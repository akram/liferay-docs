\chapter{Developer Reference}\label{developer-reference}

This developer reference contains lists of options for various APIs. The
actual API reference is stored on
\href{https://docs.liferay.com}{docs.liferay.com}. Here, you'll find
higher level descriptions of those APIs, lists of tag libraries,
descriptions of Gradle and Maven plugins, and much more.

One highlight here is a complete description of Liferay's development
tooling. This includes not only our Blade CLI (a tool that bridges the
gap between Gradle and Maven, bringing archetype-like functionality to
Liferay Gradle projects), but also our plugins for IntelliJ and Eclipse,
not to mention our Maven or Gradle-based Liferay Workspace. It also
includes a complete description of our JS Generator, which helps
front-end developers create pure JavaScript widgets.

\chapter{Classes Moved From
portal-service.jar}\label{classes-moved-from-portal-service.jar}

To leverage the benefits of modularization, many classes from
portal-service.jar have been moved into application and framework API
modules. The table below provides details about these classes and the
modules they've moved to. Package changes and each module's group,
artifact ID, and version are listed, to facilitate configuring
dependencies.

Classes Moved from portal-service.jar to Modules

This information was generated based on comparing classes in
liferay-portal-src-6.2-ee-sp20 to classes in liferay-dxp-src-7.2.10-ga1.

Class

Package

Group ID, Artifact ID, and Version

ActionHandler

Old: com.liferay.portal.kernel.mobile.device.rulegroup.action New:
com.liferay.mobile.device.rules.action

com.liferay com.liferay.mobile.device.rules.api 4.0.4

ActionHandlerManager

Old: com.liferay.portal.kernel.mobile.device.rulegroup New:
com.liferay.mobile.device.rules.action

com.liferay com.liferay.mobile.device.rules.api 4.0.4

ActionHandlerManagerUtil

Old: com.liferay.portal.kernel.mobile.device.rulegroup New:
com.liferay.mobile.device.rules.action

com.liferay com.liferay.mobile.device.rules.api 4.0.4

ActionTypeException

Old: com.liferay.portlet.mobiledevicerules New:
com.liferay.mobile.device.rules.exception

com.liferay com.liferay.mobile.device.rules.api 4.0.4

AlternateKeywordQueryHitsProcessor

Old: com.liferay.portal.kernel.search New:
com.liferay.portal.search.internal.hits

com.liferay com.liferay.portal.search 6.0.14

ArticleContentException

Old: com.liferay.portlet.journal New: com.liferay.journal.exception

com.liferay com.liferay.journal.api 4.2.1

ArticleContentSizeException

Old: com.liferay.portlet.journal New: com.liferay.journal.exception

com.liferay com.liferay.journal.api 4.2.1

ArticleCreateDateComparator

Old: com.liferay.portlet.journal.util.comparator New:
com.liferay.journal.util.comparator

com.liferay com.liferay.journal.api 4.2.1

ArticleDisplayDateComparator

Old: com.liferay.portlet.journal.util.comparator New:
com.liferay.journal.util.comparator

com.liferay com.liferay.journal.api 4.2.1

ArticleDisplayDateException

Old: com.liferay.portlet.journal New: com.liferay.journal.exception

com.liferay com.liferay.journal.api 4.2.1

ArticleExpirationDateException

Old: com.liferay.portlet.journal New: com.liferay.journal.exception

com.liferay com.liferay.journal.api 4.2.1

ArticleIDComparator

Old: com.liferay.portlet.journal.util.comparator New:
com.liferay.journal.util.comparator

com.liferay com.liferay.journal.api 4.2.1

ArticleIdException

Old: com.liferay.portlet.journal New: com.liferay.journal.exception

com.liferay com.liferay.journal.api 4.2.1

ArticleModifiedDateComparator

Old: com.liferay.portlet.journal.util.comparator New:
com.liferay.journal.util.comparator

com.liferay com.liferay.journal.api 4.2.1

ArticleResourcePKComparator

Old: com.liferay.portlet.journal.util.comparator New:
com.liferay.journal.util.comparator

com.liferay com.liferay.journal.api 4.2.1

ArticleReviewDateComparator

Old: com.liferay.portlet.journal.util.comparator New:
com.liferay.journal.util.comparator

com.liferay com.liferay.journal.api 4.2.1

ArticleReviewDateException

Old: com.liferay.portlet.journal New: com.liferay.journal.exception

com.liferay com.liferay.journal.api 4.2.1

ArticleSmallImageNameException

Old: com.liferay.portlet.journal New: com.liferay.journal.exception

com.liferay com.liferay.journal.api 4.2.1

ArticleSmallImageSizeException

Old: com.liferay.portlet.journal New: com.liferay.journal.exception

com.liferay com.liferay.journal.api 4.2.1

ArticleTitleComparator

Old: com.liferay.portlet.journal.util.comparator New:
com.liferay.journal.util.comparator

com.liferay com.liferay.journal.api 4.2.1

ArticleTitleException

Old: com.liferay.portlet.journal New: com.liferay.journal.exception

com.liferay com.liferay.journal.api 4.2.1

ArticleVersionComparator

Old: com.liferay.portlet.journal.util.comparator New:
com.liferay.journal.util.comparator

com.liferay com.liferay.journal.api 4.2.1

ArticleVersionException

Old: com.liferay.portlet.journal New: com.liferay.journal.exception

com.liferay com.liferay.journal.api 4.2.1

AuditMessageProcessor

Old: com.liferay.portal.kernel.audit New:
com.liferay.portal.security.audit

com.liferay com.liferay.portal.security.audit.api 4.0.5

AutoDeleteFileInputStream

Old: com.liferay.portal.kernel.io New: com.liferay.petra.io

com.liferay com.liferay.petra.io 3.0.2

AverageStatistics

Old: com.liferay.portal.kernel.monitoring.statistics New:
com.liferay.portal.monitoring.internal.statistics

com.liferay com.liferay.portal.monitoring 7.0.7

BackgroundTaskLocalService

Old: com.liferay.portal.service New:
com.liferay.portal.background.task.service

com.liferay com.liferay.portal.background.task.api 4.1.3

BackgroundTaskLocalServiceUtil

Old: com.liferay.portal.service New:
com.liferay.portal.background.task.service

com.liferay com.liferay.portal.background.task.api 4.1.3

BackgroundTaskLocalServiceWrapper

Old: com.liferay.portal.service New:
com.liferay.portal.background.task.service

com.liferay com.liferay.portal.background.task.api 4.1.3

BackgroundTaskModel

Old: com.liferay.portal.model New:
com.liferay.portal.background.task.model

com.liferay com.liferay.portal.background.task.api 4.1.3

BackgroundTaskPersistence

Old: com.liferay.portal.service.persistence New:
com.liferay.portal.background.task.service.persistence

com.liferay com.liferay.portal.background.task.api 4.1.3

BackgroundTaskService

Old: com.liferay.portal.service New:
com.liferay.portal.background.task.service

com.liferay com.liferay.portal.background.task.api 4.1.3

BackgroundTaskServiceUtil

Old: com.liferay.portal.service New:
com.liferay.portal.background.task.service

com.liferay com.liferay.portal.background.task.api 4.1.3

BackgroundTaskServiceWrapper

Old: com.liferay.portal.service New:
com.liferay.portal.background.task.service

com.liferay com.liferay.portal.background.task.api 4.1.3

BackgroundTaskSoap

Old: com.liferay.portal.model New:
com.liferay.portal.background.task.model

com.liferay com.liferay.portal.background.task.api 4.1.3

BackgroundTaskUtil

Old: com.liferay.portal.service.persistence New:
com.liferay.portal.background.task.service.persistence

com.liferay com.liferay.portal.background.task.api 4.1.3

BackgroundTaskWrapper

Old: com.liferay.portal.model New:
com.liferay.portal.background.task.model

com.liferay com.liferay.portal.background.task.api 4.1.3

BannedUserException

Old: com.liferay.portlet.messageboards New:
com.liferay.message.boards.exception

com.liferay com.liferay.message.boards.api 5.1.4

BaseCmisRepository

Old: com.liferay.portal.kernel.repository.cmis New:
com.liferay.document.library.repository.cmis

com.liferay com.liferay.document.library.repository.cmis.api 3.0.7

BaseCmisSearchQueryBuilder

Old: com.liferay.portal.kernel.repository.cmis.search New:
com.liferay.document.library.repository.cmis.search

com.liferay com.liferay.document.library.repository.cmis.api 3.0.7

BaseDDLExporter

Old: com.liferay.portlet.dynamicdatalists.util New:
com.liferay.dynamic.data.lists.internal.exporter

com.liferay com.liferay.dynamic.data.lists.service 3.0.12

BaseDDMDisplay

Old: com.liferay.portlet.dynamicdatamapping.util New:
com.liferay.dynamic.data.mapping.util

com.liferay com.liferay.dynamic.data.mapping.api 5.2.1

BaseFieldRenderer

Old: com.liferay.portlet.dynamicdatamapping.storage New:
com.liferay.dynamic.data.mapping.storage

com.liferay com.liferay.dynamic.data.mapping.api 5.2.1

BaseScriptingExecutor

Old: com.liferay.portal.kernel.scripting New:
com.liferay.portal.scripting

com.liferay com.liferay.portal.scripting.api 3.0.3

BaseStatistics

Old: com.liferay.portal.kernel.monitoring.statistics New:
com.liferay.portal.monitoring.internal.statistics

com.liferay com.liferay.portal.monitoring 7.0.7

BaseStorageAdapter

Old: com.liferay.portlet.dynamicdatamapping.storage New:
com.liferay.dynamic.data.mapping.storage

com.liferay com.liferay.dynamic.data.mapping.api 5.2.1

BlockingPortalCache

Old: com.liferay.portal.kernel.cache New: com.liferay.portal.cache

com.liferay com.liferay.portal.cache.api 2.0.1

BlogsEntry

Old: com.liferay.portlet.blogs.model New: com.liferay.blogs.model

com.liferay com.liferay.blogs.api 5.0.5

BlogsEntryFinder

Old: com.liferay.portlet.blogs.service.persistence New:
com.liferay.blogs.service.persistence

com.liferay com.liferay.blogs.api 5.0.5

BlogsEntryLocalService

Old: com.liferay.portlet.blogs.service New: com.liferay.blogs.service

com.liferay com.liferay.blogs.api 5.0.5

BlogsEntryLocalServiceUtil

Old: com.liferay.portlet.blogs.service New: com.liferay.blogs.service

com.liferay com.liferay.blogs.api 5.0.5

BlogsEntryLocalServiceWrapper

Old: com.liferay.portlet.blogs.service New: com.liferay.blogs.service

com.liferay com.liferay.blogs.api 5.0.5

BlogsEntryModel

Old: com.liferay.portlet.blogs.model New: com.liferay.blogs.model

com.liferay com.liferay.blogs.api 5.0.5

BlogsEntryPersistence

Old: com.liferay.portlet.blogs.service.persistence New:
com.liferay.blogs.service.persistence

com.liferay com.liferay.blogs.api 5.0.5

BlogsEntryService

Old: com.liferay.portlet.blogs.service New: com.liferay.blogs.service

com.liferay com.liferay.blogs.api 5.0.5

BlogsEntryServiceUtil

Old: com.liferay.portlet.blogs.service New: com.liferay.blogs.service

com.liferay com.liferay.blogs.api 5.0.5

BlogsEntryServiceWrapper

Old: com.liferay.portlet.blogs.service New: com.liferay.blogs.service

com.liferay com.liferay.blogs.api 5.0.5

BlogsEntrySoap

Old: com.liferay.portlet.blogs.model New: com.liferay.blogs.model

com.liferay com.liferay.blogs.api 5.0.5

BlogsEntryUtil

Old: com.liferay.portlet.blogs.service.persistence New:
com.liferay.blogs.service.persistence

com.liferay com.liferay.blogs.api 5.0.5

BlogsEntryWrapper

Old: com.liferay.portlet.blogs.model New: com.liferay.blogs.model

com.liferay com.liferay.blogs.api 5.0.5

BlogsStatsUser

Old: com.liferay.portlet.blogs.model New: com.liferay.blogs.model

com.liferay com.liferay.blogs.api 5.0.5

BlogsStatsUserFinder

Old: com.liferay.portlet.blogs.service.persistence New:
com.liferay.blogs.service.persistence

com.liferay com.liferay.blogs.api 5.0.5

BlogsStatsUserLocalService

Old: com.liferay.portlet.blogs.service New: com.liferay.blogs.service

com.liferay com.liferay.blogs.api 5.0.5

BlogsStatsUserLocalServiceUtil

Old: com.liferay.portlet.blogs.service New: com.liferay.blogs.service

com.liferay com.liferay.blogs.api 5.0.5

BlogsStatsUserLocalServiceWrapper

Old: com.liferay.portlet.blogs.service New: com.liferay.blogs.service

com.liferay com.liferay.blogs.api 5.0.5

BlogsStatsUserModel

Old: com.liferay.portlet.blogs.model New: com.liferay.blogs.model

com.liferay com.liferay.blogs.api 5.0.5

BlogsStatsUserPersistence

Old: com.liferay.portlet.blogs.service.persistence New:
com.liferay.blogs.service.persistence

com.liferay com.liferay.blogs.api 5.0.5

BlogsStatsUserSoap

Old: com.liferay.portlet.blogs.model New: com.liferay.blogs.model

com.liferay com.liferay.blogs.api 5.0.5

BlogsStatsUserUtil

Old: com.liferay.portlet.blogs.service.persistence New:
com.liferay.blogs.service.persistence

com.liferay com.liferay.blogs.api 5.0.5

BlogsStatsUserWrapper

Old: com.liferay.portlet.blogs.model New: com.liferay.blogs.model

com.liferay com.liferay.blogs.api 5.0.5

BookmarksEntry

Old: com.liferay.portlet.bookmarks.model New:
com.liferay.bookmarks.model

com.liferay com.liferay.bookmarks.api 4.0.5

BookmarksEntryFinder

Old: com.liferay.portlet.bookmarks.service.persistence New:
com.liferay.bookmarks.service.persistence

com.liferay com.liferay.bookmarks.api 4.0.5

BookmarksEntryLocalService

Old: com.liferay.portlet.bookmarks.service New:
com.liferay.bookmarks.service

com.liferay com.liferay.bookmarks.api 4.0.5

BookmarksEntryLocalServiceUtil

Old: com.liferay.portlet.bookmarks.service New:
com.liferay.bookmarks.service

com.liferay com.liferay.bookmarks.api 4.0.5

BookmarksEntryLocalServiceWrapper

Old: com.liferay.portlet.bookmarks.service New:
com.liferay.bookmarks.service

com.liferay com.liferay.bookmarks.api 4.0.5

BookmarksEntryModel

Old: com.liferay.portlet.bookmarks.model New:
com.liferay.bookmarks.model

com.liferay com.liferay.bookmarks.api 4.0.5

BookmarksEntryPersistence

Old: com.liferay.portlet.bookmarks.service.persistence New:
com.liferay.bookmarks.service.persistence

com.liferay com.liferay.bookmarks.api 4.0.5

BookmarksEntryService

Old: com.liferay.portlet.bookmarks.service New:
com.liferay.bookmarks.service

com.liferay com.liferay.bookmarks.api 4.0.5

BookmarksEntryServiceUtil

Old: com.liferay.portlet.bookmarks.service New:
com.liferay.bookmarks.service

com.liferay com.liferay.bookmarks.api 4.0.5

BookmarksEntryServiceWrapper

Old: com.liferay.portlet.bookmarks.service New:
com.liferay.bookmarks.service

com.liferay com.liferay.bookmarks.api 4.0.5

BookmarksEntrySoap

Old: com.liferay.portlet.bookmarks.model New:
com.liferay.bookmarks.model

com.liferay com.liferay.bookmarks.api 4.0.5

BookmarksEntryUtil

Old: com.liferay.portlet.bookmarks.service.persistence New:
com.liferay.bookmarks.service.persistence

com.liferay com.liferay.bookmarks.api 4.0.5

BookmarksEntryWrapper

Old: com.liferay.portlet.bookmarks.model New:
com.liferay.bookmarks.model

com.liferay com.liferay.bookmarks.api 4.0.5

BookmarksFolder

Old: com.liferay.portlet.bookmarks.model New:
com.liferay.bookmarks.model

com.liferay com.liferay.bookmarks.api 4.0.5

BookmarksFolderConstants

Old: com.liferay.portlet.bookmarks.model New:
com.liferay.bookmarks.model

com.liferay com.liferay.bookmarks.api 4.0.5

BookmarksFolderFinder

Old: com.liferay.portlet.bookmarks.service.persistence New:
com.liferay.bookmarks.service.persistence

com.liferay com.liferay.bookmarks.api 4.0.5

BookmarksFolderLocalService

Old: com.liferay.portlet.bookmarks.service New:
com.liferay.bookmarks.service

com.liferay com.liferay.bookmarks.api 4.0.5

BookmarksFolderLocalServiceUtil

Old: com.liferay.portlet.bookmarks.service New:
com.liferay.bookmarks.service

com.liferay com.liferay.bookmarks.api 4.0.5

BookmarksFolderLocalServiceWrapper

Old: com.liferay.portlet.bookmarks.service New:
com.liferay.bookmarks.service

com.liferay com.liferay.bookmarks.api 4.0.5

BookmarksFolderModel

Old: com.liferay.portlet.bookmarks.model New:
com.liferay.bookmarks.model

com.liferay com.liferay.bookmarks.api 4.0.5

BookmarksFolderPersistence

Old: com.liferay.portlet.bookmarks.service.persistence New:
com.liferay.bookmarks.service.persistence

com.liferay com.liferay.bookmarks.api 4.0.5

BookmarksFolderService

Old: com.liferay.portlet.bookmarks.service New:
com.liferay.bookmarks.service

com.liferay com.liferay.bookmarks.api 4.0.5

BookmarksFolderServiceUtil

Old: com.liferay.portlet.bookmarks.service New:
com.liferay.bookmarks.service

com.liferay com.liferay.bookmarks.api 4.0.5

BookmarksFolderServiceWrapper

Old: com.liferay.portlet.bookmarks.service New:
com.liferay.bookmarks.service

com.liferay com.liferay.bookmarks.api 4.0.5

BookmarksFolderSoap

Old: com.liferay.portlet.bookmarks.model New:
com.liferay.bookmarks.model

com.liferay com.liferay.bookmarks.api 4.0.5

BookmarksFolderUtil

Old: com.liferay.portlet.bookmarks.service.persistence New:
com.liferay.bookmarks.service.persistence

com.liferay com.liferay.bookmarks.api 4.0.5

BookmarksFolderWrapper

Old: com.liferay.portlet.bookmarks.model New:
com.liferay.bookmarks.model

com.liferay com.liferay.bookmarks.api 4.0.5

ByteArrayReportResultContainer

Old: com.liferay.portal.kernel.bi.reporting New:
com.liferay.portal.reports.engine

com.liferay com.liferay.portal.reports.engine.api 5.0.1

CMISBetweenExpression

Old: com.liferay.portal.kernel.repository.cmis.search New:
com.liferay.document.library.repository.cmis.search

com.liferay com.liferay.document.library.repository.cmis.api 3.0.7

CMISConjunction

Old: com.liferay.portal.kernel.repository.cmis.search New:
com.liferay.document.library.repository.cmis.search

com.liferay com.liferay.document.library.repository.cmis.api 3.0.7

CMISContainsExpression

Old: com.liferay.portal.kernel.repository.cmis.search New:
com.liferay.document.library.repository.cmis.search

com.liferay com.liferay.document.library.repository.cmis.api 3.0.7

CMISContainsNotExpression

Old: com.liferay.portal.kernel.repository.cmis.search New:
com.liferay.document.library.repository.cmis.search

com.liferay com.liferay.document.library.repository.cmis.api 3.0.7

CMISContainsValueExpression

Old: com.liferay.portal.kernel.repository.cmis.search New:
com.liferay.document.library.repository.cmis.search

com.liferay com.liferay.document.library.repository.cmis.api 3.0.7

CMISCriterion

Old: com.liferay.portal.kernel.repository.cmis.search New:
com.liferay.document.library.repository.cmis.search

com.liferay com.liferay.document.library.repository.cmis.api 3.0.7

CMISDisjunction

Old: com.liferay.portal.kernel.repository.cmis.search New:
com.liferay.document.library.repository.cmis.search

com.liferay com.liferay.document.library.repository.cmis.api 3.0.7

CMISFullTextConjunction

Old: com.liferay.portal.kernel.repository.cmis.search New:
com.liferay.document.library.repository.cmis.search

com.liferay com.liferay.document.library.repository.cmis.api 3.0.7

CMISInFolderExpression

Old: com.liferay.portal.kernel.repository.cmis.search New:
com.liferay.document.library.repository.cmis.search

com.liferay com.liferay.document.library.repository.cmis.api 3.0.7

CMISInTreeExpression

Old: com.liferay.portal.kernel.repository.cmis.search New:
com.liferay.document.library.repository.cmis.search

com.liferay com.liferay.document.library.repository.cmis.api 3.0.7

CMISJunction

Old: com.liferay.portal.kernel.repository.cmis.search New:
com.liferay.document.library.repository.cmis.search

com.liferay com.liferay.document.library.repository.cmis.api 3.0.7

CMISNotExpression

Old: com.liferay.portal.kernel.repository.cmis.search New:
com.liferay.document.library.repository.cmis.search

com.liferay com.liferay.document.library.repository.cmis.api 3.0.7

CMISParameterValueUtil

Old: com.liferay.portal.kernel.repository.cmis.search New:
com.liferay.document.library.repository.cmis.search

com.liferay com.liferay.document.library.repository.cmis.api 3.0.7

CMISRepositoryHandler

Old: com.liferay.portal.kernel.repository.cmis New:
com.liferay.document.library.repository.cmis

com.liferay com.liferay.document.library.repository.cmis.api 3.0.7

CMISRepositoryUtil

Old: com.liferay.portal.kernel.repository.cmis New:
com.liferay.document.library.repository.cmis.internal

com.liferay com.liferay.document.library.repository.cmis.impl 4.0.8

CMISSearchQueryBuilder

Old: com.liferay.portal.kernel.repository.cmis.search New:
com.liferay.document.library.repository.cmis.search

com.liferay com.liferay.document.library.repository.cmis.api 3.0.7

CMISSimpleExpression

Old: com.liferay.portal.kernel.repository.cmis.search New:
com.liferay.document.library.repository.cmis.search

com.liferay com.liferay.document.library.repository.cmis.api 3.0.7

CMISSimpleExpressionOperator

Old: com.liferay.portal.kernel.repository.cmis.search New:
com.liferay.document.library.repository.cmis.search

com.liferay com.liferay.document.library.repository.cmis.api 3.0.7

CharPool

Old: com.liferay.portal.kernel.util New: com.liferay.petra.string

com.liferay com.liferay.petra.string 3.0.1

CharsetDecoderUtil

Old: com.liferay.portal.kernel.nio.charset New: com.liferay.petra.nio

com.liferay com.liferay.petra.nio 3.0.1

CharsetEncoderUtil

Old: com.liferay.portal.kernel.nio.charset New: com.liferay.petra.nio

com.liferay com.liferay.petra.nio 3.0.1

ClassLoaderPool

Old: com.liferay.portal.kernel.util New: com.liferay.petra.lang

com.liferay com.liferay.petra.lang 3.0.1

ClassPathUtil

Old: com.liferay.portal.kernel.process New: com.liferay.petra.process

com.liferay com.liferay.petra.process 3.0.4

ClassResolverUtil

Old: com.liferay.portal.kernel.util New: com.liferay.petra.lang

com.liferay com.liferay.petra.lang 3.0.1

CollatedSpellCheckHitsProcessor

Old: com.liferay.portal.kernel.search New:
com.liferay.portal.search.internal.hits

com.liferay com.liferay.portal.search 6.0.14

CompoundSessionIdServletRequest

Old: com.liferay.portal.kernel.servlet.filters.compoundsessionid New:
com.liferay.portal.compound.session.id.internal

com.liferay com.liferay.portal.compound.session.id 4.0.5

Condition

Old: com.liferay.portlet.dynamicdatamapping.storage.query New:
com.liferay.adaptive.media.image.media.query

com.liferay com.liferay.adaptive.media.image.api 3.0.3

ConsumerOutputProcessor

Old: com.liferay.portal.kernel.process New: com.liferay.petra.process

com.liferay com.liferay.petra.process 3.0.4

ContactConverterKeys

Old: com.liferay.portal.security.ldap New:
com.liferay.portal.security.ldap

com.liferay com.liferay.portal.security.ldap.api 2.0.8

ContentException

Old: com.liferay.portlet.dynamicdatamapping New:
com.liferay.dynamic.data.mapping.exception

com.liferay com.liferay.dynamic.data.mapping.api 5.2.1

ContentNameException

Old: com.liferay.portlet.dynamicdatamapping New:
com.liferay.dynamic.data.mapping.exception

com.liferay com.liferay.dynamic.data.mapping.api 5.2.1

ContextClassloaderReportDesignRetriever

Old: com.liferay.portal.kernel.bi.reporting New:
com.liferay.portal.reports.engine

com.liferay com.liferay.portal.reports.engine.api 5.0.1

CountStatistics

Old: com.liferay.portal.kernel.monitoring.statistics New:
com.liferay.portal.monitoring.internal.statistics

com.liferay com.liferay.portal.monitoring 7.0.7

DDL

Old: com.liferay.portlet.dynamicdatalists.util New:
com.liferay.dynamic.data.lists.util

com.liferay com.liferay.dynamic.data.lists.api 4.0.5

DDLExporter

Old: com.liferay.portlet.dynamicdatalists.util New:
com.liferay.dynamic.data.lists.exporter

com.liferay com.liferay.dynamic.data.lists.api 4.0.5

DDLExporterFactory

Old: com.liferay.portlet.dynamicdatalists.util New:
com.liferay.dynamic.data.lists.exporter

com.liferay com.liferay.dynamic.data.lists.api 4.0.5

DDLRecord

Old: com.liferay.portlet.dynamicdatalists.model New:
com.liferay.dynamic.data.lists.model

com.liferay com.liferay.dynamic.data.lists.api 4.0.5

DDLRecordConstants

Old: com.liferay.portlet.dynamicdatalists.model New:
com.liferay.dynamic.data.lists.model

com.liferay com.liferay.dynamic.data.lists.api 4.0.5

DDLRecordFinder

Old: com.liferay.portlet.dynamicdatalists.service.persistence New:
com.liferay.dynamic.data.lists.service.persistence

com.liferay com.liferay.dynamic.data.lists.api 4.0.5

DDLRecordLocalService

Old: com.liferay.portlet.dynamicdatalists.service New:
com.liferay.dynamic.data.lists.service

com.liferay com.liferay.dynamic.data.lists.api 4.0.5

DDLRecordLocalServiceUtil

Old: com.liferay.portlet.dynamicdatalists.service New:
com.liferay.dynamic.data.lists.service

com.liferay com.liferay.dynamic.data.lists.api 4.0.5

DDLRecordLocalServiceWrapper

Old: com.liferay.portlet.dynamicdatalists.service New:
com.liferay.dynamic.data.lists.service

com.liferay com.liferay.dynamic.data.lists.api 4.0.5

DDLRecordModel

Old: com.liferay.portlet.dynamicdatalists.model New:
com.liferay.dynamic.data.lists.model

com.liferay com.liferay.dynamic.data.lists.api 4.0.5

DDLRecordPersistence

Old: com.liferay.portlet.dynamicdatalists.service.persistence New:
com.liferay.dynamic.data.lists.service.persistence

com.liferay com.liferay.dynamic.data.lists.api 4.0.5

DDLRecordService

Old: com.liferay.portlet.dynamicdatalists.service New:
com.liferay.dynamic.data.lists.service

com.liferay com.liferay.dynamic.data.lists.api 4.0.5

DDLRecordServiceUtil

Old: com.liferay.portlet.dynamicdatalists.service New:
com.liferay.dynamic.data.lists.service

com.liferay com.liferay.dynamic.data.lists.api 4.0.5

DDLRecordServiceWrapper

Old: com.liferay.portlet.dynamicdatalists.service New:
com.liferay.dynamic.data.lists.service

com.liferay com.liferay.dynamic.data.lists.api 4.0.5

DDLRecordSet

Old: com.liferay.portlet.dynamicdatalists.model New:
com.liferay.dynamic.data.lists.model

com.liferay com.liferay.dynamic.data.lists.api 4.0.5

DDLRecordSetConstants

Old: com.liferay.portlet.dynamicdatalists.model New:
com.liferay.dynamic.data.lists.model

com.liferay com.liferay.dynamic.data.lists.api 4.0.5

DDLRecordSetFinder

Old: com.liferay.portlet.dynamicdatalists.service.persistence New:
com.liferay.dynamic.data.lists.service.persistence

com.liferay com.liferay.dynamic.data.lists.api 4.0.5

DDLRecordSetLocalService

Old: com.liferay.portlet.dynamicdatalists.service New:
com.liferay.dynamic.data.lists.service

com.liferay com.liferay.dynamic.data.lists.api 4.0.5

DDLRecordSetLocalServiceUtil

Old: com.liferay.portlet.dynamicdatalists.service New:
com.liferay.dynamic.data.lists.service

com.liferay com.liferay.dynamic.data.lists.api 4.0.5

DDLRecordSetLocalServiceWrapper

Old: com.liferay.portlet.dynamicdatalists.service New:
com.liferay.dynamic.data.lists.service

com.liferay com.liferay.dynamic.data.lists.api 4.0.5

DDLRecordSetModel

Old: com.liferay.portlet.dynamicdatalists.model New:
com.liferay.dynamic.data.lists.model

com.liferay com.liferay.dynamic.data.lists.api 4.0.5

DDLRecordSetPersistence

Old: com.liferay.portlet.dynamicdatalists.service.persistence New:
com.liferay.dynamic.data.lists.service.persistence

com.liferay com.liferay.dynamic.data.lists.api 4.0.5

DDLRecordSetService

Old: com.liferay.portlet.dynamicdatalists.service New:
com.liferay.dynamic.data.lists.service

com.liferay com.liferay.dynamic.data.lists.api 4.0.5

DDLRecordSetServiceUtil

Old: com.liferay.portlet.dynamicdatalists.service New:
com.liferay.dynamic.data.lists.service

com.liferay com.liferay.dynamic.data.lists.api 4.0.5

DDLRecordSetServiceWrapper

Old: com.liferay.portlet.dynamicdatalists.service New:
com.liferay.dynamic.data.lists.service

com.liferay com.liferay.dynamic.data.lists.api 4.0.5

DDLRecordSetSoap

Old: com.liferay.portlet.dynamicdatalists.model New:
com.liferay.dynamic.data.lists.model

com.liferay com.liferay.dynamic.data.lists.api 4.0.5

DDLRecordSetUtil

Old: com.liferay.portlet.dynamicdatalists.service.persistence New:
com.liferay.dynamic.data.lists.service.persistence

com.liferay com.liferay.dynamic.data.lists.api 4.0.5

DDLRecordSetWrapper

Old: com.liferay.portlet.dynamicdatalists.model New:
com.liferay.dynamic.data.lists.model

com.liferay com.liferay.dynamic.data.lists.api 4.0.5

DDLRecordSoap

Old: com.liferay.portlet.dynamicdatalists.model New:
com.liferay.dynamic.data.lists.model

com.liferay com.liferay.dynamic.data.lists.api 4.0.5

DDLRecordUtil

Old: com.liferay.portlet.dynamicdatalists.service.persistence New:
com.liferay.dynamic.data.lists.service.persistence

com.liferay com.liferay.dynamic.data.lists.api 4.0.5

DDLRecordVersion

Old: com.liferay.portlet.dynamicdatalists.model New:
com.liferay.dynamic.data.lists.model

com.liferay com.liferay.dynamic.data.lists.api 4.0.5

DDLRecordVersionModel

Old: com.liferay.portlet.dynamicdatalists.model New:
com.liferay.dynamic.data.lists.model

com.liferay com.liferay.dynamic.data.lists.api 4.0.5

DDLRecordVersionPersistence

Old: com.liferay.portlet.dynamicdatalists.service.persistence New:
com.liferay.dynamic.data.lists.service.persistence

com.liferay com.liferay.dynamic.data.lists.api 4.0.5

DDLRecordVersionSoap

Old: com.liferay.portlet.dynamicdatalists.model New:
com.liferay.dynamic.data.lists.model

com.liferay com.liferay.dynamic.data.lists.api 4.0.5

DDLRecordVersionUtil

Old: com.liferay.portlet.dynamicdatalists.service.persistence New:
com.liferay.dynamic.data.lists.service.persistence

com.liferay com.liferay.dynamic.data.lists.api 4.0.5

DDLRecordVersionVersionComparator

Old: com.liferay.portlet.dynamicdatalists.util.comparator New:
com.liferay.dynamic.data.lists.util.comparator

com.liferay com.liferay.dynamic.data.lists.api 4.0.5

DDLRecordVersionWrapper

Old: com.liferay.portlet.dynamicdatalists.model New:
com.liferay.dynamic.data.lists.model

com.liferay com.liferay.dynamic.data.lists.api 4.0.5

DDLRecordWrapper

Old: com.liferay.portlet.dynamicdatalists.model New:
com.liferay.dynamic.data.lists.model

com.liferay com.liferay.dynamic.data.lists.api 4.0.5

DDM

Old: com.liferay.portlet.dynamicdatamapping.util New:
com.liferay.dynamic.data.mapping.util

com.liferay com.liferay.dynamic.data.mapping.api 5.2.1

DDMContent

Old: com.liferay.portlet.dynamicdatamapping.model New:
com.liferay.dynamic.data.mapping.model

com.liferay com.liferay.dynamic.data.mapping.api 5.2.1

DDMContentLocalService

Old: com.liferay.portlet.dynamicdatamapping.service New:
com.liferay.dynamic.data.mapping.service

com.liferay com.liferay.dynamic.data.mapping.api 5.2.1

DDMContentLocalServiceUtil

Old: com.liferay.portlet.dynamicdatamapping.service New:
com.liferay.dynamic.data.mapping.service

com.liferay com.liferay.dynamic.data.mapping.api 5.2.1

DDMContentLocalServiceWrapper

Old: com.liferay.portlet.dynamicdatamapping.service New:
com.liferay.dynamic.data.mapping.service

com.liferay com.liferay.dynamic.data.mapping.api 5.2.1

DDMContentModel

Old: com.liferay.portlet.dynamicdatamapping.model New:
com.liferay.dynamic.data.mapping.model

com.liferay com.liferay.dynamic.data.mapping.api 5.2.1

DDMContentPersistence

Old: com.liferay.portlet.dynamicdatamapping.service.persistence New:
com.liferay.dynamic.data.mapping.service.persistence

com.liferay com.liferay.dynamic.data.mapping.api 5.2.1

DDMContentSoap

Old: com.liferay.portlet.dynamicdatamapping.model New:
com.liferay.dynamic.data.mapping.model

com.liferay com.liferay.dynamic.data.mapping.api 5.2.1

DDMContentUtil

Old: com.liferay.portlet.dynamicdatamapping.service.persistence New:
com.liferay.dynamic.data.mapping.service.persistence

com.liferay com.liferay.dynamic.data.mapping.api 5.2.1

DDMContentWrapper

Old: com.liferay.portlet.dynamicdatamapping.model New:
com.liferay.dynamic.data.mapping.model

com.liferay com.liferay.dynamic.data.mapping.api 5.2.1

DDMDisplay

Old: com.liferay.portlet.dynamicdatamapping.util New:
com.liferay.dynamic.data.mapping.util

com.liferay com.liferay.dynamic.data.mapping.api 5.2.1

DDMDisplayRegistry

Old: com.liferay.portlet.dynamicdatamapping.util New:
com.liferay.dynamic.data.mapping.util

com.liferay com.liferay.dynamic.data.mapping.api 5.2.1

DDMIndexer

Old: com.liferay.portlet.dynamicdatamapping.util New:
com.liferay.dynamic.data.mapping.util

com.liferay com.liferay.dynamic.data.mapping.api 5.2.1

DDMIndexerUtil

Old: com.liferay.portlet.dynamicdatamapping.util New:
com.liferay.asset.list.internal.dynamic.data.mapping.util

com.liferay com.liferay.asset.list.service 1.0.11

DDMStorageLink

Old: com.liferay.portlet.dynamicdatamapping.model New:
com.liferay.dynamic.data.mapping.model

com.liferay com.liferay.dynamic.data.mapping.api 5.2.1

DDMStorageLinkLocalService

Old: com.liferay.portlet.dynamicdatamapping.service New:
com.liferay.dynamic.data.mapping.service

com.liferay com.liferay.dynamic.data.mapping.api 5.2.1

DDMStorageLinkLocalServiceUtil

Old: com.liferay.portlet.dynamicdatamapping.service New:
com.liferay.dynamic.data.mapping.service

com.liferay com.liferay.dynamic.data.mapping.api 5.2.1

DDMStorageLinkLocalServiceWrapper

Old: com.liferay.portlet.dynamicdatamapping.service New:
com.liferay.dynamic.data.mapping.service

com.liferay com.liferay.dynamic.data.mapping.api 5.2.1

DDMStorageLinkModel

Old: com.liferay.portlet.dynamicdatamapping.model New:
com.liferay.dynamic.data.mapping.model

com.liferay com.liferay.dynamic.data.mapping.api 5.2.1

DDMStorageLinkPersistence

Old: com.liferay.portlet.dynamicdatamapping.service.persistence New:
com.liferay.dynamic.data.mapping.service.persistence

com.liferay com.liferay.dynamic.data.mapping.api 5.2.1

DDMStorageLinkSoap

Old: com.liferay.portlet.dynamicdatamapping.model New:
com.liferay.dynamic.data.mapping.model

com.liferay com.liferay.dynamic.data.mapping.api 5.2.1

DDMStorageLinkUtil

Old: com.liferay.portlet.dynamicdatamapping.service.persistence New:
com.liferay.dynamic.data.mapping.service.persistence

com.liferay com.liferay.dynamic.data.mapping.api 5.2.1

DDMStorageLinkWrapper

Old: com.liferay.portlet.dynamicdatamapping.model New:
com.liferay.dynamic.data.mapping.model

com.liferay com.liferay.dynamic.data.mapping.api 5.2.1

DDMStructureConstants

Old: com.liferay.portlet.dynamicdatamapping.model New:
com.liferay.dynamic.data.mapping.model

com.liferay com.liferay.dynamic.data.mapping.api 5.2.1

DDMStructureFinder

Old: com.liferay.portlet.dynamicdatamapping.service.persistence New:
com.liferay.dynamic.data.mapping.service.persistence

com.liferay com.liferay.dynamic.data.mapping.api 5.2.1

DDMStructureLinkLocalService

Old: com.liferay.portlet.dynamicdatamapping.service New:
com.liferay.dynamic.data.mapping.service

com.liferay com.liferay.dynamic.data.mapping.api 5.2.1

DDMStructureLinkLocalServiceUtil

Old: com.liferay.portlet.dynamicdatamapping.service New:
com.liferay.dynamic.data.mapping.service

com.liferay com.liferay.dynamic.data.mapping.api 5.2.1

DDMStructureLinkLocalServiceWrapper

Old: com.liferay.portlet.dynamicdatamapping.service New:
com.liferay.dynamic.data.mapping.service

com.liferay com.liferay.dynamic.data.mapping.api 5.2.1

DDMStructureLinkModel

Old: com.liferay.portlet.dynamicdatamapping.model New:
com.liferay.dynamic.data.mapping.model

com.liferay com.liferay.dynamic.data.mapping.api 5.2.1

DDMStructureLinkPersistence

Old: com.liferay.portlet.dynamicdatamapping.service.persistence New:
com.liferay.dynamic.data.mapping.service.persistence

com.liferay com.liferay.dynamic.data.mapping.api 5.2.1

DDMStructureLinkSoap

Old: com.liferay.portlet.dynamicdatamapping.model New:
com.liferay.dynamic.data.mapping.model

com.liferay com.liferay.dynamic.data.mapping.api 5.2.1

DDMStructureLinkUtil

Old: com.liferay.portlet.dynamicdatamapping.service.persistence New:
com.liferay.dynamic.data.mapping.service.persistence

com.liferay com.liferay.dynamic.data.mapping.api 5.2.1

DDMStructureLinkWrapper

Old: com.liferay.portlet.dynamicdatamapping.model New:
com.liferay.dynamic.data.mapping.model

com.liferay com.liferay.dynamic.data.mapping.api 5.2.1

DDMStructureLocalService

Old: com.liferay.portlet.dynamicdatamapping.service New:
com.liferay.dynamic.data.mapping.service

com.liferay com.liferay.dynamic.data.mapping.api 5.2.1

DDMStructureLocalServiceUtil

Old: com.liferay.portlet.dynamicdatamapping.service New:
com.liferay.dynamic.data.mapping.service

com.liferay com.liferay.dynamic.data.mapping.api 5.2.1

DDMStructureLocalServiceWrapper

Old: com.liferay.portlet.dynamicdatamapping.service New:
com.liferay.dynamic.data.mapping.service

com.liferay com.liferay.dynamic.data.mapping.api 5.2.1

DDMStructureModel

Old: com.liferay.portlet.dynamicdatamapping.model New:
com.liferay.dynamic.data.mapping.model

com.liferay com.liferay.dynamic.data.mapping.api 5.2.1

DDMStructurePersistence

Old: com.liferay.portlet.dynamicdatamapping.service.persistence New:
com.liferay.dynamic.data.mapping.service.persistence

com.liferay com.liferay.dynamic.data.mapping.api 5.2.1

DDMStructureService

Old: com.liferay.portlet.dynamicdatamapping.service New:
com.liferay.dynamic.data.mapping.service

com.liferay com.liferay.dynamic.data.mapping.api 5.2.1

DDMStructureServiceUtil

Old: com.liferay.portlet.dynamicdatamapping.service New:
com.liferay.dynamic.data.mapping.service

com.liferay com.liferay.dynamic.data.mapping.api 5.2.1

DDMStructureServiceWrapper

Old: com.liferay.portlet.dynamicdatamapping.service New:
com.liferay.dynamic.data.mapping.service

com.liferay com.liferay.dynamic.data.mapping.api 5.2.1

DDMStructureSoap

Old: com.liferay.portlet.dynamicdatamapping.model New:
com.liferay.dynamic.data.mapping.model

com.liferay com.liferay.dynamic.data.mapping.api 5.2.1

DDMStructureUtil

Old: com.liferay.portlet.dynamicdatamapping.service.persistence New:
com.liferay.dynamic.data.mapping.service.persistence

com.liferay com.liferay.dynamic.data.mapping.api 5.2.1

DDMStructureWrapper

Old: com.liferay.portlet.dynamicdatamapping.model New:
com.liferay.dynamic.data.mapping.model

com.liferay com.liferay.dynamic.data.mapping.api 5.2.1

DDMTemplateConstants

Old: com.liferay.portlet.dynamicdatamapping.model New:
com.liferay.dynamic.data.mapping.model

com.liferay com.liferay.dynamic.data.mapping.api 5.2.1

DDMTemplateFinder

Old: com.liferay.portlet.dynamicdatamapping.service.persistence New:
com.liferay.dynamic.data.mapping.service.persistence

com.liferay com.liferay.dynamic.data.mapping.api 5.2.1

DDMTemplateHelper

Old: com.liferay.portlet.dynamicdatamapping.util New:
com.liferay.dynamic.data.mapping.util

com.liferay com.liferay.dynamic.data.mapping.api 5.2.1

DDMTemplateLocalService

Old: com.liferay.portlet.dynamicdatamapping.service New:
com.liferay.dynamic.data.mapping.service

com.liferay com.liferay.dynamic.data.mapping.api 5.2.1

DDMTemplateLocalServiceUtil

Old: com.liferay.portlet.dynamicdatamapping.service New:
com.liferay.dynamic.data.mapping.service

com.liferay com.liferay.dynamic.data.mapping.api 5.2.1

DDMTemplateLocalServiceWrapper

Old: com.liferay.portlet.dynamicdatamapping.service New:
com.liferay.dynamic.data.mapping.service

com.liferay com.liferay.dynamic.data.mapping.api 5.2.1

DDMTemplateModel

Old: com.liferay.portlet.dynamicdatamapping.model New:
com.liferay.dynamic.data.mapping.model

com.liferay com.liferay.dynamic.data.mapping.api 5.2.1

DDMTemplatePersistence

Old: com.liferay.portlet.dynamicdatamapping.service.persistence New:
com.liferay.dynamic.data.mapping.service.persistence

com.liferay com.liferay.dynamic.data.mapping.api 5.2.1

DDMTemplateService

Old: com.liferay.portlet.dynamicdatamapping.service New:
com.liferay.dynamic.data.mapping.service

com.liferay com.liferay.dynamic.data.mapping.api 5.2.1

DDMTemplateServiceUtil

Old: com.liferay.portlet.dynamicdatamapping.service New:
com.liferay.dynamic.data.mapping.service

com.liferay com.liferay.dynamic.data.mapping.api 5.2.1

DDMTemplateServiceWrapper

Old: com.liferay.portlet.dynamicdatamapping.service New:
com.liferay.dynamic.data.mapping.service

com.liferay com.liferay.dynamic.data.mapping.api 5.2.1

DDMTemplateSoap

Old: com.liferay.portlet.dynamicdatamapping.model New:
com.liferay.dynamic.data.mapping.model

com.liferay com.liferay.dynamic.data.mapping.api 5.2.1

DDMTemplateUtil

Old: com.liferay.portlet.dynamicdatamapping.service.persistence New:
com.liferay.dynamic.data.mapping.service.persistence

com.liferay com.liferay.dynamic.data.mapping.api 5.2.1

DDMTemplateWrapper

Old: com.liferay.portlet.dynamicdatamapping.model New:
com.liferay.dynamic.data.mapping.model

com.liferay com.liferay.dynamic.data.mapping.api 5.2.1

DDMUtil

Old: com.liferay.portlet.dynamicdatamapping.util New:
com.liferay.dynamic.data.mapping.util

com.liferay com.liferay.dynamic.data.mapping.api 5.2.1

DDMXML

Old: com.liferay.portlet.dynamicdatamapping.util New:
com.liferay.dynamic.data.mapping.util

com.liferay com.liferay.dynamic.data.mapping.api 5.2.1

DLContent

Old: com.liferay.portlet.documentlibrary.model New:
com.liferay.document.library.content.model

com.liferay com.liferay.document.library.content.api 2.0.3

DLContentDataBlobModel

Old: com.liferay.portlet.documentlibrary.model New:
com.liferay.document.library.content.model

com.liferay com.liferay.document.library.content.api 2.0.3

DLContentLocalService

Old: com.liferay.portlet.documentlibrary.service New:
com.liferay.document.library.content.service

com.liferay com.liferay.document.library.content.api 2.0.3

DLContentLocalServiceUtil

Old: com.liferay.portlet.documentlibrary.service New:
com.liferay.document.library.content.service

com.liferay com.liferay.document.library.content.api 2.0.3

DLContentLocalServiceWrapper

Old: com.liferay.portlet.documentlibrary.service New:
com.liferay.document.library.content.service

com.liferay com.liferay.document.library.content.api 2.0.3

DLContentModel

Old: com.liferay.portlet.documentlibrary.model New:
com.liferay.document.library.content.model

com.liferay com.liferay.document.library.content.api 2.0.3

DLContentPersistence

Old: com.liferay.portlet.documentlibrary.service.persistence New:
com.liferay.document.library.content.service.persistence

com.liferay com.liferay.document.library.content.api 2.0.3

DLContentSoap

Old: com.liferay.portlet.documentlibrary.model New:
com.liferay.document.library.content.model

com.liferay com.liferay.document.library.content.api 2.0.3

DLContentUtil

Old: com.liferay.portlet.documentlibrary.service.persistence New:
com.liferay.document.library.content.service.persistence

com.liferay com.liferay.document.library.content.api 2.0.3

DLContentVersionComparator

Old: com.liferay.portlet.documentlibrary.util.comparator New:
com.liferay.document.library.content.service.util.comparator

com.liferay com.liferay.document.library.content.service 2.0.3

DLContentWrapper

Old: com.liferay.portlet.documentlibrary.model New:
com.liferay.document.library.content.model

com.liferay com.liferay.document.library.content.api 2.0.3

DLFileRank

Old: com.liferay.portlet.documentlibrary.model New:
com.liferay.document.library.file.rank.model

com.liferay com.liferay.document.library.file.rank.api 2.0.3

DLFileRankFinder

Old: com.liferay.portlet.documentlibrary.service.persistence New:
com.liferay.document.library.file.rank.service.persistence

com.liferay com.liferay.document.library.file.rank.api 2.0.3

DLFileRankLocalService

Old: com.liferay.portlet.documentlibrary.service New:
com.liferay.document.library.file.rank.service

com.liferay com.liferay.document.library.file.rank.api 2.0.3

DLFileRankLocalServiceUtil

Old: com.liferay.portlet.documentlibrary.service New:
com.liferay.document.library.file.rank.service

com.liferay com.liferay.document.library.file.rank.api 2.0.3

DLFileRankLocalServiceWrapper

Old: com.liferay.portlet.documentlibrary.service New:
com.liferay.document.library.file.rank.service

com.liferay com.liferay.document.library.file.rank.api 2.0.3

DLFileRankModel

Old: com.liferay.portlet.documentlibrary.model New:
com.liferay.document.library.file.rank.model

com.liferay com.liferay.document.library.file.rank.api 2.0.3

DLFileRankPersistence

Old: com.liferay.portlet.documentlibrary.service.persistence New:
com.liferay.document.library.file.rank.service.persistence

com.liferay com.liferay.document.library.file.rank.api 2.0.3

DLFileRankSoap

Old: com.liferay.portlet.documentlibrary.model New:
com.liferay.document.library.file.rank.model

com.liferay com.liferay.document.library.file.rank.api 2.0.3

DLFileRankUtil

Old: com.liferay.portlet.documentlibrary.service.persistence New:
com.liferay.document.library.file.rank.service.persistence

com.liferay com.liferay.document.library.file.rank.api 2.0.3

DLFileRankWrapper

Old: com.liferay.portlet.documentlibrary.model New:
com.liferay.document.library.file.rank.model

com.liferay com.liferay.document.library.file.rank.api 2.0.3

DLSyncConstants

Old: com.liferay.portlet.documentlibrary.model New:
com.liferay.document.library.sync.constants

com.liferay com.liferay.document.library.sync.api 2.0.3

DLSyncEvent

Old: com.liferay.portlet.documentlibrary.model New:
com.liferay.document.library.sync.model

com.liferay com.liferay.document.library.sync.api 2.0.3

DLSyncEventLocalService

Old: com.liferay.portlet.documentlibrary.service New:
com.liferay.document.library.sync.service

com.liferay com.liferay.document.library.sync.api 2.0.3

DLSyncEventLocalServiceUtil

Old: com.liferay.portlet.documentlibrary.service New:
com.liferay.document.library.sync.service

com.liferay com.liferay.document.library.sync.api 2.0.3

DLSyncEventLocalServiceWrapper

Old: com.liferay.portlet.documentlibrary.service New:
com.liferay.document.library.sync.service

com.liferay com.liferay.document.library.sync.api 2.0.3

DLSyncEventModel

Old: com.liferay.portlet.documentlibrary.model New:
com.liferay.document.library.sync.model

com.liferay com.liferay.document.library.sync.api 2.0.3

DLSyncEventPersistence

Old: com.liferay.portlet.documentlibrary.service.persistence New:
com.liferay.document.library.sync.service.persistence

com.liferay com.liferay.document.library.sync.api 2.0.3

DLSyncEventSoap

Old: com.liferay.portlet.documentlibrary.model New:
com.liferay.document.library.sync.model

com.liferay com.liferay.document.library.sync.api 2.0.3

DLSyncEventUtil

Old: com.liferay.portlet.documentlibrary.service.persistence New:
com.liferay.document.library.sync.service.persistence

com.liferay com.liferay.document.library.sync.api 2.0.3

DLSyncEventWrapper

Old: com.liferay.portlet.documentlibrary.model New:
com.liferay.document.library.sync.model

com.liferay com.liferay.document.library.sync.api 2.0.3

Database

Old: com.liferay.portal.kernel.util New:
com.liferay.portal.tools.db.upgrade.client

com.liferay com.liferay.portal.tools.db.upgrade.client 3.0.0

DefaultAttributesTransformer

Old: com.liferay.portal.security.ldap New:
com.liferay.portal.security.ldap.internal

com.liferay com.liferay.portal.security.ldap.impl 2.0.5

DefaultMessageBus

Old: com.liferay.portal.kernel.messaging New:
com.liferay.portal.messaging.internal

com.liferay com.liferay.portal.messaging 6.0.5

DefaultSingleDestinationMessageSender

Old: com.liferay.portal.kernel.messaging.sender New:
com.liferay.portal.messaging.internal.sender

com.liferay com.liferay.portal.messaging 6.0.5

DefaultSingleDestinationSynchronousMessageSender

Old: com.liferay.portal.kernel.messaging.sender New:
com.liferay.portal.messaging.internal.sender

com.liferay com.liferay.portal.messaging 6.0.5

DefaultSynchronousMessageSender

Old: com.liferay.portal.kernel.messaging.sender New:
com.liferay.portal.messaging.internal.sender

com.liferay com.liferay.portal.messaging 6.0.5

DeleteFileFinalizeAction

Old: com.liferay.portal.kernel.memory New: com.liferay.petra.memory

com.liferay com.liferay.petra.memory 3.0.1

DestinationStatisticsManager

Old: com.liferay.portal.kernel.messaging.jmx New:
com.liferay.portal.messaging.internal.jmx

com.liferay com.liferay.portal.messaging 6.0.5

DestinationStatisticsManagerMBean

Old: com.liferay.portal.kernel.messaging.jmx New:
com.liferay.portal.messaging.internal.jmx

com.liferay com.liferay.portal.messaging 6.0.5

DirectSynchronousMessageSender

Old: com.liferay.portal.kernel.messaging.sender New:
com.liferay.portal.messaging.internal.sender

com.liferay com.liferay.portal.messaging 6.0.5

Dummy

Old: com.liferay.portal.model New:
com.liferay.exportimport.test.util.model

com.liferay com.liferay.exportimport.test.util 2.0.6

DummyContext

Old: com.liferay.portal.kernel.ldap New:
com.liferay.portal.security.ldap.dummy

com.liferay com.liferay.portal.security.ldap.api 2.0.8

DummyDirContext

Old: com.liferay.portal.kernel.ldap New:
com.liferay.portal.security.ldap.dummy

com.liferay com.liferay.portal.security.ldap.api 2.0.8

DummyFinalizeAction

Old: com.liferay.portal.kernel.memory New: com.liferay.petra.memory

com.liferay com.liferay.petra.memory 3.0.1

DuplicateArticleIdException

Old: com.liferay.portlet.journal New: com.liferay.journal.exception

com.liferay com.liferay.journal.api 4.2.1

DuplicateFeedIdException

Old: com.liferay.portlet.journal New: com.liferay.journal.exception

com.liferay com.liferay.journal.api 4.2.1

DuplicateLDAPServerNameException

Old: com.liferay.portal.kernel.ldap New:
com.liferay.portal.security.ldap

com.liferay com.liferay.portal.security.ldap.api 2.0.8

DuplicateNodeNameException

Old: com.liferay.portlet.wiki New: com.liferay.wiki.exception

com.liferay com.liferay.wiki.api 4.0.7

DuplicatePageException

Old: com.liferay.portlet.wiki New: com.liferay.wiki.exception

com.liferay com.liferay.wiki.api 4.0.7

DuplicateRuleGroupInstanceException

Old: com.liferay.portlet.mobiledevicerules New:
com.liferay.mobile.device.rules.exception

com.liferay com.liferay.mobile.device.rules.api 4.0.4

DuplicateVoteException

Old: com.liferay.portlet.polls New: com.liferay.polls.exception

com.liferay com.liferay.polls.api 6.0.3

EntryDisplayDateComparator

Old: com.liferay.portlet.blogs.util.comparator New:
com.liferay.blogs.util.comparator

com.liferay com.liferay.blogs.api 5.0.5

EntryModifiedDateComparator

Old: com.liferay.portlet.bookmarks.util.comparator New:
com.liferay.bookmarks.util.comparator

com.liferay com.liferay.bookmarks.api 4.0.5

EntryNameComparator

Old: com.liferay.portlet.bookmarks.util.comparator New:
com.liferay.bookmarks.util.comparator

com.liferay com.liferay.bookmarks.api 4.0.5

EntryPriorityComparator

Old: com.liferay.portlet.bookmarks.util.comparator New:
com.liferay.bookmarks.util.comparator

com.liferay com.liferay.bookmarks.api 4.0.5

EntrySmallImageNameException

Old: com.liferay.portlet.blogs New: com.liferay.blogs.exception

com.liferay com.liferay.blogs.api 5.0.5

EntryURLComparator

Old: com.liferay.portlet.bookmarks.util.comparator New:
com.liferay.bookmarks.util.comparator

com.liferay com.liferay.bookmarks.api 4.0.5

EntryVisitsComparator

Old: com.liferay.portlet.bookmarks.util.comparator New:
com.liferay.bookmarks.util.comparator

com.liferay com.liferay.bookmarks.api 4.0.5

EqualityWeakReference

Old: com.liferay.portal.kernel.memory New: com.liferay.petra.memory

com.liferay com.liferay.petra.memory 3.0.1

Fact

Old: com.liferay.portal.kernel.bi.rules New:
com.liferay.portal.rules.engine

com.liferay com.liferay.portal.rules.engine.api 4.0.4

FeedContentFieldException

Old: com.liferay.portlet.journal New: com.liferay.journal.exception

com.liferay com.liferay.journal.api 4.2.1

FeedIdException

Old: com.liferay.portlet.journal New: com.liferay.journal.exception

com.liferay com.liferay.journal.api 4.2.1

FeedNameException

Old: com.liferay.portlet.journal New: com.liferay.journal.exception

com.liferay com.liferay.journal.api 4.2.1

FeedTargetLayoutFriendlyUrlException

Old: com.liferay.portlet.journal New: com.liferay.journal.exception

com.liferay com.liferay.journal.api 4.2.1

FeedTargetPortletIdException

Old: com.liferay.portlet.journal New: com.liferay.journal.exception

com.liferay com.liferay.journal.api 4.2.1

FieldConstants

Old: com.liferay.portlet.dynamicdatamapping.storage New:
com.liferay.dynamic.data.mapping.storage

com.liferay com.liferay.dynamic.data.mapping.api 5.2.1

FieldRenderer

Old: com.liferay.portlet.dynamicdatamapping.storage New:
com.liferay.dynamic.data.mapping.storage

com.liferay com.liferay.dynamic.data.mapping.api 5.2.1

FieldRendererFactory

Old: com.liferay.portlet.dynamicdatamapping.storage New:
com.liferay.dynamic.data.mapping.storage

com.liferay com.liferay.dynamic.data.mapping.api 5.2.1

Fields

Old: com.liferay.portlet.dynamicdatamapping.storage New:
com.liferay.dynamic.data.mapping.storage

com.liferay com.liferay.dynamic.data.mapping.api 5.2.1

FileRankCreateDateComparator

Old: com.liferay.portlet.documentlibrary.util.comparator New:
com.liferay.document.library.file.rank.util.comparator

com.liferay com.liferay.document.library.file.rank.service 2.0.6

FinalizeAction

Old: com.liferay.portal.kernel.memory New: com.liferay.petra.memory

com.liferay com.liferay.petra.memory 3.0.1

FinalizeManager

Old: com.liferay.portal.kernel.memory New: com.liferay.petra.memory

com.liferay com.liferay.petra.memory 3.0.1

FlagsEntryService

Old: com.liferay.portlet.flags.service New: com.liferay.flags.service

com.liferay com.liferay.flags.api 4.0.6

FlagsEntryServiceUtil

Old: com.liferay.portlet.flags.service New: com.liferay.flags.service

com.liferay com.liferay.flags.api 4.0.6

FlagsEntryServiceWrapper

Old: com.liferay.portlet.flags.service New: com.liferay.flags.service

com.liferay com.liferay.flags.api 4.0.6

FlagsRequest

Old: com.liferay.portlet.flags.messaging New:
com.liferay.flags.internal.messaging

com.liferay com.liferay.flags.service 4.0.2

GroupConverterKeys

Old: com.liferay.portal.security.ldap New:
com.liferay.portal.security.ldap

com.liferay com.liferay.portal.security.ldap.api 2.0.8

ImportFilesException

Old: com.liferay.portlet.wiki New: com.liferay.wiki.exception

com.liferay com.liferay.wiki.api 4.0.7

JournalArticle

Old: com.liferay.portlet.journal.model New: com.liferay.journal.model

com.liferay com.liferay.journal.api 4.2.1

JournalArticleConstants

Old: com.liferay.portlet.journal.model New: com.liferay.journal.model

com.liferay com.liferay.journal.api 4.2.1

JournalArticleDisplay

Old: com.liferay.portlet.journal.model New: com.liferay.journal.model

com.liferay com.liferay.journal.api 4.2.1

JournalArticleFinder

Old: com.liferay.portlet.journal.service.persistence New:
com.liferay.journal.service.persistence

com.liferay com.liferay.journal.api 4.2.1

JournalArticleLocalService

Old: com.liferay.portlet.journal.service New:
com.liferay.journal.service

com.liferay com.liferay.journal.api 4.2.1

JournalArticleLocalServiceUtil

Old: com.liferay.portlet.journal.service New:
com.liferay.journal.service

com.liferay com.liferay.journal.api 4.2.1

JournalArticleLocalServiceWrapper

Old: com.liferay.portlet.journal.service New:
com.liferay.journal.service

com.liferay com.liferay.journal.api 4.2.1

JournalArticleModel

Old: com.liferay.portlet.journal.model New: com.liferay.journal.model

com.liferay com.liferay.journal.api 4.2.1

JournalArticlePersistence

Old: com.liferay.portlet.journal.service.persistence New:
com.liferay.journal.service.persistence

com.liferay com.liferay.journal.api 4.2.1

JournalArticleResource

Old: com.liferay.portlet.journal.model New: com.liferay.journal.model

com.liferay com.liferay.journal.api 4.2.1

JournalArticleResourceLocalService

Old: com.liferay.portlet.journal.service New:
com.liferay.journal.service

com.liferay com.liferay.journal.api 4.2.1

JournalArticleResourceLocalServiceUtil

Old: com.liferay.portlet.journal.service New:
com.liferay.journal.service

com.liferay com.liferay.journal.api 4.2.1

JournalArticleResourceLocalServiceWrapper

Old: com.liferay.portlet.journal.service New:
com.liferay.journal.service

com.liferay com.liferay.journal.api 4.2.1

JournalArticleResourceModel

Old: com.liferay.portlet.journal.model New: com.liferay.journal.model

com.liferay com.liferay.journal.api 4.2.1

JournalArticleResourcePersistence

Old: com.liferay.portlet.journal.service.persistence New:
com.liferay.journal.service.persistence

com.liferay com.liferay.journal.api 4.2.1

JournalArticleResourceSoap

Old: com.liferay.portlet.journal.model New: com.liferay.journal.model

com.liferay com.liferay.journal.api 4.2.1

JournalArticleResourceUtil

Old: com.liferay.portlet.journal.service.persistence New:
com.liferay.journal.service.persistence

com.liferay com.liferay.journal.api 4.2.1

JournalArticleResourceWrapper

Old: com.liferay.portlet.journal.model New: com.liferay.journal.model

com.liferay com.liferay.journal.api 4.2.1

JournalArticleService

Old: com.liferay.portlet.journal.service New:
com.liferay.journal.service

com.liferay com.liferay.journal.api 4.2.1

JournalArticleServiceUtil

Old: com.liferay.portlet.journal.service New:
com.liferay.journal.service

com.liferay com.liferay.journal.api 4.2.1

JournalArticleServiceWrapper

Old: com.liferay.portlet.journal.service New:
com.liferay.journal.service

com.liferay com.liferay.journal.api 4.2.1

JournalArticleSoap

Old: com.liferay.portlet.journal.model New: com.liferay.journal.model

com.liferay com.liferay.journal.api 4.2.1

JournalArticleUtil

Old: com.liferay.portlet.journal.service.persistence New:
com.liferay.journal.service.persistence

com.liferay com.liferay.journal.api 4.2.1

JournalArticleWrapper

Old: com.liferay.portlet.journal.model New: com.liferay.journal.model

com.liferay com.liferay.journal.api 4.2.1

JournalContent

Old: com.liferay.portlet.journalcontent.util New:
com.liferay.journal.util

com.liferay com.liferay.journal.api 4.2.1

JournalContentSearch

Old: com.liferay.portlet.journal.model New: com.liferay.journal.model

com.liferay com.liferay.journal.api 4.2.1

JournalContentSearchLocalService

Old: com.liferay.portlet.journal.service New:
com.liferay.journal.service

com.liferay com.liferay.journal.api 4.2.1

JournalContentSearchLocalServiceUtil

Old: com.liferay.portlet.journal.service New:
com.liferay.journal.service

com.liferay com.liferay.journal.api 4.2.1

JournalContentSearchLocalServiceWrapper

Old: com.liferay.portlet.journal.service New:
com.liferay.journal.service

com.liferay com.liferay.journal.api 4.2.1

JournalContentSearchModel

Old: com.liferay.portlet.journal.model New: com.liferay.journal.model

com.liferay com.liferay.journal.api 4.2.1

JournalContentSearchPersistence

Old: com.liferay.portlet.journal.service.persistence New:
com.liferay.journal.service.persistence

com.liferay com.liferay.journal.api 4.2.1

JournalContentSearchSoap

Old: com.liferay.portlet.journal.model New: com.liferay.journal.model

com.liferay com.liferay.journal.api 4.2.1

JournalContentSearchUtil

Old: com.liferay.portlet.journal.service.persistence New:
com.liferay.journal.service.persistence

com.liferay com.liferay.journal.api 4.2.1

JournalContentSearchWrapper

Old: com.liferay.portlet.journal.model New: com.liferay.journal.model

com.liferay com.liferay.journal.api 4.2.1

JournalConverter

Old: com.liferay.portlet.journal.util New: com.liferay.journal.util

com.liferay com.liferay.journal.api 4.2.1

JournalFeed

Old: com.liferay.portlet.journal.model New: com.liferay.journal.model

com.liferay com.liferay.journal.api 4.2.1

JournalFeedConstants

Old: com.liferay.portlet.journal.model New: com.liferay.journal.model

com.liferay com.liferay.journal.api 4.2.1

JournalFeedFinder

Old: com.liferay.portlet.journal.service.persistence New:
com.liferay.journal.service.persistence

com.liferay com.liferay.journal.api 4.2.1

JournalFeedLocalService

Old: com.liferay.portlet.journal.service New:
com.liferay.journal.service

com.liferay com.liferay.journal.api 4.2.1

JournalFeedLocalServiceUtil

Old: com.liferay.portlet.journal.service New:
com.liferay.journal.service

com.liferay com.liferay.journal.api 4.2.1

JournalFeedLocalServiceWrapper

Old: com.liferay.portlet.journal.service New:
com.liferay.journal.service

com.liferay com.liferay.journal.api 4.2.1

JournalFeedModel

Old: com.liferay.portlet.journal.model New: com.liferay.journal.model

com.liferay com.liferay.journal.api 4.2.1

JournalFeedPersistence

Old: com.liferay.portlet.journal.service.persistence New:
com.liferay.journal.service.persistence

com.liferay com.liferay.journal.api 4.2.1

JournalFeedService

Old: com.liferay.portlet.journal.service New:
com.liferay.journal.service

com.liferay com.liferay.journal.api 4.2.1

JournalFeedServiceUtil

Old: com.liferay.portlet.journal.service New:
com.liferay.journal.service

com.liferay com.liferay.journal.api 4.2.1

JournalFeedServiceWrapper

Old: com.liferay.portlet.journal.service New:
com.liferay.journal.service

com.liferay com.liferay.journal.api 4.2.1

JournalFeedSoap

Old: com.liferay.portlet.journal.model New: com.liferay.journal.model

com.liferay com.liferay.journal.api 4.2.1

JournalFeedUtil

Old: com.liferay.portlet.journal.service.persistence New:
com.liferay.journal.service.persistence

com.liferay com.liferay.journal.api 4.2.1

JournalFeedWrapper

Old: com.liferay.portlet.journal.model New: com.liferay.journal.model

com.liferay com.liferay.journal.api 4.2.1

JournalFolder

Old: com.liferay.portlet.journal.model New: com.liferay.journal.model

com.liferay com.liferay.journal.api 4.2.1

JournalFolderFinder

Old: com.liferay.portlet.journal.service.persistence New:
com.liferay.journal.service.persistence

com.liferay com.liferay.journal.api 4.2.1

JournalFolderLocalService

Old: com.liferay.portlet.journal.service New:
com.liferay.journal.service

com.liferay com.liferay.journal.api 4.2.1

JournalFolderLocalServiceUtil

Old: com.liferay.portlet.journal.service New:
com.liferay.journal.service

com.liferay com.liferay.journal.api 4.2.1

JournalFolderLocalServiceWrapper

Old: com.liferay.portlet.journal.service New:
com.liferay.journal.service

com.liferay com.liferay.journal.api 4.2.1

JournalFolderModel

Old: com.liferay.portlet.journal.model New: com.liferay.journal.model

com.liferay com.liferay.journal.api 4.2.1

JournalFolderPersistence

Old: com.liferay.portlet.journal.service.persistence New:
com.liferay.journal.service.persistence

com.liferay com.liferay.journal.api 4.2.1

JournalFolderService

Old: com.liferay.portlet.journal.service New:
com.liferay.journal.service

com.liferay com.liferay.journal.api 4.2.1

JournalFolderServiceUtil

Old: com.liferay.portlet.journal.service New:
com.liferay.journal.service

com.liferay com.liferay.journal.api 4.2.1

JournalFolderServiceWrapper

Old: com.liferay.portlet.journal.service New:
com.liferay.journal.service

com.liferay com.liferay.journal.api 4.2.1

JournalFolderSoap

Old: com.liferay.portlet.journal.model New: com.liferay.journal.model

com.liferay com.liferay.journal.api 4.2.1

JournalFolderUtil

Old: com.liferay.portlet.journal.service.persistence New:
com.liferay.journal.service.persistence

com.liferay com.liferay.journal.api 4.2.1

JournalFolderWrapper

Old: com.liferay.portlet.journal.model New: com.liferay.journal.model

com.liferay com.liferay.journal.api 4.2.1

JournalSearchConstants

Old: com.liferay.portlet.journal.model New: com.liferay.journal.model

com.liferay com.liferay.journal.api 4.2.1

JournalStructureConstants

Old: com.liferay.portlet.journal.model New: com.liferay.journal.model

com.liferay com.liferay.journal.api 4.2.1

LDAPFilterException

Old: com.liferay.portal.kernel.ldap New:
com.liferay.portal.security.ldap.validator

com.liferay com.liferay.portal.security.ldap.api 2.0.8

LDAPGroup

Old: com.liferay.portal.security.ldap New:
com.liferay.portal.security.ldap.exportimport

com.liferay com.liferay.portal.security.ldap.api 2.0.8

LDAPServerNameException

Old: com.liferay.portal.kernel.ldap New:
com.liferay.portal.security.ldap

com.liferay com.liferay.portal.security.ldap.api 2.0.8

LDAPToPortalConverter

Old: com.liferay.portal.security.ldap New:
com.liferay.portal.security.ldap.exportimport

com.liferay com.liferay.portal.security.ldap.api 2.0.8

LDAPUser

Old: com.liferay.portal.security.ldap New:
com.liferay.portal.security.ldap.exportimport

com.liferay com.liferay.portal.security.ldap.api 2.0.8

LDAPUtil

Old: com.liferay.portal.kernel.ldap New:
com.liferay.portal.security.ldap.util

com.liferay com.liferay.portal.security.ldap.api 2.0.8

LockLocalService

Old: com.liferay.portal.service New: com.liferay.portal.lock.service

com.liferay com.liferay.portal.lock.api 4.1.1

LockLocalServiceUtil

Old: com.liferay.portal.service New: com.liferay.portal.lock.service

com.liferay com.liferay.portal.lock.api 4.1.1

LockLocalServiceWrapper

Old: com.liferay.portal.service New: com.liferay.portal.lock.service

com.liferay com.liferay.portal.lock.api 4.1.1

LockModel

Old: com.liferay.portal.model New: com.liferay.portal.lock.model

com.liferay com.liferay.portal.lock.api 4.1.1

LockPersistence

Old: com.liferay.portal.service.persistence New:
com.liferay.portal.lock.service.persistence

com.liferay com.liferay.portal.lock.api 4.1.1

LockSoap

Old: com.liferay.portal.model New: com.liferay.portal.lock.model

com.liferay com.liferay.portal.lock.api 4.1.1

LockUtil

Old: com.liferay.portal.service.persistence New:
com.liferay.portal.lock.service.persistence

com.liferay com.liferay.portal.lock.api 4.1.1

LockWrapper

Old: com.liferay.portal.model New: com.liferay.portal.lock.model

com.liferay com.liferay.portal.lock.api 4.1.1

LockedThreadException

Old: com.liferay.portlet.messageboards New:
com.liferay.message.boards.exception

com.liferay com.liferay.message.boards.api 5.1.4

LoggingOutputProcessor

Old: com.liferay.portal.kernel.process New: com.liferay.petra.process

com.liferay com.liferay.petra.process 3.0.4

MBBan

Old: com.liferay.portlet.messageboards.model New:
com.liferay.message.boards.model

com.liferay com.liferay.message.boards.api 5.1.4

MBBanLocalService

Old: com.liferay.portlet.messageboards.service New:
com.liferay.message.boards.service

com.liferay com.liferay.message.boards.api 5.1.4

MBBanLocalServiceUtil

Old: com.liferay.portlet.messageboards.service New:
com.liferay.message.boards.service

com.liferay com.liferay.message.boards.api 5.1.4

MBBanLocalServiceWrapper

Old: com.liferay.portlet.messageboards.service New:
com.liferay.message.boards.service

com.liferay com.liferay.message.boards.api 5.1.4

MBBanModel

Old: com.liferay.portlet.messageboards.model New:
com.liferay.message.boards.model

com.liferay com.liferay.message.boards.api 5.1.4

MBBanPersistence

Old: com.liferay.portlet.messageboards.service.persistence New:
com.liferay.message.boards.service.persistence

com.liferay com.liferay.message.boards.api 5.1.4

MBBanService

Old: com.liferay.portlet.messageboards.service New:
com.liferay.message.boards.service

com.liferay com.liferay.message.boards.api 5.1.4

MBBanServiceUtil

Old: com.liferay.portlet.messageboards.service New:
com.liferay.message.boards.service

com.liferay com.liferay.message.boards.api 5.1.4

MBBanServiceWrapper

Old: com.liferay.portlet.messageboards.service New:
com.liferay.message.boards.service

com.liferay com.liferay.message.boards.api 5.1.4

MBBanSoap

Old: com.liferay.portlet.messageboards.model New:
com.liferay.message.boards.model

com.liferay com.liferay.message.boards.api 5.1.4

MBBanUtil

Old: com.liferay.portlet.messageboards.service.persistence New:
com.liferay.message.boards.service.persistence

com.liferay com.liferay.message.boards.api 5.1.4

MBBanWrapper

Old: com.liferay.portlet.messageboards.model New:
com.liferay.message.boards.model

com.liferay com.liferay.message.boards.api 5.1.4

MBCategory

Old: com.liferay.portlet.messageboards.model New:
com.liferay.message.boards.model

com.liferay com.liferay.message.boards.api 5.1.4

MBCategoryConstants

Old: com.liferay.portlet.messageboards.model New:
com.liferay.message.boards.constants

com.liferay com.liferay.message.boards.api 5.1.4

MBCategoryDisplay

Old: com.liferay.portlet.messageboards.model New:
com.liferay.message.boards.web.internal.display

com.liferay com.liferay.message.boards.web 3.0.17

MBCategoryFinder

Old: com.liferay.portlet.messageboards.service.persistence New:
com.liferay.message.boards.service.persistence

com.liferay com.liferay.message.boards.api 5.1.4

MBCategoryLocalService

Old: com.liferay.portlet.messageboards.service New:
com.liferay.message.boards.service

com.liferay com.liferay.message.boards.api 5.1.4

MBCategoryLocalServiceUtil

Old: com.liferay.portlet.messageboards.service New:
com.liferay.message.boards.service

com.liferay com.liferay.message.boards.api 5.1.4

MBCategoryLocalServiceWrapper

Old: com.liferay.portlet.messageboards.service New:
com.liferay.message.boards.service

com.liferay com.liferay.message.boards.api 5.1.4

MBCategoryModel

Old: com.liferay.portlet.messageboards.model New:
com.liferay.message.boards.model

com.liferay com.liferay.message.boards.api 5.1.4

MBCategoryPersistence

Old: com.liferay.portlet.messageboards.service.persistence New:
com.liferay.message.boards.service.persistence

com.liferay com.liferay.message.boards.api 5.1.4

MBCategoryService

Old: com.liferay.portlet.messageboards.service New:
com.liferay.message.boards.service

com.liferay com.liferay.message.boards.api 5.1.4

MBCategoryServiceUtil

Old: com.liferay.portlet.messageboards.service New:
com.liferay.message.boards.service

com.liferay com.liferay.message.boards.api 5.1.4

MBCategoryServiceWrapper

Old: com.liferay.portlet.messageboards.service New:
com.liferay.message.boards.service

com.liferay com.liferay.message.boards.api 5.1.4

MBCategorySoap

Old: com.liferay.portlet.messageboards.model New:
com.liferay.message.boards.model

com.liferay com.liferay.message.boards.api 5.1.4

MBCategoryUtil

Old: com.liferay.portlet.messageboards.service.persistence New:
com.liferay.message.boards.service.persistence

com.liferay com.liferay.message.boards.api 5.1.4

MBCategoryWrapper

Old: com.liferay.portlet.messageboards.model New:
com.liferay.message.boards.model

com.liferay com.liferay.message.boards.api 5.1.4

MBDiscussion

Old: com.liferay.portlet.messageboards.model New:
com.liferay.message.boards.model

com.liferay com.liferay.message.boards.api 5.1.4

MBDiscussionLocalService

Old: com.liferay.portlet.messageboards.service New:
com.liferay.message.boards.service

com.liferay com.liferay.message.boards.api 5.1.4

MBDiscussionLocalServiceUtil

Old: com.liferay.portlet.messageboards.service New:
com.liferay.message.boards.service

com.liferay com.liferay.message.boards.api 5.1.4

MBDiscussionLocalServiceWrapper

Old: com.liferay.portlet.messageboards.service New:
com.liferay.message.boards.service

com.liferay com.liferay.message.boards.api 5.1.4

MBDiscussionModel

Old: com.liferay.portlet.messageboards.model New:
com.liferay.message.boards.model

com.liferay com.liferay.message.boards.api 5.1.4

MBDiscussionPersistence

Old: com.liferay.portlet.messageboards.service.persistence New:
com.liferay.message.boards.service.persistence

com.liferay com.liferay.message.boards.api 5.1.4

MBDiscussionSoap

Old: com.liferay.portlet.messageboards.model New:
com.liferay.message.boards.model

com.liferay com.liferay.message.boards.api 5.1.4

MBDiscussionUtil

Old: com.liferay.portlet.messageboards.service.persistence New:
com.liferay.message.boards.service.persistence

com.liferay com.liferay.message.boards.api 5.1.4

MBDiscussionWrapper

Old: com.liferay.portlet.messageboards.model New:
com.liferay.message.boards.model

com.liferay com.liferay.message.boards.api 5.1.4

MBMailingList

Old: com.liferay.portlet.messageboards.model New:
com.liferay.message.boards.model

com.liferay com.liferay.message.boards.api 5.1.4

MBMailingListLocalService

Old: com.liferay.portlet.messageboards.service New:
com.liferay.message.boards.service

com.liferay com.liferay.message.boards.api 5.1.4

MBMailingListLocalServiceUtil

Old: com.liferay.portlet.messageboards.service New:
com.liferay.message.boards.service

com.liferay com.liferay.message.boards.api 5.1.4

MBMailingListLocalServiceWrapper

Old: com.liferay.portlet.messageboards.service New:
com.liferay.message.boards.service

com.liferay com.liferay.message.boards.api 5.1.4

MBMailingListModel

Old: com.liferay.portlet.messageboards.model New:
com.liferay.message.boards.model

com.liferay com.liferay.message.boards.api 5.1.4

MBMailingListPersistence

Old: com.liferay.portlet.messageboards.service.persistence New:
com.liferay.message.boards.service.persistence

com.liferay com.liferay.message.boards.api 5.1.4

MBMailingListSoap

Old: com.liferay.portlet.messageboards.model New:
com.liferay.message.boards.model

com.liferay com.liferay.message.boards.api 5.1.4

MBMailingListUtil

Old: com.liferay.portlet.messageboards.service.persistence New:
com.liferay.message.boards.service.persistence

com.liferay com.liferay.message.boards.api 5.1.4

MBMailingListWrapper

Old: com.liferay.portlet.messageboards.model New:
com.liferay.message.boards.model

com.liferay com.liferay.message.boards.api 5.1.4

MBMessage

Old: com.liferay.portlet.messageboards.model New:
com.liferay.message.boards.model

com.liferay com.liferay.message.boards.api 5.1.4

MBMessageConstants

Old: com.liferay.portlet.messageboards.model New:
com.liferay.message.boards.constants

com.liferay com.liferay.message.boards.api 5.1.4

MBMessageDisplay

Old: com.liferay.portlet.messageboards.model New:
com.liferay.message.boards.model

com.liferay com.liferay.message.boards.api 5.1.4

MBMessageFinder

Old: com.liferay.portlet.messageboards.service.persistence New:
com.liferay.message.boards.service.persistence

com.liferay com.liferay.message.boards.api 5.1.4

MBMessageLocalService

Old: com.liferay.portlet.messageboards.service New:
com.liferay.message.boards.service

com.liferay com.liferay.message.boards.api 5.1.4

MBMessageLocalServiceUtil

Old: com.liferay.portlet.messageboards.service New:
com.liferay.message.boards.service

com.liferay com.liferay.message.boards.api 5.1.4

MBMessageLocalServiceWrapper

Old: com.liferay.portlet.messageboards.service New:
com.liferay.message.boards.service

com.liferay com.liferay.message.boards.api 5.1.4

MBMessageModel

Old: com.liferay.portlet.messageboards.model New:
com.liferay.message.boards.model

com.liferay com.liferay.message.boards.api 5.1.4

MBMessagePersistence

Old: com.liferay.portlet.messageboards.service.persistence New:
com.liferay.message.boards.service.persistence

com.liferay com.liferay.message.boards.api 5.1.4

MBMessageService

Old: com.liferay.portlet.messageboards.service New:
com.liferay.message.boards.service

com.liferay com.liferay.message.boards.api 5.1.4

MBMessageServiceUtil

Old: com.liferay.portlet.messageboards.service New:
com.liferay.message.boards.service

com.liferay com.liferay.message.boards.api 5.1.4

MBMessageServiceWrapper

Old: com.liferay.portlet.messageboards.service New:
com.liferay.message.boards.service

com.liferay com.liferay.message.boards.api 5.1.4

MBMessageSoap

Old: com.liferay.portlet.messageboards.model New:
com.liferay.message.boards.model

com.liferay com.liferay.message.boards.api 5.1.4

MBMessageUtil

Old: com.liferay.portlet.messageboards.service.persistence New:
com.liferay.message.boards.service.persistence

com.liferay com.liferay.message.boards.api 5.1.4

MBMessageWrapper

Old: com.liferay.portlet.messageboards.model New:
com.liferay.message.boards.model

com.liferay com.liferay.message.boards.api 5.1.4

MBStatsUser

Old: com.liferay.portlet.messageboards.model New:
com.liferay.message.boards.model

com.liferay com.liferay.message.boards.api 5.1.4

MBStatsUserLocalService

Old: com.liferay.portlet.messageboards.service New:
com.liferay.message.boards.service

com.liferay com.liferay.message.boards.api 5.1.4

MBStatsUserLocalServiceUtil

Old: com.liferay.portlet.messageboards.service New:
com.liferay.message.boards.service

com.liferay com.liferay.message.boards.api 5.1.4

MBStatsUserLocalServiceWrapper

Old: com.liferay.portlet.messageboards.service New:
com.liferay.message.boards.service

com.liferay com.liferay.message.boards.api 5.1.4

MBStatsUserModel

Old: com.liferay.portlet.messageboards.model New:
com.liferay.message.boards.model

com.liferay com.liferay.message.boards.api 5.1.4

MBStatsUserPersistence

Old: com.liferay.portlet.messageboards.service.persistence New:
com.liferay.message.boards.service.persistence

com.liferay com.liferay.message.boards.api 5.1.4

MBStatsUserSoap

Old: com.liferay.portlet.messageboards.model New:
com.liferay.message.boards.model

com.liferay com.liferay.message.boards.api 5.1.4

MBStatsUserUtil

Old: com.liferay.portlet.messageboards.service.persistence New:
com.liferay.message.boards.service.persistence

com.liferay com.liferay.message.boards.api 5.1.4

MBStatsUserWrapper

Old: com.liferay.portlet.messageboards.model New:
com.liferay.message.boards.model

com.liferay com.liferay.message.boards.api 5.1.4

MBThread

Old: com.liferay.portlet.messageboards.model New:
com.liferay.message.boards.model

com.liferay com.liferay.message.boards.api 5.1.4

MBThreadConstants

Old: com.liferay.portlet.messageboards.model New:
com.liferay.message.boards.constants

com.liferay com.liferay.message.boards.api 5.1.4

MBThreadFinder

Old: com.liferay.portlet.messageboards.service.persistence New:
com.liferay.message.boards.service.persistence

com.liferay com.liferay.message.boards.api 5.1.4

MBThreadFlag

Old: com.liferay.portlet.messageboards.model New:
com.liferay.message.boards.model

com.liferay com.liferay.message.boards.api 5.1.4

MBThreadFlagLocalService

Old: com.liferay.portlet.messageboards.service New:
com.liferay.message.boards.service

com.liferay com.liferay.message.boards.api 5.1.4

MBThreadFlagLocalServiceUtil

Old: com.liferay.portlet.messageboards.service New:
com.liferay.message.boards.service

com.liferay com.liferay.message.boards.api 5.1.4

MBThreadFlagLocalServiceWrapper

Old: com.liferay.portlet.messageboards.service New:
com.liferay.message.boards.service

com.liferay com.liferay.message.boards.api 5.1.4

MBThreadFlagModel

Old: com.liferay.portlet.messageboards.model New:
com.liferay.message.boards.model

com.liferay com.liferay.message.boards.api 5.1.4

MBThreadFlagPersistence

Old: com.liferay.portlet.messageboards.service.persistence New:
com.liferay.message.boards.service.persistence

com.liferay com.liferay.message.boards.api 5.1.4

MBThreadFlagSoap

Old: com.liferay.portlet.messageboards.model New:
com.liferay.message.boards.model

com.liferay com.liferay.message.boards.api 5.1.4

MBThreadFlagUtil

Old: com.liferay.portlet.messageboards.service.persistence New:
com.liferay.message.boards.service.persistence

com.liferay com.liferay.message.boards.api 5.1.4

MBThreadFlagWrapper

Old: com.liferay.portlet.messageboards.model New:
com.liferay.message.boards.model

com.liferay com.liferay.message.boards.api 5.1.4

MBThreadLocalService

Old: com.liferay.portlet.messageboards.service New:
com.liferay.message.boards.service

com.liferay com.liferay.message.boards.api 5.1.4

MBThreadLocalServiceUtil

Old: com.liferay.portlet.messageboards.service New:
com.liferay.message.boards.service

com.liferay com.liferay.message.boards.api 5.1.4

MBThreadLocalServiceWrapper

Old: com.liferay.portlet.messageboards.service New:
com.liferay.message.boards.service

com.liferay com.liferay.message.boards.api 5.1.4

MBThreadModel

Old: com.liferay.portlet.messageboards.model New:
com.liferay.message.boards.model

com.liferay com.liferay.message.boards.api 5.1.4

MBThreadPersistence

Old: com.liferay.portlet.messageboards.service.persistence New:
com.liferay.message.boards.service.persistence

com.liferay com.liferay.message.boards.api 5.1.4

MBThreadService

Old: com.liferay.portlet.messageboards.service New:
com.liferay.message.boards.service

com.liferay com.liferay.message.boards.api 5.1.4

MBThreadServiceUtil

Old: com.liferay.portlet.messageboards.service New:
com.liferay.message.boards.service

com.liferay com.liferay.message.boards.api 5.1.4

MBThreadServiceWrapper

Old: com.liferay.portlet.messageboards.service New:
com.liferay.message.boards.service

com.liferay com.liferay.message.boards.api 5.1.4

MBThreadSoap

Old: com.liferay.portlet.messageboards.model New:
com.liferay.message.boards.model

com.liferay com.liferay.message.boards.api 5.1.4

MBThreadUtil

Old: com.liferay.portlet.messageboards.service.persistence New:
com.liferay.message.boards.service.persistence

com.liferay com.liferay.message.boards.api 5.1.4

MBThreadWrapper

Old: com.liferay.portlet.messageboards.model New:
com.liferay.message.boards.model

com.liferay com.liferay.message.boards.api 5.1.4

MBTreeWalker

Old: com.liferay.portlet.messageboards.model New:
com.liferay.message.boards.model

com.liferay com.liferay.message.boards.api 5.1.4

MBeanRegistry

Old: com.liferay.portal.kernel.jmx New: com.liferay.portal.jmx

com.liferay com.liferay.portal.jmx.api 3.0.1

MDRAction

Old: com.liferay.portlet.mobiledevicerules.model New:
com.liferay.mobile.device.rules.model

com.liferay com.liferay.mobile.device.rules.api 4.0.4

MDRActionLocalService

Old: com.liferay.portlet.mobiledevicerules.service New:
com.liferay.mobile.device.rules.service

com.liferay com.liferay.mobile.device.rules.api 4.0.4

MDRActionLocalServiceUtil

Old: com.liferay.portlet.mobiledevicerules.service New:
com.liferay.mobile.device.rules.service

com.liferay com.liferay.mobile.device.rules.api 4.0.4

MDRActionLocalServiceWrapper

Old: com.liferay.portlet.mobiledevicerules.service New:
com.liferay.mobile.device.rules.service

com.liferay com.liferay.mobile.device.rules.api 4.0.4

MDRActionModel

Old: com.liferay.portlet.mobiledevicerules.model New:
com.liferay.mobile.device.rules.model

com.liferay com.liferay.mobile.device.rules.api 4.0.4

MDRActionPersistence

Old: com.liferay.portlet.mobiledevicerules.service.persistence New:
com.liferay.mobile.device.rules.service.persistence

com.liferay com.liferay.mobile.device.rules.api 4.0.4

MDRActionService

Old: com.liferay.portlet.mobiledevicerules.service New:
com.liferay.mobile.device.rules.service

com.liferay com.liferay.mobile.device.rules.api 4.0.4

MDRActionServiceUtil

Old: com.liferay.portlet.mobiledevicerules.service New:
com.liferay.mobile.device.rules.service

com.liferay com.liferay.mobile.device.rules.api 4.0.4

MDRActionServiceWrapper

Old: com.liferay.portlet.mobiledevicerules.service New:
com.liferay.mobile.device.rules.service

com.liferay com.liferay.mobile.device.rules.api 4.0.4

MDRActionSoap

Old: com.liferay.portlet.mobiledevicerules.model New:
com.liferay.mobile.device.rules.model

com.liferay com.liferay.mobile.device.rules.api 4.0.4

MDRActionUtil

Old: com.liferay.portlet.mobiledevicerules.service.persistence New:
com.liferay.mobile.device.rules.service.persistence

com.liferay com.liferay.mobile.device.rules.api 4.0.4

MDRActionWrapper

Old: com.liferay.portlet.mobiledevicerules.model New:
com.liferay.mobile.device.rules.model

com.liferay com.liferay.mobile.device.rules.api 4.0.4

MDRPermission

Old: com.liferay.portlet.mobiledevicerules.service.permission New:
com.liferay.mobile.device.rules.web.internal.security.permission.resource

com.liferay com.liferay.mobile.device.rules.web 3.0.6

MDRRule

Old: com.liferay.portlet.mobiledevicerules.model New:
com.liferay.mobile.device.rules.model

com.liferay com.liferay.mobile.device.rules.api 4.0.4

MDRRuleGroup

Old: com.liferay.portlet.mobiledevicerules.model New:
com.liferay.mobile.device.rules.model

com.liferay com.liferay.mobile.device.rules.api 4.0.4

MDRRuleGroupFinder

Old: com.liferay.portlet.mobiledevicerules.service.persistence New:
com.liferay.mobile.device.rules.service.persistence

com.liferay com.liferay.mobile.device.rules.api 4.0.4

MDRRuleGroupInstanceLocalService

Old: com.liferay.portlet.mobiledevicerules.service New:
com.liferay.mobile.device.rules.service

com.liferay com.liferay.mobile.device.rules.api 4.0.4

MDRRuleGroupInstanceLocalServiceUtil

Old: com.liferay.portlet.mobiledevicerules.service New:
com.liferay.mobile.device.rules.service

com.liferay com.liferay.mobile.device.rules.api 4.0.4

MDRRuleGroupInstanceLocalServiceWrapper

Old: com.liferay.portlet.mobiledevicerules.service New:
com.liferay.mobile.device.rules.service

com.liferay com.liferay.mobile.device.rules.api 4.0.4

MDRRuleGroupInstanceModel

Old: com.liferay.portlet.mobiledevicerules.model New:
com.liferay.mobile.device.rules.model

com.liferay com.liferay.mobile.device.rules.api 4.0.4

MDRRuleGroupInstancePermission

Old: com.liferay.portlet.mobiledevicerules.service.permission New:
com.liferay.mobile.device.rules.web.internal.security.permission.resource

com.liferay com.liferay.mobile.device.rules.web 3.0.6

MDRRuleGroupInstancePersistence

Old: com.liferay.portlet.mobiledevicerules.service.persistence New:
com.liferay.mobile.device.rules.service.persistence

com.liferay com.liferay.mobile.device.rules.api 4.0.4

MDRRuleGroupInstanceService

Old: com.liferay.portlet.mobiledevicerules.service New:
com.liferay.mobile.device.rules.service

com.liferay com.liferay.mobile.device.rules.api 4.0.4

MDRRuleGroupInstanceServiceUtil

Old: com.liferay.portlet.mobiledevicerules.service New:
com.liferay.mobile.device.rules.service

com.liferay com.liferay.mobile.device.rules.api 4.0.4

MDRRuleGroupInstanceServiceWrapper

Old: com.liferay.portlet.mobiledevicerules.service New:
com.liferay.mobile.device.rules.service

com.liferay com.liferay.mobile.device.rules.api 4.0.4

MDRRuleGroupInstanceSoap

Old: com.liferay.portlet.mobiledevicerules.model New:
com.liferay.mobile.device.rules.model

com.liferay com.liferay.mobile.device.rules.api 4.0.4

MDRRuleGroupInstanceUtil

Old: com.liferay.portlet.mobiledevicerules.service.persistence New:
com.liferay.mobile.device.rules.service.persistence

com.liferay com.liferay.mobile.device.rules.api 4.0.4

MDRRuleGroupInstanceWrapper

Old: com.liferay.portlet.mobiledevicerules.model New:
com.liferay.mobile.device.rules.model

com.liferay com.liferay.mobile.device.rules.api 4.0.4

MDRRuleGroupLocalService

Old: com.liferay.portlet.mobiledevicerules.service New:
com.liferay.mobile.device.rules.service

com.liferay com.liferay.mobile.device.rules.api 4.0.4

MDRRuleGroupLocalServiceUtil

Old: com.liferay.portlet.mobiledevicerules.service New:
com.liferay.mobile.device.rules.service

com.liferay com.liferay.mobile.device.rules.api 4.0.4

MDRRuleGroupLocalServiceWrapper

Old: com.liferay.portlet.mobiledevicerules.service New:
com.liferay.mobile.device.rules.service

com.liferay com.liferay.mobile.device.rules.api 4.0.4

MDRRuleGroupModel

Old: com.liferay.portlet.mobiledevicerules.model New:
com.liferay.mobile.device.rules.model

com.liferay com.liferay.mobile.device.rules.api 4.0.4

MDRRuleGroupPermission

Old: com.liferay.portlet.mobiledevicerules.service.permission New:
com.liferay.mobile.device.rules.web.internal.security.permission.resource

com.liferay com.liferay.mobile.device.rules.web 3.0.6

MDRRuleGroupPersistence

Old: com.liferay.portlet.mobiledevicerules.service.persistence New:
com.liferay.mobile.device.rules.service.persistence

com.liferay com.liferay.mobile.device.rules.api 4.0.4

MDRRuleGroupService

Old: com.liferay.portlet.mobiledevicerules.service New:
com.liferay.mobile.device.rules.service

com.liferay com.liferay.mobile.device.rules.api 4.0.4

MDRRuleGroupServiceUtil

Old: com.liferay.portlet.mobiledevicerules.service New:
com.liferay.mobile.device.rules.service

com.liferay com.liferay.mobile.device.rules.api 4.0.4

MDRRuleGroupServiceWrapper

Old: com.liferay.portlet.mobiledevicerules.service New:
com.liferay.mobile.device.rules.service

com.liferay com.liferay.mobile.device.rules.api 4.0.4

MDRRuleGroupSoap

Old: com.liferay.portlet.mobiledevicerules.model New:
com.liferay.mobile.device.rules.model

com.liferay com.liferay.mobile.device.rules.api 4.0.4

MDRRuleGroupUtil

Old: com.liferay.portlet.mobiledevicerules.service.persistence New:
com.liferay.mobile.device.rules.service.persistence

com.liferay com.liferay.mobile.device.rules.api 4.0.4

MDRRuleGroupWrapper

Old: com.liferay.portlet.mobiledevicerules.model New:
com.liferay.mobile.device.rules.model

com.liferay com.liferay.mobile.device.rules.api 4.0.4

MDRRuleLocalService

Old: com.liferay.portlet.mobiledevicerules.service New:
com.liferay.mobile.device.rules.service

com.liferay com.liferay.mobile.device.rules.api 4.0.4

MDRRuleLocalServiceUtil

Old: com.liferay.portlet.mobiledevicerules.service New:
com.liferay.mobile.device.rules.service

com.liferay com.liferay.mobile.device.rules.api 4.0.4

MDRRuleLocalServiceWrapper

Old: com.liferay.portlet.mobiledevicerules.service New:
com.liferay.mobile.device.rules.service

com.liferay com.liferay.mobile.device.rules.api 4.0.4

MDRRuleModel

Old: com.liferay.portlet.mobiledevicerules.model New:
com.liferay.mobile.device.rules.model

com.liferay com.liferay.mobile.device.rules.api 4.0.4

MDRRulePersistence

Old: com.liferay.portlet.mobiledevicerules.service.persistence New:
com.liferay.mobile.device.rules.service.persistence

com.liferay com.liferay.mobile.device.rules.api 4.0.4

MDRRuleService

Old: com.liferay.portlet.mobiledevicerules.service New:
com.liferay.mobile.device.rules.service

com.liferay com.liferay.mobile.device.rules.api 4.0.4

MDRRuleServiceUtil

Old: com.liferay.portlet.mobiledevicerules.service New:
com.liferay.mobile.device.rules.service

com.liferay com.liferay.mobile.device.rules.api 4.0.4

MDRRuleServiceWrapper

Old: com.liferay.portlet.mobiledevicerules.service New:
com.liferay.mobile.device.rules.service

com.liferay com.liferay.mobile.device.rules.api 4.0.4

MDRRuleSoap

Old: com.liferay.portlet.mobiledevicerules.model New:
com.liferay.mobile.device.rules.model

com.liferay com.liferay.mobile.device.rules.api 4.0.4

MDRRuleUtil

Old: com.liferay.portlet.mobiledevicerules.service.persistence New:
com.liferay.mobile.device.rules.service.persistence

com.liferay com.liferay.mobile.device.rules.api 4.0.4

MDRRuleWrapper

Old: com.liferay.portlet.mobiledevicerules.model New:
com.liferay.mobile.device.rules.model

com.liferay com.liferay.mobile.device.rules.api 4.0.4

MailingListEmailAddressException

Old: com.liferay.portlet.messageboards New:
com.liferay.message.boards.exception

com.liferay com.liferay.message.boards.api 5.1.4

MailingListInServerNameException

Old: com.liferay.portlet.messageboards New:
com.liferay.message.boards.exception

com.liferay com.liferay.message.boards.api 5.1.4

MailingListInUserNameException

Old: com.liferay.portlet.messageboards New:
com.liferay.message.boards.exception

com.liferay com.liferay.message.boards.api 5.1.4

MailingListOutEmailAddressException

Old: com.liferay.portlet.messageboards New:
com.liferay.message.boards.exception

com.liferay com.liferay.message.boards.api 5.1.4

MailingListOutServerNameException

Old: com.liferay.portlet.messageboards New:
com.liferay.message.boards.exception

com.liferay com.liferay.message.boards.api 5.1.4

MailingListOutUserNameException

Old: com.liferay.portlet.messageboards New:
com.liferay.message.boards.exception

com.liferay com.liferay.message.boards.api 5.1.4

MemoryReportDesignRetriever

Old: com.liferay.portal.kernel.bi.reporting New:
com.liferay.portal.reports.engine

com.liferay com.liferay.portal.reports.engine.api 5.0.1

MessageBodyException

Old: com.liferay.portlet.messageboards New:
com.liferay.message.boards.exception

com.liferay com.liferay.message.boards.api 5.1.4

MessageBusManager

Old: com.liferay.portal.kernel.messaging.jmx New:
com.liferay.portal.messaging.internal.jmx

com.liferay com.liferay.portal.messaging 6.0.5

MessageBusManagerMBean

Old: com.liferay.portal.kernel.messaging.jmx New:
com.liferay.portal.messaging.internal.jmx

com.liferay com.liferay.portal.messaging 6.0.5

MessageCreateDateComparator

Old: com.liferay.portlet.messageboards.util.comparator New:
com.liferay.message.boards.util.comparator

com.liferay com.liferay.message.boards.api 5.1.4

MessageSubjectException

Old: com.liferay.portlet.messageboards New:
com.liferay.message.boards.exception

com.liferay com.liferay.message.boards.api 5.1.4

MessageThreadComparator

Old: com.liferay.portlet.messageboards.util.comparator New:
com.liferay.message.boards.util.comparator

com.liferay com.liferay.message.boards.api 5.1.4

Modifications

Old: com.liferay.portal.security.ldap New:
com.liferay.portal.security.ldap.exportimport

com.liferay com.liferay.portal.security.ldap.api 2.0.8

NoSuchArticleException

Old: com.liferay.portlet.journal New: com.liferay.journal.exception

com.liferay com.liferay.journal.api 4.2.1

NoSuchArticleImageException

Old: com.liferay.portlet.journal New: com.liferay.journal.exception

com.liferay com.liferay.journal.api 4.2.1

NoSuchArticleResourceException

Old: com.liferay.portlet.journal New: com.liferay.journal.exception

com.liferay com.liferay.journal.api 4.2.1

NoSuchBanException

Old: com.liferay.portlet.messageboards New:
com.liferay.message.boards.exception

com.liferay com.liferay.message.boards.api 5.1.4

NoSuchChoiceException

Old: com.liferay.portlet.polls New: com.liferay.polls.exception

com.liferay com.liferay.polls.api 6.0.3

NoSuchContentException

Old: com.liferay.portlet.dynamicdatamapping New:
com.liferay.dynamic.data.mapping.exception

com.liferay com.liferay.dynamic.data.mapping.api 5.2.1

NoSuchContentSearchException

Old: com.liferay.portlet.journal New: com.liferay.journal.exception

com.liferay com.liferay.journal.api 4.2.1

NoSuchDiscussionException

Old: com.liferay.portlet.messageboards New:
com.liferay.message.boards.exception

com.liferay com.liferay.message.boards.api 5.1.4

NoSuchFeedException

Old: com.liferay.portlet.journal New: com.liferay.journal.exception

com.liferay com.liferay.journal.api 4.2.1

NoSuchFileRankException

Old: com.liferay.portlet.documentlibrary New:
com.liferay.document.library.file.rank.exception

com.liferay com.liferay.document.library.file.rank.api 2.0.3

NoSuchMailingListException

Old: com.liferay.portlet.messageboards New:
com.liferay.message.boards.exception

com.liferay com.liferay.message.boards.api 5.1.4

NoSuchNodeException

Old: com.liferay.portlet.wiki New: com.liferay.wiki.exception

com.liferay com.liferay.wiki.api 4.0.7

NoSuchPageException

Old: com.liferay.portlet.wiki New: com.liferay.wiki.exception

com.liferay com.liferay.wiki.api 4.0.7

NoSuchPageResourceException

Old: com.liferay.portlet.wiki New: com.liferay.wiki.exception

com.liferay com.liferay.wiki.api 4.0.7

NoSuchQuestionException

Old: com.liferay.portlet.polls New: com.liferay.polls.exception

com.liferay com.liferay.polls.api 6.0.3

NoSuchRecordException

Old: com.liferay.portlet.dynamicdatalists New:
com.liferay.dynamic.data.lists.exception

com.liferay com.liferay.dynamic.data.lists.api 4.0.5

NoSuchRecordSetException

Old: com.liferay.portlet.dynamicdatalists New:
com.liferay.dynamic.data.lists.exception

com.liferay com.liferay.dynamic.data.lists.api 4.0.5

NoSuchRecordVersionException

Old: com.liferay.portlet.dynamicdatalists New:
com.liferay.dynamic.data.lists.exception

com.liferay com.liferay.dynamic.data.lists.api 4.0.5

NoSuchRuleException

Old: com.liferay.portlet.mobiledevicerules New:
com.liferay.mobile.device.rules.exception

com.liferay com.liferay.mobile.device.rules.api 4.0.4

NoSuchRuleGroupException

Old: com.liferay.portlet.mobiledevicerules New:
com.liferay.mobile.device.rules.exception

com.liferay com.liferay.mobile.device.rules.api 4.0.4

NoSuchRuleGroupInstanceException

Old: com.liferay.portlet.mobiledevicerules New:
com.liferay.mobile.device.rules.exception

com.liferay com.liferay.mobile.device.rules.api 4.0.4

NoSuchStatsUserException

Old: com.liferay.portlet.blogs New: com.liferay.blogs.exception

com.liferay com.liferay.blogs.api 5.0.5

NoSuchStorageLinkException

Old: com.liferay.portlet.dynamicdatamapping New:
com.liferay.dynamic.data.mapping.exception

com.liferay com.liferay.dynamic.data.mapping.api 5.2.1

NoSuchStructureLinkException

Old: com.liferay.portlet.dynamicdatamapping New:
com.liferay.dynamic.data.mapping.exception

com.liferay com.liferay.dynamic.data.mapping.api 5.2.1

NoSuchTemplateException

Old: com.liferay.portlet.dynamicdatamapping New:
com.liferay.dynamic.data.mapping.exception

com.liferay com.liferay.dynamic.data.mapping.api 5.2.1

NoSuchThreadException

Old: com.liferay.portlet.messageboards New:
com.liferay.message.boards.exception

com.liferay com.liferay.message.boards.api 5.1.4

NoSuchThreadFlagException

Old: com.liferay.portlet.messageboards New:
com.liferay.message.boards.exception

com.liferay com.liferay.message.boards.api 5.1.4

NoSuchVoteException

Old: com.liferay.portlet.polls New: com.liferay.polls.exception

com.liferay com.liferay.polls.api 6.0.3

NodeNameException

Old: com.liferay.portlet.wiki New: com.liferay.wiki.exception

com.liferay com.liferay.wiki.api 4.0.7

OutputProcessor

Old: com.liferay.portal.kernel.process New: com.liferay.petra.process

com.liferay com.liferay.petra.process 3.0.4

PageContentException

Old: com.liferay.portlet.wiki New: com.liferay.wiki.exception

com.liferay com.liferay.wiki.api 4.0.7

PageCreateDateComparator

Old: com.liferay.portlet.wiki.util.comparator New:
com.liferay.wiki.util.comparator

com.liferay com.liferay.wiki.api 4.0.7

PageTitleComparator

Old: com.liferay.portlet.wiki.util.comparator New:
com.liferay.wiki.util.comparator

com.liferay com.liferay.wiki.api 4.0.7

PageTitleException

Old: com.liferay.portlet.wiki New: com.liferay.wiki.exception

com.liferay com.liferay.wiki.api 4.0.7

PageVersionComparator

Old: com.liferay.portlet.wiki.util.comparator New:
com.liferay.wiki.util.comparator

com.liferay com.liferay.wiki.api 4.0.7

PageVersionException

Old: com.liferay.portlet.wiki New: com.liferay.wiki.exception

com.liferay com.liferay.wiki.api 4.0.7

PollsChoice

Old: com.liferay.portlet.polls.model New: com.liferay.polls.model

com.liferay com.liferay.polls.api 6.0.3

PollsChoiceLocalService

Old: com.liferay.portlet.polls.service New: com.liferay.polls.service

com.liferay com.liferay.polls.api 6.0.3

PollsChoiceLocalServiceUtil

Old: com.liferay.portlet.polls.service New: com.liferay.polls.service

com.liferay com.liferay.polls.api 6.0.3

PollsChoiceLocalServiceWrapper

Old: com.liferay.portlet.polls.service New: com.liferay.polls.service

com.liferay com.liferay.polls.api 6.0.3

PollsChoiceModel

Old: com.liferay.portlet.polls.model New: com.liferay.polls.model

com.liferay com.liferay.polls.api 6.0.3

PollsChoicePersistence

Old: com.liferay.portlet.polls.service.persistence New:
com.liferay.polls.service.persistence

com.liferay com.liferay.polls.api 6.0.3

PollsChoiceService

Old: com.liferay.portlet.polls.service New: com.liferay.polls.service

com.liferay com.liferay.polls.api 6.0.3

PollsChoiceServiceUtil

Old: com.liferay.portlet.polls.service New: com.liferay.polls.service

com.liferay com.liferay.polls.api 6.0.3

PollsChoiceServiceWrapper

Old: com.liferay.portlet.polls.service New: com.liferay.polls.service

com.liferay com.liferay.polls.api 6.0.3

PollsChoiceSoap

Old: com.liferay.portlet.polls.model New: com.liferay.polls.model

com.liferay com.liferay.polls.api 6.0.3

PollsChoiceUtil

Old: com.liferay.portlet.polls.service.persistence New:
com.liferay.polls.service.persistence

com.liferay com.liferay.polls.api 6.0.3

PollsChoiceWrapper

Old: com.liferay.portlet.polls.model New: com.liferay.polls.model

com.liferay com.liferay.polls.api 6.0.3

PollsQuestion

Old: com.liferay.portlet.polls.model New: com.liferay.polls.model

com.liferay com.liferay.polls.api 6.0.3

PollsQuestionLocalService

Old: com.liferay.portlet.polls.service New: com.liferay.polls.service

com.liferay com.liferay.polls.api 6.0.3

PollsQuestionLocalServiceUtil

Old: com.liferay.portlet.polls.service New: com.liferay.polls.service

com.liferay com.liferay.polls.api 6.0.3

PollsQuestionLocalServiceWrapper

Old: com.liferay.portlet.polls.service New: com.liferay.polls.service

com.liferay com.liferay.polls.api 6.0.3

PollsQuestionModel

Old: com.liferay.portlet.polls.model New: com.liferay.polls.model

com.liferay com.liferay.polls.api 6.0.3

PollsQuestionPersistence

Old: com.liferay.portlet.polls.service.persistence New:
com.liferay.polls.service.persistence

com.liferay com.liferay.polls.api 6.0.3

PollsQuestionService

Old: com.liferay.portlet.polls.service New: com.liferay.polls.service

com.liferay com.liferay.polls.api 6.0.3

PollsQuestionServiceUtil

Old: com.liferay.portlet.polls.service New: com.liferay.polls.service

com.liferay com.liferay.polls.api 6.0.3

PollsQuestionServiceWrapper

Old: com.liferay.portlet.polls.service New: com.liferay.polls.service

com.liferay com.liferay.polls.api 6.0.3

PollsQuestionSoap

Old: com.liferay.portlet.polls.model New: com.liferay.polls.model

com.liferay com.liferay.polls.api 6.0.3

PollsQuestionUtil

Old: com.liferay.portlet.polls.service.persistence New:
com.liferay.polls.service.persistence

com.liferay com.liferay.polls.api 6.0.3

PollsQuestionWrapper

Old: com.liferay.portlet.polls.model New: com.liferay.polls.model

com.liferay com.liferay.polls.api 6.0.3

PollsVote

Old: com.liferay.portlet.polls.model New: com.liferay.polls.model

com.liferay com.liferay.polls.api 6.0.3

PollsVoteLocalService

Old: com.liferay.portlet.polls.service New: com.liferay.polls.service

com.liferay com.liferay.polls.api 6.0.3

PollsVoteLocalServiceUtil

Old: com.liferay.portlet.polls.service New: com.liferay.polls.service

com.liferay com.liferay.polls.api 6.0.3

PollsVoteLocalServiceWrapper

Old: com.liferay.portlet.polls.service New: com.liferay.polls.service

com.liferay com.liferay.polls.api 6.0.3

PollsVoteModel

Old: com.liferay.portlet.polls.model New: com.liferay.polls.model

com.liferay com.liferay.polls.api 6.0.3

PollsVotePersistence

Old: com.liferay.portlet.polls.service.persistence New:
com.liferay.polls.service.persistence

com.liferay com.liferay.polls.api 6.0.3

PollsVoteService

Old: com.liferay.portlet.polls.service New: com.liferay.polls.service

com.liferay com.liferay.polls.api 6.0.3

PollsVoteServiceUtil

Old: com.liferay.portlet.polls.service New: com.liferay.polls.service

com.liferay com.liferay.polls.api 6.0.3

PollsVoteServiceWrapper

Old: com.liferay.portlet.polls.service New: com.liferay.polls.service

com.liferay com.liferay.polls.api 6.0.3

PollsVoteSoap

Old: com.liferay.portlet.polls.model New: com.liferay.polls.model

com.liferay com.liferay.polls.api 6.0.3

PollsVoteUtil

Old: com.liferay.portlet.polls.service.persistence New:
com.liferay.polls.service.persistence

com.liferay com.liferay.polls.api 6.0.3

PollsVoteWrapper

Old: com.liferay.portlet.polls.model New: com.liferay.polls.model

com.liferay com.liferay.polls.api 6.0.3

PoolAction

Old: com.liferay.portal.kernel.memory New: com.liferay.petra.memory

com.liferay com.liferay.petra.memory 3.0.1

PortalCacheClusterChannel

Old: com.liferay.portal.kernel.cache.cluster New:
com.liferay.portal.cache.multiple.internal.cluster.link

com.liferay com.liferay.portal.cache.multiple 3.0.6

PortalCacheClusterChannelFactory

Old: com.liferay.portal.kernel.cache.cluster New:
com.liferay.portal.cache.multiple.internal.cluster.link

com.liferay com.liferay.portal.cache.multiple 3.0.6

PortalCacheClusterChannelSelector

Old: com.liferay.portal.kernel.cache.cluster New:
com.liferay.portal.cache.multiple.internal.cluster.link

com.liferay com.liferay.portal.cache.multiple 3.0.6

PortalCacheClusterEvent

Old: com.liferay.portal.kernel.cache.cluster New:
com.liferay.portal.cache.multiple.internal

com.liferay com.liferay.portal.cache.multiple 3.0.6

PortalCacheClusterEventCoalesceComparator

Old: com.liferay.portal.kernel.cache.cluster New:
com.liferay.portal.cache.multiple.internal

com.liferay com.liferay.portal.cache.multiple 3.0.6

PortalCacheClusterEventType

Old: com.liferay.portal.kernel.cache.cluster New:
com.liferay.portal.cache.multiple.internal

com.liferay com.liferay.portal.cache.multiple 3.0.6

PortalCacheClusterException

Old: com.liferay.portal.kernel.cache.cluster New:
com.liferay.portal.cache.multiple.internal

com.liferay com.liferay.portal.cache.multiple 3.0.6

PortalCacheClusterLink

Old: com.liferay.portal.kernel.cache.cluster New:
com.liferay.portal.cache.multiple.internal.cluster.link

com.liferay com.liferay.portal.cache.multiple 3.0.6

PortalExecutorFactory

Old: com.liferay.portal.kernel.executor New:
com.liferay.portal.executor.internal

com.liferay com.liferay.portal.executor 4.0.2

PortalToLDAPConverter

Old: com.liferay.portal.security.ldap New:
com.liferay.portal.security.ldap.exportimport

com.liferay com.liferay.portal.security.ldap.api 2.0.8

PortletDisplayTemplate

Old: com.liferay.portlet.portletdisplaytemplate.util New:
com.liferay.portlet.display.template

com.liferay com.liferay.portlet.display.template.api 2.0.2

PortletDisplayTemplateConstants

Old: com.liferay.portlet.portletdisplaytemplate.util New:
com.liferay.portlet.display.template.constants

com.liferay com.liferay.portlet.display.template.api 2.0.2

PortletDisplayTemplateUtil

Old: com.liferay.portlet.portletdisplaytemplate.util New:
com.liferay.roles.admin.web.internal.util

com.liferay com.liferay.roles.admin.web 3.0.6

PortletDisplayTemplateUtil

Old: com.liferay.portlet.portletdisplaytemplate.util New:
com.liferay.roles.admin.web.internal.util

com.liferay com.liferay.roles.admin.web 3.0.6

PortletDisplayTemplateUtil

Old: com.liferay.portlet.portletdisplaytemplate.util New:
com.liferay.roles.admin.web.internal.util

com.liferay com.liferay.roles.admin.web 3.0.6

ProcessUtil

Old: com.liferay.portal.kernel.process New: com.liferay.petra.process

com.liferay com.liferay.petra.process 3.0.4

QueryIndexingHitsProcessor

Old: com.liferay.portal.kernel.search New:
com.liferay.portal.search.internal.hits

com.liferay com.liferay.portal.search 6.0.14

QuerySuggestionHitsProcessor

Old: com.liferay.portal.kernel.search New:
com.liferay.portal.search.internal.hits

com.liferay com.liferay.portal.search 6.0.14

QueryType

Old: com.liferay.portal.kernel.bi.rules New:
com.liferay.portal.rules.engine

com.liferay com.liferay.portal.rules.engine.api 4.0.4

QuestionChoiceException

Old: com.liferay.portlet.polls New: com.liferay.polls.exception

com.liferay com.liferay.polls.api 6.0.3

QuestionDescriptionException

Old: com.liferay.portlet.polls New: com.liferay.polls.exception

com.liferay com.liferay.polls.api 6.0.3

QuestionExpirationDateException

Old: com.liferay.portlet.polls New: com.liferay.polls.exception

com.liferay com.liferay.polls.api 6.0.3

QuestionExpiredException

Old: com.liferay.portlet.polls New: com.liferay.polls.exception

com.liferay com.liferay.polls.api 6.0.3

QuestionTitleException

Old: com.liferay.portlet.polls New: com.liferay.polls.exception

com.liferay com.liferay.polls.api 6.0.3

RecordSetDDMStructureIdException

Old: com.liferay.portlet.dynamicdatalists New:
com.liferay.dynamic.data.lists.exception

com.liferay com.liferay.dynamic.data.lists.api 4.0.5

RecordSetDuplicateRecordSetKeyException

Old: com.liferay.portlet.dynamicdatalists New:
com.liferay.dynamic.data.lists.exception

com.liferay com.liferay.dynamic.data.lists.api 4.0.5

RecordSetNameException

Old: com.liferay.portlet.dynamicdatalists New:
com.liferay.dynamic.data.lists.exception

com.liferay com.liferay.dynamic.data.lists.api 4.0.5

RegistryAwareMBeanServer

Old: com.liferay.portal.kernel.jmx New: com.liferay.portal.jmx.internal

com.liferay com.liferay.portal.jmx 6.0.2

ReportCompilerRequestMessageListener

Old: com.liferay.portal.kernel.bi.reporting.messaging New:
com.liferay.portal.reports.engine.messaging

com.liferay com.liferay.portal.reports.engine.api 5.0.1

ReportDataSourceType

Old: com.liferay.portal.kernel.bi.reporting New:
com.liferay.portal.reports.engine

com.liferay com.liferay.portal.reports.engine.api 5.0.1

ReportDesignRetriever

Old: com.liferay.portal.kernel.bi.reporting New:
com.liferay.portal.reports.engine

com.liferay com.liferay.portal.reports.engine.api 5.0.1

ReportEngine

Old: com.liferay.portal.kernel.bi.reporting New:
com.liferay.portal.reports.engine

com.liferay com.liferay.portal.reports.engine.api 5.0.1

ReportExportException

Old: com.liferay.portal.kernel.bi.reporting New:
com.liferay.portal.reports.engine

com.liferay com.liferay.portal.reports.engine.api 5.0.1

ReportFormat

Old: com.liferay.portal.kernel.bi.reporting New:
com.liferay.portal.reports.engine

com.liferay com.liferay.portal.reports.engine.api 5.0.1

ReportFormatExporter

Old: com.liferay.portal.kernel.bi.reporting New:
com.liferay.portal.reports.engine

com.liferay com.liferay.portal.reports.engine.api 5.0.1

ReportFormatExporterRegistry

Old: com.liferay.portal.kernel.bi.reporting New:
com.liferay.portal.reports.engine

com.liferay com.liferay.portal.reports.engine.api 5.0.1

ReportGenerationException

Old: com.liferay.portal.kernel.bi.reporting New:
com.liferay.portal.reports.engine

com.liferay com.liferay.portal.reports.engine.api 5.0.1

ReportRequest

Old: com.liferay.portal.kernel.bi.reporting New:
com.liferay.portal.reports.engine

com.liferay com.liferay.portal.reports.engine.api 5.0.1

ReportRequestContext

Old: com.liferay.portal.kernel.bi.reporting New:
com.liferay.portal.reports.engine

com.liferay com.liferay.portal.reports.engine.api 5.0.1

ReportRequestMessageListener

Old: com.liferay.portal.kernel.bi.reporting.messaging New:
com.liferay.portal.reports.engine.messaging

com.liferay com.liferay.portal.reports.engine.api 5.0.1

ReportResultContainer

Old: com.liferay.portal.kernel.bi.reporting New:
com.liferay.portal.reports.engine

com.liferay com.liferay.portal.reports.engine.api 5.0.1

RequestStatistics

Old: com.liferay.portal.kernel.monitoring.statistics New:
com.liferay.portal.monitoring.internal.statistics

com.liferay com.liferay.portal.monitoring 7.0.7

RequiredMessageException

Old: com.liferay.portlet.messageboards New:
com.liferay.message.boards.exception

com.liferay com.liferay.message.boards.api 5.1.4

RequiredNodeException

Old: com.liferay.portlet.wiki New: com.liferay.wiki.exception

com.liferay com.liferay.wiki.api 4.0.7

RequiredTemplateException

Old: com.liferay.portlet.dynamicdatamapping New:
com.liferay.dynamic.data.mapping.exception

com.liferay com.liferay.dynamic.data.mapping.api 5.2.1

RequiredTemplateException

Old: com.liferay.portlet.journal New:
com.liferay.dynamic.data.mapping.exception

com.liferay com.liferay.dynamic.data.mapping.api 5.2.1

RuleGroupInstancePriorityComparator

Old: com.liferay.portlet.mobiledevicerules.util New:
com.liferay.mobile.device.rules.util.comparator

com.liferay com.liferay.mobile.device.rules.api 4.0.4

RuleGroupProcessor

Old: com.liferay.portal.kernel.mobile.device.rulegroup New:
com.liferay.mobile.device.rules.rule

com.liferay com.liferay.mobile.device.rules.api 4.0.4

RuleGroupProcessorUtil

Old: com.liferay.portal.kernel.mobile.device.rulegroup New:
com.liferay.mobile.device.rules.rule

com.liferay com.liferay.mobile.device.rules.api 4.0.4

RuleHandler

Old: com.liferay.portal.kernel.mobile.device.rulegroup.rule New:
com.liferay.mobile.device.rules.rule

com.liferay com.liferay.mobile.device.rules.api 4.0.4

RulesEngine

Old: com.liferay.portal.kernel.bi.rules New:
com.liferay.portal.rules.engine

com.liferay com.liferay.portal.rules.engine.api 4.0.4

RulesEngineException

Old: com.liferay.portal.kernel.bi.rules New:
com.liferay.portal.rules.engine

com.liferay com.liferay.portal.rules.engine.api 4.0.4

RulesEngineUtil

Old: com.liferay.portal.kernel.bi.rules New:
com.liferay.portal.rules.engine

com.liferay com.liferay.portal.rules.engine.api 4.0.4

RulesLanguage

Old: com.liferay.portal.kernel.bi.rules New:
com.liferay.portal.rules.engine

com.liferay com.liferay.portal.rules.engine.api 4.0.4

RulesResourceRetriever

Old: com.liferay.portal.kernel.bi.rules New:
com.liferay.portal.rules.engine

com.liferay com.liferay.portal.rules.engine.api 4.0.4

SearchUtil

Old: com.liferay.portal.kernel.search.util New:
com.liferay.portal.vulcan.util

com.liferay com.liferay.portal.vulcan.api 3.2.2

SearchUtil

Old: com.liferay.portal.kernel.search.util New:
com.liferay.portal.vulcan.util

com.liferay com.liferay.portal.vulcan.api 3.2.2

ServletContextReportDesignRetriever

Old: com.liferay.portal.kernel.bi.reporting.servlet New:
com.liferay.portal.reports.engine.servlet

com.liferay com.liferay.portal.reports.engine.api 5.0.1

SoftReferencePool

Old: com.liferay.portal.kernel.memory New: com.liferay.petra.memory

com.liferay com.liferay.petra.memory 3.0.1

SortFactoryImpl

Old: com.liferay.portal.kernel.search New:
com.liferay.portal.search.internal

com.liferay com.liferay.portal.search 6.0.14

SplitThreadException

Old: com.liferay.portlet.messageboards New:
com.liferay.message.boards.exception

com.liferay com.liferay.message.boards.api 5.1.4

Statistics

Old: com.liferay.portal.kernel.monitoring.statistics New:
com.liferay.portal.monitoring.internal.statistics

com.liferay com.liferay.portal.monitoring 7.0.7

StatsUserLastPostDateComparator

Old: com.liferay.portlet.blogs.util.comparator New:
com.liferay.blogs.util.comparator

com.liferay com.liferay.blogs.api 5.0.5

StorageAdapter

Old: com.liferay.portlet.dynamicdatamapping.storage New:
com.liferay.dynamic.data.mapping.storage

com.liferay com.liferay.dynamic.data.mapping.api 5.2.1

StorageEngine

Old: com.liferay.portlet.dynamicdatamapping.storage New:
com.liferay.dynamic.data.mapping.storage

com.liferay com.liferay.dynamic.data.mapping.api 5.2.1

StorageException

Old: com.liferay.portlet.dynamicdatamapping New:
com.liferay.dynamic.data.mapping.exception

com.liferay com.liferay.dynamic.data.mapping.api 5.2.1

StorageFieldNameException

Old: com.liferay.portlet.dynamicdatamapping New:
com.liferay.dynamic.data.mapping.exception

com.liferay com.liferay.dynamic.data.mapping.api 5.2.1

StringQueryImpl

Old: com.liferay.portal.kernel.search New:
com.liferay.portal.search.internal.query

com.liferay com.liferay.portal.search 6.0.14

StructureDuplicateStructureKeyException

Old: com.liferay.portlet.dynamicdatamapping New:
com.liferay.dynamic.data.mapping.exception

com.liferay com.liferay.dynamic.data.mapping.api 5.2.1

StructureFieldException

Old: com.liferay.portlet.dynamicdatamapping New:
com.liferay.dynamic.data.mapping.exception

com.liferay com.liferay.dynamic.data.mapping.api 5.2.1

StructureIdComparator

Old: com.liferay.portlet.dynamicdatamapping.util.comparator New:
com.liferay.dynamic.data.mapping.util.comparator

com.liferay com.liferay.dynamic.data.mapping.api 5.2.1

StructureModifiedDateComparator

Old: com.liferay.portlet.dynamicdatamapping.util.comparator New:
com.liferay.dynamic.data.mapping.util.comparator

com.liferay com.liferay.dynamic.data.mapping.api 5.2.1

StructureStructureKeyComparator

Old: com.liferay.portlet.dynamicdatamapping.util.comparator New:
com.liferay.dynamic.data.mapping.util.comparator

com.liferay com.liferay.dynamic.data.mapping.api 5.2.1

SummaryStatistics

Old: com.liferay.portal.kernel.monitoring.statistics New:
com.liferay.portal.monitoring.internal.statistics

com.liferay com.liferay.portal.monitoring 7.0.7

SynchronousMessageListener

Old: com.liferay.portal.kernel.messaging.sender New:
com.liferay.portal.messaging.internal.sender

com.liferay com.liferay.portal.messaging 6.0.5

TemplateDuplicateTemplateKeyException

Old: com.liferay.portlet.dynamicdatamapping New:
com.liferay.dynamic.data.mapping.exception

com.liferay com.liferay.dynamic.data.mapping.api 5.2.1

TemplateIdComparator

Old: com.liferay.portlet.dynamicdatamapping.util.comparator New:
com.liferay.dynamic.data.mapping.util.comparator

com.liferay com.liferay.dynamic.data.mapping.api 5.2.1

TemplateModifiedDateComparator

Old: com.liferay.portlet.dynamicdatamapping.util.comparator New:
com.liferay.dynamic.data.mapping.util.comparator

com.liferay com.liferay.dynamic.data.mapping.api 5.2.1

TemplateNameException

Old: com.liferay.portlet.dynamicdatamapping New:
com.liferay.dynamic.data.mapping.exception

com.liferay com.liferay.dynamic.data.mapping.api 5.2.1

TemplateNameException

Old: com.liferay.portlet.journal New:
com.liferay.dynamic.data.mapping.exception

com.liferay com.liferay.dynamic.data.mapping.api 5.2.1

TemplateScriptException

Old: com.liferay.portlet.dynamicdatamapping New:
com.liferay.dynamic.data.mapping.exception

com.liferay com.liferay.dynamic.data.mapping.api 5.2.1

TemplateSmallImageNameException

Old: com.liferay.portlet.dynamicdatamapping New:
com.liferay.dynamic.data.mapping.exception

com.liferay com.liferay.dynamic.data.mapping.api 5.2.1

TemplateSmallImageNameException

Old: com.liferay.portlet.journal New:
com.liferay.dynamic.data.mapping.exception

com.liferay com.liferay.dynamic.data.mapping.api 5.2.1

TemplateSmallImageSizeException

Old: com.liferay.portlet.dynamicdatamapping New:
com.liferay.dynamic.data.mapping.exception

com.liferay com.liferay.dynamic.data.mapping.api 5.2.1

TemplateSmallImageSizeException

Old: com.liferay.portlet.journal New:
com.liferay.dynamic.data.mapping.exception

com.liferay com.liferay.dynamic.data.mapping.api 5.2.1

ThreadLastPostDateComparator

Old: com.liferay.portlet.messageboards.util.comparator New:
com.liferay.message.boards.util.comparator

com.liferay com.liferay.message.boards.api 5.1.4

UniformPortalCacheClusterChannelSelector

Old: com.liferay.portal.kernel.cache.cluster New:
com.liferay.portal.cache.multiple.internal.cluster.link

com.liferay com.liferay.portal.cache.multiple 3.0.6

UnknownRuleHandlerException

Old: com.liferay.portal.kernel.mobile.device.rulegroup.rule New:
com.liferay.mobile.device.rules.rule

com.liferay com.liferay.mobile.device.rules.api 4.0.4

UserConverterKeys

Old: com.liferay.portal.security.ldap New:
com.liferay.portal.security.ldap

com.liferay com.liferay.portal.security.ldap.api 2.0.8

WikiFormatException

Old: com.liferay.portlet.wiki New: com.liferay.wiki.exception

com.liferay com.liferay.wiki.api 4.0.7

WikiNode

Old: com.liferay.portlet.wiki.model New: com.liferay.wiki.model

com.liferay com.liferay.wiki.api 4.0.7

WikiNodeLocalService

Old: com.liferay.portlet.wiki.service New: com.liferay.wiki.service

com.liferay com.liferay.wiki.api 4.0.7

WikiNodeLocalServiceUtil

Old: com.liferay.portlet.wiki.service New: com.liferay.wiki.service

com.liferay com.liferay.wiki.api 4.0.7

WikiNodeLocalServiceWrapper

Old: com.liferay.portlet.wiki.service New: com.liferay.wiki.service

com.liferay com.liferay.wiki.api 4.0.7

WikiNodeModel

Old: com.liferay.portlet.wiki.model New: com.liferay.wiki.model

com.liferay com.liferay.wiki.api 4.0.7

WikiNodePersistence

Old: com.liferay.portlet.wiki.service.persistence New:
com.liferay.wiki.service.persistence

com.liferay com.liferay.wiki.api 4.0.7

WikiNodeService

Old: com.liferay.portlet.wiki.service New: com.liferay.wiki.service

com.liferay com.liferay.wiki.api 4.0.7

WikiNodeServiceUtil

Old: com.liferay.portlet.wiki.service New: com.liferay.wiki.service

com.liferay com.liferay.wiki.api 4.0.7

WikiNodeServiceWrapper

Old: com.liferay.portlet.wiki.service New: com.liferay.wiki.service

com.liferay com.liferay.wiki.api 4.0.7

WikiNodeSoap

Old: com.liferay.portlet.wiki.model New: com.liferay.wiki.model

com.liferay com.liferay.wiki.api 4.0.7

WikiNodeUtil

Old: com.liferay.portlet.wiki.service.persistence New:
com.liferay.wiki.service.persistence

com.liferay com.liferay.wiki.api 4.0.7

WikiNodeWrapper

Old: com.liferay.portlet.wiki.model New: com.liferay.wiki.model

com.liferay com.liferay.wiki.api 4.0.7

WikiPage

Old: com.liferay.portlet.wiki.model New: com.liferay.wiki.model

com.liferay com.liferay.wiki.api 4.0.7

WikiPageConstants

Old: com.liferay.portlet.wiki.model New: com.liferay.wiki.model

com.liferay com.liferay.wiki.api 4.0.7

WikiPageDisplay

Old: com.liferay.portlet.wiki.model New: com.liferay.wiki.model

com.liferay com.liferay.wiki.api 4.0.7

WikiPageFinder

Old: com.liferay.portlet.wiki.service.persistence New:
com.liferay.wiki.service.persistence

com.liferay com.liferay.wiki.api 4.0.7

WikiPageLocalService

Old: com.liferay.portlet.wiki.service New: com.liferay.wiki.service

com.liferay com.liferay.wiki.api 4.0.7

WikiPageLocalServiceUtil

Old: com.liferay.portlet.wiki.service New: com.liferay.wiki.service

com.liferay com.liferay.wiki.api 4.0.7

WikiPageLocalServiceWrapper

Old: com.liferay.portlet.wiki.service New: com.liferay.wiki.service

com.liferay com.liferay.wiki.api 4.0.7

WikiPageModel

Old: com.liferay.portlet.wiki.model New: com.liferay.wiki.model

com.liferay com.liferay.wiki.api 4.0.7

WikiPagePersistence

Old: com.liferay.portlet.wiki.service.persistence New:
com.liferay.wiki.service.persistence

com.liferay com.liferay.wiki.api 4.0.7

WikiPageResource

Old: com.liferay.portlet.wiki.model New: com.liferay.wiki.model

com.liferay com.liferay.wiki.api 4.0.7

WikiPageResourceLocalService

Old: com.liferay.portlet.wiki.service New: com.liferay.wiki.service

com.liferay com.liferay.wiki.api 4.0.7

WikiPageResourceLocalServiceUtil

Old: com.liferay.portlet.wiki.service New: com.liferay.wiki.service

com.liferay com.liferay.wiki.api 4.0.7

WikiPageResourceLocalServiceWrapper

Old: com.liferay.portlet.wiki.service New: com.liferay.wiki.service

com.liferay com.liferay.wiki.api 4.0.7

WikiPageResourceModel

Old: com.liferay.portlet.wiki.model New: com.liferay.wiki.model

com.liferay com.liferay.wiki.api 4.0.7

WikiPageResourcePersistence

Old: com.liferay.portlet.wiki.service.persistence New:
com.liferay.wiki.service.persistence

com.liferay com.liferay.wiki.api 4.0.7

WikiPageResourceSoap

Old: com.liferay.portlet.wiki.model New: com.liferay.wiki.model

com.liferay com.liferay.wiki.api 4.0.7

WikiPageResourceUtil

Old: com.liferay.portlet.wiki.service.persistence New:
com.liferay.wiki.service.persistence

com.liferay com.liferay.wiki.api 4.0.7

WikiPageResourceWrapper

Old: com.liferay.portlet.wiki.model New: com.liferay.wiki.model

com.liferay com.liferay.wiki.api 4.0.7

WikiPageService

Old: com.liferay.portlet.wiki.service New: com.liferay.wiki.service

com.liferay com.liferay.wiki.api 4.0.7

WikiPageServiceUtil

Old: com.liferay.portlet.wiki.service New: com.liferay.wiki.service

com.liferay com.liferay.wiki.api 4.0.7

WikiPageServiceWrapper

Old: com.liferay.portlet.wiki.service New: com.liferay.wiki.service

com.liferay com.liferay.wiki.api 4.0.7

WikiPageSoap

Old: com.liferay.portlet.wiki.model New: com.liferay.wiki.model

com.liferay com.liferay.wiki.api 4.0.7

WikiPageUtil

Old: com.liferay.portlet.wiki.service.persistence New:
com.liferay.wiki.service.persistence

com.liferay com.liferay.wiki.api 4.0.7

WikiPageWrapper

Old: com.liferay.portlet.wiki.model New: com.liferay.wiki.model

com.liferay com.liferay.wiki.api 4.0.7

\chapter{Export/Import and Staging}\label{exportimport-and-staging}

Export/Import and Staging are frameworks that help you manage your
content publication. This section provides reference documentation
complementing the
\href{/docs/7-2/frameworks/-/knowledge_base/f/content-publication-management}{Content
Publication Management} section.

\chapter{Decision to Implement
Staging}\label{decision-to-implement-staging}

Staging is an advanced publication tool that lets you create or modify
your site before releasing it to the public. Most of Liferay DXP's
included applications (e.g., Web Content, Bookmarks, etc.) support
Staging. Implementing Staging in your own application can be beneficial,
but how do you know if it's the right move?

Not every application needs to support Staging and Export/Import. The
most important question to consider during the decision process is

\emph{What part of your application are you primarily focused on using
Staging for?}

When Staging is enabled, all pages and applications are staged
automatically. Liferay DXP's architecture separates the application and
its configuration from the actual content, meaning that content can
exist without any application to display it and vice versa. Although
Staging supports all applications and their configurations by default,
not all applications' content is supported by Staging.

Implementing Staging for your application means you're defining the
logic for how the Staging framework should process, serialize, and
de-serialize your app's content, and how to insert it into a database.

Therefore, if you want to track your application's content, you should
implement Staging in your application. Here are a few other scenarios
where you should implement Staging in your application:

\begin{itemize}
\tightlist
\item
  You're using remote staging. When publishing to a remote live site,
  your content must be transferred to a different Liferay DXP
  installation. Therefore, Staging must be able to recognize the content
  to facilitate the transfer.
\item
  You want a space where you can freely edit and test your content
  before publishing it to a live audience.
\item
  Your content is being referenced from another content type that
  supports Staging.
\item
  You want to process your portlet's preferences during publication
  (i.e., you might want to publish some content with it or complete
  extra steps).
\item
  You want to process the content during publication (e.g., writing
  validation for your content during the import process).
\end{itemize}

If none of these options are beneficial for you, implementing Staging in
your application is unnecessary.

When content supports Staging and Staging is enabled, it is created in a
Staging group and is only published to a live site when that site is
published. When content is \textbf{not} supported by Staging, it is
never added to a Staging group and is not reviewable during the Staging
publication process; it's added and removed from the live site only.

From a technical standpoint, publishing an entity or content follows the
process below:

\begin{enumerate}
\def\labelenumi{\arabic{enumi}.}
\tightlist
\item
  The entity's possible references are discovered and processed.
\item
  The entity's fields are processed.
\item
  The entity is serialized into a LAR file.
\item
  The LAR is transferred to the live site (local or remote live).
\item
  After de-serialization, the entity's fields are processed.
\item
  The entity is added to the database.
\end{enumerate}

Awesome! You should now have a good idea about whether you should
implement Staging for your application.

\chapter{Liferay Archive (LAR) File}\label{liferay-archive-lar-file}

An easier way to export/import your application's data is to use a
Liferay ARchive (LAR) file. Liferay provides the LAR feature to address
the need to export/import data in a database agnostic manner. So what
exactly is a LAR file?

A LAR file is a compressed file (ZIP archive) Liferay DXP uses to
export/import data. LAR files can be created for single portlets, pages,
or sets of pages. Portlets that are LAR-capable provide an interface to
let you control how their data is imported/exported. There are several
Liferay DXP use cases that require the use of LAR files:

\begin{itemize}
\tightlist
\item
  Backing up and restoring portlet-specific data without requiring a
  full database backup.
\item
  Cloning sites.
\item
  Specifying a template to be used for users' public or private pages.
\item
  Using Local Live or Remote Live staging.
\end{itemize}

The data handler framework is available so developers don't have to
create/modify a LAR file manually. \textbf{It is strongly recommended
never to modify a LAR file.} You should always use Liferay's provided
data handler APIs to construct it.

Knowing how a LAR file is constructed, however, is beneficial to
understand the overall purpose of your application's data handlers.
Next, you'll explore a LAR file's anatomy.

\section{LAR File Anatomy}\label{lar-file-anatomy}

What is a LAR file? You know the general concept for \emph{why} it's
used, but you may want to know what lives inside to make your
export/import processes work. With a fundamental understanding for how a
LAR file is constructed, you can better understand what your data
handlers generate behind the scenes.

Below is the structure of a simple LAR file. It illustrates the
exportation of a single Bookmarks entry and the portlet's configuration:

\begin{itemize}
\tightlist
\item
  \texttt{Bookmarks\_Admin-201701091904.portlet.lar}

  \begin{itemize}
  \tightlist
  \item
    \texttt{group}

    \begin{itemize}
    \tightlist
    \item
      \texttt{20143}

      \begin{itemize}
      \tightlist
      \item
        \texttt{com.liferay.bookmarks.model.BookmarksEntry}

        \begin{itemize}
        \tightlist
        \item
          \texttt{35005.xml}
        \end{itemize}
      \item
        \texttt{portlet}

        \begin{itemize}
        \tightlist
        \item
          \texttt{com\_liferay\_bookmarks\_web\_portlet\_BookmarksAdminPortlet}

          \begin{itemize}
          \tightlist
          \item
            \texttt{20137}

            \begin{itemize}
            \tightlist
            \item
              \texttt{portlet.xml}
            \end{itemize}
          \item
            \texttt{20143}

            \begin{itemize}
            \tightlist
            \item
              \texttt{portlet-data.xml}
            \end{itemize}
          \end{itemize}
        \end{itemize}
      \end{itemize}
    \end{itemize}
  \item
    \texttt{manifest.xml}
  \end{itemize}
\end{itemize}

You'll dissect the anatomy structure next.

\section{LAR Manifest}\label{lar-manifest}

You can tell from the LAR's generated name what information is contained
in the LAR: the Bookmarks Admin app's data. The \texttt{manifest.xml}
file sits at the root of the LAR file. It provides essential information
about the export process. The \texttt{manifest.xml} for the sample
Bookmarks LAR is pretty bare since it's not exporting much content, but
this file can become large when exporting pages of content. There are
four main parts (tags) to a \texttt{manifest.xml} file.

\begin{itemize}
\tightlist
\item
  \texttt{header}: contains information about the LAR file, current
  process, and site you're exporting (if necessary). For example, it can
  include locales, build information, export date, company ID, group ID,
  layouts, themes, etc.
\item
  \texttt{missing-references}: lists entities that must be validated
  during import. For example, suppose you're exporting a web content
  article that references an image (e.g., an embedded image residing in
  the document library). If the image was not selected for export, the
  image must already exist in the site where the article is imported.
  Therefore, the image would be flagged as a missing reference in the
  LAR file. If the missing reference does not exist in the site when the
  LAR is imported, the import process fails. If your import fails, the
  Import UI shows you the missing references that weren't validated.
\item
  \texttt{portlets}: defines the portlets (i.e., portlet data) exported
  in the LAR. Each portlet definition has basic information on the
  exported portlet and points to the generated \texttt{portlet.xml} for
  more specialized portlet information.
\item
  \texttt{manifest-summary}: contains information on what has been
  exported. The Staging and Export frameworks export or publish some
  entities even though they weren't marked for it, because the process
  respects data integrity. This section holds information for all the
  entities that have been processed. The entities defining a non-zero
  \texttt{addition-count} attribute are displayed in the Export/Import
  UI.
\end{itemize}

The \texttt{manifest.xml} file also defines layout information if you've
exported pages in your LAR. For example, your manifest could have
\texttt{LayoutSet}, \texttt{Layout}, and \texttt{LayoutFriendlyURL} tags
specifying staged models and their various references in an exported
page.

Now that you've learned about the LAR's \texttt{manifest.xml} and how
it's used to store high-level data about your export process, you can
dive deeper into the LAR file's folders.

\section{LAR Folders}\label{lar-folders}

The \texttt{group} folder has two main parts:

\begin{itemize}
\tightlist
\item
  Entities
\item
  Portlets
\end{itemize}

If you look at the anatomy of the sample Bookmarks LAR, you'll notice
that \texttt{group/{[}groupId{]}} folder holds a folder named after the
entity you're exporting (e.g.,
\texttt{com.liferay.bookmarks.model.BookmarksEntry}) and a
\texttt{portlet} folder holding a folder named after the portlet from
which you're exporting (e.g.,
\texttt{com\_liferay\_bookmarks\_web\_portlet\_BookmarksAdminPortlet}).
For each entity/portlet you export, there are subsequent folders holding
data about them. Entities and portlets can also be stored in a
\texttt{company} folder. Although the majority of entities belong to a
group, some exist outside of a group scope (e.g., users).

If you open the
\texttt{/group/20143/com.liferay.bookmarks.model.BookmarksEntry/35005.xml}
file, you'll find serialized data about the entity, which is similar to
what is stored in the database.

The \texttt{portlet} folder holds all the portlets you exported. Each
portlet has its own folder that holds various XML files with data
describing the exported content. There are three main XML files that can
be generated for a single portlet:

\begin{itemize}
\tightlist
\item
  \texttt{portlet.xml}: provides essential information about the
  portlet, similar to a manifest file. For example, this can include the
  portlet ID, high-level entity information stored in the portlet (e.g.,
  web content articles in a web content portlet), permissioning, etc.
\item
  \texttt{portlet-data.xml}: describes specific entity data stored in
  the portlet. For example, for the web content portlet, articles stored
  in the portlet are defined in \texttt{staged-model} tags and are
  linked to their serialized entity XML files.
\item
  \texttt{portlet-preferences.xml}: defines the settings of the portlet.
  For example, this can include portlet preferences like the portlet
  owner, default user, article IDs, etc.
\end{itemize}

Note that when you import a LAR, it only includes the portlet data. You
have to deploy the portlet to be able to use it.

You now know how exported entities, portlets, and pages are defined in a
LAR file. For a summarized outline of what you've learned about LAR file
construction, see the diagram below.

\begin{figure}
\centering
\includegraphics{./images/lar-diagram.png}
\caption{Entities, Portlets, and Pages are defined in a LAR in different
places.}
\end{figure}

Excellent! You now have a fundamental understanding for how a LAR file
is generated and how it's structured.

\chapter{Front-End Reference}\label{front-end-reference}

This section contains resources that you might find useful for Front-End
development.

The topics below are covered in this section:

\begin{itemize}
\tightlist
\item
  \href{/docs/7-2/reference/-/knowledge_base/r/product-freemarker-macros}{Liferay
  DXP FreeMarker Macros}
\item
  \href{/docs/7-2/reference/-/knowledge_base/r/freemarker-taglib-macros}{FreeMarker
  Taglib Macros}
\item
  \href{/docs/7-2/reference/-/knowledge_base/r/front-end-taglibs}{Front-end
  Taglibs}
\item
  \href{/docs/7-2/reference/-/knowledge_base/r/liferay-npm-bundler}{Liferay
  npm Bundler}
\item
  \href{/docs/7-2/reference/-/knowledge_base/r/liferay-javascript-apis}{Liferay
  JS APIs}
\item
  \href{/docs/7-2/reference/-/knowledge_base/r/setting-up-your-npm-environment}{Setting
  up Your npm Environment}
\item
  \href{/docs/7-2/reference/-/knowledge_base/r/sitemap-page-configuration-options}{Sitemap.json
  Page Configuration Options}
\item
  \href{/docs/7-2/reference/-/knowledge_base/r/ckeditor-plugin-reference-guide}{CKEditor
  Plugin Reference Guide}
\item
  \href{/docs/7-2/reference/-/knowledge_base/r/fully-qualified-portlet-ids}{Fully
  Qualified Portlet IDs List}
\end{itemize}

\chapter{Liferay DXP FreeMarker
Macros}\label{liferay-dxp-freemarker-macros}

Liferay DXP defines several
\href{https://freemarker.apache.org/docs/ref_directive_macro.html}{macros}
in
\href{https://github.com/liferay/liferay-portal/blob/7.2.x/modules/apps/portal-template/portal-template-freemarker/src/main/resources/FTL_liferay.ftl}{\texttt{FTL\_Liferay.ftl}
template} that you can use in your theme templates to include theme
resources, standard portlets, and more. Liferay DXP also exposes its
taglibs as FreeMarker macros. See each
\href{/docs/7-2/reference/-/knowledge_base/r/front-end-taglibs}{taglib's
documentation} for more information on using the taglib in your
FreeMarker templates. This reference guide lists the available
FreeMarker macros that Liferay DXP offers.

\noindent\hrulefill

\begin{longtable}[]{@{}
  >{\raggedright\arraybackslash}p{(\columnwidth - 6\tabcolsep) * \real{0.2500}}
  >{\raggedright\arraybackslash}p{(\columnwidth - 6\tabcolsep) * \real{0.2500}}
  >{\raggedright\arraybackslash}p{(\columnwidth - 6\tabcolsep) * \real{0.2500}}
  >{\raggedright\arraybackslash}p{(\columnwidth - 6\tabcolsep) * \real{0.2500}}@{}}
\toprule\noalign{}
\begin{minipage}[b]{\linewidth}\raggedright
Macro
\end{minipage} & \begin{minipage}[b]{\linewidth}\raggedright
Parameters
\end{minipage} & \begin{minipage}[b]{\linewidth}\raggedright
Description
\end{minipage} & \begin{minipage}[b]{\linewidth}\raggedright
Example
\end{minipage} \\
\midrule\noalign{}
\endhead
\bottomrule\noalign{}
\endlastfoot
breadcrumbs & default\_preferences & Adds the Breadcrumbs portlet with
optional preferences &
\texttt{\textless{}@liferay.breadcrumbs\ /\textgreater{}} \\
control\_menu & N/A & Adds the Control Menu portlet &
\texttt{\textless{}@liferay.control\_menu\ /\textgreater{}} \\
css & file\_name & Adds an external stylesheet with the specified file
name location &
\texttt{\textless{}@liferay.css\ file\_name="\$\{css\_folder\}/mycss.css"/\textgreater{}} \\
date & format & Prints the date in the current locale with the given
format &
\texttt{\textless{}@liferay.date\ format="/yyyy/MM/dd/HH/"\ /\textgreater{}} \\
js & file\_name & Adds an external JavaScript file with the specified
file name source &
\texttt{\textless{}@liferay.js\ file\_name="\$\{javascript\_folder\}/myJs.js"/\textgreater{}} \\
language & key & Prints the specified language key in the current locale
&
\texttt{\textless{}@liferay.language\ key="last-modified"\ /\textgreater{}} \\
language\_format & argumentskey & Formats the given language key with
the specified arguments. For example, passing \texttt{go-to-x} as the
key and \texttt{Mars} as the arguments prints \emph{Go to Mars}. &
\texttt{\textless{}@liferay.language\_format\ arguments="\$\{site\_name\}"\ key="go-to-x"\ /\textgreater{}} \\
languages & default\_preferences & Adds the Languages portlet with
optional preferences &
\texttt{\textless{}@liferay.languages\ /\textgreater{}} \\
navigation\_menu & default\_preferencesinstance\_id & Adds the
Navigation Menu portlet with optional preferences and instance ID. &
\texttt{\textless{}@liferay.navigation\_menu\ /\textgreater{}} \\
search & default\_preferences & Adds the Search portlet with optional
preferences & \texttt{\textless{}@liferay.search\ /\textgreater{}} \\
search\_bar & default\_preferences & Adds the Search Bar portlet with
optional preferences &
\texttt{\textless{}@liferay.search\_bar\ /\textgreater{}} \\
user\_personal\_bar & N/A & Adds the User Personal Bar portlet &
\texttt{\textless{}@liferay.user\_personal\_bar\ /\textgreater{}} \\
\end{longtable}

\noindent\hrulefill

A few reference examples are shown below.

\section{Reference Examples}\label{reference-examples}

The example below includes a language key with the \texttt{language}
macro directive along with its language \texttt{key} parameter:

\begin{verbatim}
<@liferay.language key="powered-by" />
\end{verbatim}

This example includes the Search portlet with its
\href{/docs/7-2/frameworks/-/knowledge_base/f/theming-portlets\#portlet-decorators}{Portlet
Decorator} portlet preference set to barebone:

\begin{verbatim}
<@liferay.search default_preferences=
  freeMarkerPortletPreferences.getPreferences(
    "portletSetupPortletDecoratorId", "barebone"
  ) 
/>
\end{verbatim}

You can also pass multiple portlet preferences in an object, as shown in
the example below for the Navigation Menu portlet:

\begin{verbatim}
<#assign secondaryNavigationPreferencesMap = 
  {
    "displayStyle": "ddmTemplate_NAVBAR-BLANK-JUSTIFIED-FTL", 
    "portletSetupPortletDecoratorId": "barebone", 
    "rootLayoutType": "relative", 
    "siteNavigationMenuId": "0", 
    "siteNavigationMenuType": "1"
  } 
/>

<@liferay.navigation_menu
  default_preferences=
  freeMarkerPortletPreferences.getPreferences(secondaryNavigationPreferencesMap)
  instance_id="main_navigation_menu"
/>
\end{verbatim}

\noindent\hrulefill

\textbf{Note:} Portlet preferences are unique to each portlet, so first
you must determine which preferences you want to configure. There are
two ways to determine the proper key/value pair for a portlet
preference. The first is to set the portlet preference manually, and
then check the values in the \texttt{portletPreferences.preferences}
column of the database as a hint for what to configure.

Another approach is to search each app in your bundle for the keyword
\texttt{preferences-\/-}. This returns app JSPs that have the portlet
preferences defined for the portlet.

\chapter{Front-End Taglibs}\label{front-end-taglibs}

You have access to a powerful set of taglibs for creating commonly used
UI components in your apps, themes, and web content. The following
taglibs are covered in this section:

\begin{itemize}
\item
  AUI: create common UI components such as forms, buttons, and more.
\item
  Chart: visualize data. Create bar charts, line charts, scatter charts,
  spline charts, and much more.
\item
  Clay: create
  \href{https://clayui.com/docs/components/alerts.html}{Clay
  components}, such as alerts, buttons, drop-down menus, form elements,
  and more for your apps.
\item
  Frontend: create UI components commonly used throughout Portal's apps,
  such as add menus, cards, management bars, and more.
\item
  Liferay UI: create common UI components such as icons, tabs, and more.
\item
  Liferay Util: load additional resources, define parameters, buffer
  content, and more.
\end{itemize}

\noindent\hrulefill

\textbf{Note:} Each taglib is available as a FreeMarker macro, except
for the Chart taglib. The Chart taglib is \textbf{not} available as a
FreeMarker macro. The articles in this section provide the proper syntax
to use for each macro. See the
\href{/docs/7-2/reference/-/knowledge_base/r/product-freemarker-macros}{FreeMarker
Taglib Mappings reference} for a complete list of the available
FreeMarker taglib macros.

\noindent\hrulefill

In this section, you'll learn how to use taglibs to build awesome user
interfaces for your apps!

\chapter{Liferay Theme Objects Available in
JSPs}\label{liferay-theme-objects-available-in-jsps}

When you include the
\texttt{\textless{}liferay-theme:defineObjects\textgreater{}} tag in
your JSP, you gain access to several Liferay theme objects via
variables. These objects are described in the table below:

\noindent\hrulefill

\begin{longtable}[]{@{}
  >{\raggedright\arraybackslash}p{(\columnwidth - 2\tabcolsep) * \real{0.5000}}
  >{\raggedright\arraybackslash}p{(\columnwidth - 2\tabcolsep) * \real{0.5000}}@{}}
\toprule\noalign{}
\begin{minipage}[b]{\linewidth}\raggedright
Object
\end{minipage} & \begin{minipage}[b]{\linewidth}\raggedright
Description
\end{minipage} \\
\midrule\noalign{}
\endhead
\bottomrule\noalign{}
\endlastfoot
\texttt{account} & The user's Account object. This object maps to the
Account table in the Liferay database. \\
\texttt{colorScheme} & An object representing the current color scheme
in the theme that is being rendered by the portal \\
\texttt{company} & The current Company object. This represents the
portal instance on which the user is currently navigating. \\
\texttt{contact} & The user's Contact object. This object maps to the
Contact table in the Liferay database. \\
\texttt{layout} & The page to which the user has currently navigated \\
\texttt{layoutTypePortlet} & This object can be used to programmatically
add or remove portlets from a page. \\
\texttt{locale} & The current user's locale, as defined by Java \\
\texttt{permissionChecker} & An object that can determine---given a
particular resource---whether the current user has a particular
permission for that resource \\
\texttt{plid} & A portal layout ID. This is a unique identifier for any
page that exists in the portal, across all portal instances. \\
\texttt{portletDisplay} & An object that gives the programmer access to
many attributes of the current portlet, including the portlet name, the
portlet mode, the ID of the column on the layout in which it resides,
and more \\
\texttt{realUser} & When an administrator is impersonating a user, this
variable tracks the administrator's User object. \\
\texttt{scopeGroupId} & By default, contains the groupId for the
community or organization in which this portlet resides. If the
scopeable attribute is set to true, this may contain a unique scope
identifier for custom scopes, such as the page scope, if the portlet has
been configured to use a custom scope. \\
\texttt{theme} & An object representing the current theme that is being
rendered by the portal \\
\texttt{themeDisplay} & A runtime object that contains many useful
items, such as the logged-in user, the layout, logo information, paths,
and much more \\
\texttt{timeZone} & The current user's time zone, as defined by Java \\
\texttt{user} & The User object representing the current user \\
\end{longtable}

\chapter{Liferay Portlet Objects Available in
JSPs}\label{liferay-portlet-objects-available-in-jsps}

You may have noticed the
\texttt{\textless{}liferay-portlet:defineObjects\textgreater{}} tag in
your JSPs. Similar to the
\href{/docs/7-2/reference/-/knowledge_base/r/liferay-theme-objects-available-in-jsps}{theme:defineObjects}
tag, when you include this tag in your JSP, you gain access to several
variables that, in this case, return useful information about your
portlet. Note that the JSR-286 specification defines four lifecycle
methods for a portlet: processAction, processEvent, render, and
serveResource. Some of the variables defined by the
\texttt{\textless{}portlet:defineObjects/\textgreater{}} tag are only
available to a JSP if the JSP was included during the appropriate phase
of the portlet lifecycle. These objects are described in the table
below:

\noindent\hrulefill

\begin{longtable}[]{@{}
  >{\raggedright\arraybackslash}p{(\columnwidth - 2\tabcolsep) * \real{0.3529}}
  >{\raggedright\arraybackslash}p{(\columnwidth - 2\tabcolsep) * \real{0.6471}}@{}}
\toprule\noalign{}
\begin{minipage}[b]{\linewidth}\raggedright
Object
\end{minipage} & \begin{minipage}[b]{\linewidth}\raggedright
Description
\end{minipage} \\
\midrule\noalign{}
\endhead
\bottomrule\noalign{}
\endlastfoot
\texttt{ActionRequest\ actionRequest} & Represents the request sent to
the portlet to handle an action. \texttt{actionRequest} is only
available to a JSP if the JSP was included during the action-processing
phase. \\
\texttt{ActionResponse\ actionResponse} & Represents the portlet
response to an action request. \texttt{actionResponse} is only available
to a JSP if the JSP was included in the action-processing phase. \\
\texttt{EventRequest\ eventRequest} & Represents the request sent to the
portlet to handle an event. \texttt{eventRequest} is only available to a
JSP if the JSP was included during the event-processing phase. \\
\texttt{EventResponse\ eventResponse} & Represents the portlet response
to an event request. \texttt{eventResponse} is only available to a JSP
if the JSP was included in the event-processing phase. \\
\texttt{HeaderRequest\ headerRequest} & Represents the request sent to
the portlet to handle its HTML header or HEAD section.
\texttt{headerRequest} is only available to a JSP if the JSP was
included during the header-processing phase. \\
\texttt{HeaderResponse\ headerResponse} & Represents the portlet
response to a header request. \texttt{headerResponse} is only available
to a JSP if the JSP was included in the header-processing phase. \\
\href{https://docs.liferay.com/dxp/portal/7.2-latest/javadocs/portal-kernel/com/liferay/portal/kernel/portlet/LiferayPortletRequest.html}{\texttt{LiferayPortletRequest\ liferayPortletRequest}}
& Provides access to the \texttt{HttpServletRequest}, the
\texttt{Portlet}, and the portlet name and lifecycle value.
\texttt{liferayPortletRequest} is available in all portlet phases. \\
\href{https://docs.liferay.com/dxp/portal/7.2-latest/javadocs/portal-kernel/com/liferay/portal/kernel/portlet/LiferayPortletResponse.html}{\texttt{LiferayPortletResponse\ liferayPortletResponse}}
& Includes the properties returned to the portal and provides a means to
add or change properties. \texttt{liferayPortletResponse} is available
in all portlet phases. \\
\texttt{RenderRequest\ renderRequest} & Represents the request sent to
the portlet to render the portlet. \texttt{renderRequest} is only
available to a JSP if the JSP was included during the render request
phase. \\
\texttt{RenderResponse\ renderResponse} & Represents an object that
assists the portlet in sending a response to the portal.
\texttt{renderResponse} is only available to a JSP if the JSP was
included during the render request phase. \\
\texttt{ResourceRequest\ resourceRequest} & Represents the request sent
to the portlet for rendering resources. \texttt{resourceRequest} is only
available to a JSP if the JSP was included during the resource-serving
phase. \\
\texttt{ResourceResponse\ resourceResponse} & Represents an object that
assists the portlet in rendering a resource. \texttt{resourceResponse}
is only available to a JSP if the JSP was included in the
resource-serving phase. \\
\texttt{PortletConfig\ portletConfig} & Represents the portlet's
configuration including, the portlet's name, initialization parameters,
resource bundle, and application context. \texttt{portletConfig} is
always available to a portlet JSP, regardless of the request-processing
phase in which it was included. \\
\texttt{PortletPreferences\ portletPreferences} & Provides access to a
portlet's preferences. \texttt{portletPreferences} is always available
to a portlet JSP, regardless of the request-processing phase in which it
was included. \\
\texttt{Map\textless{}String,\ String{[}{]}\textgreater{}\ portletPreferencesValues}
& Provides a Map equivalent to the \texttt{portletPreferences.getMap()}
call or an empty Map if no portlet preferences exist. \\
\texttt{PortletSession\ portletSession} & Provides a way to identify a
user across more than one request and to store transient information
about a user. A \texttt{portletSession} is created for each user client.
\texttt{portletSession} is always available to a portlet JSP, regardless
of the request-processing phase in which it was included.
\texttt{portletSession} is \texttt{null} if no session exists. \\
\texttt{Map\textless{}String,\ Object\textgreater{}\ portletSessionScope}
& Provides a Map equivalent to the
\texttt{PortletSession.getAtrributeMap()} call or an empty Map if no
session attributes exist. \\
\end{longtable}

\noindent\hrulefill

For more details, visit the
\href{https://docs.liferay.com/portlet-api/3.0/javadocs/}{Portlet 3.0
API Javadoc}.

\chapter{Using the Liferay UI Taglib}\label{using-the-liferay-ui-taglib}

The Liferay UI tag library provides tags that implement commonly used UI
components. These tags make your markup consistent, responsive, and
accessible.

You can find a list of the available Liferay UI taglibs in the
\href{https://docs.liferay.com/dxp/portal/7.2-latest/taglibs/util-taglib/liferay-ui/tld-summary.html}{Liferay
UI taglibdocs}. Each taglib has a list of attributes that can be passed
to the tag. Some of these are required and some are optional. See the
taglibdocs to view the requirements for each tag. You'll find the full
markup generated by the tags in their JSPs in their
\href{https://github.com/liferay/liferay-portal/tree/7.2.x/portal-web/docroot/html/taglib/ui}{Liferay
Github Repo} folders.

To use the Liferay-UI taglib library in your apps, you must add the
following declaration to your JSP:

\begin{verbatim}
<%@ taglib prefix="liferay-ui" uri="http://liferay.com/tld/ui" %>
\end{verbatim}

The Liferay-UI taglib is also available via a macro for your FreeMarker
theme and web content templates. Follow this syntax:

\begin{verbatim}
<@liferay_ui["tag-name"] attribute="string value" attribute=10 />
\end{verbatim}

This section covers how to create UI components with the Liferay UI
taglibs. Each article contains code examples along with a screenshot of
the resulting UI.

\chapter{Liferay UI Icons}\label{liferay-ui-icons}

The Liferay UI taglibs provide several icons you can include in your
apps. To add an icon to your app, use the \texttt{liferay-ui:icon} tag
and specify the icon with either the \texttt{icon},
\texttt{iconCssClass}, or \texttt{image} attribute. An example of each
use case is shown below.

The \texttt{image} attribute specifies
\href{https://github.com/liferay/liferay-portal/tree/7.2.x/modules/apps/frontend-theme/frontend-theme-unstyled/src/main/resources/META-INF/resources/_unstyled/images}{Liferay
UI icons} to use (as defined in the Unstyled theme's
\texttt{images/common} folder). Here's an example configuration for a
JSP:

\begin{verbatim}
<div class="col-md-3">
    <liferay-ui:icon image="subscribe" />

    <span class="ml-2">Subscribe</span>
</div>
\end{verbatim}

\begin{figure}
\centering
\includegraphics{./images/liferay-ui-taglib-icon-subscribe.png}
\caption{Use the image attribute to use a theme icon.}
\end{figure}

The Liferay UI taglib also exposes language flag icons. To use a
language flag icon, provide the \texttt{../language/} relative path
before the icon's name. Below is an example snippet from the Web Content
Search portlet that displays the current language's flag along with a
localized message:

\begin{verbatim}
<liferay-ui:icon
    image='<%= "../language/" + languageId %>'
    message='<%= LanguageUtil.format(
        request,
        "this-result-comes-from-the-x-version-of-this-content",
        snippetLocale.getDisplayLanguage(locale),
        false
    ) %>'
/>
\end{verbatim}

You can achieve the same result in FreeMarker with the following code
that uses the available
\href{https://github.com/liferay/liferay-portal/blob/7.2.x/modules/apps/frontend-theme/frontend-theme-unstyled/src/main/resources/META-INF/resources/_unstyled/templates/init.ftl}{\texttt{init.ftl}
variables} and
\href{/docs/7-2/reference/-/knowledge_base/r/product-freemarker-macros}{Liferay
DXP macros}:

\begin{verbatim}
<#assign flag_message>
    <@liferay.language_format 
        arguments=language 
        key="this-result-comes-from-the-x-version-of-this-content" 
    />
</#assign>

<@liferay_ui["icon"]
    image="../language/${language_id}"
    message=flag_message
/>
\end{verbatim}

The full list of available icons is shown in the figures below:

\begin{figure}
\centering
\includegraphics{./images/liferay-ui-taglib-icons.png}
\caption{The Liferay UI taglib offers multiple icons for use in your
app.}
\end{figure}

\begin{figure}
\centering
\includegraphics{./images/liferay-ui-taglib-icon-flags.png}
\caption{Liferay UI icons can be configured based on language.}
\end{figure}

The \texttt{icon} attribute specifies
\href{https://fontawesome.com/v3.2.1/icons/}{Font Awesome icons} to use:

\begin{verbatim}
<liferay-ui:icon icon="angle-down" />
\end{verbatim}

\begin{figure}
\centering
\includegraphics{./images/liferay-ui-taglib-icon-angle-down.png}
\caption{You can use the icon attribute to include Font Awesome icons in
your app.}
\end{figure}

The \texttt{iconCssClass} attribute specifies a
\href{http://marcoceppi.github.io/bootstrap-glyphicons/}{glyphicon} to
use:

\begin{verbatim}
<liferay-ui:icon
    iconCssClass="icon-remove-sign"
    label="<%= true %>"
    message="unsubscribe"
    url="<%= unsubscribeURL %>"
/>
\end{verbatim}

\begin{figure}
\centering
\includegraphics{./images/liferay-ui-taglib-icon-css-class.png}
\caption{You can use Font Awesome icons in your app.}
\end{figure}

The examples above use some of the icon's available attributes. See the
\href{https://docs.liferay.com/dxp/portal/7.2-latest/taglibs/util-taglib/liferay-ui/icon.html}{Icon
taglibdocs} for the full list.

\section{Related Topics}\label{related-topics}

\begin{itemize}
\tightlist
\item
  \href{/docs/7-2/reference/-/knowledge_base/r/clay-icons}{Clay Icons}
\item
  \href{/docs/7-2/reference/-/knowledge_base/r/liferay-ui-icon-lists}{Liferay
  UI Icon Lists}
\item
  \href{/docs/7-2/reference/-/knowledge_base/r/liferay-ui-icon-menus}{Liferay
  UI Icon Menus}
\end{itemize}

\chapter{Liferay UI Icon Lists}\label{liferay-ui-icon-lists}

An icon list displays icons in a horizontal list, instead of in a pop-up
navigation menu like an
\href{/docs/7-2/reference/-/knowledge_base/r/liferay-ui-icon-menus}{icon
menu}. You can see an example of an icon list menu in a message board
thread. The thread's actions are visible at all times for
administrators:

\begin{figure}
\centering
\includegraphics{./images/liferay-ui-taglib-icon-list.png}
\caption{Icon lists display an app's actions at all times.}
\end{figure}

Create the list menu with the \texttt{liferay-ui:icon-list} tag and nest
\href{/docs/7-2/reference/-/knowledge_base/r/liferay-ui-icons}{icons}
for each list item, as shown below:

\begin{verbatim}
<div class="thread-actions">
    <liferay-ui:icon-list>

        <liferay-ui:icon
        iconCssClass="icon-lock"
        message="permissions"
        method="get"
        url="<%= permissionsURL %>"
        useDialog="<%= true %>"
        />

        <liferay-rss:rss
        delta="<%= rssDelta %>"
        displayStyle="<%= rssDisplayStyle %>"
        feedType="<%= rssFeedType %>"
        url="<%= MBRSSUtil.getRSSURL(plid, 0, message.getThreadId(), 0, themeDisplay) %>"
        />

        <liferay-ui:icon
        iconCssClass="icon-remove-sign"
        message="unsubscribe"
        url="<%= unsubscribeURL %>"
        />

        <liferay-ui:icon
        iconCssClass="icon-lock"
        message="lock"
        url="<%= lockThreadURL %>"
        />

        <liferay-ui:icon
        iconCssClass="icon-move"
        message="move"
        url="<%= editThreadURL %>"
        />

        <liferay-ui:icon-delete
        showIcon="<%= true %>"
        trash="<%= trashHelper.isTrashEnabled(themeDisplay.getScopeGroupId()) %>"
        url="<%= deleteURL %>"
        />
    </liferay-ui:icon-list>
</div>
\end{verbatim}

See the
\href{https://docs.liferay.com/dxp/portal/7.2-latest/taglibs/util-taglib/liferay-ui/icon-list.html}{Icon
List taglibdocs} for the full list of available attributes.

\section{Related Topics}\label{related-topics-1}

\begin{itemize}
\tightlist
\item
  \href{/docs/7-2/reference/-/knowledge_base/r/clay-icons}{Clay Icons}
\item
  \href{/docs/7-2/reference/-/knowledge_base/r/liferay-ui-icon-menus}{Liferay
  UI Icon Menus}
\item
  \href{/docs/7-2/reference/-/knowledge_base/r/liferay-ui-icons}{Liferay
  UI Icons}
\end{itemize}

\chapter{Liferay UI Icon Menus}\label{liferay-ui-icon-menus}

You can add a pop-up navigation menu to your app with the
\texttt{liferay-ui:icon-menu} tag. Icon menus display menu options when
needed, storing them away in a collapsed menu when they're not. This
keeps the UI clean and uncluttered. Just as with an icon list, you nest
\href{/docs/7-2/reference/-/knowledge_base/r/liferay-ui-icons}{icons}
for each navigation item. You can see an example of a icon menu in a
site's actions menu in the My Sites portlet:

\begin{figure}
\centering
\includegraphics{./images/liferay-ui-taglib-icon-menu.png}
\caption{Setting up an icon menu is a piece of cake.}
\end{figure}

Example JSP configuration:

\begin{verbatim}
<liferay-ui:icon-menu
    direction="left-side"
    icon="<%= StringPool.BLANK %>"
    markupView="lexicon"
    message="<%= StringPool.BLANK %>"
    showWhenSingleIcon="<%= true %>"
>

                <liferay-ui:icon
                    message="go-to-public-pages"
                    target="_blank"
                    url="<%= group.getDisplayURL(themeDisplay, false) %>"
                />

                <liferay-ui:icon
                    message="leave"
                    url="<%= leaveURL %>"
                />

</liferay-ui:icon-menu>
\end{verbatim}

Note that the \texttt{url} attribute is required for icons to render
properly. See the
\href{https://docs.liferay.com/dxp/portal/7.2-latest/taglibs/util-taglib/liferay-ui/icon-menu.html}{Icon
Menu taglibdocs} for the full list of attributes.

\section{Related Topics}\label{related-topics-2}

\begin{itemize}
\tightlist
\item
  \href{/docs/7-2/reference/-/knowledge_base/r/clay-icons}{Clay Icons}
\item
  \href{/docs/7-2/reference/-/knowledge_base/r/liferay-ui-icon-lists}{Liferay
  UI Icon Lists}
\item
  \href{/docs/7-2/reference/-/knowledge_base/r/liferay-ui-icons}{Liferay
  UI Icons}
\end{itemize}

\chapter{Liferay UI Tabs}\label{liferay-ui-tabs}

Tabs create dividers that organize content into individual sections.
Content can be embedded or included from another JSP.

To add tabs to your app, use the
\texttt{\textless{}liferay-ui:tabs\textgreater{}} tag and specify each
tab's name as a comma-separated list for the \texttt{names} attribute.
For example, three tabs named \texttt{tab1}, \texttt{tab2}, and
\texttt{tab3}, look like this in the JSP:

\begin{verbatim}
<liferay-ui:tabs names="tab1,tab2,tab3">

</liferay-ui:tabs>
\end{verbatim}

Each tab requires a corresponding section to display content. Nest
\texttt{liferay-ui:section} tags for each of the tabs. Within each
section, you can add HTML content or add content indirectly by including
content from another JSP (via the
\texttt{\textless{}\%@\ includefile="filepath"\%\textgreater{}}
directive). The example snippet below is from the Calendar portlet's
\texttt{configuration.jsp}:

\begin{verbatim}
<liferay-ui:tabs
    names='<%= "user-settings,display-settings,rss" %>'
    param="tabs2"
    refresh="<%= false %>"
    type="tabs nav-tabs-default"
>
    <liferay-ui:section>
        <%@ include file="/configuration/user_settings.jspf" %>
    </liferay-ui:section>

    <liferay-ui:section>
        <%@ include file="/configuration/display_settings.jspf" %>
    </liferay-ui:section>

    <liferay-ui:section>
        <%@ include file="/configuration/rss.jspf" %>
    </liferay-ui:section>
</liferay-ui:tabs>
\end{verbatim}

\begin{figure}
\centering
\includegraphics{./images/liferay-ui-taglib-tabs.png}
\caption{Tabs are a useful way to organize configuration options into
individual sections within the same UI.}
\end{figure}

The example above uses some of the tab's available attributes. See the
\href{https://docs.liferay.com/dxp/portal/7.2-latest/taglibs/util-taglib/liferay-ui/tabs.html}{Tabs
taglibdocs} for the full list of attributes.

\section{Related Topics}\label{related-topics-3}

\begin{itemize}
\tightlist
\item
  \href{/docs/7-2/reference/-/knowledge_base/r/clay-navigation-bars}{Clay
  Navigation Bars}
\item
  \href{/docs/7-2/reference/-/knowledge_base/r/clay-dropdown-menus-and-action-menus}{Clay
  Dropdown Menus and Action Menus}
\item
  \href{/docs/7-2/reference/-/knowledge_base/r/liferay-ui-icon-help}{Liferay
  UI Icon Help}
\end{itemize}

\chapter{Liferay UI Icon Help}\label{liferay-ui-icon-help}

The icon help tag lets you communicate additional information to your
users in an unobtrusive way. It renders as an iconic question mark that
provides more information through a pop-up tooltip on mouse over. You
can see an example of this in the Control Panel:

\begin{figure}
\centering
\includegraphics{./images/liferay-ui-taglib-tooltip.png}
\caption{Here's an example of the icon help tag.}
\end{figure}

\noindent\hrulefill

\textbf{Note:} If you have installed a custom theme you may also need to
add the following imports to your \texttt{view.jsp} to make
\texttt{liferay-ui:icon-help} tag work:

\begin{verbatim}
 <%@ taglib uri="http://liferay.com/tld/theme" prefix="liferay-theme"%>
 <liferay-theme:defineObjects />
\end{verbatim}

\noindent\hrulefill

Add the \texttt{\textless{}liferay-ui:icon-help/\textgreater{}} tag next
to the UI that needs tooltip information. Define the informational text
with the required \texttt{message} attribute. Below is an example
snippet for one of the Server Administration's clean up actions:

\begin{verbatim}
<h5>
    <liferay-ui:message key="clean-up-permissions" />
    <liferay-ui:icon-help message="clean-up-permissions-help" />
</h5>
\end{verbatim}

\begin{figure}
\centering
\includegraphics{./images/liferay-ui-taglib-tooltip-02.png}
\caption{help icons are used throughout the Control Panel.}
\end{figure}

Note that the message is supplied via a
\href{/docs/7-2/frameworks/-/knowledge_base/f/localizing-your-application}{language
key}. While you can use a string for the tooltip's message for testing
purposes, a language key is considered best practice and should be used
in production.

\section{Related Topics}\label{related-topics-4}

\begin{itemize}
\tightlist
\item
  \href{/docs/7-2/reference/-/knowledge_base/r/clay-badges}{Clay Badges}
\item
  \href{/docs/7-2/reference/-/knowledge_base/r/clay-stickers}{Clay
  Stickers}
\item
  \href{/docs/7-2/reference/-/knowledge_base/r/liferay-ui-icon-menus}{Liferay
  UI Icon Menus}
\end{itemize}

\chapter{Using Liferay Front-end Taglibs in Your
Portlet}\label{using-liferay-front-end-taglibs-in-your-portlet}

The Liferay Front-end tag library provides a set of tags for creating
common front-end UI components in your app.

To use the Front-end taglib in you apps, add the following declaration
to your JSP:

\begin{verbatim}
<%@ taglib prefix="liferay-frontend" uri="http://liferay.com/tld/frontend" %>
\end{verbatim}

The Liferay Front-end taglib is also available via a macro for your
FreeMarker theme templates and web content templates. Follow this
syntax:

\begin{verbatim}
<@liferay_frontend["tag-name"] attribute="string value" attribute=10 />
\end{verbatim}

The following Front-end UI components are covered in this section:

\begin{itemize}
\tightlist
\item
  \href{/docs/7-2/reference/-/knowledge_base/r/liferay-front-end-add-menu}{Add
  Menu}
\item
  \href{/docs/7-2/reference/-/knowledge_base/r/liferay-front-end-cards}{Cards}
\item
  \href{/docs/7-2/reference/-/knowledge_base/r/liferay-front-end-info-bar}{Info
  Bar}
\item
  \href{/docs/7-2/reference/-/knowledge_base/r/liferay-front-end-management-bar}{Management
  Bar}
\end{itemize}

Each article contains a set of examples along with a screenshot of the
resulting UI.

\chapter{Liferay Front-end Add Menu}\label{liferay-front-end-add-menu}

The add menu tag creates an add menu button for one or multiple items.
It's used for actions that add entities (e.g.~a new blog entry), and is
part of the Management Bar. Use the
\texttt{\textless{}liferay-frontend:add-menu\textgreater{}} tag to
create the add menu and nest a
\texttt{\textless{}liferay-frontend:add-menu-item\textgreater{}} tag for
each item.

\noindent\hrulefill

\textbf{Note:} This pattern is deprecated as of 7.0. We recommend that
you use the Clay Management Toolbar's
\href{/docs/7-2/reference/-/knowledge_base/r/clay-management-toolbar\#creation-menu}{creation
menu pattern} instead.

\noindent\hrulefill

When the menu has one item, the button triggers the item's action as
shown in the example below for the Blogs Admin App:

\begin{verbatim}
<liferay-frontend:management-bar>
  <liferay-frontend:management-bar-buttons>
      ...
      <liferay-frontend:add-menu
        inline="<%= true %>"
      >
        <liferay-frontend:add-menu-item
          title='<%= LanguageUtil.get(request, "add-blog-entry") %>'
          url="<%= addEntryURL %>"
        />
      </liferay-frontend:add-menu>

  </liferay-frontend:management-bar-buttons>
</liferay-frontend:management-bar>
\end{verbatim}

\begin{figure}
\centering
\includegraphics{./images/liferay-frontend-taglib-add-menu-one-item.png}
\caption{The add button pattern consists of an \texttt{add-menu} tag and
at least one \texttt{add-menu-item} tag.}
\end{figure}

When the menu has multiple items, they display in a pop-up menu. For
example, the Message Boards Admin application has the configuration
below:

\begin{verbatim}
<liferay-frontend:add-menu>
    ...
    <liferay-frontend:add-menu-item title='<%= LanguageUtil.get(request,
    "thread") %>' url="<%= addMessageURL.toString() %>" />
    ...
    <liferay-frontend:add-menu-item title='<%= LanguageUtil.get(request,
    (categoryId == MBCategoryConstants.DEFAULT_PARENT_CATEGORY_ID) ?
    "category[message-board]" : "subcategory[message-board]") %>'
    url="<%= addCategoryURL.toString() %>" />
    ...
</liferay-frontend:add-menu>
\end{verbatim}

\begin{figure}
\centering
\includegraphics{./images/liferay-frontend-taglib-add-menu-items.png}
\caption{The add button pattern consists of an \texttt{add-menu} tag and
at least one \texttt{add-menu-item} tag.}
\end{figure}

The examples above use some of the available attributes. See the
\href{https://docs.liferay.com/dxp/apps/foundation/latest/taglibdocs/liferay-frontend/add-menu.html}{add
menu} and
\href{https://docs.liferay.com/dxp/apps/foundation/latest/taglibdocs/liferay-frontend/add-menu-item.html}{add
menu item} taglibdocs for the full list of available attributes for the
tags.

\section{Related Topics}\label{related-topics-5}

\begin{itemize}
\tightlist
\item
  \href{/docs/7-2/reference/-/knowledge_base/r/liferay-front-end-cards}{Liferay
  Frontend Cards}
\item
  \href{/docs/7-2/reference/-/knowledge_base/r/liferay-front-end-info-bar}{Liferay
  Frontend Info Bar}
\item
  \href{/docs/7-2/reference/-/knowledge_base/r/liferay-front-end-management-bar}{Liferay
  Frontend Management Bar}
\end{itemize}

\chapter{Liferay Front-end Cards}\label{liferay-front-end-cards}

If you have data you want to compare that's heavy on image usage, cards
are the component for the job. Cards visually represent data in a
minimal and compact format. Use them for images, document libraries,
user profiles, and more. There are four main types of Cards:

\begin{itemize}
\tightlist
\item
  Horizontal Cards
\item
  Icon Cards
\item
  Vertical Cards
\item
  User Cards
\end{itemize}

Examples of each card are shown below.

\section{Horizontal Card}\label{horizontal-card}

Horizontal cards are used primarily to display documents, such as files
and folders. An example configuration is shown below:

\begin{verbatim}
<liferay-frontend:horizontal-card
    text="Documents"
  url="https://portal.liferay.dev/docs/7-1/tutorials/-/knowledge_base/t/clay-icons"
>
    <liferay-frontend:horizontal-card-col>
                <liferay-frontend:horizontal-card-icon
                    icon="folder"
                />
    </liferay-frontend:horizontal-card-col>
</liferay-frontend:horizontal-card>
\end{verbatim}

\begin{figure}
\centering
\includegraphics{./images/liferay-frontend-taglib-cards-horizontal.png}
\caption{Horizontal cards are perfect to display files and documents.}
\end{figure}

The
\texttt{\textless{}liferay-frontend:horizontal-card-icon\textgreater{}}
tag uses \href{/docs/7-2/reference/-/knowledge_base/r/clay-icons}{Clay
Icons} for its \texttt{icon} attribute.

\section{Icon Vertical Card}\label{icon-vertical-card}

Icon vertical cards, as the name suggests, are cards that display
information in a vertical format that emphasizes an icon. These cards
show content that doesn't have an associated image. Instead, an icon
representing the type of content is displayed. The example snippet below
displays information for a web content article:

\begin{verbatim}
<liferay-frontend:icon-vertical-card
  cssClass="article-preview-content"
  icon="web-content"
  title="<%= title %>"
>
  <liferay-frontend:vertical-card-sticker-bottom>
    <liferay-ui:user-portrait
      cssClass="sticker sticker-bottom"
      userId="<%= assetRenderer.getUserId() %>"
    />
  </liferay-frontend:vertical-card-sticker-bottom>

  <liferay-frontend:vertical-card-footer>
    <aui:workflow-status 
      markupView="lexicon" 
      showIcon="<%= false %>" 
      showLabel="<%= false %>" 
      status="<%= article.getStatus() %>" 
    />
  </liferay-frontend:vertical-card-footer>
</liferay-frontend:icon-vertical-card>
\end{verbatim}

\begin{figure}
\centering
\includegraphics{./images/liferay-frontend-taglib-cards-icon-vertical.png}
\caption{Vertical icon cards are perfect to display an entity selection,
such as a web content article.}
\end{figure}

\section{Vertical Card}\label{vertical-card}

Vertical cards display information in a vertical card format, as opposed
to a horizontal format. If the content has an associated image (like a
blog header image) you can use a vertical card to display the image. If
there is no associated image, you can use an icon vertical card to
represent the content's type instead (e.g.~a PDF file). The example
below displays a vertical card for a web content article when an image
preview is available:

\begin{verbatim}
<liferay-frontend:vertical-card
  cssClass="article-preview-content"
  imageUrl="<%= articleImageURL %>"
  title="<%= title %>"
>
  <liferay-frontend:vertical-card-sticker-bottom>
    <liferay-ui:user-portrait
      cssClass="sticker sticker-bottom"
      userId="<%= assetRenderer.getUserId() %>"
    />
  </liferay-frontend:vertical-card-sticker-bottom>

  <liferay-frontend:vertical-card-footer>
    <aui:workflow-status 
      markupView="lexicon" 
      showIcon="<%= false %>" 
      showLabel="<%= false %>" 
      status="<%= article.getStatus() %>" 
    />
  </liferay-frontend:vertical-card-footer>
</liferay-frontend:vertical-card>
\end{verbatim}

\begin{figure}
\centering
\includegraphics{./images/liferay-frontend-taglib-cards-vertical.png}
\caption{Vertical cards are perfect to display files and documents.}
\end{figure}

\section{HTML Vertical Card}\label{html-vertical-card}

The HTML Vertical card lets you display custom HTML in the header of the
vertical card. The example below embeds a video:

\begin{verbatim}
<liferay-util:buffer var = "customThumbnailHtml">
    <div class="embed-responsive embed-responsive-16by9">
      <iframe class="embed-responsive-item" 
    src="https://www.youtube.com/embed/8Bg9jPJpGOM?rel=0" 
    allowfullscreen></iframe>
    </div>
</liferay-util:buffer>

<div class="container">
  <div class="row">
    <div class="col-md-4">
      <liferay-frontend:html-vertical-card
        html="<%= customThumbnailHtml %>"
        title="My Video"
      >
      </liferay-frontend:html-vertical-card>
    </div>
  </div>
</div>
\end{verbatim}

\begin{figure}
\centering
\includegraphics{./images/liferay-frontend-taglib-cards-html-vertical.png}
\caption{Html vertical cards let you display custom HTML in the card's
header.}
\end{figure}

\section{User Vertical Card}\label{user-vertical-card}

The User Vertical card displays user profile selections in the icon view
of the Management Bar. Below is an example snippet from the User Admin
portlet:

\begin{verbatim}
<liferay-frontend:user-vertical-card
  actionJsp="/membership_request_action.jsp"
  actionJspServletContext="<%= application %>"
  resultRow="<%= row %>"
  subtitle="<%= membershipRequestUser.getEmailAddress() %>"
  title="<%= HtmlUtil.escape(membershipRequestUser.getFullName()) %>"
  userId="<%= membershipRequest.getUserId() %>"
>
  <liferay-frontend:vertical-card-header>
    <liferay-ui:message 
      arguments="<%= LanguageUtil.getTimeDescription(
      request, 
      System.currentTimeMillis() - membershipRequest.getCreateDate().getTime(), 
      true) %>" 
      key="x-ago" 
      translateArguments="<%= false %>" 
    />
  </liferay-frontend:vertical-card-header>
</liferay-frontend:user-vertical-card>
\end{verbatim}

\begin{figure}
\centering
\includegraphics{./images/liferay-frontend-taglib-cards-user-vertical.png}
\caption{User vertical cards are perfect to display files and
documents.}
\end{figure}

\section{Related Topics}\label{related-topics-6}

\begin{itemize}
\tightlist
\item
  \href{/docs/7-2/reference/-/knowledge_base/r/liferay-front-end-add-menu}{Liferay
  Front-end Add Menu}
\item
  \href{/docs/7-2/reference/-/knowledge_base/r/liferay-front-end-info-bar}{Liferay
  Front-end Info Bar}
\item
  \href{/docs/7-2/reference/-/knowledge_base/r/liferay-front-end-management-bar}{Liferay
  Front-end Management Bar}
\end{itemize}

\chapter{Liferay Front-end Info Bar}\label{liferay-front-end-info-bar}

An info bar provides a button that toggles the visibility of additional
sidebar information. This is perfect for providing more detailed
metadata for a search result, such as the file size, type, URL, etc.

\begin{figure}
\centering
\includegraphics{./images/liferay-frontend-taglib-info-bar-article.png}
\caption{The info bar tags create a sidebar panel toggler that reveals
additional info.}
\end{figure}

The configuration has two key parts: the info bar---and buttons---and
the sidebar panel.

Info bar:

\begin{verbatim}
<liferay-frontend:info-bar>
  <liferay-frontend:info-bar-buttons>
    <liferay-frontend:info-bar-sidenav-toggler-button
      icon="info-circle"
      label="my info"
    />
  </liferay-frontend:info-bar-buttons>
</liferay-frontend:info-bar>
\end{verbatim}

The
\texttt{\textless{}liferay-frontend:info-bar-sidenav-toggler-button\textgreater{}}
tag uses \href{/docs/7-2/reference/-/knowledge_base/r/clay-icons}{Clay
Icons} for the \texttt{icon} attribute.

Sidebar panel:

\begin{verbatim}
<div class="closed container-fluid-1280 sidenav-container sidenav-right" id="<portlet:namespace />infoPanelId">
    <liferay-frontend:sidebar-panel>
      <div>
      <h2>sidebar content</h2>
      <p>Here is some content</p>
      </div>
    </liferay-frontend:sidebar-panel>
</div>
\end{verbatim}

Note that the sidebar panel's wrapper
\texttt{\textless{}div\textgreater{}} has the classes \texttt{closed}
and \texttt{sidenav-right}. The info button toggles the classes
\texttt{open} and \texttt{closed}, showing and hiding the sidebar panel.
The \texttt{sidenav-right} class specifies that the panel should open on
the right.

\begin{figure}
\centering
\includegraphics{./images/liferay-frontend-taglib-info-bar.png}
\caption{The info bar tags create a sidebar panel toggler that reveals
additional info.}
\end{figure}

The examples above use some of the available attributes. See the
\href{https://docs.liferay.com/dxp/apps/foundation/latest/taglibdocs/liferay-frontend/info-bar.html}{info
bar},
\href{https://docs.liferay.com/dxp/apps/foundation/latest/taglibdocs/liferay-frontend/info-bar-buttons.html}{info
bar buttons},
\href{https://docs.liferay.com/dxp/apps/foundation/latest/taglibdocs/liferay-frontend/info-bar-sidenav-toggler-button.html}{info
bar sidenav toggler button}, and
\href{https://docs.liferay.com/dxp/apps/foundation/latest/taglibdocs/liferay-frontend/sidebar-panel.html}{sidebar
panel} taglibdocs for the full list of available attributes for the
tags.

\section{Related Topics}\label{related-topics-7}

\begin{itemize}
\tightlist
\item
  \href{/docs/7-2/reference/-/knowledge_base/r/liferay-front-end-add-menu}{Liferay
  Front-end Add Menu}
\item
  \href{/docs/7-2/reference/-/knowledge_base/r/liferay-front-end-cards}{Liferay
  Front-end Cards}
\item
  \href{/docs/7-2/reference/-/knowledge_base/r/liferay-front-end-management-bar}{Liferay
  Front-end Management Bar}
\end{itemize}

\chapter{Liferay Front-end Management
Bar}\label{liferay-front-end-management-bar}

The Management Bar gives administrators control over search container
results. It lets you filter, sort, and choose a display style for search
results, so you can quickly identify the document, web content, asset
entry, or whatever you're looking for in your app. The Management Bar is
fully customizable, so you can implement all the controls, or just the
ones your app requires.

\begin{figure}
\centering
\includegraphics{./images/liferay-frontend-taglib-management-bar-message-boards.png}
\caption{The Management Bar lets the user customize how the app displays
content.}
\end{figure}

\noindent\hrulefill

\textbf{Note:} The Liferay Front-end Management Bar is deprecated as of
7.0. We recommend that you use the
\href{/docs/7-2/reference/-/knowledge_base/r/clay-management-toolbar}{Clay
Management Toolbar} instead.

\noindent\hrulefill

The Management Bar has a few key sections. Each section is grouped and
configured using different taglibs:

The
\href{https://docs.liferay.com/dxp/apps/foundation/latest/taglibdocs/liferay-frontend/management-bar-buttons.html}{\texttt{\textless{}liferay-frontend:management-bar-buttons\textgreater{}}
tag} wraps the Management Bar's button elements:

\begin{figure}
\centering
\includegraphics{./images/liferay-frontend-taglib-management-bar-buttons.png}
\caption{The \texttt{management-bar-buttons} tag contains the Management
Bar's main buttons.}
\end{figure}

The
\href{https://docs.liferay.com/dxp/apps/foundation/latest/taglibdocs/liferay-frontend/management-bar-sidenav-toggler-button.html}{\texttt{\textless{}liferay-frontend:management-bar-sidenav-toggler-button\textgreater{}}
tag} implements slide-out navigation for the info button.

The
\href{https://docs.liferay.com/dxp/apps/foundation/latest/taglibdocs/liferay-frontend/management-bar-display-buttons.html}{\texttt{\textless{}liferay-frontend:management-bar-display-buttons\textgreater{}}
tag} renders the app's display style options.

\begin{figure}
\centering
\includegraphics{./images/liferay-frontend-taglib-management-bar-display-buttons.png}
\caption{The \texttt{management-bar-display-buttons} tag contains the
content's display options.}
\end{figure}

The
\href{https://docs.liferay.com/dxp/apps/foundation/latest/taglibdocs/liferay-frontend/management-bar-filters.html}{\texttt{\textless{}liferay-frontend:management-bar-filters\textgreater{}}
tag} wraps the app's filtering options. This filter should be included
in all control panel applications. Filtering options can include sort
criteria, sort ordering, and more.

\begin{figure}
\centering
\includegraphics{./images/liferay-frontend-taglib-management-bar-filters.png}
\caption{The \texttt{management-bar-filters} tag contains the content
filtering options.}
\end{figure}

Finally, the
\href{https://docs.liferay.com/dxp/apps/foundation/latest/taglibdocs/liferay-frontend/management-bar-action-buttons.html}{\texttt{\textless{}liferay-frontend:management-bar-action-buttons\textgreater{}}
tag} wraps the actions that you can execute over selected items. You can
select multiple items between pages. The management bar keeps track of
the number of selected items for you.

\begin{figure}
\centering
\includegraphics{./images/liferay-frontend-taglib-management-bar-action-buttons.png}
\caption{The management bar keeps track of the items selected and
displays the actions to execute on them.}
\end{figure}

For example, here's the Management Bar configuration in the Trash app:

\begin{verbatim}
<liferay-frontend:management-bar
   includeCheckBox="<%= true %>"
   searchContainerId="trash"
>
   <liferay-frontend:management-bar-buttons>
       <liferay-frontend:management-bar-sidenav-toggler-button />

       <liferay-portlet:actionURL name="changeDisplayStyle"
       varImpl="changeDisplayStyleURL">
           <portlet:param name="redirect" value="<%= currentURL %>" />
       </liferay-portlet:actionURL>

       <liferay-frontend:management-bar-display-buttons
           displayViews='<%= new String[] {"descriptive", "icon",
           "list"} %>'
           portletURL="<%= changeDisplayStyleURL %>"
           selectedDisplayStyle="<%= trashDisplayContext.getDisplayStyle()
           %>"
       />
   </liferay-frontend:management-bar-buttons>

   <liferay-frontend:management-bar-filters>
       <liferay-frontend:management-bar-navigation
           navigationKeys='<%= new String[] {"all"} %>'
           portletURL="<%= trashDisplayContext.getPortletURL() %>"
       />

       <liferay-frontend:management-bar-sort
           orderByCol="<%= trashDisplayContext.getOrderByCol() %>"
           orderByType="<%= trashDisplayContext.getOrderByType() %>"
           orderColumns='<%= new String[] {"removed-date"} %>'
           portletURL="<%= trashDisplayContext.getPortletURL() %>"
       />
   </liferay-frontend:management-bar-filters>

   <liferay-frontend:management-bar-action-buttons>
       <liferay-frontend:management-bar-sidenav-toggler-button />

       <liferay-frontend:management-bar-button href="javascript:;"
       icon="trash" id="deleteSelectedEntries" label="delete" />
   </liferay-frontend:management-bar-action-buttons>
</liferay-frontend:management-bar>
\end{verbatim}

\chapter{Including Actions in the Management
Bar}\label{including-actions-in-the-management-bar}

While an actions menu is typically included with each search container
result, you can also include these actions in the management bar. This
keeps everything organized within the same UI. This update adds a
checkbox next to each search container result, as well as adds one in
the management bar itself to select all results. The actions are
displayed when a checkbox is checked---individual or select all---and
hidden from view otherwise.

\begin{figure}
\centering
\includegraphics{./images/liferay-frontend-taglib-management-bar-include-checkbox.png}
\caption{You can select individual results or all results at once.}
\end{figure}

Follow these steps to include actions in your management bar:

\begin{enumerate}
\def\labelenumi{\arabic{enumi}.}
\item
  Update the
  \texttt{\textless{}liferay-frontend:management-bar\textgreater{}} tag
  to include the checkbox and provide the search container's ID:

\begin{verbatim}
<liferay-frontend:management-bar
    includeCheckBox="<%= true %>"
    searchContainerId="mySearchContainerId"
>
\end{verbatim}
\item
  After the closing
  \texttt{\textless{}/liferay-frontend:management-bar-filters\textgreater{}}
  tag, add the
  \texttt{\textless{}liferay-frontend:management-bar-action-buttons\textgreater{}}
  tags:

\begin{verbatim}
<liferay-frontend:management-bar-action-buttons>

</liferay-frontend:management-bar-action-buttons>
\end{verbatim}
\item
  Use the available management bar button taglibs
  (e.g.~\texttt{management-bar-button}) to build the action buttons for
  your app's management bar. A code snippet from the Site admin portlet
  is shown below:

\begin{verbatim}
<liferay-frontend:management-bar-action-buttons>
    <liferay-frontend:management-bar-sidenav-toggler-button
        icon="info-circle"
        label="info"
    />

    <liferay-frontend:management-bar-button
        href="javascript:deleteEntries();"
        icon="trash"
        id="deleteSites"
        label="delete"
    />
</liferay-frontend:management-bar-action-buttons>
\end{verbatim}
\end{enumerate}

\begin{figure}
\centering
\includegraphics{./images/liferay-frontend-taglib-management-bar-actions.png}
\caption{You can have as many actions as your app requires.}
\end{figure}

\section{Related Topics}\label{related-topics-8}

\begin{itemize}
\tightlist
\item
  \href{/docs/7-2/reference/-/knowledge_base/r/disabling-all-or-portions-of-the-management-bar}{Disabling
  All or Portions of the Management Bar}
\item
  \href{/docs/7-2/reference/-/knowledge_base/r/clay-management-toolbar}{Clay
  Management Toolbar}
\end{itemize}

\chapter{Disabling All or Portions of the Management
Bar}\label{disabling-all-or-portions-of-the-management-bar}

When there are no search results to display, you should disable all the
Management Bar's buttons, except the sidenav toggler button.

You can disable the Management Bar by adding the \texttt{disabled}
attribute to the \texttt{liferay-frontend:management-bar} tag:

\begin{verbatim}
<liferay-frontend:management-bar
        disabled="<%= total == 0 %>"
        includeCheckBox="<%= true %>"
        searchContainerId="<%= searchContainerId %>"
>
\end{verbatim}

You can also disable individual components by adding the
\texttt{disabled} attribute to the corresponding tag. The example below
disables the display buttons when the search container displays 0
results, since changing the display style has no effect when there
aren't any results to view:

\begin{verbatim}
<liferay-frontend:management-bar-display-buttons
        disabled="<%= total == 0 %>"
        displayViews='<%= new String[] {"descriptive", "icon", "list"} %>'
        portletURL="<%= changeDisplayStyleURL %>"
        selectedDisplayStyle="<%= displayStyle %>"
/>
\end{verbatim}

\begin{figure}
\centering
\includegraphics{./images/liferay-frontend-taglib-management-bar-disabled.png}
\caption{You can disable all or portions of the Management Bar.}
\end{figure}

\section{Related Topics}\label{related-topics-9}

\begin{itemize}
\tightlist
\item
  \href{/docs/7-2/reference/-/knowledge_base/r/including-actions-in-the-management-bar}{Including
  Actions in the Management Bar}
\item
  \href{/docs/7-2/reference/-/knowledge_base/r/clay-management-toolbar}{Clay
  Management Toolbar}
\end{itemize}

\chapter{Using the Liferay Util
Taglib}\label{using-the-liferay-util-taglib}

The Liferay Util taglib is used to pull other resources into a portlet
or theme. You can use it to specify which resources to insert at the
bottom or top of the page's HTML.

To use the Liferay-Util taglib, add the following declaration to your
JSP:

\begin{verbatim}
<%@ taglib prefix="liferay-util" uri="http://liferay.com/tld/util" %>
\end{verbatim}

The Liferay-Util taglib is also available via a macro for your
FreeMarker theme templates and web content templates. Follow this
syntax:

\begin{verbatim}
<@liferay_util["tag-name"] attribute="string value" attribute=10 />
\end{verbatim}

This section covers the available Liferay Util tags you can use in your
app to inject content into portlets and themes.

\chapter{Using Liferay Util Body
Bottom}\label{using-liferay-util-body-bottom}

The body bottom tag is not a self-closing tag. It lets you add
additional HTML or scripts to the bottom of the \texttt{body} tag.
content placed between the opening and closing of this tag is passed to
the
\href{https://github.com/liferay/liferay-portal/blob/7.2.x/portal-web/docroot/html/common/themes/body_bottom.jsp\#L26-L31}{body\_bottom.jsp}
and outputs in this JSP.

This tag also has an optional \texttt{outputKey} attribute. If several
portlets on the page include the same resource with this tag, you can
specify the same \texttt{outputKey} value for each tag so the resource
is only loaded once.

The example configuration below uses the
\texttt{\textless{}liferay-util:body-bottom\textgreater{}} tag to
include JavaScript provided by the portlet's bundle:

\begin{verbatim}
<liferay-util:body-bottom outputKey="bodybottom" >
    <script 
  src="/o/my-liferay-util-portlet/js/my_custom_javascript_body_bottom.js" 
  type="text/javascript"></script>
</liferay-util:body-bottom>
\end{verbatim}

Now you know how to use the
\texttt{\textless{}liferay-util:body-bottom\textgreater{}} tag to
include additional resources in the bottom of the page's body.

\section{Related Topics}\label{related-topics-10}

\begin{itemize}
\tightlist
\item
  \href{/docs/7-2/reference/-/knowledge_base/r/using-liferay-util-body-top}{Using
  the Liferay Util HTML Body Top Tag}
\item
  \href{/docs/7-2/reference/-/knowledge_base/r/using-liferay-util-html-top}{Using
  the Liferay Util HTML Top Tag}
\item
  \href{/docs/7-2/reference/-/knowledge_base/r/using-the-liferay-ui-taglib}{Using
  the Liferay UI Taglib}
\end{itemize}

\chapter{Using Liferay Util Body Top}\label{using-liferay-util-body-top}

The body top tag is not a self-closing tag. The content placed between
the opening and closing of this tag is moved to the top of the
\texttt{body} tag. When something is passed using this taglib, the
\href{https://github.com/liferay/liferay-portal/blob/7.2.x/portal-web/docroot/html/common/themes/body_top.jsp\#L25-L31}{body\_top.jsp}
is passed markup and outputs in this JSP.

This tag also has an optional \texttt{outputKey} attribute. If several
portlets on the page include the same resource with this tag, you can
specify the same \texttt{outputKey} value for each tag so the resource
is only loaded once.

The example configuration below uses the
\texttt{\textless{}liferay-util:body-top\textgreater{}} tag to include
JavaScript provided by the portlet's bundle:

\begin{verbatim}
<liferay-util:body-top outputKey="bodytop" >
    <script 
  src="/o/my-liferay-util-portlet/js/my_custom_javascript_body_top.js" 
  type="text/javascript"></script>
</liferay-util:body-top>
\end{verbatim}

Now you know how to use the
\texttt{\textless{}liferay-util:body-top\textgreater{}} tag to include
additional resources in the top of the page's body.

\section{Related Topics}\label{related-topics-11}

\begin{itemize}
\tightlist
\item
  \href{/docs/7-2/reference/-/knowledge_base/r/using-liferay-util-body-bottom}{Using
  the Liferay Util HTML Body Bottom Tag}
\item
  \href{/docs/7-2/reference/-/knowledge_base/r/using-liferay-util-html-bottom}{Using
  the Liferay Util HTML Bottom Tag}
\item
  \href{/docs/7-2/reference/-/knowledge_base/r/using-the-clay-taglib-in-your-portlets}{Using
  the Clay Taglib}
\end{itemize}

\chapter{Using Liferay Util Buffer}\label{using-liferay-util-buffer}

The buffer tag is not a self-closing tag. The content placed between the
opening and closing of this tag is saved to a buffer and its output is
assigned to the Java variable declared with the tag's \texttt{var}
attribute. The output is returned as a String, letting you post-process
it. For example, you can use this to
\href{/docs/7-2/customization/-/knowledge_base/c/jsp-overrides-using-osgi-fragments\#provide-the-overridden-jsp}{override
a JSP's existing contents}.

The example below saves the link's generated markup to a buffer and then
uses the returned string as the argument for a
\texttt{liferay-ui:message} key:

\begin{verbatim}
<liferay-util:buffer
        var="linkContent"
>
        <aui:a 
            href="https://www.liferay.com/" 
            target="_blank">Liferay
        </aui:a>
</liferay-util:buffer>

<liferay-ui:message 
        arguments="<%= linkContent %>" 
        key="see-x-for-more-information" 
        translateArguments="<%= false %>" 
/>
\end{verbatim}

Now you know how to use the
\texttt{\textless{}liferay-util:buffer\textgreater{}} tag to save
content to a buffer.

\begin{figure}
\centering
\includegraphics{./images/liferay-util-buffer.png}
\caption{You can use the Liferay Util Buffer tag to save pieces of
markup to reuse in your JSP.}
\end{figure}

\section{Related Topics}\label{related-topics-12}

\begin{itemize}
\tightlist
\item
  \href{/docs/7-2/customization/-/knowledge_base/c/jsp-overrides-using-osgi-fragments\#provide-the-overridden-jsp}{JSP
  Overrides Using OSGi Fragments}
\item
  \href{/docs/7-2/reference/-/knowledge_base/r/using-liferay-util-param}{Using
  the Liferay Util Param Tag}
\item
  \href{/docs/7-2/reference/-/knowledge_base/r/using-liferay-front-end-taglibs-in-your-portlet}{Using
  the Liferay Front-End Taglibs}
\end{itemize}

\chapter{Using Liferay Util Dynamic
Include}\label{using-liferay-util-dynamic-include}

The dynamic include tag lets you specify a point or points in a JSP or
theme where a developer can inject additional HTML, resources, or
functionality, using the \texttt{DynamicIncludeRegistry}. You can read
more about the OSGi Service Registry
\href{http://docs.spring.io/osgi/docs/current/reference/html/service-registry.html}{here}.
The \texttt{key} attribute identifies the extension point. See
\href{/docs/7-2/customization/-/knowledge_base/c/dynamic-includes}{Dynamic
Includes} for example configurations that use dynamic include extension
points to inject additional functionality.

The example configuration below uses the
\texttt{\textless{}liferay-util:dynamic-include\textgreater{}} tag to
include an extension point before the primary code and an extension
point after the primary code:

\begin{verbatim}
<%@ taglib uri="http://liferay.com/tld/util" prefix="liferay-util" %>

<liferay-util:dynamic-include key="/path/to/jsp#pre" />

<div>
        <p>And here we have our content</p>
</div>

<liferay-util:dynamic-include key="/path/to/jsp#post" />
\end{verbatim}

Now you know how to use the
\texttt{\textless{}liferay-util:dynamic-include\textgreater{}} tag to
add extension points to your app.

\section{Related Topics}\label{related-topics-13}

\begin{itemize}
\tightlist
\item
  \href{/docs/7-2/customization/-/knowledge_base/c/dynamic-includes}{Dynamic
  Includes}
\item
  \href{/docs/7-2/reference/-/knowledge_base/r/using-liferay-util-body-top}{Using
  the Liferay Util Body Top Tag}
\item
  \href{/docs/7-2/reference/-/knowledge_base/r/using-the-chart-taglib-in-your-portlets}{Using
  the Chart Taglib}
\end{itemize}

\chapter{Using Liferay Util Get URL}\label{using-liferay-util-get-url}

The get URL tag scrapes the URL provided by the \texttt{url} attribute.
If a value is provided for the \texttt{var} attribute, the content from
the screen scrape is scoped to that variable. Otherwise, the scraped
content is displayed where the taglib is used.

A basic configuration for the
\texttt{\textless{}liferay-util:get-url\textgreater{}} tag is shown
below:

\begin{verbatim}
<liferay-util:get-url url="https://www.liferay.com/" />
\end{verbatim}

Here is an example that uses the \texttt{var} attribute:

\begin{verbatim}
<liferay-util:get-url url="https://www.liferay.com/" var="Liferay" />

<div>
                <h2>We borrowed <a href="https://www.liferay.com/">Liferay</a>. Here it is.</h2>

                <div class="Liferay">
                                <%= Liferay %>
                </div>
</div>
\end{verbatim}

\begin{figure}
\centering
\includegraphics{./images/liferay-util-get-url-ldn.png}
\caption{You can use the Liferay Util Get URL tag to scrape URLs.}
\end{figure}

Now you know how to use the
\texttt{\textless{}liferay-util:get-url\textgreater{}} tag to scrape
URLs.

\section{Related Topics}\label{related-topics-14}

\begin{itemize}
\tightlist
\item
  \href{/docs/7-2/reference/-/knowledge_base/r/using-liferay-util-param}{Using
  the Liferay Util Param Tag}
\item
  \href{/docs/7-2/reference/-/knowledge_base/r/using-liferay-util-include}{Using
  the Liferay Util Include Tag}
\item
  \href{/docs/7-2/reference/-/knowledge_base/r/using-aui-taglibs}{Using
  the AUI Taglib}
\end{itemize}

\chapter{Using Liferay Util HTML
Bottom}\label{using-liferay-util-html-bottom}

The HTML bottom tag is not a self-closing tag. Content placed between
the opening and closing of this tag is moved to the bottom of the
\texttt{\textless{}html\textgreater{}} tag. When something is passed
using this taglib, the
\href{https://github.com/liferay/liferay-portal/blob/master/portal-web/docroot/html/common/themes/bottom.jsp\#L53-L59}{bottom.jsp}
is passed markup and outputs in this JSP.

This tag also has an optional \texttt{outputKey} attribute. If several
portlets on the page include the same resource with this tag, you can
specify the same \texttt{outputKey} value for each tag so the resource
is only loaded once.

The example configuration below uses the
\texttt{\textless{}liferay-util:html-bottom\textgreater{}} tag to
include JavaScript (a common use case) provided by the portlet's bundle:

\begin{verbatim}
<liferay-util:html-bottom outputKey="htmlbottom">

    <script src="/o/my-liferay-util-portlet/js/my_custom_javascript.js" 
    type="text/javascript"></script>

</liferay-util:html-bottom>
\end{verbatim}

Now you know how to use the
\texttt{\textless{}liferay-util:html-bottom\textgreater{}} tag to
include additional resources in the bottom of the page's HTML tag.

\section{Related Topics}\label{related-topics-15}

\begin{itemize}
\tightlist
\item
  \href{/docs/7-2/reference/-/knowledge_base/r/using-liferay-util-body-bottom}{Using
  the Liferay Util HTML Body Bottom Tag}
\item
  \href{/docs/7-2/reference/-/knowledge_base/r/using-liferay-util-html-top}{Using
  the Liferay Util HTML Top Tag}
\item
  \href{/docs/7-2/reference/-/knowledge_base/r/using-the-liferay-ui-taglib}{Using
  the Liferay UI Taglib}
\end{itemize}

\chapter{Using Liferay Util HTML Top}\label{using-liferay-util-html-top}

The HTML top tag is not a self-closing tag. The content placed between
the opening and closing of this tag is moved to the
\texttt{\textless{}head\textgreater{}} tag. When something is passed
using this taglib, the
\href{https://github.com/liferay/liferay-portal/blob/master/portal-web/docroot/html/common/themes/top_head.jsp\#L147-L153}{top\_head.jsp}
is passed markup and outputs in this JSP.

This tag also has an optional \texttt{outputKey} attribute. If several
portlets on the page include the same resource with this tag, you can
specify the same \texttt{outputKey} value for each tag so the resource
is only loaded once.

The example configuration below uses the
\texttt{\textless{}liferay-util:html-top\textgreater{}} tag to include
additional CSS styles provided by the portlet's bundle:

\begin{verbatim}
<liferay-util:html-top outputKey="htmltop">
                <link data-senna-track="permanent" 
        href="/o/my-liferay-util-portlet/css/my-custom-styles.css" 
        rel="stylesheet" type="text/css" />
</liferay-util:html-top>
\end{verbatim}

Now you know how to use the
\texttt{\textless{}liferay-util:html-top\textgreater{}} tag to include
additional resources in the top of the page's HTML tag.

\section{Related Topics}\label{related-topics-16}

\begin{itemize}
\tightlist
\item
  \href{/docs/7-2/reference/-/knowledge_base/r/using-liferay-util-html-bottom}{Using
  the Liferay Util HTML Bottom Tag}
\item
  \href{/docs/7-2/reference/-/knowledge_base/r/using-liferay-util-body-top}{Using
  the Liferay Util Body Top Tag}
\item
  \href{/docs/7-2/reference/-/knowledge_base/r/using-the-clay-taglib-in-your-portlets}{Using
  the Clay Taglib}
\end{itemize}

\chapter{Using Liferay Util Include}\label{using-liferay-util-include}

The include tag lets you include other JSP files in your portlet's JSP,
theme, or web content. This can increase readability as well as provide
separation of concerns for JSP files.

The \texttt{page} attribute is required and specifies the path to the
JSP or JSPF to include. The \texttt{servletContext} refers to the
request context that the included JSP should use. Passing
\texttt{\textless{}\%=\ application\ \%\textgreater{}} to this attribute
lets the included JSP use the same \texttt{request} object as other
objects that might be set in the prior JSP.

Below is an example configuration for the
\texttt{\textless{}liferay-util:include\textgreater{}} tag:

\begin{verbatim}
<liferay-util:include 
  page="/relative/path/to/file.jsp" 
  servletContext="<%= application %>"
/>
\end{verbatim}

Now you know how to use the
\texttt{\textless{}liferay-util:include\textgreater{}} tag to include
other JSPs in your portlets, themes, and web content.

\section{Related Topics}\label{related-topics-17}

\begin{itemize}
\tightlist
\item
  \href{/docs/7-2/reference/-/knowledge_base/r/using-liferay-util-param}{Using
  the Liferay Util Param Tag}
\item
  \href{/docs/7-2/reference/-/knowledge_base/r/using-liferay-util-dynamic-include}{Using
  the Liferay Util Dynamic Include Tag}
\item
  \href{/docs/7-2/reference/-/knowledge_base/r/using-liferay-front-end-taglibs-in-your-portlet}{Using
  the Liferay Front-End Taglibs}
\end{itemize}

\chapter{Using Liferay Util Param}\label{using-liferay-util-param}

The param tag lets you set a parameter for an
\href{/docs/7-2/reference/-/knowledge_base/r/using-liferay-util-include}{included
JSP page}. This configuration requires two JSPs. JSP A, the main view of
the app, includes JSP B and sets its parameter value. This lets you
dynamically set content when you include the JSP.

For example, say you have your main functionality in
\texttt{my-app.jsp}, and you have additional functionality provided by
\texttt{more-content.jsp}. You could have the example configuration
shown below:

\texttt{more-content.jsp}:

\begin{verbatim}
<%@ page import="com.liferay.portal.kernel.util.ParamUtil" %>

<%
String answer = ParamUtil.getString(request, "answer");
%>

<div>
  <p>The answer to life, the universe and everything is <%= answer %>.</p>
</div>
\end{verbatim}

Then in \texttt{my-app.jsp}, you can include \texttt{more-content.jsp}
and set the value of the \texttt{answer} parameter:

\begin{verbatim}
<liferay-util:include page="/path/to/more-content.jsp" servletContext="<%= application %>">
  <liferay-util:param name="answer" value="42" />
</liferay-util:include>
\end{verbatim}

This results in the following output in \texttt{my-app.jsp}:

\begin{verbatim}
The answer to life, the universe and everything is 42.
\end{verbatim}

Now you know how to use the
\texttt{\textless{}liferay-util:param\textgreater{}} tag to set
parameters for included JSPs. You can use this approach to include
common reusable pieces of code in your apps.

\section{Related Topics}\label{related-topics-18}

\begin{itemize}
\tightlist
\item
  \href{/docs/7-2/reference/-/knowledge_base/r/using-liferay-util-include}{Using
  the Liferay Util Include Tag}
\item
  \href{/docs/7-2/reference/-/knowledge_base/r/using-liferay-util-body-top}{Using
  the Liferay Util Body Top Tag}
\item
  \href{/docs/7-2/reference/-/knowledge_base/r/using-the-chart-taglib-in-your-portlets}{Using
  the Chart Taglib}
\end{itemize}

\chapter{Using Liferay Util Whitespace
Remover}\label{using-liferay-util-whitespace-remover}

The whitespace remover tag removes line breaks and tabs from code blocks
included between the opening and closing of the tag. Below is an example
configuration for the
\texttt{\textless{}liferay-util:whitespace-remover\textgreater{}} tag:

with remover:

\begin{verbatim}
<liferay-util:whitespace-remover>
    <p>Here is some text with        tabs.</p>
</liferay-util:whitespace-remover>
\end{verbatim}

result:

\begin{verbatim}
Here is some text withtabs.
\end{verbatim}

Now you know how to use the
\texttt{\textless{}liferay-util:whitespace-remover\textgreater{}} tag to
ensure that your code formatting is consistent.

\section{Related Topics}\label{related-topics-19}

\begin{itemize}
\tightlist
\item
  \href{/docs/7-2/reference/-/knowledge_base/r/using-liferay-util-param}{Using
  the Liferay Util Param Tag}
\item
  \href{/docs/7-2/reference/-/knowledge_base/r/using-liferay-util-buffer}{Using
  the Liferay Util Buffer Tag}
\item
  \href{/docs/7-2/reference/-/knowledge_base/r/using-aui-taglibs}{Using
  the AUI Taglib}
\end{itemize}

\chapter{Using the Clay Taglib in Your
portlets}\label{using-the-clay-taglib-in-your-portlets}

The Liferay Clay tag library provides a set of tags for creating
\href{https://clayui.com/}{Clay} UI components in your app.

\noindent\hrulefill

\textbf{Note:} AUI taglibs are deprecated as of 7.0. We recommend that
you use Clay taglibs to avoid future compatibility issues.

\noindent\hrulefill

To use the Clay taglib in your apps, add the following declaration to
your JSP:

\begin{verbatim}
<%@ taglib prefix="clay" uri="http://liferay.com/tld/clay" %>
\end{verbatim}

The Liferay Clay taglib is also available via a macro for your
FreeMarker theme templates and web content templates. Follow this
syntax:

\begin{verbatim}
<@clay["tag-name"] attribute="string value" attribute=10 />
\end{verbatim}

Clay taglibs provide the following UI components for your apps:

\begin{itemize}
\tightlist
\item
  \href{/docs/7-2/reference/-/knowledge_base/r/clay-alerts}{Alerts}
\item
  \href{/docs/7-2/reference/-/knowledge_base/r/clay-badges}{Badges}
\item
  \href{/docs/7-2/reference/-/knowledge_base/r/clay-buttons}{Buttons}
\item
  \href{/docs/7-2/reference/-/knowledge_base/r/clay-cards}{Cards}
\item
  \href{/docs/7-2/reference/-/knowledge_base/r/clay-dropdown-menus-and-action-menus}{Dropdown
  Menus and Action Menus}
\item
  \href{/docs/7-2/reference/-/knowledge_base/r/clay-form-elements}{Form
  Elements}
\item
  \href{/docs/7-2/reference/-/knowledge_base/r/clay-icons}{Icons}
\item
  \href{/docs/7-2/reference/-/knowledge_base/r/clay-labels-and-links}{Labels
  and links}
\item
  \href{/docs/7-2/reference/-/knowledge_base/r/clay-management-toolbar}{Management
  Toolbar}
\item
  \href{/docs/7-2/reference/-/knowledge_base/r/clay-navigation-bars}{Navigation
  Bars}
\item
  \href{/docs/7-2/reference/-/knowledge_base/r/clay-progress-bars}{Progress
  Bars}
\item
  \href{/docs/7-2/reference/-/knowledge_base/r/clay-stickers}{Stickers}
\end{itemize}

This section covers how to create these components with the Clay
taglibs. Each article contains a set of clay component examples along
with a screenshot of the resulting UI.

\chapter{Clay Alerts}\label{clay-alerts}

Clay alerts come in two types: embedded and stripe. Both types, along
with several examples of each, are shown below.

\section{Embedded Alerts}\label{embedded-alerts}

Embedded alerts are usually used inside forms. The element that contains
it determines an embedded alert's width. The close action is not
required for embedded alerts. The following embedded alerts can be
created with Clay taglibs:

Danger alert (embedded):

\begin{verbatim}
<clay:alert
    message="This is an error message."
    style="danger"
    title="Error"
/>
\end{verbatim}

\begin{figure}
\centering
\includegraphics{./images/clay-taglib-alert-danger.png}
\caption{The danger alert notifies the user of an error or issue.}
\end{figure}

Success alert (embedded):

\begin{verbatim}
<clay:alert
    message="This is a success message."
    style="success"
    title="Success"
/>
\end{verbatim}

\begin{figure}
\centering
\includegraphics{./images/clay-taglib-alert-success.png}
\caption{The success alert notifies the user when an action is
successful.}
\end{figure}

Info alert (embedded):

\begin{verbatim}
<clay:alert
    message="This is an info message."
    title="Info"
/>
\end{verbatim}

\begin{figure}
\centering
\includegraphics{./images/clay-taglib-alert-info.png}
\caption{The info alert displays general information to the user.}
\end{figure}

Warning alert (embedded):

\begin{verbatim}
<clay:alert
    message="This is a warning message."
    style="warning"
    title="Warning"
/>
\end{verbatim}

\begin{figure}
\centering
\includegraphics{./images/clay-taglib-alert-warning.png}
\caption{The warning alert displays a warning message to the user.}
\end{figure}

\section{Stripe Alerts}\label{stripe-alerts}

Stripe alerts are placed below the last navigation element (either the
header or the navigation bar), and they usually appear on \emph{Save}
action, communicating the status of the action once received from the
server. Unlike embedded alerts, stripe alerts require the close action.
A stripe alert is always the full width of the container and pushes all
the content below it. The following stripe alerts can be created with
Clay taglibs:

Danger alert (stripe):

\begin{verbatim}
<clay:stripe
    message="This is an error message."
    style="danger"
    title="Error"
/>
\end{verbatim}

\begin{figure}
\centering
\includegraphics{./images/clay-taglib-alert-danger-stripe.png}
\caption{The danger striped alert notifies the user that an action has
failed.}
\end{figure}

Success alert (stripe):

\begin{verbatim}
<clay:stripe
    message="This is a success message."
    style="success"
    title="Success"
/>
\end{verbatim}

\begin{figure}
\centering
\includegraphics{./images/clay-taglib-alert-success-stripe.png}
\caption{The success striped alert notifies the user that an action has
completed successfully.}
\end{figure}

Info alert (stripe):

\begin{verbatim}
<clay:stripe
    message="This is an info message."
    title="Info"
/>
\end{verbatim}

\begin{figure}
\centering
\includegraphics{./images/clay-taglib-alert-info-stripe.png}
\caption{The info striped alert displays general information about an
action to the user.}
\end{figure}

Warning alert (stripe):

\begin{verbatim}
<clay:stripe
    message="This is a warning message."
    style="warning"
    title="Warning"
/>
\end{verbatim}

\begin{figure}
\centering
\includegraphics{./images/clay-taglib-alert-warning-stripe.png}
\caption{The warning striped alert warns the user about an action.}
\end{figure}

Now you know how to alert users!

\section{Related Topics}\label{related-topics-20}

\begin{itemize}
\tightlist
\item
  \href{/docs/7-2/reference/-/knowledge_base/r/clay-buttons}{Clay
  Buttons}
\item
  \href{/docs/7-2/reference/-/knowledge_base/r/clay-form-elements}{Clay
  Form Elements}
\item
  \href{/docs/7-2/reference/-/knowledge_base/r/clay-labels-and-links}{Clay
  Labels and Links}
\end{itemize}

\chapter{Clay Badges}\label{clay-badges}

Badges help highlight important information such as notifications or new
and unread messages. Badges have circular borders and are only used to
specify a number. This covers the different types of Clay badges you can
add to your app.

\section{Badge Types}\label{badge-types}

The following badge styles are available:

Primary badge:

\begin{verbatim}
<div class="col-md-1">
    <clay:badge label="8" />

    <div>Primary</div>
</div>
\end{verbatim}

\begin{figure}
\centering
\includegraphics{./images/clay-taglib-badge-primary.png}
\caption{A primary badge is bright blue, commanding attention like the
primary button of a form.}
\end{figure}

Secondary badge:

\begin{verbatim}
<div class="col-md-1">
    <clay:badge label="87" style="secondary" />

    <div>Secondary</div>
</div>
\end{verbatim}

\begin{figure}
\centering
\includegraphics{./images/clay-taglib-badge-secondary.png}
\caption{A secondary badge is light-grey and draws less focus than a
primary button.}
\end{figure}

Info badge:

\begin{verbatim}
<div class="col-md-1">
    <clay:badge label="91" style="info" />

    <div>Info</div>
</div>
\end{verbatim}

\begin{figure}
\centering
\includegraphics{./images/clay-taglib-badge-info.png}
\caption{A info badge is dark blue and meant for numbers related to
general information.}
\end{figure}

Error badge:

\begin{verbatim}
<div class="col-md-1">
    <clay:badge label="130" style="danger" />

    <div>Error</div>
</div>
\end{verbatim}

\begin{figure}
\centering
\includegraphics{./images/clay-taglib-badge-error.png}
\caption{An error badge displays numbers related to an error.}
\end{figure}

Success badge:

\begin{verbatim}
<div class="col-md-1">
    <clay:badge label="1111" style="success" />

    <div>Success</div>
</div>
\end{verbatim}

\begin{figure}
\centering
\includegraphics{./images/clay-taglib-badge-success.png}
\caption{A success badge displays numbers related to a successful
action.}
\end{figure}

Warning badge:

\begin{verbatim}
<div class="col-md-1">
    <clay:badge label="21" style="warning" />

    <div>Warning</div>
</div>
\end{verbatim}

\begin{figure}
\centering
\includegraphics{./images/clay-taglib-badge-warning.png}
\caption{A warning badge displays numbers related to warnings that
should be addressed.}
\end{figure}

Now you know how to use badges to keep track of values in your app.

\section{Related Topics}\label{related-topics-21}

\begin{itemize}
\tightlist
\item
  \href{/docs/7-2/reference/-/knowledge_base/r/clay-labels-and-links}{Clay
  Labels and Links}
\item
  \href{/docs/7-2/reference/-/knowledge_base/r/clay-cards}{Clay Cards}
\item
  \href{/docs/7-2/reference/-/knowledge_base/r/clay-stickers}{Clay
  Stickers}
\end{itemize}

\chapter{Clay Buttons}\label{clay-buttons}

Buttons come in several types and variations. This tutorial covers the
different styles and variations of buttons you can create with the Clay
taglibs.

\section{Types}\label{types}

\textbf{Primary button:} Used for the most important actions. Two
primary buttons should not be together or near each other.

Primary button with label:

\begin{verbatim}
<clay:button label="Primary" />
\end{verbatim}

\begin{figure}
\centering
\includegraphics{./images/clay-taglib-button-primary.png}
\caption{A primary button is bright blue, grabbing the user's
attention.}
\end{figure}

\textbf{Secondary button:} Used for secondary actions. There can be
multiple secondary buttons together or near each other.

\begin{verbatim}
<div class="col">
    <clay:button label="Secondary" style="secondary" />
</div>
<div class="col">
    <clay:button ariaLabel="Wiki" icon="wiki" style="secondary" />
</div>
\end{verbatim}

\begin{figure}
\centering
\includegraphics{./images/clay-taglib-button-secondary.png}
\caption{A secondary button draws less attention than a primary button
and is meant for secondary actions.}
\end{figure}

\textbf{Borderless button:} Used in cases such as toolbars where the
secondary button would be too heavy for the design. This keeps the
design clean.

\begin{verbatim}
<div class="col">
    <clay:button label="Borderless" style="borderless" />
</div>
<div class="col">
    <clay:button ariaLabel="Page Template" icon="page-template" style="borderless" />
</div>
\end{verbatim}

\begin{figure}
\centering
\includegraphics{./images/clay-taglib-button-borderless.png}
\caption{Borderless buttons remove the dark outline from the button.}
\end{figure}

\textbf{Link button:} Used for Cancel actions.

\begin{verbatim}
<div class="col">
    <clay:button label="Link" style="link" />
</div>
<div class="col">
    <clay:button ariaLabel="Add Role" icon="add-role" style="link" />
</div>
\end{verbatim}

\begin{figure}
\centering
\includegraphics{./images/clay-taglib-button-link.png}
\caption{You can also turn buttons into links.}
\end{figure}

You can use labels or icons for your buttons. Below is an example of a
Primary button with an icon:

\begin{verbatim}
<clay:button ariaLabel="Workflow" icon="workflow" />
\end{verbatim}

\begin{figure}
\centering
\includegraphics{./images/clay-taglib-button-primary-icon.png}
\caption{Buttons can also display icons.}
\end{figure}

You can disable a button by adding the \texttt{disabled} attribute:

\begin{verbatim}
<div class="col">
    <clay:button disabled="<%= true %>" label="Primary" />
</div>
<div class="col">
    <clay:button ariaLabel="Workflow" disabled="<%= true %>" icon="workflow" />
</div>
\end{verbatim}

\begin{figure}
\centering
\includegraphics{./images/clay-taglib-button-primary-disabled.png}
\caption{Buttons can be disabled if you don't want the user to interact
with them.}
\end{figure}

\section{Variations}\label{variations}

Button with icon and text:

\begin{verbatim}
<clay:button icon="share" label="Share" />
\end{verbatim}

\begin{figure}
\centering
\includegraphics{./images/clay-taglib-button-icon-text.png}
\caption{Buttons can display both icons and text.}
\end{figure}

Button with monospaced text:

\begin{verbatim}
<clay:button icon="indent-less" monospaced="<%= true %>" style="secondary" />
\end{verbatim}

\begin{figure}
\centering
\includegraphics{./images/clay-taglib-button-monospaced.png}
\caption{Buttons can display monospaced text.}
\end{figure}

Block level button:

\begin{verbatim}
<clay:button block="<%= true %>" label="Button" />
\end{verbatim}

\begin{figure}
\centering
\includegraphics{./images/clay-taglib-button-block-level.png}
\caption{Block level buttons span the entire width of the container.}
\end{figure}

Plus button:

\begin{verbatim}
<clay:button icon="plus" monospaced="<%= true %>" style="secondary" />
\end{verbatim}

\begin{figure}
\centering
\includegraphics{./images/clay-taglib-button-plus.png}
\caption{: A plus button is used for add actions in an app.}
\end{figure}

Action button:

\begin{verbatim}
<clay:button icon="ellipsis-v" monospaced="<%= true %>" style="borderless" />
\end{verbatim}

\begin{figure}
\centering
\includegraphics{./images/clay-taglib-button-action.png}
\caption{: An action button is used to display actions menus.}
\end{figure}

\section{Related Topics}\label{related-topics-22}

\begin{itemize}
\tightlist
\item
  \href{/docs/7-2/reference/-/knowledge_base/r/clay-alerts}{Clay Alerts}
\item
  \href{/docs/7-2/reference/-/knowledge_base/r/clay-buttons}{Clay
  Buttons}
\item
  \href{/docs/7-2/reference/-/knowledge_base/r/clay-labels-and-links}{Clay
  Labels and Links}
\end{itemize}

\chapter{Clay Cards}\label{clay-cards}

Cards visually represent data. Use them for images, document libraries,
user profiles and more. There are four main types of Cards:

\begin{itemize}
\tightlist
\item
  Image Cards
\item
  File Cards
\item
  User Cards
\item
  Horizontal Cards
\end{itemize}

Each of these types is covered below.

\section{Image Cards}\label{image-cards}

Image Cards are used for image/document galleries.

Image Card:

\begin{verbatim}
<clay:image-card
    actionItems="<%= cardsDisplayContext.getDefaultActionItems() %>"
    href="#1"
    imageAlt="thumbnail"
    imageSrc="https://images.unsplash.com/photo-1506976773555-b3da30a63b57"
    subtitle="Author Action"
    title="Madrid"
/>
\end{verbatim}

\begin{figure}
\centering
\includegraphics{./images/clay-taglib-image-card.png}
\caption{Image Cards display images and documents.}
\end{figure}

Image Card with icon:

\begin{verbatim}
<clay:image-card
    actionItems="<%= cardsDisplayContext.getDefaultActionItems() %>"
    icon="camera"
    subtitle="Author Action"
    title="<%= SVG_FILE_TITLE %>"
/>
\end{verbatim}

\begin{figure}
\centering
\includegraphics{./images/clay-taglib-image-card-icon.png}
\caption{Image Cards can also display icons instead of images.}
\end{figure}

Image Card empty:

\begin{verbatim}
<clay:image-card 
  actionItems="<%= cardsDisplayContext.getDefaultActionItems() %>"
    subtitle="Author Action"
    title="<%= SVG_FILE_TITLE %>"
/>
\end{verbatim}

\begin{figure}
\centering
\includegraphics{./images/clay-taglib-image-card-empty.png}
\caption{Cards can also display nothing.}
\end{figure}

Cards can also contain file types. Specify the file type with the
\texttt{filetype} attribute:

\begin{verbatim}
<clay:image-card
    actionItems="<%= cardsDisplayContext.getDefaultActionItems() %>"
    fileType="JPG"
    fileTypeStyle="danger"
    href="#1"
    imageAlt="thumbnail"
    imageSrc="https://images.unsplash.com/photo-1499310226026-b9d598980b90"
    subtitle="Author Action"
    title="California"
/>
\end{verbatim}

\begin{figure}
\centering
\includegraphics{./images/clay-taglib-image-card-file-type.png}
\caption{Cards can also contain file types.}
\end{figure}

Include the \texttt{labels} attribute to add a label to a Card:

\begin{verbatim}
<clay:image-card
    actionItems="<%= cardsDisplayContext.getDefaultActionItems() %>"
    fileType="JPG"
    fileTypeStyle="danger"
    href="#1"
    imageAlt="thumbnail"
    imageSrc="https://images.unsplash.com/photo-1503703294279-c83bdf7b4bf4"
    labels="<%= cardsDisplayContext.getLabels() %>"
    subtitle="Author Action"
    title="Beetle"
/>
\end{verbatim}

\begin{figure}
\centering
\includegraphics{./images/clay-taglib-image-card-icon-label.png}
\caption{You can include labels in Cards.}
\end{figure}

Include the \texttt{selectable} attribute to make cards selectable
(include a checkbox):

\begin{verbatim}
<clay:image-card
    actionItems="<%= cardsDisplayContext.getDefaultActionItems() %>"
    fileType="JPG"
    fileTypeStyle="danger"
    href="#1"
    imageAlt="thumbnail"
    imageSrc="https://images.unsplash.com/photo-1506020647804-b04ee956dc04"
    labels="<%= cardsDisplayContext.getLabels() %>"
    selectable="<%= true %>"
    selected="<%= true %>"
    subtitle="Author Action"
    title="Beetle"
/>
\end{verbatim}

\begin{figure}
\centering
\includegraphics{./images/clay-taglib-image-card-icon-selectable.png}
\caption{Cards can be selectable.}
\end{figure}

\section{File Cards}\label{file-cards}

File Cards display an icon of the file's type. They represent file types
other than image files (i.e.~PDF, MP3, DOC, etc.).

\begin{verbatim}
<clay:file-card
    actionItems="<%= cardsDisplayContext.getDefaultActionItems() %>"
    fileType="MP3"
    fileTypeStyle="warning"
    labels="<%= cardsDisplayContext.getLabels() %>"
    labelStylesMap="<%= cardsDisplayContext.getLabelStylesMap() %>"
    selectable="<%= true %>"
    selected="<%= true %>"
    subtitle="Jimi Hendrix"
    title="<%= MP3_FILE_TITLE %>"
/>
\end{verbatim}

\begin{figure}
\centering
\includegraphics{./images/clay-taglib-file-card.png}
\caption{File Cards display file type icons.}
\end{figure}

You can optionally use the \texttt{labelStylesMap} attribute to pass a
\texttt{HashMap} of multiple labels, as shown above.

The example below specifies a list \texttt{icon} instead of the default
file icon:

\begin{verbatim}
<clay:file-card
    actionItems="<%= cardsDisplayContext.getDefaultActionItems() %>"
    fileType="DOC"
    fileTypeStyle="info"
    icon="list"
    labels="<%= cardsDisplayContext.getLabels() %>"
    selectable="<%= true %>"
    selected="<%= true %>"
    subtitle="Paco de Lucia"
    title="<%= DOC_FILE_TITLE %>"
/>
\end{verbatim}

\noindent\hrulefill

\textbf{Note:} The full list of available Liferay icons can be found on
the \href{https://clayui.com/docs/components/icon.html}{Clay CSS
website}.

\noindent\hrulefill

\section{User Cards}\label{user-cards}

User Cards display user profile images or the initials of the user's
name or name+surname.

User Card with initials:

\begin{verbatim}
<clay:user-card
    actionItems="<%= cardsDisplayContext.getDefaultActionItems() %>"
    initials="HS"
    name="User Name"
    subtitle="Latest Action"
    userColor="danger"
/>
\end{verbatim}

\begin{figure}
\centering
\includegraphics{./images/clay-taglib-user-card-initial.png}
\caption{User Cards can display a user's initials.}
\end{figure}

User Card with profile image:

\begin{verbatim}
<clay:user-card
    actionItems="<%= cardsDisplayContext.getDefaultActionItems() %>"
    disabled="<%= true %>"
    imageAlt="thumbnail"
    imageSrc="https://images.unsplash.com/photo-1502290822284-9538ef1f1291"
    name="User name"
    selectable="<%= true %>"
    selected="<%= true %>"
    subtitle="Latest Action"
/>
\end{verbatim}

\begin{figure}
\centering
\includegraphics{./images/clay-taglib-user-card-profile-image.png}
\caption{A User Card can also display a profile image.}
\end{figure}

\section{Horizontal Cards}\label{horizontal-cards}

Horizontal Cards represent folders and can have the same amount of
information as other Cards. The key difference is that horizontal Cards
let you remove the image portion of the Card, since only the folder icon
is required.

\begin{verbatim}
<clay:horizontal-card
    actionItems="<%= cardsDisplayContext.getDefaultActionItems() %>"
    selectable="<%= true %>"
    selected="<%= true %>"
    title="ReallySuperInsanelyJustIncrediblyLongAndTotallyNotPossibleWordButWeAreReallyTryingToCoverAllOurBasesHereJustInCaseSomeoneIsNutsAsPerUsual"
/>
\end{verbatim}

\begin{figure}
\centering
\includegraphics{./images/clay-taglib-horizontal-card.png}
\caption{: Horizontal Cards are good for displaying folders.}
\end{figure}

Now you know how to use Cards in your UI to display information in your
apps.

\section{Related Topics}\label{related-topics-23}

\begin{itemize}
\tightlist
\item
  \href{/docs/7-2/reference/-/knowledge_base/r/clay-badges}{Clay Badges}
\item
  \href{/docs/7-2/reference/-/knowledge_base/r/clay-labels-and-links}{Clay
  Labels and Links}
\item
  \href{/docs/7-2/reference/-/knowledge_base/r/clay-stickers}{Clay
  Stickers}
\end{itemize}

\chapter{Clay Dropdown Menus and Action
Menus}\label{clay-dropdown-menus-and-action-menus}

You can add dropdown menus to your app via the
\texttt{clay:dropdown-menu} and \texttt{clay:actions-menu} taglibs. The
Clay taglibs provide several menu variations for you to choose. Both
taglibs with several examples are shown below.

\section{Dropdown Menus}\label{dropdown-menus}

Basic dropdown menu:

\begin{verbatim}
<clay:dropdown-menu
    items="<%= dropdownsDisplayContext.getDefaultDropdownItems() %>"
    label="Default"
/>
\end{verbatim}

\begin{figure}
\centering
\includegraphics{./images/clay-taglib-dropdown-basic.png}
\caption{Clay taglibs provide everything you need to add dropdown menus
to your app.}
\end{figure}

The dropdown menu's items are defined in its Java
class--\texttt{dropdownDisplayContext} in this case. Menu items are
\texttt{NavigationItem} objects. You can disable menu items with the
\texttt{setDisabled(true)} method and make a menu item active with the
\texttt{setActive(true)} method. The \texttt{href} attribute is set with
the \texttt{setHref()} method, and labels are defined with the
\texttt{setLabel()} method. Here's an example implementation of the
\texttt{dropdownDisplayContext} class:

\begin{verbatim}
if (_defaultDropdownItems != null) {
  return _defaultDropdownItems;
}

_defaultDropdownItems = new ArrayList<>();

for (int i = 0; i < 4; i++) {
  NavigationItem navigationItem = new NavigationItem();

  if (i == 1) {
    navigationItem.setDisabled(true);
  }
  else if (i == 2) {
    navigationItem.setActive(true);
  }

  navigationItem.setHref("#" + i);
  navigationItem.setLabel("Option " + i);

  _defaultDropdownItems.add(navigationItem);
}

return _defaultDropdownItems;
}
\end{verbatim}

You can organize menu items into groups by setting the
\texttt{NavigationItem}'s type to \texttt{TYPE\_GROUP} and nesting the
items in separate \texttt{ArrayList}s. You can add a horizontal
separator to separate the groups visually with the
\texttt{setSeparator(true)} method. Below is a code snippet from the
\texttt{dropdownsDisplayContext} class:

\begin{verbatim}
group1NavigationItem.setSeparator(true);
group1NavigationItem.setType(NavigationItem.TYPE_GROUP);
\end{verbatim}

Corresponding taglib:

\begin{verbatim}
<clay:dropdown-menu
    items="<%= dropdownsDisplayContext.getGroupDropdownItems() %>"
    label="Dividers"
/>
\end{verbatim}

\begin{figure}
\centering
\includegraphics{./images/clay-taglib-dropdown-group.png}
\caption{You can organize dropdown menu items into groups.}
\end{figure}

You can also add inputs to dropdown menus. To add an input to a dropdown
menu, set the input's type with the \texttt{setType()} method
(e.g.~\texttt{NavigationItem.TYPE\_CHECKBOX}), its name with the
\texttt{setInputName()} method, and its value with the
\texttt{setInputValue()} method. Here's an example implementation:

\begin{verbatim}
navigationItem.setInputName("checkbox" + i);
navigationItem.setInputValue("checkboxValue" + i);
navigationItem.setLabel("Group 1 - Option " + i);
navigationItem.setType(NavigationItem.TYPE_CHECKBOX);
\end{verbatim}

Corresponding taglib:

\begin{verbatim}
<clay:dropdown-menu
    buttonLabel="Done"
    items="<%= dropdownsDisplayContext.getInputDropdownItems() %>"
    label="Inputs"
    searchable="<%= true %>"
/>
\end{verbatim}

\begin{figure}
\centering
\includegraphics{./images/clay-taglib-dropdown-input.png}
\caption{Inputs can be included in dropdown menus.}
\end{figure}

Menu items can also contain icons. To add an icon to a menu item, use
the \texttt{setIcon()} method. Below is an example:

\begin{verbatim}
navigationItem.setIcon("check-circle-full")
\end{verbatim}

Corresponding taglib:

\begin{verbatim}
<clay:dropdown-menu
    items="<%= dropdownsDisplayContext.getIconDropdownItems() %>"
    itemsIconAlignment="left"
    label="Icons"
/>
\end{verbatim}

\begin{figure}
\centering
\includegraphics{./images/clay-taglib-dropdown-icons.png}
\caption{Icons can be included in dropdown menus.}
\end{figure}

\section{Actions Menus}\label{actions-menus}

Basic actions menu:

\begin{verbatim}
<clay:dropdown-actions
    items="<%= dropdownsDisplayContext.getDefaultDropdownItems() %>"
/>
\end{verbatim}

\begin{figure}
\centering
\includegraphics{./images/clay-taglib-dropdown-actions.png}
\caption{You can also create Actions menus with Clay taglibs.}
\end{figure}

An actions menu can also display help text to the user:

\begin{verbatim}
<clay:dropdown-actions
    buttonLabel="More"
    buttonStyle="secondary"
    caption="Showing 4 of 32 Options"
    helpText="You can customize this menu or see all you have by pressing \"more\"."
    items="<%= dropdownsDisplayContext.getDefaultDropdownItems() %>"
/>
\end{verbatim}

\begin{figure}
\centering
\includegraphics{./images/clay-taglib-dropdown-actions-help.png}
\caption{You can provide help text in Actions menus.}
\end{figure}

Clay taglibs make it easy to add dropdown menus and action menus to your
apps.

\section{Related Topics}\label{related-topics-24}

\begin{itemize}
\tightlist
\item
  \href{/docs/7-2/reference/-/knowledge_base/r/clay-form-elements}{Clay
  Form Elements}
\item
  \href{/docs/7-2/reference/-/knowledge_base/r/clay-navigation-bars}{Clay
  Navigation Bars}
\item
  \href{/docs/7-2/reference/-/knowledge_base/r/clay-progress-bars}{Clay
  Progress Bars}
\end{itemize}

\chapter{Clay Form Elements}\label{clay-form-elements}

The Liferay Clay tag library provides several tags for creating form
elements. An example of each tag is shown below.

\section{Checkbox}\label{checkbox}

Checkboxes give the user a true or false input.

\begin{verbatim}
<clay:checkbox 
        checked="<%= true %>" 
        hideLabel="<%= true %>" 
        label="My Input" 
        name="name" 
/>
\end{verbatim}

Attributes:

\textbf{checked:} Whether the checkbox is checked

\textbf{disabled:} Whether the checkbox is enabled

\textbf{hideLabel:} Whether to display the checkbox label

\textbf{indeterminate:} Checkbox variable for multiple selection

\textbf{label:} The checkbox's label

\textbf{name:} The checkbox's name

\begin{figure}
\centering
\includegraphics{./images/clay-taglib-form-checkbox.png}
\caption{Clay taglibs provide checkboxes.}
\end{figure}

\section{Radio}\label{radio}

A radio button lets the user select one choice from a set of options in
a form.

\begin{verbatim}
<clay:radio 
        checked="<%= true %>" 
        hideLabel="<%= true %>" 
        label="My Input" 
        name="name" 
/>
\end{verbatim}

Attributes:

\textbf{checked:} Whether the radio button is checked

\textbf{hideLabel:} Whether to display the radio button label

\textbf{disabled:} Whether the radio button is enabled

\textbf{label:} The radio button's label

\textbf{name:} The radio button's name

\begin{figure}
\centering
\includegraphics{./images/clay-taglib-form-radio-button.png}
\caption{Clay taglibs provide radio buttons.}
\end{figure}

\section{Selector}\label{selector}

A selector gives the user a select box with a set of options to choose
from.

The Java scriplet below creates eight dummy options for the selector:

\begin{verbatim}
<%
List<Map<String, Object>> options = new ArrayList<>();

for (int i = 0; i < 8; i++) {
    Map<String, Object> option = new HashMap<>();

    option.put("label", "Sample " + i);
    option.put("value", i);

    options.add(option);
}
%>
\end{verbatim}

\begin{verbatim}
<clay:select 
        label="Regular Select Element" 
        name="name" 
        options="<%= options %>" 
/>
\end{verbatim}

\begin{figure}
\centering
\includegraphics{./images/clay-taglib-form-selector.png}
\caption{Clay taglibs provide select boxes.}
\end{figure}

If you want let users select multiple options at once, set the
\texttt{multiple} attribute to \texttt{true}:

\begin{verbatim}
<clay:select 
        label="Multiple Select Element" 
        multiple="<%= true %>" 
        name="name" 
        options="<%= options %>" 
/>
\end{verbatim}

\begin{figure}
\centering
\includegraphics{./images/clay-taglib-form-selector-multiple.png}
\caption{You can let users select multiple options from the select
menu.}
\end{figure}

Attributes:

\textbf{disabled:} Whether the selector is enabled \textbf{label:} The
selector's label \textbf{multiple:} Whether multiple options can be
selected \textbf{name:} The selector's name

Now you know how to use Clay taglibs to add common form elements to your
app!

\section{Related Topics}\label{related-topics-25}

\begin{itemize}
\tightlist
\item
  \href{/docs/7-2/reference/-/knowledge_base/r/clay-buttons}{Clay
  Buttons}
\item
  \href{/docs/7-2/reference/-/knowledge_base/r/clay-icons}{Clay Icons}
\item
  \href{/docs/7-2/reference/-/knowledge_base/r/clay-labels-and-links}{Clay
  Labels and Links}
\end{itemize}

\chapter{Clay Icons}\label{clay-icons}

The Liferay Clay taglibs provide several icons that you can use in your
apps. Use the \texttt{clay:icon} tag and specify the icon with the
\texttt{symbol} attribute:

\begin{verbatim}
<clay:icon symbol="folder" />
\end{verbatim}

\begin{figure}
\centering
\includegraphics{./images/clay-taglib-icon-folder.png}
\caption{You can include icons in your app with the Clay taglib.}
\end{figure}

The full list of icons is shown below:

\begin{figure}
\centering
\includegraphics{./images/clay-taglib-icon-library.png}
\caption{The Clay taglib gives you access to several Liferay DXP icons.}
\end{figure}

The Liferay Clay taglibs also provide a set of language flag icons that
you can use in your app. The full list of language flags is shown below:

\begin{figure}
\centering
\includegraphics{./images/clay-taglib-icon-language-flags.png}
\caption{You can include language flags in your apps.}
\end{figure}

\section{Related Topics}\label{related-topics-26}

\begin{itemize}
\tightlist
\item
  \href{/docs/7-2/reference/-/knowledge_base/r/clay-badges}{Clay Badges}
\item
  \href{/docs/7-2/reference/-/knowledge_base/r/clay-stickers}{Clay
  Stickers}
\item
  \href{/docs/7-2/frameworks/-/knowledge_base/f/using-clay-icons-in-a-theme}{Using
  Clay Icons in a Theme}
\end{itemize}

\chapter{Clay Labels and Links}\label{clay-labels-and-links}

Liferay Clay taglibs provide tags for creating labels and links in your
app. Both of these elements are covered below.

\section{Labels}\label{labels}

The Liferay Clay taglibs provide a few different labels for your app.
Use the \texttt{clay:label} tag to add a label to your app. You can
create color-coded labels, removable labels, and labels that contain
links. The sections below demonstrate all of these options.

\section{Color-coded Labels}\label{color-coded-labels}

The Liferay Clay labels come in four different colors: dark-blue for
info, light-gray for status, orange for pending, red for rejected, and
green for approved.

Info labels are dark-blue, and since they stand out a bit more than
status labels, they are best for conveying general information. To use
an info label, set the \texttt{style} attribute to \texttt{info}:

\begin{verbatim}
<clay:label label="Label text" style="info" />
\end{verbatim}

\begin{figure}
\centering
\includegraphics{./images/clay-taglib-label-info.png}
\caption{Info labels convey general information.}
\end{figure}

Status labels are light-gray, and due to their neutral color, they are
best for conveying basic information. Status labels are the default
label and therefore require no \texttt{style} attribute:

\begin{verbatim}
<clay:label label="Status" />
\end{verbatim}

\begin{figure}
\centering
\includegraphics{./images/clay-taglib-label-status.png}
\caption{Status labels are the least flashy and best for displaying
basic information.}
\end{figure}

Warning labels are orange, and due to their color, they are best for
conveying a warning message. To use a warning label, set the
\texttt{style} attribute to \texttt{warning}:

\begin{verbatim}
<clay:label label="Pending" style="warning" />
\end{verbatim}

\begin{figure}
\centering
\includegraphics{./images/clay-taglib-label-warning.png}
\caption{Warning labels notify the user of issues, but nothing app
breaking.}
\end{figure}

Danger labels are red and indicate that something is wrong or has
failed. To use a danger label, set the \texttt{style} attribute to
\texttt{danger}:

\begin{verbatim}
<clay:label label="Rejected" style="danger" />
\end{verbatim}

\begin{figure}
\centering
\includegraphics{./images/clay-taglib-label-danger.png}
\caption{Danger labels convey a sense of urgency that must be
addressed.}
\end{figure}

Success labels are green and indicate that something has completed
successfully. To use a success label, set the \texttt{style} attribute
to \texttt{success}:

\begin{verbatim}
<clay:label label="Approved" style="success" />
\end{verbatim}

\begin{figure}
\centering
\includegraphics{./images/clay-taglib-label-success.png}
\caption{Success labels indicate a successful action.}
\end{figure}

Labels can also be bigger. Set the \texttt{size} attribute to
\texttt{lg} to display large labels:

\begin{verbatim}
<clay:label label="Approved" size="lg" style="success" />
\end{verbatim}

\section{Removable Labels}\label{removable-labels}

If you want to let a user close a label (e.g.~a temporary notification),
you can make the label removable by setting the \texttt{closeable}
attribute to \texttt{true}.

\begin{verbatim}
<clay:label closeable="<%= true %>" label="Normal Label" />
\end{verbatim}

\begin{figure}
\centering
\includegraphics{./images/clay-taglib-label-removable.png}
\caption{Labels can be removable.}
\end{figure}

\section{Labels with Links}\label{labels-with-links}

You can make a label a link by adding the \texttt{href} attribute to it
just as you would an anchor tag:

\begin{verbatim}
<clay:label href="#" label="Label Text" />
\end{verbatim}

\begin{figure}
\centering
\includegraphics{./images/clay-taglib-label-link.png}
\caption{Labels can also be links.}
\end{figure}

\section{Links}\label{links}

You can add traditional hyperlinks to your app with the
\texttt{\textless{}clay:link\textgreater{}} tag:

\begin{verbatim}
<clay:link href="#" label="link text" />
\end{verbatim}

\begin{figure}
\centering
\includegraphics{./images/clay-taglib-link.png}
\caption{Clay taglibs also provide link elements.}
\end{figure}

Now you know how to add links and labels to your apps!

\section{Related Topics}\label{related-topics-27}

\begin{itemize}
\tightlist
\item
  \href{/docs/7-2/reference/-/knowledge_base/r/clay-badges}{Clay Badges}
\item
  \href{/docs/7-2/reference/-/knowledge_base/r/clay-cards}{Clay Cards}
\item
  \href{/docs/7-2/reference/-/knowledge_base/r/clay-form-elements}{Clay
  Form Elements}
\end{itemize}

\chapter{Clay Management Toolbar}\label{clay-management-toolbar}

The Management Toolbar gives administrators control over search
container results in their apps. It lets you filter, sort, and choose a
view type for search results, so you can quickly identify the document,
web content, asset entry, or whatever you're looking for. The Management
Toolbar is fully customizable, so you can implement all the controls or
just the ones your app requires.

\begin{figure}
\centering
\includegraphics{./images/clay-taglib-management-toolbar.png}
\caption{The Management ToolBar lets the user customize how the app
displays content.}
\end{figure}

To create a management toolbar, use the \texttt{clay:management-toolbar}
taglib. The toolbar contains a few key sections. Each section is grouped
and configured using different attributes. These attributes are
described in more detail below.

\section{Using a Display Context to Configure the Management
Toolbar}\label{using-a-display-context-to-configure-the-management-toolbar}

If you're using a Display Context---a separate class to configure your
display options for your management toolbar---to define all or some of
the configuration options for the toolbar, you can specify the Display
Context with the \texttt{displayContext} attribute. An example is shown
below:

\begin{verbatim}
<clay:management-toolbar 
    displayContext="<%= viewUADEntitiesManagementToolbarDisplayContext %>" 
/>
\end{verbatim}

You can see an example use case of a Display Context in
\href{/docs/7-2/frameworks/-/knowledge_base/f/filtering-and-sorting-items-with-the-management-toolbar}{Filtering
and Sorting Items with the Management Toolbar}. A Display Context is not
required for a management toolbar's configuration. You can provide as
much or as little of the configuration options for your management
toolbar through the Display Context as you like.

\section{Checkbox and Actions}\label{checkbox-and-actions}

The \texttt{actionItems}, \texttt{searchContainerId},
\texttt{selectable}, and \texttt{totalItems} attributes let you include
a checkbox in the toolbar to select all search container results and run
bulk actions on them. Actions and total items display when an individual
result is checked, or when the master checkbox is checked in the
toolbar.

\texttt{actionItems}: The list of dropdown items to display when a
result is checked or the master checkbox in the Management Toolbar is
checked. You can select multiple results between pages. The Management
Toolbar keeps track of the number of selected results for you.

\texttt{searchContainerId}: The ID of the search container connected to
the Management Toolbar

\texttt{selectable}: Whether to include a checkbox in the Management
Toolbar

\texttt{totalItems}: The total number of items across pagination. This
number displays when one or multiple items are selected.

An example configuration is shown below:

\begin{verbatim}
actionItems="<%=
    new JSPDropdownItemList(pageContext) {
        {
          add(
            dropdownItem -> {
              dropdownItem.setHref("#edit");
              dropdownItem.setLabel("Edit");
            });
  
          add(
            dropdownItem -> {
              dropdownItem.setHref("#download");
              dropdownItem.setIcon("download");
              dropdownItem.setLabel("Download");
              dropdownItem.setQuickAction(true);
            });
  
          add(
            dropdownItem -> {
              dropdownItem.setHref("#delete");
              dropdownItem.setLabel("Delete");
              dropdownItem.setIcon("trash");
              dropdownItem.setQuickAction(true);
            });
        }
    }
%>"
\end{verbatim}

Action items are listed in the Actions menu, along with the number of
items selected across pagination.

\begin{figure}
\centering
\includegraphics{./images/clay-taglib-management-toolbar-actions.png}
\caption{Actions are also listed in the Management Toolbar's dropdown
menu when an item, multiple items, or the master checkbox is checked.}
\end{figure}

If an action has an icon specified, such as the Delete and Download
actions in the example above, the icon is displayed next to the action
menu as well.

\begin{figure}
\centering
\includegraphics{./images/clay-taglib-management-toolbar-selectable.png}
\caption{The Management Toolbar keeps track of the results selected and
displays the actions to execute on them.}
\end{figure}

\section{Filtering and Sorting Search
Results}\label{filtering-and-sorting-search-results}

The \texttt{filterItems}, \texttt{sortingOrder}, and \texttt{sortingURL}
attributes let you filter and sort search container results. Filtering
and sorting are grouped together in one convenient dropdown menu.

\texttt{filterItems}: Sets the search container's filtering options.
This filter should be included in all control panel applications.
Filtering options can include sort criteria, sort ordering, and more.

\texttt{filterLabelItems}: Sets the search container's filter labels to
display. This lets the user know which filters are currently applied.

\texttt{sortingOrder}: The current sorting order: ascending or
descending.

\texttt{sortingURL}: The URL to change the sorting order

The example below adds two filter options and two sorting options:

\begin{verbatim}
filterItems="<%=
    new DropdownItemList(_request) {
        {
            addGroup(
                dropdownGroupItem -> {
                    dropdownGroupItem.setDropdownItemList(
                        new DropdownItemList(_request) {
                            {
                                add(
                                    dropdownItem -> {
                                        dropdownItem.setHref("#1");
                                        dropdownItem.setLabel("Filter 1");
                                    });

                                add(
                                    dropdownItem -> {
                                        dropdownItem.setHref("#2");
                                        dropdownItem.setLabel("Filter 2");
                                    });
                            }
                        }
                    );
                    dropdownGroupItem.setLabel("Filter By");
                });
                
            addGroup(
                dropdownGroupItem -> {
                    dropdownGroupItem.setDropdownItemList(
                        new DropdownItemList(_request) {
                            {
                                add(
                                    dropdownItem -> {
                                        dropdownItem.setHref("#3");
                                        dropdownItem.setLabel("Order 1");
                                    });

                                add(
                                    dropdownItem -> {
                                        dropdownItem.setHref("#4");
                                        dropdownItem.setLabel("Order 2");
                                    });
                            }
                        }
                    );
                    dropdownGroupItem.setLabel("Order By");
                });
        }
    }
%>"
\end{verbatim}

\begin{verbatim}
filterLabelItems="<%=
  new LabelItemList() {
    {
      add(
        labelItem -> {
          labelItem.setLabel("Filter 1");
        });

      add(
        labelItem -> {
          labelItem.setLabel("Filter 2");
        });
    }
  };
%>"
\end{verbatim}

\begin{figure}
\centering
\includegraphics{./images/clay-taglib-management-toolbar-filter-and-sort.png}
\caption{You can also sort and filter search container results.}
\end{figure}

\begin{figure}
\centering
\includegraphics{./images/clay-taglib-management-toolbar-filter-label-items.jpg}
\caption{You can also sort and filter search container results.}
\end{figure}

\section{Search Form}\label{search-form}

The \texttt{clearResultsURL}, \texttt{searchActionURL},
\texttt{searchFormName}, \texttt{searchInputName}, and
\texttt{searchValue} attributes let you configure the search form. The
main portion of the Management Toolbar is reserved for the search form.

\texttt{clearResultsURL}: The URL to reset the search

\texttt{searchActionURL}: The action URL to send the search form

\texttt{searchFormName}: The search form's name

\texttt{searchInputName}: The search input's name

\texttt{searchValue}: The search input's value

An example configuration is shown below:

\begin{verbatim}
<clay:management-toolbar
    clearResultsURL="<%= searchURL %>"
    disabled="<%= isDisabled %>"
    namespace="<%= renderResponse.getNamespace() %>"
    searchActionURL="<%= searchURL %>"
    searchFormName="fm"
    searchInputName="<%= DisplayTerms.KEYWORDS %>"
    searchValue="<%= ParamUtil.getString(request, searchInputName) %>"
    selectable="<%= false %>"
    totalItems="<%= totalItems %>"
/>
\end{verbatim}

\begin{figure}
\centering
\includegraphics{./images/clay-taglib-management-toolbar-search-form.png}
\caption{The search form comprises most of the Management Toolbar,
letting users search through the search container results.}
\end{figure}

\section{Info Panel}\label{info-panel}

The \texttt{infoPanelId} and \texttt{showInfoButton} attributes let you
add a retractable sidebar panel that displays additional information
related to a search container result.

\texttt{infoPanelId}: The ID of the info panel to toggle

\texttt{showInfoButton}: Whether to show the info button

In the example configuration below, the \texttt{showInfoButton}
attribute is provided in the Display Context---specified with the
\texttt{displayContext} attribute---and the \texttt{infoPanelId} is
explicitly set in the JSP:

\begin{verbatim}
<clay:management-toolbar
    displayContext="<%= journalDisplayContext %>"
    infoPanelId="infoPanelId"
    namespace="<%= renderResponse.getNamespace() %>"
    searchContainerId="<=% searchContainerId %>"
/>
\end{verbatim}

\begin{figure}
\centering
\includegraphics{./images/clay-taglib-management-toolbar-info-panel.png}
\caption{The info panel keeps your UI clutter-free.}
\end{figure}

\section{View Types}\label{view-types}

The \texttt{viewTypes} attribute specifies the display options for the
search container results. There are three display options to choose
from:

\textbf{Cards:} Displays search result columns on a horizontal or
vertical card.

\begin{figure}
\centering
\includegraphics{./images/clay-taglib-management-toolbar-view-type-card.png}
\caption{The Management Toolbar's icon display view gives a quick
summary of the content's description and status.}
\end{figure}

\textbf{List:} Displays a detailed description along with summarized
details for the search result columns.

\begin{figure}
\centering
\includegraphics{./images/clay-taglib-management-toolbar-view-type-list.png}
\caption{The Management Toolbar's List view type gives the content's
full description.}
\end{figure}

\textbf{Table:} The default view. Lists the search result columns from
left to right.

\begin{figure}
\centering
\includegraphics{./images/clay-taglib-management-toolbar-view-type-table.png}
\caption{: The Management Toolbar's Table view type list the content's
information in individual columns.}
\end{figure}

An example configuration is shown below:

\begin{verbatim}
viewTypes="<%=
    new JSPViewTypeItemList(pageContext, baseURL, selectedType) {
        {
            addCardViewTypeItem(
                viewTypeItem -> {
                    viewTypeItem.setActive(true);
                    viewTypeItem.setLabel("Card");
                });

            addListViewTypeItem(
                viewTypeItem -> {
                    viewTypeItem.setLabel("List");
                });

            addTableViewTypeItem(
                viewTypeItem -> {
                    viewTypeItem.setLabel("Table");
                });
        }
    }
%>"
\end{verbatim}

While the example above shows how to configure the view types in the
JSP, you must also
\href{/docs/7-2/frameworks/-/knowledge_base/f/implementing-the-view-types}{specify
when to use each view type}.

\begin{figure}
\centering
\includegraphics{./images/clay-taglib-management-toolbar-view-types.png}
\caption{: The Management Toolbar offers three view type options.}
\end{figure}

\section{Creation Menu}\label{creation-menu}

The \texttt{creationMenu} attribute creates an add menu button for one
or multiple items. It's used for creating new entities (e.g.~a new blog
entry).

Use the \texttt{addPrimaryDropdownItem()} method to add the top level
items to the dropdown menu, or use the
\texttt{addFavoriteDropdownItem()} method to add secondary items to the
dropdown menu.

The example configuration below adds two primary creation menu items and
two secondary creation menu items:

\begin{verbatim}
creationMenu="<%= 
    new JSPCreationMenu(pageContext) {
            {
                addPrimaryDropdownItem(
                    dropdownItem -> {
                        dropdownItem.setHref("#1");
                        dropdownItem.setLabel("Sample 1");
                    });
  
                addPrimaryDropdownItem(
                    dropdownItem -> {
                        dropdownItem.setHref("#2");
                        dropdownItem.setLabel("Sample 2");
                    });
  
                addFavoriteDropdownItem(
                    dropdownItem -> {
                        dropdownItem.setHref("#3");
                        dropdownItem.setLabel("Favorite 1");
                    });
  
                addFavoriteDropdownItem(
                    dropdownItem -> {
                        dropdownItem.setHref("#4");
                        dropdownItem.setLabel("Other item");
                    });
            }
    };
%>"
\end{verbatim}

\begin{figure}
\centering
\includegraphics{./images/clay-taglib-management-toolbar-creation-menu.png}
\caption{: The Management Toolbar lets you optionally add a Creation
Menu for creating new entities.}
\end{figure}

\section{Related Topics}\label{related-topics-28}

\begin{itemize}
\tightlist
\item
  \href{/docs/7-2/reference/-/knowledge_base/r/clay-dropdown-menus-and-action-menus}{Clay
  Dropdown Menus and Action Menus}
\item
  \href{/docs/7-2/reference/-/knowledge_base/r/clay-icons}{Clay Icons}
\item
  \href{/docs/7-2/reference/-/knowledge_base/r/clay-navigation-bars}{Clay
  Navigation Bars}
\end{itemize}

\chapter{Clay Navigation Bars}\label{clay-navigation-bars}

Similar to dropdown menus, navigation bars display a list of navigation
items. The key difference is navigation bars are displayed in a
horizontal bar with all navigation items visible at all times. The
navigation bar also indicates the active navigation item with an
underline. Navigation bars come in two styles: white background with
dark-grey text (default) and dark-grey background with white text
(inverted).

Default navigation bar:

\begin{verbatim}
<clay:navigation-bar 
    navigationItems="<%= navigationBarsDisplayContext.getNavigationItems() %>" 
/>
\end{verbatim}

\begin{figure}
\centering
\includegraphics{./images/clay-taglib-nav-bars.png}
\caption{You can include navigation bars in your apps.}
\end{figure}

Inverted navigation bar (set \texttt{inverted} attribute to
\texttt{true}):

\begin{verbatim}
<clay:navigation-bar 
    inverted="<%= true %>" 
    navigationItems="<%= navigationBarsDisplayContext.getNavigationItems() %>" 
/>
\end{verbatim}

\begin{figure}
\centering
\includegraphics{./images/clay-taglib-nav-bars-inverted.png}
\caption{Navigation bars can be inverted if you prefer.}
\end{figure}

\section{Related Topics}\label{related-topics-29}

\begin{itemize}
\tightlist
\item
  \href{/docs/7-2/reference/-/knowledge_base/r/clay-dropdown-menus-and-action-menus}{Clay
  Dropdown Menus and Action Menus}
\item
  \href{/docs/7-2/reference/-/knowledge_base/r/clay-form-elements}{Clay
  Form Elements}
\item
  \href{/docs/7-2/reference/-/knowledge_base/r/clay-progress-bars}{Clay
  Progress Bars}
\end{itemize}

\chapter{Clay Progress Bars}\label{clay-progress-bars}

You can add progress bars to your app with the \texttt{clay:progressbar}
tag. These indicate the completion percentage of a task and come in
three status styles: \texttt{default} (blue), \texttt{warning} (red),
and \texttt{complete} (green with checkmark). You can provide a minimum
value (\texttt{minValue}) and a maximum value (\texttt{maxValue}).

Default progress bar:

\begin{verbatim}
<clay:progressbar 
    maxValue="<%= 100 %>" 
    minValue="<%= 0 %>" 
    value="<%= 30 %>" 
/>
\end{verbatim}

\begin{figure}
\centering
\includegraphics{./images/clay-taglib-progress-bar.png}
\caption{You can include progress bars in your apps.}
\end{figure}

Warning progress bar:

\begin{verbatim}
<clay:progressbar 
    maxValue="<%= 100 %>" 
    minValue="<%= 0 %>" 
    status="warning" 
    value="<%= 70 %>" 
/>
\end{verbatim}

\begin{figure}
\centering
\includegraphics{./images/clay-taglib-progress-bar-warning.png}
\caption{warning progress bars indicate that the progress has not
completed due to an error.}
\end{figure}

Complete progress bar:

\begin{verbatim}
<clay:progressbar 
    status="complete" 
/>
\end{verbatim}

\begin{figure}
\centering
\includegraphics{./images/clay-taglib-progress-bar-complete.png}
\caption{The complete progress bar indicates the progress is complete.}
\end{figure}

Clay taglibs make it easy to track progress in your apps.

\section{Related Topics}\label{related-topics-30}

\begin{itemize}
\tightlist
\item
  \href{/docs/7-2/reference/-/knowledge_base/r/clay-dropdown-menus-and-action-menus}{Clay
  Dropdown Menus and Action Menus}
\item
  \href{/docs/7-2/reference/-/knowledge_base/r/clay-icons}{Clay Icons}
\item
  \href{/docs/7-2/reference/-/knowledge_base/r/clay-navigation-bars}{Clay
  Navigation Bars}
\end{itemize}

\chapter{Clay Stickers}\label{clay-stickers}

Whereas badges display numbers and labels display short information,
stickers are small visual indicators of the content (usually the content
type). They can include a small label or a Liferay icon, and they come
in two shapes: circle and square.

Square sticker with label:

\begin{verbatim}
<clay:sticker label="JPG" />
\end{verbatim}

\begin{figure}
\centering
\includegraphics{./images/clay-taglib-sticker-square-label.png}
\caption{You can include stickers in your apps.}
\end{figure}

Square sticker with icon:

\begin{verbatim}
<clay:sticker icon="picture" />
\end{verbatim}

\begin{figure}
\centering
\includegraphics{./images/clay-taglib-sticker-square-icon.png}
\caption{Stickers can include icons.}
\end{figure}

Circle sticker:

\begin{verbatim}
<clay:sticker label="JPG" shape="circle" />
\end{verbatim}

\begin{figure}
\centering
\includegraphics{./images/clay-taglib-sticker-round.png}
\caption{You can also have circle stickers.}
\end{figure}

Stickers can be positioned in any corner of a div. Indicate their
position with the \texttt{position} attribute: \texttt{top-left},
\texttt{bottom-left}, \texttt{top-right}, or \texttt{bottom-right}:

\begin{verbatim}
<div class="aspect-ratio">

    <img class="aspect-ratio-item-fluid" src="](./images/thumbnail_hot_air_ballon.jpg" />

    <clay:sticker label="PDF" position="top-left" style="danger" />
</div>
\end{verbatim}

\begin{figure}
\centering
\includegraphics{./images/clay-taglib-sticker-position.png}
\caption{You can specify the position of the sticker within a
container.}
\end{figure}

Now you know how to use Clay stickers in your app!

\section{Related Topics}\label{related-topics-31}

\begin{itemize}
\tightlist
\item
  \href{/docs/7-2/reference/-/knowledge_base/r/clay-badges}{Clay Badges}
\item
  \href{/docs/7-2/reference/-/knowledge_base/r/clay-cards}{Clay Cards}
\item
  \href{/docs/7-2/reference/-/knowledge_base/r/clay-icons}{Clay Icons}
\end{itemize}

\chapter{Using the Chart Taglib in Your
Portlets}\label{using-the-chart-taglib-in-your-portlets}

Lines, splines, bars, pies and more, the Chart tag Library provides
everything you need to model data. Each taglib gives you access to the
corresponding
\href{https://github.com/liferay/clay/tree/2.x-stable/packages/clay-charts/src}{Clay
component}. These components contain the default configuration for the
UI.

To use the Chart taglib in your apps, add the following declaration to
your JSP:

\begin{verbatim}
<%@ taglib prefix="chart" uri="http://liferay.com/tld/chart" %>
\end{verbatim}

This section covers the types of charts you can create with the Chart
taglibs. Each article contains a set of chart examples along with sample
Java data and a figure displaying the rendered results.

\begin{figure}
\centering
\includegraphics{./images/chart-taglib-sample-portlet.png}
\caption{You can create many different types of charts with the chart
taglibs.}
\end{figure}

\chapter{Bar Charts}\label{bar-charts}

Bar charts contain multiple sets of data. A bar chart models the data in
bars. Each data series (created with the \texttt{addColumns()} method)
is defined with a new instance of the
\href{https://docs.liferay.com/portal/7.2-latest/apps/frontend-taglib-2.0.2/javadocs/com/liferay/frontend/taglib/chart/model/MultiValueColumn.html}{\texttt{MultiValueColumn}
object}, which takes an ID and a set of values. Follow these steps to
configure your portlet to use bar charts.

\begin{enumerate}
\def\labelenumi{\arabic{enumi}.}
\item
  Import the chart taglib along with the \texttt{BarChartConfig} and
  \texttt{MultiValueColumn} classes into your bundle's \texttt{init.jsp}
  file:

\begin{verbatim}
<%@ taglib prefix="chart" uri="http://liferay.com/tld/chart" %>
<%@ page import="com.liferay.frontend.taglib.chart.model.point.bar.BarChartConfig" %>
<%@ page import="com.liferay.frontend.taglib.chart.model.MultiValueColumn" %>
\end{verbatim}
\item
  Add the following Java scriptlet to the top of your \texttt{view.jsp}:

\begin{verbatim}
<%
BarChartConfig _barChartConfig = new BarChartConfig();

_barChartConfig.addColumns(
    new MultiValueColumn("data1", 100, 20, 30),
    new MultiValueColumn("data2", 20, 70, 100)
);
%>
\end{verbatim}
\item
  Add the \texttt{\textless{}chart\textgreater{}} taglib to the
  \texttt{view.jsp}, passing the \texttt{\_barChartConfig} as the
  \texttt{config} attribute's value:

\begin{verbatim}
<chart:bar
  config="<%= _barChartConfig %>"
/>
\end{verbatim}
\end{enumerate}

\begin{figure}
\centering
\includegraphics{./images/chart-taglib-bar.png}
\caption{A bar chart models the data in bars.}
\end{figure}

Awesome! Now you know how to create bar charts for your apps.

\section{Related Topics}\label{related-topics-32}

\begin{itemize}
\tightlist
\item
  \href{/docs/7-2/reference/-/knowledge_base/r/line-charts}{Line Charts}
\item
  \href{/docs/7-2/reference/-/knowledge_base/r/donut-charts}{Donut
  Charts}
\item
  \href{/docs/7-2/reference/-/knowledge_base/r/combination-charts}{Combination
  Charts}
\end{itemize}

\chapter{Line Charts}\label{line-charts}

Line charts contain multiple sets of data. A Line chart displays the
data linearly. Each data series (created with the \texttt{addColumns()}
method) is defined with a new instance of the
\href{https://docs.liferay.com/portal/7.2-latest/apps/frontend-taglib-2.0.2/javadocs/com/liferay/frontend/taglib/chart/model/MultiValueColumn.html}{\texttt{MultiValueColumn}
object}, which takes an ID and a set of values. Follow these steps to
configure your portlet to use line charts.

\begin{enumerate}
\def\labelenumi{\arabic{enumi}.}
\item
  Import the chart taglib along with the \texttt{LineChartConfig} and
  \texttt{MultiValueColumn} classes into your bundle's \texttt{init.jsp}
  file:

\begin{verbatim}
<%@ taglib prefix="chart" uri="http://liferay.com/tld/chart" %>
<%@ page import="com.liferay.frontend.taglib.chart.model.point.line.LineChartConfig" %>
<%@ page import="com.liferay.frontend.taglib.chart.model.MultiValueColumn" %>
\end{verbatim}
\item
  Add the following Java scriptlet to the top of your \texttt{view.jsp}:

\begin{verbatim}
<%
LineChartConfig _lineChartConfig = new LineChartConfig();

_lineChartConfig.addColumns(
  new MultiValueColumn("data1", 100, 20, 30),
  new MultiValueColumn("data2", 20, 70, 100)
);
%>
\end{verbatim}
\item
  Add the \texttt{\textless{}chart\textgreater{}} taglib to the
  \texttt{view.jsp}, passing the \texttt{\_lineChartConfig} as the
  \texttt{config} attribute's value:

\begin{verbatim}
<chart:line
  config="<%= _lineChartConfig %>"
/>
\end{verbatim}
\end{enumerate}

\begin{figure}
\centering
\includegraphics{./images/chart-taglib-line.png}
\caption{A Line chart displays the data linearly.}
\end{figure}

Awesome! Now you know how to create line charts for your apps.

\section{Related Topics}\label{related-topics-33}

\begin{itemize}
\tightlist
\item
  \href{/docs/7-2/reference/-/knowledge_base/r/spline-charts}{Spline
  Charts}
\item
  \href{/docs/7-2/reference/-/knowledge_base/r/step-charts}{Step Charts}
\item
  \href{/docs/7-2/reference/-/knowledge_base/r/predictive-charts}{Predictive
  Charts}
\end{itemize}

\chapter{Scatter Charts}\label{scatter-charts}

Scatter charts contain multiple sets of data. A scatter chart models the
data as individual points. Each data series (created with the
\texttt{addColumns()} method) is defined with a new instance of the
\href{https://docs.liferay.com/portal/7.2-latest/apps/frontend-taglib-2.0.2/javadocs/com/liferay/frontend/taglib/chart/model/MultiValueColumn.html}{\texttt{MultiValueColumn}
object}, which takes an ID and a set of values. Follow these steps to
configure your portlet to use scatter charts.

\begin{enumerate}
\def\labelenumi{\arabic{enumi}.}
\item
  Import the chart taglib along with the \texttt{ScatterChartConfig} and
  \texttt{MultiValueColumn} classes into your bundle's \texttt{init.jsp}
  file:

\begin{verbatim}
<%@ taglib prefix="chart" uri="http://liferay.com/tld/chart" %>
<%@ page import="com.liferay.frontend.taglib.chart.model.point.scatter.ScatterChartConfig" %>
<%@ page import="com.liferay.frontend.taglib.chart.model.MultiValueColumn" %>
\end{verbatim}
\item
  Add the following Java scriptlet to the top of your \texttt{view.jsp}:

\begin{verbatim}
<%
ScatterChartConfig _scatterChartConfig = new ScatterChartConfig();

_scatterChartConfig.addColumns(
  new MultiValueColumn("data1", 100, 20, 30),
  new MultiValueColumn("data2", 20, 70, 100));
%>
\end{verbatim}
\item
  Add the \texttt{\textless{}chart\textgreater{}} taglib to the
  \texttt{view.jsp}, passing the \texttt{\_scatterChartConfig} as the
  \texttt{config} attribute's value:

\begin{verbatim}
<chart:scatter
  config="<%= _scatterChartConfig %>"
/>
\end{verbatim}
\end{enumerate}

\begin{figure}
\centering
\includegraphics{./images/chart-taglib-scatter.png}
\caption{A scatter chart models the data as individual points.}
\end{figure}

Awesome! Now you know how to create scatter charts for your apps.

\section{Related Topics}\label{related-topics-34}

\begin{itemize}
\tightlist
\item
  \href{/docs/7-2/reference/-/knowledge_base/r/line-charts}{Line Charts}
\item
  \href{/docs/7-2/reference/-/knowledge_base/r/step-charts}{Step Charts}
\item
  \href{/docs/7-2/reference/-/knowledge_base/r/predictive-charts}{Predictive
  Charts}
\end{itemize}

\chapter{Spline Charts}\label{spline-charts}

Spline charts contain multiple sets of data. A spline chart connects
points of data with a smooth curve. Each data series (created with the
\texttt{addColumns()} method) is defined with a new instance of the
\href{https://docs.liferay.com/portal/7.2-latest/apps/frontend-taglib-2.0.2/javadocs/com/liferay/frontend/taglib/chart/model/MultiValueColumn.html}{\texttt{MultiValueColumn}
object}, which takes an ID and a set of values. Follow these steps to
configure your portlet to use spline charts.

\begin{enumerate}
\def\labelenumi{\arabic{enumi}.}
\item
  Import the chart taglib along with the \texttt{SplineChartConfig} and
  \texttt{MultiValueColumn} classes into your bundle's \texttt{init.jsp}
  file:

\begin{verbatim}
<%@ taglib prefix="chart" uri="http://liferay.com/tld/chart" %>
<%@ page import="com.liferay.frontend.taglib.chart.model.point.spline.SplineChartConfig" %>
<%@ page import="com.liferay.frontend.taglib.chart.model.MultiValueColumn" %>
\end{verbatim}
\item
  Add the following Java scriptlet to the top of your \texttt{view.jsp}:

\begin{verbatim}
<%
SplineChartConfig _splineChartConfig = new SplineChartConfig();

_splineChartConfig.addColumns(
  new MultiValueColumn("data1", 100, 20, 30),
  new MultiValueColumn("data2", 20, 70, 100)
);
%>
\end{verbatim}
\item
  Add the \texttt{\textless{}chart\textgreater{}} taglib to the
  \texttt{view.jsp}, passing the \texttt{\_splineChartConfig} as the
  \texttt{config} attribute's value:

\begin{verbatim}
<chart:spline
  config="<%= _splineChartConfig %>"
/>
\end{verbatim}
\end{enumerate}

\begin{figure}
\centering
\includegraphics{./images/chart-taglib-spline.png}
\caption{A spline chart connects points of data with a smooth curve.}
\end{figure}

You can also use an area spline chart if you prefer. An area spline
chart highlights the area under the spline curve.

\begin{verbatim}
<chart:area-spline
  config="<%= _splineChartConfig %>"
/>
\end{verbatim}

\begin{figure}
\centering
\includegraphics{./images/chart-taglib-area-spline.png}
\caption{An area spline chart highlights the area under the spline
curve.}
\end{figure}

Awesome! Now you know how to create spline charts for your apps.

\section{Related Topics}\label{related-topics-35}

\begin{itemize}
\tightlist
\item
  \href{/docs/7-2/reference/-/knowledge_base/r/line-charts}{Line Charts}
\item
  \href{/docs/7-2/reference/-/knowledge_base/r/step-charts}{Step Charts}
\item
  \href{/docs/7-2/reference/-/knowledge_base/r/scatter-charts}{Scatter
  Charts}
\end{itemize}

\chapter{Step Charts}\label{step-charts}

Step charts contain multiple sets of data. A step chart steps between
the points of data, resembling steps. Each data series (created with the
\texttt{addColumns()} method) is defined with a new instance of the
\href{https://docs.liferay.com/portal/7.2-latest/apps/frontend-taglib-2.0.2/javadocs/com/liferay/frontend/taglib/chart/model/MultiValueColumn.html}{\texttt{MultiValueColumn}
object}, which takes an ID and a set of values. Follow these steps to
configure your portlet to use step charts.

\begin{enumerate}
\def\labelenumi{\arabic{enumi}.}
\item
  Import the chart taglib along with the \texttt{StepChartConfig} and
  \texttt{MultiValueColumn} classes into your bundle's \texttt{init.jsp}
  file:

\begin{verbatim}
<%@ taglib prefix="chart" uri="http://liferay.com/tld/chart" %>
<%@ page import="com.liferay.frontend.taglib.chart.model.point.step.StepChartConfig" %>
<%@ page import="com.liferay.frontend.taglib.chart.model.MultiValueColumn" %>
\end{verbatim}
\item
  Add the following Java scriptlet to the top of your \texttt{view.jsp}:

\begin{verbatim}
<%
StepChartConfig _stepChartConfig = new StepChartConfig();

_stepChartConfig.addColumns(
  new MultiValueColumn("data1", 100, 20, 30),
  new MultiValueColumn("data2", 20, 70, 100)
);
%>
\end{verbatim}
\item
  Add the \texttt{\textless{}chart\textgreater{}} taglib to the
  \texttt{view.jsp}, passing the \texttt{\_stepChartConfig} as the
  \texttt{config} attribute's value:

\begin{verbatim}
<chart:step
  config="<%= _stepChartConfig %>"
/>
\end{verbatim}
\end{enumerate}

\begin{figure}
\centering
\includegraphics{./images/chart-taglib-step.png}
\caption{A step chart steps between the points of data, resembling
steps.}
\end{figure}

You can also use an area step chart if you prefer. An area step chart
highlights the area covered by a step graph.

\begin{verbatim}
<chart:area-step
  config="<%= _stepChartConfig %>"
/>
\end{verbatim}

\begin{figure}
\centering
\includegraphics{./images/chart-taglib-area-step.png}
\caption{An area step chart highlights the area covered by a step
graph.}
\end{figure}

Awesome! Now you know how to create step charts for your apps.

\section{Related Topics}\label{related-topics-36}

\begin{itemize}
\tightlist
\item
  \href{/docs/7-2/reference/-/knowledge_base/r/line-charts}{Line Charts}
\item
  \href{/docs/7-2/reference/-/knowledge_base/r/spline-charts}{Spline
  Charts}
\item
  \href{/docs/7-2/reference/-/knowledge_base/r/scatter-charts}{Scatter
  Charts}
\end{itemize}

\chapter{Combination Charts}\label{combination-charts}

Combination charts have minor differences from other charts. In a
combination chart, you must define the representation type of each data
set: \texttt{AREA}, \texttt{AREA\_SPLINE}, \texttt{AREA\_STEP},
\texttt{BAR}, \texttt{BUBBLE}, \texttt{DONUT}, \texttt{GAUGE},
\texttt{LINE}, \texttt{PIE}, \texttt{SCATTER}, \texttt{SPLINE}, or
\texttt{STEP}. Each data set in a combination chart is an instance of
the
\href{https://docs.liferay.com/portal/7.2-latest/apps/frontend-taglib-2.0.2/javadocs/com/liferay/frontend/taglib/chart/model/TypedMultiValueColumn.html}{\texttt{TypedMultiValueColumn}
object}. Each object receives an ID, the representation type, and values
for the data. Follow these steps to configure your portlet to use
combination charts.

\begin{enumerate}
\def\labelenumi{\arabic{enumi}.}
\item
  Import the chart taglib along with the
  \texttt{CombinationChartConfig}, \texttt{MultiValueColumn}, and
  \texttt{MultiValueColumn.Type} classes into your bundle's
  \texttt{init.jsp} file:

\begin{verbatim}
<%@ taglib prefix="chart" uri="http://liferay.com/tld/chart" %>
<%@ page import="com.liferay.frontend.taglib.chart.model.combination.CombinationChartConfig" %>
<%@ page import="com.liferay.frontend.taglib.chart.model.MultiValueColumn" %>
<%@ page import="com.liferay.frontend.taglib.chart.model.TypedMultiValueColumn.Type" %>
\end{verbatim}
\item
  Add the following Java scriptlet to the top of your \texttt{view.jsp}:

\begin{verbatim}
<%
CombinationChartConfig _combinationChartConfig =
new CombinationChartConfig();

_combinationChartConfig.addColumns(
  new TypedMultiValueColumn(
    "data1", Type.BAR, 30, 20, 50, 40, 60, 50),
  new TypedMultiValueColumn(
    "data2", Type.BAR, 200, 130, 90, 240, 130, 220),
  new TypedMultiValueColumn(
    "data3", Type.SPLINE, 300, 200, 160, 400, 250, 250),
  new TypedMultiValueColumn(
    "data4", Type.LINE, 200, 130, 90, 240, 130, 220),
  new TypedMultiValueColumn(
    "data5", Type.BAR, 130, 120, 150, 140, 160, 150),
  new TypedMultiValueColumn(
    "data6", Type.AREA, 90, 70, 20, 50, 60, 120)
  );

_combinationChartConfig.addGroup("data1", "data2");
%>
\end{verbatim}
\item
  Add the \texttt{\textless{}chart\textgreater{}} taglib to the
  \texttt{view.jsp}, passing the \texttt{\_combinationChartConfig} as
  the \texttt{config} attribute's value:

\begin{verbatim}
<chart:combination
  config="<%= _combinationChartConfig %>"
/>
\end{verbatim}
\end{enumerate}

\begin{figure}
\centering
\includegraphics{./images/chart-taglib-combination.png}
\caption{A combination chart displays a variety of data set types.}
\end{figure}

Awesome! Now you know how to create combination charts for your apps.

\section{Related Topics}\label{related-topics-37}

\begin{itemize}
\tightlist
\item
  \href{/docs/7-2/reference/-/knowledge_base/r/bar-charts}{Bar Charts}
\item
  \href{/docs/7-2/reference/-/knowledge_base/r/line-charts}{Line Charts}
\item
  \href{/docs/7-2/reference/-/knowledge_base/r/geomap-charts}{Geomap
  Charts}
\end{itemize}

\chapter{Donut Charts}\label{donut-charts}

Donut charts are percentage-based. A donut chart is similar to a pie
chart, but it has a hole in the center. Each data set must be defined as
a new instance of the
\href{https://docs.liferay.com/portal/7.2-latest/apps/frontend-taglib-2.0.2/javadocs/com/liferay/frontend/taglib/chart/model/SingleValueColumn.html}{\texttt{SingleValueColumn}
object}. Follow these steps to configure your portlet to use donut
charts.

\begin{enumerate}
\def\labelenumi{\arabic{enumi}.}
\item
  Import the chart taglib along with the \texttt{DonutChartConfig} and
  \texttt{SingleValueColumn} classes into your bundle's
  \texttt{init.jsp} file:

\begin{verbatim}
<%@ taglib prefix="chart" uri="http://liferay.com/tld/chart" %>
<%@ page import="com.liferay.frontend.taglib.chart.model.percentage.donut.DonutChartConfig" %>
<%@ page import="com.liferay.frontend.taglib.chart.model.SingleValueColumn" %>
\end{verbatim}
\item
  Add the following Java scriptlet to the top of your \texttt{view.jsp}:

\begin{verbatim}
<%
DonutChartConfig _donutChartConfig = new DonutChartConfig();

_donutChartConfig.addColumns(
  new SingleValueColumn("data1", 30),
  new SingleValueColumn("data2", 70)
);
%>
\end{verbatim}
\item
  Add the \texttt{\textless{}chart\textgreater{}} taglib to the
  \texttt{view.jsp}, passing the \texttt{\_donutChartConfig} as the
  \texttt{config} attribute's value:

\begin{verbatim}
<chart:donut
  config="<%= _donutChartConfig %>"
/>
\end{verbatim}
\end{enumerate}

\begin{figure}
\centering
\includegraphics{./images/chart-taglib-donut.png}
\caption{A donut chart is similar to a pie chart, but it has a hole in
the center.}
\end{figure}

Awesome! Now you know how to create donut charts for your apps.

\section{Related Topics}\label{related-topics-38}

\begin{itemize}
\tightlist
\item
  \href{/docs/7-2/reference/-/knowledge_base/r/pie-charts}{Pie Charts}
\item
  \href{/docs/7-2/reference/-/knowledge_base/r/gauge-charts}{Gauge
  Charts}
\item
  \href{/docs/7-2/reference/-/knowledge_base/r/bar-charts}{Bar Charts}
\end{itemize}

\chapter{Gauge Charts}\label{gauge-charts}

Gauge charts are percentage-based. A gauge chart shows where
percentage-based data falls over a given range. Each data set must be
defined as a new instance of the
\href{https://docs.liferay.com/portal/7.2-latest/apps/frontend-taglib-2.0.2/javadocs/com/liferay/frontend/taglib/chart/model/SingleValueColumn.html}{\texttt{SingleValueColumn}
object}. Follow these steps to configure your portlet to use gauge
charts.

\begin{enumerate}
\def\labelenumi{\arabic{enumi}.}
\item
  Import the chart taglib along with the \texttt{GaugeChartConfig} and
  \texttt{SingleValueColumn} classes into your bundle's
  \texttt{init.jsp} file:

\begin{verbatim}
<%@ taglib prefix="chart" uri="http://liferay.com/tld/chart" %>
<%@ page import="com.liferay.frontend.taglib.chart.model.gauge.GaugeChartConfig" %>
<%@ page import="com.liferay.frontend.taglib.chart.model.SingleValueColumn" %>
\end{verbatim}
\item
  Add the following Java scriptlet to the top of your \texttt{view.jsp}:

\begin{verbatim}
<%
GaugeChartConfig _gaugeChartConfig = new GaugeChartConfig();

_gaugeChartConfig.addColumn(
  new SingleValueColumn("data1", 85.4)
);
%>
\end{verbatim}
\item
  Add the \texttt{\textless{}chart\textgreater{}} taglib to the
  \texttt{view.jsp}, passing the \texttt{\_gaugeChartConfig} as the
  \texttt{config} attribute's value:

\begin{verbatim}
<chart:gauge
  config="<%= _gaugeChartConfig %>"
/>
\end{verbatim}
\end{enumerate}

\begin{figure}
\centering
\includegraphics{./images/chart-taglib-gauge.png}
\caption{A gauge chart shows where percentage-based data falls over a
given range.}
\end{figure}

Awesome! Now you know how to create gauge charts for your apps.

\section{Related Topics}\label{related-topics-39}

\begin{itemize}
\tightlist
\item
  \href{/docs/7-2/reference/-/knowledge_base/r/pie-charts}{Pie Charts}
\item
  \href{/docs/7-2/reference/-/knowledge_base/r/donut-charts}{Donut
  Charts}
\item
  \href{/docs/7-2/reference/-/knowledge_base/r/bar-charts}{Bar Charts}
\end{itemize}

\chapter{Pie Charts}\label{pie-charts}

Pie charts are percentage-based. A pie chart models percentage-based
data as individual slices of pie. Each data set must be defined as a new
instance of the
\href{https://docs.liferay.com/portal/7.2-latest/apps/frontend-taglib-2.0.2/javadocs/com/liferay/frontend/taglib/chart/model/SingleValueColumn.html}{\texttt{SingleValueColumn}
object}. Follow these steps to configure your portlet to use pie charts.

\begin{enumerate}
\def\labelenumi{\arabic{enumi}.}
\item
  Import the chart taglib along with the \texttt{PieChartConfig} and
  \texttt{SingleValueColumn} classes into your bundle's
  \texttt{init.jsp} file:

\begin{verbatim}
<%@ taglib prefix="chart" uri="http://liferay.com/tld/chart" %>
<%@ page import="com.liferay.frontend.taglib.chart.model.percentage.pie.PieChartConfig" %>
<%@ page import="com.liferay.frontend.taglib.chart.model.SingleValueColumn" %>
\end{verbatim}
\item
  Add the following Java scriptlet to the top of your \texttt{view.jsp}:

\begin{verbatim}
<%
PieChartConfig _pieChartConfig = new PieChartConfig();

_pieChartConfig.addColumn(
  new SingleValueColumn("data1", 85.4)
);
%>
\end{verbatim}
\item
  Add the \texttt{\textless{}chart\textgreater{}} taglib to the
  \texttt{view.jsp}, passing the \texttt{\_pieChartConfig} as the
  \texttt{config} attribute's value:

\begin{verbatim}
<chart:pie
  config="<%= _pieChartConfig %>"
/>
\end{verbatim}
\end{enumerate}

\begin{figure}
\centering
\includegraphics{./images/chart-taglib-pie.png}
\caption{A pie chart models percentage-based data as individual slices
of pie.}
\end{figure}

Awesome! Now you know how to create pie charts for your apps.

\section{Related Topics}\label{related-topics-40}

\begin{itemize}
\tightlist
\item
  \href{/docs/7-2/reference/-/knowledge_base/r/donut-charts}{Donut
  Charts}
\item
  \href{/docs/7-2/reference/-/knowledge_base/r/gauge-charts}{Gauge
  Charts}
\item
  \href{/docs/7-2/reference/-/knowledge_base/r/spine-charts}{Spline
  Charts}
\end{itemize}

\chapter{Geomap Charts}\label{geomap-charts}

A Geomap Chart lets you visualize data based on geography, given a
specified color range---a lighter color representing a lower rank and a
darker a higher rank usually. The default configuration comes from the
Clay charts
\href{https://github.com/liferay/clay/blob/2.x-stable/packages/clay-charts/src/Geomap.js\#L90-L104}{geomap
component}: which ranges from light-blue (\#b1d4ff) to dark-blue
(\#0065e4) and ranks the geography based on the location's
\texttt{pop\_est} value (specified in the geomap's JSON file).

\begin{figure}
\centering
\includegraphics{./images/chart-taglib-geomap-default.png}
\caption{A Geomap chart displays a heatmap representing the data.}
\end{figure}

Follow these steps to configure your portlet to use geomap charts.

\begin{enumerate}
\def\labelenumi{\arabic{enumi}.}
\item
  Import the chart taglib along with the \texttt{GeomapConfig},
  \texttt{GeomapColor}, and \texttt{GeomapColorRange} classes into your
  bundle's \texttt{init.jsp} file:

\begin{verbatim}
<%@ taglib prefix="chart" uri="http://liferay.com/tld/chart" %>
<%@ page import="com.liferay.frontend.taglib.chart.model.geomap.GeomapConfig" %>
<%@ page import="com.liferay.frontend.taglib.chart.model.geomap.GeomapColor" %>
<%@ page import="com.liferay.frontend.taglib.chart.model.geomap.GeomapColorRange" %>
\end{verbatim}
\item
  Add the following Java scriptlet to the top of your \texttt{view.jsp}.
  The colors---a color for minimum and a color for maximum---are
  completely configurable, as shown in the second example configuration
  below: \texttt{\_geomapConfig2}. Create a new
  \texttt{GeomapColorRange} and set the minimum and maximum color values
  with the \texttt{setMax()} and \texttt{setMin()} methods. Specify the
  highlight color---the color displayed when you mouse over an
  area---with the \texttt{setSelected()} method. use the
  \texttt{geomapColor.setValue()} method to specify the JSON property to
  determine the geomap's ranking. Specify the JSON filepath with the
  \texttt{setDataHREF()} method. The example below displays a geomap
  based on the length of each location's name:

\begin{verbatim}
<%
GeomapConfig _geomapConfig1 = new GeomapConfig();
GeomapConfig _geomapConfig2 = new GeomapConfig();

GeomapColor geomapColor = new GeomapColor();
GeomapColorRange geomapColorRange = new GeomapColorRange();

geomapColorRange.setMax("#b2150a");
geomapColorRange.setMin("#ee3e32");

geomapColor.setGeomapColorRange(geomapColorRange);

geomapColor.setSelected("#a9615c");

geomapColor.setValue("name_len");

_geomapConfig2.setColor(geomapColor);

StringBuilder sb = new StringBuilder();

sb.append(_portletRequest.getScheme());
sb.append(StringPool.COLON);
sb.append(StringPool.SLASH);
sb.append(StringPool.SLASH);
sb.append(_portletRequest.getServerName());
sb.append(StringPool.COLON);
sb.append(_portletRequest.getServerPort());
sb.append(_portletRequest.getContextPath());
sb.append(StringPool.SLASH);
sb.append("geomap.geo.json");

_geomapConfig1.setDataHREF(sb.toString());
_geomapConfig2.setDataHREF(sb.toString());
%>
\end{verbatim}
\item
  Add the \texttt{\textless{}chart\textgreater{}} taglib to the
  \texttt{view.jsp} along with any styling information for the geomap,
  such as the size and margins as shown below:

\begin{verbatim}
<style type="text/css">
    .geomap {
        margin: 10px 0 10px 0;
    }
    .geomap svg {
        width: 100%;
        height: 500px !important;
    }
</style>

<chart:geomap
    config="<%= _geomapConfig1 %>"
    id="geomap-default-colors"
/>

<chart:geomap
    config="<%= _geomapConfig2 %>"
    id="geomap-custom-colors"
/>
\end{verbatim}
\end{enumerate}

\begin{figure}
\centering
\includegraphics{./images/chart-taglib-geomap-custom.png}
\caption{Geomap charts can be customized to fit the look and feel you
desire.}
\end{figure}

Awesome! Now you know how to create geomap charts for your apps.

\section{Related Topics}\label{related-topics-41}

\begin{itemize}
\tightlist
\item
  \href{/docs/7-2/reference/-/knowledge_base/r/bar-charts}{Bar Charts}
\item
  \href{/docs/7-2/reference/-/knowledge_base/r/pie-charts}{Pie Charts}
\item
  \href{/docs/7-2/reference/-/knowledge_base/r/combination-charts}{Combination
  Charts}
\end{itemize}

\chapter{Predictive Charts}\label{predictive-charts}

Predictive charts let you visualize current data along with
predicted/forecasted data within a given value range.

\begin{figure}
\centering
\includegraphics{./images/chart-taglib-predictive-value-range.png}
\caption{Predicted/forecasted data is surrounded by a highlighted area
of possible values.}
\end{figure}

Follow these steps to use predictive charts.

\begin{enumerate}
\def\labelenumi{\arabic{enumi}.}
\item
  Import the chart taglib along with the \texttt{PredictiveChartConfig}
  and \texttt{MixedDataColumn} classes into your bundle's
  \texttt{init.jsp} file:

\begin{verbatim}
<%@ taglib prefix="chart" uri="http://liferay.com/tld/chart" %>
<%@ page import="com.liferay.frontend.taglib.chart.model.predictive.PredictiveChartConfig" %>
<%@ page import="com.liferay.frontend.taglib.chart.model.MixedDataColumn" %>
\end{verbatim}
\item
  Add the following Java scriptlet to the top of your \texttt{view.jsp}.
  Add a
  \href{https://docs.liferay.com/portal/7.2-latest/apps/frontend-taglib-2.0.2/javadocs/com/liferay/frontend/taglib/chart/model/MixedDataColumn.html}{\texttt{MixedDataColumn}
  object} ---a column that supports both single number values and arrays
  of three numbers---for each data series. Single number values define
  existing data. Arrays of numbers are used as the prediction/forecast
  data and contain three numbers: a minimum value, an estimated value,
  and a maximum value. The estimated value is rendered solid and
  surrounded by a highlighted area with borders specified by the minimum
  and maximum values. This lets you visualize your estimated values,
  while also giving you an idea of the possible value ranges. Use the
  \texttt{addDataColumn()} method to add each data series:

\begin{verbatim}
<%
private PredictiveChartConfig _predictiveChartConfig = new
PredictiveChartConfig();

MixedDataColumn mixedDataColumn1 = new MixedDataColumn(
  "data1", 130, 340, 200, 500, 80, 240, 40,
  new Number[] {370, 400, 450}, new Number[] {210, 240, 270},
  new Number[] {150, 180, 210}, new Number[] {60, 90, 120},
  new Number[] {310, 340, 370}
);

_predictiveChartConfig.addDataColumn(mixedDataColumn1);

MixedDataColumn mixedDataColumn2 = new MixedDataColumn(
  "data2", 210, 160, 50, 125, 230, 110, 90,
  Arrays.asList(170, 200, 230), Arrays.asList(10, 40, 70),
  Arrays.asList(350, 380, 410), Arrays.asList(260, 290, 320),
  Arrays.asList(30, 70, 150)
);

_predictiveChartConfig.addDataColumn(mixedDataColumn2);

_predictiveChartConfig.setAxisXTickFormat("%b");

_predictiveChartConfig.setPredictionDate("2018-07-01");

List<String> timeseries = new ArrayList<>();

timeseries.add("2018-01-01");
timeseries.add("2018-02-01");
timeseries.add("2018-03-01");
timeseries.add("2018-04-01");
timeseries.add("2018-05-01");
timeseries.add("2018-06-01");
timeseries.add("2018-07-01");
timeseries.add("2018-08-01");
timeseries.add("2018-09-01");
timeseries.add("2018-10-01");
timeseries.add("2018-11-01");
timeseries.add("2018-12-01");

_predictiveChartConfig.setTimeseries(timeseries);
%>
\end{verbatim}

  Predictive charts have these properties:

  \textbf{axisXTickFormat:} An optional string which specfies the time
  formatting on the X axis. For more information on which formats can be
  specified please refer to
  \href{https://github.com/d3/d3-time-format/blob/master/README.md\#locale_format}{d3's
  time format README}. This value is set using the
  \texttt{setAxisXTickFormat()} method.

  \textbf{Prediction Date:} A date as a string that represents the point
  in the timeline from when the forecast/prediction is shown. This value
  is parsed as a Date object in JavaScript and set using the
  \texttt{setPredictionDate()} method.

  \textbf{Time Series:} A timeline for the data which is displayed on
  the X axis of the chart. This value is set as an array of dates
  (\texttt{2018-01-01} for example).
\item
  Add the \texttt{\textless{}chart\textgreater{}} taglib to the
  \texttt{view.jsp}, passing the \texttt{\_predictiveChartConfig} as the
  \texttt{config} attribute's value:

\begin{verbatim}
<chart:predictive
  config="<%= _predictiveChartConfig %>"
/>
\end{verbatim}
\end{enumerate}

The area contained within the light-blue rectangle is the point from
which the predicted/forecasted values are shown:

\begin{figure}
\centering
\includegraphics{./images/chart-taglib-predictive.png}
\caption{A predictive chart lets you visualize estimated future data
alongside existing data.}
\end{figure}

Awesome! Now you know how to create predictive charts for your apps.

\section{Related Topics}\label{related-topics-42}

\begin{itemize}
\tightlist
\item
  \href{/docs/7-2/reference/-/knowledge_base/r/Line-charts}{Line Charts}
\item
  \href{/docs/7-2/reference/-/knowledge_base/r/combination-charts}{Combination
  Charts}
\item
  \href{/docs/7-2/reference/-/knowledge_base/r/geomap-charts}{Geomap
  Charts}
\end{itemize}

\chapter{Refreshing Charts to Reflect Real Time
Data}\label{refreshing-charts-to-reflect-real-time-data}

The polling interval property is an optional property for all charts. It
specifies the time in milliseconds for the chart's data to refresh. You
can use this for charts that receive any kind of real time data, such as
a JSON file that changes periodically. This ensures that the chart is up
to date, reflecting the most recent data. Follow these steps to
configure your chart to use real time data.

\begin{enumerate}
\def\labelenumi{\arabic{enumi}.}
\item
  Add a new java scriptlet and create a new instance of the chart's
  object, and put the data into the \texttt{data} attribute. Finally,
  set the chart's polling interval with the
  \texttt{setPollingInterval()} method. An example \texttt{view.jsp}
  configuration is shown below:

\begin{verbatim}
```java
<%
LineChartConfig _pollingIntervalLineChartConfig = new LineChartConfig();

_pollingIntervalLineChartConfig.put("data", "/foo.json");

_pollingIntervalLineChartConfig.setPollingInterval(2000);
%>
```
\end{verbatim}
\item
  Set the chart taglib's \texttt{config} attribute to the updated
  configuration object that you created in the last step, as shown in
  the example below:

\begin{verbatim}
```markup
<chart:line
    componentId="polling-interval-line-chart"
    config="<%= _pollingIntervalLineChartConfig %>"
/>
```
\end{verbatim}
\end{enumerate}

\begin{figure}
\centering
\includegraphics{./images/chart-polling-interval.png}
\caption{The polling interval property lets you refresh charts at a
given interval to reflect real time data.}
\end{figure}

Now you know how to reflect real time data in your charts!

\section{Related Topics}\label{related-topics-43}

\begin{itemize}
\tightlist
\item
  \href{/docs/7-2/reference/-/knowledge_base/r/bar-charts}{Bar Charts}
\item
  \href{/docs/7-2/reference/-/knowledge_base/r/scatter-charts}{Scatter
  Charts}
\item
  \href{/docs/7-2/reference/-/knowledge_base/r/donut-charts}{Donut
  Charts}
\end{itemize}

\chapter{Using AUI Taglibs}\label{using-aui-taglibs}

The AUI tag library provides tags that implement commonly used UI
components. These tags make your markup consistent, responsive, and
accessible.

You can find a list of the available
\texttt{\textless{}aui\textgreater{}} taglibs in the
\href{https://docs.liferay.com/portal/7.2-latest/taglibs/util-taglib/aui/tld-summary.html}{AUI
taglibdocs}. Each taglib has a list of attributes that can be passed to
the tag. Some of these are required, and some are optional. See the
taglibdocs to view the requirements for each tag. You'll find the full
markup generated by the tags in their JSPs in their
\href{https://github.com/liferay/liferay-portal/tree/7.2.x/portal-web/docroot/html/taglib/aui}{Liferay
Github Repo} folders.

To use the AUI taglib library in your apps, you must add the following
declaration to your JSP:

\begin{verbatim}
<%@ taglib prefix="aui" uri="http://liferay.com/tld/aui" %>
\end{verbatim}

The AUI taglib is also available via a macro for your FreeMarker theme
templates and web content templates. Follow this syntax:

\begin{verbatim}
<@liferay_aui["tag-name"] attribute="string value" attribute=10 />
\end{verbatim}

This section covers how to create UI components with the AUI taglibs.
Each article contains code examples along with a screenshot of the
resulting UI.

\chapter{Building Forms with AUI
Tags}\label{building-forms-with-aui-tags}

The
\href{https://docs.liferay.com/portal/7.2-latest/taglibs/util-taglib/aui/tld-summary.html}{AUI
tag library} provides all the components you need to build forms for
your applications. AUI tags provide many benefits to standard form
elements, such as custom namespacing, localization, and even validation.
They provide multiple attributes that let you create the experience you
want for your users.

Follow these steps to build a form using AUI tags:

\begin{enumerate}
\def\labelenumi{\arabic{enumi}.}
\item
  Add the \texttt{aui} taglib declaration to your portlet's
  \texttt{view.jsp} if you haven't already:

\begin{verbatim}
<%@ taglib prefix="aui" uri="http://liferay.com/tld/aui" %>
\end{verbatim}
\item
  Build your form using the tags shown below. Each tag links to the
  corresponding taglibdoc that list the available attributes:

  \begin{itemize}
  \tightlist
  \item
    \href{https://docs.liferay.com/ce/portal/7.2-latest/taglibs/util-taglib/aui/input.html}{\texttt{\textless{}aui:input\textgreater{}}}
  \item
    \href{https://docs.liferay.com/ce/portal/7.2-latest/taglibs/util-taglib/aui/button.html}{\texttt{\textless{}aui:button\textgreater{}}}
  \item
    \href{https://docs.liferay.com/ce/portal/7.2-latest/taglibs/util-taglib/aui/button-row.html}{\texttt{\textless{}aui:button-row\textgreater{}}}
  \item
    \href{https://docs.liferay.com/ce/portal/7.2-latest/taglibs/util-taglib/aui/container.html}{\texttt{\textless{}aui:container\textgreater{}}}
  \item
    \href{https://docs.liferay.com/ce/portal/7.2-latest/taglibs/util-taglib/aui/col.html}{\texttt{\textless{}aui:col\textgreater{}}}
  \item
    \href{https://docs.liferay.com/ce/portal/7.2-latest/taglibs/util-taglib/aui/row.html}{\texttt{\textless{}aui:row\textgreater{}}}
  \item
    \href{https://docs.liferay.com/ce/portal/7.2-latest/taglibs/util-taglib/aui/field-wrapper.html}{\texttt{\textless{}aui:field-wrapper\textgreater{}}}
  \item
    \href{https://docs.liferay.com/ce/portal/7.2-latest/taglibs/util-taglib/aui/fieldset.html}{\texttt{\textless{}aui:fieldset\textgreater{}}}
  \item
    \href{https://docs.liferay.com/ce/portal/7.2-latest/taglibs/util-taglib/aui/fieldset-group.html}{\texttt{\textless{}aui:fieldset-group\textgreater{}}}
  \item
    \href{https://docs.liferay.com/ce/portal/7.2-latest/taglibs/util-taglib/aui/form.html}{\texttt{\textless{}aui:form\textgreater{}}}
  \item
    \href{https://docs.liferay.com/ce/portal/7.2-latest/taglibs/util-taglib/aui/select.html}{\texttt{\textless{}aui:select\textgreater{}}}
  \item
    \href{https://docs.liferay.com/ce/portal/7.2-latest/taglibs/util-taglib/aui/option.html}{\texttt{\textless{}aui:option\textgreater{}}}
  \end{itemize}

  An example form is shown below:

\begin{verbatim}
<aui:form name="fm">
    <aui:fieldset-group markupView="lexicon">
        <aui:fieldset label="Personal Information">
            <aui:row>
                <aui:col width="50">
                    <aui:input label="First Name" name="firstName" type="text" />
                </aui:col>
                <aui:col width="50">
                    <aui:input label="Last Name" name="lastName" type="text" />
                </aui:col>
            </aui:row>
            <aui:row>
                <aui:col width="50">
                    <aui:input label="Username" name="username" type="text" />
                </aui:col>
                <aui:col width="50">
                    <aui:input label="Email" name="email" type="email" />
                </aui:col>
            </aui:row>
        </aui:fieldset>
    </aui:fieldset-group>
    <aui:fieldset-group markupView="lexicon">
        <aui:fieldset label="Miscellaneous">
            <aui:input label="Hobbies" name="hobbies" type="textarea" />
            <aui:input label="Receive email updates" name="emailUpdates" type="checkbox" />
        </aui:fieldset>
    </aui:fieldset-group>
    <aui:button-row>
        <aui:button name="submitButton" type="submit" value="Submit" />
    </aui:button-row>
</aui:form>
\end{verbatim}

  \begin{figure}
  \centering
  \includegraphics{./images/aui-taglib-basic-form.png}
  \caption{The AUI tags provide everything you need to build forms for
  your applications.}
  \end{figure}
\item
  Optionally add validation to your form fields. Nest a
  \texttt{\textless{}aui:validator\textgreater{}} tag inside each form
  field that you want to validate. Specify the validation rule with the
  \texttt{\textless{}aui:validator\textgreater{}} tag's \texttt{name}
  attribute (The available validation rules are shown in the table
  below). You can override a field's default validation error message
  with the \texttt{errorMessage} attribute. An example configuration is
  shown below:

\begin{verbatim}
<aui:form name="myForm">
    <aui:input name="password" id="password" label="Password" 
    required="true" />
    <aui:input name="confirmPassword" id="password" 
    label="Confirm Password" required="true">
        <aui:validator name="equalTo" 
        errorMessage="The passwords much match. Please try again." >
        '#<portlet:namespace>password'
        </aui:validator>
    </aui:input>
</aui:form>
\end{verbatim}

  \begin{figure}
  \centering
  \includegraphics{./images/aui-taglib-form-validation.png}
  \caption{The AUI tags also provide validation for form fields.}
  \end{figure}

  The full list of available validation rules is shown in the table
  below:
\end{enumerate}

\noindent\hrulefill

\begin{verbatim}
 Rule | Description | Default Error Message |
 --- | --- | --- |
 `acceptFiles` | Specifies that the field can only contain the file types given. Each file extension must be separated by a comma. For example </br> `<aui:validator name="acceptFiles">'jpg,png,tif,gif'</aui:validator>` | 'Please enter a file with a valid extension ([supported extensions]).' |
 `alpha` | Permits alphabetic characters | 'Please enter only alpha characters.' |
 `alphanum` | Permits alphanumeric characters | 'Please enter only alphanumeric characters.' |
 `date` | Permits dates | 'Please enter a valid date.' |
 `digits` | Permits digits | 'Please enter only digits.' |
 `email` | Permits an email address | 'Please enter a valid email address.' |
 `equalTo` | Permits contents equal to another field with the specified field ID. For example, </br> `<aui:validator name="equalTo">'#<portlet:namespace/>password'</aui:validator>` | 'Please enter the same value again.' |
 `max` | Permits an integer value less than the specified value. For example, a max value of 20 is specified with </br> `<aui:validator name="max">20</aui:validator>` | 'Please enter a value less than or equal to [max value].' |
 `maxLength` | Permits a maximum field length of the specified size (follows the same syntax as `max`) | 'Please enter no more than [max] characters.' |
 `min` | Permits an integer value greater than the specified minimum value (follows the same syntax as `max`) | 'Please enter a value greater than or equal to [min value].' |
 `minLength` | Permits a field length longer than the specified size (follows the same syntax as `max`). | 'Please enter at least [min] characters.' |
 `number` | Permits numerical values | 'Please enter a valid number.' |
 `range` | Permits a number between the specified range. For example, a range between 1.23 and 10 is specified here </br> `<aui:validator name="range">[1.23,10]</aui:validator>` | 'Please enter a value between [0] and [1].' |
 `rangeLength` | Permits a field length between the specified range (follows the same syntax as `range`)  | 'Please enter a value between [0] and [1] characters long.' |
 `required` | Prevents a blank field  | 'This field is required.' |
 `url` | Permits a URL value | 'Please enter a valid URL.' |
\end{verbatim}

\noindent\hrulefill

Now you know how to build user-friendly forms for your applications.

\section{Related Topics}\label{related-topics-44}

\begin{itemize}
\tightlist
\item
  \href{/docs/7-2/reference/-/knowledge_base/r/using-the-chart-taglib-in-your-portlets}{Using
  the Chart Taglib in Your Portlets}
\item
  \href{/docs/7-2/reference/-/knowledge_base/r/using-liferay-front-end-taglibs-in-your-portlet}{Using
  Liferay Front-end Taglibs in Your Portlet}
\item
  \href{/docs/7-2/reference/-/knowledge_base/r/using-the-clay-taglib-in-your-portlets}{Using
  the Clay Taglib in Your portlets}
\end{itemize}

\chapter{liferay-npm-bundler}\label{liferay-npm-bundler}

The liferay-npm-bundler is a bundler (like
\href{https://webpack.github.io/}{Webpack} or
\href{http://browserify.org/}{Browserify} ) that targets Liferay DXP as
a platform and assumes you're using your npm packages from widgets (as
opposed to typical web applications).

The workflow for running npm packages inside widgets is slightly
different from standard bundlers. Instead of bundling the JavaScript in
a single file, you must \emph{link} all packages together in the browser
when the full web page is assembled. This lets widgets share common
versions of modules instead of each one loading its own copy. The
liferay-npm-bundler handles this for you.

\noindent\hrulefill

\textbf{Note:} You can also find information for the liferay-npm-bundler
in the project's
\href{https://github.com/liferay/liferay-npm-build-tools/wiki}{Wiki}.

\noindent\hrulefill

\section{How the Liferay npm Bundler Works
Internally}\label{how-the-liferay-npm-bundler-works-internally}

The liferay-npm-bundler takes a widget project and outputs its files
(including npm packages) to a build folder, so the standard widget build
(Gradle) can produce an OSGi bundle. You can learn more about the build
folder's structure in
\href{/docs/7-2/reference/-/knowledge_base/r/the-structure-of-osgi-bundles-containing-npm-packages}{The
Structure of OSGi Bundles Containing NPM Packages} reference.

The liferay-npm-bundler uses the process below to create the OSGi
bundle:

\begin{enumerate}
\def\labelenumi{\arabic{enumi}.}
\item
  Copy the project's \texttt{package.json} file to the output directory.
\item
  Traverse the project's dependency tree to determine its dependencies.
\item
  For the project,

  \begin{enumerate}
  \def\labelenumii{\alph{enumii}.}
  \item
    Run the source files, specified in the \texttt{.npmbundlerrc}
    configuration, through the rules.
  \item
    Pre-process the project's package with any configured plugins.
  \item
    Run \href{https://babeljs.io/}{Babel} with configured plugins for
    each \texttt{.js} file inside the project.
  \item
    Post-process the project package with any configured plugins.
  \end{enumerate}
\item
  For each npm package dependency:

  \begin{enumerate}
  \def\labelenumii{\alph{enumii}.}
  \item
    Copy the npm package to the output folder and prefix the bundle's
    name to it. Note that the bundler stores packages in a plain
    \emph{bundle-name\$package}@\emph{version} format, rather than the
    standard node\_modules tree format. To determine what is copied, the
    bundler invokes a plugin to filter the package file list.
  \item
    Run rules on the package files.
  \item
    Pre-process the npm package with any configured plugins.
  \item
    Run \href{https://babeljs.io/}{Babel} with configured plugins for
    each \texttt{.js} file inside the npm package.
  \item
    Post-process the npm package with any configured plugins.
  \end{enumerate}
\end{enumerate}

The only difference between the pre-process and post-process steps are
when they are run (before or after Babel is run, respectively). During
this workflow, liferay-npm-bundler calls all the configured plugins so
they can perform transformations on the npm packages (for instance,
modifying their \texttt{package.json} files, or deleting or moving
files).

\noindent\hrulefill

\textbf{Note:} that the pre, post, and Babel phases were designed for
the old mode of operation (See the
\href{/docs/7-2/frameworks/-/knowledge_base/f/migrating-your-project-to-use-the-new-mode}{Migrating
Your Project to Use the New Mode} for more information) and they will
gradually be replaced with rules for the new mode.

\noindent\hrulefill

In this reference section, you'll learn more about the
liferay-npm-bundler's configuration, default presets, format, and more.

\chapter{\texorpdfstring{Understanding the \texttt{.npmbundlerrc}'s
Structure}{Understanding the .npmbundlerrc's Structure}}\label{understanding-the-.npmbundlerrcs-structure}

The liferay-npm-bundler is configured via a \texttt{.npmbundlerrc} file
placed in the widget project's root folder. You can create a complete
configuration manually or extend a configuration preset (via Babel).

This article explains the \texttt{.npmbundlerrc} file's structure. See
the
\href{/docs/7-2/reference/-/knowledge_base/r/how-the-default-preset-configures-the-liferay-npm-bundler}{default
preset reference} to learn how the default preset configures the
liferay-npm-bundler. See
\href{/docs/7-2/frameworks/-/knowledge_base/f/creating-and-bundling-javascript-widgets-with-javascript-tooling}{Creating
JavaScript Widgets with JavaScript Tooling} to learn how to use the
liferay-npm-bundler along with the Liferay JS Generator to create
JavaScript widgets.

\section{The Structure}\label{the-structure}

The \texttt{.npmbundlerrc} file has four possible phase definitions:
\emph{copy-process}, \emph{pre-process}, \emph{post-process}, and
\emph{babel}. These phase definitions are explained in more detail
below:

\textbf{Copy-Process:} Defined with the \texttt{copy-plugins} property
(only available for dependency packages). Specifies which files should
be copied or excluded from each given package.

\textbf{Pre-Process:} Defined with the \texttt{plugins} property.
Specifies plugins to run before the Babel phase is run.

\textbf{Babel:} Defined with the \texttt{.babelrc} definition. Specifies
the \texttt{.babelrc} file to use when running Babel through the
package's \texttt{.js} files.

\noindent\hrulefill

\textbf{Note:} During this phase, Babel transforms package files (for
example, to convert them to AMD format, if necessary), but doesn't
transpile them. In theory, you could also transpile them by configuring
the proper plugins. We recommend transpiling before running the bundler,
to avoid mixing both unrelated processes.

\noindent\hrulefill

\textbf{Post-Process:} Defined with the \texttt{post-plugins} property.
An alternative to using the \emph{pre-process} phase, this specifies
plugins to run after the Babel phase has completed.

Here's an example of a \texttt{.npmbundlerrc} configuration:

\begin{verbatim}
{
    "exclude": {
        "*": [
            "test/**/*"
        ],
        "some-package-name": [
            "test/**/*",
            "bin/**/*"
        ],
        "another-package-name@1.0.10": [
            "test/**/*",
            "bin/**/*",
            "lib/extras-1.0.10.js"
        ]
    },
    "include-dependencies": [
        "isobject", "isarray"
    ],
    "output": "build",
    "verbose": false,
    "dump-report": true,
    "config": {
        "imports": {
            "npm-angular5-provider": {
                "@angular/common": "^5.0.0",
            "@angular/core": "^5.0.0"
            }
        }
    },
    "/": {
    "plugins": ["resolve-linked-dependencies"],
    ".babelrc": {
      "presets": ["liferay-standard"]
    },
    "post-plugins": [
            "namespace-packages",
            "inject-imports-dependencies"
        ]
    },
    "*": {
      "copy-plugins": ["exclude-imports"],
      "plugins": ["replace-browser-modules"],
      ".babelrc": {
        "presets": ["liferay-standard"]
      },
      "post-plugins": [
        "namespace-packages",
        "inject-imports-dependencies",
        "inject-peer-dependencies"
      ]
    },
    "packages": {
        "a-package-name": [
        "copy-plugins": ["exclude-imports"],
        "plugins": ["replace-browser-modules"],
        ".babelrc": {
          "presets": ["liferay-standard"]
        },
        "post-plugins": [
          "namespace-packages",
          "inject-imports-dependencies",
          "inject-peer-dependencies"
        ]
        ],
        "other-package-name@1.0.10": [
          "copy-plugins": ["exclude-imports"],
          "plugins": ["replace-browser-modules"],
          ".babelrc": {
            "presets": ["liferay-standard"]
          },
          "post-plugins": [
            "namespace-packages",
            "inject-imports-dependencies",
            "inject-peer-dependencies"
          ]
        ]
    }
}
\end{verbatim}

\noindent\hrulefill

\textbf{Note:} Not all definition formats (\texttt{*},
\texttt{some-package-name}, and \texttt{some-package-name@version})
shown above are required. In most cases, the wildcard definition
(\texttt{*}) is enough. The non-wildcard formats
(\texttt{some-package-name} and \texttt{some-package-name@version}) are
rare exceptions for packages that require a more specific configuration
than the wildcard definition provides.

\noindent\hrulefill

\section{Standard Configuration
Options}\label{standard-configuration-options}

Below are the standard configuration options for the
\texttt{.npmbundlerrc} file:

\texttt{config}: Defines the global configuration that is made available
to all liferay-npm-bundler and Babel plugins. Please refer to each
plugin's documentation to find the available options for each specific
plugin.

\begin{verbatim}
{
  "config": {
    "imports": {
      "vuejs-provider": {
        "vue": "^2.0.0"
      }
    }
  }
}
\end{verbatim}

\texttt{dump-report:} Sets whether to generate a debugging report. If
\texttt{true}, a \texttt{liferay-npm-bundler-report.html} file is
generated in the project directory that describes all actions and
decisions taken when processing project and npm modules. Note that you
can also pass this as the build flag
\texttt{\$\ liferay-npm-bundler\ -\/-dump-report} or
\texttt{\$\ liferay-npm-bundler\ -r}. The default value is
\texttt{false}.

\texttt{no-tracking:} whether to send usage analytics to our servers.
Note that you can also pass this as a build flag with the CLI argument
\texttt{\$\ liferay-npm-bundler\ -\/-no-tracking}, or by creating a
marker file called \texttt{.liferay-npm-bundler-no-tracking} in the
project's root folder or any of its ancestors, or by setting the
environment variable
\texttt{LIFERAY\_NPM\_BUNDLER\_NO\_TRACKING=\textquotesingle{}\textquotesingle{}}.
The default value is \texttt{false}.

\texttt{output:} by default the bundler writes packages to the standard
Gradle resources folder:
\texttt{build/resources/main/META-INF/resources}. Set this value to
override the default output folder. Note that the dependency npm
packages are placed in a \texttt{node\_modules} folder inside the build
folder. Note if \texttt{create-jar} is set, the default output folder is
\texttt{build}.

\texttt{preset:} specifies the \texttt{liferay-npm-bundler} preset to
use as a base configuration. Note that if a \texttt{.npmbundlerrc} file
is not provided, the default
\texttt{liferay-npm-bundler-preset-standard} preset is used. All
settings provided by the preset are inherited, but they can be
overridden.

\texttt{verbose:} Sets whether to output detailed information about what
the tool is doing to the console. The default value is \texttt{false}.

\section{Package Processing Options}\label{package-processing-options}

\texttt{"/"}: plugins' configuration for the project's package.

\texttt{"\textbackslash{}"}: plugins' configuration for dependency
packages.

\emph{(asterisk)}: Defines the default plugin configuration for all npm
packages. It contains four values identified by a corresponding key.
Keys \texttt{copy-plugins}, \texttt{plugins} and \texttt{post-plugins}
identify arrays of \texttt{liferay-npm-bundler} plugins to apply in the
copy, pre and post process steps. Key \texttt{.babelrc} identifies an
object specifying the configuration to use in the Babel step and has the
same structure of a standard \texttt{.babelrc} file.

\texttt{exclude:} defines glob expressions of files to exclude from
bundling from all or specific packages. Each list is an array identified
by one of the following keys: \texttt{*} (any package),
\texttt{\{package\ name\}} (any version of the package), or
\texttt{\{package\ name\}@\{version\}} (a specific version of a
package). Below is an example configuration:

\begin{verbatim}
{
  "exclude": {
    "*": ["__tests__/**/*"],
    "is-object": ["test/**/*"],
    "is-array@1.0.1": ["test/**/*", "Makefile"]
  }
}
\end{verbatim}

\texttt{ignore:} skips processing the specified JavaScript files with
Babel for the project. An example configuration is shown below:

\begin{verbatim}
{
  "ignore": ["lib/legacy/**/*.js"]
}
\end{verbatim}

\texttt{include-dependencies:} defines packages to include in bundling,
even if they are not listed under the \texttt{dependencies} section of
\texttt{package.json}. These packages must be available in the
\texttt{node\_modules} folder (i.e.~installed manually, without saving
them to \texttt{package.json}, or listed in the \texttt{devDependencies}
section).

\texttt{packages:} defines plugin configuration for npm packages, per
package.

\texttt{max-parallel-files:} Defines the maximum number of files to
process in parallel to avoid EMFILE errors (especially on Windows). The
default value is \texttt{128}.

\texttt{process-serially:} \textbf{Note}: removed since v 2.7.0.
Replaced with \texttt{max-parallel-files}.

\texttt{rules:} defines rules to apply to the projects source files with
the loader. Rules must have a \texttt{use} array property that defines
the loader to use, which can be specified using a package name or an
object with \texttt{loader} and \texttt{options} properties if
applicable, and one or more of the properties below:

\begin{itemize}
\tightlist
\item
  \texttt{test}: defines a regular expression to filter files in the
  \texttt{sources} folders to determine whether to apply rules to them.
  The project-relative path of each eligible file is compared against
  the regular expression and files that match are processed by the
  loaders.
\item
  \texttt{exclude}: refines the \texttt{test} expression by specifying
  files to exclude.
\item
  \texttt{include}: refines the \texttt{test} expression by specifying
  files to include.
\end{itemize}

Here's an example configuration:

\begin{verbatim}
{
  "rules": [
    {
      "test": "\\.js$",
      "exclude": "node_modules",
      "use": [
        {
          "loader": "babel-loader",
          "options": {
            "presets": ["env", "react"]
          }
        }
      ]
    },
    {
      "test": "\\.css$",
      "use": ["style-loader"]
    },
    {
      "test": "\\.json$",
      "use": ["json-loader"]
    }
  ]
}
\end{verbatim}

\texttt{sources:} rules apply to files in these project folders. Folders
can be nested (e.g.~\texttt{/src/main/resources/}) and must be written
using POSIX path separators (i.e.~use \texttt{/} instead of
\texttt{\textbackslash{}} on Win32 systems). Note that rules are
automatically applied to package dependency files of the project.

An example configuration is shown below:

\begin{verbatim}
{
  "sources": ["src", "assets"]
}
\end{verbatim}

\section{OSGi Bundle Creation
Options}\label{osgi-bundle-creation-options}

Since version 2.2.0, the liferay-npm-bundler can create widget OSGi
bundles for you. See
\href{/docs/7-2/frameworks/-/knowledge_base/f/creating-and-bundling-javascript-widgets-with-javascript-tooling}{Creating
and Bundling JavaScript Widgets with JavaScript Tooling} for complete
instructions. The configuration options for OSGi bundle creation are
shown below:

\begin{itemize}
\tightlist
\item
  \textbf{create-jar}: Creates an OSGi bundle when set to a truthy
  value. When set to \texttt{true}, all sub-options take default values.
  When an object is passed, each sub-option can be configured
  individually. Note that you can also pass this as a build flag:
  \texttt{\$\ liferay-npm-bundler\ -\/-create-} or
  \texttt{\$\ liferay-npm-bundler\ -j}. The default value is
  \texttt{false}.
\end{itemize}

\begin{verbatim}
{
  "create-jar": true
}
\end{verbatim}

\begin{itemize}
\item
  \textbf{create-jar.auto-deploy-portlet}: \textbf{Note} that this
  option is deprecated. Use the \texttt{create-jar.features.js-extender}
  option instead.
\item
  \textbf{create-jar.features.configuration}: specifies the file
  describing the system (OSGi) and widget instance (widget preferences,
  as defined in the Portlet spec) configuration to use. (see
  \href{/docs/7-2/frameworks/-/knowledge_base/f/configuring-system-settings-and-instance-settings-for-your-js-widget}{Configuring
  System Settings and Instance Settings for Your JavaScript Widgets} for
  more information on the required settings configuration). The default
  value is \texttt{features/configuration.json} if that file exists,
  otherwise the default is \texttt{undefined}.
\end{itemize}

\begin{verbatim}
{
  "create-jar": {
    "features": {
      "configuration": "features/configuration.json"
    }
  }
}
\end{verbatim}

\begin{itemize}
\tightlist
\item
  \textbf{create-jar.output-dir:} specifies where to place the final JAR
\end{itemize}

\begin{verbatim}
{
  "create-jar": {
    "output-dir": "dist"
  }
}
\end{verbatim}

\begin{itemize}
\tightlist
\item
  \textbf{create-jar.features.js-extender:} controls whether to process
  the OSGi bundle with the JS Portlet Extender. You can also specify the
  minimum required version of the Extender to use for the bundle. This
  can be useful if you want to use advanced features in your bundle, but
  you want it to be deployable in older versions of the Extender. Pass
  the string \texttt{"any"} to let the bundle deploy in any version of
  the Extender. If \texttt{true}, the liferay-npm-bundler automatically
  determines the minimum version of the Extender required for the
  features used in the bundle. the default value is \texttt{true}. An
  example configuration is shown below:
\end{itemize}

\begin{verbatim}
{
  "create-jar": {
    "features": {
      "js-extender": "1.1.0"
    }
  }
}
\end{verbatim}

\begin{itemize}
\tightlist
\item
  \textbf{create-jar.features.web-context:} specifies the context path
  to use for publishing bundle's static resources. The default value is
  \texttt{/\{project\ name\}-\{project\ version\}}.
\end{itemize}

\begin{verbatim}
{
  "create-jar": {
    "features": {
      "web-context": "/my-project"
    }
  }
}
\end{verbatim}

\begin{itemize}
\tightlist
\item
  \textbf{create-jar.features.localization:} specifies the L10N file to
  use for the bundle (see
  \href{/docs/7-2/frameworks/-/knowledge_base/f/localizing-your-widget}{Providing
  Localization in Your JavaScript Widgets} for more information on using
  localization in your widget. The default value is
  \texttt{features/localization/Language} if a properties file with that
  base name exists, otherwise the default is \texttt{undefined}.
\end{itemize}

\begin{verbatim}
{
  "create-jar": {
    "features": {
      "localization": "features/localization/Language"
    }
  }
}
\end{verbatim}

\begin{itemize}
\tightlist
\item
  \textbf{create-jar.features.settings:} \textbf{Note} that this option
  is deprecated. Use the \texttt{create-jar.features.configuration}
  option instead.
\end{itemize}

\noindent\hrulefill

\textbf{Note:} Plugins' configuration specifies the options for
configuring plugins in all the possible phases, as well as the
\texttt{.babelrc} file to use when running Babel (see
\href{https://babeljs.io/docs/usage/babelrc/}{Babel's documentation} for
more information on that file format).

\noindent\hrulefill

\noindent\hrulefill

\textbf{Note:} Prior to version 1.4.0 of the liferay-npm-bundler,
package configurations were placed next to the tools options
(\texttt{*}, \texttt{output}, \texttt{exclude}, etc.) To prevent package
name collisions, package configurations are now namespaced and placed
under the \texttt{packages} section. To maintain backwards
compatibility, the liferay-npm-bundler falls back to the root section
outside \texttt{packages} for package configuration, if no package
configurations (\texttt{package-name@version}, \texttt{package-name}, or
\texttt{*}) are found in the \texttt{packages} section.

\noindent\hrulefill

Now you know the structure of the \texttt{.npmbundlerrc} file!

\chapter{How the Default Preset Configures the
liferay-npm-bundler}\label{how-the-default-preset-configures-the-liferay-npm-bundler}

The liferay-npm-bundler comes with a default configuration preset:
\href{https://github.com/liferay/liferay-npm-build-tools/tree/master/packages/liferay-npm-bundler-preset-standard}{\texttt{liferay-npm-bundler-preset-standard}}
in your \texttt{.npmbundlerrc} file. This preset configures several
plugins for the build process and is automatically used (even if the
\texttt{.npmbundlerrc} is missing), unless you override it with one of
your own. Running the liferay-npm-bundler with this preset applies the
\href{https://github.com/liferay/liferay-npm-build-tools/blob/master/packages/liferay-npm-bundler-preset-standard/config.json}{config
file} from \texttt{liferay-npm-bundler-preset-standard}:

\begin{verbatim}
{
  "/": {
    "plugins": ["resolve-linked-dependencies"],
    ".babelrc": {
      "presets": ["liferay-standard"]
    },
    "post-plugins": ["namespace-packages", "inject-imports-dependencies"]
  },
  "*": {
    "copy-plugins": ["exclude-imports"],
    "plugins": ["replace-browser-modules"],
    ".babelrc": {
      "presets": ["liferay-standard"]
    },
    "post-plugins": [
      "namespace-packages",
      "inject-imports-dependencies",
      "inject-peer-dependencies"
    ]
  }
}
\end{verbatim}

The configuration above states that for all npm packages (\texttt{*})
the pre-process phase (\texttt{plugins}) must run the
\texttt{replace-browser-modules} plugin. Setting this to
\texttt{post-plugins} would run it during the post phase instead.

\noindent\hrulefill

\textbf{Note:} You can override configuration preset values by adding
your own configuration to your project's \texttt{.npmbundlerrc} file.
For instance, using the configuration preset example above, you can
define your own \texttt{.babelrc} value in \texttt{.npmbundlerrc} file
to override the defined ``liferay-standard'' babelrc preset.

\noindent\hrulefill

The
\href{https://github.com/liferay/liferay-npm-build-tools/tree/master/packages/babel-preset-liferay-standard}{\texttt{liferay-standard}
preset} applies the following plugins to packages:

\begin{itemize}
\item
  \href{https://github.com/liferay/liferay-npm-build-tools/tree/master/packages/liferay-npm-bundler-plugin-exclude-imports}{exclude-imports}:
  Exclude packages declared in the \texttt{imports} section from the
  build.
\item
  \href{https://github.com/liferay/liferay-npm-build-tools/tree/master/packages/liferay-npm-bundler-plugin-inject-imports-dependencies}{inject-imports-dependencies}:
  Inject dependencies declared in the \texttt{imports} section in the
  dependencies' \texttt{package.json} files.
\item
  \href{https://github.com/liferay/liferay-npm-build-tools/tree/master/packages/liferay-npm-bundler-plugin-inject-peer-dependencies}{inject-peer-dependencies}:
  Inject declared peer dependencies (as they are resolved in the
  project's \texttt{node\_modules} folder) in the dependencies'
  \texttt{package.json} files.
\item
  \href{https://github.com/liferay/liferay-npm-build-tools/tree/master/packages/liferay-npm-bundler-plugin-namespace-packages}{namespace-packages}:
  Namespace package names based on the root project's package name to
  isolate packages per project and avoid collisions. This prepends
  \texttt{\textless{}project-package-name\textgreater{}\$} to each
  package name appearance in \texttt{package.json} files.
\item
  \href{https://github.com/liferay/liferay-npm-build-tools/tree/master/packages/liferay-npm-bundler-plugin-replace-browser-modules}{replace-browser-modules}:
  Replaces the server side files for modules listed under
  \texttt{browser}/\texttt{unpkg}/\texttt{jsdelivr} section of
  \texttt{package.json} with their browser counterparts.
\item
  \href{https://github.com/liferay/liferay-npm-build-tools/tree/master/packages/liferay-npm-bundler-plugin-resolve-linked-dependencies}{resolve-linked-dependencies}:
  Replace linked dependencies versions appearing in
  \texttt{package.json} files (those obtained from local file system or
  GitHub, for example) by their real version number, as resolved in the
  project's \texttt{node\_modules} directory.
\end{itemize}

In addition, the bundler runs Babel with the
\href{https://github.com/liferay/liferay-npm-build-tools/tree/master/packages/babel-preset-liferay-standard}{babel-preset-liferay-standard}
preset, that invokes the following plugins:

\begin{itemize}
\item
  \href{https://github.com/liferay/liferay-npm-build-tools/tree/master/packages/babel-plugin-normalize-requires}{babel-plugin-normalize-requires}:
  Normalize AMD \texttt{require()} calls.
\item
  \href{https://github.com/babel/minify/tree/master/packages/babel-plugin-transform-node-env-inline}{babel-plugin-transform-node-env-inline}:
  Inline the \texttt{NODE\_ENV} environment variable, and if it's part
  of a binary expression (eg.
  \texttt{process.env.NODE\_ENV\ ===\ "development"}), then statically
  evaluate and replace it.
\item
  \href{https://www.npmjs.com/package/babel-plugin-minify-dead-code-elimination}{babel-plugin-minify-dead-code-elimination}:
  Inline bindings when possible. Tries to evaluate expressions and
  prunes unreachable as a result.
\item
  \href{https://github.com/liferay/liferay-npm-build-tools/tree/master/packages/babel-plugin-wrap-modules-amd}{babel-plugin-wrap-modules-amd}:
  Wrap modules inside an AMD \texttt{define()} module.
\item
  \href{https://github.com/liferay/liferay-npm-build-tools/tree/master/packages/babel-plugin-name-amd-modules}{babel-plugin-name-amd-modules}:
  Name AMD modules based on package name, version, and module path.
\item
  \href{https://github.com/liferay/liferay-npm-build-tools/tree/master/packages/babel-plugin-namespace-modules}{babel-plugin-namespace-modules}:
  Namespace modules based on the root project's package name, prepending
  \texttt{\textless{}project-package-name\textgreater{}\$}. Wrap modules
  inside an AMD \texttt{define()} module for each module name appearance
  (in \texttt{define()} or \texttt{require()} calls) so that the
  packages are localized per project and don't clash.
\item
  \href{https://github.com/liferay/liferay-npm-build-tools/tree/master/packages/babel-plugin-namespace-amd-define}{babel-plugin-namespace-amd-define}:
  Add a prefix to AMD \texttt{define()} calls (by default
  \texttt{Liferay.Loader.}).
\end{itemize}

Now you know the available configuration presets for
\texttt{.npmbundlerrc} and how they work.

\chapter{The Structure of OSGi Bundles Containing npm
Packages}\label{the-structure-of-osgi-bundles-containing-npm-packages}

To deploy JavaScript modules, you must create an OSGi bundle with the
npm dependencies extracted from the project's \texttt{node\_modules}
folder and modify them to work with the
\href{https://github.com/liferay/liferay-amd-loader}{Liferay AMD
Loader}. The liferay-npm-bundler automates this process for you,
creating a bundle similar to the one below:

\begin{itemize}
\tightlist
\item
  \texttt{my-bundle/}

  \begin{itemize}
  \tightlist
  \item
    \texttt{META-INF/}

    \begin{itemize}
    \tightlist
    \item
      \texttt{resources/}

      \begin{itemize}
      \tightlist
      \item
        \texttt{package.json}

        \begin{itemize}
        \tightlist
        \item
          name: my-bundle-package
        \item
          version: 1.0.0
        \item
          main: lib/index
        \item
          dependencies:

          \begin{itemize}
          \tightlist
          \item
            my-bundle-package\$isarray: 2.0.0
          \item
            my-bundle-package\$isobject: 2.1.0
          \end{itemize}
        \item
          \ldots{}
        \end{itemize}
      \item
        \texttt{lib/}

        \begin{itemize}
        \tightlist
        \item
          \texttt{index.js}
        \item
          \ldots{}
        \end{itemize}
      \item
        \ldots{}
      \item
        \texttt{node\_modules/}

        \begin{itemize}
        \tightlist
        \item
          \texttt{my-bundle-package\$isobject@2.1.0/}

          \begin{itemize}
          \tightlist
          \item
            \texttt{package.json}

            \begin{itemize}
            \tightlist
            \item
              name: my-bundle-package\$isobject
            \item
              version: 2.1.0
            \item
              main: lib/index
            \item
              dependencies:

              \begin{itemize}
              \tightlist
              \item
                my-bundle-package\$isarray: 1.0.0
              \end{itemize}
            \item
              \ldots{}
            \end{itemize}
          \item
            \ldots{}
          \end{itemize}
        \item
          \texttt{my-bundle-package\$isarray@1.0.0/}

          \begin{itemize}
          \tightlist
          \item
            \texttt{package.json}

            \begin{itemize}
            \tightlist
            \item
              name: my-bundle-package\$isarray
            \item
              version: 1.0.0
            \item
              \ldots{}
            \end{itemize}
          \item
            \ldots{}
          \end{itemize}
        \item
          \texttt{my-bundle-package\$isarray@2.0.0/}

          \begin{itemize}
          \tightlist
          \item
            \texttt{package.json}

            \begin{itemize}
            \tightlist
            \item
              name: my-bundle-package\$isarray
            \item
              version: 2.0.0
            \item
              \ldots{}
            \end{itemize}
          \item
            \ldots{}
          \end{itemize}
        \end{itemize}
      \end{itemize}
    \end{itemize}
  \end{itemize}
\end{itemize}

The packages inside \texttt{node\_modules} are the same format as the
npm tool and can be copied (after a little processing for things like
converting to AMD, for example) from a standard \texttt{node\_modules}
folder. The \texttt{node\_modules} folder can hold any number of npm
packages (even different versions of the same package), or no npm
packages at all.

Now that you know the structure for OSGi bundles containing npm
packages, you can learn how the liferay-npm-bundler handles inline
JavaScript packages.

\section{Inline JavaScript packages}\label{inline-javascript-packages}

The resulting OSGi bundle that the liferay-npm-bundler creates lets you
deploy one inline JavaScript package (named \texttt{my-bundle-package}
in the example) with several npm packages that are placed inside the
\texttt{node\_modules} folder, one package per folder.

The inline package is nested in the OSGi standard
\texttt{META-INF/resources} folder and is defined by a standard npm
\texttt{package.json} file.

The inline package is optional, but only one inline package is allowed
per OSGi bundle. The inline package usually provides the JavaScript code
for a widget, when the OSGi bundle contains one. Note that the
architecture does not differentiate between inline and npm packages once
they are published. The inline package is only used for organizational
purposes.

Now you know how the liferay-npm-bundler creates OSGi bundles for npm
packages!

\chapter{How the Liferay npm Bundler Publishes npm
Packages}\label{how-the-liferay-npm-bundler-publishes-npm-packages}

When you deploy an OSGi bundle with the specified structure, as
explained in
\href{/docs/7-2/reference/-/knowledge_base/r/the-structure-of-osgi-bundles-containing-npm-packages}{The
Structure of OSGi Bundles Containing NPM Packages} reference, its
modules are made available for consumption through canonical URLs. To
better illustrate resolved modules, the example structure below is the
standard structure that the liferay-npm-bundler 1.x generates, and
therefore doesn't have the namespaced packages that the 2.x version
generates. Please refer to the last sections of this article to know how
liferay-npm-bundler 2.0 overrides this de-duplication mechanism to
implement isolated dependencies and imports.

\begin{itemize}
\tightlist
\item
  \texttt{my-bundle/}

  \begin{itemize}
  \tightlist
  \item
    \texttt{META-INF/}

    \begin{itemize}
    \tightlist
    \item
      \texttt{resources/}

      \begin{itemize}
      \tightlist
      \item
        \texttt{package.json}

        \begin{itemize}
        \tightlist
        \item
          name: my-bundle-package
        \item
          version: 1.0.0
        \item
          main: lib/index
        \item
          dependencies:

          \begin{itemize}
          \tightlist
          \item
            isarray: 2.0.0
          \item
            isobject: 2.1.0
          \end{itemize}
        \item
          \ldots{}
        \end{itemize}
      \item
        \texttt{lib/}

        \begin{itemize}
        \tightlist
        \item
          \texttt{index.js}
        \item
          \ldots{}
        \end{itemize}
      \item
        \ldots{}
      \item
        \texttt{node\_modules/}

        \begin{itemize}
        \tightlist
        \item
          \texttt{isobject@2.1.0/}

          \begin{itemize}
          \tightlist
          \item
            \texttt{package.json}

            \begin{itemize}
            \tightlist
            \item
              name: isobject
            \item
              version: 2.1.0
            \item
              main: lib/index
            \item
              dependencies:

              \begin{itemize}
              \tightlist
              \item
                isarray: 1.0.0
              \end{itemize}
            \item
              \ldots{}
            \end{itemize}
          \item
            \ldots{}
          \end{itemize}
        \item
          \texttt{isarray@1.0.0/}

          \begin{itemize}
          \tightlist
          \item
            \texttt{package.json}

            \begin{itemize}
            \tightlist
            \item
              name: isarray
            \item
              version: 1.0.0
            \item
              \ldots{}
            \end{itemize}
          \item
            \ldots{}
          \end{itemize}
        \item
          \texttt{isarray@2.0.0/}

          \begin{itemize}
          \tightlist
          \item
            \texttt{package.json}

            \begin{itemize}
            \tightlist
            \item
              name: isarray
            \item
              version: 2.0.0
            \item
              \ldots{}
            \end{itemize}
          \item
            \ldots{}
          \end{itemize}
        \end{itemize}
      \end{itemize}
    \end{itemize}
  \end{itemize}
\end{itemize}

If you deploy the example OSGi bundle shown above, the following URLs
are made available (one for each module):

\begin{itemize}
\item
  \url{http://localhost/o/js/module/598/my-bundle-package@1.0.0/lib/index.js}
\item
  \url{http://localhost/o/js/module/598/isobject@2.1.0/index.js}
\item
  \url{http://localhost/o/js/module/598/isarray@1.0.0/index.js}
\item
  \url{http://localhost/o/js/module/598/isarray@2.0.0/index.js}
\end{itemize}

\noindent\hrulefill

\textbf{NOTE:} The OSGi bundle ID (598) may vary.

\noindent\hrulefill

You can learn about package de-duplication next.

\section{Package De-duplication}\label{package-de-duplication}

Since two or more OSGi modules may export multiple copies of the same
package and version, Liferay Portal must de-duplicate such collisions.
To accomplish de-duplication, a new concept called \emph{resolved
module} was created.

A resolved module is the reference package exported to Liferay Portal's
front-end, when multiple copies of the same package and version exist.
It's randomly referenced from one of the several bundles exporting the
same copies of the package.

Using the example from the previous section, for each group of canonical
URLs referring to the same module inside different OSGi bundles, there's
another canonical URL for the resolved module. The example structure has
the resolved module URLs shown below:

\begin{itemize}
\item
  \url{http://localhost/o/js/resolved-module/my-bundle-package@1.0.0/lib/index.js}
\item
  {[}http://localhost/o/js/resolved-module/my-bundle-package\(isobject@2.1.0/index.js](http://localhost/o/js/resolved-module/my-bundle-package\)isobject@2.1.0/index.js)
\item
  {[}http://localhost/o/js/resolved-module/my-bundle-package\(isarray@1.0.0/index.js](http://localhost/o/js/resolved-module/my-bundle-package\)isarray@1.0.0/index.js)
\item
  {[}http://localhost/o/js/resolved-module/my-bundle-package\(isarray@2.0.0/index.js](http://localhost/o/js/resolved-module/my-bundle-package\)isarray@2.0.0/index.js)
\end{itemize}

\noindent\hrulefill

\textbf{NOTE:} The OSGi bundle ID (598 in the example) is removed and
module is replaced by \texttt{resolved-module}.

\noindent\hrulefill

Next you can learn how the bundler (since version 2.0.0) isolates
package dependencies. See
\href{/docs/7-2/reference/-/knowledge_base/r/what-changed-between-liferay-npm-bundler-1-x-and-2-x}{What
Changed Between liferay-npm-bundler 1.x and 2.x} for more information on
why this change was made.

\section{Isolated Package
Dependencies}\label{isolated-package-dependencies}

A typical OSGi bundle structure generated with liferay-npm-bundler 2.x
is shown below:

\begin{itemize}
\tightlist
\item
  \texttt{my-bundle/}

  \begin{itemize}
  \tightlist
  \item
    \texttt{META-INF/}

    \begin{itemize}
    \tightlist
    \item
      \texttt{resources/}

      \begin{itemize}
      \tightlist
      \item
        \texttt{package.json}

        \begin{itemize}
        \tightlist
        \item
          name: my-bundle-package
        \item
          version: 1.0.0
        \item
          main: lib/index
        \item
          dependencies:

          \begin{itemize}
          \tightlist
          \item
            my-bundle-package\$isarray: 2.0.0
          \item
            my-bundle-package\$isobject: 2.1.0
          \end{itemize}
        \item
          \ldots{}
        \end{itemize}
      \item
        \texttt{lib/}

        \begin{itemize}
        \tightlist
        \item
          \texttt{index.js}
        \item
          \ldots{}
        \end{itemize}
      \item
        \ldots{}
      \item
        \texttt{node\_modules/}

        \begin{itemize}
        \tightlist
        \item
          \texttt{my-bundle-package\$isobject@2.1.0/}

          \begin{itemize}
          \tightlist
          \item
            \texttt{package.json}

            \begin{itemize}
            \tightlist
            \item
              name: my-bundle-package\$isobject
            \item
              version: 2.1.0
            \item
              main: lib/index
            \item
              dependencies:

              \begin{itemize}
              \tightlist
              \item
                my-bundle-package\$isarray: 1.0.0
              \end{itemize}
            \item
              \ldots{}
            \end{itemize}
          \item
            \ldots{}
          \end{itemize}
        \item
          \texttt{my-bundle-package\$isarray@1.0.0/}

          \begin{itemize}
          \tightlist
          \item
            \texttt{package.json}

            \begin{itemize}
            \tightlist
            \item
              name: my-bundle-package\$isarray
            \item
              version: 1.0.0
            \item
              \ldots{}
            \end{itemize}
          \item
            \ldots{}
          \end{itemize}
        \item
          \texttt{my-bundle-package\$isarray@2.0.0/}

          \begin{itemize}
          \tightlist
          \item
            \texttt{package.json}

            \begin{itemize}
            \tightlist
            \item
              name: my-bundle-package\$isarray
            \item
              version: 2.0.0
            \item
              \ldots{}
            \end{itemize}
          \item
            \ldots{}
          \end{itemize}
        \end{itemize}
      \end{itemize}
    \end{itemize}
  \end{itemize}
\end{itemize}

Note that each package dependency is namespaced with the bundle's name
(\texttt{my-bundle-package\$} in the example structure). This lets each
project load its own dependencies and avoid potential collisions with
projects that export the same package. For example, consider the two
widget projects below:

\begin{itemize}
\tightlist
\item
  \texttt{my-widget}

  \begin{itemize}
  \tightlist
  \item
    package.json

    \begin{itemize}
    \tightlist
    \item
      dependencies:

      \begin{itemize}
      \tightlist
      \item
        a-library 1.0.0
      \item
        a-helper 1.0.0
      \end{itemize}
    \end{itemize}
  \item
    node\_modules

    \begin{itemize}
    \tightlist
    \item
      a-library

      \begin{itemize}
      \tightlist
      \item
        version: 1.0.0
      \item
        dependencies:

        \begin{itemize}
        \tightlist
        \item
          a-helper \^{}1.0.0
        \end{itemize}
      \end{itemize}
    \item
      a-helper

      \begin{itemize}
      \tightlist
      \item
        version: 1.0.0
      \end{itemize}
    \end{itemize}
  \end{itemize}
\item
  \texttt{another-widget}

  \begin{itemize}
  \tightlist
  \item
    package.json

    \begin{itemize}
    \tightlist
    \item
      dependencies:

      \begin{itemize}
      \tightlist
      \item
        a-library 1.0.0
      \item
        a-helper 1.2.0
      \end{itemize}
    \end{itemize}
  \item
    node\_modules

    \begin{itemize}
    \tightlist
    \item
      a-library

      \begin{itemize}
      \tightlist
      \item
        version: 1.0.0
      \item
        dependencies:

        \begin{itemize}
        \tightlist
        \item
          a-helper \^{}1.0.0
        \end{itemize}
      \end{itemize}
    \item
      a-helper

      \begin{itemize}
      \tightlist
      \item
        version: 1.2.0
      \end{itemize}
    \end{itemize}
  \end{itemize}
\end{itemize}

In this example, \texttt{a-library} depends on \texttt{a-helper} at
version 1.0.0 or higher (note the caret \^{} expression in the
dependencies). The bundler implements isolated dependencies by prefixing
the name of the bundle to the modules, so that \texttt{my-widget} gets
its \texttt{a-helper} at 1.0.0, while \texttt{another-widget} gets its
\texttt{a-helper} at 1.2.0.

The dependencies isolation not only avoids collisions between bundles,
but also makes peer dependencies behave deterministically as each widget
gets what it had in its \texttt{node\_modules} folder when it was
developed.

Now that you understand how namespacing modules isolates bundle
dependencies, avoiding collisions, you can learn about de-duplication
next.

\section{De-duplication through
Importing}\label{de-duplication-through-importing}

Isolated dependencies are very useful, but there are times when sharing
the same package between modules would be more beneficial. To do this,
the liferay-npm-bundler lets you import packages from an external OSGi
bundle, instead of using your own. This lets you put shared dependencies
in one project and reference them from the rest.

Imagine that you have three widgets that compose the homepage of your
site: \texttt{my-toolbar}, \texttt{my-menu}, and \texttt{my-content}.
These widgets depend on the fake, but awesome, Wonderful UI Components
(WUI) framework. This quite limited framework is composed of only three
packages:

\begin{enumerate}
\def\labelenumi{\arabic{enumi}.}
\tightlist
\item
  \texttt{component-core}
\item
  \texttt{button}
\item
  \texttt{textfield}
\end{enumerate}

Since the bundler namespaces each dependency package with the widget's
name by default, you would end up with three namespaced copies of the
WUI package on the page. This is not what you want. Since they share the
same package, instead you can create a fourth bundle that contains the
WUI package, and import the WUI package in the three widgets. This
results in the structure below:

\begin{itemize}
\tightlist
\item
  \texttt{my-toolbar/}

  \begin{itemize}
  \tightlist
  \item
    \texttt{.npmbundlerrc}

    \begin{itemize}
    \tightlist
    \item
      config:

      \begin{itemize}
      \tightlist
      \item
        imports:

        \begin{itemize}
        \tightlist
        \item
          wui-provider:

          \begin{itemize}
          \tightlist
          \item
            component-core: \^{}1.0.0
          \item
            button: \^{}1.0.0
          \item
            textfield: \^{}1.0.0
          \end{itemize}
        \end{itemize}
      \end{itemize}
    \end{itemize}
  \end{itemize}
\item
  \texttt{my-menu/}

  \begin{itemize}
  \tightlist
  \item
    \texttt{.npmbundlerrc}

    \begin{itemize}
    \tightlist
    \item
      config:

      \begin{itemize}
      \tightlist
      \item
        imports:

        \begin{itemize}
        \tightlist
        \item
          wui-provider:

          \begin{itemize}
          \tightlist
          \item
            component-core: \^{}1.0.0
          \item
            button: \^{}1.0.0
          \item
            textfield: \^{}1.0.0
          \end{itemize}
        \end{itemize}
      \end{itemize}
    \end{itemize}
  \end{itemize}
\item
  \texttt{my-content/}

  \begin{itemize}
  \tightlist
  \item
    \texttt{.npmbundlerrc}

    \begin{itemize}
    \tightlist
    \item
      config:

      \begin{itemize}
      \tightlist
      \item
        imports:

        \begin{itemize}
        \tightlist
        \item
          wui-provider:

          \begin{itemize}
          \tightlist
          \item
            component-core: \^{}1.0.0
          \item
            button: \^{}1.0.0
          \item
            textfield: \^{}1.0.0
          \end{itemize}
        \end{itemize}
      \end{itemize}
    \end{itemize}
  \end{itemize}
\item
  \texttt{wui-provider/}

  \begin{itemize}
  \tightlist
  \item
    \texttt{.package.json}

    \begin{itemize}
    \tightlist
    \item
      name: wui-provider
    \item
      dependencies:

      \begin{itemize}
      \tightlist
      \item
        component-core: 1.0.0
      \item
        button: 1.0.0
      \item
        textfield: 1.0.0
      \end{itemize}
    \end{itemize}
  \end{itemize}
\end{itemize}

The bundler switches the namespace of certain packages, thus pointing
them to an external bundle. Say that you have the following code in
\texttt{my-toolbar} widget:

\begin{verbatim}
var Button = require('button');
\end{verbatim}

By default, the bundler 2.x transforms this into the following when not
imported from another bundle:

\begin{verbatim}
var Button = require('my-toolbar$button');
\end{verbatim}

But, because \texttt{button} is imported from \texttt{wui-provider}, it
is instead changed to the value below:

\begin{verbatim}
var Button = require('wui-provider$button');
\end{verbatim}

Also, a dependency on \texttt{wui-provider\$button} at version
\texttt{\^{}1.0.0} is included in \texttt{my-toolbar}'s
\texttt{package.json} file so that the loader finds the correct version.
That's all you need. Once \texttt{wui-provider\$button} is required at
runtime, it jumps to \texttt{wui-provider}'s context and loads the
subdependencies from there on, even if code is executed from
\texttt{my-toolbar}. This works because, as you can imagine,
\texttt{wui-provider}'s modules are namespaced too, and once you load a
module from it, it keeps requiring \texttt{wui-provider\$} prefixed
modules all the way down.

Next, you will learn possible strategies for importing.

\section{Strategies When Importing
Packages}\label{strategies-when-importing-packages}

De-duplication by importing is a powerful tool, but you must design a
versioning strategy suitable for you so that you don't run into errors.

First of all, you must decide if you want to declare imported
dependencies only in the \texttt{.npmbundlerrc} file or in the
\texttt{package.json} too. Listing an imported dependency in
\texttt{.npmbundlerrc} is enough, even if it isn't present in your
\texttt{node\_modules} folder because during runtime the loader will
find it. Listing an imported dependency in \texttt{.npmbundlerrc} is
enough, even if it isn't present in your \texttt{node\_modules} folder,
because during runtime the loader finds it. If you have previous
experience with dynamic linking support in standard operating systems,
think of it as a DLL or shared object.

You may need to install your dependencies in \texttt{node\_modules} too
if you use them for tests, or if they contain types needed to compile
(like in Typescript), etc. If that is the case, then you can place them
in the \texttt{dependencies} or \texttt{devDependencies} section of your
\texttt{package.json}. If you list them under the latter, they are
automatically excluded from the output bundle by the
liferay-npm-bundler. Otherwise, you need to exclude them in the
\texttt{.npmbundlerrc} file so they don't redundantly appear in the
output.

If you list dependencies both in \texttt{package.json} and
\texttt{.npmbundlerrc}, decide how to keep versions in sync. The best
advice is to use the same version constraints in both files, but you may
decide not to do so if it is necessary. For example, imagine that you
import one of your dependencies from another bundle during runtime to
run tests. Say you are using version constraint \^{}1.5.1. It would be
desirable that if you have tested your code with a version
\textgreater=1.5.1 and \textless2.0.0 (that's what \^{}1.5.1 means), you
get a compatible version during runtime. Thus, you would declare the
dependency with \^{}1.5.1 in \texttt{.npmbundlerrc} too.

However, there are times when you may want to be more lenient, and you
may need to get a lower version (1.4.0 for example) during runtime, even
if you are developing against \^{}1.5.1. In that case, you can declare
\^{}1.5.1 in your \texttt{package.json} and \^{}1.0.0 in
\texttt{.npmbundlerrc}.

In the end, it's up to you to decide how you want to handle your
dependencies:

\begin{enumerate}
\def\labelenumi{\arabic{enumi}.}
\item
  \texttt{package.json} (While developing)
\item
  \texttt{.npmbundlerrc} (During runtime)
\end{enumerate}

we recommend that you choose a versioning strategy and stick to it, to
ensure dependencies are satisfied at runtime.

\chapter{Understanding How liferay-npm-bundler Formats JavaScript
Modules for
AMD}\label{understanding-how-liferay-npm-bundler-formats-javascript-modules-for-amd}

Liferay AMD Loader is based on the
\href{https://github.com/amdjs/amdjs-api/wiki/AMD}{AMD specification}.
All modules inside an npm OSGi bundle must be in AMD format. This is
done for \href{http://www.commonjs.org/}{CommonJS} modules by wrapping
the module code inside a \texttt{define} call. The liferay-npm-bundler
helps automate this process by wrapping the module for you. This article
references the OSGi structure below as an example. You can learn more
about this structure in
\href{/docs/7-2/reference/-/knowledge_base/r/the-structure-of-osgi-bundles-containing-npm-packages}{The
Structure of OSGi Bundles Containing NPM Packages} reference.

\begin{itemize}
\tightlist
\item
  \texttt{my-bundle/}

  \begin{itemize}
  \tightlist
  \item
    \texttt{META-INF/}

    \begin{itemize}
    \tightlist
    \item
      \texttt{resources/}

      \begin{itemize}
      \tightlist
      \item
        \texttt{package.json}

        \begin{itemize}
        \tightlist
        \item
          name: my-bundle-package
        \item
          version: 1.0.0
        \item
          main: lib/index
        \item
          dependencies:

          \begin{itemize}
          \tightlist
          \item
            my-bundle-package\$isarray: 2.0.0
          \item
            my-bundle-package\$isobject: 2.1.0
          \end{itemize}
        \item
          \ldots{}
        \end{itemize}
      \item
        \texttt{lib/}

        \begin{itemize}
        \tightlist
        \item
          \texttt{index.js}
        \item
          \ldots{}
        \end{itemize}
      \item
        \ldots{}
      \item
        \texttt{node\_modules/}

        \begin{itemize}
        \tightlist
        \item
          \texttt{my-bundle-package\$isobject@2.1.0/}

          \begin{itemize}
          \tightlist
          \item
            \texttt{package.json}

            \begin{itemize}
            \tightlist
            \item
              name: my-bundle-package\$isobject
            \item
              version: 2.1.0
            \item
              main: lib/index
            \item
              dependencies:

              \begin{itemize}
              \tightlist
              \item
                my-bundle-package\$isarray: 1.0.0
              \end{itemize}
            \item
              \ldots{}
            \end{itemize}
          \item
            \ldots{}
          \end{itemize}
        \item
          \texttt{my-bundle-package\$isarray@1.0.0/}

          \begin{itemize}
          \tightlist
          \item
            \texttt{package.json}

            \begin{itemize}
            \tightlist
            \item
              name: my-bundle-package\$isarray
            \item
              version: 1.0.0
            \item
              \ldots{}
            \end{itemize}
          \item
            \ldots{}
          \end{itemize}
        \item
          \texttt{my-bundle-package\$isarray@2.0.0/}

          \begin{itemize}
          \tightlist
          \item
            \texttt{package.json}

            \begin{itemize}
            \tightlist
            \item
              name: my-bundle-package\$isarray
            \item
              version: 2.0.0
            \item
              \ldots{}
            \end{itemize}
          \item
            \ldots{}
          \end{itemize}
        \end{itemize}
      \end{itemize}
    \end{itemize}
  \end{itemize}
\end{itemize}

For example, the \texttt{my-bundle-package\$isobject@2.1.0} package's
\texttt{index.js} file contains the following code:

\begin{verbatim}
'use strict';

var isArray = require('my-bundle-package$isarray');

module.exports = function isObject(val) {
    return val != null && typeof val === 'object' && isArray(val) === false;
};
\end{verbatim}

The updated module code configured for AMD format is shown below:

\begin{verbatim}
define(
    'my-bundle-package$isobject@2.1.0/index', 
    ['module', 'require', 'my-bundle-package$isarray'], 
    function (module, require) {
        'use strict';

        var define = undefined;

        var isArray = require('my-bundle-package$isarray');

        module.exports = function isObject(val) {
            return val != null && typeof val === 'object' 
            && isArray(val) === false;
        };
    }
);
\end{verbatim}

\noindent\hrulefill

\textbf{Note:} The module's name must be based on its package, version,
and file path (for example
\texttt{my-bundle-package\$isobject@2.1.0/index}), otherwise Liferay AMD
Loader can't find it.

\noindent\hrulefill

Note the module's dependencies:
\texttt{{[}\textquotesingle{}module\textquotesingle{},\ \textquotesingle{}require\textquotesingle{},\ \textquotesingle{}my-bundle-package\$isarray\textquotesingle{}{]}}.

\texttt{module} and \texttt{require} must be used to get a reference to
the \texttt{module.exports} object and the local \texttt{require}
function, as defined in the AMD specification.

The subsequent dependencies state the modules on which this module
depends. Note that \texttt{my-bundle-package\$isarray} in the example is
not a package but rather an alias of the
\texttt{my-bundle-package\$isarray} package's main module (thus, it is
equivalent to \texttt{my-bundle-package\$isarray/index}).

Also note that there is enough information in the \texttt{package.json}
files to know that \texttt{my-bundle-package\$isarray} refers to
\texttt{my-bundle-package\$isarray/index}, but also that it must be
resolved to version \texttt{1.0.0} of such package, i.e., that
\texttt{my-bundle-package\$isarray/index} in this case refers to
\texttt{my-bundle-package\$isarray@1.0.0/index}.

You may also have noted the \texttt{var\ define\ =\ undefined;} addition
to the top of the file. This is introduced by
\texttt{liferay-npm-bundler} to make the module think that it is inside
a CommonJS environment (instead of an AMD one). This is because some npm
packages are written in UMD format and, because we are wrapping it
inside our AMD \texttt{define()} call, we don't want them to execute
their own \texttt{define()} but prefer them to take the CommonJS path,
where the exports are done through the \texttt{module.exports} global.

Now you have a better understanding of how liferay-npm-bundler formats
JavaScript modules for AMD!

\chapter{Understanding How Liferay AMD Loader Configuration is
Exported}\label{understanding-how-liferay-amd-loader-configuration-is-exported}

\noindent\hrulefill

\textbf{NOTE:} This article is for users who know how Liferay AMD Loader
works under the hood. See
\href{/docs/7-2/frameworks/-/knowledge_base/f/loading-amd-modules-in-liferay}{Liferay
AMD Module Loader} for more information.

\noindent\hrulefill

With
\href{/docs/7-2/reference/-/knowledge_base/r/how-the-liferay-npm-bundler-publishes-npm-packages\#package-de-duplication}{de-duplication}
in place, JavaScript modules are made available to Liferay AMD Loader
through the configuration returned by the
\texttt{/o/js\_loader\_modules} URL.

The OSGi bundle shown below is used for reference in this article:

\begin{itemize}
\tightlist
\item
  \texttt{my-bundle/}

  \begin{itemize}
  \tightlist
  \item
    \texttt{META-INF/}

    \begin{itemize}
    \tightlist
    \item
      \texttt{resources/}

      \begin{itemize}
      \tightlist
      \item
        \texttt{package.json}

        \begin{itemize}
        \tightlist
        \item
          name: my-bundle-package
        \item
          version: 1.0.0
        \item
          main: lib/index
        \item
          dependencies:

          \begin{itemize}
          \tightlist
          \item
            isarray: 2.0.0
          \item
            isobject: 2.1.0
          \end{itemize}
        \item
          \ldots{}
        \end{itemize}
      \item
        \texttt{lib/}

        \begin{itemize}
        \tightlist
        \item
          \texttt{index.js}
        \item
          \ldots{}
        \end{itemize}
      \item
        \ldots{}
      \item
        \texttt{node\_modules/}

        \begin{itemize}
        \tightlist
        \item
          \texttt{isobject@2.1.0/}

          \begin{itemize}
          \tightlist
          \item
            \texttt{package.json}

            \begin{itemize}
            \tightlist
            \item
              name: isobject
            \item
              version: 2.1.0
            \item
              main: lib/index
            \item
              dependencies:

              \begin{itemize}
              \tightlist
              \item
                isarray: 1.0.0
              \end{itemize}
            \item
              \ldots{}
            \end{itemize}
          \item
            \ldots{}
          \end{itemize}
        \item
          \texttt{isarray@1.0.0/}

          \begin{itemize}
          \tightlist
          \item
            \texttt{package.json}

            \begin{itemize}
            \tightlist
            \item
              name: isarray
            \item
              version: 1.0.0
            \item
              \ldots{}
            \end{itemize}
          \item
            \ldots{}
          \end{itemize}
        \item
          \texttt{isarray@2.0.0/}

          \begin{itemize}
          \tightlist
          \item
            \texttt{package.json}

            \begin{itemize}
            \tightlist
            \item
              name: isarray
            \item
              version: 2.0.0
            \item
              \ldots{}
            \end{itemize}
          \item
            \ldots{}
          \end{itemize}
        \end{itemize}
      \end{itemize}
    \end{itemize}
  \end{itemize}
\end{itemize}

For example, for the specified structure (shown above), as explained in
\href{/docs/7-2/reference/-/knowledge_base/r/the-structure-of-osgi-bundles-containing-npm-packages}{The
Structure of OSGi Bundles Containing npm Packages} reference, the
following configuration is published for Liferay AMD loader to consume:

\begin{verbatim}
Liferay.PATHS = {
  ...
  'my-bundle-package@1.0.0/lib/index': '/o/js/resolved-module/my-bundle-package@1.0.0/lib/index',
  'isobject@2.1.0/index': '/o/js/resolved-module/isobject@2.1.0/index',
  'isarray@1.0.0/index': '/o/js/resolved-module/isarray@1.0.0/index',
  'isarray@2.0.0/index': '/o/js/resolved-module/isarray@2.0.0/index',
  ...
}
Liferay.MODULES = {
  ...
  "my-bundle-package@1.0.0/lib/index.es": {
    "dependencies": ["exports", "isarray", "isobject"],
    "map": {
      "isarray": "isarray@2.0.0", 
      "isobject": "isobject@2.1.0"
    }
  },
  "isobject@2.1.0/index": {
    "dependencies": ["module", "require", "isarray"],
    "map": {
      "isarray": "isarray@1.0.0"
    }
  },
  "isarray@1.0.0/index": {
    "dependencies": ["module", "require"],
    "map": {}
  },
  "isarray@2.0.0/index": {
    "dependencies": ["module", "require"],
    "map": {}
  },
  ...
}
Liferay.MAPS = {
  ...
  'my-bundle-package@1.0.0': { value: 'my-bundle-package@1.0.0/lib/index', exactMatch: true}
  'isobject@2.1.0': { value: 'isobject@2.1.0/index', exactMatch: true},
  'isarray@2.0.0': { value: 'isarray@2.0.0/index', exactMatch: true},
  'isarray@1.0.0': { value: 'isarray@1.0.0/index', exactMatch: true},
  ...
}
\end{verbatim}

Note:

\begin{itemize}
\item
  The \texttt{Liferay.PATHS} property describes paths to the JavaScript
  module files.
\item
  The \texttt{Liferay.MODULES} property describes the dependency names
  and versions of each module.
\item
  The \texttt{Liferay.MAPS} property describes the aliases of the
  package's main modules.
\end{itemize}

\chapter{What Changed Between Liferay npm Bundler 1.x and
2.x}\label{what-changed-between-liferay-npm-bundler-1.x-and-2.x}

This reference doc outlines the key changes between liferay-npm-bundler
version 1.x and 2.x.

\section{Automatically Formatting Modules for
AMD}\label{automatically-formatting-modules-for-amd}

In version series 1.x of the bundler it was the developer's
responsibility to wrap project modules in an AMD \texttt{define()} call.
However, since 2.x the bundler does it for you, so the only requisite is
that the project's code is transpiled/written for CommonJS modules model
(the standard model for module handling in Node.js, that uses
\texttt{require()} calls to load modules).

\section{Isolating Project
Dependencies}\label{isolating-project-dependencies}

Package names are prefixed with the bundle name since version 2.0.0 of
the bundler, but were left intact in previous versions. This strategy is
used to isolate packages from different bundles. You can still deploy
bundler 1.x packages (without prefix), and they will still work as they
did for previous versions of the bundler.

\section{Improved Peer Dependency
Support}\label{improved-peer-dependency-support}

In bundler 1.x, there was only one shared peer dependency package
available between widgets. With isolated dependencies per widget, it's
easy to honor peer dependencies perfectly. Peer dependencies can be
resolved exactly as stated in projects because their names are prefixed
with the project's name. This is possible because of the new
\href{https://github.com/liferay/liferay-npm-build-tools/tree/master/packages/liferay-npm-bundler-plugin-inject-peer-dependencies}{liferay-npm-bundler-plugin-inject-peer-dependencies}
plugin. It scans all JS modules for \texttt{require} calls. If the
bundler finds a required package in the \texttt{main.js} file, but it is
not declared in the \texttt{package.json}, it resolves it to the proper
version that is found in the \texttt{node\_modules} folder. The plugin
then injects a new dependency in the output \texttt{package.json} for
the required package.

Note that injected dependency version constraints are the specific
version number required, without caret or any other semantic version
operator. This is to honor the exact peer dependency found in the
project. Injecting more relaxed semantic version expressions could lead
to unstable results.

\section{Manually De-duplicating Through
Importing}\label{manually-de-duplicating-through-importing}

Namespacing means that each widget gets its own dependencies. Only using
the bundler this way obtains the same functionality as standard bundlers
like webpack or Browserify, so you wouldn't need a specific tool like
liferay-npm-bundler. Since Liferay DXP is a widget based architecture,
sharing dependencies among different widgets would be very beneficial.

In bundler 1.x that deduplication was made automatically, but there was
no control over it. However, with version 2.x, you may now import
packages from an external OSGi bundle, instead of using your own. This
lets you put shared dependencies in one project, and reference them from
the rest. Though This new way of de-duplication is not automatic, it
leads to full control (during build time) of how each package is
resolved.

Now that you understand what changed between version 1.x and 2.x of the
liferay-npm-bundler, you can follow the steps in the
\href{/docs/7-2/frameworks/-/knowledge_base/f/migrating-a-liferay-npm-bundler-project-from-1-x-to-2-x}{Migrating
a liferay-npm-bundler Project from 1.x to 2.x} to migrate your 1.x
projects to 2.x.

\chapter{Understanding liferay-npm-bundler's
Loaders}\label{understanding-liferay-npm-bundlers-loaders}

liferay-npm-bundler's mechanism is inspired by webpack. Like webpack,
the liferay-npm-bundler processes files using a set of rules that
include loaders that transform a project's source files before producing
the final output.

\noindent\hrulefill

\textbf{Note:} While webpack creates a single JS bundle file,
liferay-npm-bundler targets an AMD loader, so webpack and
liferay-npm-bundler loaders are not compatible.

\noindent\hrulefill

Loaders are npm packages that export a function in their main module
that receives source files and returns modified files, and optionally
new files, based on the loader's configuration. For example, the
\href{https://github.com/liferay/liferay-js-toolkit/tree/master/packages/liferay-npm-bundler-loader-babel-loader}{babel-loader}
receives ES6+ JavaScript files, runs Babel on them, and returns
transpiled ES5 files along with a generated source map. You can use this
pattern to
\href{/docs/7-2/frameworks/-/knowledge_base/f/creating-custom-loaders-for-the-liferay-npm-bundler}{create
custom loaders}. A few example loader functions are shown below:

\begin{itemize}
\tightlist
\item
  Pass JavaScript files through Babel or TSC
\item
  Convert CSS files into JS modules that dynamically inject the CSS into
  the HTML page
\item
  Process CSS files with SASS
\item
  Create tools that generate code based on
  \href{https://en.wikipedia.org/wiki/Interface_description_language}{IDL}
  files
\end{itemize}

Loaders are configured via the project's \texttt{.npmbundlerrc} file. A
loader's configuration is specified using two key options:
\texttt{sources} (the folders that contain the sources files to process)
and \texttt{rules} (the loaders, options---if applicable---and regular
expressions that determine which files to process). See
\href{/docs/7-2/reference/-/knowledge_base/r/understanding-the-npmbundlerrcs-structure\#package-processing-options}{Understanding
the \texttt{.npmbundlerrc}'s Structure} for more information on the
configuration requirements and options.

Loaders can be chained. Files are processed by the loaders in the order
they are listed in the \texttt{use} property. The files are passed to
the first loader, processed, sent to the next loader, and so on, until
the files are processed by the rules. You can run complex processes,
such as converting a SASS file into CSS with the sass-loader, and then
convert it into a JavaScript module with the style-loader. Once the
rules are applied, the liferay-npm-bundler continues with the pre, post,
and babel phases of the bundler plugins.

\chapter{Default liferay-npm-bundler
Loaders}\label{default-liferay-npm-bundler-loaders}

Several
\href{/docs/7-2/frameworks/-/knowledge_base/f/understanding-liferay-npm-bundlers-loaders}{loaders}
are available for the liferay-npm-bundler by default:

\href{https://github.com/liferay/liferay-js-toolkit/tree/master/packages/liferay-npm-bundler-loader-babel-loader}{\texttt{babel-loader}}:
processes source files with \href{https://babeljs.io/}{Babel}. This
avoids an extra build step before the bundler.

\href{https://github.com/liferay/liferay-js-toolkit/tree/master/packages/liferay-npm-bundler-loader-copy-loader}{\texttt{copy-loader}}:
copies source files (static assets) to the output folder.

\href{https://github.com/liferay/liferay-js-toolkit/tree/master/packages/liferay-npm-bundler-loader-css-loader}{\texttt{css-loader}}:
converts a CSS file into a JavaScript module that's inserted into the
DOM once it's loaded.

\href{https://github.com/liferay/liferay-js-toolkit/tree/master/packages/liferay-npm-bundler-loader-json-loader}{\texttt{json-loader}}:
generates JavaScript modules that export the contents of a JSON file as
an object, so you can include JSON files with the \texttt{require()}
call.

\href{https://github.com/liferay/liferay-js-toolkit/tree/master/packages/liferay-npm-bundler-loader-sass-loader}{\texttt{sass-loader}}:
runs \texttt{node-sass} or \texttt{sass} on source files. This lets you
generate static CSS files. It can be chained before
\texttt{style-loader}.

\href{https://github.com/liferay/liferay-js-toolkit/tree/master/packages/liferay-npm-bundler-loader-style-loader}{\texttt{style-loader}}:
converts a CSS file into a JavaScript module that inserts the CSS
contents into the DOM once it's loaded. This lets you include CSS files
with a \texttt{require()} call.

See the
\href{https://github.com/izaera/liferay-js-toolkit-showcase/tree/loaders}{liferay-js-toolkit
loaders showcase} for an example use case of the liferay-npm-bundler's
loaders. If the default loaders don't meet your requirements, you can
follow the instructions in
\href{/docs/7-2/frameworks/-/knowledge_base/f/creating-custom-loaders-for-the-liferay-npm-bundler}{Creating
Custom Loaders for the Bundler} to create your own loaders.

\chapter{Liferay JavaScript APIs}\label{liferay-javascript-apis}

The \texttt{Liferay} JavaScript object exposes methods, objects, and
properties that you can use to access Liferay DXP-specific information.
This section contains a comprehensive list of some of the most useful
utilities you can find inside the \texttt{Liferay} object.

\chapter{Accessing ThemeDisplay
Information}\label{accessing-themedisplay-information}

The \texttt{Liferay} global JavaScript Object exposes useful methods,
objects, and properties, each containing a wealth of information, one of
which is \texttt{ThemeDisplay}. If you have experience with Java
development in Liferay DXP, you may be familiar with ThemeDisplay. The
JavaScript object exposes the same information as the ThemeDisplay Java
Class. It gives you access to valuable information that you can use in
your applications, such as the Portal instance, the current user, the
user's language, whether the user is signed in or being impersonated,
the file path to the theme's resources, and much more.

The \texttt{Liferay} global object is automatically available in Liferay
DXP at runtime. To access the \texttt{ThemeDisplay} object, use the
following dot notation in your app:

\begin{verbatim}
Liferay.ThemeDisplay.method-name
\end{verbatim}

This reference describes some of the most commonly used
\texttt{ThemeDisplay} methods for retrieving IDs, file paths, and login
information. An exhaustive list of all of the available methods is
displayed in the table below:

\noindent\hrulefill

\begin{longtable}[]{@{}
  >{\raggedright\arraybackslash}p{(\columnwidth - 4\tabcolsep) * \real{0.3333}}
  >{\raggedright\arraybackslash}p{(\columnwidth - 4\tabcolsep) * \real{0.3333}}
  >{\raggedright\arraybackslash}p{(\columnwidth - 4\tabcolsep) * \real{0.3333}}@{}}
\toprule\noalign{}
\begin{minipage}[b]{\linewidth}\raggedright
Method
\end{minipage} & \begin{minipage}[b]{\linewidth}\raggedright
Type
\end{minipage} & \begin{minipage}[b]{\linewidth}\raggedright
Description
\end{minipage} \\
\midrule\noalign{}
\endhead
\bottomrule\noalign{}
\endlastfoot
getLayoutId & number & \\
getLayoutRelativeURL & string & Returns the relative URL for the page \\
getLayoutURL & string & \\
getParentLayoutId & number & \\
isControlPanel & boolean & \\
isPrivateLayout & boolean & \\
isVirtualLayout & boolean & \\
getBCP47LanguageId & number & \\
getCDNBaseURL & string & Returns the content delivery network (CDN) base
URL, or the current portal URL if the CDN base URL is null \\
getCDNDynamicResourcesHost & string & Returns the content delivery
network (CDN) dynamic resources host, or the current portal URL if the
CDN dynamic resources host is null \\
getCDNHost & string & \\
getCompanyGroupId & number & \\
getCompanyId & number & Returns the portal instance ID \\
getDefaultLanguageId & number & \\
getDoAsUserIdEncoded & string & \\
getLanguageId & number & Returns the user's language ID \\
getParentGroupId & number & \\
getPathContext & string & \\
getPathImage & string & Returns the relative path of the portlet's image
directory \\
getPathJavaScript & string & Returns the relative path of the directory
containing the portlet's JavaScript source files \\
getPathMain & string & Returns the path of the portal instance's main
directory \\
getPathThemeImages & string & Returns the path of the current theme's
image directory \\
getPathThemeRoot & string & Returns the relative path of the current
theme's root directory \\
getPlid & string & Returns the primary key of the page \\
getPortalURL & string & Returns the portal instance's base URL \\
getScopeGroupId & number & Returns the ID of the scoped or sub-scoped
active group (e.g.~site) \\
getScopeGroupIdOrLiveGroupId & number & \\
getSessionId & number & Returns the session ID, or a blank string if the
session ID is not available to the application \\
getSiteGroupId & number & \\
getURLControlPanel & string & \\
getURLHome & string & \\
getUserId & number & Returns the ID of the user for which the current
request is being handled \\
getUserName & string & Returns the user's name \\
isAddSessionIdToURL & boolean & \\
isFreeformLayout & boolean & \\
isImpersonated & boolean & Returns \texttt{true} if the current user is
being impersonated. Authorized administrative users can
\href{/docs/7-2/user/-/knowledge_base/u/adding-editing-and-deleting-users\#editing-users}{impersonate}
act as another user to test that user's account \\
isSignedIn & boolean & Returns \texttt{true} if the user is logged in to
the portal \\
isStateExclusive & boolean & \\
isStateMaximized & boolean & \\
isStatePopUp & boolean & \\
\end{longtable}

\noindent\hrulefill

The example configuration below alerts users with a standard message if
they are a guest or a personal greeting if they are signed in. This is a
basic example, and perhaps a bit invasive, but it illustrates how you
can create unique experiences for each user with the
\texttt{ThemeDisplay} APIs:

\begin{verbatim}
if(Liferay.ThemeDisplay.isSignedIn()){
    alert('Hello ' + Liferay.ThemeDisplay.getUserName() + '. Welcome Back.')
}
else {
    alert('Hello Guest.')
}
\end{verbatim}

\chapter{Working with URLs in
JavaScript}\label{working-with-urls-in-javascript}

The \texttt{Liferay} global JavaScript Object exposes methods, objects,
and properties that access the portal context. Four of these are helpful
when working with URLS: \texttt{authToken}, \texttt{currentURL},
\texttt{currentURLEncoded}, and \texttt{PortletURL}. If you have
experience with Java development in Liferay DXP, you may have worked
with some of these before. The \texttt{Liferay} global object is
automatically available at runtime, so no additional dependencies are
required.

\noindent\hrulefill

\textbf{Note:} Since Liferay DXP SP1 and Liferay Portal CE 7.2 GA2, the
\texttt{Liferay.PortletURL} utilities are deprecated and have been
replaced with \texttt{Liferay.Util.PortletURL} utilities. We recommend
that you use the updated versions to ensure future compatibility. The
examples below use the updated utilities.

\noindent\hrulefill

This covers how to use the \texttt{Liferay} global JavaScript object to
manipulate URLs. A list of the available methods and properties appears
in the tables shown below. Example configurations are shown below the
tables.

\section{Portlet URL Methods}\label{portlet-url-methods}

\texttt{Liferay.Util.PortletURL} Methods:

\noindent\hrulefill

\begin{longtable}[]{@{}
  >{\raggedright\arraybackslash}p{(\columnwidth - 4\tabcolsep) * \real{0.3333}}
  >{\raggedright\arraybackslash}p{(\columnwidth - 4\tabcolsep) * \real{0.3333}}
  >{\raggedright\arraybackslash}p{(\columnwidth - 4\tabcolsep) * \real{0.3333}}@{}}
\toprule\noalign{}
\begin{minipage}[b]{\linewidth}\raggedright
Method
\end{minipage} & \begin{minipage}[b]{\linewidth}\raggedright
Parameters
\end{minipage} & \begin{minipage}[b]{\linewidth}\raggedright
Returns
\end{minipage} \\
\midrule\noalign{}
\endhead
\bottomrule\noalign{}
\endlastfoot
\texttt{createPortletURL} & \texttt{basePortletURL}, \texttt{parameters}
& A portlet URL as a \href{https://url.spec.whatwg.org/\#api}{URL}
object \\
\texttt{createActionURL} & \texttt{basePortletURL}, \texttt{parameters}
& A portlet URL as a \href{https://url.spec.whatwg.org/\#api}{URL}
object \\
\texttt{createRenderURL} & \texttt{basePortletURL}, \texttt{parameters}
& A portlet URL as a \href{https://url.spec.whatwg.org/\#api}{URL}
object \\
\texttt{createResourceURL} & \texttt{basePortletURL},
\texttt{parameters} & A portlet URL as a
\href{https://url.spec.whatwg.org/\#api}{URL} object \\
\end{longtable}

\noindent\hrulefill

\section{Liferay Util PortletURL}\label{liferay-util-portleturl}

\texttt{Liferay.Util.PortletURL} provides APIs for creating portlet URLs
(\texttt{actionURL}, \texttt{renderURL}, and \texttt{resourceURL}) with
JavaScript in your JSPs. Below is an example configuration for a JSP:

\begin{verbatim}
var basePortletURL = 'https://localhost:8080/group/control_panel/manage?p_p_id=com_liferay_roles_admin_web_portlet_RolesAdminPortlet';

var actionURL = Liferay.Util.PortletURL.createActionURL(
  basePortletURL,
  {
    'javax.portlet.action': 'addUser',
    foo: 'bar'
  }  
);

console.log(actionURL.toString());
// https://localhost:8080/group/control_panel/manage?p_p_id=com_liferay_roles_admin_web_portlet_RolesAdminPortlet&javax.portlet.action=addUser&com_liferay_roles_admin_web_portlet_RolesAdminPortlet_foo=bar&p_p_lifecycle=1
\end{verbatim}

The same API is available as a module for use in your JavaScript files.
The ES6 example below uses the \texttt{createActionURL} module:

\begin{verbatim}
import {createActionURL} from 'frontend-js-web';

var basePortletURL = 'https://localhost:8080/group/control_panel/manage?p_p_id=com_liferay_roles_admin_web_portlet_RolesAdminPortlet';

var actionURL = createActionURL(
  basePortletURL,
  {
    'p_p_id': Liferay.PortletKeys.DOCUMENT_LIBRARY,
    foo: 'bar'
  }  
);
\end{verbatim}

See the \hyperref[portlet-url-methods]{Portlet URL Methods} section for
more information about the method used in the example above.

\section{Liferay AuthToken}\label{liferay-authtoken}

The \texttt{Liferay.authToken} property holds the current authentication
token value as a String. The \texttt{authToken} is used to validate
permissions when you make calls to services. To use the
\texttt{authToken} in a URL, pass \texttt{Liferay.authToken} as the
URL's \texttt{p\_auth} parameter, as shown in the example below:

\begin{verbatim}
import {createActionURL} from 'frontend-js-web';

var basePortletURL = 'https://localhost:8080/group/control_panel/manage?p_p_id=com_liferay_roles_admin_web_portlet_RolesAdminPortlet';

var actionURL = createActionURL(
  basePortletURL,
  {
    'p_auth': Liferay.authToken
  }  
);
\end{verbatim}

\section{Liferay CurrentURL}\label{liferay-currenturl}

The \texttt{Liferay.currentURL} property holds the path of the current
URL from the server root.

For example, if checked from \texttt{my.domain.com/es/web/guest/home},
the value is \texttt{/es/web/guest/home}, as shown below:

\begin{verbatim}
// Inside my.domain.com/es/web/guest/home
console.log(Liferay.currentURL); // "/es/web/guest/home"
\end{verbatim}

\section{Liferay CurrentURLEncoded}\label{liferay-currenturlencoded}

The \texttt{Liferay.currentURLEncoded} property holds the path of the
current URL, encoded in ASCII for safe transmission over the Internet,
from the server root.

For example, if checked from \texttt{my.domain.com/es/web/guest/home},
the value is \texttt{\%2Fes\%2Fweb\%2Fguest\%2Fhome}, as shown below:

\begin{verbatim}
// Inside my.domain.com/es/web/guest/home
console.log(Liferay.currentURLEncoded); // "%2Fes%2Fweb%2Fguest%2Fhome"
\end{verbatim}

Now you know how to manipulate URLs using methods within the
\texttt{Liferay} global JavaScript object.

\chapter{Liferay DXP JavaScript
Utilities}\label{liferay-dxp-javascript-utilities}

This reference explains some of the utility methods and objects inside
the \texttt{Liferay} global JavaScript object.

\section{Retrieve Browser
Information}\label{retrieve-browser-information}

The \texttt{Liferay.Browser} object contains methods that expose the
current user agent characteristics without the need of accessing and
parsing the global \texttt{window.navigator} object.

The available methods for the \texttt{Liferay.Browser} object are listed
in the table below:

\noindent\hrulefill

\begin{longtable}[]{@{}
  >{\raggedright\arraybackslash}p{(\columnwidth - 4\tabcolsep) * \real{0.3333}}
  >{\raggedright\arraybackslash}p{(\columnwidth - 4\tabcolsep) * \real{0.3333}}
  >{\raggedright\arraybackslash}p{(\columnwidth - 4\tabcolsep) * \real{0.3333}}@{}}
\toprule\noalign{}
\begin{minipage}[b]{\linewidth}\raggedright
Method
\end{minipage} & \begin{minipage}[b]{\linewidth}\raggedright
Return Type
\end{minipage} & \begin{minipage}[b]{\linewidth}\raggedright
Description
\end{minipage} \\
\midrule\noalign{}
\endhead
\bottomrule\noalign{}
\endlastfoot
acceptsGzip & boolean & Returns whether the browser accepts gzip file
compression \\
getMajorVersion & number & Returns the major version of the browser \\
getRevision & number & Returns the revision version of the browser \\
getVersion & number & Returns the major.minor version of the browser \\
isAir & boolean & Returns whether the browser is Adobe AIR \\
isChrome & boolean & Returns whether the browser is Chrome \\
isFirefox & boolean & Returns whether the browser is Firefox \\
isGecko & boolean & Returns whether the browser is Gecko \\
isIe & boolean & Returns whether the browser is Internet Explorer \\
isIphone & boolean & Returns whether the browser is on an Iphone \\
isLinux & boolean & Returns whether the browser is being viewed on
Linux \\
isMac & boolean & Returns whether the browser is being viewed on Mac \\
isMobile & boolean & Returns whether the browser is being viewed on a
mobile device \\
isMozilla & boolean & Returns whether the browser is Mozilla \\
isOpera & boolean & Returns whether the browser is Opera \\
isRtf & boolean & Returns whether the browser supports RTF \\
isSafari & boolean & Returns whether the browser is Safari \\
isSun & boolean & Returns whether the browser is being viewed on Sun
OS \\
isWebKit & boolean & Returns whether the browser is WebKit \\
isWindows & boolean & Returns whether the browser is being viewed on
Windows \\
\end{longtable}

\noindent\hrulefill

Below is an example configuration:

\begin{verbatim}
Liferay.Browser.isChrome(); //returns true in Chrome
\end{verbatim}

\section{Format XML}\label{format-xml}

The \texttt{Liferay.Util.formatXML} utility takes XML content, as a
String, and returns it formatted.

Parameters: - \texttt{content}: The XML string to format -
\texttt{options}: An optional configuration object \texttt{\{\}} that
contains additional parameters for formatting the XML

The default configuration contains these options:

\begin{verbatim}
const DEFAULT_OPTIONS = {
  newLine: NEW_LINE, //'\r\n'
  tagIndent: TAG_INDENT //'\t'
};
\end{verbatim}

Below is an example configuration for a JSP that overwrites the default
options:

\begin{verbatim}
var options = {newLine: '\n', tagIndent: ' '};

var input = `<?xml xlmns:a="http://www.w3.org/TR/html4/" version="1.0" encoding="UTF-8"?>
<!DOCTYPE note>
<a:note>                         <a:to>Foo</a:to>
<a:from>Bar</a:from><a:heading>FooBar</a:heading>
<a:body>FooBarBaz!</a:body>
                  </a:note>
`;

var formattedXMLString = Liferay.Util.formatXML(input, options);

console.log(formattedXMLString);
/*results:
<?xml xlmns:a="http://www.w3.org/TR/html4/" version="1.0" encoding="UTF-8"?>\n'
<!DOCTYPE note>\n'
<a:note>\n'
<a:to>Foo</a:to>\n'
<a:from>Bar</a:from>\n'
<a:heading>FooBar</a:heading>\n'
<a:body>FooBarBaz!</a:body>\n'
</a:note>';
*/
\end{verbatim}

\section{Format Storage Size}\label{format-storage-size}

The \texttt{Liferay.Util.formatStorage} utility takes a storage size
number (in bytes) and returns it in the proper format (KB, MB, or GB) as
a String.

Parameters: - \texttt{size}: The numerical value of the storage size in
bytes - \texttt{options}: An optional configuration object \texttt{\{\}}
that contains additional parameters for formatting the storage size

The default configuration contains these options:

\begin{verbatim}
const DEFAULT_OPTIONS = {
  addSpaceBeforeSuffix: false,
  decimalSeparator: '.',
  denominator: 1024.0,
  suffixGB: 'GB',
  suffixKB: 'KB',
  suffixMB: 'MB'
};
\end{verbatim}

Below is an example configuration that overwrites some of the default
options:

\begin{verbatim}
var formattedSize = Liferay.Util.formatStorage(1048576, {
  addSpaceBeforeSuffix: true,
  decimalSeparator: ',',
  suffixMB: 'megabytes'
});

console.log(formattedSize); //1,0 megabytes
\end{verbatim}

\section{Store and Retrieve Session Form
data}\label{store-and-retrieve-session-form-data}

\texttt{Liferay.Util.setSessionValue()}: Sets a key, value pair for the
Store utility's fetch value.

Parameters: - \texttt{key}: The \texttt{formData} key (String) -
\texttt{value}: The \texttt{formData} key's corresponding value
(Object\textbar String)

\texttt{Liferay.Util.getSessionValue()}: Retrieves the Store utility's
fetch value for the given \texttt{key}.

Parameters: - \texttt{key}: The key (String) to fetch the value for.

Below is an example configuration for a JSP:

\begin{verbatim}
Liferay.Util.Session.set('state', 'open');

Liferay.Util.Session.get('state').then(function(value) {
  console.log(value); //open
});
\end{verbatim}

Here is an example configuration that uses ES6:

\begin{verbatim}
import {getSessionValue, setSessionValue} from 'frontend-js-web';

setSessionValue('state', 'open');

getSessionValue('state').then(value =>{
  console.log(value); //open
});
\end{verbatim}

\chapter{Invoking Liferay Services}\label{invoking-liferay-services}

Liferay DXP provides many web services out-of-the-box. To see a
comprehensive list of the available web services, navigate to
\texttt{http://localhost:8080/api/jsonws} (assuming your localhost is
running on port 8080).

This reference covers how to invoke these web services using JavaScript.

\section{Invoking Web Services via
JavaScript}\label{invoking-web-services-via-javascript}

7.0 contains a global JavaScript object called \texttt{Liferay} that has
many useful utilities. One method is \texttt{Liferay.Service}, which
invokes JSON web services.

The \texttt{Liferay.Service} method takes four possible arguments:

\textbf{service \{string\textbar object\}:} Specify the service name or
an object with the keys as the service to call, and the value as the
service configuration object. (Required)

\textbf{data \{object\textbar node\textbar string\}:} Specify the data
to send to the service. If the object passed is the ID of a form or a
form element, the form fields will be serialized and used as the data.

\textbf{successCallback \{function\}:} A function to execute when the
server returns a response. It receives a JSON object as its first
parameter.

\textbf{exceptionCallback \{function\}:} A function to execute when the
response from the server contains a service exception. It receives an
exception message as its first parameter.

One of the benefits of using the \texttt{Liferay.Service} method versus
using a standard AJAX request is that it handles the authentication for
you.

Below is an example configuration of the \texttt{Liferay.Service}
method:

\begin{verbatim}
Liferay.Service(
        '/user/get-user-by-email-address',
        {
                companyId: Liferay.ThemeDisplay.getCompanyId(),
                emailAddress: 'test@example.com'
        },
        function(obj) {
                console.log(obj);
        }
);
\end{verbatim}

The example above retrieves information about a user by passing its
\texttt{companyId} and \texttt{emailAddress}. The response data
resembles the following JSON object:

\begin{verbatim}
{
        "agreedToTermsOfUse": true,
        "comments": "",
        "companyId": "20116",
        "contactId": "20157",
        "createDate": 1471990639779,
        "defaultUser": false,
        "emailAddress": "test@example.com",
        "emailAddressVerified": true,
        "facebookId": "0",
        "failedLoginAttempts": 0,
        "firstName": "Test",
        "googleUserId": "",
        "graceLoginCount": 0,
        "greeting": "Welcome Test Test!",
        "jobTitle": "",
        "languageId": "en_US",
        "lastFailedLoginDate": null,
        "lastLoginDate": 1471996720765,
        "lastLoginIP": "127.0.0.1",
        "lastName": "Test",
        "ldapServerId": "-1",
        "lockout": false,
        "lockoutDate": null,
        "loginDate": 1472077523149,
        "loginIP": "127.0.0.1",
        "middleName": "",
        "modifiedDate": 1472077523149,
        "mvccVersion": "7",
        "openId": "",
        "portraitId": "0",
        "reminderQueryAnswer": "test",
        "reminderQueryQuestion": "what-is-your-father's-middle-name",
        "screenName": "test",
        "status": 0,
        "timeZoneId": "UTC",
        "userId": "20156",
        "uuid": "c641a7c9-5acb-aa68-b3ea-5575e1845d2f"
}
\end{verbatim}

Now that you know how to send an individual request, you're ready to run
batch requests.

\section{Batching Requests}\label{batching-requests}

Another way to invoke the \texttt{Liferay.Service} method is by passing
an object with the keys of the service to call and the value of the
service configuration object.

Below is an example configuration for a batch request:

\begin{verbatim}
Liferay.Service(
        {
                '/user/get-user-by-email-address': {
                        companyId: Liferay.ThemeDisplay.getCompanyId(),
                        emailAddress: 'test@example.com'
                }
        },
        function(obj) {
                console.log(obj);
        }
);
\end{verbatim}

You can invoke multiple services with the same request by passing in an
array of service objects. Here's an example:

\begin{verbatim}
Liferay.Service(
        [
                {
                        '/user/get-user-by-email-address': {
                                companyId: Liferay.ThemeDisplay.getCompanyId(),
                                emailAddress: 'test@example.com'
                        }
                },
                {
                        '/role/get-user-roles': {
                                userId: Liferay.ThemeDisplay.getUserId()
                        }
                }
        ],
        function(obj) {
                // obj is now an array of response objects
                // obj[0] == /user/get-user-by-email-address data
                // obj[1] == /role/get-user-roles data

                console.log(obj);
        }
);
\end{verbatim}

Next you can learn how to nest your requests.

\section{Nesting Requests}\label{nesting-requests}

Nested service calls bind information from related objects together in a
JSON object. You can call other services in the same HTTP request and
conveniently nest returned objects.

You can use variables to reference objects returned from service calls.
Variable names must start with a dollar sign (\texttt{\$}).

The example in this section retrieves user data with
\texttt{/user/get-user-by-id} and uses the \texttt{contactId} returned
from that service to then invoke \texttt{/contact/get-contact} in the
same request.

\noindent\hrulefill

\textbf{Note:} You must flag parameters that take values from existing
variables. To flag a parameter, insert the \texttt{@} prefix before the
parameter name.

\noindent\hrulefill

Below is an example configuration that demonstrates these concepts:

\begin{verbatim}
Liferay.Service(
        {
                "$user = /user/get-user-by-id": {
                        "userId": Liferay.ThemeDisplay.getUserId(),
                        "$contact = /contact/get-contact": {
                                "@contactId": "$user.contactId"
                        }
                }
        },
        function(obj) {
                console.log(obj);
        }
);
\end{verbatim}

Here is what the response data would look like for the request above:

\begin{verbatim}
{
        "agreedToTermsOfUse": true,
        "comments": "",
        "companyId": "20116",
        "contactId": "20157",
        "createDate": 1471990639779,
        "defaultUser": false,
        "emailAddress": "test@example.com",
        "emailAddressVerified": true,
        "facebookId": "0",
        "failedLoginAttempts": 0,
        "firstName": "Test",
        "googleUserId": "",
        "graceLoginCount": 0,
        "greeting": "Welcome Test Test!",
        "jobTitle": "",
        "languageId": "en_US",
        "lastFailedLoginDate": null,
        "lastLoginDate": 1472231639378,
        "lastLoginIP": "127.0.0.1",
        [...]
        "screenName": "test",
        "status": 0,
        "timeZoneId": "UTC",
        "userId": "20156",
        "uuid": "c641a7c9-5acb-aa68-b3ea-5575e1845d2f",
        "contact": {
                "accountId": "20118",
                "birthday": 0,
                [...]
                "createDate": 1471990639779,
                "emailAddress": "test@example.com",
                "employeeNumber": "",
                "employeeStatusId": "",
                "facebookSn": "",
                "firstName": "Test",
                "lastName": "Test",
                "male": true,
                "middleName": "",
                "modifiedDate": 1471990639779,
                [...]
                "userName": ""
        }
}
\end{verbatim}

Now that you know how to process requests, you can learn how to filter
the results.

\section{Filtering Results}\label{filtering-results}

If you don't want all the properties returned by a service, you can
define a whitelist of properties. This returns only the specific
properties you request in the object.

Below is an example of whitelisting properties:

\begin{verbatim}
Liferay.Service(
        {
                '$user[emailAddress,firstName] = /user/get-user-by-id': {
                        userId: Liferay.ThemeDisplay.getUserId()
                }
        },
        function(obj) {
                console.log(obj);
        }
);
\end{verbatim}

To specify whitelist properties, place the properties in square brackets
(e.g., \texttt{{[}whiteList{]}}) immediately following the name of your
variable. The example above requests only the \texttt{emailAddress} and
\texttt{firstName} of the user.

Below is the filtered response:

\begin{verbatim}
{
        "firstName": "Test",
        "emailAddress": "test@example.com"
}
\end{verbatim}

Next you can learn how to populate the inner parameters of the request.

\section{Inner Parameters}\label{inner-parameters}

When you pass in an object parameter, you'll often need to populate its
inner parameters (i.e., fields).

Consider a default parameter \texttt{serviceContext} of type
\texttt{ServiceContext}. To make an appropriate call to JSON web
services you might need to set \texttt{serviceContext} fields such as
\texttt{scopeGroupId}, as shown below:

\begin{verbatim}
Liferay.Service(
        '/example/some-web-service',
        {
                serviceContext: {
                        scopeGroupId: 123
                }
        },
        function(obj) {
                console.log(obj);
        }
);
\end{verbatim}

\chapter{\texorpdfstring{Handling AJAX Requests with
\texttt{Liferay.Util.fetch}}{Handling AJAX Requests with Liferay.Util.fetch}}\label{handling-ajax-requests-with-liferay.util.fetch}

When you make Ajax requests (referred to as Service Resource
actions/requests in Liferay DXP), they must protect against
\href{https://en.wikipedia.org/wiki/Cross-site_request_forgery}{CSRF}
and include the proper credentials. Since Liferay DXP 7.2 SP1 and
Liferay CE Portal 7.2 GA2, Liferay DXP provides a
\texttt{Liferay.Util.fetch} utility based on the standard
\href{https://fetch.spec.whatwg.org/}{\texttt{fetch}} API that you can
use to make AJAX requests. It includes these key features:

\begin{itemize}
\tightlist
\item
  A thin wrapper on ES6
  \href{https://fetch.spec.whatwg.org/}{\texttt{fetch}} that shares the
  same API
\item
  Sets \texttt{credentials:include} to each request
\item
  Sets \texttt{x-csrf-token} header to each request
\item
  Requires no dependencies
\end{itemize}

Below is an example configuration in ES6:

\begin{verbatim}
import {fetch} from 'frontend-js-web';

fetch(url, {
  body: new FormData(form),
  method: 'POST'
})
  .then(response => response.json())
  .then(response => processData(response))
  .then(response => failureCallback(error));
\end{verbatim}

Example use case in JSPs:

\begin{verbatim}
Liferay.Util.fetch(url, {
  body: new FormData(form),
  method: 'POST'
}).then(function(response) {
  return response.json();
}).then(function(response) {
  message.innerHTML = response.message;
}).catch(function() {
  failureCallback();
});
\end{verbatim}

\noindent\hrulefill

\textbf{NOTE:} global access through \texttt{Liferay.Util} is only meant
for use in JSP code. In ES6, you must use the \texttt{fetch} module, as
shown in the JavaScript example above.

\chapter{Working with Addresses}\label{working-with-addresses}

The \texttt{Liferay} global JavaScript Object exposes methods, objects,
and properties that access the portal context. The
\texttt{Liferay.Address} utility contains methods for retrieving
information about the addresses country and region. The \texttt{Liferay}
global object is automatically available at runtime, so no additional
dependencies are required.

The available methods are listed below, along with an example
configuration.

\texttt{Liferay.Address.getCountries(callback)}: returns an Array of the
available countries.

Parameters: - \texttt{callback}: A callback function to post-process the
Array of countries

The example below prints the list of available regions for the selected
country (the United States in this case) in a table:

\begin{verbatim}
Liferay.Address.getCountries(function(e){console.table(e)}, 19);
\end{verbatim}

\texttt{Liferay.Address.getRegions(callback,\ selectKey)}: returns an
Array of the available regions, by country, for the specified region ID.

Parameters: - \texttt{callback}: A callback function to post-process the
Array of regions - \texttt{selectKey}: The selected region ID

The example below prints the list of available countries in a table to
the console:

\begin{verbatim}
Liferay.Address.getCountries(function(e){console.table(e)});
\end{verbatim}

This example uses an AUI \texttt{DynamicSelect} module to create a pair
of select fields in a JSP. The first field retrieves the countries with
the \texttt{Liferay.Address.getCountries()} method, and the second
select field is dynamically populated with the selected country's
available regions with the \texttt{Liferay.Address.getRegions()} method:

\begin{verbatim}
<aui:script use="liferay-dynamic-select">
  new Liferay.DynamicSelect(
    [
      {
        select: '<portlet:namespace />countryId',
        selectData: Liferay.Address.getCountries,
        selectDesc: 'nameCurrentValue',
        selectId: 'countryId',
        selectSort: '<%= true %>',
        selectVal: '<%= countryId %>'
      },
      {
        select: '<portlet:namespace />regionId',
        selectData: Liferay.Address.getRegions,
        selectDesc: 'name',
        selectDisableOnEmpty: true,
        selectId: 'regionId',
        selectVal: '<%= regionId %>'
      }
    ]
  );
</aui:script>
\end{verbatim}

\chapter{FreeMarker Taglib Macros}\label{freemarker-taglib-macros}

Liferay DXP's taglibs are mapped to FreeMarker macros, so you can use
them in your FreeMarker templates. See the
\href{/docs/7-2/reference/-/knowledge_base/r/front-end-taglibs}{Taglib
reference} for more information on using each taglib in your theme
templates. The taglib macros are defined in
\texttt{taglib-mappings.properties} files. For convenience, these macros
are listed in the table below:

Macro

Taglib

TLD

\texttt{liferay\_aui}

liferay-aui

liferay-aui.tld

\texttt{liferay\_portlet}

liferay-portlet

liferay-portlet-ext.tld

\texttt{liferay\_security}

liferay-security

liferay-security.tld

\texttt{liferay\_theme}

liferay-theme

liferay-theme.tld

\texttt{liferay\_ui}

liferay-ui

liferay-ui.tld

\texttt{liferay\_util}

liferay-util

liferay-util.tld

\texttt{portlet}

portlet

liferay-portlet.tld

\texttt{liferay\_frontend}

liferay-frontend

liferay-frontend.tld

\texttt{clay}

clay

liferay-clay.tld

\texttt{liferay\_map}

liferay-map

liferay-map.tld

\texttt{liferay\_rss}

liferay-rss

liferay-rss.tld

\texttt{liferay\_flags}

liferay-flags

liferay-flags.tld

\texttt{liferay\_expando}

liferay-expando

liferay-expando.tld

\texttt{liferay\_journal}

liferay-journal

liferay-journal.tld

\texttt{liferay\_social\_bookmarks}

liferay-social-bookmarks

liferay-social-bookmarks.tld

\texttt{liferay\_site}

liferay-site

liferay-site.tld

\texttt{liferay\_comment}

liferay-comment

liferay-comment.tld

\texttt{liferay\_social\_activities}

liferay-social-activities

liferay-social-activities.tld

\texttt{liferay\_asset}

liferay-asset

liferay-asset.tld

\texttt{liferay\_trash}

liferay-trash

liferay-trash.tld

\texttt{liferay\_item\_selector}

liferay-item-selector

liferay-item-selector.tld

\texttt{liferay\_layout}

liferay-layout

liferay-layout.tld

\texttt{liferay\_editor}

liferay-editor

liferay-editor.tld

\texttt{liferay-fragment}

liferay-fragment

liferay-fragment.tld

\texttt{liferay\_reading\_time}

liferay-reading-time

liferay-reading-time.tld

\texttt{liferay\_sharing}

liferay-sharing

liferay-sharing.tld

\texttt{liferay\_site\_navigation}

liferay-site-navigation

liferay-site-navigation.tld

\texttt{adaptive\_media\_image}

liferay-adaptive-media

liferay-adaptive-media.tld

\texttt{liferay\_product\_navigation}

liferay-product-navigation

liferay-product-navigation.tld

\chapter{Setting up Your npm
Environment}\label{setting-up-your-npm-environment}

If you're using npm for development in Liferay DXP, you should set up
your npm environment to avoid potential permissions issues. Follow these
steps to configure your npm environment:

\begin{enumerate}
\def\labelenumi{\arabic{enumi}.}
\item
  Create an \texttt{.npmrc} file in your user's home directory. This
  helps bypass npm permission-related issues.
\item
  In the \texttt{.npmrc} file, specify a \texttt{prefix} property based
  on your user's home directory, like the one shown below. This value
  specifies where to install global npm packages:

\begin{verbatim}
prefix=/Users/[username]/.npm-packages
\end{verbatim}
\item
  Set the \texttt{NPM\_PACKAGES} system environment variable to the
  \texttt{prefix} value you just specified:

\begin{verbatim}
NPM_PACKAGES=/Users/[username]/.npm-packages (same as prefix value)
\end{verbatim}
\item
  Since npm installs Yeoman and gulp executables to
  \texttt{\$\{NPM\_PACKAGES\}/bin} on UNIX and to
  \texttt{\%NPM\_PACKAGES\%} on Windows, make sure to add the
  appropriate directory to your system path. For example, on UNIX you'd
  set this:

\begin{verbatim}
PATH=${PATH}:${NPM_PACKAGES}/bin
\end{verbatim}
\end{enumerate}

\chapter{Sitemap Page Configuration
Options}\label{sitemap-page-configuration-options}

If you're importing resources with your themes, you must define the
pages for the site in the theme's \texttt{sitemap.json}. Below is the
full list of available configuration options for pages in the theme's
\texttt{sitemap.json}:

\textbf{colorSchemeId:} Specifies a different color scheme (by ID) than
the default color scheme to use for the page.

\textbf{columns:} Specifies the column contents for the page.

\textbf{friendlyURL:} Sets the page's friendly URL.

\textbf{hidden:} Sets whether the page is hidden.

\textbf{layoutCss:} Sets custom CSS for the page to load after the
theme.

\textbf{layoutPrototypeLinkEnabled:} Sets whether the page inherits
changes made to the page template (if the page has one).

\textbf{layoutPrototypeName:} Specifies the page template (by name) to
use for the page. If this is defined, the page template's UUID is
retrieved using the name, and \texttt{layoutPrototypeUuid} is not
required.

\textbf{layoutPrototypeUuid:} Specifies the page template (by UUID) to
use for the page. If \texttt{layoutPrototypeName} is defined, this is
not required.

\textbf{layouts:} Specifies child pages for a page set.

\textbf{name:} The page's name.

\textbf{nameMap:} Passes a name object with multiple name key/value
pairs. You can use this to pass translations for a page's title, as
shown in the example above.

\textbf{privatePages:} Specifies private pages.

\textbf{publicPages:} Specifies public pages.

\textbf{themeId:} Specifies a different theme (by ID) than the default
theme bundled with the \texttt{sitemap.json} to use for the page.

\textbf{title:} The page's title.

\textbf{type:} Sets the page type. The default value is \texttt{portlet}
(empty page). Possible values are \texttt{copy} (copy of a page of this
site), \texttt{embedded}, \texttt{full\_page\_application},
\texttt{link\_to\_layout}, \texttt{node} (page set), \texttt{panel},
\texttt{portlet}, and \texttt{url} (link to URL).

\textbf{typeSettings:} Specifies settings (using key/value pairs) for
the page \texttt{type}.

\chapter{CKEditor Plugin Reference
Guide}\label{ckeditor-plugin-reference-guide}

This reference guide provides a list of the default CKEditor plugins
bundled with Liferay DXP's AlloyEditor. You can
\href{/docs/7-2/frameworks/-/knowledge_base/f/adding-buttons-to-alloyeditor-toolbars}{use
these existing CKEditor plugins in your custom AlloyEditor
configurations}. Each plugin below links to its \texttt{plugin.js} file
for reference, specifying the plugin's name and buttons if applicable:

\begin{itemize}
\tightlist
\item
  \href{https://github.com/ckeditor/ckeditor-dev/tree/master/plugins/about/plugin.js}{about}
\item
  \href{https://github.com/ckeditor/ckeditor-dev/tree/master/plugins/a11yhelp/plugin.js}{allyhelp}
\item
  \href{https://github.com/liferay/liferay-portal/tree/7.2.x/modules/apps/frontend-editor/frontend-editor-ckeditor-web/src/main/resources/META-INF/resources/_diffs/plugins/a11yhelpbtn/plugin.js}{allyhelpbtn}
\item
  \href{https://github.com/liferay/liferay-portal/tree/7.2.x/modules/apps/frontend-editor/frontend-editor-ckeditor-web/src/main/resources/META-INF/resources/_diffs/plugins/ajaxsave/plugin.js}{ajaxsave}
\item
  \href{https://github.com/liferay/liferay-portal/tree/7.2.x/modules/apps/frontend-editor/frontend-editor-ckeditor-web/src/main/resources/META-INF/resources/_diffs/plugins/autocomplete/plugin.js}{autocomplete}
\item
  \href{https://github.com/ckeditor/ckeditor-dev/tree/master/plugins/basicstyles/plugin.js}{basicstyles}
\item
  \href{https://github.com/liferay/liferay-portal/tree/7.2.x/modules/apps/frontend-editor/frontend-editor-ckeditor-web/src/main/resources/META-INF/resources/_diffs/plugins/bbcode/plugin.js}{bbcode}
\item
  \href{https://github.com/ckeditor/ckeditor-dev/tree/master/plugins/bidi/plugin.js}{bidi}
\item
  \href{https://github.com/ckeditor/ckeditor-dev/tree/master/plugins/blockquote/plugin.js}{blockquote}
\item
  \href{https://github.com/ckeditor/ckeditor-dev/tree/master/plugins/clipboard/plugin.js}{clipboard}
\item
  \href{https://github.com/ckeditor/ckeditor-dev/tree/master/plugins/colorbutton/plugin.js}{colorbutton}
\item
  \href{https://github.com/ckeditor/ckeditor-dev/tree/master/plugins/colordialog/plugin.js}{colordialog}
\item
  \href{https://github.com/ckeditor/ckeditor-dev/blob/master/plugins/contextmenu/plugin.js}{contextmenu}
\item
  \href{https://github.com/liferay/liferay-portal/blob/7.2.x/modules/apps/frontend-editor/frontend-editor-ckeditor-web/src/main/resources/META-INF/resources/_diffs/plugins/creole/plugin.js}{creole}
\item
  \href{https://github.com/ckeditor/ckeditor-dev/blob/master/plugins/dialogadvtab/plugin.js}{dialogadvtab}
\item
  \href{https://github.com/ckeditor/ckeditor-dev/blob/master/plugins/div/plugin.js}{div}
\item
  \href{https://github.com/ckeditor/ckeditor-dev/blob/master/plugins/elementspath/plugin.js}{elementspath}
\item
  \href{https://github.com/ckeditor/ckeditor-dev/blob/master/plugins/enterkey/plugin.js}{enterkey}
\item
  \href{https://github.com/ckeditor/ckeditor-dev/blob/master/plugins/entities/plugin.js}{entities}
\item
  \href{https://github.com/ckeditor/ckeditor-dev/blob/master/plugins/filebrowser/plugin.js}{filebrowse}
\item
  \href{https://github.com/ckeditor/ckeditor-dev/blob/master/plugins/find/plugin.js}{find}
\item
  \href{https://github.com/ckeditor/ckeditor-dev/blob/master/plugins/flash/plugin.js}{flash}
\item
  \href{https://github.com/ckeditor/ckeditor-dev/blob/master/plugins/floatingspace/plugin.js}{floatingspace}
\item
  \href{https://github.com/ckeditor/ckeditor-dev/blob/master/plugins/font/plugin.js}{font}
\item
  \href{https://github.com/ckeditor/ckeditor-dev/blob/master/plugins/format/plugin.js}{format}
\item
  \href{https://github.com/ckeditor/ckeditor-dev/blob/master/plugins/forms/plugin.js}{forms}
\item
  \href{https://github.com/ckeditor/ckeditor-dev/blob/master/plugins/horizontalrule/plugin.js}{horizontalrule}
\item
  \href{https://github.com/ckeditor/ckeditor-dev/blob/master/plugins/htmlwriter/plugin.js}{htmlwriter}
\item
  \href{https://github.com/ckeditor/ckeditor-dev/blob/master/plugins/image/plugin.js}{image}
\item
  \href{https://github.com/ckeditor/ckeditor-dev/blob/master/plugins/iframe/plugin.js}{iframe}
\item
  \href{https://github.com/ckeditor/ckeditor-dev/blob/master/plugins/indent/plugin.js}{indent}
\item
  \href{https://github.com/liferay/liferay-portal/blob/7.2.x/modules/apps/frontend-editor/frontend-editor-ckeditor-web/src/main/resources/META-INF/resources/_diffs/plugins/itemselector/plugin.js}{itemselector}
\item
  \href{https://github.com/ckeditor/ckeditor-dev/blob/master/plugins/justify/plugin.js}{justify}
\item
  \href{https://github.com/ckeditor/ckeditor-dev/blob/master/plugins/link/plugin.js}{link}
\item
  \href{https://github.com/ckeditor/ckeditor-dev/blob/master/plugins/list/plugin.js}{list}
\item
  \href{https://github.com/ckeditor/ckeditor-dev/blob/master/plugins/liststyle/plugin.js}{liststyle}
\item
  \href{https://github.com/liferay/liferay-portal/blob/7.2.x/modules/apps/frontend-editor/frontend-editor-ckeditor-web/src/main/resources/META-INF/resources/_diffs/plugins/lfrpopup/plugin.js}{lfrpopup}
\item
  \href{https://github.com/ckeditor/ckeditor-dev/blob/master/plugins/magicline/plugin.js}{magicline}
\item
  \href{https://github.com/liferay/liferay-portal/blob/7.2.x/modules/apps/frontend-editor/frontend-editor-ckeditor-web/src/main/resources/META-INF/resources/_diffs/plugins/media/plugin.js}{media}
\item
  \href{https://github.com/ckeditor/ckeditor-dev/blob/master/plugins/newpage/plugin.js}{newpage}
\item
  \href{https://github.com/ckeditor/ckeditor-dev/blob/master/plugins/pagebreak/plugin.js}{pagebreak}
\item
  \href{https://github.com/ckeditor/ckeditor-dev/blob/master/plugins/pastefromword/plugin.js}{pastefromword}
\item
  \href{https://github.com/ckeditor/ckeditor-dev/blob/master/plugins/pastetext/plugin.js}{pastetext}
\item
  \href{https://github.com/ckeditor/ckeditor-dev/blob/master/plugins/preview/plugin.js}{preview}
\item
  \href{https://github.com/ckeditor/ckeditor-dev/blob/master/plugins/removeformat/plugin.js}{removeformat}
\item
  \href{https://github.com/ckeditor/ckeditor-dev/blob/master/plugins/resize/plugin.js}{resize}
\item
  \href{https://github.com/liferay/liferay-portal/blob/7.2.x/modules/apps/frontend-editor/frontend-editor-ckeditor-web/src/main/resources/META-INF/resources/_diffs/plugins/restore/plugin.js}{restore}
\item
  \href{https://github.com/ckeditor/ckeditor-dev/blob/master/plugins/selectall/plugin.js}{selectall}
\item
  \href{https://github.com/ckeditor/ckeditor-dev/blob/master/plugins/showblocks/plugin.js}{showblocks}
\item
  \href{https://github.com/ckeditor/ckeditor-dev/blob/master/plugins/showborders/plugin.js}{showborders}
\item
  \href{https://github.com/ckeditor/ckeditor-dev/blob/master/plugins/smiley/plugin.js}{smiley}
\item
  \href{https://github.com/ckeditor/ckeditor-dev/blob/master/plugins/sourcearea/plugin.js}{sourcearea}
\item
  \href{https://github.com/ckeditor/ckeditor-dev/blob/master/plugins/specialchar/plugin.js}{specialchar}
\item
  \href{https://github.com/ckeditor/ckeditor-dev/blob/master/plugins/stylescombo/plugin.js}{stylescombo}
\item
  \href{https://github.com/ckeditor/ckeditor-dev/blob/master/plugins/tab/plugin.js}{tab}
\item
  \href{https://github.com/ckeditor/ckeditor-dev/blob/master/plugins/table/plugin.js}{table}
\item
  \href{https://github.com/ckeditor/ckeditor-dev/blob/master/plugins/tabletools/plugin.js}{tabletools}
\item
  \href{https://github.com/ckeditor/ckeditor-dev/blob/master/plugins/templates/plugin.js}{templates}
\item
  \href{https://github.com/ckeditor/ckeditor-dev/blob/master/plugins/toolbar/plugin.js}{toolbar}
\item
  \href{https://github.com/ckeditor/ckeditor-dev/blob/master/plugins/undo/plugin.js}{undo}
\item
  \href{https://github.com/liferay/liferay-portal/blob/7.2.x/modules/apps/frontend-editor/frontend-editor-ckeditor-web/src/main/resources/META-INF/resources/_diffs/plugins/wikilink/plugin.js}{wikilink}
\item
  \href{https://github.com/ckeditor/ckeditor-dev/blob/master/plugins/wysiwygarea/plugin.js}{wysiwygarea}
\end{itemize}

\noindent\hrulefill

\textbf{Note:} The following CKEditor plugins are not available for
inline mode in AlloyEditor at this time, but you can still use them in
the classic CKEditor:

\begin{itemize}
\tightlist
\item
  \href{https://github.com/ckeditor/ckeditor-dev/blob/master/plugins/maximize/plugin.js}{maximize}
\item
  \href{https://github.com/ckeditor/ckeditor-dev/blob/master/plugins/print/plugin.js}{print}
\item
  \href{https://github.com/ckeditor/ckeditor-dev/blob/master/plugins/save/plugin.js}{save}
\end{itemize}

To use the Classic CKEditor instead of AlloyEditor, there are a few
properties to set, depending on the portlet. Add the
\href{https://github.com/liferay/liferay-portal/blob/7.2.x/portal-impl/src/portal.properties\#L5432-L5441}{properties}
that you need to your \texttt{portal-ext.properties} file:

\begin{verbatim}
 editor.wysiwyg.default=ckeditor
 editor.wysiwyg.portal-impl.portlet.ddm.text_html.ftl=ckeditor
 editor.wysiwyg.portal-web.docroot.html.portlet.announcements.edit_entry.jsp=ckeditor
 editor.wysiwyg.portal-web.docroot.html.portlet.blogs.edit_entry.jsp=ckeditor
 editor.wysiwyg.portal-web.docroot.html.portlet.mail.edit.jsp=ckeditor
 editor.wysiwyg.portal-web.docroot.html.portlet.mail.edit_message.jsp=ckeditor
 editor.wysiwyg.portal-web.docroot.html.portlet.message_boards.edit_message.html.jsp=ckeditor
 editor.wysiwyg.portal-web.docroot.html.taglib.ui.discussion.jsp=ckeditor
 editor.wysiwyg.portal-web.docroot.html.taglib.ui.email_notification_settings.jsp=ckeditor
\end{verbatim}

\chapter{Fully Qualified Portlet IDs}\label{fully-qualified-portlet-ids}

Below is a listing of the portlet IDs for the default portlets in
Liferay DXP. You can use these IDs to embed portlets in your theme's
\href{/docs/7-2/frameworks/-/knowledge_base/f/defining-portlets-in-a-sitemap}{sitemap}.

\textbf{Collaboration}

\noindent\hrulefill

\begin{longtable}[]{@{}
  >{\raggedright\arraybackslash}p{(\columnwidth - 2\tabcolsep) * \real{0.5000}}
  >{\raggedright\arraybackslash}p{(\columnwidth - 2\tabcolsep) * \real{0.5000}}@{}}
\toprule\noalign{}
\begin{minipage}[b]{\linewidth}\raggedright
Portlet
\end{minipage} & \begin{minipage}[b]{\linewidth}\raggedright
ID
\end{minipage} \\
\midrule\noalign{}
\endhead
\bottomrule\noalign{}
\endlastfoot
Blogs & \texttt{com\_liferay\_blogs\_web\_portlet\_BlogsPortlet} \\
Blogs Aggregator &
\texttt{com\_liferay\_blogs\_web\_portlet\_BlogsAgreggatorPortlet} \\
Calendar &
\texttt{com\_liferay\_calendar\_web\_portlet\_CalendarPortlet} \\
Dynamic Data Lists Display &
\texttt{com\_liferay\_dynamic\_data\_lists\_web\_portlet\_DDLDisplayPortlet} \\
Form &
\texttt{com\_liferay\_dynamic\_data\_mapping\_form\_web\_portlet\_DDMFormPortlet} \\
Invite Members &
\texttt{com\_liferay\_invitation\_invite\_members\_web\_portlet\_InviteMembersPortlet} \\
Message Boards &
\texttt{com\_liferay\_message\_boards\_web\_portlet\_MBPortlet} \\
Recent Bloggers &
\texttt{com\_liferay\_blogs\_recent\_bloggers\_web\_portlet\_RecentBloggersPortlet} \\
\end{longtable}

\noindent\hrulefill

\textbf{Community}

\noindent\hrulefill

\begin{longtable}[]{@{}
  >{\raggedright\arraybackslash}p{(\columnwidth - 2\tabcolsep) * \real{0.5000}}
  >{\raggedright\arraybackslash}p{(\columnwidth - 2\tabcolsep) * \real{0.5000}}@{}}
\toprule\noalign{}
\begin{minipage}[b]{\linewidth}\raggedright
Portlet
\end{minipage} & \begin{minipage}[b]{\linewidth}\raggedright
ID
\end{minipage} \\
\midrule\noalign{}
\endhead
\bottomrule\noalign{}
\endlastfoot
My Sites &
\texttt{com\_liferay\_site\_my\_sites\_web\_portlet\_MySitesPortlet} \\
Page Comments &
\texttt{com\_liferay\_comment\_page\_comments\_web\_portlet\_PageCommentsPortlet} \\
Page Flags &
\texttt{com\_liferay\_flags\_web\_portlet\_PageFlagsPortlet} \\
Page Ratings &
\texttt{com\_liferay\_ratings\_page\_ratings\_web\_portlet\_PageRatingsPortlet} \\
\end{longtable}

\noindent\hrulefill

\textbf{Content Management}

\noindent\hrulefill

\begin{longtable}[]{@{}
  >{\raggedright\arraybackslash}p{(\columnwidth - 2\tabcolsep) * \real{0.5000}}
  >{\raggedright\arraybackslash}p{(\columnwidth - 2\tabcolsep) * \real{0.5000}}@{}}
\toprule\noalign{}
\begin{minipage}[b]{\linewidth}\raggedright
Portlet
\end{minipage} & \begin{minipage}[b]{\linewidth}\raggedright
ID
\end{minipage} \\
\midrule\noalign{}
\endhead
\bottomrule\noalign{}
\endlastfoot
Asset Publisher &
\texttt{com\_liferay\_asset\_publisher\_web\_portlet\_AssetPublisherPortlet} \\
Breadcrumb &
\texttt{com\_liferay\_site\_navigation\_breadcrumb\_web\_portlet\_SiteNavigationBreadcrumbPortlet} \\
Categories Navigation &
\texttt{com\_liferay\_asset\_categories\_navigation\_web\_portlet\_AssetCategoriesNavigationPortlet} \\
Documents and Media &
\texttt{com\_liferay\_document\_library\_web\_portlet\_DLPortlet} \\
Highest Rated Assets &
\texttt{com\_liferay\_asset\_publisher\_web\_portlet\_HighestRatedAssetsPortlet} \\
Knowledge Base Article &
\texttt{com\_liferay\_knowledge\_base\_web\_portlet\_ArticlePortlet} \\
Knowledge Base Display &
\texttt{com\_liferay\_knowledge\_base\_web\_portlet\_DisplayPortlet} \\
Knowledge Base Search &
\texttt{com\_liferay\_knowledge\_base\_web\_portlet\_SearchPortlet} \\
Knowledge Base Section &
\texttt{com\_liferay\_knowledge\_base\_web\_portlet\_SectionPortlet} \\
Media Gallery &
\texttt{com\_liferay\_document\_library\_web\_portlet\_IGDisplayPortlet} \\
Most Viewed Assets &
\texttt{com\_liferay\_asset\_publisher\_web\_portlet\_MostViewedAssetsPortlet} \\
Navigation Menu &
\texttt{com\_liferay\_site\_navigation\_menu\_web\_portlet\_SiteNavigationMenuPortlet} \\
Nested Applications &
\texttt{com\_liferay\_nested\_portlets\_web\_portlet\_NestedPortletsPortlet} \\
Polls Display Portlet &
\texttt{com\_liferay\_polls\_web\_portlet\_PollsDisplayPortlet} \\
Related Assets &
\texttt{com\_liferay\_asset\_publisher\_web\_portlet\_RelatedAssetsPortlet} \\
Site Map &
\texttt{com\_liferay\_site\_navigation\_site\_map\_web\_portlet\_SiteNavigationSiteMapPortlet} \\
Sites Directory &
\texttt{com\_liferay\_site\_navigation\_directory\_web\_portlet\_SitesDirectoryPortlet} \\
Tag Cloud &
\texttt{com\_liferay\_asset\_tags\_navigation\_web\_portlet\_AssetTagsCloudPortlet} \\
Tags Navigation &
\texttt{com\_liferay\_asset\_tags\_navigation\_web\_portlet\_AssetTagsNavigationPortlet} \\
Web Content Display &
\texttt{com\_liferay\_journal\_content\_web\_portlet\_JournalContentPortlet} \\
\end{longtable}

\noindent\hrulefill

\textbf{News}

\noindent\hrulefill

\begin{longtable}[]{@{}
  >{\raggedright\arraybackslash}p{(\columnwidth - 2\tabcolsep) * \real{0.5000}}
  >{\raggedright\arraybackslash}p{(\columnwidth - 2\tabcolsep) * \real{0.5000}}@{}}
\toprule\noalign{}
\begin{minipage}[b]{\linewidth}\raggedright
Portlet
\end{minipage} & \begin{minipage}[b]{\linewidth}\raggedright
ID
\end{minipage} \\
\midrule\noalign{}
\endhead
\bottomrule\noalign{}
\endlastfoot
Alerts &
\texttt{com\_liferay\_announcements\_web\_portlet\_AlertsPortlet} \\
Announcements &
\texttt{com\_liferay\_announcements\_web\_portlet\_AnnouncementsPortlet} \\
Recent Content Portlet &
\texttt{com\_liferay\_asset\_publisher\_web\_portlet\_RecentContentPortlet} \\
\end{longtable}

\noindent\hrulefill

\textbf{Sample}

\noindent\hrulefill

\begin{longtable}[]{@{}ll@{}}
\toprule\noalign{}
Portlet & ID \\
\midrule\noalign{}
\endhead
\bottomrule\noalign{}
\endlastfoot
Hello World &
\texttt{com\_liferay\_hello\_world\_web\_portlet\_HelloWorldPortlet} \\
IFrame & \texttt{com\_liferay\_iframe\_web\_portlet\_IFramePortlet} \\
\end{longtable}

\noindent\hrulefill

\textbf{Search}

\noindent\hrulefill

\begin{longtable}[]{@{}
  >{\raggedright\arraybackslash}p{(\columnwidth - 2\tabcolsep) * \real{0.5000}}
  >{\raggedright\arraybackslash}p{(\columnwidth - 2\tabcolsep) * \real{0.5000}}@{}}
\toprule\noalign{}
\begin{minipage}[b]{\linewidth}\raggedright
Portlet
\end{minipage} & \begin{minipage}[b]{\linewidth}\raggedright
ID
\end{minipage} \\
\midrule\noalign{}
\endhead
\bottomrule\noalign{}
\endlastfoot
Category Facet &
\texttt{com\_liferay\_portal\_search\_web\_category\_facet\_portlet\_CategoryFacetPortlet} \\
Custom Facet &
\texttt{com\_liferay\_portal\_search\_web\_custom\_facet\_portlet\_CustomFacetPortlet} \\
Folder Facet &
\texttt{com\_liferay\_portal\_search\_web\_folder\_facet\_portlet\_FolderFacetPortlet} \\
Modified Facet &
\texttt{com\_liferay\_portal\_search\_web\_modified\_facet\_portlet\_ModifiedFacetPortlet} \\
Search Bar &
\texttt{com\_liferay\_portal\_search\_web\_search\_bar\_portlet\_SearchBarPortlet} \\
Search Insights &
\texttt{com\_liferay\_portal\_search\_web\_search\_insights\_portlet\_SearchInsightsPortlet} \\
Search Options &
\texttt{com\_liferay\_portal\_search\_web\_search\_options\_portlet\_SearchOptionsPortlet} \\
Search Results &
\texttt{com\_liferay\_portal\_search\_web\_search\_results\_portlet\_SearchResultsPortlet} \\
Site Facet &
\texttt{com\_liferay\_portal\_search\_web\_site\_facet\_portlet\_SiteFacetPortlet} \\
Suggestions &
\texttt{com\_liferay\_portal\_search\_web\_suggestions\_portlet\_SuggestionsPortlet} \\
Tag Facet &
\texttt{com\_liferay\_portal\_search\_web\_tag\_facet\_portlet\_TagFacetPortlet} \\
Type Facet &
\texttt{com\_liferay\_portal\_search\_web\_type\_facet\_portlet\_TypeFacetPortlet} \\
User Facet &
\texttt{com\_liferay\_portal\_search\_web\_user\_facet\_portlet\_UserFacetPortlet} \\
\end{longtable}

\noindent\hrulefill

\textbf{Social}

\noindent\hrulefill

\begin{longtable}[]{@{}
  >{\raggedright\arraybackslash}p{(\columnwidth - 2\tabcolsep) * \real{0.5000}}
  >{\raggedright\arraybackslash}p{(\columnwidth - 2\tabcolsep) * \real{0.5000}}@{}}
\toprule\noalign{}
\begin{minipage}[b]{\linewidth}\raggedright
Portlet
\end{minipage} & \begin{minipage}[b]{\linewidth}\raggedright
ID
\end{minipage} \\
\midrule\noalign{}
\endhead
\bottomrule\noalign{}
\endlastfoot
Activities &
\texttt{com\_liferay\_social\_activities\_web\_portlet\_SocialActivitiesPortlet} \\
Contacts Center &
\texttt{com\_liferay\_contacts\_web\_portlet\_ContactsCenterPortlet} \\
Members &
\texttt{com\_liferay\_social\_networking\_web\_members\_portlet\_MembersPortlet} \\
My Contacts &
\texttt{com\_liferay\_contacts\_web\_portlet\_MyContactsPortlet} \\
Profile &
\texttt{com\_liferay\_contacts\_web\_portlet\_ProfilePortlet} \\
\end{longtable}

\noindent\hrulefill

\textbf{Tools}

\noindent\hrulefill

\begin{longtable}[]{@{}
  >{\raggedright\arraybackslash}p{(\columnwidth - 2\tabcolsep) * \real{0.5000}}
  >{\raggedright\arraybackslash}p{(\columnwidth - 2\tabcolsep) * \real{0.5000}}@{}}
\toprule\noalign{}
\begin{minipage}[b]{\linewidth}\raggedright
Portlet
\end{minipage} & \begin{minipage}[b]{\linewidth}\raggedright
ID
\end{minipage} \\
\midrule\noalign{}
\endhead
\bottomrule\noalign{}
\endlastfoot
Language Selector &
\texttt{com\_liferay\_site\_navigation\_language\_web\_portlet\_SiteNavigationLanguagePortlet} \\
Search &
\texttt{com\_liferay\_portal\_search\_web\_portlet\_SearchPortlet} \\
Sign In & \texttt{com\_liferay\_login\_web\_portlet\_LoginPortlet} \\
\end{longtable}

\noindent\hrulefill

\textbf{Wiki}

\noindent\hrulefill

\begin{longtable}[]{@{}
  >{\raggedright\arraybackslash}p{(\columnwidth - 2\tabcolsep) * \real{0.5000}}
  >{\raggedright\arraybackslash}p{(\columnwidth - 2\tabcolsep) * \real{0.5000}}@{}}
\toprule\noalign{}
\begin{minipage}[b]{\linewidth}\raggedright
Portlet
\end{minipage} & \begin{minipage}[b]{\linewidth}\raggedright
ID
\end{minipage} \\
\midrule\noalign{}
\endhead
\bottomrule\noalign{}
\endlastfoot
Page Menu &
\texttt{com\_liferay\_wiki\_navigation\_web\_portlet\_WikiNavigationPageMenuPortlet} \\
Tree Menu &
\texttt{com\_liferay\_wiki\_navigation\_web\_portlet\_WikiNavigationTreeMenuPortlet} \\
Wiki & \texttt{com\_liferay\_wiki\_web\_portlet\_WikiPortlet} \\
Wiki Display &
\texttt{com\_liferay\_wiki\_web\_portlet\_WikiDisplayPortlet} \\
\end{longtable}

\chapter{Available SPA Lifecycle
Events}\label{available-spa-lifecycle-events}

During development, you may need to know when navigation has started or
stopped in your SPA. SennaJS makes this easy by exposing lifecycle
events that represent state changes in the application. The available
lifecycle events are listed in the table below:

\noindent\hrulefill

\begin{longtable}[]{@{}
  >{\raggedright\arraybackslash}p{(\columnwidth - 4\tabcolsep) * \real{0.3333}}
  >{\raggedright\arraybackslash}p{(\columnwidth - 4\tabcolsep) * \real{0.3333}}
  >{\raggedright\arraybackslash}p{(\columnwidth - 4\tabcolsep) * \real{0.3333}}@{}}
\toprule\noalign{}
\begin{minipage}[b]{\linewidth}\raggedright
Event
\end{minipage} & \begin{minipage}[b]{\linewidth}\raggedright
Description
\end{minipage} & \begin{minipage}[b]{\linewidth}\raggedright
Ex payload
\end{minipage} \\
\midrule\noalign{}
\endhead
\bottomrule\noalign{}
\endlastfoot
\texttt{beforeNavigate} & Fires before navigation starts. This event
passes a JSON object with the path to the content you are navigating to
and whether to update the history. &
\texttt{\{\ path:\ \textquotesingle{}/pages/page1.html\textquotesingle{},\ replaceHistory:\ false\ \}} \\
\texttt{startNavigate} & Fires when navigation begins & `\{ form: ' \\
\end{longtable}

\noindent\hrulefill replaceHistory: false \}` \textbar{}

\noindent\hrulefill

\texttt{endNavigate} \textbar{} Fired after the content has been
retrieved and inserted onto

\noindent\hrulefill the page \textbar{}
\texttt{\{\ form:\ \textquotesingle{}\textless{}form\ name="form"\textgreater{}\textless{}/form\textgreater{}\textquotesingle{},\ path:\ \textquotesingle{}/pages/page1.html\textquotesingle{}\ \}}
\textbar{}

These events can be leveraged easily by listening for them on the
Liferay global object. For example, the JavaScript below alerts the user
to ``Get ready to navigate to'' the URL that has been clicked, just
before SPA navigation begins:

\begin{verbatim}
Liferay.on('beforeNavigate', function(event) {
    alert("Get ready to navigate to " + event.path);
});
\end{verbatim}

The alert takes advantage of the payload for the \texttt{beforeNavigate}
event, retrieving the URL from the \texttt{path} attribute of the JSON
payload object.

\begin{figure}
\centering
\includegraphics{./images/private-messaging-before-navigate.png}
\caption{You can leverage SPA lifecycle events in your apps.}
\end{figure}

\chapter{Theme Anatomy Reference
Guide}\label{theme-anatomy-reference-guide}

A theme is made up of several files. Although most of the files are
named after their matching components, their functions may be unclear.
This reference guide explains each file's usage to make clear which
files to modify.

Themes built with the
\href{https://github.com/liferay/liferay-js-themes-toolkit/tree/master/packages}{Liferay
JS Theme Toolkit} have the anatomy shown below:

\begin{itemize}
\tightlist
\item
  \texttt{theme-name/}

  \begin{itemize}
  \tightlist
  \item
    \texttt{src/}

    \begin{itemize}
    \tightlist
    \item
      \texttt{css/}

      \begin{itemize}
      \tightlist
      \item
        \href{/docs/7-2/reference/-/knowledge_base/r/theme-reference-guide\#-clay-customscss}{\texttt{\_clay\_custom.scss}}
      \item
        \href{/docs/7-2/reference/-/knowledge_base/r/theme-reference-guide\#-clay-variablesscss}{\texttt{\_clay\_variables.scss}}
      \item
        \href{/docs/7-2/reference/-/knowledge_base/r/theme-reference-guide\#-customscss}{\texttt{\_custom.scss}}
      \item
        \href{/docs/7-2/reference/-/knowledge_base/r/theme-reference-guide\#-liferay-variables-customscss}{\texttt{\_liferay\_variables\_custom.scss}}
      \end{itemize}
    \item
      \texttt{images/}

      \begin{itemize}
      \tightlist
      \item
        (custom images)
      \end{itemize}
    \item
      \texttt{js/}

      \begin{itemize}
      \tightlist
      \item
        \href{/docs/7-2/reference/-/knowledge_base/r/theme-reference-guide\#mainjs}{\texttt{main.js}}
      \end{itemize}
    \item
      \texttt{templates/}

      \begin{itemize}
      \tightlist
      \item
        \href{/docs/7-2/reference/-/knowledge_base/r/theme-reference-guide\#init-customftl}{\texttt{init\_custom.ftl}}
      \item
        \href{/docs/7-2/reference/-/knowledge_base/r/theme-reference-guide\#navigationftl}{\texttt{navigation.ftl}}
      \item
        \href{/docs/7-2/reference/-/knowledge_base/r/theme-reference-guide\#portal-normalftl}{\texttt{portal\_normal.ftl}}
      \item
        \href{/docs/7-2/reference/-/knowledge_base/r/theme-reference-guide\#portal-pop-upftl}{\texttt{portal\_pop\_up.ftl}}
      \item
        \href{/docs/7-2/reference/-/knowledge_base/r/theme-reference-guide\#portletftl}{\texttt{portlet.ftl}}
      \end{itemize}
    \item
      \texttt{WEB-INF/}

      \begin{itemize}
      \tightlist
      \item
        \href{/docs/7-2/reference/-/knowledge_base/r/theme-reference-guide\#liferay-look-and-feelxml}{\texttt{liferay-look-and-feel.xml}}
      \item
        \href{/docs/7-2/reference/-/knowledge_base/r/theme-reference-guide\#liferay-plugin-packageproperties}{\texttt{liferay-plugin-package.properties}}
      \item
        \texttt{src/}

        \begin{itemize}
        \tightlist
        \item
          \texttt{resources-importer/}

          \begin{itemize}
          \tightlist
          \item
            (Many directories)
          \end{itemize}
        \end{itemize}
      \end{itemize}
    \end{itemize}
  \item
    \href{/docs/7-2/reference/-/knowledge_base/r/theme-reference-guide\#liferay-themejson}{\texttt{liferay-theme.json}}
  \item
    \href{/docs/7-2/reference/-/knowledge_base/r/theme-reference-guide\#packagejson}{\texttt{package.json}}
  \end{itemize}
\end{itemize}

Regarding CSS files, you should only modify
\texttt{\_clay\_custom.scss}, \texttt{\_clay\_variables.scss},
\texttt{\_custom.scss}, and \texttt{\_liferay\_variables\_custom.scss}.

You can of course overwrite any CSS file you want, but if you modify any
other files, you're removing styling that 7.0 needs to work properly.

\section{Theme Files}\label{theme-files}

\section{\_clay\_custom.scss}\label{clay_custom.scss}

Used for Clay custom styles, i.e.~styles for a third party Bootstrap
theme. Anything written in this file is compiled in the same scope as
Bootstrap/Lexicon, so you can use their variables, mixins, etc. You can
also implement any of the variables you define in
\texttt{\_clay\_variables.scss}.

\section{\_clay\_variables.scss}\label{clay_variables.scss}

Used to store custom Sass variables. This file gets injected into the
Bootstrap/Lexicon build, so you can overwrite variables and change how
those libraries are compiled.

\section{\_custom.scss}\label{custom.scss}

Used for custom CSS styles. You should place all of your custom CSS
modifications in this file.

\section{\_liferay\_variables\_custom.scss}\label{liferay_variables_custom.scss}

Used for overwriting variables defined in
\texttt{\_liferay\_variables.scss} without wiping out the whole file.

\section{init\_custom.ftl}\label{init_custom.ftl}

Used for custom FreeMarker variables i.e.~
\href{/docs/7-2/frameworks/-/knowledge_base/f/making-configurable-theme-settings}{theme
setting} variables.

\section{navigation.ftl}\label{navigation.ftl}

The theme template for the theme's navigation.

\section{portal\_normal.ftl}\label{portal_normal.ftl}

Similar to a static site's \texttt{index.html}, this file acts as a hub
for all theme templates.

\section{portal\_pop\_up.ftl}\label{portal_pop_up.ftl}

The theme template for pop up dialogs for the theme's portlets.

\section{portlet.ftl}\label{portlet.ftl}

The theme template for the theme's portlets. If your theme uses
\href{/docs/7-2/frameworks/-/knowledge_base/f/theming-portlets\#portlet-decorators}{Application
Decorators}, you can modify this file to create application
decorator-specific theme settings.

\section{liferay-theme.json}\label{liferay-theme.json}

Contains the configuration settings for your app server, in Node.js
tool-based themes. You can change this file manually at any time to
update your server settings. The file can also be updated via the
\href{/docs/7-2/frameworks/-/knowledge_base/f/updating-your-themes-app-server}{\texttt{gulp\ init}
task}.

\section{package.json}\label{package.json}

Contains theme setting information such as the theme template language,
version, and base theme, for Node.js tool developed themes. You can
update this file manually. The
\href{/docs/7-2/frameworks/-/knowledge_base/f/changing-your-base-theme}{\texttt{gulp\ extend}
task} can also be used to change the base theme.

\section{main.js}\label{main.js}

Used for custom JavaScript.

\section{liferay-look-and-feel.xml}\label{liferay-look-and-feel.xml}

Contains basic information for the theme. If your theme has
\href{/docs/7-2/frameworks/-/knowledge_base/f/making-configurable-theme-settings}{theme
settings}, they are defined in this file. For a full explanation of this
file, please see the
\href{https://docs.liferay.com/dxp/portal/7.2-latest/definitions/liferay-look-and-feel_7_2_0.dtd.html}{Definitions
docs}.

\section{liferay-plugin-package.properties}\label{liferay-plugin-package.properties}

Contains general properties for the theme.
\href{/docs/7-2/frameworks/-/knowledge_base/f/importing-resources-with-a-theme}{Resources
Importer} configuration settings are also placed in this file. For a
full explanation of the properties available for this file please see
the
\href{https://docs.liferay.com/dxp/portal/7.2-latest/propertiesdoc/liferay-plugin-package_7_2_0.properties.html}{7.2
Properties documentation}.

\chapter{Freemarker Variable Reference
Guide}\label{freemarker-variable-reference-guide}

By default, FreeMarker templates have access to several variables
defined in
\href{https://github.com/liferay/liferay-portal/blob/7.2.x/modules/apps/frontend-theme/frontend-theme-unstyled/src/main/resources/META-INF/resources/_unstyled/templates/init.ftl}{\texttt{init.ftl}}
that you can use in your
\href{/docs/7-2/frameworks/-/knowledge_base/f/themes-introduction}{themes}
to access several theme objects, settings, and resources. Several of
these variables are listed below for reference:

\textbf{Common Variables}

\noindent\hrulefill

\begin{longtable}[]{@{}
  >{\raggedright\arraybackslash}p{(\columnwidth - 2\tabcolsep) * \real{0.5000}}
  >{\raggedright\arraybackslash}p{(\columnwidth - 2\tabcolsep) * \real{0.5000}}@{}}
\toprule\noalign{}
\begin{minipage}[b]{\linewidth}\raggedright
Variable
\end{minipage} & \begin{minipage}[b]{\linewidth}\raggedright
Description
\end{minipage} \\
\midrule\noalign{}
\endhead
\bottomrule\noalign{}
\endlastfoot
\texttt{theme\_display} & Returns the \texttt{themeDisplay} Java Object
and all its methods \\
\texttt{portlet\_display} & Returns the \texttt{portletDisplay} Java
Object and all its methods \\
\texttt{layoutSet} & Returns the page set \\
\texttt{theme\_timestamp} & Prints the date in the current locale with
the given format \\
\texttt{theme\_settings} & Retrieves theme settings. See
\href{/docs/7-2/frameworks/-/knowledge_base/f/making-configurable-theme-settings}{configurable
theme settings} for more information. \\
\texttt{root\_css\_class} & Returns the root CSS class which indicates
the direction of the page (\texttt{ltr} (left-to-right) by default) \\
\texttt{css\_class} & Returns a string of the current classes applied to
the body of the page \\
\texttt{page\_group} & Retrieves the page group \\
\texttt{css\_folder} & Returns the path to the theme's \texttt{css}
folder \\
\texttt{images\_folder} & Returns the path to the theme's
\texttt{images} folder \\
\texttt{javascript\_folder} & Returns the path to the theme's
\texttt{javascript} folder \\
\texttt{templates\_folder} & Returns the path to the theme's
\texttt{templates} folder \\
\texttt{full\_css\_path} & Returns the full path, which includes the
servlet context, to the theme's \texttt{css} \\
\texttt{full\_templates\_path} & returns the full path, which includes
the servlet context, to the theme's \texttt{templates} \\
\texttt{css\_main\_file} & Returns the path to \texttt{main.css} \\
\texttt{js\_main\_file} & Returns the path to \texttt{main.js} \\
\texttt{company\_id} & Returns the company ID \\
\texttt{company\_name} & Returns the company name \\
\texttt{company\_logo} & Returns the company logo's URL \\
\texttt{company\_logo\_height} & Returns the company logo's height \\
\texttt{company\_logo\_width} & Returns the company logo's width \\
\texttt{company\_url} & Returns the URL of the home page for the
company \\
\texttt{time\_zone} & Returns the time zone for the current user \\
\texttt{is\_login\_redirect\_required} & Returns whether a login
redirect is required for the user \\
\texttt{is\_signed\_in} & Returns whether the user is signed in \\
\texttt{group\_id} & Returns the group ID for the current user \\
\texttt{time\_zone} & Returns the time zone for the current user \\
\texttt{is\_default\_user} & Returns if the user has a default role \\
\texttt{is\_female} & Returns if the current user is Female \\
\texttt{is\_male} & Returns if the current user is Male \\
\texttt{is\_setup\_complete} & Returns whether the user has configured
their profile \\
\texttt{language} & Returns the native language for the current user \\
\texttt{language\_id} & Returns the ID of the current locale \\
\texttt{user\_birthday} & Returns the current user's birthday \\
\texttt{user\_comments} & Returns comments from the user's profile \\
\texttt{user\_email\_address} & Returns the user's email address \\
\texttt{user\_first\_name} & Returns the user's first name \\
\texttt{user\_greeting} & Returns the user's greeting \\
\texttt{user\_id} & Returns the ID of the current user \\
\texttt{user\_last\_login\_ip} & Returns the IP address that the user
last logged in from \\
\texttt{user\_last\_name} & Returns the last name of the current user \\
\texttt{user\_login\_ip} & Returns the current user's current IP
address \\
\texttt{user\_middle\_name} & Returns the user's middle name \\
\texttt{user\_name} & Returns the current user's username \\
\texttt{w3c\_language\_id} & Returns the W3C language code of the
current language \\
\end{longtable}

\noindent\hrulefill

\textbf{URLs}

\noindent\hrulefill

\begin{longtable}[]{@{}
  >{\raggedright\arraybackslash}p{(\columnwidth - 2\tabcolsep) * \real{0.5000}}
  >{\raggedright\arraybackslash}p{(\columnwidth - 2\tabcolsep) * \real{0.5000}}@{}}
\toprule\noalign{}
\begin{minipage}[b]{\linewidth}\raggedright
Variable
\end{minipage} & \begin{minipage}[b]{\linewidth}\raggedright
Description
\end{minipage} \\
\midrule\noalign{}
\endhead
\bottomrule\noalign{}
\endlastfoot
\texttt{show\_control\_panel} & Returns whether the current user has
permission to view the Control Panel \\
\texttt{control\_panel\_text} & Returns the ``control-panel'' language
key in the current user's locale, if they have permission to view the
Control Panel \\
\texttt{control\_panel\_url} & Returns the URL to the Control Panel, if
the current user has permission to view the Control Panel \\
\texttt{show\_home} & Returns whether the current user is on a page \\
\texttt{home\_text} & Returns the ``home'' language key in the current
user's locale \\
\texttt{home\_url} & Returns the URL to the home page \\
\texttt{show\_my\_account} & Returns whether the current user's account
icon is visible \\
\texttt{my\_account\_text} & Returns the ``my-account'' language key in
the current user's locale, if the user's account icon is visible \\
\texttt{my\_account\_url} & Returns the URL to the user's Account
Settings page if the user's account icon is visible \\
\texttt{show\_sign\_in} & Returns whether the sign in link is visible \\
\texttt{sign\_in\_text} & Returns the ``sign-in'' language key in the
current user's locale, if they are signed out \\
\texttt{sign\_in\_url} & Returns the sign in URL, if the current user is
signed out \\
\texttt{show\_sign\_out} & Returns whether the sign out link is
visible \\
\texttt{sign\_out\_text} & Returns the ``sign-out'' language key in the
current user's locale, if they are signed in \\
\texttt{sign\_out\_url} & Returns the sign out URL, if the current user
is signed in \\
\end{longtable}

\noindent\hrulefill

\textbf{Page}

\noindent\hrulefill

\begin{longtable}[]{@{}
  >{\raggedright\arraybackslash}p{(\columnwidth - 2\tabcolsep) * \real{0.5000}}
  >{\raggedright\arraybackslash}p{(\columnwidth - 2\tabcolsep) * \real{0.5000}}@{}}
\toprule\noalign{}
\begin{minipage}[b]{\linewidth}\raggedright
Variable
\end{minipage} & \begin{minipage}[b]{\linewidth}\raggedright
Description
\end{minipage} \\
\midrule\noalign{}
\endhead
\bottomrule\noalign{}
\endlastfoot
\texttt{the\_title} & Returns the current page's title \\
\texttt{selectable} & Returns whether the current page is selectable \\
\texttt{is\_maximized} & Returns whether the page is maximized \\
\texttt{page} & Returns the current page (layout) \\
\texttt{is\_first\_child} & Returns whether the current page is the
first child page in the navigation \\
\texttt{is\_first\_parent} & Returns whether the current page is the
first parent page in the navigation \\
\texttt{is\_portlet\_page} & Returns whether the current page is a
widget page (portlet) \\
\texttt{site\_name} & Returns the site's name \\
\texttt{is\_guest\_group} & Returns whether the current page group is
for guests \\
\texttt{site\_type} & Returns the type of the current site: site,
company site, organization site, or user site \\
\texttt{site\_default\_url} & Returns the default URL for the site \\
\texttt{layout\_friendly\_url} & Returns the friendly URL of the current
page \\
\texttt{portlet\_id} & Returns the portlet ID for the specified
portlet \\
\end{longtable}

\noindent\hrulefill

\textbf{Logo}

\noindent\hrulefill

\begin{longtable}[]{@{}
  >{\raggedright\arraybackslash}p{(\columnwidth - 2\tabcolsep) * \real{0.5000}}
  >{\raggedright\arraybackslash}p{(\columnwidth - 2\tabcolsep) * \real{0.5000}}@{}}
\toprule\noalign{}
\begin{minipage}[b]{\linewidth}\raggedright
Variable
\end{minipage} & \begin{minipage}[b]{\linewidth}\raggedright
Description
\end{minipage} \\
\midrule\noalign{}
\endhead
\bottomrule\noalign{}
\endlastfoot
\texttt{logo\_css\_class} & Returns a string of the current classes
applied to the logo. \\
\texttt{use\_company\_logo} & Returns whether the logo is displayed \\
\texttt{site\_logo\_height} & Returns the logo's height \\
\texttt{site\_logo\_width} & Returns the logo's width \\
\texttt{show\_site\_name\_supported} & Returns whether the logo is
configured to show the site name. The value is \texttt{true} if
\texttt{show\_site\_name\_default} is true. \\
\texttt{show\_site\_name\_default} & Returns whether the Show Site Name
Default theme setting is enabled \\
\texttt{show\_site\_name} & Returns whether the \texttt{showSiteName}
property for the current pageset is enabled \\
\texttt{logo\_description} & Returns the Site's name or nothing if
\texttt{show\_site\_name} is enabled. It is used for alternate text for
the logo by default. \\
\end{longtable}

\noindent\hrulefill

\textbf{Navigation}

\noindent\hrulefill

\begin{longtable}[]{@{}
  >{\raggedright\arraybackslash}p{(\columnwidth - 2\tabcolsep) * \real{0.5000}}
  >{\raggedright\arraybackslash}p{(\columnwidth - 2\tabcolsep) * \real{0.5000}}@{}}
\toprule\noalign{}
\begin{minipage}[b]{\linewidth}\raggedright
Variable
\end{minipage} & \begin{minipage}[b]{\linewidth}\raggedright
Description
\end{minipage} \\
\midrule\noalign{}
\endhead
\bottomrule\noalign{}
\endlastfoot
\texttt{has\_navigation} & Returns whether navigation exists (i.e.~at
least one page exists) \\
\texttt{nav\_items} & Returns the current pages as list \\
\texttt{nav\_css\_class} & Returns a string of the current classes
applied to the page's navigation \\
\end{longtable}

\noindent\hrulefill

\textbf{My Sites}

\noindent\hrulefill

\begin{longtable}[]{@{}
  >{\raggedright\arraybackslash}p{(\columnwidth - 2\tabcolsep) * \real{0.5000}}
  >{\raggedright\arraybackslash}p{(\columnwidth - 2\tabcolsep) * \real{0.5000}}@{}}
\toprule\noalign{}
\begin{minipage}[b]{\linewidth}\raggedright
Variable
\end{minipage} & \begin{minipage}[b]{\linewidth}\raggedright
Description
\end{minipage} \\
\midrule\noalign{}
\endhead
\bottomrule\noalign{}
\endlastfoot
\texttt{show\_my\_sites} & Returns whether the current user has a My
Sites page \\
\texttt{show\_my\_places} & Returns whether the current user has a My
Sites page \\
\texttt{my\_sites\_text} & Returns the ``my-sites'' language key in the
current user's locale \\
\texttt{my\_places\_text} & Returns whether the current user has a My
Sites page \\
\end{longtable}

\noindent\hrulefill

\textbf{Includes}

\noindent\hrulefill

\begin{longtable}[]{@{}
  >{\raggedright\arraybackslash}p{(\columnwidth - 2\tabcolsep) * \real{0.5000}}
  >{\raggedright\arraybackslash}p{(\columnwidth - 2\tabcolsep) * \real{0.5000}}@{}}
\toprule\noalign{}
\begin{minipage}[b]{\linewidth}\raggedright
Variable
\end{minipage} & \begin{minipage}[b]{\linewidth}\raggedright
Description
\end{minipage} \\
\midrule\noalign{}
\endhead
\bottomrule\noalign{}
\endlastfoot
\texttt{dir\_include} & Returns ``/html'' \\
\texttt{body\_bottom\_include} & Returns
``\({dir_include}/common/themes/body_bottom.jsp" |
 `body_top_include` | Returns "\)\{dir\_include\}/common/themes/body\_top.jsp'' \\
\texttt{bottom\_include} & Returns
``\({dir_include}/common/themes/bottom.jsp" |
 `top_head_include` | Returns "\)\{dir\_include\}/common/themes/top\_head.jsp'' \\
\texttt{top\_messages\_include} & Returns
``\$\{dir\_include\}/common/themes/top\_messages.jsp'' \\
\end{longtable}

\noindent\hrulefill

\textbf{Date}

\noindent\hrulefill

\begin{longtable}[]{@{}
  >{\raggedright\arraybackslash}p{(\columnwidth - 2\tabcolsep) * \real{0.5000}}
  >{\raggedright\arraybackslash}p{(\columnwidth - 2\tabcolsep) * \real{0.5000}}@{}}
\toprule\noalign{}
\begin{minipage}[b]{\linewidth}\raggedright
Variable
\end{minipage} & \begin{minipage}[b]{\linewidth}\raggedright
Description
\end{minipage} \\
\midrule\noalign{}
\endhead
\bottomrule\noalign{}
\endlastfoot
\texttt{date} & Gives access to the \texttt{dateUtil} Java Object and
all its methods \\
\texttt{current\_time} & Returns the current time \\
\texttt{the\_year} & Returns the current year \\
\end{longtable}

\chapter{Gradle Plugins}\label{gradle-plugins}

Liferay provides plugins that you can apply to your Gradle project. This
reference documentation describes how to apply and use Liferay's Gradle
plugins.

\textbf{Important:} If you're using
\href{/docs/7-2/reference/-/knowledge_base/r/liferay-workspace}{Liferay
Workspace} to create Liferay apps, most of the Liferay Gradle plugins
covered in this section are already applied by default. The
\href{https://github.com/liferay/liferay-portal/tree/master/modules/sdk/gradle-plugins-workspace}{com.liferay.gradle.plugins.workspace}
and
\href{https://github.com/liferay/liferay-portal/tree/master/modules/sdk/gradle-plugins}{com.liferay.gradle.plugins}
dependencies provide them, both of which are preset in workspace by
default.

Do not apply a Liferay Gradle plugin to an app that already has access
to it.

Each article in this section describes how to apply the plugin, what
Gradle tasks the plugin provides, the plugin's configuration properties,
and the plugin's dependencies.

\chapter{App Javadoc Builder Gradle
Plugin}\label{app-javadoc-builder-gradle-plugin}

The App Javadoc Builder Gradle plugin lets you generate API
documentation as a single, combined HTML document for an application
that spans different subprojects, each one representing a different
component of the same application.

The plugin has been successfully tested with Gradle 4.10.2.

\section{Usage}\label{usage}

To use the plugin, include it in the build script of the root project:

\begin{verbatim}
buildscript {
    dependencies {
        classpath group: "com.liferay", name: "com.liferay.gradle.plugins.app.javadoc.builder", version: "1.2.2"
    }

    repositories {
        maven {
            url "https://repository-cdn.liferay.com/nexus/content/groups/public"
        }
    }
}

apply plugin: "com.liferay.app.javadoc.builder"
\end{verbatim}

The App Javadoc Builder plugin automatically applies the
\href{https://docs.gradle.org/current/userguide/standard_plugins.html\#N135C1}{\texttt{base}}
and \texttt{reporting-base} plugins.

\section{Project Extension}\label{project-extension}

The App Javadoc Builder plugin exposes the following properties through
the extension named \texttt{appJavadocBuilder}:

Property Name \textbar{} Type \textbar{} Default Value \textbar{}
Description \texttt{copyTags} \textbar{} \texttt{boolean} \textbar{}
\texttt{true} \textbar{} Whether to copy the custom block tags
configuration from the subprojects. It sets the Javadoc
\href{http://docs.oracle.com/javase/8/docs/technotes/tools/windows/javadoc.html\#tag}{\texttt{-tag}}
argument for the \hyperref[appjavadoc]{\texttt{appJavadoc}} task.
\texttt{doclintDisabled} \textbar{} \texttt{boolean} \textbar{}
\texttt{true} on JDK8+, \texttt{false} otherwise. \textbar{} Whether to
ignore Javadoc errors. It sets the Javadoc
\href{docs.oracle.com/javase/8/docs/technotes/tools/windows/javadoc.html\#BEJEFABE}{\texttt{-Xdoclint}}
and
\href{http://docs.oracle.com/javase/8/docs/technotes/tools/windows/javadoc.html\#CHDGFHAA}{\texttt{-quiet}}
arguments for the \hyperref[appjavadoc]{\texttt{appJavadoc}} task.
\texttt{groupNameClosure} \textbar{}
\texttt{Closure\textless{}String\textgreater{}} \textbar{} The
subproject's description, or the subproject's name if the description is
empty. \textbar{} The closure invoked in order to get the group heading
for a subproject. The given closure is passed a
\href{https://docs.gradle.org/current/javadoc/org/gradle/api/Project.html}{\texttt{Project}}
as its parameter. If \texttt{groupPackages} is \texttt{false}, this
property is not used. \texttt{groupPackages} \textbar{} \texttt{boolean}
\textbar{} \texttt{true} \textbar{} Whether to separate packages on the
overview page based on the subprojects they belong to. It sets the
\href{docs.oracle.com/javase/8/docs/technotes/tools/unix/javadoc.html\#CHDIGGII}{\texttt{-group}}
argument for the \hyperref[appjavadoc]{\texttt{appJavadoc}} task.
\texttt{subprojects} \textbar{}
\texttt{Set\textless{}Project\textgreater{}} \textbar{}
\texttt{project.subprojects} \textbar{} The subprojects to include in
the API documentation of the app.

The same extension exposes the following methods:

Method \textbar{} Description
\texttt{AppJavadocBuilderExtension\ onlyIf(Closure\textless{}Boolean\textgreater{}\ onlyIfClosure)}
\textbar{} Includes a subproject in the API documentation if the given
closure returns \texttt{true}. The closure is evaluated at the end of
the subproject configuration phase and is passed a single parameter: the
subproject. If the closure returns \texttt{false}, the subproject is not
included in the API documentation.
\texttt{AppJavadocBuilderExtension\ onlyIf(Spec\textless{}Project\textgreater{}\ onlyIfSpec)}
\textbar{} Includes a subproject in the API documentation if the given
spec is satisfied. The spec is evaluated at the end of the subproject
configuration phase. If the spec is not satisfied, the subproject is not
included in the API documentation.
\texttt{AppJavadocBuilderExtension\ subprojects(Iterable\textless{}Project\textgreater{}\ subprojects)}
\textbar{} Include additional projects in the API documentation of the
app.
\texttt{AppJavadocBuilderExtension\ subprojects(Project...\ subprojects)}
\textbar{} Include additional projects in the API documentation of the
app.

\section{Tasks}\label{tasks}

The plugin adds two tasks to your project:

Name \textbar{} Depends On \textbar{} Type \textbar{} Description
\texttt{appJavadoc} \textbar{} The \texttt{javadoc} tasks of the
subprojects. \textbar{}
\href{https://docs.gradle.org/current/dsl/org.gradle.api.tasks.javadoc.Javadoc.html}{\texttt{Javadoc}}
\textbar{} Generates Javadoc API documentation for the app.
\texttt{jarAppJavadoc} \textbar{} \texttt{appJavadoc} \textbar{}
\href{https://docs.gradle.org/current/dsl/org.gradle.api.tasks.bundling.Jar.html}{\texttt{Jar}}
\textbar{} Assembles a JAR archive containing the Javadoc files for this
app.

The \texttt{appJavadoc} task is automatically configured with sensible
defaults:

Property Name \textbar{} Default Value
\href{https://docs.gradle.org/current/dsl/org.gradle.api.tasks.javadoc.Javadoc.html\#org.gradle.api.tasks.javadoc.Javadoc:classpath}{\texttt{classpath}}
\textbar{} The \texttt{javadoc.classpath} of all the subprojects.
\href{https://docs.gradle.org/current/dsl/org.gradle.api.tasks.javadoc.Javadoc.html\#org.gradle.api.tasks.javadoc.Javadoc:destinationDir}{\texttt{destinationDir}}
\textbar{} \texttt{\$\{project.buildDir\}/docs/javadoc}
\href{https://docs.gradle.org/current/javadoc/org/gradle/external/javadoc/MinimalJavadocOptions.html\#getEncoding()}{\texttt{options.encoding}}
\textbar{} \texttt{"UTF-8"}
\href{https://docs.gradle.org/current/dsl/org.gradle.api.tasks.javadoc.Javadoc.html\#org.gradle.api.tasks.javadoc.Javadoc:source}{\texttt{source}}
\textbar{} The \texttt{javadoc.source} of all the subprojects.
\href{https://docs.gradle.org/current/dsl/org.gradle.api.tasks.javadoc.Javadoc.html\#org.gradle.api.tasks.javadoc.Javadoc:title}{\texttt{title}}
\textbar{} \texttt{project.reporting.apiDocTitle}

\chapter{Baseline Gradle Plugin}\label{baseline-gradle-plugin}

The Baseline Gradle plugin lets you verify that the OSGi
\href{http://semver.org/}{semantic versioning} rules are obeyed by your
OSGi bundle.

When you run the \hyperref[baseline]{\texttt{baseline}} task, the plugin
\emph{baselines} the new bundle against the latest released non-snapshot
bundle (i.e., the \emph{baseline}). That is, it compares the public
exported API of the new bundle with the baseline. If there are any
changes, it uses the OSGi semantic versioning rules to calculate the
minimum new version. If the new bundle has a lower version, errors are
thrown.

The plugin has been successfully tested with Gradle 4.10.2.

\section{Usage}\label{usage-1}

To use the plugin, include it in your build script:

\begin{verbatim}
buildscript {
    dependencies {
        classpath group: "com.liferay", name: "com.liferay.gradle.plugins.baseline", version: "2.1.0"
    }

    repositories {
        maven {
            url "https://repository-cdn.liferay.com/nexus/content/groups/public"
        }
    }
}

apply plugin: "com.liferay.baseline"
\end{verbatim}

The Baseline plugin automatically applies the
\href{https://docs.gradle.org/current/userguide/java_plugin.html}{\texttt{java}}
and
\href{https://docs.gradle.org/current/userguide/standard_plugins.html\#sec:base_plugins}{\texttt{reporting-base}}
plugins.

Since the plugin needs to download the baseline, you have to configure a
\href{https://docs.gradle.org/current/userguide/artifact_dependencies_tutorial.html\#sec:repositories_tutorial}{repository}
that hosts it; for example, the central Maven 2 repository:

\begin{verbatim}
repositories {
    mavenCentral()
}
\end{verbatim}

\section{Project Extension}\label{project-extension-1}

The Baseline plugin exposes the following properties through the
\texttt{baselineConfiguration} extension:

Property Name \textbar{} Type \textbar{} Default Value \textbar{}
Description \texttt{allowMavenLocal} \textbar{} \texttt{boolean}
\textbar{} \texttt{false} \textbar{} Whether to let the baseline come
from the local Maven cache (by default: \texttt{\$\{user.home\}/.m2}).
If the local Maven cache is not
\href{https://docs.gradle.org/current/userguide/dependency_management.html\#sub:maven_local}{configured}
as a project repository, this property has no effect.
\texttt{lowestBaselineVersion} \textbar{} \texttt{String} \textbar{}
\texttt{"1.0.0"} \textbar{} The greatest project version to ignore for
the baseline check. If the
\href{https://docs.gradle.org/current/dsl/org.gradle.api.tasks.bundling.Jar.html\#org.gradle.api.tasks.bundling.Jar:version}{project
version} is less than or equal to the value of this property, the
\hyperref[baseline]{\texttt{baseline}} task is skipped.
\texttt{lowestMajorVersion} \textbar{} \texttt{Integer} \textbar{}
Content of the file
\texttt{\$\{project.projectDir\}/.lfrbuild-lowest-major-version}, where
the default file name can be changed by setting the project property
\texttt{baseline.lowest.major.version.file}. \textbar{} The lowest major
version of the released artifact to use in the baseline check.
\texttt{lowestMajorVersionRequired} \textbar{} \texttt{boolean}
\textbar{} \texttt{false} \textbar{} Whether to fail the build if the
\hyperref[lowestmajorversion]{\texttt{lowestMajorVersion}} is not
specified.

If the \texttt{lowestMajorVersion} is not specified, the plugin runs the
check using the most recent released non-snapshot bundle as baseline,
which matches the
\href{http://ant.apache.org/ivy/history/latest-milestone/settings/version-matchers.html}{version
range} \texttt{(,\$\{project.version\})}. Otherwise, if the
\texttt{lowestMajorVersion} is equal to a value \texttt{L} and the
project has version \texttt{M.x.y} (with \texttt{L} less or equal than
\texttt{M}), multiple checks are performed in order, using the following
version ranges as baseline:

\begin{enumerate}
\def\labelenumi{\arabic{enumi}.}
\tightlist
\item
  \texttt{{[}L.0.0,\ (L\ +\ 1).0.0)}
\item
  \texttt{{[}(L\ +\ 1).0.0,\ (L\ +\ 2).0.0)}
\item
  \ldots{}
\item
  \texttt{{[}(M\ -\ 2).0.0,\ (M\ -\ 1).0.0)}
\item
  \texttt{{[}(M\ -\ 1).0.0,\ M.0.0)}
\item
  \texttt{{[}M.0.0,\ M.x.y)}
\end{enumerate}

The first failing check fails the whole build.

\section{Tasks}\label{tasks-1}

The plugin adds one task to your project:

Name \textbar{} Depends On \textbar{} Type \textbar{} Description
\texttt{baseline} \textbar{}
\href{(https://docs.gradle.org/current/userguide/java_plugin.html\#sec:jar)}{\texttt{jar}}
\textbar{} \hyperref[baselinetask]{\texttt{BaselineTask}} \textbar{}
Compares the public API of this project with the public API of the
previous released version, if found.

The \texttt{baseline} task is automatically configured with sensible
defaults:

Property Name \textbar{} Default Value
\hyperref[baselineconfiguration]{\texttt{baselineConfiguration}}
\textbar{}
\hyperref[baseline-dependency]{\texttt{configurations.baseline}}
\hyperref[bndfile]{\texttt{bndFile}} \textbar{}
\texttt{\$\{project.projectDir\}/bnd.bnd}
\hyperref[newjarfile]{\texttt{newJarFile}} \textbar{}
\href{https://docs.gradle.org/current/dsl/org.gradle.api.tasks.bundling.Jar.html\#org.gradle.api.tasks.bundling.Jar:archivePath}{\texttt{project.tasks.jar.archivePath}}
\hyperref[sourcedir]{\texttt{sourceDir}} \textbar{} The first
\texttt{resources} directory of the \texttt{main} source set (by
default: \texttt{src/main/resources}).

\section{BaselineTask}\label{baselinetask}

\subsection{Task Properties}\label{task-properties}

Property Name \textbar{} Type \textbar{} Default Value \textbar{}
Description \texttt{baselineConfiguration} \textbar{}
\texttt{Configuration} \textbar{} \texttt{null} \textbar{} The
configuration that contains exactly one dependency to the baseline
bundle. \texttt{bndFile} \textbar{} \texttt{File} \textbar{}
\texttt{null} \textbar{} The BND file of the project. If provided, the
task will automatically update the
\href{http://bnd.bndtools.org/heads/bundle_version.html}{\texttt{Bundle-Version}}
header. \texttt{forceCalculatedVersion} \textbar{} \texttt{boolean}
\textbar{} \texttt{false} \textbar{} Whether to fail the baseline check
if the \texttt{Bundle-Version} has been excessively increased.
\texttt{ignoreExcessiveVersionIncreases} \textbar{} \texttt{boolean}
\textbar{} \texttt{false} \textbar{} Whether to ignore excessive package
version increase warnings. \texttt{ignoreFailures} \textbar{}
\texttt{boolean} \textbar{} \texttt{false} \textbar{} Whether the build
should not break when semantic versioning errors are found.
\texttt{logFile} \textbar{} \texttt{File} \textbar{} \texttt{null}
\textbar{} The file to which the results of the baseline check are
written. \emph{(Read-only)} \texttt{logFileName} \textbar{}
\texttt{String} \textbar{} \texttt{"baseline/\$\{task.name\}.log"}
\textbar{} The name of the file to which the results of the baseline
check are written. If the \texttt{reporting-base} plugin is applied, the
file name is relative to
\href{https://docs.gradle.org/current/dsl/org.gradle.api.reporting.ReportingExtension.html\#org.gradle.api.reporting.ReportingExtension:baseDir}{\texttt{reporting.baseDir}};
otherwise, it's relative to the project directory. \texttt{newJarFile}
\textbar{} \texttt{File} \textbar{} \texttt{null} \textbar{} The file of
the new OSGi bundle. \texttt{reportDiff} \textbar{} \texttt{boolean}
\textbar{} \texttt{true} if the project property
\texttt{baseline.jar.report.level} has either value \texttt{"diff"} or
\texttt{"persist"}; \texttt{false} otherwise \textbar{} Whether to show
a granular, differential report of all changes that occurred in the
exported packages of the OSGi bundle. \texttt{reportOnlyDirtyPackages}
\textbar{} \texttt{boolean} \textbar{} Value of the project property
\texttt{baseline.jar.report.only.dirty.packages} if specified;
\texttt{true} otherwise. \textbar{} Whether to show only packages with
API changes in the report. \texttt{sourceDir} \textbar{} \texttt{File}
\textbar{} \texttt{null} \textbar{} The directory to which the
\href{http://bnd.bndtools.org/chapters/170-versioning.html\#versioning-packages}{\texttt{packageinfo}}
files are generated or updated.

The properties of type \texttt{File} support any type that can be
resolved by
\href{https://docs.gradle.org/current/dsl/org.gradle.api.Project.html\#org.gradle.api.Project:file(java.css.Object)}{\texttt{project.file}}.
Moreover, it is possible to use Closures and Callables as values for the
\texttt{String} properties to defer evaluation until task execution.

\section{Helper Tasks}\label{helper-tasks}

If the \hyperref[lowestmajorversion]{\texttt{lowestMajorVersion}}
property is specified with a value \texttt{L}, the plugin creates a
series of helper tasks of type
\hyperref[baselinetask]{\texttt{BaselineTask}} at the end of the
\href{https://docs.gradle.org/current/userguide/build_lifecycle.html\#N11BAE}{project
evaluation}, one for each major version between \texttt{L} and the major
version \texttt{M} of the project:

\begin{enumerate}
\def\labelenumi{\arabic{enumi}.}
\tightlist
\item
  Task \texttt{baseline\$\{L\ +\ 1\}}, which depends on
  \texttt{baseline\$\{L\ +\ 2\}} and uses the version range
  \texttt{{[}(L\ +\ 1).0.0,\ (L\ +\ 2).0.0)} as baseline.
\item
  Task \texttt{baseline\$\{L\ +\ 2\}}, which depends on
  \texttt{baseline\$\{L\ +\ 3\}} and uses the version range
  \texttt{{[}(L\ +\ 2).0.0,\ (L\ +\ 3).0.0)} as baseline.
\item
  \ldots{}
\item
  Task \texttt{baseline\$\{M\ -\ 2\}}, which depends on
  \texttt{baseline\$\{M\ -\ 1\}} and uses the version range
  \texttt{{[}(M\ -\ 2).0.0,\ (M\ -\ 1).0.0)} as baseline.
\item
  Task \texttt{baseline\$\{M\ -\ 1\}}, which depends on
  \texttt{baseline\$\{M\}} and uses the version range
  \texttt{{[}(M\ -\ 1).0.0,\ M.0.0)} as baseline.
\item
  Task \texttt{baseline\$\{M\}}, which uses the version range
  \texttt{{[}M.0.0,\ M.x.y)} as baseline.
\end{enumerate}

The \texttt{baseline} task is also configured to use the version range
\texttt{{[}L.0.0,\ (L\ +\ 1).0.0)} as baseline, and to depend on the
task \texttt{baseline\$\{L\ +\ 1\}}. This means that running the
\texttt{baseline} task runs the baseline check against multiple
versions, starting from the most recent \texttt{M} and going back to
\texttt{L}.

Moreover, all tasks except \texttt{baseline\$\{M\}} have the property
\hyperref[ignoreexcessiveversionincreases]{\texttt{ignoreExcessiveVersionIncreases}}
set to \texttt{true}.

\section{Additional Configuration}\label{additional-configuration}

There are additional configurations that can help you baseline your OSGi
bundle.

\section{Baseline Dependency}\label{baseline-dependency}

The plugin creates a configuration called \texttt{baseline} with a
default dependency to a released non-snapshot version of the bundle:

\begin{itemize}
\tightlist
\item
  version range \texttt{{[}L.0.0,\ (L\ +\ 1).0.0)} if the
  \hyperref[lowestmajorversion]{\texttt{lowestMajorVersion}} property is
  specified with a value \texttt{L}.
\item
  version range \texttt{(,\$\{project.version\})} otherwise.
\end{itemize}

It is possible to override this setting and use a different version of
the bundle as baseline.

\section{System Properties}\label{system-properties}

It is possible to set the default values of the
\hyperref[ignorefailures]{\texttt{ignoreFailures}} property for a
\texttt{BaselineTask} task via system properties:

\begin{verbatim}
-D${task.name}.ignoreFailures=true
\end{verbatim}

For example, run the following Bash command to execute the baseline
check without breaking the build, in case of errors:

\begin{verbatim}
./gradlew baseline -Dbaseline.ignoreFailures=true
\end{verbatim}

\chapter{Change Log Builder Gradle
Plugin}\label{change-log-builder-gradle-plugin}

The Change Log Builder Gradle plugin lets you generate and maintain a
change log file based on the Git commits in your project. A change log
file generated by this plugin looks like this

\begin{verbatim}
#
# Bundle Version 1.0.1
#
9c77ff4c95cb1a325db3bdd089be105206e8b63c^..b421f00ac84b065685b131833fecc594fc01c760=LPS-123 LPS-1321

#
# Bundle Version 1.0.2
#
b421f00ac84b065685b131833fecc594fc01c760^..bc15d8d84e12b9544f78e4e3743c510dbaec2d89=LPS-456
\end{verbatim}

Every time the \hyperref[buildchangelog]{\texttt{buildChangeLog}} task
is executed, a new line is added to the change log, which lists all Git
\hyperref[ticketidprefixes]{commit prefixes} (usually issue ticket IDs)
that occurred in a certain range. The end of the range is always the tip
of the current branch. The start range can vary, depending on the case:

\begin{itemize}
\tightlist
\item
  If \texttt{buildChangeLog} has never been executed for the project,
  the change log does not exist. Therefore, the most recent commit from
  two years ago is used for the range start.
\item
  If a change log already exists for your project, the start range
  begins at the range end of the last line in the change log.
\end{itemize}

The plugin has been successfully tested with Gradle 4.10.2.

\section{Usage}\label{usage-2}

To use the plugin, include it in your build script:

\begin{verbatim}
buildscript {
    dependencies {
        classpath group: "com.liferay", name: "com.liferay.gradle.plugins.change.log.builder", version: "1.1.3"
    }

    repositories {
        maven {
            url "https://repository-cdn.liferay.com/nexus/content/groups/public"
        }
    }
}

apply plugin: "com.liferay.change.log.builder"
\end{verbatim}

\section{Tasks}\label{tasks-2}

The plugin adds one task to your project:

Name \textbar{} Depends On \textbar{} Type \textbar{} Description
\texttt{buildChangeLog} \textbar{} - \textbar{}
\hyperref[buildchangelogtask]{\texttt{BuildChangeLogTask}} \textbar{}
Builds the change log file for this project.

The \texttt{buildChangeLog} task is automatically configured with
sensible defaults, depending on whether the
\href{https://docs.gradle.org/current/userguide/java_plugin.html}{\texttt{java}}
plugin is applied:

Property Name \textbar{} Default Value
\hyperref[changelogheader]{\texttt{changeLogHeader}} \textbar{}
\texttt{"Bundle\ Version\ \$\{project.version\}"}
\hyperref[changelogfile]{\texttt{changeLogFile}} \textbar{}

\textbf{If the \texttt{java} plugin is applied:} The
\texttt{META-INF/liferay-releng.changelog} file in the first
\texttt{resources} directory of the \texttt{main} source set (by
default, \texttt{src/main/resources/META-INF/liferay-releng.changelog}).

\textbf{Otherwise:}
\texttt{"\$\{project.projectDir\}/liferay-releng.changelog"}

\hyperref[dirs]{\texttt{dirs}} \textbar{}
\texttt{{[}project.projectDir{]}}

\section{BuildChangeLogTask}\label{buildchangelogtask}

\subsection{Task Properties}\label{task-properties-1}

Property Name \textbar{} Type \textbar{} Default Value \textbar{}
Description \texttt{changeLogFile} \textbar{} \texttt{File} \textbar{}
\texttt{null} \textbar{} The change log file to build.
\texttt{changeLogHeader} \textbar{} \texttt{String} \textbar{}
\texttt{null} \textbar{} The header for the new line in the change log.
\texttt{dirs} \textbar{} \texttt{FileCollection} \textbar{}
\texttt{{[}{]}} \textbar{} The directories to consider when listing the
commits in the range specified. \texttt{gitDir} \textbar{} \texttt{File}
\textbar{} \texttt{project.rootDir} \textbar{} The base directory to
start searching for the \texttt{.git} directory. The search proceeds in
all the ancestors of the directory specified. \texttt{rangeEnd}
\textbar{} \texttt{String} \textbar{} \texttt{null} \textbar{} The hash
of the last commit to consider. If not set, it corresponds to the range
end of the last line in the change log, or the most recent commit from
at least two years ago if the change log file does not exist yet.
\texttt{rangeStart} \textbar{} \texttt{String} \textbar{} \texttt{null}
\textbar{} The hash of the first commit to consider. If not set, it
corresponds to the hash of the tip of the current branch.
\texttt{ticketIdPrefixes} \textbar{}
\texttt{Set\textless{}String\textgreater{}} \textbar{}
\texttt{{[}"CLDSVCS",\ "LPS",\ "SOS",\ "SYNC"{]}} \textbar{} The valid
prefix of the Git commit messages to add to the change log. For example,
if a commit message is \texttt{"LPS-123\ Bugfix"}, \texttt{"LPS-123"}
will be added to the change log.

The properties of type \texttt{File} support any type that can be
resolved by
\href{https://docs.gradle.org/current/dsl/org.gradle.api.Project.html\#org.gradle.api.Project:file(java.css.Object)}{\texttt{project.file}}.
Moreover, it is possible to use Closures and Callables as values for the
\texttt{String} properties to defer evaluation until task execution.

\subsection{Task Methods}\label{task-methods}

Method \textbar{} Description
\texttt{BuildChangeLogTask\ dirs(Iterable\textless{}?\textgreater{}\ dirs)}
\textbar{} Adds directories to consider when listing the commits in the
range specified. \texttt{BuildChangeLogTask\ dirs(Object...\ dirs)}
\textbar{} Adds directories to consider when listing the commits in the
range specified.
\texttt{BuildChangeLogTask\ ticketIdPrefixes(Iterable\textless{}String\textgreater{}\ ticketIdPrefixes)}
\textbar{} Adds valid prefixes of the Git commit messages to add to the
change log.
\texttt{BuildChangeLogTask\ ticketIdPrefixes(String...\ ticketIdPrefixes)}
\textbar{} Adds valid prefixes of the Git commit messages to add to the
change log.

\chapter{CSS Builder Gradle Plugin}\label{css-builder-gradle-plugin}

The CSS Builder Gradle plugin lets you run the
\href{https://github.com/liferay/liferay-portal/tree/master/modules/util/css-builder}{Liferay
CSS Builder} tool to compile \href{http://sass-lang.com/}{Sass} files in
your project.

The plugin has been successfully tested with Gradle 4.10.2.

\section{Usage}\label{usage-3}

To use the plugin, include it in your build script:

\begin{verbatim}
buildscript {
    dependencies {
        classpath group: "com.liferay", name: "com.liferay.gradle.plugins.css.builder", version: "3.0.0"
    }

    repositories {
        maven {
            url "https://repository-cdn.liferay.com/nexus/content/groups/public"
        }
    }
}

apply plugin: "com.liferay.css.builder"
\end{verbatim}

Since the plugin automatically resolves the Liferay CSS Builder library
as a dependency, you have to configure a repository that hosts the
library and its transitive dependencies. The Liferay CDN repository
hosts them all:

\begin{verbatim}
repositories {
    maven {
        url "https://repository-cdn.liferay.com/nexus/content/groups/public"
    }
}
\end{verbatim}

\section{Tasks}\label{tasks-3}

The plugin adds one task to your project:

Name \textbar{} Depends On \textbar{} Type \textbar{} Description
\texttt{buildCSS} \textbar{} - \textbar{}
\hyperref[buildcsstask]{\texttt{BuildCSSTask}} \textbar{} Compiles the
Sass files in this project.

The plugin also adds the following dependencies to tasks defined by the
\href{https://docs.gradle.org/current/userguide/java_plugin.html}{\texttt{java}}
plugin:

Name \textbar{} Depends On \texttt{processResources} \textbar{}
\texttt{buildCSS}

The \texttt{buildCSS} task is automatically configured with sensible
defaults, depending on whether the
\href{https://docs.gradle.org/current/userguide/java_plugin.html}{\texttt{java}}
or the
\href{https://docs.gradle.org/current/userguide/war_plugin.html}{\texttt{war}}
plugins are applied:

Property Name \textbar{} Default Value
\hyperref[basedir]{\texttt{baseDir}} \textbar{}

\textbf{If the \texttt{java} plugin is applied:} The first
\texttt{resources} directory of the \texttt{main} source set (by
default: \texttt{src/main/resources}).

\textbf{If the \texttt{war} plugin is applied:}
\texttt{project.webAppDir}.

\textbf{Otherwise:} \texttt{null}

\section{BuildCSSTask}\label{buildcsstask}

Tasks of type \texttt{BuildCSSTask} extend
\href{https://docs.gradle.org/current/dsl/org.gradle.api.tasks.JavaExec.html}{\texttt{JavaExec}},
so all its properties and methods, such as
\href{https://docs.gradle.org/current/dsl/org.gradle.api.tasks.JavaExec.html\#org.gradle.api.tasks.JavaExec:args(java.css.Iterable)}{\texttt{args}}
and
\href{https://docs.gradle.org/current/dsl/org.gradle.api.tasks.JavaExec.html\#org.gradle.api.tasks.JavaExec:maxHeapSize}{\texttt{maxHeapSize}},
are available. They also have the following properties set by default:

Property Name \textbar{} Default Value
\href{https://docs.gradle.org/current/dsl/org.gradle.api.tasks.JavaExec.html\#org.gradle.api.tasks.JavaExec:args}{\texttt{args}}
\textbar{} CSS Builder command line arguments
\href{https://docs.gradle.org/current/dsl/org.gradle.api.tasks.JavaExec.html\#org.gradle.api.tasks.JavaExec:classpath}{\texttt{classpath}}
\textbar{}
\hyperref[liferay-css-builder-dependency]{\texttt{project.configurations.cssBuilder}}
\href{https://docs.gradle.org/current/javadoc/org/gradle/api/tasks/JavaExec.html\#setDefaultCharacterEncoding(java.lang.String)}{\texttt{defaultCharacterEncoding}}
\textbar{} \texttt{"UTF-8"}
\href{https://docs.gradle.org/current/dsl/org.gradle.api.tasks.JavaExec.html\#org.gradle.api.tasks.JavaExec:main}{\texttt{main}}
\textbar{} \texttt{"com.liferay.css.builder.CSSBuilder"}
\href{https://docs.gradle.org/current/dsl/org.gradle.api.tasks.JavaExec.html\#org.gradle.api.tasks.JavaExec:systemProperties}{\texttt{systemProperties}}
\textbar{} \texttt{{[}"sass.compiler.jni.clean.temp.dir",\ true{]}}

\subsection{Task Properties}\label{task-properties-2}

Property Name \textbar{} Type \textbar{} Default Value \textbar{}
Description \texttt{appendCssImportTimestamps} \textbar{}
\texttt{boolean} \textbar{} \texttt{true} \textbar{} Whether to append
the current timestamp to the URLs in the \texttt{@import} CSS at-rules.
It sets the \texttt{sass.append.css.import.timestamps} argument.
\texttt{baseDir} \textbar{} \texttt{File} \textbar{} \texttt{null}
\textbar{} The base directory that contains the SCSS files to compile.
It sets the \texttt{sass.docroot.dir} argument. \texttt{cssFiles}
\textbar{} \texttt{FileCollection} \textbar{} - \textbar{} The SCSS
files to compile. \emph{(Read-only)} \texttt{dirNames} \textbar{}
\texttt{List\textless{}String\textgreater{}} \textbar{}
\texttt{{[}"/"{]}} \textbar{} The name of the directories, relative to
\hyperref[basedir]{\texttt{baseDir}}, which contain the SCSS files to
compile. All sub-directories are searched for SCSS files as well. It
sets the \texttt{sass.dir} argument. \texttt{generateSourceMap}
\textbar{} \texttt{boolean} \textbar{} \texttt{false} \textbar{} Whether
to generate
\href{https://developers.google.com/web/tools/chrome-devtools/debug/readability/source-maps}{source
maps} for easier debugging. It sets the
\texttt{sass.generate.source.map} argument. \texttt{importDir}
\textbar{} \texttt{File} \textbar{} \texttt{null} \textbar{} The
\texttt{META-INF/resources} directory of the
\href{https://github.com/liferay/liferay-portal/tree/master/modules/apps/frontend-css/frontend-css-common}{Liferay
Frontend Common CSS} artifact. This is required in order to make
\href{http://bourbon.io}{Bourbon} and other CSS libraries available to
the compilation. \texttt{importFile} \textbar{} \texttt{File} \textbar{}
\hyperref[liferay-frontend-common-css-dependency]{\texttt{configurations.portalCommonCSS.singleFile}}
\textbar{} The Liferay Frontend Common CSS JAR file. If
\hyperref[importdir]{\texttt{importDir}} is set, this property has no
effect. \texttt{importPath} \textbar{} \texttt{File} \textbar{} -
\textbar{} The value of the \texttt{importDir} property if set;
otherwise \texttt{importFile}. It sets the
\texttt{sass.portal.common.path} argument. \emph{(Read-only)}
\texttt{outputDirName} \textbar{} \texttt{String} \textbar{}
\texttt{".sass-cache/"} \textbar{} The name of the sub-directories where
the SCSS files are compiled to. For each directory that contains SCSS
files, a sub-directory with this name is created. It sets the
\texttt{sass.output.dir} argument. \texttt{outputDirs} \textbar{}
\texttt{FileCollection} \textbar{} - \textbar{} The directories where
the SCSS files are compiled to. Usually, these directories are ignored
by the Version Control System. \emph{(Read-only)} \texttt{precision}
\textbar{} \texttt{int} \textbar{} \texttt{5} \textbar{} The numeric
precision of numbers in Sass. It sets the \texttt{sass.precision}
argument. \texttt{rtlExcludedPathRegexps} \textbar{}
\texttt{List\textless{}String\textgreater{}} \textbar{} \texttt{{[}{]}}
\textbar{} The SCSS file patterns to exclude when converting for
right-to-left (RTL) support. It sets the
\texttt{sass.rtl.excluded.path.regexps} argument.
\texttt{sassCompilerClassName} \textbar{} \texttt{String} \textbar{}
\texttt{null} \textbar{} The type of Sass compiler to use. Supported
values are \texttt{"jni"} and \texttt{"ruby"}. If not set, defaults to
\texttt{"jni"}. It sets the \texttt{sass.compiler.class.name} argument.

\noindent\hrulefill

\textbf{Note:} Liferay's CSS Builder is supported for Oracle's JDK and
uses a native compiler for increased speed. If you're using an IBM JDK,
you may experience issues when building your Sass files (e.g., when
building a theme). It's recommended to switch to using the Oracle JDK,
but if you prefer using the IBM JDK, you must use the fallback Ruby
compiler. You can do this two ways:

\begin{itemize}
\tightlist
\item
  If you're working in a
  \href{/docs/7-2/reference/-/knowledge_base/r/liferay-workspace}{Liferay
  Workspace} or using the
  \href{https://github.com/liferay/liferay-portal/tree/master/modules/sdk/gradle-plugins}{Liferay
  Gradle Plugins} plugin, set \texttt{sass.compiler.class.name=ruby} in
  your \texttt{gradle.properties} file.
\item
  Otherwise, set
  \texttt{buildCSS.sassCompilerClassName=\textquotesingle{}ruby\textquotesingle{}}
  in the project's \texttt{build.gradle} file.
\end{itemize}

The \texttt{sass.compiler.class.name=ruby} Gradle property only works
for modules, so if you're using the Ruby compiler in a WAR project
(e.g., theme), you must use the second option.

Be aware that the Ruby-based compiler doesn't perform as well as the
native compiler, so expect longer compile times.

\noindent\hrulefill

The properties of type \texttt{File} support any type that can be
resolved by
\href{https://docs.gradle.org/current/dsl/org.gradle.api.Project.html\#org.gradle.api.Project:file(java.css.Object)}{\texttt{project.file}}.
Moreover, it is possible to use Closures and Callables as values for the
\texttt{int} and \texttt{String} properties, to defer evaluation until
task execution.

\subsection{Task Methods}\label{task-methods-1}

Method \textbar{} Description
\texttt{BuildCSSTask\ dirNames(Iterable\textless{}Object\textgreater{}\ dirNames)}
\textbar{} Adds sub-directory names, relative to
\hyperref[basedir]{\texttt{baseDir}}, which contain the SCSS files to
compile. \texttt{BuildCSSTask\ dirNames(Object...\ dirNames)} \textbar{}
Adds sub-directory names, relative to
\hyperref[basedir]{\texttt{baseDir}}, which contain the SCSS files to
compile.
\texttt{BuildCSSTask\ rtlExcludedPathRegexps(Iterable\textless{}Object\textgreater{}\ rtlExcludedPathRegexps)}
\textbar{} Adds SCSS file patterns to exclude when converting for
right-to-left (RTL) support.
\texttt{BuildCSSTask\ rtlExcludedPathRegexps(Object...\ rtlExcludedPathRegexps)}
\textbar{} Adds SCSS file patterns to exclude when converting for
right-to-left (RTL) support.

\section{Additional Configuration}\label{additional-configuration-1}

There are additional configurations that can help you use the CSS
Builder.

\section{Liferay CSS Builder
Dependency}\label{liferay-css-builder-dependency}

By default, the plugin creates a configuration called
\texttt{cssBuilder} and adds a dependency to the latest released version
of the Liferay CSS Builder. It is possible to override this setting and
use a specific version of the tool by manually adding a dependency to
the \texttt{cssBuilder} configuration:

\begin{verbatim}
dependencies {
    cssBuilder group: "com.liferay", name: "com.liferay.css.builder", version: "3.0.0"
}
\end{verbatim}

\section{Liferay Frontend Common CSS
Dependency}\label{liferay-frontend-common-css-dependency}

By default, the plugin creates a configuration called
\texttt{portalCommonCSS} and adds a dependency to the latest released
version of the Liferay Frontend Common CSS artifact. It is possible to
override this setting and use a specific version of the artifact by
manually adding a dependency to the \texttt{portalCommonCSS}
configuration:

\begin{verbatim}
dependencies {
    portalCommonCSS group: "com.liferay", name: "com.liferay.frontend.css.common", version: "2.0.1"
}
\end{verbatim}

\chapter{DB Support Gradle Plugin}\label{db-support-gradle-plugin}

The DB Support Gradle plugin lets you run the
\href{https://github.com/liferay/liferay-portal/tree/master/modules/util/portal-tools-db-support}{Liferay
DB Support} tool to execute certain actions on a local Liferay database.
So far, the following actions are available:

\begin{itemize}
\tightlist
\item
  Cleans the Liferay database from the Service Builder tables and rows
  of a module.
\end{itemize}

The plugin has been successfully tested with Gradle 4.10.2.

\section{Usage}\label{usage-4}

To use the plugin, include it in your build script:

\begin{verbatim}
buildscript {
    dependencies {
        classpath group: "com.liferay", name: "com.liferay.gradle.plugins.db.support", version: "1.0.5"
    }

    repositories {
        maven {
            url "https://repository-cdn.liferay.com/nexus/content/groups/public"
        }
    }
}

apply plugin: "com.liferay.portal.tools.db.support"
\end{verbatim}

Since the plugin automatically resolves the Liferay DB Support library
as a dependency, you have to configure a repository that hosts the
library and its transitive dependencies. The Liferay CDN repository
hosts them all:

\begin{verbatim}
repositories {
    maven {
        url "https://repository-cdn.liferay.com/nexus/content/groups/public"
    }
}
\end{verbatim}

\section{Tasks}\label{tasks-4}

The plugin adds one task to your project:

Name \textbar{} Depends On \textbar{} Type \textbar{} Description
\texttt{cleanServiceBuilder} \textbar{} - \textbar{}
\hyperref[cleanservicebuildertask]{\texttt{CleanServiceBuilderTask}}
\textbar{} Cleans the Liferay database from the Service Builder tables
and rows of a module.

The \texttt{cleanServiceBuilder} task is automatically configured with
sensible defaults, depending on whether the
\href{https://docs.gradle.org/current/userguide/standard_plugins.html\#N135C1}{\texttt{base}}
plugin is applied:

Property Name \textbar{} Default Value
\hyperref[servletcontextname]{\texttt{servletContextName}} \textbar{}

\textbf{If the \texttt{base} plugin is applied:} The bundle symbolic
name of the project inferred via the
\href{https://github.com/gradle/gradle/blob/master/subprojects/osgi/src/main/java/org/gradle/api/internal/plugins/osgi/OsgiHelper.java}{\texttt{OsgiHelper}}
class.

\textbf{Otherwise:} \texttt{null}

\hyperref[servicexmlfile]{\texttt{serviceXmlFile}} \textbar{}
\texttt{"\$\{project.projectDir\}/service.xml"}

\section{CleanServiceBuilderTask}\label{cleanservicebuildertask}

Tasks of type \texttt{BuildDeploymentHelperTask} extend
\href{https://docs.gradle.org/current/dsl/org.gradle.api.tasks.JavaExec.html}{\texttt{JavaExec}},
so all its properties and methods, such as
\href{https://docs.gradle.org/current/dsl/org.gradle.api.tasks.JavaExec.html\#org.gradle.api.tasks.JavaExec:args(java.lang.Iterable)}{\texttt{args}}
and
\href{https://docs.gradle.org/current/dsl/org.gradle.api.tasks.JavaExec.html\#org.gradle.api.tasks.JavaExec:maxHeapSize}{\texttt{maxHeapSize}},
are available. They also have the following properties set by default:

Property Name \textbar{} Default Value
\href{https://docs.gradle.org/current/dsl/org.gradle.api.tasks.JavaExec.html\#org.gradle.api.tasks.JavaExec:args}{\texttt{args}}
\textbar{} The DB Support command line arguments.
\href{https://docs.gradle.org/current/dsl/org.gradle.api.tasks.JavaExec.html\#org.gradle.api.tasks.JavaExec:classpath}{\texttt{classpath}}
\textbar{}
\hyperref[jdbc-drivers-dependency]{\texttt{project.configurations.dbSupport}}
+
\hyperref[liferay-db-support-dependency]{\texttt{project.configurations.dbSupportTool}}
\href{https://docs.gradle.org/current/dsl/org.gradle.api.tasks.JavaExec.html\#org.gradle.api.tasks.JavaExec:main}{\texttt{main}}
\textbar{} \texttt{"com.liferay.portal.tools.db.support.DBSupport"}

\subsection{Task Properties}\label{task-properties-3}

Property Name \textbar{} Type \textbar{} Default Value \textbar{}
Description \texttt{password} \textbar{} \texttt{String} \textbar{}
\texttt{null} \textbar{} The user password for connecting to the Liferay
database. It sets the \texttt{-\/-password} argument. If
\hyperref[propertiesfile]{\texttt{propertiesFile}} is set, this property
has no effect. \texttt{propertiesFile} \textbar{} \texttt{File}
\textbar{} \texttt{null} \textbar{} The \texttt{portal-ext.properties}
file that contains the JDBC settings for connecting to the Liferay
database. It sets the \texttt{-\/-properties-file} argument.
\texttt{servletContextName} \textbar{} \texttt{String} \textbar{}
\texttt{null} \textbar{} The servlet context name (usually the value of
the \texttt{Bundle-Symbolic-Name} manifest header) of the module. It
sets the \texttt{-\/-servlet-context-name} argument.
\texttt{serviceXmlFile} \textbar{} \texttt{File} \textbar{}
\texttt{null} \textbar{} The \texttt{service.xml} file of the module. It
sets the \texttt{-\/-service-xml-file} argument. \texttt{url} \textbar{}
\texttt{String} \textbar{} \texttt{null} \textbar{} The JDBC URL for
connecting to the Liferay database. It sets the \texttt{-\/-url}
argument. If \hyperref[propertiesfile]{\texttt{propertiesFile}} is set,
this property has no effect. \texttt{userName} \textbar{}
\texttt{String} \textbar{} \texttt{null} \textbar{} The user name for
connecting to the Liferay database. It sets the \texttt{-\/-user-name}
argument. If \hyperref[propertiesfile]{\texttt{propertiesFile}} is set,
this property has no effect.

The properties of type \texttt{File} support any type that can be
resolved by
\href{https://docs.gradle.org/current/dsl/org.gradle.api.Project.html\#org.gradle.api.Project:file(java.css.Object)}{\texttt{project.file}}.
Moreover, it is possible to use Closures and Callables as values for the
\texttt{int} and \texttt{String} properties to defer evaluation until
task execution.

\section{Additional Configuration}\label{additional-configuration-2}

There are additional configurations that can help you use the Deployment
Helper.

\section{JDBC Drivers Dependency}\label{jdbc-drivers-dependency}

The plugin creates a configuration called \texttt{dbSupport}, which can
be used to provide the suitable JDBC driver for your Liferay database:

\begin{verbatim}
dependencies {
    dbSupport group: "mysql", name: "mysql-connector-java", version: "5.1.23"
    dbSupport group: "org.mariadb.jdbc", name: "mariadb-java-client", version: "1.1.9"
    dbSupport group: "org.postgresql", name: "postgresql", version: "9.4-1201-jdbc41"
}
\end{verbatim}

\section{Liferay DB Support
Dependency}\label{liferay-db-support-dependency}

By default, the plugin creates a configuration called
\texttt{dbSupportTool} and adds a dependency to the latest released
version of the Liferay DB Support. It is possible to override this
setting and use a specific version of the tool by manually adding a
dependency to the \texttt{dbSupportTool} configuration:

\begin{verbatim}
dependencies {
    dbSupportTool group: "com.liferay", name: "com.liferay.portal.tools.db.support", version: "1.0.8"
}
\end{verbatim}

\chapter{Dependency Checker Gradle
Plugin}\label{dependency-checker-gradle-plugin}

The Dependency Checker Gradle plugin lets you warn users if a specific
configuration dependency is not the latest one available from the Maven
central repository. The plugin eventually fails the build if the
dependency age (the difference between the timestamp of the current
version and the latest version) is above a predetermined threshold.

The plugin has been successfully tested with Gradle 4.10.2.

\section{Usage}\label{usage-5}

To use the plugin, include it in your build script:

\begin{verbatim}
buildscript {
    dependencies {
        classpath group: "com.liferay", name: "com.liferay.gradle.plugins.dependency.checker", version: "1.0.3"
    }

    repositories {
        maven {
            url "https://repository-cdn.liferay.com/nexus/content/groups/public"
        }
    }
}

apply plugin: "com.liferay.dependency.checker"
\end{verbatim}

\section{Project Extension}\label{project-extension-2}

The Dependency Checker Gradle plugin exposes the following properties
through the extension named \texttt{dependencyChecker}:

Property Name \textbar{} Type \textbar{} Default Value \textbar{}
Description \texttt{ignoreFailures} \textbar{} \texttt{boolean}
\textbar{} \texttt{true} \textbar{} Whether to print an error message
instead of failing the build when the dependency check fails, either for
a network error or because the dependency is out-of-date.

The same extension exposes the following methods:

Method \textbar{} Description
\texttt{void\ maxAge(Map\textless{}?,\ ?\textgreater{}\ args)}
\textbar{} Declares the max age allowed for a dependency. The
\texttt{args} map must contain the following entries:

\texttt{configuration}: the configuration name

\texttt{group}: the dependency group

\texttt{name}: the dependency name

\texttt{maxAge}: an instance of
\href{http://docs.groovy-lang.org/latest/html/api/groovy/time/Duration.html}{\texttt{groovy.time.Duration}}
that represents the maximum age allowed for the dependency

\texttt{throwError}: a \texttt{boolean} value representing whether to
throw an error if the dependency is out-of-date

\section{Additional Configuration}\label{additional-configuration-3}

There are additional configurations that can help you use the Deployment
Helper.

\section{Project Properties}\label{project-properties}

It is possible to set the default values of the
\hyperref[ignorefailures]{\texttt{ignoreFailures}} property via the
project property \texttt{dependencyCheckerIgnoreFailures}:

\begin{verbatim}
-PdependencyCheckerIgnoreFailures=false
\end{verbatim}

\chapter{Deployment Helper Gradle
Plugin}\label{deployment-helper-gradle-plugin}

The Deployment Helper Gradle plugin lets you run the
\href{https://github.com/liferay/liferay-portal/tree/master/modules/util/deployment-helper}{Liferay
Deployment Helper} tool to create a cluster deployable WAR from your
OSGi artifacts.

The plugin has been successfully tested with Gradle 4.10.2.

\section{Usage}\label{usage-6}

To use the plugin, include it in your build script:

\begin{verbatim}
buildscript {
    dependencies {
        classpath group: "com.liferay", name: "com.liferay.gradle.plugins.deployment.helper", version: "1.0.5"
    }

    repositories {
        maven {
            url "https://repository-cdn.liferay.com/nexus/content/groups/public"
        }
    }
}

apply plugin: "com.liferay.deployment.helper"
\end{verbatim}

Since the plugin automatically resolves the Liferay Deployment Helper
library as a dependency, you have to configure a repository that hosts
the library and its transitive dependencies. The Liferay CDN repository
hosts them all:

\begin{verbatim}
repositories {
    maven {
        url "https://repository-cdn.liferay.com/nexus/content/groups/public"
    }
}
\end{verbatim}

\section{Tasks}\label{tasks-5}

The plugin adds one task to your project:

Name \textbar{} Depends On \textbar{} Type \textbar{} Description
\texttt{buildDeploymentHelper} \textbar{} - \textbar{}
\hyperref[builddeploymenthelpertask]{\texttt{BuildDeploymentHelperTask}}
\textbar{} Builds a WAR which contains one or more files that are copied
once the WAR is deployed.

\section{BuildDeploymentHelperTask}\label{builddeploymenthelpertask}

Tasks of type \texttt{BuildDeploymentHelperTask} extend
\href{https://docs.gradle.org/current/dsl/org.gradle.api.tasks.JavaExec.html}{\texttt{JavaExec}},
so all its properties and methods, such as
\href{https://docs.gradle.org/current/dsl/org.gradle.api.tasks.JavaExec.html\#org.gradle.api.tasks.JavaExec:args(java.lang.Iterable)}{\texttt{args}}
and
\href{https://docs.gradle.org/current/dsl/org.gradle.api.tasks.JavaExec.html\#org.gradle.api.tasks.JavaExec:maxHeapSize}{\texttt{maxHeapSize}},
are available. They also have the following properties set by default:

Property Name \textbar{} Default Value
\href{https://docs.gradle.org/current/dsl/org.gradle.api.tasks.JavaExec.html\#org.gradle.api.tasks.JavaExec:args}{\texttt{args}}
\textbar{} The Deployment Helper command line arguments.
\href{https://docs.gradle.org/current/dsl/org.gradle.api.tasks.JavaExec.html\#org.gradle.api.tasks.JavaExec:classpath}{\texttt{classpath}}
\textbar{}
\hyperref[liferay-deployment-helper-dependency]{\texttt{project.configurations.deploymentHelper}}
\hyperref[deploymentfiles]{\texttt{deploymentFiles}} \textbar{} The
output files of the
\href{https://docs.gradle.org/current/userguide/java_plugin.html\#sec:jar}{\texttt{jar}}
tasks of this project and all its sub-projects.
\href{https://docs.gradle.org/current/dsl/org.gradle.api.tasks.JavaExec.html\#org.gradle.api.tasks.JavaExec:main}{\texttt{main}}
\textbar{} \texttt{"com.liferay.deployment.helper.DeploymentHelper"}
\hyperref[outputfile]{\texttt{outputFile}} \textbar{}
\texttt{"\$\{project.buildDir\}/\$\{project.name\}.war"}

\subsection{Task Properties}\label{task-properties-4}

Property Name \textbar{} Type \textbar{} Default Value \textbar{}
Description \texttt{deploymentFiles} \textbar{} \texttt{FileCollection}
\textbar{} \texttt{{[}{]}} \textbar{} The files or directories to
include in the WAR and copy once the WAR is deployed. If a directory is
added to this collection, all the JAR files contained in the directory
are included in the WAR. \texttt{deploymentPath} \textbar{}
\texttt{File} \textbar{} \texttt{null} \textbar{} The directory to which
the included files are copied. \texttt{outputFile} \textbar{}
\texttt{File} \textbar{} \texttt{null} \textbar{} The WAR file to build.

The properties of type \texttt{File} support any type that can be
resolved by
\href{https://docs.gradle.org/current/dsl/org.gradle.api.Project.html\#org.gradle.api.Project:file(java.css.Object)}{\texttt{project.file}}.

\subsection{Task Methods}\label{task-methods-2}

Method \textbar{} Description
\texttt{BuildDeploymentHelperTask\ deploymentFiles(Iterable\textless{}?\textgreater{}\ deploymentFiles)}
\textbar{} Adds files or directories to include in the WAR and copy once
the WAR is deployed. The values are evaluated as per
\href{https://docs.gradle.org/current/dsl/org.gradle.api.Project.html\#org.gradle.api.Project:files(java.lang.Object\%5B\%5D)}{\texttt{project.files}}.
\texttt{BuildDeploymentHelperTask\ deploymentFiles(Object...\ deploymentFiles)}
\textbar{} Adds files or directories to include in the WAR and copy once
the WAR is deployed. The values are evaluated as per
\href{https://docs.gradle.org/current/dsl/org.gradle.api.Project.html\#org.gradle.api.Project:files(java.lang.Object\%5B\%5D)}{\texttt{project.files}}.

\section{Additional Configuration}\label{additional-configuration-4}

There are additional configurations that can help you use the Deployment
Helper.

\section{Liferay Deployment Helper
Dependency}\label{liferay-deployment-helper-dependency}

By default, the plugin creates a configuration called
\texttt{deploymentHelper} and adds a dependency to the latest released
version of the Liferay Deployment Helper. It is possible to override
this setting and use a specific version of the tool by manually adding a
dependency to the \texttt{deploymentHelper} configuration:

\begin{verbatim}
dependencies {
    deploymentHelper group: "com.liferay", name: "com.liferay.deployment.helper", version: "1.0.4"
}
\end{verbatim}

\chapter{Go Gradle Plugin}\label{go-gradle-plugin}

The Go Gradle plugin lets you run \href{https://golang.org/}{Go} as part
of your build.

The plugin has been successfully tested with Gradle 3.5.1 up to 4.10.2.

\section{Usage}\label{usage-7}

To use the plugin, include it in your build script:

\begin{verbatim}
buildscript {
    dependencies {
        classpath group: "com.liferay", name: "com.liferay.gradle.plugins.go", version: "1.0.0"
    }

    repositories {
        maven {
            url "https://repository-cdn.liferay.com/nexus/content/groups/public"
        }
    }
}

apply plugin: "com.liferay.go"
\end{verbatim}

\section{Project Extension}\label{project-extension-3}

The Go Gradle plugin exposes the following properties through the
extension named \texttt{go}:

Property Name \textbar{} Type \textbar{} Default Value \textbar{}
Description \texttt{goDir} \textbar{} \texttt{File} \textbar{}
\texttt{"\$\{project.buildDir\}/go"} \textbar{} The directory where the
Go distribution is unpacked. \texttt{goUrl} \textbar{} \texttt{String}
\textbar{}
\texttt{"https://dl.google.com/go/go\$\{go.goVersion\}.\$\{platform\}-\$\{bitMode\}.\$\{extension\}}
\textbar{} The URL of the Go distribution to download.
\texttt{goVersion} \textbar{} \texttt{String} \textbar{}
\texttt{"1.11.4"} \textbar{} The Go distribution's version to use.
\texttt{workingDir} \textbar{} \texttt{File} \textbar{}
\texttt{"\$\{project.projectDir\}"} \textbar{} The directory that
contains the project's Go source code.

\section{Tasks}\label{tasks-6}

The plugin adds a series of tasks to your project:

Name \textbar{} Depends On \textbar{} Type \textbar{} Description
\texttt{downloadGo} \textbar{} - \textbar{}
\hyperref[downloadgotask]{\texttt{DownloadGoTask}} \textbar{} Downloads
and unpacks the local Go distribution for the project.
\hyperref[gocommandprogramname-task]{\texttt{goBuild\$\{programName\}}}
\textbar{} \texttt{downloadGo} \textbar{}
\hyperref[executegotask]{\texttt{ExecuteGoTask}} \textbar{} Compiles
packages and dependencies for the Go program.
\hyperref[gocommandprogramname-task]{\texttt{goClean\$\{programName\}}}
\textbar{} \texttt{downloadGo} \textbar{}
\hyperref[executegotask]{\texttt{ExecuteGoTask}} \textbar{} Removes
object files for the Go program.
\hyperref[gocommandprogramname-task]{\texttt{goRun\$\{programName\}}}
\textbar{} \texttt{downloadGo} \textbar{}
\hyperref[executegotask]{\texttt{ExecuteGoTask}} \textbar{} Compiles and
runs the Go program.
\hyperref[gocommandprogramname-task]{\texttt{goTest\$\{programName\}}}
\textbar{} \texttt{downloadGo} \textbar{}
\hyperref[executegotask]{\texttt{ExecuteGoTask}} \textbar{} Tests
packages for the Go program.

\section{DownloadGoTask}\label{downloadgotask}

The purpose of this task is to download and unpack a Go distribution.

\subsection{Task Properties}\label{task-properties-5}

Property Name \textbar{} Type \textbar{} Default Value \textbar{}
Description \texttt{goDir} \textbar{} \texttt{File} \textbar{}
\texttt{null} \textbar{} The directory where the Go distribution is
unpacked. \texttt{goUrl} \textbar{} \texttt{String} \textbar{}
\texttt{null} \textbar{} The URL of the Go distribution to download.

The \texttt{File} type support any type that can be resolved by
\href{https://docs.gradle.org/current/dsl/org.gradle.api.Project.html\#org.gradle.api.Project:file(java.css.Object)}{\texttt{project.file}}.
Moreover, it is possible to use Closures and Callables as values for the
\texttt{String} properties, to defer evaluation until task execution.

\section{ExecuteGoTask}\label{executegotask}

This is the base task to run Go in a Gradle build. All tasks of type
\texttt{ExecuteGoTask} automatically depend on
\hyperref[downloadgo]{\texttt{downloadGo}}.

\subsection{Task Properties}\label{task-properties-6}

Property Name \textbar{} Type \textbar{} Default Value \textbar{}
Description \texttt{args} \textbar{}
\texttt{List\textless{}Object\textgreater{}} \textbar{} \texttt{{[}{]}}
\textbar{} The arguments for the Go invocation. \texttt{command}
\textbar{} \texttt{String} \textbar{} \texttt{"go"} \textbar{} The file
name of the executable to invoke. \texttt{environment} \textbar{}
\texttt{Map\textless{}Object,\ Object\textgreater{}} \textbar{}
\texttt{{[}{]}} \textbar{} The environment variables for the Go
invocation. \texttt{inheritProxy} \textbar{} \texttt{boolean} \textbar{}
\texttt{true} \textbar{} Whether to set the \texttt{http\_proxy},
\texttt{https\_proxy}, and \texttt{no\_proxy} environment variables in
the Go invocation based on the values of the system properties
\texttt{https.proxyHost}, \texttt{https.proxyPort},
\texttt{https.proxyUser}, \texttt{https.proxyPassword},
\texttt{https.nonProxyHosts}, \texttt{https.proxyHost},
\texttt{https.proxyPort}, \texttt{https.proxyUser},
\texttt{https.proxyPassword}, and \texttt{https.nonProxyHosts}. If these
environment variables are already set, their values will not be
overwritten. \texttt{goDir} \textbar{} \texttt{File} \textbar{}
\texttt{go.goDir}{]}(\#godir) \textbar{} The directory that contains the
executable to invoke. \texttt{useGradleExec} \textbar{} \texttt{boolean}
\textbar{}

\textbf{If running in a
\href{https://docs.gradle.org/current/userguide/gradle_daemon.html}{Gradle
Daemon}:} \texttt{true}

\textbf{Otherwise:} \texttt{false}

\textbar{} Whether to invoke Go using
\href{https://docs.gradle.org/current/dsl/org.gradle.api.Project.html\#org.gradle.api.Project:exec(org.gradle.api.Action)}{\texttt{project.exec}},
which can solve hanging problems with the Gradle Daemon.
\texttt{workingDir} \textbar{} \texttt{File} \textbar{}
\texttt{go.workingDir}{]}(\#workingdir) \textbar{} The working directory
to use in the Go invocation.

The type \texttt{File} properties support any type that can be resolved
by
\href{https://docs.gradle.org/current/dsl/org.gradle.api.Project.html\#org.gradle.api.Project:file(java.css.Object)}{\texttt{project.file}}.
Moreover, it is possible to use Closures and Callables as values for the
\texttt{String} properties to defer evaluation until task execution.

\subsection{Task Methods}\label{task-methods-3}

Method \textbar{} Description
\texttt{ExecuteGoTask\ args(Iterable\textless{}?\textgreater{}\ args)}
\textbar{} Adds arguments for the Go invocation.
\texttt{ExecuteGoTask\ args(Object...\ args)} \textbar{} Adds arguments
for the Go invocation.
\texttt{ExecuteGoTask\ environment(Map\textless{}?,\ ?\textgreater{}\ environment)}
\textbar{} Adds environment variables for the Go invocation.
\texttt{ExecuteGoTask\ environment(Object\ key,\ Object\ value)}
\textbar{} Adds an environment variable for the Go invocation.

\section{\texorpdfstring{go\({command}\)\{programName\}
Task}{go\{command\}\{programName\} Task}}\label{gocommandprogramname-task}

For each Go program in
\hyperref[workingdirproperty]{\texttt{workingDir}}, four tasks of type
\hyperref[executegotask]{\texttt{ExecuteGoTask}} are added. Each of
these tasks are automatically configured with sensible defaults:

Property Name \textbar{} Default Value \texttt{args} \textbar{}
\texttt{{[}"\$\{command\}",\ "\$\{programFile.absolutePath\}"{]}}

\chapter{Gulp Gradle Plugin}\label{gulp-gradle-plugin}

The Gulp Gradle plugin lets you run \href{http://gulpjs.com/}{Gulp}
tasks as part of your build.

The plugin has been successfully tested with Gradle 4.10.2.

\section{Usage}\label{usage-8}

To use the plugin, include it in your build script:

\begin{verbatim}
buildscript {
    dependencies {
        classpath group: "com.liferay", name: "com.liferay.gradle.plugins.gulp", version: "2.0.59"
    }

    repositories {
        maven {
            url "https://repository-cdn.liferay.com/nexus/content/groups/public"
        }
    }
}

apply plugin: "com.liferay.gulp"
\end{verbatim}

The Gulp plugin automatically applies the
\href{https://github.com/liferay/liferay-portal/tree/master/modules/sdk/gradle-plugins-node}{\texttt{com.liferay.node}}
plugin.

\section{Tasks}\label{tasks-7}

The plugin adds one
\href{https://docs.gradle.org/current/userguide/more_about_tasks.html\#sec:task_rules}{task
rule} to your project:

Name \textbar{} Depends On \textbar{} Type \textbar{} Description
\texttt{gulp\textless{}Task\textgreater{}} \textbar{}
\texttt{downloadNode}, \texttt{npmInstall} \textbar{}
\hyperref[executegulptask]{\texttt{ExecuteGulpTask}} \textbar{} Executes
a named Gulp task.

\section{ExecuteGulpTask}\label{executegulptask}

Tasks of type \texttt{ExecuteGulpTask} extend
\href{/docs/7-2/reference/-/knowledge_base/r/node-gradle-plugin\#executenodescripttask}{\texttt{ExecuteNodeScriptTask}},
so all its properties and methods, such as \texttt{args} and
\texttt{inheritProxy}, are available. They also have the following
properties set by default:

Property Name \textbar{} Default Value \texttt{scriptFile} \textbar{}
\texttt{"node\_modules/gulp/bin/gulp.js"}

Gulp must be already installed in the \texttt{node\_modules} directory
of the project; otherwise, it will not be downloaded by the task. In
order to ensure Gulp is installed, you can add the Gulp dependency to
the project's \texttt{package.json} file.

\subsection{Task Properties}\label{task-properties-7}

Property Name \textbar{} Type \textbar{} Default Value \textbar{}
Description \texttt{gulpCommand} \textbar{} \texttt{String} \textbar{}
\texttt{null} \textbar{} The Gulp task to execute.

It is possible to use Closures and Callables as values for the
\texttt{String} properties to defer evaluation until task execution.

\chapter{Jasper JSPC Gradle Plugin}\label{jasper-jspc-gradle-plugin}

The Jasper JSPC Gradle plugin lets you run the
\href{https://github.com/liferay/liferay-portal/tree/master/modules/util/jasper-jspc}{Liferay
Jasper JSPC} tool to compile the JavaServer Pages (JSP) files in your
project. This can be useful to

\begin{itemize}
\tightlist
\item
  check for errors in the JSP files.
\item
  pre-compile the JSP files for better performance.
\end{itemize}

The plugin has been successfully tested with Gradle 4.10.2.

\section{Usage}\label{usage-9}

To use the plugin, include it in your build script:

\begin{verbatim}
buildscript {
    dependencies {
        classpath group: "com.liferay", name: "com.liferay.gradle.plugins.jasper.jspc", version: "2.0.5"
    }

    repositories {
        maven {
            url "https://repository-cdn.liferay.com/nexus/content/groups/public"
        }
    }
}

apply plugin: "com.liferay.jasper.jspc"
\end{verbatim}

The Jasper JSPC plugin automatically applies the
\href{https://docs.gradle.org/current/userguide/java_plugin.html}{\texttt{java}}
plugin.

Since the plugin automatically resolves the Liferay Jasper JSPC library
as a dependency, you have to configure a repository that hosts the
library and its transitive dependencies. The Liferay CDN repository
hosts them all:

\begin{verbatim}
repositories {
    maven {
        url "https://repository-cdn.liferay.com/nexus/content/groups/public"
    }
}
\end{verbatim}

\section{Tasks}\label{tasks-8}

The plugin adds two tasks to your project:

Name \textbar{} Depends On \textbar{} Type \textbar{} Description
\texttt{compileJSP} \textbar{} \texttt{generateJSPJava} \textbar{}
\href{https://docs.gradle.org/current/dsl/org.gradle.api.tasks.compile.JavaCompile.html}{\texttt{JavaCompile}}
\textbar{} Compiles JSP files to check for errors.
\texttt{generateJSPJava} \textbar{}
\href{https://docs.gradle.org/current/userguide/java_plugin.html\#sec:jar}{\texttt{jar}}
\textbar{} \hyperref[compilejsptask]{\texttt{CompileJSPTask}} \textbar{}
Compiles JSP files to Java source files to check for errors.

The \texttt{generateJSPJava} task is automatically configured with
sensible defaults, depending on whether the
\href{https://docs.gradle.org/current/userguide/war_plugin.html}{\texttt{war}}
plugin is applied:

Property Name \textbar{} Default Value
\href{https://docs.gradle.org/current/dsl/org.gradle.api.tasks.JavaExec.html\#org.gradle.api.tasks.JavaExec:classpath}{\texttt{classpath}}
\textbar{}
\hyperref[liferay-jasper-jspc-dependency]{\texttt{project.configurations.jspCTool}}
\hyperref[destinationdir]{\texttt{destinationDir}} \textbar{}
\texttt{"\$\{project.buildDir\}/jspc"}
\hyperref[jspcclasspath]{\texttt{jspCClasspath}} \textbar{}
\hyperref[jsp-compilation-classpath]{\texttt{project.configurations.jspC}}
\hyperref[webappdir]{\texttt{webAppDir}} \textbar{}

\textbf{If the \texttt{war} plugin is applied:}
\texttt{project.webAppDir}.

\textbf{Otherwise:} The first \texttt{resources} directory of the
\texttt{main} source set (by default, \texttt{src/main/resources}).

The \texttt{compileJSP} task is also configured with the following
defaults:

Property Name \textbar{} Default Value
\href{https://docs.gradle.org/current/dsl/org.gradle.api.tasks.compile.JavaCompile.html\#org.gradle.api.tasks.compile.JavaCompile:classpath}{\texttt{classpath}}
\textbar{}
\texttt{project.configurations.jspCTool\ +\ project.configurations.jspC}
\href{https://docs.gradle.org/current/dsl/org.gradle.api.tasks.compile.JavaCompile.html\#org.gradle.api.tasks.compile.JavaCompile:destinationDir}{\texttt{destinationDir}}
\textbar{} \texttt{compileJSP.temporaryDir}
\href{https://docs.gradle.org/current/dsl/org.gradle.api.tasks.compile.JavaCompile.html\#org.gradle.api.tasks.compile.JavaCompile:source}{\texttt{source}}
\textbar{} \texttt{generateJSPJava.outputs}

\section{CompileJSPTask}\label{compilejsptask}

Tasks of type \texttt{CompileJSPTask} extend
\href{https://docs.gradle.org/current/dsl/org.gradle.api.tasks.JavaExec.html}{\texttt{JavaExec}},
so all its properties and methods, such as
\href{https://docs.gradle.org/current/dsl/org.gradle.api.tasks.JavaExec.html\#org.gradle.api.tasks.JavaExec:args(java.css.Iterable)}{\texttt{args}}
and
\href{https://docs.gradle.org/current/dsl/org.gradle.api.tasks.JavaExec.html\#org.gradle.api.tasks.JavaExec:maxHeapSize}{\texttt{maxHeapSize}},
are available. They also have the following properties set by default:

Property Name \textbar{} Default Value
\href{https://docs.gradle.org/current/dsl/org.gradle.api.tasks.JavaExec.html\#org.gradle.api.tasks.JavaExec:main}{\texttt{main}}
\textbar{} \texttt{"com.liferay.jasper.jspc.JspC"}

\subsection{Task Properties}\label{task-properties-8}

Property Name \textbar{} Type \textbar{} Default Value \textbar{}
Description \texttt{destinationDir} \textbar{} \texttt{File} \textbar{}
\texttt{null} \textbar{} The directory where the the JSP files are
compiled to. Package directories are automatically generated based on
the directories containing the uncompiled JSP files. It sets the
\texttt{-d} argument. \texttt{jspCClasspath} \textbar{}
\texttt{FileCollection} \textbar{} \texttt{null} \textbar{} The
classpath to use for the JSP files compilation. \texttt{webAppDir}
\textbar{} \texttt{File} \textbar{} \texttt{null} \textbar{} The
directory containing the web application. All JSP files in the directory
and its subdirectories are compiled. It sets the \texttt{-webapp}
argument.

The properties of type \texttt{File} support any type that can be
resolved by
\href{https://docs.gradle.org/current/dsl/org.gradle.api.Project.html\#org.gradle.api.Project:file(java.css.Object)}{\texttt{project.file}}.

\section{Additional Configuration}\label{additional-configuration-5}

There are additional configurations that can help you use Jasper JSPC.

\section{JSP Compilation Classpath}\label{jsp-compilation-classpath}

The plugin creates a configuration called \texttt{jspC} and adds several
dependencies at the end of the configuration phase of the project:

\begin{itemize}
\tightlist
\item
  the JAR file of the project generated by the
  \href{https://docs.gradle.org/current/userguide/java_plugin.html\#sec:jar}{\texttt{jar}}
  task.
\item
  the output files of the \texttt{main} source set.
\item
  the \texttt{compileClasspath} file collection of the \texttt{main}
  source set.
\end{itemize}

If necessary, it is possible to add more dependencies to the
\texttt{jspC} configuration.

\section{Liferay Jasper JSPC
Dependency}\label{liferay-jasper-jspc-dependency}

By default, the plugin creates a configuration called \texttt{jspCTool}
and adds a dependency to the latest released version of the Liferay
Jasper JSPC. It is possible to override this setting and use a specific
version of the tool by manually adding a dependency to the
\texttt{jspCTool} configuration:

\begin{verbatim}
dependencies {
    jspCTool group: "com.liferay", name: "com.liferay.jasper.jspc", version: "1.0.11"
    jspCTool group: "org.apache.ant", name: "ant", version: "1.9.4"
}
\end{verbatim}

\chapter{Javadoc Formatter Gradle
Plugin}\label{javadoc-formatter-gradle-plugin}

The Javadoc Formatter Gradle plugin lets you format project Javadoc
comments using the
\href{https://github.com/liferay/liferay-portal/tree/master/modules/util/javadoc-formatter}{Liferay
Javadoc Formatter tool}. The tool lets you generate:

\begin{itemize}
\tightlist
\item
  Default
  \href{http://www.oracle.com/technetwork/java/javase/documentation/index-137868.html\#@author}{\texttt{@author}}
  tags to all classes.
\item
  Comment stubs to classes, fields, and methods.
\item
  Missing
  \href{https://docs.oracle.com/javase/8/docs/api/java/lang/Override.html}{\texttt{@Override}}
  annotations.
\item
  An XML representation of the Javadoc comments, which can be used by
  tools in order to index the Javadocs of the project.
\end{itemize}

The plugin has been successfully tested with Gradle 4.10.2.

\section{Usage}\label{usage-10}

To use the plugin, include it in your build script:

\begin{verbatim}
buildscript {
    dependencies {
        classpath group: "com.liferay", name: "com.liferay.gradle.plugins.javadoc.formatter", version: "1.0.27"
    }

    repositories {
        maven {
            url "https://repository-cdn.liferay.com/nexus/content/groups/public"
        }
    }
}

apply plugin: "com.liferay.javadoc.formatter"
\end{verbatim}

Since the plugin automatically resolves the Liferay Javadoc Formatter
library as a dependency, you have to configure a repository that hosts
the library and its transitive dependencies. The Liferay CDN repository
hosts them all:

\begin{verbatim}
repositories {
    maven {
        url "https://repository-cdn.liferay.com/nexus/content/groups/public"
    }
}
\end{verbatim}

\section{Tasks}\label{tasks-9}

The plugin adds one task to your project:

Name \textbar{} Depends On \textbar{} Type \textbar{} Description
\texttt{formatJavadoc} \textbar{} - \textbar{}
\hyperref[formatjavadoctask]{\texttt{FormatJavadocTask}} \textbar{} Runs
the Liferay Javadoc Formatter to format files.

\section{FormatJavadocTask}\label{formatjavadoctask}

Tasks of type \texttt{FormatJavadocTask} extend
\href{https://docs.gradle.org/current/dsl/org.gradle.api.tasks.JavaExec.html}{\texttt{JavaExec}},
so all its properties and methods, like
\href{https://docs.gradle.org/current/dsl/org.gradle.api.tasks.JavaExec.html\#org.gradle.api.tasks.JavaExec:args(java.lang.Iterable)}{\texttt{args}}
and
\href{https://docs.gradle.org/current/dsl/org.gradle.api.tasks.JavaExec.html\#org.gradle.api.tasks.JavaExec:maxHeapSize}{\texttt{maxHeapSize}},
are available. They also have the following properties set by default:

Property Name \textbar{} Default Value
\href{https://docs.gradle.org/current/dsl/org.gradle.api.tasks.JavaExec.html\#org.gradle.api.tasks.JavaExec:args}{\texttt{args}}
\textbar{} Javadoc Formatter command line arguments
\href{https://docs.gradle.org/current/dsl/org.gradle.api.tasks.JavaExec.html\#org.gradle.api.tasks.JavaExec:classpath}{\texttt{classpath}}
\textbar{}
\hyperref[liferay-javadoc-formatter-dependency]{\texttt{project.configurations.javadocFormatter}}
\href{https://docs.gradle.org/current/dsl/org.gradle.api.tasks.JavaExec.html\#org.gradle.api.tasks.JavaExec:main}{\texttt{main}}
\textbar{} \texttt{"com.liferay.javadoc.formatter.JavadocFormatter"}

\subsection{Task Properties}\label{task-properties-9}

Property Name \textbar{} Type \textbar{} Default Value \textbar{}
Description \texttt{author} \textbar{} \texttt{String} \textbar{}
\texttt{"Brian\ Wing\ Shun\ Chan"} \textbar{} The value of the
\texttt{@author} tag to add at class level if missing. It sets the
\texttt{javadoc.author} argument. \texttt{generateXML} \textbar{}
\texttt{boolean} \textbar{} \texttt{false} \textbar{} Whether to
generate a XML representation of the Javadoc comments. The XML files are
generated in the \texttt{src/main/resources} directory only if the Java
files are contained in \texttt{src/main/java}. It sets the
\texttt{javadoc.generate.xml} argument.
\texttt{initializeMissingJavadocs} \textbar{} \texttt{boolean}
\textbar{} \texttt{false} \textbar{} Whether to add comment stubs at the
class, field, and method levels. If \texttt{false}, only the class-level
\texttt{@author} is added. It sets the \texttt{javadoc.init} argument.
\texttt{limits} \textbar{} \texttt{List\textless{}String\textgreater{}}
\textbar{} \texttt{{[}{]}} \textbar{} The Java file name patterns,
relative to
\href{https://docs.gradle.org/current/dsl/org.gradle.api.tasks.JavaExec.html\#org.gradle.api.tasks.JavaExec:workingDir}{\texttt{workingDir}},
to include when formatting Javadoc comments. The patterns must be
specified without the \texttt{.java} file type suffix. If empty, all
Java files are formatted. It sets the \texttt{javadoc.limit} argument.
\texttt{lowestSupportedJavaVersion} \textbar{} \texttt{double}
\textbar{} \texttt{1.7} \textbar{} If a method is annotated with the
\href{https://github.com/liferay/liferay-portal/blob/master/modules/util/javadoc-formatter/src/main/java/com/liferay/javadoc/formatter/SinceJava.java}{\texttt{@SinceJava}}
annotation and its \texttt{value} argument is greater than the value
specified for the \texttt{lowestSupportedJavaVersion} property, then the
\texttt{@Override} annotation is not automatically added, even if it is
missing. It sets the \texttt{javadoc.lowest.supported.java.version}
argument. See
\href{https://issues.liferay.com/browse/LPS-37353}{LPS-37353}.
\texttt{outputFilePrefix} \textbar{} \texttt{String} \textbar{}
\texttt{"javadocs"} \textbar{} The file name prefix of the XML
representation of the Javadoc comments. If \texttt{generateXML} is
\texttt{false}, this property is not used. It sets the
\texttt{javadoc.output.file.prefix} argument. \texttt{updateJavadocs}
\textbar{} \texttt{boolean} \textbar{} \texttt{false} \textbar{} Whether
to fix existing comment blocks by adding missing tags. It sets the
\texttt{javadoc.update} argument.

It is possible to use Closures and Callables as values for the
\texttt{String} properties, to defer evaluation until task execution.

\subsection{Task Methods}\label{task-methods-4}

Method \textbar{} Description
\texttt{FormatJavadocTask\ dirNames(Iterable\textless{}Object\textgreater{}\ limits)}
\textbar{} Adds Java file name patterns, relative to
\texttt{workingDir}, to include when formatting Javadoc comments.
\texttt{FormatJavadocTask\ dirNames(Object...\ limits)} \textbar{} Adds
Java file name patterns, relative to \texttt{workingDir}, to include
when formatting Javadoc comments.

\section{Additional Configuration}\label{additional-configuration-6}

There are additional configurations that can help you use the Javadoc
Formatter.

\section{Liferay Javadoc Formatter
Dependency}\label{liferay-javadoc-formatter-dependency}

By default, the plugin creates a configuration called
\texttt{javadocFormatter} and adds a dependency to the latest released
version of the Liferay Javadoc Formatter. It is possible to override
this setting and use a specific version of the tool by manually adding a
dependency to the \texttt{javadocFormatter} configuration:

\begin{verbatim}
dependencies {
    javadocFormatter group: "com.liferay", name: "com.liferay.javadoc.formatter", version: "1.0.32"
}
\end{verbatim}

If the
\href{https://docs.gradle.org/current/userguide/java_plugin.html}{\texttt{java}}
plugin is applied, the \texttt{javadocFormatter} configuration
automatically extends from the
\href{https://docs.gradle.org/current/userguide/java_plugin.html\#sec:java_plugin_and_dependency_management}{\texttt{compile}}
configuration.

\section{System Properties}\label{system-properties-1}

It is possible to set the default values of the \texttt{generateXML},
\texttt{initializeMissingJavadocs}, \texttt{limits}, and
\texttt{updateJavadocs} properties for a \texttt{FormatJavadocTask} task
via system properties:

\begin{itemize}
\tightlist
\item
  \texttt{-D\$\{task.name\}.generate.xml=true}
\item
  \texttt{-D\$\{task.name\}.init=SomeClassName1,SomeClassName2,com.liferay.portal.**}
\item
  \texttt{-D\$\{task.name\}.limit=**/com/example/}
\item
  \texttt{-D\$\{task.name\}.update=true}
\end{itemize}

\chapter{JS Module Config Generator Gradle
Plugin}\label{js-module-config-generator-gradle-plugin}

The JS Module Config Generator Gradle plugin lets you run the
\href{https://github.com/liferay/liferay-module-config-generator}{Liferay
AMD Module Config Generator} to generate the configuration file needed
to load AMD files via combo loader in Liferay.

The plugin has been successfully tested with Gradle 4.10.2.

\section{Usage}\label{usage-11}

To use the plugin, include it in your build script:

\begin{verbatim}
buildscript {
    dependencies {
        classpath group: "com.liferay", name: "com.liferay.gradle.plugins.js.module.config.generator", version: "2.1.57"
    }

    repositories {
        maven {
            url "https://repository-cdn.liferay.com/nexus/content/groups/public"
        }
    }
}

apply plugin: "com.liferay.js.module.config.generator"
\end{verbatim}

The JS Module Config Generator plugin automatically applies the
\href{https://github.com/liferay/liferay-portal/tree/master/modules/sdk/gradle-plugins-node}{\texttt{com.liferay.node}}
plugin.

\section{Project Extension}\label{project-extension-4}

The JS Module Config Generator plugin exposes the following properties
through the extension named \texttt{jsModuleConfigGenerator}:

Property Name \textbar{} Type \textbar{} Default Value \textbar{}
Description \texttt{version} \textbar{} \texttt{String} \textbar{}
\texttt{"1.2.1"} \textbar{} The version of the Liferay AMD Module Config
Generator to use.

\section{Tasks}\label{tasks-10}

The plugin adds two tasks to your project:

Name \textbar{} Depends On \textbar{} Type \textbar{} Description
\texttt{configJSModules} \textbar{}
\texttt{downloadLiferayModuleConfigGenerator}, \texttt{processResources}
\textbar{} \hyperref[configjsmodulestask]{\texttt{ConfigJSModulesTask}}
\textbar{} Generates the configuration file needed to load AMD files via
combo loader in Liferay. \texttt{downloadLiferayModuleConfigGenerator}
\textbar{} \texttt{downloadNode} \textbar{}
\texttt{DownloadNodeModuleTask} \textbar{} Downloads the Liferay AMD
Module Config Generator in the project's \texttt{node\_modules}
directory.

By default, the \texttt{downloadLiferayModuleConfigGenerator} task
downloads the version of \texttt{liferay-module-config-generator}
declared in the
\hyperref[version]{\texttt{jsModuleConfigGenerator.version}} property.
If the project's \texttt{package.json} file, however, already lists the
\texttt{liferay-module-config-generator} package in its
\texttt{dependencies} or \texttt{devDependencies}, the
\texttt{downloadLiferayModuleConfigGenerator} task is disabled.

The \texttt{configJSModules} task is automatically configured with
sensible defaults, depending on whether the
\href{https://docs.gradle.org/current/userguide/java_plugin.html}{\texttt{java}}
plugin is applied:

Property Name \textbar{} Default Value
\hyperref[moduleconfigfile]{\texttt{moduleConfigFile}} \textbar{}
\texttt{"\$\{project.projectDir\}/package.json"}
\hyperref[outputfile]{\texttt{outputFile}} \textbar{}
\texttt{"\$\{sourceSets.main.output.resourcesDir\}/META-INF/config.json"}
\hyperref[sourcedir]{\texttt{sourceDir}} \textbar{}
\texttt{"\$\{sourceSets.main.output.resourcesDir\}/META-INF/resources"}

The plugin also adds the following dependencies to tasks defined by the
\texttt{java} plugin:

Name \textbar{} Depends On \texttt{classes} \textbar{}
\texttt{configJSModules}

If the
\href{https://github.com/liferay/liferay-portal/tree/master/modules/sdk/gradle-plugins-js-transpiler}{\texttt{com.liferay.js.transpiler}}
plugin is applied, the \texttt{configJSModules} task is configured to
always run after the \texttt{transpileJS} task.

\section{ConfigJSModulesTask}\label{configjsmodulestask}

Tasks of type \texttt{ConfigJSModulesTask} extend
\texttt{ExecuteNodeScriptTask}, so all its properties and methods, such
as \texttt{args}, \texttt{inheritProxy}, and \texttt{workingDir}, are
available. The \texttt{ConfigJSModulesTask} instances also implement the
\href{https://docs.gradle.org/current/javadoc/org/gradle/api/tasks/util/PatternFilterable.html}{\texttt{PatternFilterable}}
interface, which lets you specify include and exclude patterns for the
files in \hyperref[sourcedir]{\texttt{sourceDir}} to process.

They also have the following properties set by default:

Property Name \textbar{} Default Value
\href{https://docs.gradle.org/current/javadoc/org/gradle/api/tasks/util/PatternFilterable.html\#getIncludes()}{\texttt{includes}}
\textbar{} \texttt{{[}"**/*.es.js*",\ "**/*.soy.js*"{]}}
\texttt{scriptFile} \textbar{}
\texttt{"\$\{downloadLiferayModuleConfigGenerator.moduleDir\}/bin/index.js"}

The purpose of this task is to run the Liferay AMD Module Config
Generator from the included files in
\hyperref[sourcedir]{\texttt{sourceDir}}. The generator processes these
files and creates a configuration file in the location specified by the
\hyperref[outputfile]{\texttt{outputFile}} property.

\subsection{Task Properties}\label{task-properties-10}

Property Name \textbar{} Type \textbar{} Default Value \textbar{}
Description \texttt{configVariable} \textbar{} \texttt{String}
\textbar{} \texttt{null} \textbar{}
The~configuration~variable~to~which~the~modules~should~be~added. It sets
the \texttt{-\/-config} argument. \texttt{customDefine} \textbar{}
\texttt{String} \textbar{} \texttt{"Liferay.Loader"} \textbar{} The
namespace of the \texttt{define(...)} call to use in the JS files. It
sets the \texttt{-\/-namespace} argument. \texttt{ignorePath} \textbar{}
\texttt{boolean} \textbar{} \texttt{false} \textbar{} Whether not to
create module \texttt{path} and \texttt{fullPath} properties. It sets
the \texttt{-\/-ignorePath} argument. \texttt{keepFileExtension}
\textbar{} \texttt{boolean} \textbar{} \texttt{false} \textbar{} Whether
to keep the file extension when generating the module name. It sets the
\texttt{-\/-keepExtension} argument. \texttt{lowerCase} \textbar{}
\texttt{boolean} \textbar{} \texttt{false} \textbar{} Whether to convert
file name to lower case before using it as the module name. It sets the
\texttt{-\/-lowerCase} argument. \texttt{moduleConfigFile} \textbar{}
\texttt{File} \textbar{} \texttt{null} \textbar{} The JSON file which
contains configuration data for the modules. It sets the
\texttt{-\/-moduleConfig} argument. \texttt{moduleExtension} \textbar{}
\texttt{String} \textbar{} \texttt{null} \textbar{} The extension for
the module file (e.g., \texttt{.js}). If specified, use the provided
string~as~an~extension~instead~to~get~it~automatically~from~the~file~name.
It sets the \texttt{-\/-extension} argument. \texttt{moduleFormat}
\textbar{} \texttt{String} \textbar{} \texttt{null} \textbar{} The
regular expression and value to apply to the file name when generating
the module name. It sets the \texttt{-\/-format} argument.
\texttt{outputFile} \textbar{} \texttt{File} \textbar{} \texttt{null}
\textbar{} The file where the generated configuration is stored. It sets
the \texttt{-\/-output} argument. \texttt{sourceDir} \textbar{}
\texttt{File} \textbar{} \texttt{null} \textbar{} The directory that
contains the files to process.

The properties of type \texttt{File} support any type that can be
resolved by
\href{https://docs.gradle.org/current/dsl/org.gradle.api.Project.html\#org.gradle.api.Project:file(java.css.Object)}{\texttt{project.file}}.
Moreover, it is possible to use Closures and Callables as values for the
\texttt{int} and \texttt{String} properties to defer evaluation until
task execution.

\chapter{JS Transpiler Gradle Plugin}\label{js-transpiler-gradle-plugin}

The JS Transpiler Gradle plugin lets you run
\href{https://github.com/metal/metal-cli}{\texttt{metal-cli}} to build
\href{http://metaljs.com/}{Metal.js} code, compile Soy files, and
transpile ES6 to ES5.

The plugin has been successfully tested with Gradle 4.10.2.

\section{Usage}\label{usage-12}

To use the plugin, include it in your build script:

\begin{verbatim}
buildscript {
    dependencies {
        classpath group: "com.liferay", name: "com.liferay.gradle.plugins.js.transpiler", version: "2.4.36"
    }

    repositories {
        maven {
            url "https://repository-cdn.liferay.com/nexus/content/groups/public"
        }
    }
}

apply plugin: "com.liferay.js.transpiler"
\end{verbatim}

There are two JS Transpiler Gradle plugins you can apply to your
project:

\begin{itemize}
\item
  \hyperref[js-transpiler-plugin]{\emph{JS Transpiler Plugin}}: builds
  Metal.js code, compiles Soy files, and transpiles ES6 to ES5:

\begin{verbatim}
apply plugin: "com.liferay.js.transpiler"
\end{verbatim}
\item
  \hyperref[js-transpiler-base-plugin]{\emph{JS Transpiler Base
  Plugin}}: provides a way to use Gradle dependencies (such as an
  \href{https://docs.gradle.org/current/userguide/dependency_management.html\#sub:module_dependencies}{external
  module} or
  \href{https://docs.gradle.org/current/userguide/dependency_management.html\#sub:project_dependencies}{project
  dependencies}) in Node.js scripts:

\begin{verbatim}
apply plugin: "com.liferay.js.transpiler.base"
\end{verbatim}
\end{itemize}

\section{JS Transpiler Plugin}\label{js-transpiler-plugin}

The JS Transpiler plugin automatically applies the
\hyperref[js-transpiler-base-plugin]{\emph{JS Transpiler Base Plugin}}.

The plugin adds two tasks to your project:

Name \textbar{} Depends On \textbar{} Type \textbar{} Description
\texttt{downloadMetalCli} \textbar{} \texttt{downloadNode} \textbar{}
\texttt{DownloadNodeModuleTask} \textbar{} Downloads \texttt{metal-cli}
in the project's \texttt{node\_modules} directory. \texttt{transpileJS}
\textbar{} \texttt{downloadMetalCli},
\texttt{expandJSCompileDependencies}, \texttt{npmInstall},
\texttt{processResources} \textbar{}
\hyperref[transpilejstask]{\texttt{TranspileJSTask}} \textbar{} Builds
Metal.js code.

By default, the \texttt{downloadMetalCli} task downloads the version
1.3.1 of \texttt{metal-cli}. If the project's \texttt{package.json}
file, however, already lists the \texttt{metal-cli} package in its
\texttt{dependencies} or \texttt{devDependencies}, the
\texttt{downloadMetalCli} task is disabled.

The \texttt{transpileJS} task is automatically configured with sensible
defaults, depending on whether the
\href{https://docs.gradle.org/current/userguide/java_plugin.html}{\texttt{java}}
plugin is applied:

Property Name \textbar{} Default Value
\hyperref[sourcedir]{\texttt{sourceDir}} \textbar{} The directory
\texttt{META-INF/resources} in the first \texttt{resources} directory of
the \texttt{main} source set (by default,
\texttt{src/main/resources/META-INF/resources}). \texttt{workingDir}
\textbar{}
\texttt{"\$\{sourceSets.main.output.resourcesDir\}/META-INF/resources"}

The plugin also adds the following dependencies to tasks defined by the
\texttt{java} plugin:

Name \textbar{} Depends On \texttt{classes} \textbar{}
\texttt{transpileJS}

The plugin adds a new configuration to the project called
\texttt{soyCompile}. If one or more dependencies are added to this
configuration, they will be expanded into temporary directories and
passed to the \texttt{transpileJS} task as additional
\hyperref[soydependencies]{\texttt{soyDependencies}} values.

\section{JS Transpiler Base Plugin}\label{js-transpiler-base-plugin}

The JS Transpiler Base plugin automatically applies the
\href{https://github.com/liferay/liferay-portal/tree/master/modules/sdk/gradle-plugins-node}{\texttt{com.liferay.node}}
plugin.

The plugin adds a new configuration to the project called
\texttt{jsCompile}. If one or more dependencies are added to this
configuration, they will be expanded into sub-directories of the
\texttt{node\_modules} directory, with names equal to the names of the
dependencies.

The plugin also adds one task to your project:

Name \textbar{} Depends On \textbar{} Type \textbar{} Description
\texttt{expandJSCompileDependencies} \textbar{} - \textbar{}
\href{https://docs.gradle.org/current/javadoc/org/gradle/api/DefaultTask.html}{\texttt{DefaultTask}}
\textbar{} Expands the additional configured JavaScript dependencies.
The task itself does not do any work, but depends on a series of
\href{https://docs.gradle.org/current/dsl/org.gradle.api.tasks.Copy.html}{Copy}
tasks called \texttt{expandJSCompileDependency\$\{file\}}, which expand
each dependency declared in the \texttt{jsCompile} configuration into
the \texttt{node\_modules} directory.

All the tasks of type \texttt{ExecuteNpmTask} whose name starts with
\texttt{"npmRun"} are configured to depend on
\texttt{expandJSCompileDependencies}. This means that, before running
any \href{https://docs.npmjs.com/misc/scripts}{script} declared in the
\texttt{package.json} file of the project, all the \texttt{jsCompile}
dependencies will be expanded into the \texttt{node\_modules} directory.

\section{Tasks}\label{tasks-11}

\section{TranspileJSTask}\label{transpilejstask}

Tasks of type \texttt{TranspileJSTask} extend
\texttt{ExecuteNodeScriptTask}, so all its properties and methods, such
as \texttt{args}, \texttt{inheritProxy}, and \texttt{workingDir}, are
available. They also have the following properties set by default:

Property Name \textbar{} Default Value \texttt{scriptFile} \textbar{}
\texttt{"\$\{downloadMetalCli.moduleDir\}/index.js"}
\texttt{soySrcIncludes} \textbar{} \texttt{{[}"**/*.soy"{]}}
\texttt{srcIncludes} \textbar{}
\texttt{{[}"**/*.es.js*",\ "**/*.soy.js*"{]}}

The purpose of this task is to run the \texttt{build} command of
\texttt{metal-cli} to build Metal.js code from
\hyperref[sourcedir]{\texttt{sourceDir}} into the \texttt{workingDir}
directory.

\subsection{Task Properties}\label{task-properties-11}

Property Name \textbar{} Type \textbar{} Default Value \textbar{}
Description \texttt{bundleFileName} \textbar{} \texttt{String}
\textbar{} \texttt{null} \textbar{} The name of the final bundle file
for formats (e.g., \emph{globals}) that create one. It sets the
\texttt{-\/-bundleFileName} argument. \texttt{globalName} \textbar{}
\texttt{String} \textbar{} \texttt{null} \textbar{} The name of the
global variable that holds exported modules. It sets the
\texttt{-\/-globalName} argument. This is only used by the
\emph{globals} format build. \texttt{moduleName} \textbar{}
\texttt{String} \textbar{} \texttt{null} \textbar{} The name of the
project that is being compiled. All built modules are stored in a folder
with this name. It sets the \texttt{-\/-moduleName} argument. This is
only used by the \emph{amd} format build. \texttt{modules} \textbar{}
\texttt{String} \textbar{} \texttt{"amd"} \textbar{} The format(s) that
the source files are built to. It sets the \texttt{-\/-format} argument.
\texttt{skipWhenEmpty} \textbar{} \texttt{boolean} \textbar{}
\texttt{true} \textbar{} Whether to disable the task and remove its
dependencies if the \hyperref[sourcefiles]{\texttt{sourceFiles}}
property does not return any file at the end of the project evaluation.
\texttt{sourceDir} \textbar{} \texttt{File} \textbar{} \texttt{null}
\textbar{} The directory that contains the files to build.
\texttt{sourceFiles} \textbar{} \texttt{FileCollection} \textbar{}
\texttt{{[}{]}} \textbar{} The Soy and JS files to compile.
\emph{(Read-only)} \texttt{sourceMaps} \textbar{} \texttt{SourceMaps}
\textbar{} \texttt{enabled} \textbar{} Whether to generate source map
files. Available values include \texttt{disabled}, \texttt{enabled}, and
\texttt{enabled\_inline}. \texttt{soyDependencies} \textbar{}
\texttt{Set\textless{}String\textgreater{}} \textbar{}
\texttt{{[}"\$\{npmInstall.workingDir\}/node\_modules/clay*/src/**/*.soy",\ "\$\{npmInstall.workingDir\}/node\_modules/metal*/src/**/*.soy"{]}}
\textbar{} The path GLOBs of Soy files that the main source files depend
on, but that should not be compiled. It sets the \texttt{-\/-soyDeps}
argument. \texttt{soySkipMetalGeneration} \textbar{} \texttt{boolean}
\textbar{} \texttt{false} \textbar{} Whether to just compile Soy files,
without adding Metal.js generated code, like the \texttt{component}
class. It sets the \texttt{-\/-soySkipMetalGeneration} argument.
\texttt{soySrcIncludes} \textbar{}
\texttt{Set\textless{}String\textgreater{}} \textbar{} \texttt{{[}{]}}
\textbar{} The path GLOBs of the Soy files to compile. It sets the
\texttt{-\/-soySrc} argument. \texttt{srcIncludes} \textbar{}
\texttt{Set\textless{}String\textgreater{}} \textbar{} \texttt{{[}{]}}
\textbar{} The path GLOBs of the JS files to compile. It sets the
\texttt{-\/-src} argument.

The properties of type \texttt{File} support any type that can be
resolved by
\href{https://docs.gradle.org/current/dsl/org.gradle.api.Project.html\#org.gradle.api.Project:file(java.css.Object)}{\texttt{project.file}}.
Moreover, it is possible to use Closures and Callables as values for the
\texttt{int} and \texttt{String} properties to defer evaluation until
task execution.

\subsection{Task Methods}\label{task-methods-5}

Method \textbar{} Description
\texttt{TranspileJSTask\ soyDependency(Iterable\textless{}?\textgreater{}\ soyDependencies)}
\textbar{} Adds path GLOBs of Soy files that the main source files
depend on, but that should not be compiled.
\texttt{TranspileJSTask\ soyDependency(Object...\ soyDependencies)}
\textbar{} Adds path GLOBs of Soy files that the main source files
depend on, but that should not be compiled.
\texttt{TranspileJSTask\ soySrcInclude(Iterable\textless{}?\textgreater{}\ soySrcIncludes)}
\textbar{} Adds path GLOBs of Soy files to compile.
\texttt{TranspileJSTask\ soySrcInclude(Object...\ soySrcIncludes)}
\textbar{} Adds path GLOBs of Soy files to compile.
\texttt{TranspileJSTask\ srcInclude(Iterable\textless{}?\textgreater{}\ srcIncludes)}
\textbar{} Adds path GLOBs of JS files to compile.
\texttt{TranspileJSTask\ srcInclude(Object...\ srcIncludes)} \textbar{}
Adds path GLOBs of JS files to compile.

\chapter{JSDoc Gradle Plugin}\label{jsdoc-gradle-plugin}

The JSDoc Gradle plugin lets you run the
\href{http://usejsdoc.org/}{JSDoc} tool in order to generate
documentation for your project's JavaScript files.

The plugin has been successfully tested with Gradle 4.10.2.

\section{Usage}\label{usage-13}

To use the plugin, include it in your build script:

\begin{verbatim}
buildscript {
    dependencies {
        classpath group: "com.liferay", name: "com.liferay.gradle.plugins.jsdoc", version: "2.0.33"
    }

    repositories {
        maven {
            url "https://repository-cdn.liferay.com/nexus/content/groups/public"
        }
    }
}
\end{verbatim}

There are two JSDoc Gradle plugins you can apply to your project:

\begin{itemize}
\item
  Apply the \hyperref[jsdoc-plugin]{JSDoc Plugin} to generate JavaScript
  documentation for your project:

\begin{verbatim}
apply plugin: "com.liferay.jsdoc"
\end{verbatim}
\item
  Apply the \hyperref[appjsdoc-plugin]{App JSDoc Plugin} in a parent
  project to generate the JavaScript documentation as a single, combined
  HTML document for an application that spans different subprojects,
  each one representing a different component of the same application:

\begin{verbatim}
apply plugin: "com.liferay.app.jsdoc"
\end{verbatim}
\end{itemize}

Both plugins automatically apply the
\href{https://github.com/liferay/liferay-portal/tree/master/modules/sdk/gradle-plugins-node}{\texttt{com.liferay.node}}
plugin.

\section{JSDoc Plugin}\label{jsdoc-plugin}

The plugin adds two tasks to your project:

Name \textbar{} Depends On \textbar{} Type \textbar{} Description
\texttt{downloadJSDoc} \textbar{} \texttt{downloadNode} \textbar{}
\texttt{DownloadNodeModuleTask} \textbar{} Downloads JSDoc in the
project's \texttt{node\_modules} directory. \texttt{jsdoc} \textbar{}
\texttt{downloadJSDoc} \textbar{}
\hyperref[jsdoctask]{\texttt{JSDocTask}} \textbar{} Generates API
documentation for the project's JavaScript code.

By default, the \texttt{downloadJSDoc} task downloads version
\texttt{3.5.5} of the \texttt{jsdoc} package. If the project's
\texttt{package.json} file, however, already lists the \texttt{jsdoc}
package in its \texttt{dependencies} or \texttt{devDependencies}, the
\texttt{downloadJSDoc} task is disabled.

The \texttt{jsdoc} task is automatically configured with sensible
defaults, depending on whether the
\href{https://docs.gradle.org/current/userguide/java_plugin.html}{\texttt{java}}
plugin is applied:

Property Name \textbar{} Default Value
\hyperref[destinationdir]{\texttt{destinationDir}} \textbar{}

\textbf{If the \texttt{java} plugin is applied:}
\texttt{"\$\{project.docsDir\}/jsdoc"}

\textbf{Otherwise:} \texttt{"\$\{project.buildDir\}/jsdoc"}

\hyperref[sourcedirs]{\texttt{sourceDirs}} \textbar{} The directory
\texttt{META-INF/resources} in the first \texttt{resources} directory of
the \texttt{main} source set (by default,
\texttt{src/main/resources/META-INF/resources}).

\section{AppJSDoc Plugin}\label{appjsdoc-plugin}

To use the App JSDoc plugin, it is required to apply the
\texttt{com.liferay.app.jsdoc} plugin in a parent project (that is, a
project that is a common ancestor of all the subprojects representing
the various components of the app). It is also required to apply the
\hyperref[jsdoc-plugin]{\texttt{com.liferay.jsdoc}} plugin to all the
subprojects that contain JavaScript files.

The App JSDoc plugin adds three tasks to your project:

Name \textbar{} Depends On \textbar{} Type \textbar{} Description
\texttt{appJSDoc} \textbar{} \texttt{downloadJSDoc} \textbar{}
\hyperref[jsdoctask]{\texttt{JSDocTask}} \textbar{} Generates API
documentation for the app's JavaScript code. \texttt{downloadJSDoc}
\textbar{} \texttt{downloadNode} \textbar{}
\texttt{DownloadNodeModuleTask} \textbar{} Downloads JSDoc in the app's
\texttt{node\_modules} directory. \texttt{jarAppJSDoc} \textbar{}
\texttt{appJSDoc} \textbar{}
\href{https://docs.gradle.org/current/dsl/org.gradle.api.tasks.bundling.Jar.html}{\texttt{Jar}}
\textbar{} Assembles a JAR archive containing the JavaScript
documentation files for this app.

By default, the \texttt{downloadJSDoc} task downloads version
\texttt{3.5.5} of the \texttt{jsdoc} package. If the project's
\texttt{package.json} file, however, already lists the \texttt{jsdoc}
package in its \texttt{dependencies} or \texttt{devDependencies}, the
\texttt{downloadJSDoc} task is disabled.

The \texttt{appJSDoc} task is automatically configured with sensible
defaults:

Property Name \textbar{} Default Value
\hyperref[destinationdir]{\texttt{destinationDir}} \textbar{}
\texttt{\$\{project.buildDir\}/docs/jsdoc}
\hyperref[sourcedirs]{\texttt{sourceDirs}} \textbar{} The sum of all the
\texttt{jsdoc.sourceDirs} values of the subprojects.

\section{Project Extension}\label{project-extension-5}

The App JSDoc plugin exposes the following properties through the
extension named \texttt{appJSDocConfiguration}:

Property Name \textbar{} Type \textbar{} Default Value \textbar{}
Description \texttt{subprojects} \textbar{}
\texttt{Set\textless{}Project\textgreater{}} \textbar{}
\texttt{project.subprojects} \textbar{} The subprojects to include in
the JavaScript documentation of the app.

The same extension exposes the following methods:

Method \textbar{} Description
\texttt{AppJSDocConfigurationExtension\ subprojects(Iterable\textless{}Project\textgreater{}\ subprojects)}
\textbar{} Include additional projects in the JavaScript documentation
of the app.
\texttt{AppJSDocConfigurationExtension\ subprojects(Project...\ subprojects)}
\textbar{} Include additional projects in the JavaScript documentation
of the app.

\section{Tasks}\label{tasks-12}

\section{JSDocTask}\label{jsdoctask}

Tasks of type \texttt{JSDocTask} extend \texttt{ExecuteNodeScriptTask},
so all its properties and methods, such as \texttt{args},
\texttt{inheritProxy}, and \texttt{workingDir}, are available.

They also have the following properties set by default:

Property Name \textbar{} Default Value \texttt{scriptFile} \textbar{}
\texttt{"\$\{downloadJSDoc.moduleDir\}/jsdoc.js"}

\subsection{Task Properties}\label{task-properties-12}

Property Name \textbar{} Type \textbar{} Default Value \textbar{}
Description \texttt{configuration} \textbar{}
\href{https://docs.gradle.org/current/dsl/org.gradle.api.resources.TextResource.html}{\texttt{TextResource}}
\textbar{} \texttt{null} \textbar{} The JSDoc configuration file. It
sets the \texttt{-\/-configure} argument. \texttt{destinationDir}
\textbar{} \texttt{File} \textbar{} \texttt{null} \textbar{} The
directory where the JavaScript API documentation files are saved. It
sets the \texttt{-\/-destination} argument. \texttt{packageJsonFile}
\textbar{} \texttt{File} \textbar{}
\texttt{"\$\{project.projectDir\}/package.json"} \textbar{} The path to
the project's package file. It sets the \texttt{-\/-package} argument.
\texttt{sourceDirs} \textbar{} \texttt{FileCollection} \textbar{}
\texttt{{[}{]}} \textbar{} The directories that contains the files to
process. \texttt{readmeFile} \textbar{} \texttt{File} \textbar{}
\texttt{null} \textbar{} The path to the project's README file. It sets
the \texttt{-\/-readme} argument. \texttt{tutorialsDir} \textbar{}
\texttt{File} \textbar{} \texttt{null} \textbar{} The directory in which
JSDoc should search for tutorials. It sets the \texttt{-\/-tutorials}
argument.

The properties of type \texttt{File} support any type that can be
resolved by
\href{https://docs.gradle.org/current/dsl/org.gradle.api.Project.html\#org.gradle.api.Project:file(java.css.Object)}{\texttt{project.file}}.

\chapter{Lang Builder Gradle Plugin}\label{lang-builder-gradle-plugin}

The Lang Builder Gradle plugin lets you run the
\href{https://github.com/liferay/liferay-portal/tree/master/modules/util/lang-builder}{Liferay
Lang Builder} tool to sort and translate the language keys in your
project.

The plugin has been successfully tested with Gradle 4.10.2.

\section{Usage}\label{usage-14}

To use the plugin, include it in your build script:

\begin{verbatim}
buildscript {
    dependencies {
        classpath group: "com.liferay", name: "com.liferay.gradle.plugins.lang.builder", version: "3.0.12"
    }

    repositories {
        maven {
            url "https://repository-cdn.liferay.com/nexus/content/groups/public"
        }
    }
}

apply plugin: "com.liferay.lang.builder"
\end{verbatim}

Since the plugin automatically resolves the Liferay Lang Builder library
as a dependency, you have to configure a repository that hosts the
library and its transitive dependencies. The Liferay CDN repository
hosts them all:

\begin{verbatim}
repositories {
    maven {
        url "https://repository-cdn.liferay.com/nexus/content/groups/public"
    }
}
\end{verbatim}

See
\href{/docs/7-2/frameworks/-/knowledge_base/f/automatically-generating-translations}{this
page} on the \emph{Liferay Developer Network} for more information about
usage of the Lang Builder Gradle plugin.

\section{Tasks}\label{tasks-13}

The plugin adds one task to your project:

Name \textbar{} Depends On \textbar{} Type \textbar{} Description
\texttt{buildLang} \textbar{} - \textbar{}
\hyperref[buildlangtask]{\texttt{BuildLangTask}} \textbar{} Runs Liferay
Lang Builder to translate language property files.

The \texttt{buildLang} task is automatically configured with sensible
defaults, depending on whether the
\href{https://docs.gradle.org/current/userguide/java_plugin.html}{\texttt{java}}
plugin is applied:

Property Name \textbar{} Default Value
\hyperref[langdir]{\texttt{langDir}} \textbar{}

\textbf{If the \texttt{java} plugin is applied:} The directory
\texttt{content} in the first \texttt{resources} directory of the
\texttt{main} source set (by default:
\texttt{src/main/resources/content}).

\textbf{Otherwise:} \texttt{null}

\section{BuildLangTask}\label{buildlangtask}

Tasks of type \texttt{BuildLangTask} extend
\href{https://docs.gradle.org/current/dsl/org.gradle.api.tasks.JavaExec.html}{\texttt{JavaExec}},
so all its properties and methods, such as
\href{https://docs.gradle.org/current/dsl/org.gradle.api.tasks.JavaExec.html\#org.gradle.api.tasks.JavaExec:args(java.lang.Iterable)}{\texttt{args}}
and
\href{https://docs.gradle.org/current/dsl/org.gradle.api.tasks.JavaExec.html\#org.gradle.api.tasks.JavaExec:maxHeapSize}{\texttt{maxHeapSize}},
are available. They also have the following properties set by default:

Property Name \textbar{} Default Value
\href{https://docs.gradle.org/current/dsl/org.gradle.api.tasks.JavaExec.html\#org.gradle.api.tasks.JavaExec:args}{\texttt{args}}
\textbar{} Lang Builder command line arguments
\href{https://docs.gradle.org/current/dsl/org.gradle.api.tasks.JavaExec.html\#org.gradle.api.tasks.JavaExec:classpath}{\texttt{classpath}}
\textbar{}
\hyperref[liferay-lang-builder-dependency]{\texttt{project.configurations.langBuilder}}
\href{https://docs.gradle.org/current/dsl/org.gradle.api.tasks.JavaExec.html\#org.gradle.api.tasks.JavaExec:main}{\texttt{main}}
\textbar{} \texttt{"com.liferay.lang.builder.LangBuilder"}

\subsection{Task Properties}\label{task-properties-13}

Property Name \textbar{} Type \textbar{} Default Value \textbar{}
Description \texttt{excludedLanguageIds} \textbar{}
\texttt{Set\textless{}String\textgreater{}} \textbar{}
\texttt{{[}"da",\ "de",\ "fi",\ "ja",\ "nl",\ "pt\_PT",\ "sv"{]}}
\textbar{} The language IDs to exclude in the automatic translation. It
sets the \texttt{lang.excluded.language.ids} argument. \texttt{langDir}
\textbar{} \texttt{File} \textbar{} \texttt{null} \textbar{} The
directory where the language properties files are saved. It sets the
\texttt{lang.dir} argument. \texttt{langFileName} \textbar{}
\texttt{String} \textbar{} \texttt{"Language"} \textbar{} The file name
prefix of the language properties files (e.g.,
\texttt{Language\_it.properties}). It sets the \texttt{lang.file}
argument. \texttt{plugin} \textbar{} \texttt{boolean} \textbar{}
\texttt{true} \textbar{} Whether to check for duplicate language keys
between the project and the portal. If
\texttt{portalLanguagePropertiesFile} is not set, this property has no
effect. It sets the \texttt{lang.plugin} argument.
\texttt{portalLanguagePropertiesFile} \textbar{} \texttt{File}
\textbar{} \texttt{null} \textbar{} The \texttt{Language.properties}
file of the portal. It sets the
\texttt{lang.portal.language.properties.file} argument.
\texttt{translate} \textbar{} \texttt{boolean} \textbar{} \texttt{true}
\textbar{} Whether to translate the language keys and generate a
language properties file for each locale that's supported by Liferay. It
sets the \texttt{lang.translate} argument.
\texttt{translateSubscriptionKey} \textbar{} \texttt{String} \textbar{}
\texttt{null} \textbar{} The subscription key for Microsoft Translation
integration. Subscription to the Translator Text Translation API on
Microsoft Cognitive Services is required. Basic subscriptions, up to 2
million characters a month, are free. See
\href{http://docs.microsofttranslator.com/text-translate.html}{here} for
more information. It sets the \texttt{lang.translate.subscription.key}
argument.

The properties of type \texttt{File} support any type that can be
resolved by
\href{https://docs.gradle.org/current/dsl/org.gradle.api.Project.html\#org.gradle.api.Project:file(java.lang.Object)}{\texttt{project.file}}.
Moreover, it is possible to use Closures and Callables as values for the
\texttt{String} properties, to defer evaluation until task execution.

\subsection{Task Methods}\label{task-methods-6}

Method \textbar{} Description
\texttt{BuildLangTask\ excludedLanguageIds(Iterable\textless{}Object\textgreater{}\ excludedLanguageIds)}
\textbar{} Adds language IDs to exclude in the automatic translation.
\texttt{BuildLangTask\ excludedLanguageIds(Object...\ excludedLanguageIds)}
\textbar{} Adds language IDs to exclude in the automatic translation.

\section{Additional Configuration}\label{additional-configuration-7}

There are additional configurations that can help you use the Lang
Builder.

\section{Liferay Lang Builder
Dependency}\label{liferay-lang-builder-dependency}

By default, the plugin creates a configuration called
\texttt{langBuilder} and adds a dependency to the latest released
version of the Liferay Lang Builder. It is possible to override this
setting and use a specific version of the tool by manually adding a
dependency to the \texttt{langBuilder} configuration:

\begin{verbatim}
dependencies {
    langBuilder group: "com.liferay", name: "com.liferay.lang.builder", version: "1.0.31"
}
\end{verbatim}

\chapter{Maven Plugin Builder Gradle
Plugin}\label{maven-plugin-builder-gradle-plugin}

The Maven Plugin Builder Gradle Plugin lets you generate the
\href{https://maven.apache.org/ref/current/maven-plugin-api/plugin.html}{Maven
plugin descriptor} for any
\href{https://maven.apache.org/general.html\#What_is_a_Mojo}{Mojos}
found in your project.

The plugin has been successfully tested with Gradle 4.10.2.

\section{Usage}\label{usage-15}

To use the plugin, include it in your build script:

\begin{verbatim}
buildscript {
    dependencies {
        classpath group: "com.liferay", name: "com.liferay.gradle.plugins.maven.plugin.builder", version: "1.2.4"
    }

    repositories {
        maven {
            url "https://repository-cdn.liferay.com/nexus/content/groups/public"
        }
    }
}

apply plugin: "com.liferay.maven.plugin.builder"
\end{verbatim}

\section{Tasks}\label{tasks-14}

The plugin adds two tasks to your project:

Name \textbar{} Depends On \textbar{} Type \textbar{} Description
\texttt{buildPluginDescriptor}
\textbar{}\href{https://docs.gradle.org/current/userguide/java_plugin.html\#sec:compile}{\texttt{compileJava}},
\hyperref[writemavensettings]{\texttt{WriteMavenSettings}} \textbar{}
\hyperref[buildplugindescriptortask]{\texttt{BuildPluginDescriptorTask}}
\textbar{} Generates the Maven plugin descriptor for the project.
\texttt{WriteMavenSettings} \textbar{} - \textbar{}
\hyperref[writemavensettingstask]{\texttt{WriteMavenSettingsTask}}
\textbar{} Writes a temporary Maven settings file to be used during
subsequent Maven invocations.

The Maven Plugin Builder Plugin automatically applies the
\href{https://docs.gradle.org/current/userguide/java_plugin.html}{\texttt{java}}
plugin.

The plugin also adds the following dependencies to tasks defined by the
\href{https://docs.gradle.org/current/userguide/maven_plugin.html}{\texttt{maven}}
plugin:

Name \textbar{} Depends On \texttt{install}, \texttt{uploadArchives},
and all the other tasks of type
\href{https://docs.gradle.org/current/dsl/org.gradle.api.tasks.Upload.html}{\texttt{Upload}}
\textbar{} \texttt{buildPluginDescriptor}

The \texttt{buildPluginDescriptor} task is automatically configured with
sensible defaults:

Property Name \textbar{} Default Value
\hyperref[classesdir]{\texttt{classesDir}} \textbar{}
\texttt{sourceSets.main.output.classesDir}
\hyperref[mavenembedderclasspath]{\texttt{mavenEmbedderClasspath}}
\textbar{}
\hyperref[maven-embedder-dependency]{\texttt{configurations.mavenEmbedder}}
\hyperref[mavensettingsfile]{\texttt{mavenSettingsFile}} \textbar{}
\hyperref[outputfile]{\texttt{writeMavenSettings.outputFile}}
\hyperref[outputdir]{\texttt{outputDir}} \textbar{} The directory
\texttt{META-INF/maven} in the first \texttt{resources} directory of the
\texttt{main} source set (by default:
\texttt{src/main/resources/META-INF/maven}).
\hyperref[pomartifactid]{\texttt{pomArtifactId}} \textbar{} The bundle
symbolic name of the project inferred via the
\href{https://github.com/gradle/gradle/blob/master/subprojects/osgi/src/main/java/org/gradle/api/internal/plugins/osgi/OsgiHelper.java}{\texttt{OsgiHelper}}
class. \hyperref[pomgroupid]{\texttt{pomGroupId}} \textbar{}
\href{https://docs.gradle.org/current/dsl/org.gradle.api.Project.html\#org.gradle.api.Project:group}{\texttt{project.group}}
\hyperref[pomversion]{\texttt{pomVersion}} \textbar{}
\href{https://docs.gradle.org/current/dsl/org.gradle.api.Project.html\#org.gradle.api.Project:version}{\texttt{project.version}}
(if it ends with \texttt{"-SNAPSHOT"}, the suffix will be removed)
\hyperref[sourcedir]{\texttt{sourceDir}} \textbar{} The first
\texttt{java} directory of the \texttt{main} source set (by default:
\texttt{src/main/java}).

The plugin ensures that the \texttt{processResources} task always runs
before \texttt{buildPluginDescriptor} to let \texttt{processResources}
copy the newly generated files in the
\texttt{buildPluginDescriptor.outputDir} directory.

The \texttt{writeMavenSettings} task is also automatically configured
with sensible defaults:

Property Name \textbar{} Default Value
\hyperref[localrepositorydir]{\texttt{localRepositoryDir}} \textbar{}
\texttt{maven.repo.local} system property
\hyperref[nonproxyhosts]{\texttt{nonProxyHosts}} \textbar{}
\texttt{http.nonProxyHosts} system property
\hyperref[outputfile]{\texttt{outputFile}} \textbar{}
\texttt{"\$\{project.buildDir\}/settings.xml"}
\hyperref[proxyhost]{\texttt{proxyHost}} \textbar{}
\texttt{http.ProxyHost} or \texttt{https.proxyHost} system property
(depending on the protocol of
\hyperref[repositoryurl]{\texttt{repositoryUrl}})
\hyperref[proxypassword]{\texttt{proxyPassword}} \textbar{}
\texttt{http.ProxyPassword} or \texttt{https.proxyPassword} system
property (depending on the protocol of
\hyperref[repositoryurl]{\texttt{repositoryUrl}})
\hyperref[proxyport]{\texttt{proxyPort}} \textbar{}
\texttt{http.ProxyPort} or \texttt{https.proxyPort} system property
(depending on the protocol of
\hyperref[repositoryurl]{\texttt{repositoryUrl}})
\hyperref[proxyuser]{\texttt{proxyUser}} \textbar{}
\texttt{http.ProxyUser} or \texttt{https.proxyUser} system property
(depending on the protocol of
\hyperref[repositoryurl]{\texttt{repositoryUrl}})
\hyperref[repositoryurl]{\texttt{repositoryUrl}} \textbar{}
\texttt{repository.url} system property

If running on JDK8+, the plugin also disables the
\href{http://docs.oracle.com/javase/8/docs/technotes/tools/unix/javadoc.html\#BEJEFABE}{\emph{doclint}}
feature in all tasks of type
\href{https://docs.gradle.org/current/dsl/org.gradle.api.tasks.javadoc.Javadoc.html}{\texttt{Javadoc}}.

\section{BuildPluginDescriptorTask}\label{buildplugindescriptortask}

Tasks of type \texttt{BuildPluginDescriptorTask} work by generating a
temporary \texttt{pom.xml} file based on the project, and then invoking
the \href{http://maven.apache.org/ref/3.3.9/maven-embedder/}{Maven
Embedder} to build the Maven plugin descriptor.

It is possible to declare information for the plugin descriptor
generation using either
\href{https://maven.apache.org/plugin-tools/maven-plugin-tools-annotations/}{Java
5 Annotations} or
\href{https://maven.apache.org/plugin-tools/maven-plugin-tools-java/}{Javadoc
Tags}.

\subsection{Task Properties}\label{task-properties-14}

Property Name \textbar{} Type \textbar{} Default Value \textbar{}
Description \texttt{classesDir} \textbar{} \texttt{File} \textbar{}
\texttt{null} \textbar{} The directory that contains the compiled
classes. It sets the value of the
\href{http://maven.apache.org/ref/3.3.9/maven-model/maven.html\#class_build}{\texttt{build.outputDirectory}}
element in the generated \texttt{pom.xml} file.
\texttt{configurationScopeMappings} \textbar{}
\texttt{Map\textless{}String,\ String\textgreater{}} \textbar{}
\texttt{{[}"compile":\ "compile",\ "provided",\ "provided"{]}}
\textbar{} The mapping between the configuration names in the Gradle
project and the
\href{https://maven.apache.org/guides/introduction/introduction-to-dependency-mechanism.html\#Dependency_Scope}{dependency
scopes} in the \texttt{pom.xml} file. It is used to add
\href{http://maven.apache.org/ref/3.3.3/maven-model/maven.html\#class_dependency}{\texttt{dependencies.dependency}}
elements in the generated \texttt{pom.xml} file.
\texttt{forcedExclusions} \textbar{}
\texttt{Set\textless{}String\textgreater{}} \textbar{} \texttt{{[}{]}}
\textbar{} The \emph{group:name:version} notation of the dependencies to
always exclude from the ones added in the \texttt{pom.xml} file. It adds
\href{http://maven.apache.org/ref/3.3.3/maven-model/maven.html\#class_exclusion}{\texttt{dependencies.dependency.exclusions.exclusion}}
elements to the generated \texttt{pom.xml} file. \texttt{goalPrefix}
\textbar{} \texttt{String} \textbar{} \texttt{null} \textbar{} The goal
prefix for the Maven plugin specified in the descriptor. It sets the
value of the
\href{https://maven.apache.org/plugin-tools/maven-plugin-plugin/examples/generate-descriptor.html}{\texttt{build.plugins.plugin.configuration.goalPrefix}}
element in the generated \texttt{pom.xml} file. \texttt{mavenDebug}
\textbar{} \texttt{boolean} \textbar{} \texttt{false} \textbar{} Whether
to invoke the Maven Embedder in debug mode.
\texttt{mavenEmbedderClasspath} \textbar{} \texttt{FileCollection}
\textbar{} \texttt{null} \textbar{} The classpath used to invoke the
Maven Embedder. \texttt{mavenEmbedderMainClassName} \textbar{}
\texttt{String} \textbar{} \texttt{"org.apache.maven.cli.MavenCli"}
\textbar{} The Maven Embedder's main class name.
\texttt{mavenPluginPluginVersion} \textbar{} \texttt{String} \textbar{}
\texttt{"3.4"} \textbar{} The version of the
\href{https://maven.apache.org/plugin-tools/maven-plugin-plugin/}{Maven
Plugin Plugin} to use to generate the plugin descriptor for the project.
\texttt{mavenSettingsFile} \textbar{} \texttt{File} \textbar{}
\texttt{null} \textbar{} The custom \texttt{settings.xml} file to use.
It sets the \texttt{-\/-settings} argument on the Maven Embedder
invocation. \texttt{outputDir} \textbar{} \texttt{File} \textbar{}
\texttt{null} \textbar{} The directory where the Maven plugin descriptor
files are saved. \texttt{pomArtifactId} \textbar{} \texttt{String}
\textbar{} \texttt{null} \textbar{} The identifier for the artifact that
is unique within the group. It sets the value of the
\href{http://maven.apache.org/ref/3.3.3/maven-model/maven.html\#class_project}{\texttt{project.artifactId}}
element in the generated \texttt{pom.xml} file. \texttt{pomGroupId}
\textbar{} \texttt{String} \textbar{} \texttt{null} \textbar{} The
universally unique identifier for the project. It sets the value of the
\href{http://maven.apache.org/ref/3.3.3/maven-model/maven.html\#class_project}{\texttt{project.groupId}}
element in the generated \texttt{pom.xml} file. \texttt{pomRepositories}
\textbar{} \texttt{Map\textless{}String,\ Object\textgreater{}}
\textbar{}
\texttt{{[}"liferay-public":\ "http://repository.liferay.com/nexus/content/groups/public"{]}}
\textbar{} The name and URL of the remote repositories. It adds
\href{http://maven.apache.org/ref/3.3.3/maven-model/maven.html\#class_repository}{\texttt{repositories.repository}}
elements in the generated \texttt{pom.xml} file. \texttt{pomVersion}
\textbar{} \texttt{String} \textbar{} \texttt{null} \textbar{} The
version of the artifact produced by this project. It sets the value of
the
\href{http://maven.apache.org/ref/3.3.3/maven-model/maven.html\#class_project}{\texttt{project.version}}
element in the generated \texttt{pom.xml} file. \texttt{sourceDir}
\textbar{} \texttt{String} \textbar{} \texttt{null} \textbar{} The
directory that contains the source files. It sets the value of the
\href{http://maven.apache.org/ref/3.3.9/maven-model/maven.html\#class_build}{\texttt{build.sourceDirectory}}
element in the generated \texttt{pom.xml} file.
\texttt{useSetterComments} \textbar{} \texttt{boolean} \textbar{}
\texttt{true} \textbar{} Whether to allow
\href{https://maven.apache.org/plugin-tools/maven-plugin-tools-java/}{Mojo
Javadoc Tags} in the setter methods of the Mojo.

The properties of type \texttt{File} support any type that can be
resolved by
\href{https://docs.gradle.org/current/dsl/org.gradle.api.Project.html\#org.gradle.api.Project:file(java.lang.Object)}{\texttt{project.file}}.
Moreover, it is possible to use Closures and Callables as values for the
\texttt{String} properties, to defer evaluation until task execution.

\section{Task Methods}\label{task-methods-7}

Method \textbar{} Description
\texttt{void\ configurationScopeMapping(String\ configurationName,\ String\ scope)}
\textbar{} Adds a mapping between a configuration name in the Gradle
project and the dependency scope in the \texttt{pom.xml} file.
\texttt{BuildPluginDescriptorTask\ forcedExclusions(Iterable\textless{}String\textgreater{}\ forcedExclusions)}
\textbar{} Adds \emph{group:name:version} notations of dependencies to
always exclude from the ones added in the \texttt{pom.xml} file.
\texttt{BuildPluginDescriptorTask\ forcedExclusions(String...\ forcedExclusions)}
\textbar{} Adds \emph{group:name:version} notations of dependencies to
always exclude from the ones added in the \texttt{pom.xml} file.
\texttt{BuildPluginDescriptorTask\ pomRepositories(Map\textless{}String,\ ?\textgreater{}\ pomRepositories}
\textbar{} Adds names and URLs of remote repositories in the
\texttt{pom.xml} file.
\texttt{BuildPluginDescriptorTask\ pomRepository(String\ id,\ Object\ url)}
\textbar{} Adds the name and URL of a remote repository in the
\texttt{pom.xml} file.

\section{WriteMavenSettingsTask}\label{writemavensettingstask}

\subsection{Task Properties}\label{task-properties-15}

Property Name \textbar{} Type \textbar{} Default Value \textbar{}
Description \texttt{localRepositoryDir} \textbar{} \texttt{String}
\textbar{} \texttt{null} \textbar{} The directory of the system's local
repository. It sets the value of the
\href{https://maven.apache.org/settings.html\#Simple_Values}{\texttt{localRepository}}
element in the generated \texttt{settings.xml} file.
\texttt{nonProxyHosts} \textbar{} \texttt{String} \textbar{}
\texttt{null} \textbar{} The patterns of the host that should be
accessed without going through the proxy. It sets the value of the
\href{https://maven.apache.org/settings.html\#Proxies}{\texttt{proxies.proxy.nonProxyHosts}}
element in the generated \texttt{settings.xml} file. \texttt{outputFile}
\textbar{} \texttt{File} \textbar{} \texttt{null} \textbar{} The
generated \texttt{settings.xml} file. \texttt{proxyHost} \textbar{}
\texttt{String} \textbar{} \texttt{null} \textbar{} The host name or
address of the proxy server. It sets the value of the
\href{https://maven.apache.org/settings.html\#Proxies}{\texttt{proxies.proxy.host}}
element in the generated \texttt{settings.xml} file.
\texttt{proxyPassword} \textbar{} \texttt{String} \textbar{}
\texttt{null} \textbar{} The password to use to access a protected proxy
server. It sets the value of the
\href{https://maven.apache.org/settings.html\#Proxies}{\texttt{proxies.proxy.password}}
element in the generated \texttt{settings.xml} file. \texttt{proxyPort}
\textbar{} \texttt{String} \textbar{} \texttt{null} \textbar{} The port
number of the proxy server. It sets the value of the
\href{https://maven.apache.org/settings.html\#Proxies}{\texttt{proxies.proxy.port}}
element in the generated \texttt{settings.xml} file. \texttt{proxyUser}
\textbar{} \texttt{String} \textbar{} \texttt{null} \textbar{} The user
name to use to access a protected proxy server. It sets the value of the
\href{https://maven.apache.org/settings.html\#Proxies}{\texttt{proxies.proxy.username}}
element in the generated \texttt{settings.xml} file.
\texttt{repositoryUrl} \textbar{} \texttt{String} \textbar{}
\texttt{null} \textbar{} The URL of the repository
\href{https://maven.apache.org/guides/mini/guide-mirror-settings.html\#Using_A_Single_Repository}{mirror}.
It sets the value of the
\href{https://maven.apache.org/settings.html\#Mirrors}{\texttt{mirrors.mirror.url}}
element in the generated \texttt{settings.xml} file.

The properties of type \texttt{File} support any type that can be
resolved by
\href{https://docs.gradle.org/current/dsl/org.gradle.api.Project.html\#org.gradle.api.Project:file(java.lang.Object)}{\texttt{project.file}}.
Moreover, it is possible to use Closures and Callables as values for the
\texttt{String} properties, to defer evaluation until task execution.

\section{Additional Configuration}\label{additional-configuration-8}

There are additional configurations that can help you use the Maven
Plugin Builder.

\section{Maven Embedder Dependency}\label{maven-embedder-dependency}

By default, the plugin creates a configuration called
\texttt{mavenEmbedder} and adds a dependency to the 3.3.9 version of the
Maven Embedder. It is possible to override this setting and use a
specific version of the tool by manually adding a dependency to the
\texttt{mavenEmbedder} configuration:

\begin{verbatim}
dependencies {
    mavenEmbedder group: "org.apache.maven", name: "maven-embedder", version: "3.3.9"
    mavenEmbedder group: "org.apache.maven.wagon", name: "wagon-http", version: "2.10"
    mavenEmbedder group: "org.eclipse.aether", name: "aether-connector-basic", version: "1.0.2.v20150114"
    mavenEmbedder group: "org.eclipse.aether", name: "aether-transport-wagon", version: "1.0.2.v20150114"
    mavenEmbedder group: "org.slf4j", name: "slf4j-simple", version: "1.7.5"
}
\end{verbatim}

\section{System Properties}\label{system-properties-2}

It is possible to set the default value of the \texttt{mavenDebug}
property for a \texttt{BuildPluginDescriptorTask} task via system
property:

\begin{itemize}
\tightlist
\item
  \texttt{-D\$\{task.name\}.maven.debug=true}
\end{itemize}

For example, run the following Bash command to invoke the Maven Embedder
in debug mode to attach a remote debugger:

\begin{verbatim}
./gradlew buildPluginDescriptor -DbuildPluginDescriptor.maven.debug=true
\end{verbatim}

\chapter{Node Gradle Plugin}\label{node-gradle-plugin}

The Node Gradle plugin lets you run \href{https://nodejs.org/}{Node.js}
and \href{https://www.npmjs.com/}{NPM} as part of your build.

The plugin has been successfully tested with Gradle 4.10.2.

\section{Usage}\label{usage-16}

To use the plugin, include it in your build script:

\begin{verbatim}
buildscript {
    dependencies {
        classpath group: "com.liferay", name: "com.liferay.gradle.plugins.node", version: "4.6.18"
    }

    repositories {
        maven {
            url "https://repository-cdn.liferay.com/nexus/content/groups/public"
        }
    }
}

apply plugin: "com.liferay.node"
\end{verbatim}

\section{Project Extension}\label{project-extension-6}

The Node Gradle plugin exposes the following properties through the
extension named \texttt{node}:

Property Name \textbar{} Type \textbar{} Default Value \textbar{}
Description \texttt{download} \textbar{} \texttt{boolean} \textbar{}
\texttt{true} \textbar{} Whether to download and use a local and
isolated Node.js distribution instead of the one installed in the
system. \texttt{global} \textbar{} \texttt{boolean} \textbar{}
\texttt{false} \textbar{} Whether to use a single Node.js installation
for the whole multi-project build. This reduces the time required to
unpack the Node.js distribution and the time required to download NPM
packages thanks to a shared packages cache. If \texttt{download} is
\texttt{false}, this property has no effect. \texttt{nodeDir} \textbar{}
\texttt{File} \textbar{}

\textbf{If \texttt{global} is \texttt{true}:}
\texttt{"\$\{rootProject.buildDir\}/node"}

\textbf{Otherwise:} \texttt{"\$\{project.buildDir\}/node"}

\textbar{} The directory where the Node.js distribution is unpacked. If
\texttt{download} is \texttt{false}, this property has no effect.
\texttt{nodeUrl} \textbar{} \texttt{String} \textbar{}
\texttt{"http://nodejs.org/dist/v\$\{node.nodeVersion\}/node-v\$\{node.nodeVersion\}-\$\{platform\}-x\$\{bitMode\}.\$\{extension\}"}
\textbar{} The URL of the Node.js distribution to download. If
\texttt{download} is \texttt{false}, this property has no effect.
\texttt{nodeVersion} \textbar{} \texttt{String} \textbar{}
\texttt{"5.5.0"} \textbar{} The version of the Node.js distribution to
use. If \texttt{download} is \texttt{false}, this property has no
effect. \texttt{npmArgs} \textbar{}
\texttt{List\textless{}String\textgreater{}} \textbar{} \texttt{{[}{]}}
\textbar{} The arguments added automatically to every task of type
\hyperref[executenpmtask]{\texttt{ExecuteNpmTask}}. \texttt{npmUrl}
\textbar{} \texttt{String} \textbar{}
\texttt{"https://registry.npmjs.org/npm/-/npm-\$\{node.npmVersion\}.tgz"}
\textbar{} The URL of the NPM version to download. If \texttt{download}
is \texttt{false}, this property has no effect. \texttt{npmVersion}
\textbar{} \texttt{String} \textbar{} \texttt{null} \textbar{} The
version of NPM to use. If \texttt{null}, the version of NPM embedded
inside the Node.js distribution is used. If \texttt{download} is
\texttt{false}, this property has no effect.

It is possible to override the default value of the \texttt{download}
property by setting the \texttt{nodeDownload} project property. For
example, this can be done via command line argument:

\begin{verbatim}
./gradlew -PnodeDownload=false npmInstall
\end{verbatim}

The same extension exposes the following methods:

Method \textbar{} Description
\texttt{NodeExtension\ npmArgs(Iterable\textless{}?\textgreater{}\ npmArgs)}
\textbar{} Adds arguments to automatically add to every task of type
\hyperref[executenpmtask]{\texttt{ExecuteNpmTask}}.
\texttt{NodeExtension\ npmArgs(Object...\ npmArgs)} \textbar{} Adds
arguments to automatically add to every task of type
\hyperref[executenpmtask]{\texttt{ExecuteNpmTask}}.

The properties of type \texttt{File} support any type that can be
resolved by
\href{https://docs.gradle.org/current/dsl/org.gradle.api.Project.html\#org.gradle.api.Project:file(java.css.Object)}{\texttt{project.file}}.
Moreover, it is possible to use Closures and Callables as values for
\texttt{String}, to defer evaluation until execution.

Please note that setting the \texttt{global} property of the
\texttt{node} extension via the command line is not supported. It can
only be set via Gradle script, which can be done by adding the following
code to the \texttt{build.gradle} file in the root of a project (e.g.,
Liferay Workspace):

\begin{verbatim}
allprojects {
    plugins.withId("com.liferay.node") {
        node.global = true
    }
}
\end{verbatim}

\section{Tasks}\label{tasks-15}

The plugin adds a series of tasks to your project:

Name \textbar{} Depends On \textbar{} Type \textbar{} Description
\texttt{cleanNPM} \textbar{} - \textbar{}
\href{https://docs.gradle.org/current/dsl/org.gradle.api.tasks.Delete.html}{\texttt{Delete}}
\textbar{} Deletes the \texttt{node\_modules} directory, the
\texttt{npm-shrinkwrap.json} file and the \texttt{package-lock.json}
files from the project, if present. \texttt{downloadNode} \textbar{} -
\textbar{} \hyperref[downloadnodetask]{\texttt{DownloadNodeTask}}
\textbar{} Downloads and unpacks the local Node.js distribution for the
project. If \texttt{node.download} is \texttt{false}, this task is
disabled. \texttt{npmInstall} \textbar{} \texttt{downloadNode}
\textbar{} \hyperref[npminstalltask]{\texttt{NpmInstallTask}} \textbar{}
Runs \texttt{npm\ install} to install the dependencies declared in the
project's \texttt{package.json} file, if present. By default, the task
is \hyperref[npminstallretries]{configured} to run \texttt{npm\ install}
two more times if it fails.
\hyperref[npmrunscript-task]{\texttt{npmRun\$\{script\}}} \textbar{}
\texttt{npmInstall} \textbar{}
\hyperref[executenpmtask]{\texttt{ExecuteNpmTask}} \textbar{} Runs the
\texttt{\$\{script\}} NPM script. \texttt{npmPackageLock} \textbar{}
\texttt{cleanNPM}, \texttt{npmInstall} \textbar{}
\href{https://docs.gradle.org/current/javadoc/org/gradle/api/DefaultTask.html}{\texttt{DefaultTask}}
\textbar{} Deletes the NPM files and runs \texttt{npm\ install} to
install the dependencies declared in the project's \texttt{package.json}
file, if present. \texttt{npmShrinkwrap} \textbar{} \texttt{cleanNPM},
\texttt{npmInstall} \textbar{}
\hyperref[npmshrinkwraptask]{\texttt{NpmShrinkwrapTask}} \textbar{}
Locks down the versions of a package's dependencies in order to control
which dependency versions are used.

\section{DownloadNodeTask}\label{downloadnodetask}

The purpose of this task is to download and unpack a Node.js
distribution.

\subsection{Task Properties}\label{task-properties-16}

Property Name \textbar{} Type \textbar{} Default Value \textbar{}
Description \texttt{nodeDir} \textbar{} \texttt{File} \textbar{}
\texttt{null} \textbar{} The directory where the Node.js distribution is
unpacked. \texttt{nodeExeUrl} \textbar{} \texttt{String} \textbar{}
\texttt{null} \textbar{} The URL of \texttt{node.exe} to download when
on Windows. \texttt{nodeUrl} \textbar{} \texttt{String} \textbar{}
\texttt{null} \textbar{} The URL of the Node.js distribution to
download. \texttt{npmUrl} \textbar{} \texttt{String} \textbar{}
\texttt{null} \textbar{} The URL of the NPM version to download.

The properties of type \texttt{File} support any type that can be
resolved by
\href{https://docs.gradle.org/current/dsl/org.gradle.api.Project.html\#org.gradle.api.Project:file(java.css.Object)}{\texttt{project.file}}.
Moreover, it is possible to use Closures and Callables as values for the
\texttt{String} properties, to defer evaluation until task execution.

\section{ExecuteNodeTask}\label{executenodetask}

This is the base task to run Node.js in a Gradle build. All tasks of
type \texttt{ExecuteNodeTask} automatically depend on
\hyperref[downloadnode]{\texttt{downloadNode}}.

\subsection{Task Properties}\label{task-properties-17}

Property Name \textbar{} Type \textbar{} Default Value \textbar{}
Description \texttt{args} \textbar{}
\texttt{List\textless{}Object\textgreater{}} \textbar{} \texttt{{[}{]}}
\textbar{} The arguments for the Node.js invocation. \texttt{command}
\textbar{} \texttt{String} \textbar{} \texttt{"node"} \textbar{} The
file name of the executable to invoke. \texttt{environment} \textbar{}
\texttt{Map\textless{}Object,\ Object\textgreater{}} \textbar{}
\texttt{{[}{]}} \textbar{} The environment variables for the Node.js
invocation. \texttt{inheritProxy} \textbar{} \texttt{boolean} \textbar{}
\texttt{true} \textbar{} Whether to set the \texttt{http\_proxy},
\texttt{https\_proxy}, and \texttt{no\_proxy} environment variables in
the Node.js invocation based on the values of the system properties
\texttt{https.proxyHost}, \texttt{https.proxyPort},
\texttt{https.proxyUser}, \texttt{https.proxyPassword},
\texttt{https.nonProxyHosts}, \texttt{https.proxyHost},
\texttt{https.proxyPort}, \texttt{https.proxyUser},
\texttt{https.proxyPassword}, and \texttt{https.nonProxyHosts}. If these
environment variables are already set, their values will not be
overwritten. \texttt{nodeDir} \textbar{} \texttt{File} \textbar{}

\textbf{If \hyperref[download]{\texttt{node.download}} is
\texttt{true}:} \hyperref[nodedir]{\texttt{node.nodeDir}}

\textbf{Otherwise:} \texttt{null}

\textbar{} The directory that contains the executable to invoke. If
\texttt{null}, the executable must be available in the system
\texttt{PATH}. \texttt{npmInstallRetries} \textbar{} \texttt{int}
\textbar{} \texttt{0} \textbar{} The number of times the
\texttt{node\_modules} is deleted, the NPM cached data is verified
(\texttt{npm\ cache\ verify}), and \texttt{npm\ install} is retried in
case the Node.js invocation defined by this task fails. This can help
solving corrupted \texttt{node\_modules} directories by re-downloading
the project's dependencies. \texttt{workingDir} \textbar{} \texttt{File}
\textbar{} \texttt{project.projectDir} \textbar{} The working directory
to use in the Node.js invocation.

The properties of type \texttt{File} support any type that can be
resolved by
\href{https://docs.gradle.org/current/dsl/org.gradle.api.Project.html\#org.gradle.api.Project:file(java.css.Object)}{\texttt{project.file}}.
Moreover, it is possible to use Closures and Callables as values for the
\texttt{String} properties to defer evaluation until task execution.

\subsection{Task Methods}\label{task-methods-8}

Method \textbar{} Description
\texttt{ExecuteNodeTask\ args(Iterable\textless{}?\textgreater{}\ args)}
\textbar{} Adds arguments for the Node.js invocation.
\texttt{ExecuteNodeTask\ args(Object...\ args)} \textbar{} Adds
arguments for the Node.js invocation.
\texttt{ExecuteNodeTask\ environment(Map\textless{}?,\ ?\textgreater{}\ environment)}
\textbar{} Adds environment variables for the Node.js invocation.
\texttt{ExecuteNodeTask\ environment(Object\ key,\ Object\ value)}
\textbar{} Adds an environment variable for the Node.js invocation.

\section{ExecuteNodeScriptTask}\label{executenodescripttask}

The purpose of this task is to execute a Node.js script. Tasks of type
\texttt{ExecuteNodeScriptTask} extend
\hyperref[executenodetask]{\texttt{ExecuteNodeTask}}.

\subsection{Task Properties}\label{task-properties-18}

Property Name \textbar{} Type \textbar{} Default Value \textbar{}
Description \texttt{scriptFile} \textbar{} \texttt{File} \textbar{}
\texttt{null} \textbar{} The Node.js script to execute.

The properties of type \texttt{File} support any type that can be
resolved by
\href{https://docs.gradle.org/current/dsl/org.gradle.api.Project.html\#org.gradle.api.Project:file(java.css.Object)}{\texttt{project.file}}.

\section{ExecuteNpmTask}\label{executenpmtask}

The purpose of this task is to execute an NPM command. Tasks of type
\texttt{ExecuteNpmTask} extend
\hyperref[executenodescripttask]{\texttt{ExecuteNodeScriptTask}} with
the following properties set by default:

Property Name \textbar{} Default Value \texttt{command} \textbar{}

\textbf{If \texttt{nodeDir} is \texttt{null}:} \texttt{"npm"}

\textbf{Otherwise:} \texttt{"node"}

\texttt{scriptFile} \textbar{}

\textbf{If \texttt{nodeDir} is \texttt{null}:} \texttt{null}

\textbf{Otherwise:}
\texttt{"\$\{nodeDir\}/lib/node\_modules/npm/bin/npm-cli.js"} or
\texttt{"\$\{nodeDir\}/node\_modules/npm/bin/npm-cli.js"} on Windows.

\subsection{Task Properties}\label{task-properties-19}

Property Name \textbar{} Type \textbar{} Default Value \textbar{}
Description \texttt{cacheConcurrent} \textbar{} \texttt{boolean}
\textbar{}

\textbf{If \texttt{node.npmVersion} is greater than or equal to
\texttt{5.0.0}, or \texttt{node.nodeVersion} is greater than or equal to
\texttt{8.0.0}:} \texttt{true}

\textbf{Otherwise:} \texttt{false}

\textbar{} Whether to run this task concurrently, in case the version of
NPM in use supports multiple concurrent accesses to the same cache
directory. \texttt{cacheDir} \textbar{} \texttt{File} \textbar{}

\textbf{If \texttt{nodeDir} is \texttt{null}, or
\texttt{node.npmVersion} is greater than or equal to \texttt{5.0.0}, or
\texttt{node.nodeVersion} is greater than or equal to \texttt{8.0.0}:}
\texttt{null}

\textbf{Otherwise:} \texttt{"\$\{nodeDir\}/.cache"}

\textbar{} The location of NPM's cache directory. It sets the
\href{https://docs.npmjs.com/misc/config\#cache}{\texttt{-\/-cache}}
argument. Leave the property \texttt{null} to keep the default value.
\texttt{logLevel} \textbar{} \texttt{String} \textbar{} Value to mirror
the log level set in the task's
\href{https://docs.gradle.org/current/dsl/org.gradle.api.Task.html\#org.gradle.api.Task:logger}{\texttt{logger}}
object. \textbar{} The NPM log level. It sets the
\href{https://docs.npmjs.com/misc/config\#loglevel}{--loglevel}
argument. \texttt{production} \textbar{} \texttt{boolean} \textbar{}
\texttt{false} \textbar{} Whether to run in production mode during the
NPM invocation. It sets the
\href{https://docs.npmjs.com/misc/config\#production}{\texttt{-\/-production}}
argument. \texttt{progress} \textbar{} \texttt{boolean} \textbar{}
\texttt{true} \textbar{} Whether to show a progress bar during the NPM
invocation. It sets the
\href{https://docs.npmjs.com/misc/config\#progress}{\texttt{-\/-progress}}
argument. \texttt{registry} \textbar{} \texttt{String} \textbar{}
\texttt{null} \textbar{} The base URL of the NPM package registry. It
sets the
\href{https://docs.npmjs.com/misc/config\#registry}{\texttt{-\/-registry}}
argument. Leave the property \texttt{null} or empty to keep the default
value.

The properties of type \texttt{File} support any type that can be
resolved by
\href{https://docs.gradle.org/current/dsl/org.gradle.api.Project.html\#org.gradle.api.Project:file(java.css.Object)}{\texttt{project.file}}.
Moreover, it is possible to use Closures and Callables as values for the
\texttt{String} properties, to defer evaluation until task execution.

\section{DownloadNodeModuleTask}\label{downloadnodemoduletask}

The purpose of this task is to download a Node.js package. The packages
are downloaded in the \texttt{\$\{workingDir\}/node\_modules} directory,
which is equal, by default, to the \texttt{node\_modules} directory of
the project. Tasks of type \texttt{DownloadNodeModuleTask} extend
\hyperref[executenpmtask]{\texttt{ExecuteNpmTask}} in order to execute
the command
\href{https://docs.npmjs.com/cli/install}{\texttt{npm\ install\ \$\{moduleName\}@\$\{moduleVersion\}}}.

\texttt{DownloadNodeModuleTask} instances are automatically disabled if
the project's \texttt{package.json} file already lists a module with the
same name in its \texttt{dependencies} or \texttt{devDependencies}
object.

\subsection{Task Properties}\label{task-properties-20}

Property Name \textbar{} Type \textbar{} Default Value \textbar{}
Description \texttt{moduleName} \textbar{} \texttt{String} \textbar{}
\texttt{null} \textbar{} The name of the Node.js package to download.
\texttt{moduleVersion} \textbar{} \texttt{String} \textbar{}
\texttt{null} \textbar{} The version of the Node.js package to download.

It is possible to use Closures and Callables as values for the
\texttt{String} properties, to defer evaluation until task execution.

\section{NpmInstallTask}\label{npminstalltask}

Purpose of these tasks is to install the dependencies declared in a
\texttt{package.json} file. Tasks of type \texttt{NpmInstallTask} extend
\hyperref[executenpmtask]{\texttt{ExecuteNpmTask}} in order to run the
command
\href{https://docs.npmjs.com/cli/install}{\texttt{npm\ install}}.

\texttt{NpmInstallTask} instances are automatically disabled if the
\texttt{package.json} file does not declare any dependency in the
\texttt{dependency} or \texttt{devDependencies} object.

\subsection{Task Properties}\label{task-properties-21}

Property Name \textbar{} Type \textbar{} Default Value \textbar{}
Description \texttt{nodeModulesCacheDir} \textbar{} \texttt{File}
\textbar{} \texttt{null} \textbar{}

The directory where \texttt{node\_modules} directories are cached. By
setting this property, it is possible to cache the
\texttt{node\_modules} directory of a project and avoid unnecessary
invocations of \texttt{npm\ install}, useful especially in Continuous
Integration environments.

The \texttt{node\_modules} directory is cached based on the content of
the project's \texttt{package-lock.json} (or
\texttt{npm-shrinkwrap.json}, or \texttt{package.json} if absent).
Therefore, if \texttt{NpmInstallTask} tasks in multiple projects are
configured with the same \texttt{nodeModulesCacheDir}, and their
\texttt{package-lock.json}, \texttt{npm-shrinkwrap.json} or
\texttt{package.json} declare the same dependencies, their
\texttt{node\_modules} caches will be shared.

This feature is not available if the
\href{https://github.com/liferay/liferay-portal/tree/master/modules/sdk/gradle-plugins-cache}{\texttt{com.liferay.cache}}
plugin is applied.

\texttt{nodeModulesCacheNativeSync} \textbar{} \texttt{boolean}
\textbar{} \texttt{true} \textbar{} Whether to use \texttt{rsync} (on
Linux/macOS) or \texttt{robocopy} (on Windows) to cache and restore the
\texttt{node\_modules} directory. If \texttt{nodeModulesCacheDir} is not
set, this property has no effect. \texttt{nodeModulesDigestFile}
\textbar{} \texttt{File} \textbar{} \texttt{null} \textbar{}

If this property is set, the content of the project's
\texttt{package-lock.json} (or \texttt{npm-shrinkwrap.json}, or
\texttt{package.json} if absent) is checked with the digest from the
\texttt{node\_modules} directory. If the digests match, do nothing. If
the digests don't match, the \texttt{node\_modules} directory is deleted
before running \texttt{npm\ install}.

This feature is not available if the \texttt{com.liferay.cache} plugin
is applied or if the property \texttt{nodeModulesCacheDir} is set.

\texttt{removeShrinkwrappedUrls} \textbar{} \texttt{boolean} \textbar{}
\texttt{true} if the \hyperref[registry]{registry} property has a value,
\texttt{false} otherwise. \textbar{} Whether to temporarily remove all
the hard-coded URLs in the \texttt{from} and \texttt{resolved} fields of
the \texttt{npm-shinkwrap.json} file before invoking
\texttt{npm\ install}. This way, it is possible to force NPM to download
all dependencies from a custom registry declared in the
\hyperref[registry]{\texttt{registry}} property. \texttt{useNpmCI}
\textbar{} \texttt{boolean} \textbar{} \texttt{false} \textbar{} Whether
to run \texttt{npm\ ci} instead of \texttt{npm\ install}. If the
\texttt{package-lock.json} file does not exist, this property has no
effect.

The properties of type \texttt{File} support any type that can be
resolved by
\href{https://docs.gradle.org/current/dsl/org.gradle.api.Project.html\#org.gradle.api.Project:file(java.css.Object)}{\texttt{project.file}}.

\section{NpmShrinkwrapTask}\label{npmshrinkwraptask}

The purpose of this task is to lock down the versions of a package's
dependencies so that you can control exactly which dependency versions
are used when your package is installed. Tasks of type
\texttt{NpmShrinkwrapTask} extend
\hyperref[executenpmtask]{\texttt{ExecuteNpmTask}} to execute the
command
\href{https://docs.npmjs.com/cli/shrinkwrap}{\texttt{npm\ shrinkwrap}}.

The generated \texttt{npm-shrinkwrap.json} file is automatically sorted
and formatted, so it's easier to see the changes with the previous
version.

\texttt{NpmShrinkwrapTask} instances are automatically disabled if the
\texttt{package.json} file does not exist.

\subsection{Task Properties}\label{task-properties-22}

Property Name \textbar{} Type \textbar{} Default Value \textbar{}
Description \texttt{excludedDependencies} \textbar{}
\texttt{List\textless{}String\textgreater{}} \textbar{} \texttt{{[}{]}}
\textbar{} The package names to exclude from the generated
\texttt{npm-shrinkwrap.json} file. \texttt{includeDevDependencies}
\textbar{} \texttt{boolean} \textbar{} \texttt{true} \textbar{} Whether
to include the package's \texttt{devDependencies}. It sets the
\href{https://docs.npmjs.com/cli/shrinkwrap\#other-notes}{\texttt{-\/-dev}}
argument.

It is possible to use Closures and Callables as values for the
\texttt{String} properties to defer evaluation until task execution.

\subsection{Task Methods}\label{task-methods-9}

Method \textbar{} Description
\texttt{NpmShrinkwrapTask\ excludeDependencies(Iterable\textless{}?\textgreater{}\ excludedDependencies)}
\textbar{} Adds package names to exclude from the generated
\texttt{npm-shrinkwrap.json} file.
\texttt{NpmShrinkwrapTask\ excludeDependencies(Object...\ excludedDependencies)}
\textbar{} Adds package names to exclude from the generated
\texttt{npm-shrinkwrap.json} file.

\section{PublishNodeModuleTask}\label{publishnodemoduletask}

The purpose of this task is to publish a package to the
\href{https://www.npmjs.com/}{NPM registry}. Tasks of type
\texttt{PublishNodeModuleTask} extend
\hyperref[executenpmtask]{\texttt{ExecuteNpmTask}} in order to execute
the command
\href{https://docs.npmjs.com/cli/publish}{\texttt{npm\ publish}}.

These tasks generate a new temporary \texttt{package.json} file in the
directory assigned to the \hyperref[workingdir]{\texttt{workingDir}}
property; then the \texttt{npm\ publish} command is executed. If the
\texttt{package.json} file in that location does not exist, the one in
the root of the project directory (if found) is copied; otherwise, a new
file is created.

The \texttt{package.json} is then processed by adding the values
provided by the task properties, if not already present in the file
itself. It is still possible to override one or more fields of the
\texttt{package.json} file with the values provided by the task
properties by adding one or more keys (e.g., \texttt{"version"}) to the
\texttt{overriddenPackageJsonKeys} property.

\subsection{Task Properties}\label{task-properties-23}

Property Name \textbar{} Type \textbar{} Default Value \textbar{}
Description \texttt{moduleAuthor} \textbar{} \texttt{String} \textbar{}
\texttt{null} \textbar{} The value of the
\href{https://docs.npmjs.com/files/package.json\#people-fields-author-contributors}{\texttt{author}}
field in the generated \texttt{package.json} file.
\texttt{moduleBugsUrl} \textbar{} \texttt{String} \textbar{}
\texttt{null} \textbar{} The value of the
\href{https://docs.npmjs.com/files/package.json\#bugs}{\texttt{bugs.url}}
field in the generated \texttt{package.json} file.
\texttt{moduleDescription} \textbar{} \texttt{String} \textbar{}
\texttt{project.description} \textbar{} The value of the
\href{https://docs.npmjs.com/files/package.json\#description-1}{\texttt{description}}
field in the generated \texttt{package.json} file.
\texttt{moduleKeywords} \textbar{}
\texttt{List\textless{}String\textgreater{}} \textbar{} \texttt{{[}{]}}
\textbar{} The value of the
\href{https://docs.npmjs.com/files/package.json\#keywords}{\texttt{keywords}}
field in the generated \texttt{package.json} file.
\texttt{moduleLicense} \textbar{} \texttt{String} \textbar{}
\texttt{null} \textbar{} The value of the
\href{https://docs.npmjs.com/files/package.json\#license}{\texttt{license}}
field in the generated \texttt{package.json} file. \texttt{moduleMain}
\textbar{} \texttt{String} \textbar{} \texttt{null} \textbar{} The value
of the
\href{https://docs.npmjs.com/files/package.json\#main}{\texttt{main}}
field in the generated \texttt{package.json} file. \texttt{moduleName}
\textbar{} \texttt{String} \textbar{} Name based on
\href{https://github.com/gradle/gradle/blob/master/subprojects/osgi/src/main/java/org/gradle/api/internal/plugins/osgi/OsgiHelper.java}{\texttt{osgiHelper.bundleSymbolicName}}:
for example, if \texttt{osgiHelper.bundleSymbolicName} is
\texttt{"com.liferay.gradle.plugins.node"}, the default value for the
\texttt{moduleName} property is \texttt{"liferay-gradle-plugins-node"}.
\textbar{} The value of the
\href{https://docs.npmjs.com/files/package.json\#name}{\texttt{name}}
field in the generated \texttt{package.json} file.
\texttt{moduleRepository} \textbar{} \texttt{String} \textbar{}
\texttt{null} \textbar{} The value of the
\href{https://docs.npmjs.com/files/package.json\#repository}{\texttt{repository}}
field in the generated \texttt{package.json} file.
\texttt{moduleVersion} \textbar{} \texttt{String} \textbar{}
\texttt{project.version} \textbar{} The value of the
\href{https://docs.npmjs.com/files/package.json\#version}{\texttt{version}}
field in the generated \texttt{package.json} file.
\texttt{npmEmailAddress} \textbar{} \texttt{String} \textbar{}
\texttt{null} \textbar{} The email address of the npmjs.com user that
publishes the package. \texttt{npmPassword} \textbar{} \texttt{String}
\textbar{} \texttt{null} \textbar{} The password of the npmjs.com user
that publishes the package. \texttt{npmUserName} \textbar{}
\texttt{String} \textbar{} \texttt{null} \textbar{} The name of the
npmjs.com user that publishes the package.
\texttt{overriddenPackageJsonKeys} \textbar{}
\texttt{Set\textless{}String\textgreater{}} \textbar{} \texttt{{[}{]}}
\textbar{} The field values to override in the generated
\texttt{package.json} file.

\subsection{Task Methods}\label{task-methods-10}

\begin{longtable}[]{@{}
  >{\raggedright\arraybackslash}p{(\columnwidth - 2\tabcolsep) * \real{0.3529}}
  >{\raggedright\arraybackslash}p{(\columnwidth - 2\tabcolsep) * \real{0.6471}}@{}}
\toprule\noalign{}
\begin{minipage}[b]{\linewidth}\raggedright
Method
\end{minipage} & \begin{minipage}[b]{\linewidth}\raggedright
Description
\end{minipage} \\
\midrule\noalign{}
\endhead
\bottomrule\noalign{}
\endlastfoot
\texttt{PublishNodeModuleTask\ overriddenPackageJsonKeys(Iterable\textless{}String\textgreater{}\ overriddenPackageJsonKeys)}
& Adds field values to override in the generated \texttt{package.json}
file. \\
\texttt{PublishNodeModuleTask\ overriddenPackageJsonKeys(String...\ overriddenPackageJsonKeys)}
& Adds field values to override in the generated \texttt{package.json}
file. \\
\end{longtable}

\section{npmRun\$\{script\} Task}\label{npmrunscript-task}

For each \href{https://docs.npmjs.com/misc/scripts}{script} declared in
the \texttt{package.json} file of the project, one task
\texttt{npmRun\$\{script\}} of type
\hyperref[executenpmtask]{\texttt{ExecuteNpmTask}} is added. Each of
these tasks is automatically configured with sensible defaults:

Property Name \textbar{} Default Value \texttt{args} \textbar{}
\texttt{{[}"run-script",\ "\$\{script\}"{]}}

If the
\href{https://docs.gradle.org/current/userguide/java_plugin.html}{\texttt{java}}
plugin is applied and the \texttt{package.json} file declares a script
named \texttt{"build"}, the script is executed before the
\texttt{classes} task but after the
\href{https://docs.gradle.org/4.0/userguide/java_plugin.html\#sec:java_resources}{\texttt{processResources}}
task.

If the
\href{https://docs.gradle.org/current/javadoc/org/gradle/language/base/plugins/LifecycleBasePlugin.html}{\texttt{lifecycle-base}}
plugin is applied and the \texttt{package.json} file declares a script
named \texttt{test}, the script is executed when running the
\texttt{check} task.

\chapter{REST Builder Gradle Plugin}\label{rest-builder-gradle-plugin}

The REST Builder Gradle plugin lets you generate a REST layer defined in
the REST Builder \texttt{rest-config.yaml} and
\texttt{rest-openapi.yaml} files.

The plugin has been successfully tested with Gradle 4.10.2.

\section{Usage}\label{usage-17}

To use the plugin, include it in your build script:

\begin{verbatim}
buildscript {
    dependencies {
        classpath group: "com.liferay", name: "com.liferay.gradle.plugins.rest.builder", version: "1.0.21"
    }

    repositories {
        maven {
            url "https://repository-cdn.liferay.com/nexus/content/groups/public"
        }
    }
}

apply plugin: "com.liferay.portal.tools.rest.builder"
\end{verbatim}

The REST Builder plugin automatically applies the
\href{https://docs.gradle.org/current/userguide/java_plugin.html}{\texttt{java}}
plugin.

Since the plugin automatically resolves the
\href{https://github.com/liferay/liferay-portal/tree/master/modules/util/portal-tools-rest-builder}{Liferay
REST Builder} library as a dependency, you have to configure a
repository that hosts the library and its transitive dependencies. The
Liferay CDN repository hosts them all:

\begin{verbatim}
repositories {
    maven {
        url "https://repository-cdn.liferay.com/nexus/content/groups/public"
    }
}
\end{verbatim}

\section{Tasks}\label{tasks-16}

The plugin adds one task to your project:

Name \textbar{} Depends On \textbar{} Type \textbar{} Description
\texttt{buildREST} \textbar{} - \textbar{}
\hyperref[buildresttask]{\texttt{BuildRESTTask}} \textbar{} Runs the
Liferay REST Builder.

\section{BuildRESTTask}\label{buildresttask}

Tasks of type \texttt{BuildRESTTask} extend
\href{https://docs.gradle.org/current/dsl/org.gradle.api.tasks.JavaExec.html}{\texttt{JavaExec}},
so all its properties and methods, such as
\href{https://docs.gradle.org/current/dsl/org.gradle.api.tasks.JavaExec.html\#org.gradle.api.tasks.JavaExec:args(java.lang.Iterable)}{\texttt{args}}
and
\href{https://docs.gradle.org/current/dsl/org.gradle.api.tasks.JavaExec.html\#org.gradle.api.tasks.JavaExec:maxHeapSize}{\texttt{maxHeapSize}}
are available. They also have the following properties set by default:

Property Name \textbar{} Default Value
\href{https://docs.gradle.org/current/dsl/org.gradle.api.tasks.JavaExec.html\#org.gradle.api.tasks.JavaExec:args}{\texttt{args}}
\textbar{} REST Builder command line arguments
\href{https://docs.gradle.org/current/dsl/org.gradle.api.tasks.JavaExec.html\#org.gradle.api.tasks.JavaExec:classpath}{\texttt{classpath}}
\textbar{}
\hyperref[liferay-rest-builder-dependency]{\texttt{project.configurations.restBuilder}}
\href{https://docs.gradle.org/current/dsl/org.gradle.api.tasks.JavaExec.html\#org.gradle.api.tasks.JavaExec:main}{\texttt{main}}
\textbar{} \texttt{"com.liferay.portal.tools.rest.builder.RESTBuilder"}
\href{https://docs.gradle.org/current/dsl/org.gradle.api.tasks.JavaExec.html\#org.gradle.api.tasks.JavaExec:systemProperties}{\texttt{systemProperties}}
\textbar{} \texttt{{[}{]}}

\subsection{Task Properties}\label{task-properties-24}

Property Name \textbar{} Type \textbar{} Default Value \textbar{}
Description \texttt{copyrightFile} \textbar{} \texttt{File} \textbar{}
\texttt{null} \textbar{} The file that contains the copyright header.
\texttt{restConfigDir} \textbar{} \texttt{File}
\textbar{}\texttt{\$\{project.projectDir\}} \textbar{} The directory
that contains the \texttt{rest-config.yaml} and
\texttt{rest-openapi.yaml} files.

In the typical scenario, the \texttt{rest-config.yaml} and
\texttt{rest-openapi.yaml} files are contained in the project directory
of \texttt{my-rest-app-impl}. In the \texttt{build.gradle} of the same
module, apply the \texttt{com.liferay.rest.builder} plugin.

The properties of type \texttt{File} supports any type that can be
resolved by
\href{https://docs.gradle.org/current/dsl/org.gradle.api.Project.html\#org.gradle.api.Project:file(java.lang.Object)}{\texttt{project.file}}.
Moreover, it is possible to use Closures and Callables as values for the
\texttt{String} properties, to defer evaluation until task execution.

\section{Additional Configuration}\label{additional-configuration-9}

There are additional configurations added to use REST Builder.

\section{Liferay REST Builder
Dependency}\label{liferay-rest-builder-dependency}

By default, the plugin creates a configuration called
\texttt{restBuilder} and adds a dependency to the latest released
version of Liferay REST Builder.

\begin{verbatim}
dependencies {
    restBuilder group: "com.liferay", name: "com.liferay.portal.tools.rest.builder", version: "1.0.22"
}
\end{verbatim}

\chapter{Service Builder Gradle
Plugin}\label{service-builder-gradle-plugin}

The Service Builder Gradle plugin lets you generate a service layer
defined in a
\href{/docs/7-2/appdev/-/knowledge_base/a/service-builder}{Service
Builder} \texttt{service.xml} file.

The plugin has been successfully tested with Gradle 4.10.2.

\section{Usage}\label{usage-18}

To use the plugin, include it in your build script:

\begin{verbatim}
buildscript {
    dependencies {
        classpath group: "com.liferay", name: "com.liferay.gradle.plugins.service.builder", version: "2.2.46"
    }

    repositories {
        maven {
            url "https://repository-cdn.liferay.com/nexus/content/groups/public"
        }
    }
}

apply plugin: "com.liferay.portal.tools.service.builder"
\end{verbatim}

The Service Builder plugin automatically applies the
\href{https://docs.gradle.org/current/userguide/java_plugin.html}{\texttt{java}}
plugin.

Since the plugin automatically resolves the
\href{https://github.com/liferay/liferay-portal/tree/master/modules/util/portal-tools-service-builder}{Liferay
Service Builder} library as a dependency, you have to configure a
repository that hosts the library and its transitive dependencies. The
Liferay CDN repository hosts them all:

\begin{verbatim}
repositories {
    maven {
        url "https://repository-cdn.liferay.com/nexus/content/groups/public"
    }
}
\end{verbatim}

\section{Tasks}\label{tasks-17}

The plugin adds one task to your project:

Name \textbar{} Depends On \textbar{} Type \textbar{} Description
\texttt{buildService} \textbar{} - \textbar{}
\hyperref[buildservicetask]{\texttt{BuildServiceTask}} \textbar{} Runs
the Liferay Service Builder.

The \texttt{buildService} task is automatically configured with sensible
defaults, depending on whether the
\href{https://docs.gradle.org/current/userguide/war_plugin.html}{\texttt{war}}
plugin is applied, or whether the
\hyperref[osgimodule]{\texttt{osgiModule}} property is \texttt{true}:

Property Name \textbar{} Default Value
\hyperref[apidir]{\texttt{apiDir}} \textbar{}

\textbf{If the \texttt{war} plugin is applied:}
\texttt{\$\{project.webAppDir\}/WEB-INF/service}

\textbf{Otherwise:} \texttt{null}

\hyperref[hbmfile]{\texttt{hbmFile}} \textbar{}

\textbf{If \texttt{osgiModule} is \texttt{true}:}
\texttt{\$\{buildService.resourcesDir\}/META-INF/module-hbm.xml}

\textbf{Otherwise:}
\texttt{\$\{buildService.resourcesDir\}/META-INF/portlet-hbm.xml}

\hyperref[impldir]{\texttt{implDir}} \textbar{} The first \texttt{java}
directory of the \texttt{main} source set (by default:
\texttt{src/main/java}). \hyperref[inputfile]{\texttt{inputFile}}
\textbar{}

\textbf{If the \texttt{war} plugin is applied:}
\texttt{\$\{project.webAppDir\}/WEB-INF/service.xml}

\textbf{Otherwise:} \texttt{\$\{project.projectDir\}/service.xml}

\hyperref[modelhintsfile]{\texttt{modelHintsFile}} \textbar{} The file
\texttt{META-INF/portlet-model-hints.xml} in the first
\texttt{resources} directory of the \texttt{main} source set (by
default: \texttt{src/main/resources/META-INF/portlet-model-hints.xml}).
\hyperref[pluginname]{\texttt{pluginName}} \textbar{}

\textbf{If \texttt{osgiModule} is \texttt{true}:} \texttt{""}

\textbf{Otherwise:} \texttt{project.name}

\hyperref[pluginname]{\texttt{propsUtil}} \textbar{}

\textbf{If \texttt{osgiModule} is \texttt{true}:}
\texttt{"\$\{bundleSymbolicName\}.util.ServiceProps"}The
\texttt{bundleSymbolicName} of the project is inferred via the
\href{https://github.com/gradle/gradle/blob/master/subprojects/osgi/src/main/java/org/gradle/api/internal/plugins/osgi/OsgiHelper.java}{\texttt{OsgiHelper}}
class.

\textbf{Otherwise:} \texttt{"com.liferay.util.service.ServiceProps"}

\hyperref[resourcesdir]{\texttt{resourcesDir}} \textbar{} The first
\texttt{resources} directory of the \texttt{main} source set (by
default: \texttt{src/main/resources}).
\hyperref[springfile]{\texttt{springFile}} \textbar{}

\textbf{If \texttt{osgiModule} is \texttt{true}:} the file
\texttt{META-INF/spring/module-spring.xml} in the first
\texttt{resources} directory of the \texttt{main} source set (by
default: \texttt{src/main/resources/META-INF/spring/module-spring.xml})

\textbf{Otherwise:} the file \texttt{META-INF/portlet-spring.xml} in the
first \texttt{resources} directory of the \texttt{main} source set (by
default: \texttt{src/main/resources/META-INF/portlet-spring.xml})

\hyperref[sqldir]{\texttt{sqlDir}} \textbar{}

\textbf{If the \texttt{war} plugin is applied:}
\texttt{\$\{project.webAppDir\}/WEB-INF/sql}

\textbf{Otherwise:} The directory \texttt{META-INF/sql} in the first
\texttt{resources} directory of the \texttt{main} source set (by
default: \texttt{src/main/resources/META-INF/sql}).

In the typical scenario of a data-driven Liferay OSGi application split
in \texttt{myapp-app}, \texttt{myapp-service} and \texttt{myapp-web}
modules, the \texttt{service.xml} file is usually contained in the root
directory of \texttt{myapp-service}. In the \texttt{build.gradle} of the
same module, it is enough to apply the
\texttt{com.liferay.service.builder} plugin \hyperref[usage]{as
described}, and then add the following snippet to enable the use of
Liferay Service Builder:

\begin{verbatim}
buildService {
    apiDir = "../myapp-api/src/main/java"
    testDir = "../myapp-test/src/testIntegration/java"
}
\end{verbatim}

While \texttt{apiDir} is required, the \texttt{testDir} property
assignment can be left out, in which case Arquillian-based integration
test classes are generated.

\section{BuildServiceTask}\label{buildservicetask}

Tasks of type \texttt{BuildWSDDTask} extend
\href{https://docs.gradle.org/current/dsl/org.gradle.api.tasks.JavaExec.html}{\texttt{JavaExec}},
so all its properties and methods, such as
\href{https://docs.gradle.org/current/dsl/org.gradle.api.tasks.JavaExec.html\#org.gradle.api.tasks.JavaExec:args(java.lang.Iterable)}{\texttt{args}}
and
\href{https://docs.gradle.org/current/dsl/org.gradle.api.tasks.JavaExec.html\#org.gradle.api.tasks.JavaExec:maxHeapSize}{\texttt{maxHeapSize}}
are available. They also have the following properties set by default:

Property Name \textbar{} Default Value
\href{https://docs.gradle.org/current/dsl/org.gradle.api.tasks.JavaExec.html\#org.gradle.api.tasks.JavaExec:args}{\texttt{args}}
\textbar{} Service Builder command line arguments
\href{https://docs.gradle.org/current/dsl/org.gradle.api.tasks.JavaExec.html\#org.gradle.api.tasks.JavaExec:classpath}{\texttt{classpath}}
\textbar{}
\hyperref[liferay-service-builder-dependency]{\texttt{project.configurations.serviceBuilder}}
\href{https://docs.gradle.org/current/dsl/org.gradle.api.tasks.JavaExec.html\#org.gradle.api.tasks.JavaExec:main}{\texttt{main}}
\textbar{}
\texttt{"com.liferay.portal.tools.service.builder.ServiceBuilder"}
\href{https://docs.gradle.org/current/dsl/org.gradle.api.tasks.JavaExec.html\#org.gradle.api.tasks.JavaExec:systemProperties}{\texttt{systemProperties}}
\textbar{} \texttt{{[}"file.encoding":\ "UTF-8"{]}}

\subsection{Task Properties}\label{task-properties-25}

Property Name \textbar{} Type \textbar{} Default Value \textbar{}
Description \texttt{apiDir} \textbar{} \texttt{File} \textbar{}
\texttt{null} \textbar{} A directory where the service API Java source
files are generated. It sets the \texttt{service.api.dir} argument.
\texttt{autoImportDefaultReferences} \textbar{} \texttt{boolean}
\textbar{} \texttt{true} \textbar{} Whether to automatically add default
references, like \texttt{com.liferay.portal.ClassName},
\texttt{com.liferay.portal.Resource} and
\texttt{com.liferay.portal.User}, to the services. It sets the
\texttt{service.auto.import.default.references} argument.
\texttt{autoNamespaceTables} \textbar{} \texttt{boolean} \textbar{}
\texttt{true} \textbar{} Whether to prefix table names by the namespace
specified in the \texttt{service.xml} file. It sets the
\texttt{service.auto.namespace.tables} argument.
\texttt{beanLocatorUtil} \textbar{} \texttt{String} \textbar{}
\texttt{"com.liferay.util.bean.PortletBeanLocatorUtil"} \textbar{} The
fully qualified class name of a bean locator class to use in the
generated service classes. It sets the
\texttt{service.bean.locator.util} argument. \texttt{buildNumber}
\textbar{} \texttt{long} \textbar{} \texttt{1} \textbar{} A specific
value to assign the \texttt{build.number} property in the
\texttt{service.properties} file. It sets the
\texttt{service.build.number} argument. \texttt{buildNumberIncrement}
\textbar{} \texttt{boolean} \textbar{} \texttt{true} \textbar{} Whether
to automatically increment the \texttt{build.number} property in the
\texttt{service.properties} file by one at every service generation. It
sets the \texttt{service.build.number.increment} argument.
\texttt{databaseNameMaxLength} \textbar{} \texttt{int} \textbar{}
\texttt{30} \textbar{} The upper bound for database table and column
name lengths to ensure it works on all databases. It sets the
\texttt{service.database.name.max.length} argument. \texttt{hbmFile}
\textbar{} \texttt{File} \textbar{} \texttt{null} \textbar{} A Hibernate
Mapping file to generate. It sets the \texttt{service.hbm.file}
argument. \texttt{implDir} \textbar{} \texttt{File} \textbar{}
\texttt{null} \textbar{} A directory where the service Java source files
are generated. It sets the \texttt{service.impl.dir} argument.
\texttt{inputFile} \textbar{} \texttt{File} \textbar{} \texttt{null}
\textbar{} The project's \texttt{service.xml} file. It sets the
\texttt{service.input.file} argument. \texttt{modelHintsConfigs}
\textbar{} \texttt{Set} \textbar{}
\texttt{{[}"classpath*:META-INF/portal-model-hints.xml",\ "META-INF/portal-model-hints.xml",\ "classpath*:META-INF/ext-model-hints.xml",\ "classpath*:META-INF/portlet-model-hints.xml"{]}}
\textbar{} Paths to the model hints files for Liferay Service Builder to
use in generating the service layer. It sets the
\texttt{service.model.hints.configs} argument. \texttt{modelHintsFile}
\textbar{} \texttt{File} \textbar{} \texttt{null} \textbar{} A model
hints file for the project. It sets the
\texttt{service.model.hints.file} argument. \texttt{osgiModule}
\textbar{} \texttt{boolean} \textbar{} \texttt{false} \textbar{} Whether
to generate the service layer for OSGi modules. It sets the
\texttt{service.osgi.module} argument. \texttt{pluginName} \textbar{}
\texttt{String} \textbar{} \texttt{null} \textbar{} If specified, a
plugin can enable additional generation features, such as \texttt{Clp}
class generation, for non-OSGi modules. It sets the
\texttt{service.plugin.name} argument. \texttt{propsUtil} \textbar{}
\texttt{String} \textbar{} \texttt{null} \textbar{} The fully qualified
class name of the service properties util class to generate. It sets the
\texttt{service.props.util} argument. \texttt{readOnlyPrefixes}
\textbar{} \texttt{Set} \textbar{}
\texttt{{[}"fetch",\ "get",\ "has",\ "is",\ "load",\ "reindex",\ "search"{]}}
\textbar{} Prefixes of methods to consider read-only. It sets the
\texttt{service.read.only.prefixes} argument.
\texttt{resourceActionsConfigs} \textbar{} \texttt{Set} \textbar{}
\texttt{{[}"META-INF/resource-actions/default.xml",\ "resource-actions/default.xml"{]}}
\textbar{} Paths to the
\href{/docs/7-2/frameworks/-/knowledge_base/f/defining-application-permissions}{resource
actions} files for Liferay Service Builder to use in generating the
service layer. It sets the \texttt{service.resource.actions.configs}
argument. \texttt{resourcesDir} \textbar{} \texttt{File} \textbar{}
\texttt{null} \textbar{} A directory where the service non-Java files
are generated. It sets the \texttt{service.resources.dir} argument.
\texttt{springFile} \textbar{} \texttt{File} \textbar{} \texttt{null}
\textbar{} A service Spring file to generate. It sets the
\texttt{service.spring.file} argument. \texttt{springNamespaces}
\textbar{} \texttt{Set} \textbar{} \texttt{{[}"beans"{]}} \textbar{}
Namespaces of Spring XML Schemas to add to the service Spring file. It
sets the \texttt{service.spring.namespaces} argument. \texttt{sqlDir}
\textbar{} \texttt{File} \textbar{} \texttt{null} \textbar{} A directory
where the SQL files are generated. It sets the \texttt{service.sql.dir}
argument. \texttt{sqlFileName} \textbar{} \texttt{String} \textbar{}
\texttt{"tables.sql"} \textbar{} A name (relative to \texttt{sqlDir})
for the file in which the SQL table creation instructions are generated.
It sets the \texttt{service.sql.file} argument.
\texttt{sqlIndexesFileName} \textbar{} \texttt{String} \textbar{}
\texttt{"indexes.sql"} \textbar{} A name (relative to \texttt{sqlDir})
for the file in which the SQL index creation instructions are generated.
It sets the \texttt{service.sql.indexes.file} argument.
\texttt{sqlSequencesFileName} \textbar{} \texttt{String} \textbar{}
\texttt{"sequences.sql"} \textbar{} A name (relative to \texttt{sqlDir})
for the file in which the SQL sequence creation instructions are
generated. It sets the \texttt{service.sql.sequences.file} argument.
\texttt{targetEntityName} \textbar{} \texttt{String} \textbar{}
\texttt{null} \textbar{} If specified, it's the name of the entity for
which Liferay Service Builder should generate the service. It sets the
\texttt{service.target.entity.name} argument. \texttt{testDir}
\textbar{} \texttt{File} \textbar{} \texttt{null} \textbar{} If
specified, it's a directory where integration test Java source files are
generated. It sets the \texttt{service.test.dir} argument.
\texttt{uadDir} \textbar{} \texttt{File} \textbar{} \texttt{null}
\textbar{} A directory where the UAD (user-associated data) Java source
files are generated. It sets the \texttt{service.uad.dir} argument.
\texttt{uadTestIntegrationDir} \textbar{} \texttt{File} \textbar{}
\texttt{null} \textbar{} A directory where integration test UAD
(user-associated data) Java source files are generated. It sets the
\texttt{service.uad.test.integration.dir} argument.

The properties of type \texttt{File} supports any type that can be
resolved by
\href{https://docs.gradle.org/current/dsl/org.gradle.api.Project.html\#org.gradle.api.Project:file(java.lang.Object)}{\texttt{project.file}}.
Moreover, it is possible to use Closures and Callables as values for the
\texttt{String} properties, to defer evaluation until task execution.

\section{Additional Configuration}\label{additional-configuration-10}

There are additional configurations that can help you use Service
Builder.

\section{Liferay Service Builder
Dependency}\label{liferay-service-builder-dependency}

By default, the plugin creates a configuration called
\texttt{serviceBuilder} and adds a dependency to the latest released
version of Liferay Service Builder. It is possible to override this
setting and use a specific version of the tool by manually adding a
dependency to the \texttt{serviceBuilder} configuration:

\begin{verbatim}
dependencies {
    serviceBuilder group: "com.liferay", name: "com.liferay.portal.tools.service.builder", version: "1.0.292"
}
\end{verbatim}

If you're applying the
\href{https://github.com/liferay/liferay-portal/tree/master/modules/sdk/gradle-plugins}{\texttt{com.liferay.gradle.plugins}}
or
\href{https://github.com/liferay/liferay-portal/blob/master/modules/sdk/gradle-plugins-workspace}{\texttt{com.liferay.gradle.plugins.workspace}}
plugins to your project, the Service Builder dependency is already added
to the \texttt{serviceBuilder} configuration. Therefore, if you try to
apply a customized version of Service Builder, it's not recognized; you
must override the configuration already applied.

To do this, you must customize the classpath of the
\texttt{buildService} task. If you're supplying the customized Service
Builder plugin through a module named \texttt{custom-sb-api}, you could
modify the \texttt{buildService} task like this:

\begin{verbatim}
buildService {
    apiDir = "../custom-sb-api/src/main/java"
    classpath = configurations.serviceBuilder.filter { file -> !file.name.contains("com.liferay.portal.tools.service.builder") }
}
\end{verbatim}

If you do this in conjunction with the \texttt{serviceBuilder}
dependency configuration, the custom Service Builder version is used.

\chapter{Source Formatter Gradle
Plugin}\label{source-formatter-gradle-plugin}

The Source Formatter Gradle plugin lets you format project files using
the
\href{https://github.com/liferay/liferay-portal/tree/master/modules/util/source-formatter}{Liferay
Source Formatter} tool.

The plugin has been successfully tested with Gradle 4.10.2.

\section{Usage}\label{usage-19}

To use the plugin, include it in your build script:

\begin{verbatim}
buildscript {
    dependencies {
        classpath group: "com.liferay", name: "com.liferay.gradle.plugins.source.formatter", version: "2.3.413"
    }

    repositories {
        maven {
            url "https://repository-cdn.liferay.com/nexus/content/groups/public"
        }
    }
}

apply plugin: "com.liferay.source.formatter"
\end{verbatim}

Since the plugin automatically resolves the Liferay Source Formatter
library as a dependency, you have to configure a repository that hosts
the library and its transitive dependencies. The Liferay CDN repository
hosts them all:

\begin{verbatim}
repositories {
    maven {
        url "https://repository-cdn.liferay.com/nexus/content/groups/public"
    }
}
\end{verbatim}

\section{Tasks}\label{tasks-18}

The plugin adds two tasks to your project:

Name \textbar{} Depends On \textbar{} Type \textbar{} Description
\texttt{checkSourceFormatting} \textbar{} - \textbar{}
\hyperref[formatsourcetask]{\texttt{FormatSourceTask}} \textbar{} Runs
the Liferay Source Formatter to check for source formatting errors.
\texttt{formatSource} \textbar{} - \textbar{}
\hyperref[formatsourcetask]{\texttt{FormatSourceTask}} \textbar{} Runs
the Liferay Source Formatter to format the project files.

If desired, it is possible to check for source formatting errors while
executing the
\href{https://docs.gradle.org/current/userguide/java_plugin.html\#N15056}{\texttt{check}}
task by adding the following dependency:

\begin{verbatim}
check {
    dependsOn checkSourceFormatting
}
\end{verbatim}

The same can be achieved by adding the following snippet to the
\texttt{build.gradle} file in the root directory of a
\href{/docs/7-2/reference/-/knowledge_base/r/liferay-workspace}{\emph{Liferay
Workspace}}:

\begin{verbatim}
subprojects {
    afterEvaluate {
        if (plugins.hasPlugin("base") && plugins.hasPlugin("com.liferay.source.formatter")) {
            check.dependsOn checkSourceFormatting
        }
    }
}
\end{verbatim}

The tasks \texttt{checkSourceFormatting} and \texttt{formatSource} are
automatically skipped if another task with the same name is being
executed in a parent project.

\section{FormatSourceTask}\label{formatsourcetask}

Tasks of type \texttt{FormatSourceTask} extend
\href{https://docs.gradle.org/current/dsl/org.gradle.api.tasks.JavaExec.html}{\texttt{JavaExec}},
so all its properties and methods, like
\href{https://docs.gradle.org/current/dsl/org.gradle.api.tasks.JavaExec.html\#org.gradle.api.tasks.JavaExec:args(java.lang.Iterable)}{\texttt{args}}
and
\href{https://docs.gradle.org/current/dsl/org.gradle.api.tasks.JavaExec.html\#org.gradle.api.tasks.JavaExec:maxHeapSize}{\texttt{maxHeapSize}}
are available. They also have the following properties set by default:

Property Name \textbar{} Default Value
\href{https://docs.gradle.org/current/dsl/org.gradle.api.tasks.JavaExec.html\#org.gradle.api.tasks.JavaExec:args}{\texttt{args}}
\textbar{} Source Formatter command line arguments
\href{https://docs.gradle.org/current/dsl/org.gradle.api.tasks.JavaExec.html\#org.gradle.api.tasks.JavaExec:classpath}{\texttt{classpath}}
\textbar{}
\hyperref[liferay-source-formatter-dependency]{\texttt{project.configurations.sourceFormatter}}
\href{https://docs.gradle.org/current/dsl/org.gradle.api.tasks.JavaExec.html\#org.gradle.api.tasks.JavaExec:main}{\texttt{main}}
\textbar{} \texttt{"com.liferay.source.formatter.SourceFormatter"}

\subsection{Task Properties}\label{task-properties-26}

Property Name \textbar{} Type \textbar{} Default Value \textbar{}
Description \texttt{autoFix} \textbar{} \texttt{boolean} \textbar{}
\texttt{false} \textbar{} Whether to automatically fix source formatting
errors. It sets the \texttt{source.auto.fix} argument. \texttt{baseDir}
\textbar{} \texttt{File} \textbar{} \textbar{} The Source Formatter base
directory. It sets the \texttt{source.base.dir} argument.
\emph{(Read-only)} \texttt{baseDirName} \textbar{} \texttt{String}
\textbar{} \texttt{"./"} \textbar{} The name of the Source Formatter
base directory, relative to the project directory.
\texttt{fileExtensions} \textbar{}
\texttt{List\textless{}String\textgreater{}} \textbar{} \texttt{{[}{]}}
\textbar{} The file extensions to format. If empty, all file extensions
will be formatted. It sets the \texttt{source.file.extensions} argument.
\texttt{files} \textbar{} \texttt{List\textless{}File\textgreater{}}
\textbar{} \textbar{} The list of files to format. It sets the
\texttt{source.files} argument. \emph{(Read-only)} \texttt{fileNames}
\textbar{} \texttt{List\textless{}String\textgreater{}} \textbar{}
\texttt{null} \textbar{} The file names to format, relative to the
project directory. If \texttt{null}, all files contained in
\texttt{baseDir} will be formatted. \texttt{formatCurrentBranch}
\textbar{} \texttt{boolean} \textbar{} \texttt{false} \textbar{} Whether
to format only the files contained in \texttt{baseDir} that are added or
modified in the current Git branch. It sets the
\texttt{format.current.branch} argument. \texttt{formatLatestAuthor}
\textbar{} \texttt{boolean} \textbar{} \texttt{false} \textbar{} Whether
to format only the files contained in \texttt{baseDir} that are added or
modified in the latest Git commits of the same author. It sets the
\texttt{format.latest.author} argument. \texttt{formatLocalChanges}
\textbar{} \texttt{boolean} \textbar{} \texttt{false} \textbar{} Whether
to format only the unstaged files contained in \texttt{baseDir}. It sets
the \texttt{format.local.changes} argument.
\texttt{gitWorkingBranchName} \textbar{} \texttt{String} \textbar{}
\texttt{"master"} \textbar{} The Git working branch name. It sets the
\texttt{git.working.branch.name} argument.
\texttt{includeSubrepositories} \textbar{} \texttt{boolean} \textbar{}
\texttt{false} \textbar{} Whether to format files that are in read-only
subrepositories. It sets the \texttt{include.subrepositories} argument.
\texttt{maxLineLength} \textbar{} \texttt{int} \textbar{} \texttt{80}
\textbar{} The maximum number of characters allowed in Java files. It
sets the \texttt{max.line.length} argument. \texttt{printErrors}
\textbar{} \texttt{boolean} \textbar{} \texttt{true} \textbar{} Whether
to print formatting errors on the Standard Output stream. It sets the
\texttt{source.print.errors} argument. \texttt{processorThreadCount}
\textbar{} \texttt{int} \textbar{} \texttt{5} \textbar{} The number of
threads used by Source Formatter. It sets the
\texttt{processor.thread.count} argument. \texttt{showDebugInformation}
\textbar{} \texttt{boolean} \textbar{} \texttt{false} \textbar{} Whether
to show debug information, if present. It sets the
\texttt{show.debug.information} argument. \texttt{showDocumentation}
\textbar{} \texttt{boolean} \textbar{} \texttt{false} \textbar{} Whether
to show the documentation for the source formatting issues, if present.
It sets the \texttt{show.documentation} argument.
\texttt{showStatusUpdates} \textbar{} \texttt{boolean} \textbar{}
\texttt{false} \textbar{} Whether to show status updates during source
formatting, if present. It sets the \texttt{show.status.updates}
argument. \texttt{throwException} \textbar{} \texttt{boolean} \textbar{}
\texttt{false} \textbar{} Whether to fail the build if formatting errors
are found. It sets the \texttt{source.throw.exception} argument.

\section{Additional Configuration}\label{additional-configuration-11}

There are additional configurations that can help you use the Source
Formatter.

\section{Liferay Source Formatter
Dependency}\label{liferay-source-formatter-dependency}

By default, the plugin creates a configuration called
\texttt{sourceFormatter} and adds a dependency to the latest released
version of Liferay Source Formatter. It is possible to override this
setting and use a specific version of the tool by manually adding a
dependency to the \texttt{sourceFormatter} configuration:

\begin{verbatim}
dependencies {
    sourceFormatter group: "com.liferay", name: "com.liferay.source.formatter", version: "1.0.885"
}
\end{verbatim}

\section{System Properties}\label{system-properties-3}

It is possible to set the default values of the \texttt{fileExtensions},
\texttt{fileNames}, \texttt{formatCurrentBranch},
\texttt{formatLatestAuthor}, and \texttt{formatLocalChanges} properties
for a \texttt{FormatSourceTask} task via system properties:

\begin{itemize}
\tightlist
\item
  \texttt{-D\$\{task.name\}.file.extensions=java,xml}
\item
  \texttt{-D\$\{task.name\}.file.names=README.markdown,src/main/resources/hello.txt}
\item
  \texttt{-D\$\{task.name\}.format.current.branch=true}
\item
  \texttt{-D\$\{task.name\}.format.latest.author=true}
\item
  \texttt{-D\$\{task.name\}.format.local.changes=true}
\end{itemize}

For example, run the following Bash command to format only the unstaged
files in the project:

\begin{verbatim}
./gradlew formatSource -DformatSource.format.local.changes=true
\end{verbatim}

\chapter{Soy Gradle Plugin}\label{soy-gradle-plugin}

The Soy Gradle plugin lets you compile
\href{https://developers.google.com/closure/templates/}{Closure
Templates} into JavaScript functions. It also lets you use a custom
localization mechanism in the generated \texttt{.soy.js} files by
replacing
\href{https://developers.google.com/closure/templates/docs/translation\#closurecompiler}{\texttt{goog.getMsg}}
definitions with a different function call (e.g.,
\texttt{Liferay.Language.get}).

The plugin has been successfully tested with Gradle 4.10.2.

\section{Usage}\label{usage-20}

To use the plugin, include it in your build script:

\begin{verbatim}
buildscript {
    dependencies {
        classpath group: "com.liferay", name: "com.liferay.gradle.plugins.soy", version: "3.1.8"
    }

    repositories {
        maven {
            url "https://repository-cdn.liferay.com/nexus/content/groups/public"
        }
    }
}
\end{verbatim}

There are two Soy Gradle plugins you can apply to your project:

\begin{itemize}
\item
  Apply the \hyperref[soy-plugin]{\emph{Soy Plugin}} to compile Closure
  Templates into JavaScript functions:

\begin{verbatim}
apply plugin: "com.liferay.soy"
\end{verbatim}
\item
  Apply the \hyperref[soy-translation-plugin]{\emph{Soy Translation
  Plugin}} to use a custom localization mechanism in the generated
  \texttt{.soy.js} files:

\begin{verbatim}
apply plugin: "com.liferay.soy.translation"
\end{verbatim}
\end{itemize}

Since the Soy Gradle plugin automatically resolves the Soy library as a
dependency, you have to configure a repository that hosts the library
and its transitive dependencies. The Liferay CDN repository hosts them
all:

\begin{verbatim}
repositories {
    maven {
        url "https://repository-cdn.liferay.com/nexus/content/groups/public"
    }
}
\end{verbatim}

\section{Soy Plugin}\label{soy-plugin}

The Soy plugin adds two tasks to your project:

Name \textbar{} Depends On \textbar{} Type \textbar{} Description
\texttt{buildSoy} \textbar{} - \textbar{}
\hyperref[buildsoytask]{\texttt{BuildSoyTask}} \textbar{} Compiles
Closure Templates into JavaScript functions.
\texttt{wrapSoyAlloyTemplate} \textbar{} - \texttt{configJSModules} if
\href{https://github.com/liferay/liferay-portal/tree/master/modules/sdk/gradle-plugins-js-module-config-generator}{\texttt{com.liferay.js.module.config.generator}}
is applied - \texttt{processResources} if \texttt{java} is applied -
\texttt{transpileJS} if
\href{https://github.com/liferay/liferay-portal/tree/master/modules/sdk/gradle-plugins-js-transpiler}{\texttt{com.liferay.js.transpiler}}
is applied \textbar{}
\hyperref[wrapsoyalloytemplatetask]{\texttt{WrapSoyAlloyTemplateTask}}
\textbar{} Wraps the JavaScript functions compiled from Closure
Templates into AlloyUI modules.

The plugin also adds the following dependencies to tasks defined by the
\texttt{java} plugin:

Name \textbar{} Depends On \texttt{classes} \textbar{}
\texttt{wrapSoyAlloyTemplate}

The \texttt{buildSoy} task is automatically configured with sensible
defaults, depending on whether the
\href{https://docs.gradle.org/current/userguide/java_plugin.html}{\texttt{java}}
plugin is applied:

Property Name \textbar{} Default Value
\href{https://docs.gradle.org/current/dsl/org.gradle.api.tasks.SourceTask.html\#org.gradle.api.tasks.SourceTask:includes}{\texttt{includes}}
\textbar{} \texttt{{[}"**/*.soy"{]}}
\href{https://docs.gradle.org/current/dsl/org.gradle.api.tasks.SourceTask.html\#org.gradle.api.tasks.SourceTask:source}{\texttt{source}}
\textbar{}

\textbf{If the \texttt{java} plugin is applied:} The first
\texttt{resources} directory of the \texttt{main} source set (by
default, \texttt{src/main/resources}).

\textbf{Otherwise:} \texttt{{[}{]}}

The \texttt{wrapSoyAlloyTemplate} task is \textbf{disabled by default},
and it is automatically configured with sensible defaults, depending on
whether the \texttt{java} plugin is applied:

Property Name \textbar{} Default Value
\href{https://docs.gradle.org/current/dsl/org.gradle.api.Task.html\#org.gradle.api.Task:enabled}{\texttt{enabled}}
\textbar{} \texttt{false}
\href{https://docs.gradle.org/current/dsl/org.gradle.api.tasks.SourceTask.html\#org.gradle.api.tasks.SourceTask:includes}{\texttt{includes}}
\textbar{} \texttt{{[}"**/*.soy.js"{]}}
\href{https://docs.gradle.org/current/dsl/org.gradle.api.tasks.SourceTask.html\#org.gradle.api.tasks.SourceTask:source}{\texttt{source}}
\textbar{}

\textbf{If the \texttt{java} plugin is applied:}
\texttt{project.sourceSets.main.output.resourcesDir}

\textbf{Otherwise:} \texttt{{[}{]}}

\section{Additional Configuration}\label{additional-configuration-12}

There are additional configurations that can help you use the Soy
library.

\subsection{Soy Dependency}\label{soy-dependency}

By default, the plugin creates a configuration called \texttt{soy} and
adds a dependency to the \texttt{2015-04-10} version of the Soy library.
It is possible to override this setting and use a specific version of
the tool by manually adding a dependency to the \texttt{soy}
configuration:

\begin{verbatim}
dependencies {
    soy group: "com.google.template", name: "soy", version: "2015-04-10"
}
\end{verbatim}

\section{Soy Translation Plugin}\label{soy-translation-plugin}

The Soy Translation plugin adds one task to your project:

Name \textbar{} Depends On \textbar{} Type \textbar{} Description
\texttt{replaceSoyTranslation} \textbar{} - \texttt{configJSModules} if
\href{https://github.com/liferay/liferay-portal/tree/master/modules/sdk/gradle-plugins-js-module-config-generator}{\texttt{com.liferay.js.module.config.generator}}
is applied - \texttt{processResources} if \texttt{java} is applied -
\texttt{transpileJS} if
\href{https://github.com/liferay/liferay-portal/tree/master/modules/sdk/gradle-plugins-js-transpiler}{\texttt{com.liferay.js.transpiler}}
is applied \textbar{}
\hyperref[replacesoytranslationtask]{\texttt{ReplaceSoyTranslationTask}}
\textbar{} Replaces \texttt{goog.getMsg} definitions with
\texttt{Liferay.Language.get} calls.

The plugin also adds the following dependencies to tasks defined by the
\texttt{java} plugin:

Name \textbar{} Depends On \texttt{classes} \textbar{}
\texttt{replaceSoyTranslation}

The \texttt{replaceSoyTranslation} task is automatically configured with
sensible defaults, depending on whether the \texttt{java} plugin is
applied:

Property Name \textbar{} Default Value
\href{https://docs.gradle.org/current/dsl/org.gradle.api.tasks.SourceTask.html\#org.gradle.api.tasks.SourceTask:includes}{\texttt{includes}}
\textbar{} \texttt{{[}"**/*.soy.js"{]}}
\hyperref[replacementclosure]{\texttt{replacementClosure}} \textbar{}
Replaces \texttt{goog.getMsg} definitions with
\texttt{Liferay.Language.get} calls.
\href{https://docs.gradle.org/current/dsl/org.gradle.api.tasks.SourceTask.html\#org.gradle.api.tasks.SourceTask:source}{\texttt{source}}
\textbar{}

\textbf{If the \texttt{java} plugin is applied:}
\texttt{project.sourceSets.main.output.resourcesDir}

\textbf{Otherwise:} \texttt{{[}{]}}

\section{Tasks}\label{tasks-19}

\section{BuildSoyTask}\label{buildsoytask}

Tasks of type \texttt{BuildSoyTask} extend
\href{https://docs.gradle.org/current/dsl/org.gradle.api.tasks.SourceTask.html}{\texttt{SourceTask}},
so all its properties and methods, such as
\href{https://docs.gradle.org/current/dsl/org.gradle.api.tasks.SourceTask.html\#org.gradle.api.tasks.SourceTask:include(java.lang.Iterable)}{\texttt{include}}
and
\href{https://docs.gradle.org/current/dsl/org.gradle.api.tasks.SourceTask.html\#org.gradle.api.tasks.SourceTask:exclude(java.lang.Iterable)}{\texttt{exclude}},
are available.

\subsection{Task Properties}\label{task-properties-27}

Property Name \textbar{} Type \textbar{} Default Value \textbar{}
Description \texttt{classpath} \textbar{}
\href{https://docs.gradle.org/current/javadoc/org/gradle/api/file/FileCollection.html}{\texttt{FileCollection}}
\textbar{}
\hyperref[soy-dependency]{\texttt{project.configurations.soy}}
\textbar{} The classpath for executing the
\href{https://github.com/liferay/liferay-portal/tree/master/modules/util/portal-tools-soy-builder}{Liferay
Portal Tools Soy Builder}.

\section{WrapSoyAlloyTemplateTask}\label{wrapsoyalloytemplatetask}

Tasks of type \texttt{WrapSoyAlloyTemplateTask} extend
\href{https://docs.gradle.org/current/dsl/org.gradle.api.tasks.SourceTask.html}{\texttt{SourceTask}},
so all its properties and methods, such as
\href{https://docs.gradle.org/current/dsl/org.gradle.api.tasks.SourceTask.html\#org.gradle.api.tasks.SourceTask:include(java.lang.Iterable)}{\texttt{include}}
and
\href{https://docs.gradle.org/current/dsl/org.gradle.api.tasks.SourceTask.html\#org.gradle.api.tasks.SourceTask:exclude(java.lang.Iterable)}{\texttt{exclude}},
are available.

\subsection{Task Properties}\label{task-properties-28}

Property Name \textbar{} Type \textbar{} Default Value \textbar{}
Description \texttt{moduleName} \textbar{} \texttt{String} \textbar{}
\texttt{null} \textbar{} The name of the AlloyUI module.
\texttt{namespace} \textbar{} \texttt{String} \textbar{} \texttt{null}
\textbar{} The namespace of the Closure Templates of the project.

It is possible to use Closures and Callables as values for the
\texttt{String} properties to defer evaluation until task execution.

\section{ReplaceSoyTranslationTask}\label{replacesoytranslationtask}

The \texttt{ReplaceSoyTranslationTask} task type finds all the
\texttt{goog.getMsg} definitions in the project's files and replaces
them with a custom function call.

\begin{verbatim}
var MSG_EXTERNAL_123 = goog.getMsg('welcome-to-{$releaseInfo}', { 'releaseInfo': opt_data.releaseInfo });
\end{verbatim}

A \texttt{goog.getMsg} definition looks like the example above, and it
has the following components:

\begin{itemize}
\tightlist
\item
  \emph{variable name}: \texttt{MSG\_EXTERNAL\_123}
\item
  \emph{language key}: \texttt{welcome-to-\{\$releaseInfo\}}
\item
  \emph{arguments object}:
  \texttt{\{\ \textquotesingle{}releaseInfo\textquotesingle{}:\ opt\_data.releaseInfo\ \}}
\end{itemize}

Tasks of type \texttt{ReplaceSoyTranslationTask} extend
\href{https://docs.gradle.org/current/dsl/org.gradle.api.tasks.SourceTask.html}{\texttt{SourceTask}},
so all its properties and methods, such as
\href{https://docs.gradle.org/current/dsl/org.gradle.api.tasks.SourceTask.html\#org.gradle.api.tasks.SourceTask:include(java.lang.Iterable)}{\texttt{include}}
and
\href{https://docs.gradle.org/current/dsl/org.gradle.api.tasks.SourceTask.html\#org.gradle.api.tasks.SourceTask:exclude(java.lang.Iterable)}{\texttt{exclude}},
are available.

\subsection{Task Properties}\label{task-properties-29}

Property Name \textbar{} Type \textbar{} Default Value \textbar{}
Description \texttt{replacementClosure} \textbar{}
\texttt{Closure\textless{}String\textgreater{}} \textbar{} \texttt{null}
\textbar{} The Closure invoked in order to get the replacement for
\texttt{goog.getMsg} definitions. The given Closure is passed the
\emph{variable name}, \emph{language key}, and \emph{arguments object}
as its parameters.

\chapter{Target Platform Gradle
Plugin}\label{target-platform-gradle-plugin}

The Target Platform Gradle plugin helps with building multiple projects
against a declared API target platform. Java dependencies can be managed
with Maven BOMs and OSGi modules can be resolved against an OSGi
distribution.

The plugin has been successfully tested with Gradle 4.10.2.

\section{Usage}\label{usage-21}

To use the plugin, include it in your build script:

\begin{verbatim}
buildscript {
    dependencies {
        classpath group: "com.liferay", name: "com.liferay.gradle.plugins.target.platform", version: "2.0.0"
    }

    repositories {
        maven {
            url "https://repository-cdn.liferay.com/nexus/content/groups/public"
        }
    }
}
\end{verbatim}

There are two Target Platform Gradle plugins you can apply to your
project. If you have a multi-module Gradle project, you only need to
apply these plugins to the root project.

\begin{itemize}
\item
  The \hyperref[target-platform-plugin]{\emph{Target Platform Plugin}}
  helps to configure your projects to build against an established set
  of platform artifacts, including Java and OSGi dependencies.

\begin{verbatim}
apply plugin: "com.liferay.target.platform"
\end{verbatim}
\item
  The \hyperref[target-platform-ide-plugin]{\emph{Target Platform IDE
  Plugin}} is a superset of the Target Platform Plugin (it applies the
  above plugin) and also adds IDE integration for searching and
  debugging source code in the target platform artifacts.

\begin{verbatim}
apply plugin: "com.liferay.target.platform.ide"
\end{verbatim}
\end{itemize}

Since the plugin automatically resolves target platform configurations
as dependencies, you must configure a repository that hosts these
artifacts. The Liferay CDN repository hosts them all:

\begin{verbatim}
repositories {
    maven {
        url "https://repository-cdn.liferay.com/nexus/content/groups/public"
    }
}
\end{verbatim}

\section{Target Platform Plugin}\label{target-platform-plugin}

The plugin applies the
\href{https://github.com/spring-gradle-plugins/dependency-management-plugin}{Spring
Dependency Management Plugin} and then adds several specific
configurations to configure the BOMs that are imported to manage Java
dependencies and the various artifacts used in resolving OSGi
dependencies. Also, a new \texttt{resolve} task is added to resolve all
OSGi requirements against a declared distribution artifact.

The plugin adds a series of configurations to your project:

Name \textbar{} Description \texttt{targetPlatformBoms} \textbar{}
Configures all the BOMs to import as managed dependencies.
\texttt{targetPlatformBundles} \textbar{} Configures all the bundles in
addition to the distro to resolve against. \texttt{targetPlatformDistro}
\textbar{} Configures the distro JAR file to use as base for resolving
against. \texttt{targetPlatformRequirements} \textbar{} Configures the
list of JAR files to use as run requirements for resolving.

The plugin adds a task \texttt{resolve} of type
\hyperref[resolvetask]{\texttt{ResolveTask}} to your project that
performs an OSGi resolve operation using the
\texttt{targetPlatformRequirements} configuration as the basis of the
requirements. The \texttt{targetPlatformBundles} configuration is used
as a repository for the resolver to resolve requirements. Lastly, the
\texttt{targetPlatformDistro} configuration is used to provide the
\emph{distro} artifact for the resolve process. The \emph{distro} is the
artifact that provides all the OSGi capabilities of the target platform.
All of these parameters are used to create a \texttt{bndrun} file that
can be used as input into the Bndrun resolve operation.

\section{Target Platform IDE Plugin}\label{target-platform-ide-plugin}

The plugin applies the \hyperref[target-platform-plugin]{Target
Platform} and the
\href{https://docs.gradle.org/current/userguide/eclipse_plugin.html}{\texttt{eclipse}}
plugins to your project, and also adds a special
\texttt{targetPlatformIDE} configuration, which is used to configure
both the \texttt{eclipse} model and \texttt{idea} plugin model in Gradle
to add all target platform artifacts to the classpath so they are
visible to both Eclipse and IntelliJ's Java Model Search (for looking up
sources to classes).

\section{Project Extension}\label{project-extension-7}

The Target Platform plugin exposes the following properties through the
extension named \texttt{targetPlatform}:

Property Name \textbar{} Type \textbar{} Default Value \textbar{}
Description \texttt{ignoreResolveFailures} \textbar{} \texttt{boolean}
\textbar{} \texttt{true} \textbar{} Whether to ignore resolve failures
found when executing tasks of type
\hyperref[resolvetask]{\texttt{ResolveTask}}. \texttt{subprojects}
\textbar{} \texttt{Set\textless{}Project\textgreater{}} \textbar{}
\texttt{project.subprojects} \textbar{} The subprojects to configure
with target platform support, including dependency management and the
\texttt{resolve} task.

The same extension exposes the following methods:

Method \textbar{} Description
\texttt{TargetPlatformExtension\ applyToConfiguration(Iterable\textless{}?\textgreater{}\ configurationNames)}
\textbar{} Adds additional configurations to configure the BOMs that are
imported to manage Java dependencies and the various artifacts used in
resolving OSGi dependencies.
\texttt{TargetPlatformExtension\ applyToConfiguration(Object...\ configurationNames)}
\textbar{} Adds additional configurations to configure the BOMs that are
imported to manage Java dependencies and the various artifacts used in
resolving OSGi dependencies.
\texttt{TargetPlatformExtension\ onlyIf(Closure\textless{}Boolean\textgreater{}\ onlyIfClosure)}
\textbar{} Includes a subproject in the target platform configuration if
the given closure returns \texttt{true}. The closure is evaluated at the
end of the subproject configuration phase and is passed a single
parameter: the subproject. If the closure returns \texttt{false}, the
subproject is not included in the target platform configuration
\texttt{TargetPlatformExtension\ onlyIf(Spec\textless{}Project\textgreater{}\ onlyIfSpec)}
\textbar{} Includes a subproject in the target platform configuration if
the given spec is satisfied. The spec is evaluated at the end of the
subproject configuration phase. If the spec is not satisfied, the
subproject is not included in the target platform configuration.
\texttt{TargetPlatformExtension\ resolveOnlyIf(Closure\textless{}Boolean\textgreater{}\ resolveOnlyIfClosure)}
\textbar{} Includes a subproject in the resolving process (including
both the requirements and bundles configuration) if the given closure
returns \texttt{true}. The closure is evaluated at the end of the
subproject configuration phase and is passed a single parameter: the
subproject. If the closure returns \texttt{false}, the subproject is the
resolution process.
\texttt{TargetPlatformExtension\ resolveOnlyIf(Spec\textless{}Project\textgreater{}\ resolveOnlyIfSpec)}
\textbar{} Includes a subproject in the resolving platform configuration
if the given spec is satisfied. The spec is evaluated at the end of the
subproject configuration phase. If the spec is not satisfied, the
subproject is not included in the target platform configuration.
\texttt{TargetPlatformExtension\ subprojects(Iterable\textless{}Project\textgreater{}\ subprojects)}
\textbar{} Includes additional projects to be configured with Target
Platform support.
\texttt{TargetPlatformExtension\ subprojects(Project...\ subprojects)}
\textbar{} Includes additional projects to be configured with Target
Platform support.

\section{Tasks}\label{tasks-20}

\section{ResolveTask}\label{resolvetask}

The purpose of this task is to resolve an OSGi module (or all OSGi
modules of subprojects) against the available
\texttt{targetPlatformBundles} and \texttt{targetPlatformDistro}
configurations. By default, the \texttt{targetPlatformBundles} are all
the artifacts created by all the subprojects. The
\texttt{targetPlatformDistro} must be set explicitly to a valid
distribution artifact. When the task is performed, a \texttt{bndrun}
file is generated using the specified \texttt{targetPlatformDistro} as
the \texttt{-distro} instruction; the \texttt{-runrequirements} are a
set of \texttt{osgi.identity} requirements for the
\texttt{targetPlatformRequirements} configuration. If the resolve
operation is able to find a valid set of \texttt{-runbundles} that match
the \texttt{-runrequirements}, then the task passes successfully (the
resolution is valid). If a set of run bundles can't be found, the
resolution has failed and the failed requirements are listed as output
of the task.

\subsection{Task Properties}\label{task-properties-30}

Property Name \textbar{} Type \textbar{} Default Value \textbar{}
Description \texttt{bndrunFile} \textbar{} \texttt{File} \textbar{}
\texttt{null} \textbar{} If this property is specified, it is used as
the \texttt{bndrun} file to input into the resolver.
\texttt{bundlesFileCollection} \textbar{} \texttt{FileCollection}
\textbar{} All JAR files of subprojects with \texttt{jar} task
\textbar{} The input to \texttt{bndrun} resolve operation.
\texttt{distroFileCollection} \textbar{} \texttt{FileCollection}
\textbar{} \texttt{null} \textbar{} The \emph{distro} parameter for the
generated \texttt{bndrun} file. \texttt{ignoreFailures} \textbar{}
\texttt{boolean} \textbar{} \texttt{false} \textbar{} Whether the
\texttt{resolve} task should ignore failing the build for resolution
errors. \texttt{offline} \textbar{} \texttt{boolean} \textbar{}
\texttt{null} \textbar{} Whether to run the bndrun resolve operation in
offline mode. \texttt{requirementsFileCollection} \textbar{}
\texttt{FileCollection} \textbar{}

\textbf{For the root project:} All the output JAR files of the
subprojects.

\textbf{For subprojects:} The output JAR file of the subproject.

\textbar{} For each resolve operation, the requirements must be
specified in the \texttt{bndrun} file; each of the JARs in this
collection generate an \texttt{osgi.identify} requirement in the
\texttt{bndrun} file.

\section{Additional Configuration}\label{additional-configuration-13}

There are additional configurations that you can use to configure the
target platform.

\section{Target Platform BOMs
Dependency}\label{target-platform-boms-dependency}

The plugin creates a configuration called \texttt{targetPlatformBoms}
with no defaults. You can use this dependency to set which BOMs to
import to configure your target platform.

\begin{verbatim}
dependencies {
    targetPlatformBoms group: "com.liferay.portal", name: "release.portal.bom", version: "7.2.0"
    targetPlatformBoms group: "com.liferay.portal", name: "release.portal.bom.compile.only", version: "7.2.0"
}
\end{verbatim}

\section{Target Platform Bundles
Dependency}\label{target-platform-bundles-dependency}

The plugin creates a configuration called
\texttt{targetPlatformBundles}. It is configured with default
dependencies to all resolvable bundles in a multi-project build (e.g.,
all projects in multi-project build that have a \texttt{jar} task). This
can be used to specify additional bundles that should be added to the
set of bundles given to \texttt{resolve} task to resolve against when
checking for OSGi requirements.

\begin{verbatim}
dependencies {
    targetPlatformBundles group: "com.google.guava", name: "guava", version: "23.0"
}
\end{verbatim}

\section{Target Platform Distro
Dependency}\label{target-platform-distro-dependency}

The plugin creates a configuration called \texttt{targetPlatformDistro}.
It is has no default so you must specify which artifact you want to use
as the distribution to resolve against.

\begin{verbatim}
dependencies {
    targetPlatformDistro group: "com.liferay.portal", name: "release.portal.distro", version: "7.2.0"
}
\end{verbatim}

If you have created your own custom distro JAR that is available
locally, you can use the \texttt{files} method to add it to the
configuration.

\begin{verbatim}
dependencies {
    targetPlatformDistro files("custom-distro.jar")
}
\end{verbatim}

\section{Target Platform Requirements
Dependency}\label{target-platform-requirements-dependency}

The plugin creates a configuration called
\texttt{targetPlatformRequirements}. It is configured with default
dependencies to all resolvable bundles in a multi-project build (e.g.,
all projects in multi-project build that have a \texttt{jar} task). This
is can be used to specify additional bundles that should be added to the
set of bundles given to the \texttt{resolve} task to set as
\texttt{osgi.identity} requirements.

\begin{verbatim}
dependencies {
    targetPlatformRequirements group: "com.liferay", name: "com.liferay.other.bundle", version: "1.0"
}
\end{verbatim}

\chapter{Theme Builder Gradle Plugin}\label{theme-builder-gradle-plugin}

The Theme Builder Gradle plugin lets you run the
\href{https://github.com/liferay/liferay-portal/tree/master/modules/util/portal-tools-theme-builder}{Liferay
Theme Builder} tool to build the Liferay theme files in your project.

The plugin has been successfully tested with Gradle 4.10.2.

\section{Usage}\label{usage-22}

To use the plugin, include it in your build script:

\begin{verbatim}
buildscript {
    dependencies {
        classpath group: "com.liferay", name: "com.liferay.gradle.plugins.theme.builder", version: "2.0.7"
    }

    repositories {
        maven {
            url "https://repository-cdn.liferay.com/nexus/content/groups/public"
        }
    }
}

apply plugin: "com.liferay.portal.tools.theme.builder"
\end{verbatim}

The Theme Builder plugin automatically applies the
\href{https://docs.gradle.org/current/userguide/war_plugin.html}{\texttt{war}}
plugin. It also applies the
\href{https://github.com/liferay/liferay-portal/tree/master/modules/sdk/gradle-plugins-css-builder}{\texttt{com.liferay.css.builder}}
plugin to compile the \href{http://sass-lang.com/}{Sass} files in the
theme.

Since the plugin automatically resolves the Liferay Theme Builder
library as a dependency, you have to configure a repository that hosts
the library and its transitive dependencies. The Liferay CDN repository
hosts them all:

\begin{verbatim}
repositories {
    maven {
        url "https://repository-cdn.liferay.com/nexus/content/groups/public"
    }
}
\end{verbatim}

\section{Tasks}\label{tasks-21}

The plugin adds one task to your project:

Name \textbar{} Depends On \textbar{} Type \textbar{} Description
\texttt{buildTheme} \textbar{} - \textbar{}
\hyperref[buildthemetask]{\texttt{BuildThemeTask}} \textbar{} Builds the
theme files.

The plugin also adds the following dependencies to tasks defined by the
\texttt{com.liferay.css.builder} and \texttt{war} plugins:

Name \textbar{} Depends On
\href{https://github.com/liferay/liferay-portal/tree/master/modules/sdk/gradle-plugins-css-builder\#tasks}{\texttt{buildCSS}}
\textbar{} \texttt{buildTheme}
\href{https://docs.gradle.org/current/userguide/war_plugin.html\#sec:war_default_settings}{\texttt{war}}
\textbar{} \texttt{buildTheme}

The \texttt{buildCSS} dependency compiles the Sass files contained in
the directory specified by the
\hyperref[outputdir]{\texttt{buildTheme.outputDir}} property. Moreover,
the \texttt{war} task is configured as follows

\begin{itemize}
\tightlist
\item
  exclude the directory specified in the
  \hyperref[diffsdir]{\texttt{buildTheme.diffsDir}} property from the
  WAR file.
\item
  include the files contained in the
  \hyperref[outputdir]{\texttt{buildTheme.outputDir}} directory into the
  WAR file.
\item
  include only the compiled CSS files, not SCSS files, into the WAR
  file.
\end{itemize}

The \texttt{buildTheme} task is automatically configured with sensible
defaults:

Property Name \textbar{} Default Value
\hyperref[diffsdir]{\texttt{diffsDir}} \textbar{}
\texttt{project.webAppDir} \hyperref[outputdir]{\texttt{outputDir}}
\textbar{} \texttt{"\$\{project.buildDir\}/buildTheme"}
\hyperref[parentfile]{\texttt{parentFile}} \textbar{} The first JAR file
in the \hyperref[parent-theme-dependencies]{\texttt{parentThemes}}
configuration that contains a
\texttt{META-INF/resources/\$\{buildTheme.parentName\}} directory, or
the first WAR file in the \texttt{parentThemes} configuration whose name
starts with \texttt{\$\{parentName\}-theme-}.
\hyperref[parentname]{\texttt{parentName}} \textbar{}
\texttt{"\_styled"}
\hyperref[templateextension]{\texttt{templateExtension}} \textbar{}
\texttt{"ftl"} \hyperref[themename]{\texttt{themeName}} \textbar{}
\texttt{project.name} \hyperref[unstyledfile]{\texttt{unstyledFile}}
\textbar{} The first JAR file in the
\hyperref[parent-theme-dependencies]{\texttt{parentThemes}}
configuration that contains a \texttt{META-INF/resources/\_unstyled}
directory.

\section{BuildThemeTask}\label{buildthemetask}

Tasks of type \texttt{BuildThemeTask} extend
\href{https://docs.gradle.org/current/dsl/org.gradle.api.tasks.JavaExec.html}{\texttt{JavaExec}},
so all its properties and methods, such as
\href{https://docs.gradle.org/current/dsl/org.gradle.api.tasks.JavaExec.html\#org.gradle.api.tasks.JavaExec:args(java.css.Iterable)}{\texttt{args}}
and
\href{https://docs.gradle.org/current/dsl/org.gradle.api.tasks.JavaExec.html\#org.gradle.api.tasks.JavaExec:maxHeapSize}{\texttt{maxHeapSize}},
are available. They also have the following properties set by default:

Property Name \textbar{} Default Value
\href{https://docs.gradle.org/current/dsl/org.gradle.api.tasks.JavaExec.html\#org.gradle.api.tasks.JavaExec:args}{\texttt{args}}
\textbar{} Theme Builder command line arguments
\href{https://docs.gradle.org/current/dsl/org.gradle.api.tasks.JavaExec.html\#org.gradle.api.tasks.JavaExec:classpath}{\texttt{classpath}}
\textbar{}
\hyperref[liferay-theme-builder-dependency]{\texttt{project.configurations.themeBuilder}}
\href{https://docs.gradle.org/current/dsl/org.gradle.api.tasks.JavaExec.html\#org.gradle.api.tasks.JavaExec:main}{\texttt{main}}
\textbar{}
\texttt{"com.liferay.portal.tools.theme.builder.ThemeBuilder"}

\subsection{Task Properties}\label{task-properties-31}

Property Name \textbar{} Type \textbar{} Default Value \textbar{}
Description \texttt{diffsDir} \textbar{} \texttt{File} \textbar{}
\texttt{null} \textbar{} The directory that contains the files to copy
over the parent theme. It sets the \texttt{-\/-diffs-dir} argument.
\texttt{outputDir} \textbar{} \texttt{File} \textbar{} \texttt{null}
\textbar{} The directory where to build the theme. It sets the
\texttt{-\/-output-dir} argument. \texttt{parentDir} \textbar{}
\texttt{File} \textbar{} \texttt{null} \textbar{} The directory of the
parent theme. It sets the \texttt{-\/-parent-path} argument.
\texttt{parentFile} \textbar{} \texttt{File} \textbar{} \texttt{null}
\textbar{} The JAR file of the parent theme. If \texttt{parentDir} is
specified, this property has no effect. It sets the
\texttt{-\/-parent-path} argument. \texttt{parentName} \textbar{}
\texttt{String} \textbar{} \texttt{null} \textbar{} The name of the
parent theme. It sets the \texttt{-\/-parent-name} argument.
\texttt{templateExtension} \textbar{} \texttt{String} \textbar{}
\texttt{null} \textbar{} The extension of the template files, usually
\texttt{"ftl"} or \texttt{"vm"}. It sets the
\texttt{-\/-template-extension} argument. \texttt{themeName} \textbar{}
\texttt{String} \textbar{} \texttt{null} \textbar{} The name of the new
theme. It sets the \texttt{-\/-name} argument. \texttt{unstyledDir}
\textbar{} \texttt{File} \textbar{} \texttt{null} \textbar{} The
directory of
\href{https://github.com/liferay/liferay-portal/tree/master/modules/apps/frontend-theme/frontend-theme-unstyled}{Liferay
Frontend Theme Unstyled}. It sets the \texttt{-\/-unstyled-dir}
argument. \texttt{unstyledFile} \textbar{} \texttt{File} \textbar{}
\texttt{null} \textbar{} The JAR file of
\href{https://github.com/liferay/liferay-portal/tree/master/modules/apps/frontend-theme/frontend-theme-unstyled}{Liferay
Frontend Theme Unstyled}. If \texttt{unstyledDir} is specified, this
property has no effect. It sets the \texttt{-\/-unstyled-dir} argument.

The properties of type \texttt{File} support any type that can be
resolved by
\href{https://docs.gradle.org/current/dsl/org.gradle.api.Project.html\#org.gradle.api.Project:file(java.css.Object)}{\texttt{project.file}}.
Moreover, it is possible to use Closures and Callables as values for the
\texttt{String} properties to defer evaluation until task execution.

\section{Additional Configuration}\label{additional-configuration-14}

There are additional configurations that can help you use the CSS
Builder.

\section{Liferay Theme Builder
Dependency}\label{liferay-theme-builder-dependency}

By default, the plugin creates a configuration called
\texttt{themeBuilder} and adds a dependency to the latest released
version of the Liferay Theme Builder. It is possible to override this
setting and use a specific version of the tool by manually adding a
dependency to the \texttt{themeBuilder} configuration:

\begin{verbatim}
dependencies {
    themeBuilder group: "com.liferay", name: "com.liferay.portal.tools.theme.builder", version: "1.1.7"
}
\end{verbatim}

\section{Parent Theme Dependencies}\label{parent-theme-dependencies}

By default, the plugin creates a configuration called
\texttt{parentThemes} and adds dependencies to the latest released
versions of the
\href{https://github.com/liferay/liferay-portal/tree/master/modules/apps/frontend-theme/frontend-theme-styled}{Liferay
Frontend Theme Styled},
\href{https://github.com/liferay/liferay-portal/tree/master/modules/apps/frontend-theme/frontend-theme-unstyled}{Liferay
Frontend Theme Unstyled}, and
\href{https://github.com/liferay/liferay-portal/tree/master/modules/apps/frontend-theme/frontend-theme-classic}{Liferay
Frontend Theme Classic} artifacts. It is possible to override this
setting and use a specific version of the artifacts by manually adding
dependencies to the \texttt{parentThemes} configuration. For example,

\begin{verbatim}
dependencies {
    parentThemes group: "com.liferay", name: "com.liferay.frontend.theme.styled", version: "VERSION"
    parentThemes group: "com.liferay", name: "com.liferay.frontend.theme.unstyled", version: "VERSION"
    parentThemes group: "com.liferay.plugins", name: "classic-theme", version: "VERSION"
}
\end{verbatim}

Specifying dependency versions is not required when leveraging
workspace's
\href{/docs/7-2/reference/-/knowledge_base/r/managing-the-target-platform}{Target
Platform} functionality. All dependencies with the group ID
\texttt{com.liferay} or \texttt{com.liferay.portal} are automatically
set when targeting a platform. For external theme dependencies (e.g.,
\texttt{classic-theme} with the group ID \texttt{com.liferay.plugins}),
you can find the version used by your specific Liferay DXP instance by
leveraging the
\href{/docs/7-2/customization/-/knowledge_base/c/using-the-felix-gogo-shell}{Gogo
shell}. In a Gogo shell prompt, execute the following command:

\begin{verbatim}
lb -s theme
\end{verbatim}

This lists the deployed theme bundles and their versions. Extract the
versions for the theme dependencies you want to leverage and add them to
your configuration.

\chapter{TLD Formatter Gradle Plugin}\label{tld-formatter-gradle-plugin}

The TLD Formatter Gradle plugin lets you format a project's TLD files
using the
\href{https://github.com/liferay/liferay-portal/tree/master/modules/util/tld-formatter}{Liferay
TLD Formatter} tool.

The plugin has been successfully tested with Gradle 4.10.2.

\section{Usage}\label{usage-23}

To use the plugin, include it in your build script:

\begin{verbatim}
buildscript {
    dependencies {
        classpath group: "com.liferay", name: "com.liferay.gradle.plugins.tld.formatter", version: "1.0.9"
    }

    repositories {
        maven {
            url "https://repository-cdn.liferay.com/nexus/content/groups/public"
        }
    }
}

apply plugin: "com.liferay.tld.formatter"
\end{verbatim}

Since the plugin automatically resolves the Liferay TLD Formatter
library as a dependency, you have to configure a repository that hosts
the library and its transitive dependencies. The Liferay CDN repository
hosts them all:

\begin{verbatim}
repositories {
    maven {
        url "https://repository-cdn.liferay.com/nexus/content/groups/public"
    }
}
\end{verbatim}

\section{Tasks}\label{tasks-22}

The plugin adds one task to your project:

Name \textbar{} Depends On \textbar{} Type \textbar{} Description
\texttt{formatTLD} \textbar{} - \textbar{}
\hyperref[formattldtask]{\texttt{FormatTLDTask}} \textbar{} Runs the
Liferay TLD Formatter to format files.

\section{FormatTLDTask}\label{formattldtask}

Tasks of type \texttt{FormatTLDTask} extend
\href{https://docs.gradle.org/current/dsl/org.gradle.api.tasks.JavaExec.html}{\texttt{JavaExec}},
so all its properties and methods, such as
\href{https://docs.gradle.org/current/dsl/org.gradle.api.tasks.JavaExec.html\#org.gradle.api.tasks.JavaExec:args(java.lang.Iterable)}{\texttt{args}}
and
\href{https://docs.gradle.org/current/dsl/org.gradle.api.tasks.JavaExec.html\#org.gradle.api.tasks.JavaExec:maxHeapSize}{\texttt{maxHeapSize}},
are available. They also have the following properties set by default:

Property Name \textbar{} Default Value
\href{https://docs.gradle.org/current/dsl/org.gradle.api.tasks.JavaExec.html\#org.gradle.api.tasks.JavaExec:args}{\texttt{args}}
\textbar{} TLD Formatter command line arguments
\href{https://docs.gradle.org/current/dsl/org.gradle.api.tasks.JavaExec.html\#org.gradle.api.tasks.JavaExec:classpath}{\texttt{classpath}}
\textbar{}
\hyperref[liferay-tld-formatter-dependency]{\texttt{project.configurations.tldFormatter}}
\href{https://docs.gradle.org/current/dsl/org.gradle.api.tasks.JavaExec.html\#org.gradle.api.tasks.JavaExec:main}{\texttt{main}}
\textbar{} \texttt{"com.liferay.tld.formatter.TLDFormatter"}

\subsection{Task Properties}\label{task-properties-32}

Property Name \textbar{} Type \textbar{} Default Value \textbar{}
Description \texttt{plugin} \textbar{} \texttt{boolean} \textbar{}
\texttt{true} \textbar{} Whether to format all the TLD files contained
in the
\href{https://docs.gradle.org/current/dsl/org.gradle.api.tasks.JavaExec.html\#org.gradle.api.tasks.JavaExec:workingDir}{\texttt{workingDir}}
directory. If \texttt{false}, all \texttt{liferay-portlet-ext.tld} files
are ignored. It sets the \texttt{tld.plugin} argument.

\section{Additional Configuration}\label{additional-configuration-15}

There are additional configurations that can help you use the TLD
Formatter.

\section{Liferay TLD Formatter
Dependency}\label{liferay-tld-formatter-dependency}

By default, the plugin creates a configuration called
\texttt{tldFormatter} and adds a dependency to the latest released
version of Liferay TLD Formatter. It is possible to override this
setting and use a specific version of the tool by manually adding a
dependency to the \texttt{tldFormatter} configuration:

\begin{verbatim}
dependencies {
    tldFormatter group: "com.liferay", name: "com.liferay.tld.formatter", version: "1.0.5"
}
\end{verbatim}

\chapter{TLDDoc Builder Gradle
Plugin}\label{tlddoc-builder-gradle-plugin}

The TLDDoc Builder Gradle plugin lets you run the
\href{http://web.archive.org/web/20070624180825/https://taglibrarydoc.dev.java.net/}{Tag
Library Documentation Generator} tool in order to generate documentation
for the JSP Tag Library Descriptor (TLD) files in your project.

The plugin has been successfully tested with Gradle 4.10.2.

\section{Usage}\label{usage-24}

To use the plugin, include it in your build script:

\begin{verbatim}
buildscript {
    dependencies {
        classpath group: "com.liferay", name: "com.liferay.gradle.plugins.tlddoc.builder", version: "1.3.3"
    }

    repositories {
        maven {
            url "https://repository-cdn.liferay.com/nexus/content/groups/public"
        }
    }
}
\end{verbatim}

There are two TLDDoc Builder Gradle plugins you can apply to your
project:

\begin{itemize}
\item
  Apply the \hyperref[tlddoc-builder-plugin]{\emph{TLDDoc Builder
  Plugin}} to generate tag library documentation for your project:

\begin{verbatim}
apply plugin: "com.liferay.tlddoc.builder"
\end{verbatim}
\item
  Apply the \hyperref[app-tlddoc-builder-plugin]{\emph{App TLDDoc
  Builder Plugin}} in a parent project to generate the tag library
  documentation as a single, combined HTML document for an application
  that spans different subprojects, each one representing a different
  component of the same application:

\begin{verbatim}
apply plugin: "com.liferay.app.tlddoc.builder"
\end{verbatim}
\end{itemize}

Since the plugin automatically resolves the Tag Library Documentation
Generator library as a dependency, you must configure a repository that
hosts the library and its transitive dependencies. The Liferay CDN
repository hosts them all:

\begin{verbatim}
repositories {
    maven {
        url "https://repository-cdn.liferay.com/nexus/content/groups/public"
    }
}
\end{verbatim}

\section{TLDDoc Builder Plugin}\label{tlddoc-builder-plugin}

The plugin adds three tasks to your project:

Name \textbar{} Depends On \textbar{} Type \textbar{} Description
\texttt{copyTLDDocResources} \textbar{} - \textbar{}
\href{https://docs.gradle.org/current/dsl/org.gradle.api.tasks.Copy.html}{\texttt{Copy}}
\textbar{} Copies the tag library documentation resources from
\texttt{src/main/tlddoc} to the \hyperref[destinationdir]{destination
directory} of the \texttt{tlddoc} task. \texttt{tlddoc} \textbar{}
\texttt{copyTLDDocResources}, \texttt{validateTLD} \textbar{}
\hyperref[tlddoctask]{\texttt{TLDDocTask}} \textbar{} Generates the tag
library documentation. \texttt{validateTLD} \textbar{} - \textbar{}
\hyperref[validateschematask]{\texttt{ValidateSchemaTask}} \textbar{}
Validates the TLD files in the project.

The \texttt{tlddoc} task is automatically configured with sensible
defaults, depending on whether the
\href{https://docs.gradle.org/current/userguide/java_plugin.html}{\texttt{java}}
plugin is applied:

Property Name \textbar{} Default Value with the \texttt{java} plugin
\hyperref[destinationdir]{\texttt{destinationDir}} \textbar{}
\texttt{\$\{project.docsDir\}/tlddoc}
\hyperref[includes]{\texttt{includes}} \textbar{}
\texttt{{[}"**/*.tld"{]}} \hyperref[source]{\texttt{source}} \textbar{}
\texttt{project.sourceSets.main.resources.srcDirs}

The \texttt{validateTLD} task is also automatically configured with
sensible defaults, depending on whether the \texttt{java} plugin is
applied:

Property Name \textbar{} Default Value \texttt{includes} \textbar{}

\textbf{If the \texttt{java} plugin is applied:}
\texttt{{[}"**/*.tld"{]}}

\textbf{Otherwise:} \texttt{{[}{]}}

\texttt{source} \textbar{}

\textbf{If the \texttt{java} plugin is applied:}
\texttt{project.sourceSets.main.resources.srcDirs}

\textbf{Otherwise:} \texttt{null}

By default, the \texttt{tlddoc} task generates the documentation for all
the TLD files that are found in the resources directories of the
\texttt{main} source set. The documentation files are saved in
\texttt{build/docs/tlddoc} and include the files copied from
\texttt{src/main/tlddoc}.

The \texttt{copyTLDDocResources} task lets you add references to images
and other resources directly in the TLD files. For example, if the
project includes an image called \texttt{breadcrumb.png} in the
\texttt{src/main/tlddoc/images} directory, you can reference it in a TLD
file contained in the \texttt{src/main/resources} directory:

\begin{verbatim}
<description>Hello World <![CDATA[<img src="..](./images/breadcrumb.png"]]></description>
\end{verbatim}

\section{App TLDDoc Builder Plugin}\label{app-tlddoc-builder-plugin}

In order to use the App TLDDoc Builder plugin, it is required to apply
the \texttt{com.liferay.app.tlddoc.builder} plugin in a parent project
(that is, a project that is a common ancestor of all the subprojects
representing the various components of the app). It is also required to
apply the
\hyperref[tlddoc-builder-plugin]{\texttt{com.liferay.tlddoc.builder}}
plugin to all the subprojects that contain TLD files.

The App TLDDoc Builder plugin automatically applies the
\href{https://docs.gradle.org/current/userguide/standard_plugins.html\#N135C1}{\texttt{base}}
plugin. It also adds three tasks to your project:

Name \textbar{} Depends On \textbar{} Type \textbar{} Description
\texttt{appTLDDoc} \textbar{} \texttt{copyAppTLDDocResources}, the
\hyperref[validatetld]{\texttt{validateTLD}} tasks of the subprojects
\textbar{} \hyperref[tlddoctask]{\texttt{TLDDocTask}} \textbar{}
Generates tag library documentation for the app.
\texttt{copyAppTLDDocResources} \textbar{} - \textbar{}
\href{https://docs.gradle.org/current/dsl/org.gradle.api.tasks.Copy.html}{\texttt{Copy}}
\textbar{} Copies the tag library documentation resources defined as
\href{https://docs.gradle.org/current/javadoc/org/gradle/api/tasks/TaskInputs.html\#getFiles()}{inputs}
for the \hyperref[copytlddocresources]{\texttt{copyTDLDocResources}}
tasks of the subprojects, aggregating them into the
\hyperref[destinationdir]{destination directory} of the
\texttt{appTLDDoc} task. \texttt{jarAppTLDDoc} \textbar{}
\texttt{appTLDDoc} \textbar{}
\href{https://docs.gradle.org/current/dsl/org.gradle.api.tasks.bundling.Jar.html}{\texttt{Jar}}
\textbar{} Assembles a JAR archive containing the tag library
documentation files for this app.

The \texttt{appTLDDoc} task is automatically configured with sensible
defaults:

Property Name \textbar{} Default Value
\hyperref[destinationdir]{\texttt{destinationDir}} \textbar{}
\texttt{\$\{project.buildDir\}/docs/tlddoc}
\hyperref[source]{\texttt{source}} \textbar{} The sum of all the
\texttt{tlddoc.source} values of the subprojects

\section{Project Extension}\label{project-extension-8}

The App TLDDoc Builder plugin exposes the following properties through
the extension named \texttt{appTLDDocBuilder}:

Property Name \textbar{} Type \textbar{} Default Value \textbar{}
Description \texttt{subprojects} \textbar{}
\texttt{Set\textless{}Project\textgreater{}} \textbar{}
\texttt{project.subprojects} \textbar{} The subprojects to include in
the tag library documentation of the app.

The same extension exposes the following methods:

Method \textbar{} Description
\texttt{AppTLDDocBuilderExtension\ subprojects(Iterable\textless{}Project\textgreater{}\ subprojects)}
\textbar{} Include additional projects in the tag library documentation
of the app.
\texttt{AppTLDDocBuilderExtension\ subprojects(Project...\ subprojects)}
\textbar{} Include additional projects in the tag library documentation
of the app.

\section{Tasks}\label{tasks-23}

\section{TLDDocTask}\label{tlddoctask}

Tasks of type \texttt{TLDDocTask} extend
\href{https://docs.gradle.org/current/dsl/org.gradle.api.tasks.JavaExec.html}{\texttt{JavaExec}},
so all its properties and methods, such as
\href{https://docs.gradle.org/current/dsl/org.gradle.api.tasks.JavaExec.html\#org.gradle.api.tasks.JavaExec:args(java.tlddoc.Iterable)}{\texttt{args}}
and
\href{https://docs.gradle.org/current/dsl/org.gradle.api.tasks.JavaExec.html\#org.gradle.api.tasks.JavaExec:maxHeapSize}{\texttt{maxHeapSize}},
are available. They also have the following properties set by default:

Property Name \textbar{} Default Value
\href{https://docs.gradle.org/current/dsl/org.gradle.api.tasks.JavaExec.html\#org.gradle.api.tasks.JavaExec:args}{\texttt{args}}
\textbar{} Tag Library Documentation Generator command line arguments
\href{https://docs.gradle.org/current/dsl/org.gradle.api.tasks.JavaExec.html\#org.gradle.api.tasks.JavaExec:classpath}{\texttt{classpath}}
\textbar{}
\hyperref[tag-library-documentation-generator-dependency]{\texttt{project.configurations.tlddoc}}
\href{https://docs.gradle.org/current/dsl/org.gradle.api.tasks.JavaExec.html\#org.gradle.api.tasks.JavaExec:main}{\texttt{main}}
\textbar{} \texttt{"com.sun.tlddoc.TLDDoc"}
\href{https://docs.gradle.org/current/dsl/org.gradle.api.tasks.JavaExec.html\#org.gradle.api.tasks.JavaExec:maxHeapSize}{\texttt{maxHeapSize}}
\textbar{} \texttt{"256m"}

The \texttt{TLDDocTask} class is also very similar to
\href{https://docs.gradle.org/current/dsl/org.gradle.api.tasks.SourceTask.html}{\texttt{SourceTask}},
which means it provides a \texttt{source} property and lets you specify
include and exclude patterns.

\subsection{Task Properties}\label{task-properties-33}

Property Name \textbar{} Type \textbar{} Default Value \textbar{}
Description \texttt{destinationDir} \textbar{} \texttt{File} \textbar{}
\texttt{null} \textbar{} The directory where the tag library
documentation files are saved. \texttt{excludes} \textbar{}
\texttt{Set\textless{}String\textgreater{}} \textbar{} \texttt{{[}{]}}
\textbar{} The TLD file patterns to exclude. \texttt{includes}
\textbar{} \texttt{Set\textless{}String\textgreater{}} \textbar{}
\texttt{{[}{]}} \textbar{} The TLD file patterns to include.
\texttt{source} \textbar{}
\href{https://docs.gradle.org/current/javadoc/org/gradle/api/file/FileTree.html}{\texttt{FileTree}}
\textbar{} \texttt{{[}{]}} \textbar{} The TLD files to generate
documentation for, after the include and exclude patterns have been
applied. \texttt{xsltDir} \textbar{} \texttt{File} \textbar{}
\texttt{null} \textbar{} The directory that contains the custom XSLT
stylesheets used by the Tag Library Documentation Generator to produce
the final documentation files. It sets the \texttt{-xslt} argument.

The properties of type \texttt{File} support any type that can be
resolved by
\href{https://docs.gradle.org/current/dsl/org.gradle.api.Project.html\#org.gradle.api.Project:file(java.tlddoc.Object)}{\texttt{project.file}}.

\subsection{Task Methods}\label{task-methods-11}

The methods available for \texttt{TLDDocTask} are exactly the same as
the one defined in the
\href{https://docs.gradle.org/current/dsl/org.gradle.api.tasks.SourceTask.html}{\texttt{SourceTask}}
class.

\section{ValidateSchemaTask}\label{validateschematask}

Tasks of type \texttt{ValidateSchemaTask} extend
\href{https://docs.gradle.org/current/dsl/org.gradle.api.tasks.SourceTask.html}{\texttt{SourceTask}},
so all its properties and methods, such as
\href{https://docs.gradle.org/current/dsl/org.gradle.api.tasks.SourceTask.html\#org.gradle.api.tasks.SourceTask:include(java.lang.Iterable)}{\texttt{include}}
and
\href{https://docs.gradle.org/current/dsl/org.gradle.api.tasks.SourceTask.html\#org.gradle.api.tasks.SourceTask:exclude(java.lang.Iterable)}{\texttt{exclude}},
are available.

Tasks of this type invoke the
\href{http://ant.apache.org/manual/Tasks/schemavalidate.html}{\texttt{schemavalidate}}
Ant task in order to validate XML files described by an XML schema.

\subsection{Task Properties}\label{task-properties-34}

Property Name \textbar{} Type \textbar{} Default Value \textbar{}
Description \texttt{dtdDisabled} \textbar{} \texttt{boolean} \textbar{}
\texttt{false} \textbar{} Whether to disable DTD support.
\texttt{fullChecking} \textbar{} \texttt{boolean} \textbar{}
\texttt{true} \textbar{} Whether to enable full schema checking.
\texttt{lenient} \textbar{} \texttt{boolean} \textbar{} \texttt{false}
\textbar{} Whether to only check if the XML document is well-formed.
\texttt{xmlParserClassName} \textbar{} \texttt{String} \textbar{}
\texttt{null} \textbar{} The class name of the XML parser to use.
\texttt{xmlParserClasspath} \textbar{} \texttt{FileCollection}
\textbar{} \texttt{null} \textbar{} The classpath with the XML parser.

It is possible to use Closures and Callables as values for the
\texttt{String} properties to defer evaluation until task execution.

\section{Additional Configuration}\label{additional-configuration-16}

There are additional configurations that can help you use the TLDDoc
Builder.

\section{Tag Library Documentation Generator
Dependency}\label{tag-library-documentation-generator-dependency}

By default, the plugin creates a configuration called \texttt{tlddoc}
and adds a dependency to the 1.3 version of the Tag Library
Documentation Generator. It is possible to override this setting and use
a specific version of the tool by manually adding a dependency to the
\texttt{tlddoc} configuration:

\begin{verbatim}
dependencies {
    tlddoc group: "taglibrarydoc", name: "tlddoc", version: "1.3"
}
\end{verbatim}

\chapter{Whip Gradle Plugin}\label{whip-gradle-plugin}

The Whip Gradle plugin lets you use the
\href{https://github.com/liferay/liferay-portal/tree/master/modules/test/whip}{Liferay
Whip} library to ensure that unit tests fully cover your project's code.
See
\href{https://github.com/liferay/liferay-portal/tree/master/modules/sdk/gradle-plugins-whip/src/gradleTest/smoke}{here}
for a usage sample.

The plugin has been successfully tested with Gradle 4.10.2.

\section{Usage}\label{usage-25}

To use the plugin, include it in your build script:

\begin{verbatim}
buildscript {
    dependencies {
        classpath group: "com.liferay", name: "com.liferay.gradle.plugins.whip", version: "1.0.7"
    }

    repositories {
        maven {
            url "https://repository-cdn.liferay.com/nexus/content/groups/public"
        }
    }
}

apply plugin: "com.liferay.whip"
\end{verbatim}

Since the plugin automatically resolves the Liferay Whip library as a
dependency, you have to configure a repository that hosts the library
and its transitive dependencies. The Liferay CDN repository hosts them
all:

\begin{verbatim}
repositories {
    maven {
        url "https://repository-cdn.liferay.com/nexus/content/groups/public"
    }
}
\end{verbatim}

By default, Whip is automatically applied to all tasks of type
\href{https://docs.gradle.org/current/javadoc/org/gradle/api/tasks/testing/Test.html}{\texttt{Test}}.
If a task has Whip applied and Whip is \hyperref[enabled]{enabled}, then
Whip is configured as a Java Agent.

\section{Project Extension}\label{project-extension-9}

The Whip Gradle plugin exposes the following properties through the
extension named \texttt{whip}:

Property Name \textbar{} Type \textbar{} Default Value \textbar{}
Description \texttt{version} \textbar{} \texttt{String} \textbar{}
\texttt{latest.release} \textbar{} The version of the Liferay Whip
library to use.

The same extension exposes the following methods:

Method \textbar{} Description \texttt{void\ applyTo(Task\ task)}
\textbar{} Applies Whip to a task. The task instance must implement the
\href{https://docs.gradle.org/current/javadoc/org/gradle/process/JavaForkOptions.html}{\texttt{JavaForkOptions}}
interface.

\section{Task Extension}\label{task-extension}

If Whip is applied, the following task properties are available through
the extension named \texttt{whip}:

Property Name \textbar{} Type \textbar{} Default Value \textbar{}
Description \texttt{dataFile} \textbar{} \texttt{File} \textbar{}
\texttt{test-coverage/whip.dat} \textbar{} \texttt{enabled} \textbar{}
\texttt{boolean} \textbar{} \texttt{true} \textbar{} Whether to
configure Whip as a Java Agent. \texttt{excludes} \textbar{}
\texttt{List\textless{}String\textgreater{}} \textbar{} \texttt{{[}{]}}
\textbar{} The class name patterns to exclude when checking for unit
test code coverage. For example, a value could be
\texttt{{[}\textquotesingle{}.*Test\textquotesingle{},\ \textquotesingle{}.*Test\textbackslash{}\textbackslash{}\$.*\textquotesingle{},\ \textquotesingle{}.*\textbackslash{}\textbackslash{}\$Proxy.*\textquotesingle{},\ \textquotesingle{}com/liferay/whip/.*\textquotesingle{}{]}}.
\texttt{includes} \textbar{}
\texttt{List\textless{}String\textgreater{}} \textbar{} \texttt{{[}{]}}
\textbar{} The class name patterns to include when checking for unit
test code coverage. \texttt{instrumentDump} \textbar{} \texttt{boolean}
\textbar{} \texttt{false} \textbar{} \texttt{whipJarFile} \textbar{}
\texttt{File} \textbar{} The first file in the \texttt{whip}
configuration whose name starts with \texttt{com.liferay.whip-}.
\textbar{} The Whip JAR file.

The same extension exposes the following methods:

Method \textbar{} Description
\texttt{WhipTaskExtension\ excludes(Iterable\textless{}Object\textgreater{}\ excludes)}
\textbar{} Adds class name patterns to exclude when checking for unit
test coverage. \texttt{WhipTaskExtension\ excludes(Object...\ excludes)}
\textbar{} Adds class name patterns to exclude when checking for unit
test coverage.
\texttt{WhipTaskExtension\ includes(Iterable\textless{}Object\textgreater{}\ includes)}
\textbar{} Adds class name patterns to include when checking for unit
test coverage. \texttt{WhipTaskExtension\ includes(Object...\ includes)}
\textbar{} Adds class name patterns to include when checking for unit
test coverage.

\section{Additional Configuration}\label{additional-configuration-17}

There are additional configurations that can help you use Whip.

\section{Liferay Whip Dependency}\label{liferay-whip-dependency}

By default, the Whip Gradle plugin creates a configuration called
\texttt{whip} and adds a dependency to the version of Liferay Whip
configured in the \hyperref[version]{\texttt{whip.version}} extension
property. It is possible to override this setting and use a specific
version of the library by manually adding a dependency to the
\texttt{whip} configuration:

\begin{verbatim}
dependencies {
    whip group: "com.liferay", name: "com.liferay.whip", version: "1.0.1"
}
\end{verbatim}

In order to leverage the sensible default of the
\hyperref[whipjarfile]{\texttt{whip.whipJarFile}} task property, the
name of the dependency must be \texttt{com.liferay.whip}. Otherwise, it
will be necessary to set the value of the \texttt{whip.whipJarFile}
property manually.

\chapter{WSDD Builder Gradle Plugin}\label{wsdd-builder-gradle-plugin}

The WSDD Builder Gradle plugin lets you run the
\href{https://github.com/liferay/liferay-portal/tree/master/modules/util/portal-tools-wsdd-builder}{Liferay
WSDD Builder} tool to generate the
\href{http://axis.apache.org/axis/}{Apache Axis} Web Service Deployment
Descriptor (WSDD) files from a
\href{/docs/7-2/appdev/-/knowledge_base/a/service-builder}{Service
Builder} \texttt{service.xml} file.

The plugin has been successfully tested with Gradle 4.10.2.

\section{Usage}\label{usage-26}

To use the plugin, include it in your build script:

\begin{verbatim}
buildscript {
    dependencies {
        classpath group: "com.liferay", name: "com.liferay.gradle.plugins.wsdd.builder", version: "1.0.13"
    }

    repositories {
        maven {
            url "https://repository-cdn.liferay.com/nexus/content/groups/public"
        }
    }
}

apply plugin: "com.liferay.portal.tools.wsdd.builder"
\end{verbatim}

The WSDD Builder plugin automatically applies the
\href{https://docs.gradle.org/current/userguide/java_plugin.html}{\texttt{java}}
plugin.

Since the plugin automatically resolves the Liferay WSDD Builder library
as a dependency, you have to configure a repository that hosts the
library and its transitive dependencies. The Liferay CDN repository
hosts them all:

\begin{verbatim}
repositories {
    maven {
        url "https://repository-cdn.liferay.com/nexus/content/groups/public"
    }
}
\end{verbatim}

\section{Tasks}\label{tasks-24}

The plugin adds one task to your project:

Name \textbar{} Depends On \textbar{} Type \textbar{} Description
\texttt{buildWSDD} \textbar{}
\href{https://docs.gradle.org/current/userguide/java_plugin.html\#sec:compile}{\texttt{compileJava}}
\textbar{} \hyperref[buildwsddtask]{\texttt{BuildWSDDTask}} \textbar{}
Runs the Liferay WSDD Builder.

By default, the \texttt{buildWSDD} task uses the
\texttt{\$\{project.projectDir\}/service.xml} file as input. Then, it
generates \texttt{\$\{project.projectDir\}/server-config.wsdd} and the
\texttt{*\_deploy.wsdd} and \texttt{*\_undeploy.wsdd} files in the first
\href{https://docs.gradle.org/current/dsl/org.gradle.api.tasks.SourceSet.html\#org.gradle.api.tasks.SourceSet:resources}{\texttt{resources}}
directory of the \texttt{main}
\href{https://docs.gradle.org/current/userguide/java_plugin.html\#N1503E}{source
set} (by default: \texttt{src/main/resources}).

If the
\href{https://docs.gradle.org/current/userguide/war_plugin.html}{\texttt{war}}
plugin is applied, the task uses
\texttt{\$\{project.webAppDir\}/WEB-INF/service.xml} as input to
generate \texttt{\$\{project.webAppDir\}/WEB-INF/server-config.wsdd}.
The \texttt{*\_deploy.wsdd} and \texttt{*\_undeploy.wsdd} files are
still generated in the first \texttt{resources} directory of the
\texttt{main} source set.

Liferay WSDD Build Service requires an additional classpath (configured
with the \texttt{buildWSDD.builderClasspath} property), to correctly
generate the WSDD files. The \texttt{buildWSDD} task uses the following
default value, which creates an implicit dependency to the
\texttt{compileJava} task:

\begin{verbatim}
tasks.compileJava.outputs.files + sourceSets.main.compileClasspath + sourceSets.main.runtimeClasspath
\end{verbatim}

\section{BuildWSDDTask}\label{buildwsddtask}

Tasks of type \texttt{BuildWSDDTask} extend
\href{https://docs.gradle.org/current/dsl/org.gradle.api.tasks.JavaExec.html}{\texttt{JavaExec}},
so all its properties and methods, such as
\href{https://docs.gradle.org/current/dsl/org.gradle.api.tasks.JavaExec.html\#org.gradle.api.tasks.JavaExec:args(java.lang.Iterable)}{\texttt{args}}
and
\href{https://docs.gradle.org/current/dsl/org.gradle.api.tasks.JavaExec.html\#org.gradle.api.tasks.JavaExec:maxHeapSize}{\texttt{maxHeapSize}},
are available. They also have the following properties set by default:

Property Name \textbar{} Default Value
\href{https://docs.gradle.org/current/dsl/org.gradle.api.tasks.JavaExec.html\#org.gradle.api.tasks.JavaExec:args}{\texttt{args}}
\textbar{} WSDD Builder command line arguments
\href{https://docs.gradle.org/current/dsl/org.gradle.api.tasks.JavaExec.html\#org.gradle.api.tasks.JavaExec:classpath}{\texttt{classpath}}
\textbar{}
\hyperref[liferay-wsdd-builder-dependency]{\texttt{project.configurations.wsddBuilder}}
\href{https://docs.gradle.org/current/dsl/org.gradle.api.tasks.JavaExec.html\#org.gradle.api.tasks.JavaExec:main}{\texttt{main}}
\textbar{} \texttt{"com.liferay.portal.tools.wsdd.builder.WSDDBuilder"}

\subsection{Task Properties}\label{task-properties-35}

Property Name \textbar{} Type \textbar{} Default Value \textbar{}
Description \texttt{builderClasspath} \textbar{} \texttt{String}
\textbar{} \texttt{null} \textbar{} A classpath that the Liferay WSDD
Builder uses to generate WSDD files. It sets the
\texttt{wsdd.class.path} argument. \texttt{inputFile} \textbar{}
\texttt{File} \textbar{} \texttt{null} \textbar{} A \texttt{service.xml}
from which to generate the WSDD files. It sets the
\texttt{wsdd.input.file} argument. \texttt{outputDir} \textbar{}
\texttt{File} \textbar{} \texttt{null} \textbar{} A directory where the
\texttt{*\_deploy.wsdd} and \texttt{*\_undeploy.wsdd} files are
generated. It sets the \texttt{wsdd.output.path} argument.
\texttt{serverConfigFile} \textbar{} \texttt{File} \textbar{}
\texttt{\$\{project.projectDir\}/server-config.wsdd} \textbar{} A
\texttt{server-config.wsdd} file to generate. It sets the
\texttt{wsdd.server.config.file} argument. \texttt{serviceNamespace}
\textbar{} \texttt{String} \textbar{} \texttt{"Plugin"} \textbar{} A
namespace for the WSDD Service. It sets the
\texttt{wsdd.service.namespace} argument.

The properties of type \texttt{File} support any type that can be
resolved by
\href{https://docs.gradle.org/current/dsl/org.gradle.api.Project.html\#org.gradle.api.Project:file(java.lang.Object)}{\texttt{project.file}}.
Moreover, it is possible to use Closures and Callables as values for the
\texttt{String} properties, to defer evaluation until task execution.

\section{Additional Configuration}\label{additional-configuration-18}

There are additional configurations that can help you use the WSDD
Builder.

\section{Liferay WSDD Builder
Dependency}\label{liferay-wsdd-builder-dependency}

By default, the plugin creates a configuration called
\texttt{wsddBuilder} and adds a dependency to the latest released
version of the Liferay WSDD Builder. It is possible to override this
setting and use a specific version of the tool by manually adding a
dependency to the \texttt{wsddBuilder} configuration:

\begin{verbatim}
dependencies {
    wsddBuilder group: "com.liferay", name: "com.liferay.portal.tools.wsdd.builder", version: "1.0.10"
}
\end{verbatim}

\chapter{WSDL Builder Gradle Plugin}\label{wsdl-builder-gradle-plugin}

The WSDL Builder Gradle plugin lets you generate
\href{http://axis.apache.org/axis/}{Apache Axis} client stubs from Web
Service Description (WSDL) files.

The plugin has been successfully tested with Gradle 4.10.2.

\section{Usage}\label{usage-27}

To use the plugin, include it in your build script:

\begin{verbatim}
buildscript {
    dependencies {
        classpath group: "com.liferay", name: "com.liferay.gradle.plugins.wsdl.builder", version: "2.0.3"
    }

    repositories {
        maven {
            url "https://repository-cdn.liferay.com/nexus/content/groups/public"
        }
    }
}

apply plugin: "com.liferay.wsdl.builder"
\end{verbatim}

The WSDL Builder plugin automatically applies the
\href{https://docs.gradle.org/current/userguide/java_plugin.html}{\texttt{java}}
plugin.

Since the plugin automatically resolves the Apache Axis library as a
dependency, you have to configure a repository that hosts the library
and its transitive dependencies. The Liferay CDN repository hosts them
all:

\begin{verbatim}
repositories {
    maven {
        url "https://repository-cdn.liferay.com/nexus/content/groups/public"
    }
}
\end{verbatim}

\section{Tasks}\label{tasks-25}

The plugin adds one main task to your project:

Name \textbar{} Depends On \textbar{} Type \textbar{} Description
\texttt{buildWSDL} \textbar{} - \textbar{}
\hyperref[buildwsdltask]{\texttt{BuildWSDLTask}} \textbar{} Generates
WSDL client stubs.

By default, the \texttt{buildWSDL} task looks for WSDL files in the
\texttt{\$\{project.projectDir\}/wsdl} directory. If the
\href{https://docs.gradle.org/current/userguide/war_plugin.html}{\texttt{war}}
plugin is applied, it looks in the
\texttt{\$\{project.webAppDir\}/WEB-INF/wsdl} directory.

For each WSDL file that can be found, the task generates client stubs
via direct invocation of the
\href{http://axis.apache.org/axis/java/user-guide.html\#Client-side_bindings}{\emph{WSDL2Java}}
tool, saving them in the first
\href{https://docs.gradle.org/current/dsl/org.gradle.api.tasks.SourceSet.html\#org.gradle.api.tasks.SourceSet:java}{\texttt{java}}
directory of the \texttt{main}
\href{https://docs.gradle.org/current/userguide/java_plugin.html\#N1503E}{source
set} (by default: \texttt{src/main/java}).

If configured to do so, \texttt{buildWSDL} can instead save the client
stub Java files in a temporary directory, compile them, and package them
in JAR files. The JAR files are named after the WSDL file and saved in
\texttt{\$\{project.projectDir\}/lib}, by default, or in
\texttt{\$\{project.webAppDir\}/WEB-INF/lib}, if the \texttt{war} plugin
is applied.

\section{BuildWSDLTask}\label{buildwsdltask}

Tasks of type \texttt{FormatWSDLTask} extend
\href{https://docs.gradle.org/current/dsl/org.gradle.api.tasks.SourceTask.html}{\texttt{SourceTask}},
so all its properties and methods, such as
\href{https://docs.gradle.org/current/dsl/org.gradle.api.tasks.SourceTask.html\#org.gradle.api.tasks.SourceTask:include(java.lang.Iterable)}{\texttt{include}}
and
\href{https://docs.gradle.org/current/dsl/org.gradle.api.tasks.SourceTask.html\#org.gradle.api.tasks.SourceTask:exclude(java.lang.Iterable)}{\texttt{exclude}},
are available.

\subsection{Task Properties}\label{task-properties-36}

Property Name \textbar{} Type \textbar{} Default Value \textbar{}
Description \texttt{buildLibs} \textbar{} \texttt{boolean} \textbar{}
\texttt{true} \textbar{} Whether to package the client stub classes of
each WSDL file in JAR files, saved to the directory the
\texttt{destinationDir} property references. If \texttt{false}, the task
generates the client stub Java files to the \texttt{destinationDir}
directory. \texttt{destinationDir} \textbar{} \texttt{File} \textbar{}
\texttt{null} \textbar{} A directory where the client stub Java files
(if \texttt{buildLibs} is \texttt{false}) or the client stub JAR files
(if \texttt{buildLibs} is \texttt{true}) are saved.
\texttt{generateOptions.mapping} \textbar{} \texttt{Map} \textbar{}
\texttt{{[}:{]}} \textbar{} Namespace-to-package mappings (sets the
\texttt{-\/-NStoPkg} argument in the \emph{WSDL2Java} invocation). It is
possible to use a \texttt{Closure} or a \texttt{Callable}, to defer
evaluation until task execution.. \texttt{generateOptions.noWrapped}
\textbar{} \texttt{boolean} \textbar{} \texttt{false} \textbar{} Whether
to turn off support for ``wrapped'' document/literal (sets the
\texttt{-\/-noWrapped} argument in the \emph{WSDL2Java} invocation).
\texttt{generateOptions.serverSide} \textbar{} \texttt{boolean}
\textbar{} \texttt{false} \textbar{} Whether to emit server-side
bindings for the web service (sets the \texttt{-\/-server-side} argument
in the \emph{WSDL2Java} invocation). \texttt{generateOptions.verbose}
\textbar{} \texttt{boolean} \textbar{} \texttt{false} \textbar{} Whether
to print informational messages (sets the \texttt{-\/-verbose} argument
in the \emph{WSDL2Java} invocation). \texttt{includeSource} \textbar{}
\texttt{boolean} \textbar{} \texttt{true} \textbar{} Whether to package
the client stub Java files in the JAR file's \texttt{OSGI-OPT/src}
directory. If \texttt{buildLibs} is \texttt{false}, this property has no
effect. \texttt{includeWSDLs} \textbar{} \texttt{boolean} \textbar{}
\texttt{true} \textbar{} Whether to configure the
\href{https://docs.gradle.org/current/userguide/java_plugin.html\#sec:resources}{\texttt{processResources}}
task to include the WSDL files in the project JAR's \texttt{wsdl}
directory.

The properties of type \texttt{File} support any type that can be
resolved by
\href{https://docs.gradle.org/current/dsl/org.gradle.api.Project.html\#org.gradle.api.Project:file(java.lang.Object)}{\texttt{project.file}}.

\subsection{Task Methods}\label{task-methods-12}

Method Signature \textbar{} Description
\texttt{generateOptions.mapping(Object\ namespace,\ Object\ packageName)}
\textbar{} Adds a namespace-to-package mapping.
\texttt{generateOptions.mappings(Map\ mappings)} \textbar{} Adds
multiple namespace-to-package mappings.

\subsection{Helper Tasks}\label{helper-tasks-1}

At the end of the
\href{https://docs.gradle.org/current/userguide/build_lifecycle.html\#N11BAE}{project
evaluation}, a series of helper tasks are created for each WSDL file
returned by the
\href{https://docs.gradle.org/current/dsl/org.gradle.api.tasks.SourceTask.html\#org.gradle.api.tasks.SourceTask:source}{\texttt{source}}
property of the \texttt{BuildWSDLTask} tasks. The names of the helper
tasks start with the WSDL file name, without any extension.

\begin{itemize}
\tightlist
\item
  \texttt{\$\{WSDL\ file\ title\}Generate} of type
  \href{https://docs.gradle.org/current/dsl/org.gradle.api.tasks.JavaExec.html}{\texttt{JavaExec}}:
  invokes
  \href{https://axis.apache.org/axis/java/reference.html\#WSDL2Java_Reference}{\emph{WSDL2Java}}
  to generate the client stubs for the WSDL file.
\end{itemize}

If \texttt{buildWSDLTask.buildLibs} is \texttt{true}, the following
helper tasks are also created:

\begin{itemize}
\tightlist
\item
  \texttt{\$\{WSDL\ file\ title\}Compile} of type
  \href{https://docs.gradle.org/current/dsl/org.gradle.api.tasks.compile.JavaCompile.html}{\texttt{JavaCompile}}:
  compiles the client stub Java files for the WSDL file.
\item
  \texttt{\$\{WSDL\ file\ title\}Jar} of type
  \href{https://docs.gradle.org/current/dsl/org.gradle.api.tasks.bundling.Jar.html}{\texttt{Jar}}:
  packages in a JAR file called \texttt{\$\{WSDL\ file\ title\}-ws.jar},
  the client stub for the WSDL file.
\end{itemize}

\section{Additional Configuration}\label{additional-configuration-19}

There are additional configurations that can help you use WSDL Builder.

\section{Apache Axis Dependency}\label{apache-axis-dependency}

By default, the plugin creates a configuration called
\texttt{wsdlBuilder} and adds the following dependencies:

\begin{itemize}
\tightlist
\item
  \texttt{axis:axis-wsdl4j:1.5.1}
\item
  \texttt{com.liferay:org.apache.axis:1.4.LIFERAY-PATCHED-1}
\item
  \texttt{commons-discovery:commons-discovery:0.2}
\item
  \texttt{commons-logging:commons-logging:1.0.4}
\item
  \texttt{javax.activation:activation:1.1}
\item
  \texttt{javax.mail:mail:1.4}
\item
  \texttt{org.apache.axis:axis-jaxrpc:1.4}
\item
  \texttt{org.apache.axis:axis-saaj:1.4}
\end{itemize}

It is possible to override this setting and use a specific version of
Apache Axis, by manually populating the \texttt{wsdlBuilder}
configuration with the desired dependencies.

\chapter{XML Formatter Gradle Plugin}\label{xml-formatter-gradle-plugin}

The XML Formatter Gradle plugin lets you format a project's XML files
using the
\href{https://github.com/liferay/liferay-portal/tree/master/modules/util/xml-formatter}{Liferay
XML Formatter} tool.

The plugin has been successfully tested with Gradle 4.10.2.

\section{Usage}\label{usage-28}

To use the plugin, include it in your build script:

\begin{verbatim}
buildscript {
    dependencies {
        classpath group: "com.liferay", name: "com.liferay.gradle.plugins.xml.formatter", version: "1.0.11"
    }

    repositories {
        maven {
            url "https://repository-cdn.liferay.com/nexus/content/groups/public"
        }
    }
}

apply plugin: "com.liferay.xml.formatter"
\end{verbatim}

Since the plugin automatically resolves the Liferay XML Formatter
library as a dependency, you have to configure a repository that hosts
the library and its transitive dependencies. The Liferay CDN repository
hosts them all:

\begin{verbatim}
repositories {
    maven {
        url "https://repository-cdn.liferay.com/nexus/content/groups/public"
    }
}
\end{verbatim}

\section{Tasks}\label{tasks-26}

The plugin adds one task to your project:

Name \textbar{} Depends On \textbar{} Type \textbar{} Description
\texttt{formatXML} \textbar{} - \textbar{}
\hyperref[formatxmltask]{\texttt{FormatXMLTask}} \textbar{} Runs the
Liferay XML Formatter to format the project files.

If the
\href{https://docs.gradle.org/current/userguide/java_plugin.html}{\texttt{java}}
plugin is applied, the task formats XML files contained in the
\href{https://docs.gradle.org/current/dsl/org.gradle.api.tasks.SourceSet.html\#org.gradle.api.tasks.SourceSet:resources}{\texttt{resources}}
directories of the \texttt{main}
\href{https://docs.gradle.org/current/userguide/java_plugin.html\#N1503E}{source
set} (by default: \texttt{src/main/resources/**/*.xml}).

\section{FormatXMLTask}\label{formatxmltask}

Tasks of type \texttt{FormatXMLTask} extend
\href{https://docs.gradle.org/current/dsl/org.gradle.api.tasks.SourceTask.html}{\texttt{SourceTask}},
so all its properties and methods, such as
\href{https://docs.gradle.org/current/dsl/org.gradle.api.tasks.SourceTask.html\#org.gradle.api.tasks.SourceTask:include(java.lang.Iterable)}{\texttt{include}}
and
\href{https://docs.gradle.org/current/dsl/org.gradle.api.tasks.SourceTask.html\#org.gradle.api.tasks.SourceTask:exclude(java.lang.Iterable)}{\texttt{exclude}},
are available.

\subsection{Task Properties}\label{task-properties-37}

Property Name \textbar{} Type \textbar{} Default Value \textbar{}
Description \texttt{classpath} \textbar{}
\href{https://docs.gradle.org/current/javadoc/org/gradle/api/file/FileCollection.html}{\texttt{FileCollection}}
\textbar{}
\hyperref[liferay-xml-formatter-dependency]{\texttt{project.configurations.xmlFormatter}}
\textbar{} The classpath for executing the main class.
\texttt{mainClassName} \textbar{} \texttt{String} \textbar{}
\texttt{"com.liferay.xml.formatter.XMLFormatter"} \textbar{} The fully
qualified name of the XML Formatter Main class. \texttt{stripComments}
\textbar{} \texttt{boolean} \textbar{} \texttt{false} \textbar{} Whether
to remove all the comments from the XML files. It sets the
\texttt{xml.formatter.strip.comments} argument.

\section{Additional Configuration}\label{additional-configuration-20}

There are additional configurations that can help you use the XML
Formatter.

\section{Liferay XML Formatter
Dependency}\label{liferay-xml-formatter-dependency}

By default, the plugin creates a configuration called
\texttt{xmlFormatter} and adds a dependency to the latest released
version of the Liferay XML Formatter. It is possible to override this
setting and use a specific version of the tool by manually adding a
dependency to the \texttt{xmlFormatter} configuration:

\begin{verbatim}
dependencies {
    xmlFormatter group: "com.liferay", name: "com.liferay.xml.formatter", version: "1.0.5"
}
\end{verbatim}

\chapter{XSD Builder Gradle Plugin}\label{xsd-builder-gradle-plugin}

The XSD Builder Gradle plugin lets you generate
\href{https://xmlbeans.apache.org/}{Apache XMLBeans} bindings from XML
Schema (XSD) files.

The plugin has been successfully tested with Gradle 4.10.2.

\section{Usage}\label{usage-29}

To use the plugin, include it in your build script:

\begin{verbatim}
buildscript {
    dependencies {
        classpath group: "com.liferay", name: "com.liferay.gradle.plugins.xsd.builder", version: "1.0.7"
    }

    repositories {
        maven {
            url "https://repository-cdn.liferay.com/nexus/content/groups/public"
        }
    }
}

apply plugin: "com.liferay.xsd.builder"
\end{verbatim}

The XSD Builder plugin automatically applies the
\href{https://docs.gradle.org/current/userguide/java_plugin.html}{\texttt{java}}
plugin.

Since the plugin automatically resolves the Liferay Service Builder
library as a dependency, you have to configure a repository that hosts
the library and its transitive dependencies. The Liferay CDN repository
hosts them all:

\begin{verbatim}
repositories {
    maven {
        url "https://repository-cdn.liferay.com/nexus/content/groups/public"
    }
}
\end{verbatim}

\section{Tasks}\label{tasks-27}

The plugin adds three tasks to your project:

Name \textbar{} Depends On \textbar{} Type \textbar{} Description
\texttt{buildXSD} \textbar{} \texttt{buildXSDCompile} \textbar{}
\hyperref[buildxsdtask]{\texttt{BuildXSDTask}} \textbar{} Generates
XMLBeans bindings and compiles them in a JAR file.
\texttt{buildXSDGenerate} \textbar{} \texttt{cleanBuildXSDGenerate}
\textbar{}
\href{https://docs.gradle.org/current/dsl/org.gradle.api.tasks.JavaExec.html}{\texttt{JavaExec}}
\textbar{} Invokes the
\href{https://xmlbeans.apache.org/docs/2.6.0/guide/tools.html\#scomp}{XMLBeans
Schema Compiler} to generate Java types from XML Schema.
\texttt{buildXSDCompile} \textbar{} \texttt{buildXSDGenerate},
\texttt{cleanBuildXSDCompile} \textbar{}
\href{https://docs.gradle.org/current/dsl/org.gradle.api.tasks.compile.JavaCompile.html}{\texttt{JavaCompile}}
\textbar{} Compiles the generated Java types.

By default, the \texttt{buildXSD} task looks for XSD files in the
\texttt{\$\{project.projectDir\}/xsd} directory, and saves the generated
JAR file as
\texttt{\$\{project.projectDir\}/lib/\$\{project.archivesBaseName\}-xbean.jar}.

If the
\href{https://docs.gradle.org/current/userguide/war_plugin.html}{\texttt{war}}
plugin is applied, the task looks for XSD files in the
\texttt{\$\{project.webAppDir\}/WEB-INF/xsd} directory, and saves the
generated JAR file as
\texttt{\$\{project.webAppDir\}/WEB-INF/lib/\$\{project.archivesBaseName\}-xbean.jar}.

\section{BuildXSDTask}\label{buildxsdtask}

Tasks of type \texttt{BuildXSDTask} extend
\href{https://docs.gradle.org/current/dsl/org.gradle.api.tasks.bundling.Zip.html}{\texttt{Zip}}.
They also have the following properties set by default:

Property Name \textbar{} Default Value
\href{https://docs.gradle.org/current/dsl/org.gradle.api.tasks.bundling.Zip.html\#org.gradle.api.tasks.bundling.Zip:appendix}{\texttt{appendix}}
\textbar{} \texttt{"xbean"}
\href{https://docs.gradle.org/current/dsl/org.gradle.api.tasks.bundling.Zip.html\#org.gradle.api.tasks.bundling.Zip:extension}{\texttt{extension}}
\textbar{} \texttt{"jar"}
\href{https://docs.gradle.org/current/dsl/org.gradle.api.tasks.bundling.Zip.html\#org.gradle.api.tasks.bundling.Zip:version}{\texttt{version}}
\textbar{} \texttt{null}

For each task of type \texttt{BuildXSDTask}, the following helper tasks
are created:

\begin{itemize}
\tightlist
\item
  \texttt{\$\{buildXSDTask.name\}Compile}
\item
  \texttt{\$\{buildXSDTask.name\}Generate}
\end{itemize}

\subsection{Task Properties}\label{task-properties-38}

Property Name \textbar{} Type \textbar{} Default Value \textbar{}
Description \texttt{inputDir} \textbar{} \texttt{File} \textbar{}
\texttt{null} \textbar{} A directory containing XSD files from which to
generate \href{https://xmlbeans.apache.org/}{Apache XMLBeans} bindings.

The properties of type \texttt{File} support any type that can be
resolved by
\href{https://docs.gradle.org/current/dsl/org.gradle.api.Project.html\#org.gradle.api.Project:file(java.lang.Object)}{\texttt{project.file}}.

\section{Additional Configuration}\label{additional-configuration-21}

There are additional configurations that can help you use the XSD
Builder.

\section{Apache XMLBeans Dependency}\label{apache-xmlbeans-dependency}

By default, the XSD Builder Gradle plugin creates a configuration called
\texttt{xsdBuilder} and adds a dependency to the 2.5.0 version of Apache
XMLBeans. It is possible to override this setting and use a specific
version of the library by manually adding a dependency to the
\texttt{xsdBuilder} configuration:

\begin{verbatim}
dependencies {
    xsdBuilder group: "org.apache.xmlbeans", name: "xmlbeans", version: "2.6.0"
}
\end{verbatim}

\chapter{Liferay Faces}\label{liferay-faces}

Liferay Faces is an umbrella project that provides support for the
JavaServer™ Faces (JSF) standard within Liferay DXP. It encompasses the
following projects:

\begin{itemize}
\tightlist
\item
  \href{/docs/7-2/reference/-/knowledge_base/r/understanding-liferay-faces-bridge}{Liferay
  Faces Bridge} enables you to deploy JSF web apps as portlets without
  writing portlet-specific Java code. It also contains innovative
  features that make it possible to leverage the power of JSF 2.x inside
  a portlet application. Liferay Faces Bridge implements the JSR 329
  Portlet Bridge Standard.
\item
  \href{/docs/7-2/reference/-/knowledge_base/r/understanding-liferay-faces-alloy}{Liferay
  Faces Alloy} enables you to use AlloyUI components in a way that is
  consistent with JSF development.
\item
  \href{/docs/7-2/reference/-/knowledge_base/r/understanding-liferay-faces-portal}{Liferay
  Faces Portal} enables you to leverage Liferay-specific utilities and
  UI components in JSF portlets.
\end{itemize}

In this section of reference documentation, you'll learn more about each
of these projects. You'll also learn about the Liferay Faces version
scheme.

\chapter{Liferay Faces Version
Scheme}\label{liferay-faces-version-scheme}

In this article, you'll learn which Liferay Faces artifacts should be
used with your portlet and explore the Liferay Faces versioning scheme
by discovering what each component of a version means. Once you have the
versioning scheme mastered, you can view several example configurations.

\section{Using The Liferay Faces Archetype
Portlet}\label{using-the-liferay-faces-archetype-portlet}

The \href{http://liferayfaces.org}{Liferay Faces Archetype portlet} can
be used to determine the Liferay Faces artifacts and versions that you
must include in your portlet. Select your preferred Liferay Portal
version, JSF version, component suite (optional), and build tool, and
the portlet will provide you with both a command to generate your
portlet from a Maven archetype and a list of dependencies that can be
copied into your build files. In the next section, you'll be provided
with compatibility information about each version of the Liferay Faces
artifacts.

\section{Liferay Faces Alloy}\label{liferay-faces-alloy}

Provides a suite of JSF components that utilize
\href{http://alloyui.com/}{AlloyUI}.

\noindent\hrulefill

Branch\textbar Example Artifact\textbar AlloyUI\textbar JSF
API\textbar Additional Info\textbar{}
\href{https://github.com/liferay/liferay-faces-alloy/tree/master}{master
(4.x)}\textbar com.liferay.faces.alloy-4.1.0.jar\textbar3.1.x\textbar2.2+\textbar{}\emph{AlloyUI
3.1.x is the version that comes bundled with Liferay Portal
7.3.}\textbar{}
\href{https://github.com/liferay/liferay-faces-alloy/tree/3.x}{3.x}\textbar com.liferay.faces.alloy-3.1.0.jar\textbar3.0.x\textbar2.2+\textbar{}\emph{AlloyUI
3.0.x is the version that comes bundled with Liferay Portal
7.0/7.1/7.2.}\textbar{}
\href{https://github.com/liferay/liferay-faces-alloy/tree/2.x}{2.x}\textbar com.liferay.faces.alloy-2.0.1.jar\textbar2.0.x\textbar2.1+\textbar{}\emph{AlloyUI
2.0.x is the version that comes bundled with Liferay Portal
6.2.}\textbar{}
\href{https://github.com/liferay/liferay-faces-alloy/tree/1.x}{1.x}\textbar com.liferay.faces.alloy-1.0.1.jar\textbar2.0.x\textbar1.2\textbar{}\emph{AlloyUI
2.0.x is the version that comes bundled with Liferay Portal
6.2.}\textbar{}

\noindent\hrulefill

\section{Liferay Faces Bridge}\label{liferay-faces-bridge}

Provides the ability to deploy JSF web applications as portlets within
\href{https://portals.apache.org/pluto/}{Apache Pluto}, the reference
implementation for JSR 286 (Portlet 2.0) and JSR 362 (Portlet 3.0).

\noindent\hrulefill

Branch\textbar Example Artifacts\textbar Portlet API\textbar JSF
API\textbar JCP Specification\textbar Additional Info\textbar{} API:
\href{https://github.com/liferay/liferay-faces-bridge-api/tree/5.x}{5.x}IMPL:
\href{https://github.com/liferay/liferay-faces-bridge-impl/tree/5.x}{5.x}\textbar com.liferay.faces.bridge.api-5.0.0.jarcom.liferay.faces.bridge.impl-5.0.0.jar\textbar3.0\textbar2.2\textbar{}\href{https://www.jcp.org/en/jsr/detail?id=378}{JSR
378}\textbar{}\emph{Under ``Final Review'' by the JCP and scheduled for
release in 2020.}\textbar{} API:
\href{https://github.com/liferay/liferay-faces-bridge-api/tree/4.x}{4.x}IMPL:
\href{https://github.com/liferay/liferay-faces-bridge-impl/tree/4.x}{4.x}\textbar com.liferay.faces.bridge.api-4.1.0.jarcom.liferay.faces.bridge.impl-4.0.0.jar\textbar2.0\textbar2.2\textbar{}\href{https://www.jcp.org/en/jsr/detail?id=329}{JSR
329}\textbar{}\emph{Includes non-standard bridge extensions for JSF
2.2.}\textbar{} API:
\href{https://github.com/liferay/liferay-faces-bridge-api/tree/3.x}{3.x}IMPL:
\href{https://github.com/liferay/liferay-faces-bridge-impl/tree/3.x}{3.x}\textbar com.liferay.faces.bridge.api-3.1.0.jarcom.liferay.faces.bridge.impl-3.0.0.jar\textbar2.0\textbar2.1\textbar{}\href{https://www.jcp.org/en/jsr/detail?id=329}{JSR
329}\textbar{}\emph{Includes non-standard bridge extensions for JSF
2.1.}\textbar{} API:
\href{https://github.com/liferay/liferay-faces-bridge-api/tree/2.x}{2.x}IMPL:
\href{https://github.com/liferay/liferay-faces-bridge-impl/tree/2.x}{2.x}\textbar com.liferay.faces.bridge.api-2.1.0.jarcom.liferay.faces.bridge.impl-2.0.0.jar\textbar2.0\textbar1.2\textbar{}\href{https://www.jcp.org/en/jsr/detail?id=329}{JSR
329} (MR1)\textbar{}\emph{Includes support for Maintenance Release 1
(MR1).}\textbar{}
1.x\textbar N/A\textbar1.0\textbar1.2\textbar{}\href{https://www.jcp.org/en/jsr/detail?id=301}{JSR
301}\textbar{}\emph{N/A (Not Applicable) since Liferay Faces Bridge has
never implemented JSR 301.}\textbar{}

\noindent\hrulefill

\section{Liferay Faces Bridge Ext}\label{liferay-faces-bridge-ext}

Extension to Liferay Faces Bridge that provides compatibility with
\href{https://liferay.dev/-/portal}{Liferay Portal} and also takes
advantage of Liferay-specific features such as friendly URLs.

\noindent\hrulefill

Branch \textbar Example Artifact \textbar~~Liferay Portal
API~~\textbar~~Bridge API~~\textbar~~Portlet API~~\textbar JSF
API\textbar{}
\href{https://github.com/liferay/liferay-faces-bridge-ext/tree/master}{8.x}\textbar com.liferay.faces.bridge.ext-8.0.0.jar\textbar7.3.0+\textbar5.x\textbar3.0\textbar2.3\textbar{}
\href{https://github.com/liferay/liferay-faces-bridge-ext/tree/7.x}{7.x}\textbar com.liferay.faces.bridge.ext-7.0.0.jar\textbar7.3.0+\textbar5.x\textbar3.0\textbar2.2\textbar{}
\href{https://github.com/liferay/liferay-faces-bridge-ext/tree/6.x}{6.x}\textbar com.liferay.faces.bridge.ext-6.0.0.jar\textbar7.3.0+\textbar4.x\textbar2.0\textbar2.2\textbar{}
\href{https://github.com/liferay/liferay-faces-bridge-ext/tree/5.x}{5.x}\textbar com.liferay.faces.bridge.ext-5.0.4.jar\textbar7.0.x/7.1.x/7.2.x\textbar4.x\textbar2.0\textbar2.2\textbar{}
\href{https://github.com/liferay/liferay-faces-bridge-ext/tree/4.x}{4.x}\textbar UNUSED\textbar N/A\textbar N/A\textbar N/A\textbar N/A\textbar{}
\href{https://github.com/liferay/liferay-faces-bridge-ext/tree/3.x}{3.x}\textbar com.liferay.faces.bridge.ext-3.0.1.jar\textbar6.2.x\textbar4.x\textbar2.0\textbar2.2\textbar{}
\href{https://github.com/liferay/liferay-faces-bridge-ext/tree/2.x}{2.x}\textbar com.liferay.faces.bridge.ext-2.0.1.jar\textbar6.2.x\textbar3.x\textbar2.0\textbar2.1\textbar{}
\href{https://github.com/liferay/liferay-faces-bridge-ext/tree/1.x}{1.x}\textbar com.liferay.faces.bridge.ext-1.0.1.jar\textbar6.2.x\textbar2.x\textbar2.0\textbar1.2\textbar{}

\noindent\hrulefill

\section{Liferay Faces Portal}\label{liferay-faces-portal}

Provides a suite of JSF components that are based on the JSP tags
provided by \href{https://liferay.dev/-/portal}{Liferay Portal}.

\noindent\hrulefill

Branch\textbar Example Artifact\textbar Liferay Portal
API~~\textbar~~Portlet API\textbar~~JSF API\textbar{}
\href{https://github.com/liferay/liferay-faces-portal/tree/master}{6.x}\textbar com.liferay.faces.portal-6.0.0.jar\textbar7.2+\textbar3.0\textbar2.3\textbar{}
\href{https://github.com/liferay/liferay-faces-portal/tree/5.x}{5.x}\textbar com.liferay.faces.portal-5.0.0.jar\textbar7.2+\textbar3.0\textbar2.2\textbar{}
\href{https://github.com/liferay/liferay-faces-portal/tree/4.x}{4.x}\textbar com.liferay.faces.portal-4.0.0.jar\textbar7.2/7.3\textbar2.0\textbar2.2\textbar{}
\href{https://github.com/liferay/liferay-faces-portal/tree/3.x}{3.x}\textbar com.liferay.faces.portal-3.0.1.jar\textbar7.0/7.1/7.2\textbar2.0\textbar2.2\textbar{}
\href{https://github.com/liferay/liferay-faces-portal/tree/2.x}{2.x}\textbar com.liferay.faces.portal-2.0.1.jar\textbar6.2\textbar2.0\textbar2.1/2.2\textbar{}
\href{https://github.com/liferay/liferay-faces-portal/tree/1.x}{1.x}\textbar com.liferay.faces.portal-1.0.1.jar\textbar6.2\textbar2.0\textbar1.2\textbar{}

\noindent\hrulefill

\section{Liferay Faces Util}\label{liferay-faces-util}

Library that contains general purpose JSF utilities to support many of
the sub-projects that comprise Liferay Faces.

\noindent\hrulefill

Branch\textbar Example Artifact\textbar~~JSF API\textbar{}
\href{https://github.com/liferay/liferay-faces-util/tree/4.x}{4.x}\textbar com.liferay.faces.util-3.1.0.jar\textbar2.3\textbar{}
\href{https://github.com/liferay/liferay-faces-util/tree/3.x}{3.x}\textbar com.liferay.faces.util-3.1.0.jar\textbar2.2\textbar{}
\href{https://github.com/liferay/liferay-faces-util/tree/2.x}{2.x}\textbar com.liferay.faces.util-2.1.0.jar\textbar2.1\textbar{}
\href{https://github.com/liferay/liferay-faces-util/tree/1.x}{1.x}\textbar com.liferay.faces.util-1.1.0.jar\textbar1.2\textbar{}

\noindent\hrulefill

Now that you know all about the Liferay Faces versioning scheme, you may
be curious as to how these components interact with each other. Refer to
the following figure to view the Liferay Faces dependency diagram.

\begin{figure}
\centering
\includegraphics{./images/liferay-faces-dependency-diagram.png}
\caption{The Liferay Faces dependency diagram helps visualize how
components interact and depend on each other.}
\end{figure}

Next, you can view some example configurations to see the new versioning
scheme in action.

\chapter{Understanding Liferay Faces
Bridge}\label{understanding-liferay-faces-bridge}

The Liferay Faces Bridge enables you to deploy JSF web apps as portlets
without writing portlet-specific code. It also contains innovative
features that make it possible to leverage the power of JSF 2.x inside a
portlet application.

Liferay Faces Bridge is distributed in a \texttt{.jar} file. You can add
Liferay Faces Bridge as a dependency to your portlet projects, in order
to deploy your JSF web applications as portlets within JSR 286 (Portlet
2.0) compliant portlet containers, like Liferay Portal 5.2, 6.0, 6.1,
6.2, and 7.0.

The Liferay Faces Bridge project home page can be found
\href{https://community.liferay.com/-/faces}{here}.

To fully understand Liferay Faces Bridge, you must first understand the
portlet bridge standard. Because the Portlet 1.0 and JSF 1.0 specs were
being created at essentially the same time, the Expert Group (EG) for
the JSF specification constructed the JSF framework to be compliant with
portlets. For example, the
\href{https://javaee.github.io/javaee-spec/javadocs/javax/faces/context/ExternalContext.html\#getRequest--}{ExternalContext.getRequest()}
method returns an \texttt{Object} instead of an
\href{https://javaee.github.io/javaee-spec/javadocs/javax/servlet/http/HttpServletRequest.html}{javax.servlet.http.HttpServletRequest}.
When this method is used in a portal, the \texttt{Object} can be cast to
a
\href{http://portals.apache.org/pluto/portlet-2.0-apidocs/javax/portlet/PortletRequest.html}{javax.portlet.PortletRequest}.
Despite the EG's consciousness of portlet compatibility within the
design of JSF, the gap between the portlet and JSF lifecycles had to be
bridged.

Portlet bridge standards and implementations evolved over time.

Starting in 2004, several different JSF portlet bridge implementations
were developed in order to provide JSF developers with the ability to
deploy their JSF web apps as portlets. In 2006, the JCP formed the
Portlet Bridge 1.0 (\href{http://www.jcp.org/en/jsr/detail?id=301}{JSR
301}) EG in order to define a standard bridge API, as well as detailed
requirements for bridge implementations. JSR 301 was released in 2010,
targeting Portlet 1.0 and JSF 1.2.

When the Portlet 2.0 (\href{http://www.jcp.org/en/jsr/detail?id=286}{JSR
286}) standard was released in 2008, it became necessary for the JCP to
form the Portlet Bridge 2.0
(\href{http://www.jcp.org/en/jsr/detail?id=329}{JSR 329}) EG. JSR 329
was also released in 2010, targeting Portlet 2.0 and JSF 1.2.

After the \href{http://www.jcp.org/en/jsr/detail?id=314}{JSR 314} EG
released JSF 2.0 in 2009 and JSF 2.1 in 2010, it became evident that a
Portlet Bridge 3.0 standard would be beneficial. In 2015 the JCP formed
\href{http://www.jcp.org/en/jsr/detail?id=378}{JSR 378}) which is
defining a bridge for Portlet 3.0 and JSF 2.2. There are also variants
of \emph{Liferay Faces Bridge} that support Portlet 2.0 and JSF
1.2/2.1/2.2.

Liferay Faces Bridge is the Reference Implementation (RI) of the Portlet
Bridge Standard. It also contains innovative features that make it
possible to leverage the power of JSF 2.x inside a portlet application.

Now that you're familiar with some of the history of the Portlet Bridge
standards, you'll learn about the responsibilities required of the
portlet bridge.

A JSF portlet bridge aligns the correct phases of the JSF lifecycle with
each phase of the portlet lifecycle. For instance, if a browser sends an
HTTP GET request to a portal page with a JSF portlet in it, the
\texttt{RENDER\_PHASE} is performed in the portlet's lifecycle. The JSF
portlet bridge then initiates the \texttt{RESTORE\_VIEW} and
\texttt{RENDER\_RESPONSE} phases in the JSF lifecycle. Likewise, when an
HTTP POST is executed on a portlet and the portlet enters the
\texttt{ACTION\_PHASE}, then the full JSF lifecycle is initiated by the
bridge.

\begin{figure}
\centering
\includegraphics{./images/lifecycle-bridge.png}
\caption{The different phases of the JSF Lifecycle are executed
depending on which phase of the Portlet lifecycle is being executed.}
\end{figure}

Besides ensuring that the two lifecycles connect correctly, the JSF
portlet bridge also acts as a mediator between the portal URL generator
and JSF navigation rules. JSF portlet bridges ensure that URLs created
by the portal comply with JSF navigation rules, so that a JSF portlet is
able to switch to different views.

The JSR 329/378 standards defines several configuration options prefixed
with the \texttt{javax.portlet.faces} namespace. Liferay Faces Bridge
defines additional implementation-specific options prefixed with the
\texttt{com.liferay.faces.bridge} namespace.

Liferay Faces Bridge is an essential part of the JSF development process
for Liferay DXP. Visit the
\href{/docs/7-1/tutorials/-/knowledge_base/t/jsf-portlets-with-liferay-faces}{JSF
Portlets with Liferay Faces} section of tutorials for more information
on JSF development for Liferay DXP.

\section{Related Topics}\label{related-topics-45}

\href{/docs/7-2/reference/-/knowledge_base/r/understanding-liferay-faces-alloy}{Understanding
Liferay Faces Alloy}

\href{/docs/7-2/reference/-/knowledge_base/r/understanding-liferay-faces-portal}{Understanding
Liferay Faces Portal}

\href{/docs/7-2/appdev/-/knowledge_base/a/service-builder}{Service
Builder}

\chapter{Understanding Liferay Faces
Alloy}\label{understanding-liferay-faces-alloy}

Liferay Faces Alloy is distributed in a \texttt{.jar} file. You can add
Liferay Faces Alloy as a dependency to your portlet projects, to use
AlloyUI in a way that is consistent with JSF development.

\noindent\hrulefill

\textbf{Note:} AlloyUI is deprecated in Liferay DXP 7.2.

\noindent\hrulefill

During the creation of a JSF portlet in Liferay IDE/Developer Studio,
you have the option of choosing the portlet's JSF Component Suite. The
options include \emph{JSF standard},
\href{http://www.icesoft.org/java/projects/ICEfaces/overview.jsf}{\emph{ICEfaces}},
\href{http://primefaces.org/}{\emph{PrimeFaces}},
\href{http://richfaces.jboss.org/}{\emph{RichFaces}}, and \emph{Liferay
Faces Alloy}.

If you selected the Liferay Faces Alloy JSF Component Suite during your
portlet's setup, the \texttt{.jar} file is included in your portlet
project.

The Liferay Faces Alloy project provides a set of UI components that
utilize AlloyUI. For example, a brief list of some of the supported
\texttt{aui:} tags are listed below:

\begin{itemize}
\tightlist
\item
  Input: \texttt{alloy:inputText}, \texttt{alloy:inputDate},
  \texttt{alloy:inputFile}
\item
  Panel: \texttt{alloy:accordion}, \texttt{alloy:column},
  \texttt{alloy:fieldset}, \texttt{alloy:row}
\item
  Select: \texttt{alloy:selectOneMenu}, \texttt{alloy:selectOneRadio},
  \texttt{alloy:selectStarRating}
\end{itemize}

If you want to utilize Liferay's AlloyUI technology based on YUI3, you
must include the Liferay Faces Alloy \texttt{.jar} file in your JSF
portlet project. If you selected \emph{Liferay Faces Alloy} during your
JSF portlet's setup, you have Liferay Faces Alloy preconfigured in your
project, so you're automatically able to use the \texttt{alloy:} tags.

As you can see, it's extremely easy to configure your JSF application to
use Liferay's AlloyUI tags.

\section{Related Topics}\label{related-topics-46}

\href{/docs/7-2/appdev/-/knowledge_base/a/developing-a-jsf-portlet-application}{Developing
a JSF Portlet Application}

\href{/docs/7-2/reference/-/knowledge_base/r/understanding-liferay-faces-bridge}{Understanding
Liferay Faces Bridge}

\href{/docs/7-2/reference/-/knowledge_base/r/understanding-liferay-faces-portal}{Understanding
Liferay Faces Portal}

\chapter{Understanding Liferay Faces
Portal}\label{understanding-liferay-faces-portal}

\emph{Liferay Faces Portal} is distributed in a \texttt{.jar} file. You
can add Liferay Faces Portal as a dependency for your portlet projects
to use its Liferay-specific utilities and UI components. When Liferay
Faces Portal is included in a JSF portlet project, the
\texttt{com.liferay.faces.portal.{[}version{]}.jar} file resides in the
portlet's library.

\begin{figure}
\centering
\includegraphics{./images/jsf-jars-package-explorer.png}
\caption{The required \texttt{.jar} files are downloaded for your JSF
portlet based on the JSF UI Component Suite you configured.}
\end{figure}

Some of the features included in Liferay Faces Portal are:

\begin{itemize}
\item
  Utilities: Provides the \texttt{LiferayPortletHelperUtil} which
  contains a variety Portlet-API and Liferay-specific convenience
  methods.
\item
  JSF Components: Provides a set of JSF equivalents for popular Liferay
  DXP JSP tags (not exhaustive):

  \begin{itemize}
  \tightlist
  \item
    \texttt{liferay-ui:captcha} → \texttt{portal:captcha}
  \item
    \texttt{liferay-ui:input-editor} → \texttt{portal:inputRichText}
  \item
    \texttt{liferay-ui:search} → \texttt{portal:inputSearch}
  \item
    \texttt{liferay-ui:header} → \texttt{portal:header}
  \item
    \texttt{aui:nav} → \texttt{portal:nav}
  \item
    \texttt{aui:nav-item} → \texttt{portal:navItem}
  \item
    \texttt{aui:nav-bar} → \texttt{portal:navBar}
  \item
    \texttt{liferay-security:permissionsURL} →
    \texttt{portal:permissionsURL}
  \item
    \texttt{liferay-portlet:runtime} → \texttt{portal:runtime}
  \end{itemize}

  For more information, visit
  \url{https://liferayfaces.org/web/guest/portal-showcase}.
\item
  Expression Language: Adds a set of EL keywords such as
  \texttt{liferay} for getting Liferay-specific info, and \texttt{i18n}
  for integration with out-of-the-box Liferay internationalized
  messages.
\end{itemize}

Great! You now have an understanding of what Liferay Faces Portal is,
and what it accomplishes in your JSF application.

\section{Related Topics}\label{related-topics-47}

\href{/docs/7-2/appdev/-/knowledge_base/a/developing-a-jsf-portlet-application}{Developing
a JSF Portlet Application}

\href{/docs/7-2/reference/-/knowledge_base/r/understanding-liferay-faces-bridge}{Understanding
Liferay Faces Bridge}

\href{/docs/7-2/reference/-/knowledge_base/r/understanding-liferay-faces-alloy}{Understanding
Liferay Faces Alloy}

\chapter{Maven Plugins}\label{maven-plugins}

Liferay provides plugins that you can apply to your Maven project. This
reference documentation describes

\begin{itemize}
\tightlist
\item
  Configuring the plugin in your \texttt{pom.xml} file.
\item
  The plugin's available goals you can leverage.
\item
  The plugin's configuration properties.
\end{itemize}

If you're looking for additional instructions on using Maven with your
modules, see the
\href{/docs/7-2/reference/-/knowledge_base/r/maven}{Maven} articles.

\chapter{Bundle Support Plugin}\label{bundle-support-plugin}

The Bundle Support plugin lets you use
\href{/docs/7-2/reference/-/knowledge_base/r/liferay-workspace}{Liferay
Workspace} as a Maven project.

\section{Usage}\label{usage-30}

To use the plugin, include it in your project's root \texttt{pom.xml}
file:

\begin{verbatim}
<build>
    <plugins>
    ...
        <plugin>
            <groupId>com.liferay</groupId>
            <artifactId>com.liferay.portal.tools.bundle.support</artifactId>
            <version>3.2.5</version>
            <executions>
                <execution>
                    <id>clean</id>
                    <goals>
                        <goal>clean</goal>
                    </goals>
                    <phase>clean</phase>
                    <configuration>
                    </configuration>
                </execution>
                <execution>
                    <id>deploy</id>
                    <goals>
                        <goal>deploy</goal>
                    </goals>
                    <phase>pre-integration-test</phase>
                    <configuration>
                    </configuration>
                </execution>
            </executions>
        </plugin>
        ...
    </plugins>
</build>
\end{verbatim}

\section{Goals}\label{goals}

The plugin adds five Maven goals to your project:

Name \textbar{} Description
\hyperref[clean-goals-available-parameters]{bundle-support:clean}
\textbar{} Deletes a file from the \texttt{deploy} directory of a
Liferay bundle.
\hyperref[create-token-goals-available-parameters]{bundle-support:create-token}
\textbar{} Creates a token used to validate your user credentials when
downloading a DXP bundle.
\hyperref[deploy-goals-available-parameters]{bundle-support:deploy}
\textbar{} Deploys the Maven project to the specified Liferay DXP
bundle. \hyperref[dist-goals-available-parameters]{bundle-support:dist}
\textbar{} Creates a distributable Liferay DXP bundle archive file
(e.g., ZIP).
\hyperref[init-goals-available-parameters]{bundle-support:init}
\textbar{} Downloads and installs the specified Liferay DXP version.

\section{clean Goal's Available
Parameters}\label{clean-goals-available-parameters}

You can set the following parameters in the \texttt{clean} execution's
\texttt{\textless{}configuration\textgreater{}} section of the POM:

Parameter Name \textbar{} Type \textbar{} Default Value \textbar{}
Description \texttt{liferayHome} \textbar{} \texttt{String} \textbar{}
\texttt{bundles} \textbar{} The directory where your Liferay DXP
instance resides. This can be specified from the command line as
\texttt{-DliferayHome=}. \texttt{fileName} \textbar{} \texttt{String}
\textbar{} \texttt{\$\{project.artifactId\}.\$\{project.packaging\}}
\textbar{} The name of the file to delete from your bundle.

\section{create-token Goal's Available
Parameters}\label{create-token-goals-available-parameters}

You can change the default parameter values of the \texttt{create-token}
goal by creating an \texttt{\textless{}execution\textgreater{}} section
containing \texttt{\textless{}configuration\textgreater{}} tags. For
example,

\begin{verbatim}
<execution>
    <id>create-token</id>
    <goals>
        <goal>create-token</goal>
    </goals>
    <configuration>
    </configuration>
</execution>
\end{verbatim}

You can set the following parameters in the \texttt{create-token}
execution's \texttt{\textless{}configuration\textgreater{}} section of
the POM:

Parameter Name \textbar{} Type \textbar{} Default Value \textbar{}
Description \texttt{emailAddress} \textbar{} \texttt{String} \textbar{}
\texttt{null} \textbar{} The email address to use when downloading a DXP
bundle. This email address must match the one registered for your DXP
subscription. \texttt{force} \textbar{} \texttt{boolean} \textbar{}
\texttt{false} \textbar{} Whether to override the existing token with a
newly generated one. \texttt{password} \textbar{} \texttt{String}
\textbar{} \texttt{null} \textbar{} The password to use when downloading
a DXP bundle. This password must match the one registered for your DXP
subscription. \texttt{passwordFile} \textbar{} \texttt{File} \textbar{}
\texttt{null} \textbar{} The file to hold your password used when
downloading a DXP bundle. \texttt{tokenFile} \textbar{} \texttt{File}
\textbar{} \texttt{\$\{user.home\}/.liferay/token} \textbar{} The file
to hold the Liferay bundle authentication token. \texttt{tokenUrl}
\textbar{} \texttt{URL} \textbar{}
\texttt{https://releases-cdn.liferay.com/portal/7.1.0-b3/liferay-ce-portal-tomcat-7.1-b3-20180611140920623.zip}
\textbar{} The URL pointing to the bundle Zip to download.

After executing the \texttt{create-token} goal, you're prompted for your
email address and password, both of which are used to generate your
token. It's recommended to configure your email and password from the
command line rather than specifying them in your POM file.

\section{deploy Goal's Available
Parameters}\label{deploy-goals-available-parameters}

You can set the following parameters in the \texttt{deploy} execution's
\texttt{\textless{}configuration\textgreater{}} section of the POM:

Parameter Name \textbar{} Type \textbar{} Default Value \textbar{}
Description \texttt{liferayHome} \textbar{} \texttt{String} \textbar{}
\texttt{bundles} \textbar{} The directory where your Liferay DXP
instance resides. This can be specified from the command line as
\texttt{-DliferayHome=}. \texttt{deployFile} \textbar{} \texttt{File}
\textbar{}
\texttt{\$\{project.build.directory\}/\$\{project.build.finalName\}.\$\{project.packaging\}}
\textbar{} The packaged file (e.g., JAR) to deploy to the Liferay
bundle. \texttt{outputFileName} \textbar{} \texttt{String} \textbar{}
\texttt{\$\{project.artifactId\}.\$\{project.packaging\}} \textbar{} The
name of the output file.

\section{dist Goal's Available
Parameters}\label{dist-goals-available-parameters}

You can change the default parameter values of the \texttt{dist} goal by
creating an \texttt{\textless{}execution\textgreater{}} section
containing \texttt{\textless{}configuration\textgreater{}} tags. For
example,

\begin{verbatim}
<execution>
    <id>dist</id>
    <goals>
        <goal>dist</goal>
    </goals>
    <configuration>
    </configuration>
</execution>
\end{verbatim}

You can set the following parameters in the \texttt{dist} execution's
\texttt{\textless{}configuration\textgreater{}} section of the POM:

Parameter Name \textbar{} Type \textbar{} Default Value \textbar{}
Description \texttt{liferayHome} \textbar{} \texttt{String} \textbar{}
\texttt{bundles} \textbar{} The directory where your Liferay DXP
instance resides. This can be specified from the command line as
\texttt{-DliferayHome=}. \texttt{archiveFileName} \textbar{}
\texttt{String} \textbar{} \texttt{null} \textbar{} The name for the
generated archive file. \texttt{cacheDir} \textbar{} \texttt{File}
\textbar{} \texttt{\$\{user.home\}/.liferay/bundles} \textbar{} The
directory where the downloaded bundle Zip files are stored.
\texttt{configs} \textbar{} \texttt{String} \textbar{} \texttt{configs}
\textbar{} The directory that contains the configuration files.
\texttt{deployFile} \textbar{} \texttt{File}
\textbar{}\texttt{\$\{project.build.directory\}/\$\{project.build.finalName\}.\$\{project.packaging\}}
\textbar{} The packaged file (e.g., JAR) to deploy to the Liferay
bundle. \texttt{environment} \textbar{} \texttt{String} \textbar{}
\texttt{\$\{liferay.workspace.environment\}} \textbar{} The environment
of your Liferay home deployment. (e.g., \texttt{common}, \texttt{dev},
\texttt{local}, \texttt{prod}, and \texttt{uat}). \texttt{format}
\textbar{} \texttt{String} \textbar{} \texttt{zip} \textbar{} The format
type to use when packaging the Liferay bundle as an archive.
\texttt{includeFolder} \textbar{} \texttt{boolean} \textbar{}
\texttt{true} \textbar{} Whether to add a parent folder to the archive.
\texttt{outputFileName} \textbar{} \texttt{String} \textbar{}
\texttt{\$\{project.artifactId\}.\$\{project.packaging\}} \textbar{} The
path to the archive file. \texttt{password} \textbar{} \texttt{String}
\textbar{} \texttt{null} \textbar{} The password if your Liferay
bundle's URL requires authentication. \texttt{stripComponents}
\textbar{} \texttt{int} \textbar{} \texttt{1} \textbar{} The number of
directories to strip when expanding your bundle. \texttt{token}
\textbar{} \texttt{boolean} \textbar{} \texttt{false} \textbar{} Whether
to use a token to download a Liferay DXP bundle. This should be set to
\texttt{true} when downloading a DXP bundle. \texttt{tokenFile}
\textbar{} \texttt{File} \textbar{}
\texttt{\$\{user.home\}/.liferay/token} \textbar{} The file to hold the
Liferay bundle authentication token. \texttt{url} \textbar{}
\texttt{URL} \textbar{} \texttt{\$\{liferay.workspace.bundle.url\}}
\textbar{} The URL of the Liferay bundle to expand. \texttt{userName}
\textbar{} \texttt{String} \textbar{} \texttt{null} \textbar{} The user
name if your Liferay bundle's URL requires authentication.

\section{init Goal's Available
Parameters}\label{init-goals-available-parameters}

You can change the default parameter values of the \texttt{init} goal by
creating an \texttt{\textless{}execution\textgreater{}} section
containing \texttt{\textless{}configuration\textgreater{}} tags. For
example,

\begin{verbatim}
<execution>
    <id>init</id>
    <goals>
        <goal>init</goal>
    </goals>
    <configuration>
    </configuration>
</execution>
\end{verbatim}

You can set the following parameters in the \texttt{init} execution's
\texttt{\textless{}configuration\textgreater{}} section of the POM:

Parameter Name \textbar{} Type \textbar{} Default Value \textbar{}
Description \texttt{liferayHome} \textbar{} \texttt{String} \textbar{}
\texttt{bundles} \textbar{} The directory where your Liferay DXP
instance resides. This can be specified from the command line as
\texttt{-DliferayHome=}. \texttt{cacheDir} \textbar{} \texttt{File}
\textbar{} \texttt{\$\{user.home\}/.liferay/bundles} \textbar{} The
directory where the downloaded bundle Zip files are stored.
\texttt{configs} \textbar{} \texttt{String} \textbar{} \texttt{configs}
\textbar{} The directory that contains the configuration files.
\texttt{environment} \textbar{} \texttt{String} \textbar{}
\texttt{\$\{liferay.workspace.environment\}} \textbar{} The environment
with the settings appropriate for current development (e.g.,
\texttt{common}, \texttt{dev}, \texttt{local}, \texttt{prod}, and
\texttt{uat}). \texttt{password} \textbar{} \texttt{String} \textbar{}
\texttt{null} \textbar{} The password if your Liferay bundle's URL
requires authentication. \texttt{stripComponents} \textbar{}
\texttt{int} \textbar{} \texttt{1} \textbar{} The number of directories
to strip when expanding your bundle. \texttt{token} \textbar{}
\texttt{boolean} \textbar{} \texttt{false} \textbar{} Whether to use a
token to download a Liferay DXP bundle. This should be set to
\texttt{true} when downloading a DXP bundle. \texttt{tokenFile}
\textbar{} \texttt{File} \textbar{}
\texttt{\$\{user.home\}/.liferay/token} \textbar{} The file to hold the
Liferay bundle authentication token. \texttt{url} \textbar{}
\texttt{URL} \textbar{} \texttt{\$\{liferay.workspace.bundle.url\}}
\textbar{} The URL of the Liferay bundle to expand. \texttt{userName}
\textbar{} \texttt{String} \textbar{} \texttt{null} \textbar{} The user
name if your Liferay bundle's URL requires authentication.

\chapter{CSS Builder Plugin}\label{css-builder-plugin}

The CSS Builder plugin lets you compile
\href{http://sass-lang.com/}{Sass} files in your project.

\section{Usage}\label{usage-31}

To use the plugin, include it in your project's root \texttt{pom.xml}
file:

\begin{verbatim}
<build>
    <plugins>
    ...
        <plugin>
            <groupId>com.liferay</groupId>
            <artifactId>com.liferay.css.builder</artifactId>
            <version>3.0.0</version>
            <executions>
                <execution>
                    <id>default-build</id>
                    <phase>compile</phase>
                    <goals>
                        <goal>build</goal>
                    </goals>
                </execution>
            </executions>
            <configuration>
            </configuration>
        </plugin>
    ...
    </plugins>
</build>
\end{verbatim}

You can view an example POM containing the CSS Builder configuration
\href{https://github.com/liferay/liferay-portal/blob/master/modules/util/css-builder/samples/pom.xml}{here}.

\section{Goals}\label{goals-1}

The plugin adds one Maven goal to your project:

Name \textbar{} Description \texttt{css-builder:build} \textbar{}
Compiles the Sass files in the project.

\section{Available Parameters}\label{available-parameters}

You can set the following parameters in the
\texttt{\textless{}configuration\textgreater{}} section of the POM:

Parameter Name \textbar{} Type \textbar{} Default Value \textbar{}
Description \texttt{appendCssImportTimestamps} \textbar{}
\texttt{boolean} \textbar{} \texttt{true} \textbar{} Whether to append
the current timestamp to the URLs in the \texttt{@import} CSS at-rules.
\texttt{baseDir} \textbar{} \texttt{File} \textbar{}
\texttt{"src/META-INF/resources"} \textbar{} The base directory that
contains the SCSS files to compile. \texttt{dirNames} \textbar{}
\texttt{List\textless{}String\textgreater{}} \textbar{}
\texttt{{[}"/"{]}} \textbar{} The name of the directories, relative to
\hyperref[basedir]{\texttt{baseDir}}, which contain the SCSS files to
compile. \texttt{generateSourceMap} \textbar{} \texttt{boolean}
\textbar{} \texttt{false} \textbar{} Whether to generate
\href{https://developers.google.com/web/tools/chrome-devtools/debug/readability/source-maps}{source
maps} for easier debugging. \texttt{importDir} \textbar{} \texttt{File}
\textbar{} \texttt{null} \textbar{} The \texttt{META-INF/resources}
directory of the
\href{https://github.com/liferay/liferay-portal/tree/master/modules/apps/frontend-css/frontend-css-common}{Liferay
Frontend Common CSS} artifact. This is required in order to make
\href{http://bourbon.io}{Bourbon} and other CSS libraries available to
the compilation. \texttt{outputDirName} \textbar{} \texttt{String}
\textbar{} \texttt{".sass-cache/"} \textbar{} The name of the
sub-directories where the SCSS files are compiled to. For each directory
that contains SCSS files, a sub-directory with this name is created.
\texttt{precision} \textbar{} \texttt{int} \textbar{} \texttt{9}
\textbar{} The numeric precision of numbers in Sass.
\texttt{rtlExcludedPathRegexps} \textbar{}
\texttt{List\textless{}String\textgreater{}} \textbar{} \textbar{} The
SCSS file patterns to exclude when converting for right-to-left (RTL)
support. \texttt{sassCompilerClassName} \textbar{} \texttt{String}
\textbar{} \texttt{"jni"} \textbar{} The type of Sass compiler to use.
Supported values are \texttt{"jni"} and \texttt{"ruby"}. The Ruby Sass
compiler requires \texttt{com.liferay.sass.compiler.ruby.jar},
\texttt{com.liferay.ruby.gems.jar}, and \texttt{jruby-complete.jar} to
be added to the classpath.

You can also manage the \texttt{com.liferay.frontend.css.common} default
theme dependency provided by the CSS Builder in your \texttt{pom.xml}.
This can be modified by adding it as a project dependency:

\begin{verbatim}
<project>
    ...
    <dependencies>
        <dependency>
            <groupId>com.liferay</groupId>
            <artifactId>com.liferay.frontend.css.common</artifactId>
            <version>3.0.1</version>
            <scope>provided</scope>
        </dependency>
        ...
    </dependencies>
</project>
\end{verbatim}

There are additional Liferay theme-related dependencies you can manage
this way that are provided by the Theme Builder. See
\href{/docs/7-2/reference/-/knowledge_base/r/theme-builder-plugin}{this
section} for more information.

\chapter{DB Support Plugin}\label{db-support-plugin}

The DB Support plugin lets you run the Liferay DB Support tool to
execute certain actions on a local Liferay DXP database. The following
actions are available:

\begin{itemize}
\tightlist
\item
  Cleans the Liferay database from the Service Builder tables and rows
  of a module.
\end{itemize}

\section{Usage}\label{usage-32}

To use the plugin, include it in your project's \texttt{pom.xml} file:

\begin{verbatim}
<build>
    <plugins>
    ...
        <plugin>
            <groupId>com.liferay</groupId>
            <artifactId>com.liferay.portal.tools.db.support</artifactId>
            <version>1.0.6</version>
            <configuration>
            </configuration>
            <dependencies>
                <dependency>
                    <groupId>org.hsqldb</groupId>
                    <artifactId>hsqldb</artifactId>
                    <version>2.4.0</version>
                </dependency>
            </dependencies>
        </plugin>
    ...
    </plugins>
</build>
\end{verbatim}

Also notice the configured plugin dependency. You must configure the
JDBC driver used by your Liferay DXP bundle so the DB Support plugin can
properly manage your database. Replace the HSQLDB driver listed above
with your custom database's JDBC driver.

\section{Goals}\label{goals-2}

The plugin adds one Maven goal to your project:

Name \textbar{} Description \texttt{db-support:clean-service-builder}
\textbar{} Cleans the Liferay DXP database from the Service Builder
tables and rows of a module.

\section{Available Parameters}\label{available-parameters-1}

You can set the following parameters in the
\texttt{\textless{}configuration\textgreater{}} section of the POM:

Parameter Name \textbar{} Type \textbar{} Default Value \textbar{}
Description \texttt{password} \textbar{} \texttt{String} \textbar{}
\texttt{jdbc.default.password} \textbar{} The user password for
connecting to the Liferay DXP database. \texttt{propertiesFile}
\textbar{} \texttt{File} \textbar{} \texttt{null} \textbar{} The
\texttt{portal-ext.properties} file which contains the JDBC settings for
connecting to the Liferay DXP database. \texttt{serviceXmlFile}
\textbar{} \texttt{File} \textbar{} \texttt{null} \textbar{} The
\texttt{service.xml} file of the module. \texttt{servletContextName}
\textbar{} \texttt{String} \textbar{} \texttt{null} \textbar{} The
servlet context name (usually the value of the
\texttt{Bundle-Symbolic-Name} manifest header) of the module.
\texttt{url} \textbar{} \texttt{String} \textbar{}
\texttt{jdbc.default.url} \textbar{} The JDBC URL for connecting to the
Liferay DXP database. \texttt{userName} \textbar{} \texttt{String}
\textbar{} \texttt{jdbc.default.username} \textbar{} The user name for
connecting to the Liferay DXP database.

\chapter{Deployment Helper Plugin}\label{deployment-helper-plugin}

The Deployment Helper plugin lets you create a cluster deployable WAR
from your OSGi artifacts.

\section{Usage}\label{usage-33}

To use the plugin, include it in your project's root \texttt{pom.xml}
file:

\begin{verbatim}
<build>
    <plugins>
    ...
        <plugin>
            <groupId>com.liferay</groupId>
            <artifactId>com.liferay.deployment.helper</artifactId>
            <version>1.0.4</version>
            <configuration>
            </configuration>
        </plugin>
    ...
    </plugins>
</build>
\end{verbatim}

You can view an example POM containing the Deployment Helper
configuration
\href{https://github.com/liferay/liferay-portal/blob/master/modules/util/deployment-helper/samples/pom.xml}{here}.

\section{Goals}\label{goals-3}

The plugin adds one Maven goal to your project:

Name \textbar{} Description \texttt{deployment-helper:build} \textbar{}
Builds a WAR which contains one or more files that are copied once the
WAR is deployed.

\section{Available Parameters}\label{available-parameters-2}

You can set the following parameters in the
\texttt{\textless{}configuration\textgreater{}} section of the POM:

Parameter Name \textbar{} Type \textbar{} Default Value \textbar{}
Description \texttt{deploymentFileNames} \textbar{} \texttt{String}
\textbar{} \texttt{null} \textbar{} The files or directories to include
in the WAR and copy once the WAR is deployed. If a directory is added to
this collection, all the JAR files contained in the directory are
included in the WAR. \texttt{deploymentPath} \textbar{} \texttt{String}
\textbar{} \texttt{null} \textbar{} The directory to which the included
files are copied. \texttt{outputFileName} \textbar{} \texttt{String}
\textbar{} \texttt{null} \textbar{} The WAR file to build.

\chapter{Javadoc Formatter Plugin}\label{javadoc-formatter-plugin}

The Javadoc Formatter plugin lets you format project Javadoc comments.
The tool lets you generate:

\begin{itemize}
\tightlist
\item
  Default
  \href{http://www.oracle.com/technetwork/java/javase/documentation/index-137868.html\#@author}{\texttt{@author}}
  tags to all classes.
\item
  Comment stubs to classes, fields, and methods.
\item
  Missing
  \href{https://docs.oracle.com/javase/8/docs/api/java/lang/Override.html}{\texttt{@Override}}
  annotations.
\item
  An XML representation of the Javadoc comments, which can be used by
  tools in order to index the Javadocs of the project.
\end{itemize}

\section{Usage}\label{usage-34}

To use the plugin, include it in your project's root \texttt{pom.xml}
file:

\begin{verbatim}
<build>
    <plugins>
    ...
        <plugin>
            <groupId>com.liferay</groupId>
            <artifactId>com.liferay.javadoc.formatter</artifactId>
            <version>1.0.32</version>
            <configuration>
            </configuration>
        </plugin>
    ...
    </plugins>
</build>
\end{verbatim}

You can view an example POM containing the Javadoc Formatter
configuration
\href{https://github.com/liferay/liferay-portal/blob/master/modules/util/javadoc-formatter/samples/pom.xml}{here}.

\section{Goals}\label{goals-4}

The plugin adds one Maven goal to your project:

Name \textbar{} Description \texttt{javadoc-formatter:format} \textbar{}
Runs the Liferay Javadoc Formatter to format files.

\section{Available Parameters}\label{available-parameters-3}

You can set the following parameters in the
\texttt{\textless{}configuration\textgreater{}} section of the POM:

Parameter Name \textbar{} Type \textbar{} Default Value \textbar{}
Description \texttt{author} \textbar{} \texttt{String} \textbar{}
\texttt{"Brian\ Wing\ Shun\ Chan"} \textbar{} The value of the
\texttt{@author} tag to add at class level if missing.
\texttt{generateXml} \textbar{} \texttt{boolean} \textbar{}
\texttt{false} \textbar{} Whether to generate a XML representation of
the Javadoc comments. The XML files are generated in the
\texttt{src/main/resources} directory only if the Java files are
contained in \texttt{src/main/java}. \texttt{initializeMissingJavadocs}
\textbar{} \texttt{boolean} \textbar{} \texttt{false} \textbar{} Whether
to add comment stubs at the class, field, and method levels. If
\texttt{false}, only the class-level \texttt{@author} is added.
\texttt{inputDirName} \textbar{} \texttt{String} \textbar{}
\texttt{"./"} \textbar{} The root directory to begin searching for Java
files to format. \texttt{limits} \textbar{} \texttt{String{[}{]}}
\textbar{} \texttt{{[}{]}} \textbar{} The Java file name patterns,
relative to the working directory, to include when formatting Javadoc
comments. The patterns must be specified without the \texttt{.java} file
type suffix. If empty, all Java files are formatted.
\texttt{outputFilePrefix} \textbar{} \texttt{String} \textbar{}
\texttt{"javadocs"} \textbar{} The file name prefix of the XML
representation of the Javadoc comments. If \texttt{generateXML} is
\texttt{false}, this property is not used. \texttt{updateJavadocs}
\textbar{} \texttt{boolean} \textbar{} \texttt{false} \textbar{} Whether
to fix existing comment blocks by adding missing tags.

\chapter{Lang Builder Plugin}\label{lang-builder-plugin}

The Lang Builder plugin lets you sort and translate the language keys in
your project.

\section{Usage}\label{usage-35}

To use the plugin, include it in your project's root \texttt{pom.xml}
file:

\begin{verbatim}
<build>
    <plugins>
    ...
        <plugin>
            <groupId>com.liferay</groupId>
            <artifactId>com.liferay.lang.builder</artifactId>
            <version>1.0.31</version>
            <configuration>
            </configuration>
        </plugin>
    ...
    </plugins>
</build>
\end{verbatim}

You can view an example POM containing the Lang Builder configuration
\href{https://github.com/liferay/liferay-portal/blob/master/modules/util/lang-builder/samples/pom.xml}{here}.

\section{Goals}\label{goals-5}

The plugin adds one Maven goal to your project:

Name \textbar{} Description \texttt{lang-builder:build} \textbar{} Runs
Liferay Lang Builder to translate language property files.

\section{Available Parameters}\label{available-parameters-4}

You can set the following parameters in the
\texttt{\textless{}configuration\textgreater{}} section of the POM:

Parameter Name \textbar{} Type \textbar{} Default Value \textbar{}
Description \texttt{excludedLanguageIds} \textbar{}
\texttt{String{[}{]}} \textbar{}
\texttt{\{"da",\ "de",\ "fi",\ "ja",\ "nl",\ "pt\_PT",\ "sv"\}}
\textbar{} The language IDs to exclude in the automatic translation.
\texttt{langDirName} \textbar{} \texttt{String} \textbar{}
\texttt{"src/content"} \textbar{} The directory where the language
properties files are saved. \texttt{langFileName} \textbar{}
\texttt{String} \textbar{} \texttt{"Language"} \textbar{} The file name
prefix of the language properties files (e.g.,
\texttt{Language\_it.properties}). \texttt{plugin} \textbar{}
\texttt{boolean} \textbar{} \texttt{true} \textbar{} Whether to check
for duplicate language keys between the project and the portal.
\texttt{portalLanguagePropertiesFileName} \textbar{} \texttt{String}
\textbar{} \texttt{null} \textbar{} The \texttt{Language.properties}
file of the portal. \texttt{translate} \textbar{} \texttt{boolean}
\textbar{} \texttt{true} \textbar{} Whether to translate the language
keys and generate a language properties file for each locale that's
supported by Liferay DXP. \texttt{translateSubscriptionKey} \textbar{}
\texttt{String} \textbar{} \texttt{null} \textbar{} The subscription key
for Microsoft Translation integration. Subscription to the Translator
Text Translation API on Microsoft Cognitive Services is required. Basic
subscriptions, up to 2 million characters a month, are free.

\chapter{REST Builder Plugin}\label{rest-builder-plugin}

The REST Builder plugin lets you generate a REST layer defined in the
REST Builder \texttt{rest-config.yaml} and \texttt{rest-openapi.yaml}
files.

\section{Usage}\label{usage-36}

To use the plugin, include it in your project's root \texttt{pom.xml}
file:

\begin{verbatim}
<build>
    <plugins>
    ...
        <plugin>
            <groupId>com.liferay</groupId>
            <artifactId>com.liferay.portal.tools.rest.builder</artifactId>
            <version>1.0.22</version>
            <configuration>
            </configuration>
        </plugin>
    ...
    </plugins>
</build>
\end{verbatim}

You can view an example POM containing the REST Builder configuration
\href{https://github.com/liferay/liferay-portal/blob/master/modules/util/portal-tools-rest-builder/samples/pom.xml}{here}.

\section{Goals}\label{goals-6}

The plugin adds one Maven goal to your project:

Name \textbar{} Description \texttt{rest-builder:build} \textbar{} Runs
the Liferay REST Builder.

\section{Available Parameters}\label{available-parameters-5}

You can set the following parameters in the
\texttt{\textless{}configuration\textgreater{}} section of the POM:

Parameter Name \textbar{} Type \textbar{} Default Value \textbar{}
Description \texttt{copyrightFile} \textbar{} \texttt{File} \textbar{}
\texttt{null} \textbar{} The file that contains the copyright header.
\texttt{restConfigDir} \textbar{} \texttt{File} \textbar{}
\texttt{\$\{project.projectDir\}} \textbar{} The directory that contains
the \texttt{rest-config.yaml} and \texttt{rest-openapi.yaml} files.

\chapter{Service Builder Plugin}\label{service-builder-plugin}

The Service Builder plugin lets you generate a service layer defined in
a \href{/docs/7-2/appdev/-/knowledge_base/a/service-builder}{Service
Builder} \texttt{service.xml} file. Visit the
\href{/docs/7-2/reference/-/knowledge_base/r/using-service-builder-in-a-maven-project}{Using
Service Builder in a Maven Project} tutorial to learn more about
applying Service Builder to your Maven project.

\section{Usage}\label{usage-37}

To use the plugin, include it in your project's root \texttt{pom.xml}
file:

\begin{verbatim}
<build>
    <plugins>
    ...
        <plugin>
            <groupId>com.liferay</groupId>
            <artifactId>com.liferay.portal.tools.service.builder</artifactId>
            <version>1.0.292</version>
            <configuration>
            </configuration>
        </plugin>
    ...
    </plugins>
</build>
\end{verbatim}

You can view an example POM containing the Service Builder configuration
\href{https://github.com/liferay/liferay-portal/blob/master/modules/util/portal-tools-service-builder/samples/pom.xml}{here}.

\section{Goals}\label{goals-7}

The plugin adds one Maven goal to your project:

Name \textbar{} Description \texttt{service-builder:build} \textbar{}
Runs the Liferay Service Builder.

\section{Available Parameters}\label{available-parameters-6}

You can set the following parameters in the
\texttt{\textless{}configuration\textgreater{}} section of the POM:

Parameter Name \textbar{} Type \textbar{} Default Value \textbar{}
Description \texttt{apiDirName} \textbar{} \texttt{String} \textbar{}
\texttt{"../portal-kernel/src"} \textbar{} A directory where the service
API Java source files are generated.
\texttt{autoImportDefaultReferences} \textbar{} \texttt{boolean}
\textbar{} \texttt{true} \textbar{} Whether to automatically add default
references, like \texttt{com.liferay.portal.ClassName},
\texttt{com.liferay.portal.Resource} and
\texttt{com.liferay.portal.User}, to the services.
\texttt{autoNamespaceTables} \textbar{} \texttt{boolean} \textbar{}
\texttt{null} \textbar{} Whether to prefix table names by the namespace
specified in the \texttt{service.xml} file. \texttt{beanLocatorUtil}
\textbar{} \texttt{String} \textbar{}
\texttt{"com.liferay.portal.kernel.bean.PortalBeanLocatorUtil"}
\textbar{} The fully qualified class name of a bean locator class to use
in the generated service classes. \texttt{buildNumber} \textbar{}
\texttt{long} \textbar{} \texttt{1} \textbar{} A specific value to
assign the \texttt{build.number} property in the
\texttt{service.properties} file. \texttt{buildNumberIncrement}
\textbar{} \texttt{boolean} \textbar{} \texttt{true} \textbar{} Whether
to automatically increment the \texttt{build.number} property in the
\texttt{service.properties} file by one at every service generation.
\texttt{databaseNameMaxLength} \textbar{} \texttt{int} \textbar{}
\texttt{30} \textbar{} The upper bound for database table and column
name lengths to ensure it works on all databases. \texttt{hbmFileName}
\textbar{} \texttt{String} \textbar{}
\texttt{"src/META-INF/portal-hbm.xml"} \textbar{} A Hibernate Mapping
file to generate. \texttt{implDirName} \textbar{} \texttt{String}
\textbar{} \texttt{"src"} \textbar{} A directory where the service Java
source files are generated. \texttt{inputFileName} \textbar{}
\texttt{String} \textbar{} \texttt{"service.xml"} \textbar{} The
project's \texttt{service.xml} file. \texttt{modelHintsConfigs}
\textbar{} \texttt{String} \textbar{}
\texttt{"classpath*:META-INF/portal-model-hints.xml,\ META-INF/portal-model-hints.xml,\ classpath*:META-INF/ext-model-hints.xml,\ classpath*:META-INF/portlet-model-hints.xml"}
\textbar{} Paths to the model hints files for Liferay Service Builder to
use in generating the service layer. \texttt{modelHintsFileName}
\textbar{} \texttt{String} \textbar{}
\texttt{"src/META-INF/portal-model-hints.xml"} \textbar{} A model hints
file for the project. \texttt{osgiModule} \textbar{} \texttt{boolean}
\textbar{} \texttt{null} \textbar{} Whether to generate the service
layer for OSGi modules. \texttt{pluginName} \textbar{} \texttt{String}
\textbar{} \texttt{null} \textbar{} If specified, a plugin can enable
additional generation features, such as \texttt{Clp} class generation,
for non-OSGi modules. \texttt{propsUtil} \textbar{} \texttt{String}
\textbar{} \texttt{"com.liferay.portal.util.PropsUtil"} \textbar{} The
fully qualified class name of the service properties util class to
generate. \texttt{readOnlyPrefixes} \textbar{} \texttt{String}
\textbar{} \texttt{"fetch,\ get,\ has,\ is,\ load,\ reindex,\ search"}
\textbar{} Prefixes of methods to consider read-only.
\texttt{resourceActionsConfigs} \textbar{} \texttt{String} \textbar{}
\texttt{"META-INF/resource-actions/default.xml,\ resource-actions/default.xml"}
\textbar{} Paths to the
\href{/docs/7-2/frameworks/-/knowledge_base/f/defining-application-permissions}{resource
actions} files for Liferay Service Builder to use in generating the
service layer. \texttt{resourcesDirName} \textbar{} \texttt{String}
\textbar{} \texttt{"src"} \textbar{} A directory where the service
non-Java files are generated. \texttt{springFileName} \textbar{}
\texttt{String} \textbar{} \texttt{"src/META-INF/portal-spring.xml"}
\textbar{} A service Spring file to generate. \texttt{springNamespaces}
\textbar{} \texttt{String} \textbar{} \texttt{"beans"} \textbar{}
Namespaces of Spring XML Schemas to add to the service Spring file.
\texttt{sqlDirName} \textbar{} \texttt{String} \textbar{}
\texttt{"../sql"} \textbar{} A directory where the SQL files are
generated. \texttt{sqlFileName} \textbar{} \texttt{String} \textbar{}
\texttt{"portal-tables.sql"} \textbar{} A name (relative to
\texttt{sqlDir}) for the file in which the SQL table creation
instructions are generated. \texttt{sqlIndexesFileName} \textbar{}
\texttt{String} \textbar{} \texttt{"indexes.sql"} \textbar{} A name
(relative to \texttt{sqlDir}) for the file in which the SQL index
creation instructions are generated. \texttt{sqlSequencesFileName}
\textbar{} \texttt{String} \textbar{} \texttt{"sequences.sql"}
\textbar{} A name (relative to \texttt{sqlDir}) for the file in which
the SQL sequence creation instructions are generated.
\texttt{targetEntityName} \textbar{} \texttt{String} \textbar{}
\texttt{null} \textbar{} If specified, it's the name of the entity for
which Liferay Service Builder should generate the service.
\texttt{testDirName} \textbar{} \texttt{String} \textbar{}
\texttt{"test/integration"} \textbar{} If specified, it's a directory
where integration test Java source files are generated.

\chapter{Source Formatter Plugin}\label{source-formatter-plugin}

The Source Formatter plugin formats project files according to Liferay's
source formatting standards. For more documentation on Source Formatter
specific functionality, visit the tool's
\href{https://github.com/liferay/liferay-portal/tree/master/modules/util/source-formatter/documentation}{documentation}
folder.

\section{Usage}\label{usage-38}

To use the plugin, include it in your project's root \texttt{pom.xml}
file:

\begin{verbatim}
<build>
    <plugins>
    ...
        <plugin>
            <groupId>com.liferay</groupId>
            <artifactId>com.liferay.source.formatter</artifactId>
            <version>1.0.885</version>
            <executions>
                <execution>
                    <phase>process-sources</phase>
                    <goals>
                        <goal>format</goal>
                    </goals>
                </execution>
            </executions>
            <configuration>
            </configuration>
        </plugin>
    ...
    </plugins>
</build>
\end{verbatim}

You can view an example POM containing the Source Formatter
configuration
\href{https://github.com/liferay/liferay-portal/blob/master/modules/util/source-formatter/samples/pom.xml}{here}.

\section{Goals}\label{goals-8}

The plugin adds one Maven goal to your project:

Name \textbar{} Description \texttt{source-formatter:format} \textbar{}
Runs the Liferay Source Formatter to format source formatting errors.

\section{Available Parameters}\label{available-parameters-7}

You can set the following parameters in the
\texttt{\textless{}configuration\textgreater{}} section of the POM:

Parameter Name \textbar{} Type \textbar{} Default Value \textbar{}
Description \texttt{autoFix} \textbar{} \texttt{boolean} \textbar{}
\texttt{true} \textbar{} Whether to automatically fix source formatting
errors. \texttt{baseDir} \textbar{} \texttt{String} \textbar{}
\texttt{"./"} \textbar{} The Source Formatter base directory.
\emph{(Read-only)} \texttt{fileNames} \textbar{} \texttt{String{[}{]}}
\textbar{} \texttt{null} \textbar{} The file names to format, relative
to the project directory. If \texttt{null}, all files contained in
\texttt{baseDir} will be formatted. \texttt{formatCurrentBranch}
\textbar{} \texttt{boolean} \textbar{} \texttt{false} \textbar{} Whether
to format only the files contained in \texttt{baseDir} that are added or
modified in the current Git branch. \texttt{formatLatestAuthor}
\textbar{} \texttt{boolean} \textbar{} \texttt{false} \textbar{} Whether
to format only the files contained in \texttt{baseDir} that are added or
modified in the latest Git commits of the same author.
\texttt{formatLocalChanges} \textbar{} \texttt{boolean} \textbar{}
\texttt{false} \textbar{} Whether to format only the unstaged files
contained in \texttt{baseDir}. \texttt{gitWorkingBranchName} \textbar{}
\texttt{String} \textbar{} \texttt{"master"} \textbar{} The Git working
branch name. \texttt{includeSubrepositories} \textbar{} \texttt{boolean}
\textbar{} \texttt{false} \textbar{} Whether to format files that are in
read-only subrepositories. \texttt{maxLineLength} \textbar{}
\texttt{int} \textbar{} \texttt{80} \textbar{} The maximum number of
characters allowed in Java files. \texttt{printErrors} \textbar{}
\texttt{boolean} \textbar{} \texttt{true} \textbar{} Whether to print
formatting errors on the Standard Output stream.
\texttt{processorThreadCount} \textbar{} \texttt{int} \textbar{}
\texttt{5} \textbar{} The number of threads used by Source Formatter.
\texttt{showDocumentation} \textbar{} \texttt{boolean} \textbar{}
\texttt{false} \textbar{} Whether to show the documentation for the
source formatting issues, if present. \texttt{throwException} \textbar{}
\texttt{boolean} \textbar{} \texttt{false} \textbar{} Whether to fail
the build if formatting errors are found.

\chapter{Theme Builder Plugin}\label{theme-builder-plugin}

The Theme Builder plugin lets you build Liferay theme files in your
project. Visit the
\href{/docs/7-2/reference/-/knowledge_base/r/building-a-theme-with-maven}{Building
a Theme with Maven} tutorial to learn more about applying Theme Builder
to your Maven project.

\section{Usage}\label{usage-39}

To use the plugin, include it in your project's root \texttt{pom.xml}
file:

\begin{verbatim}
<build>
    <plugins>
    ...
        <plugin>
            <groupId>com.liferay</groupId>
            <artifactId>com.liferay.portal.tools.theme.builder</artifactId>
            <version>1.1.7</version>
            <executions>
                <execution>
                    <phase>generate-resources</phase>
                    <goals>
                        <goal>build</goal>
                    </goals>
                    <configuration>
                    </configuration>
                </execution>
            </executions>
        </plugin>
        ...
    </plugins>
</build>
\end{verbatim}

You can view an example POM containing the Theme Builder configuration
\href{https://github.com/liferay/liferay-portal/blob/master/modules/util/portal-tools-theme-builder/samples/pom.xml}{here}.

\section{Goals}\label{goals-9}

The plugin adds one Maven goal to your project:

Name \textbar{} Description \texttt{theme-builder:build} \textbar{}
Builds the theme files.

\section{Available Parameters}\label{available-parameters-8}

You can set the following parameters in the
\texttt{\textless{}configuration\textgreater{}} section of the POM:

Parameter Name \textbar{} Type \textbar{} Default Value \textbar{}
Description \texttt{diffsDir} \textbar{} \texttt{File} \textbar{}
\texttt{\$\{maven.war.src\}} \textbar{} The directory that contains the
files to copy over the parent theme. \texttt{name} \textbar{}
\texttt{String} \textbar{} \texttt{\$\{project.artifactId\}} \textbar{}
The name of the new theme. \texttt{outputDir} \textbar{} \texttt{File}
\textbar{}
\texttt{\$\{project.build.directory\}/\$\{project.build.finalName\}}
\textbar{} The directory where to build the theme. \texttt{parentDir}
\textbar{} \texttt{File} \textbar{} \texttt{null} \textbar{} The
directory of the parent theme. \texttt{parentName} \textbar{}
\texttt{String} \textbar{} \texttt{null} \textbar{} The name of the
parent theme. \texttt{templateExtension} \textbar{} \texttt{String}
\textbar{} \texttt{"ftl"} \textbar{} The extension of the template
files, usually \texttt{"ftl"} or \texttt{"vm"}. \texttt{unstyledDir}
\textbar{} \texttt{File} \textbar{} \texttt{null} \textbar{} The
directory of
\href{https://github.com/liferay/liferay-portal/tree/master/modules/apps/frontend-theme/frontend-theme-unstyled}{Liferay
Frontend Theme Unstyled}.

You can also manage the \texttt{com.liferay.frontend.theme.styled} and
\texttt{com.liferay.frontend.theme.unstyled} default theme dependencies
provided by the Theme Builder in your \texttt{pom.xml}. They can be
modified by adding them as project dependencies:

\begin{verbatim}
<project>
    ...
    <dependencies>
        ...
        <dependency>
            <groupId>com.liferay</groupId>
            <artifactId>com.liferay.frontend.theme.styled</artifactId>
            <version>3.0.4</version>
            <scope>provided</scope>
        </dependency>
        <dependency>
            <groupId>com.liferay</groupId>
            <artifactId>com.liferay.frontend.theme.unstyled</artifactId>
            <version>3.0.4</version>
            <scope>provided</scope>
        </dependency>
    </dependencies>
</project>
\end{verbatim}

There is an additional Liferay theme-related dependency you can manage
this way that's provided by the CSS Builder. See
\href{/docs/7-2/reference/-/knowledge_base/r/css-builder-plugin}{this
section} for more information.

\chapter{TLD Formatter Plugin}\label{tld-formatter-plugin}

The TLD Formatter plugin lets you format a project's TLD files.

\section{Usage}\label{usage-40}

To use the plugin, include it in your project's root \texttt{pom.xml}
file:

\begin{verbatim}
<build>
    <plugins>
    ...
        <plugin>
            <groupId>com.liferay</groupId>
            <artifactId>com.liferay.tld.formatter</artifactId>
            <version>1.0.5</version>
            <configuration>
            </configuration>
        </plugin>
    ...
    </plugins>
</build>
\end{verbatim}

You can view an example POM containing the TLD Formatter configuration
\href{https://github.com/liferay/liferay-portal/blob/master/modules/util/tld-formatter/samples/pom.xml}{here}.

\section{Goals}\label{goals-10}

The plugin adds one Maven goal to your project:

Name \textbar{} Description \texttt{tld-formatter:format} \textbar{}
Runs the Liferay TLD Formatter to format files.

\section{Available Parameters}\label{available-parameters-9}

You can set the following parameters in the
\texttt{\textless{}configuration\textgreater{}} section of the POM:

Parameter Name \textbar{} Type \textbar{} Default Value \textbar{}
Description \texttt{baseDirName} \textbar{} \texttt{String} \textbar{}
\texttt{"./"} \textbar{} The base directory to begin searching for TLD
files to format. \texttt{plugin} \textbar{} \texttt{boolean} \textbar{}
\texttt{true} \textbar{} Whether to format all the TLD files contained
in the working directory. If \texttt{false}, all
\texttt{liferay-portlet-ext.tld} files are ignored.

\chapter{WSDD Builder Plugin}\label{wsdd-builder-plugin}

The WSDD Builder plugin lets you generate the
\href{http://axis.apache.org/axis/}{Apache Axis} Web Service Deployment
Descriptor (WSDD) files from a
\href{/docs/7-2/appdev/-/knowledge_base/a/service-builder}{Service
Builder} \texttt{service.xml} file.

\section{Usage}\label{usage-41}

To use the plugin, include it in your project's root \texttt{pom.xml}
file:

\begin{verbatim}
<build>
    <plugins>
    ...
        <plugin>
            <groupId>com.liferay</groupId>
            <artifactId>com.liferay.portal.tools.wsdd.builder</artifactId>
            <version>1.0.10</version>
            <configuration>
            </configuration>
        </plugin>
    ...
    </plugins>
</build>
\end{verbatim}

You can view an example POM containing the WSDD Builder configuration
\href{https://github.com/liferay/liferay-portal/blob/master/modules/util/portal-tools-wsdd-builder/samples/pom.xml}{here}.

\section{Goals}\label{goals-11}

The plugin adds one Maven goal to your project:

Name \textbar{} Description \texttt{wsdd-builder:build} \textbar{} Runs
the Liferay WSDD Builder to generate the WSDD files.

\section{Available Parameters}\label{available-parameters-10}

You can set the following parameters in the
\texttt{\textless{}configuration\textgreater{}} section of the POM:

Parameter Name \textbar{} Type \textbar{} Default Value \textbar{}
Description \texttt{classPath} \textbar{} \texttt{String} \textbar{}
\texttt{null} \textbar{} The classpath that the Liferay WSDD Builder
uses to generate WSDD files. \texttt{inputFileName} \textbar{}
\texttt{String} \textbar{} \texttt{"service.xml"} \textbar{} The file
from which to generate the WSDD files. \texttt{outputDirName} \textbar{}
\texttt{String} \textbar{} \texttt{"src"} \textbar{} The directory where
the \texttt{*\_deploy.wsdd} and \texttt{*\_undeploy.wsdd} files are
generated. \texttt{serverConfigFileName} \textbar{} \texttt{String}
\textbar{} \texttt{"server-config.wsdd"} \textbar{} The file to
generate. \texttt{serviceNamespace} \textbar{} \texttt{String}
\textbar{} \texttt{"Plugin"} \textbar{} The namespace for the WSDD
Service.

\chapter{XML Formatter Plugin}\label{xml-formatter-plugin}

The XML Formatter plugin lets you format a project's XML files.

\section{Usage}\label{usage-42}

To use the plugin, include it in your project's root \texttt{pom.xml}
file:

\begin{verbatim}
<build>
    <plugins>
    ...
        <plugin>
            <groupId>com.liferay</groupId>
            <artifactId>com.liferay.xml.formatter</artifactId>
            <version>1.0.5</version>
            <configuration>
            </configuration>
        </plugin>
    ...
    </plugins>
</build>
\end{verbatim}

You can view an example POM containing the XML Formatter configuration
\href{https://github.com/liferay/liferay-portal/blob/master/modules/util/xml-formatter/samples/pom.xml}{here}.

\section{Goals}\label{goals-12}

The plugin adds one Maven goal to your project:

Name \textbar{} Description \texttt{xml-formatter:format} \textbar{}
Runs the Liferay XML Formatter to format the project files.

\section{Available Parameters}\label{available-parameters-11}

You can set the following parameters in the
\texttt{\textless{}configuration\textgreater{}} section of the POM:

Parameter Name \textbar{} Type \textbar{} Default Value \textbar{}
Description \texttt{fileName} \textbar{} \texttt{String} \textbar{}
\texttt{null} \textbar{} The XML file to format. This plugin only lets
you format one XML file at a time. \texttt{stripComments} \textbar{}
\texttt{boolean} \textbar{} \texttt{false} \textbar{} Whether to remove
all the comments from the XML file.

\chapter{PortletMVC4Spring}\label{portletmvc4spring}

PortletMVC4Spring integrates Spring, the Spring Web Framework, and the
MVC design pattern with portlet development. As such, it uses
configuration files from each of these areas and provides new
annotations that facilitate portlet application development. Here are
the PortletMVC4Spring reference topics:

\begin{itemize}
\item
  \href{/docs/7-2/reference/-/knowledge_base/r/portletmvc4spring-annotations}{PortletMVC4Spring
  Annotations}
\item
  \href{/docs/7-2/reference/-/knowledge_base/r/portletmvc4spring-configuration-files}{PortletMVC4Spring
  Configuration Files}
\end{itemize}

\chapter{PortletMVC4Spring Project
Anatomy}\label{portletmvc4spring-project-anatomy}

PortletMVC4Spring portlets are packaged in WARs. Liferay provide Maven
archetypes for creating projects configured to use JSP/JSPX and
Thymeleaf templates. Their commands are listed below. The
PortletMVC4Spring project structure follows the commands.

\section{Maven Commands for Generating PortletMVC4Spring
Projects}\label{maven-commands-for-generating-portletmvc4spring-projects}

Here are Maven commands for generating PortletMVC4Spring portlet
projects that use JSPX and \href{https://www.thymeleaf.org}{Thymeleaf}
View templates:

\section{SP/JSPX Form Portlet}\label{spjspx-form-portlet}

\begin{verbatim}
mvn archetype:generate \
-DarchetypeGroupId=com.liferay.portletmvc4spring.archetype \
-DarchetypeArtifactId=com.liferay.portletmvc4spring.archetype.form.jsp.portlet \
-DarchetypeVersion=5.1.0 \ 
-DgroupId=com.mycompany \ 
-DartifactId=com.mycompany.my.form.jsp.portlet
\end{verbatim}

\section{Thymeleaf Form Portlet}\label{thymeleaf-form-portlet}

\begin{verbatim}
mvn archetype:generate \
-DarchetypeGroupId=com.liferay.portletmvc4spring.archetype \
-DarchetypeArtifactId=com.liferay.portletmvc4spring.archetype.form.thymeleaf.portlet \
-DarchetypeVersion=5.1.0 \
-DgroupId=com.mycompany \
-DartifactId=com.mycompany.my.form.thymeleaf.portlet
\end{verbatim}

\section{Project Structure}\label{project-structure}

The Maven commands generate a project that includes Model and Controller
classes, View templates, a resource bundle, a stylesheet, and more. The
\href{/docs/7-2/reference/-/knowledge_base/r/portletmvc4spring-configuration-files}{Spring
contexts and configuration files} set PortletMVC4Spring development
essentials. Here's the resulting project structure:

\begin{itemize}
\tightlist
\item
  \texttt{{[}com.mycompany.my.form.jsp.portlet{]}}/ → Arbitrary project
  name

  \begin{itemize}
  \tightlist
  \item
    \texttt{src/}

    \begin{itemize}
    \tightlist
    \item
      \texttt{main/}

      \begin{itemize}
      \tightlist
      \item
        \texttt{java/{[}my-package-path{]}/}

        \begin{itemize}
        \tightlist
        \item
          \texttt{controller/} → Sub-package for Controller classes
          (optional)
        \item
          \texttt{dto/} → Sub-package for Model (data transfer object)
          classes (optional)
        \item
          \texttt{resources/} → Resources to include in the class path -
          \texttt{content/} → Resource bundles -
          \texttt{log4j.properties} → Log4J logging configuration
        \item
          \texttt{webapp/}

          \begin{itemize}
          \tightlist
          \item
            \texttt{resources/}

            \begin{itemize}
            \tightlist
            \item
              \texttt{css/} → Style sheets
            \item
              \texttt{images/} → Images
            \end{itemize}
          \item
            \texttt{WEB-INF/}

            \begin{itemize}
            \tightlist
            \item
              \texttt{spring-context/} → Contexts

              \begin{itemize}
              \tightlist
              \item
                \texttt{portlet/} → Portlet contexts

                \begin{itemize}
                \tightlist
                \item
                  \texttt{portlet1-context.xml} → Portlet context
                \end{itemize}
              \item
                \texttt{portlet-application-context.xml} → Application
                context
              \end{itemize}
            \item
              \texttt{views/} → View templates
            \item
              \texttt{liferay-display.xml} → Portlet display
              configuration
            \item
              \texttt{liferay-plugin-package.properties} → Packaging
              descriptor
            \item
              \texttt{liferay-portlet.xml} → Liferay-specific portlet
              configuration
            \item
              \texttt{portlet.xml} → Portlet configuration
            \item
              \texttt{web.xml} → Web application configuration
            \end{itemize}
          \end{itemize}
        \end{itemize}
      \end{itemize}
    \item
      \texttt{test/java/} → Test source files
    \end{itemize}
  \item
    \texttt{build.gradle} → Gradle build file
  \item
    \texttt{pom.xml} → Maven POM
  \end{itemize}
\end{itemize}

\chapter{PortletMVC4Spring
Annotations}\label{portletmvc4spring-annotations}

PortletMVC4Spring provides several annotations for mapping requests to
controller classes and controller methods.

\section{\texorpdfstring{\texttt{@RenderMapping} Annotation
Examples}{@RenderMapping Annotation Examples}}\label{rendermapping-annotation-examples}

The following table describes some \texttt{@RenderMapping} annotation
examples.

\noindent\hrulefill

\begin{longtable}[]{@{}
  >{\raggedright\arraybackslash}p{(\columnwidth - 2\tabcolsep) * \real{0.5769}}
  >{\raggedright\arraybackslash}p{(\columnwidth - 2\tabcolsep) * \real{0.4231}}@{}}
\toprule\noalign{}
\begin{minipage}[b]{\linewidth}\raggedright
Example
\end{minipage} & \begin{minipage}[b]{\linewidth}\raggedright
Description
\end{minipage} \\
\midrule\noalign{}
\endhead
\bottomrule\noalign{}
\endlastfoot
\texttt{@RenderMapping} & Handle primary render requests if no other
handler methods match the render request. \\
\texttt{@RenderMapping(params\ =\ "javax.portlet.action=success")} &
Handle the render request if has parameter a parameter setting
\texttt{javax.portlet.action=success}. \\
\texttt{@RenderMapping(param\ =\ "foo")} & Handle the request if it has
a parameter named \texttt{foo}, regardless of its value. \\
\texttt{@RenderMapping(param\ =\ "!bar")} & Handle the request as long
as it has not parameter named \texttt{bar}. \\
\texttt{@RenderMapping(windowState\ =\ "MAXIMIZED")} & Handle the
request if the window state is \texttt{MAXIMIZED}. Note, supported
portlet window states are \texttt{NORMAL}, \texttt{MAXIMIZED}, and
\texttt{MINIMIZED}. \\
\end{longtable}

\noindent\hrulefill

\section{\texorpdfstring{\texttt{@ActionMapping} Annotation
Examples}{@ActionMapping Annotation Examples}}\label{actionmapping-annotation-examples}

The table below describes some \texttt{@ActionMapping} annotation
examples.

\noindent\hrulefill

\begin{longtable}[]{@{}
  >{\raggedright\arraybackslash}p{(\columnwidth - 2\tabcolsep) * \real{0.5769}}
  >{\raggedright\arraybackslash}p{(\columnwidth - 2\tabcolsep) * \real{0.4231}}@{}}
\toprule\noalign{}
\begin{minipage}[b]{\linewidth}\raggedright
Example
\end{minipage} & \begin{minipage}[b]{\linewidth}\raggedright
Description
\end{minipage} \\
\midrule\noalign{}
\endhead
\bottomrule\noalign{}
\endlastfoot
\texttt{@ActionMapping} & Handle primary action requests if no other
handler methods match the action request. \\
\texttt{@ActionMapping(params\ =\ some.param=yourValue")} & Handle the
action request if has parameter a parameter setting
\texttt{javax.portlet.action=success}. \\
\texttt{@ActionMapping(param\ =\ "foo")} & Handle the request if it has
a parameter named \texttt{foo}, regardless of its value. \\
\texttt{@ActionMapping(param\ =\ "!bar")} & Handle the request as long
as it has not parameter named \texttt{bar}. \\
\end{longtable}

\chapter{PortletMVC4Spring Configuration
Files}\label{portletmvc4spring-configuration-files}

A PortletMVC4Spring application has these descriptors, Spring contexts,
and properties files in its \texttt{WEB-INF} folder:

\begin{itemize}
\tightlist
\item
  \texttt{web.xml} → Web application descriptor
\item
  \texttt{portlet.xml} → Portlet application descriptor
\item
  \texttt{liferay-portlet.xml} → Liferay-specific portlet descriptor
\item
  \texttt{liferay-display.xml} → Liferay-specific display descriptor
\item
  \texttt{spring-context/portlet-application-context.xml} → Portlet
  application context
\item
  \texttt{spring-context/portlet/{[}portlet{]}-context.xml} → Portlet
  context
\item
  \texttt{liferay-plugin-package.properties} → Packaging descriptor
\end{itemize}

Examples of each file are provided and portlet-specific content is
highlighted.

\section{web.xml}\label{web.xml}

The servlet container processes the \texttt{web.xml}. This file
specifies the servlet that render's the portlet and the portlet
application's context, servlet, filters, listeners, and more. Here's an
example \texttt{web.xml}:

\begin{verbatim}
<?xml version="1.0"?>

<web-app version="3.1" xmlns="http://xmlns.jcp.org/xml/ns/javaee" xmlns:xsi="http://www.w3.org/2001/XMLSchema-instance" xsi:schemaLocation="http://xmlns.jcp.org/xml/ns/javaee http://xmlns.jcp.org/xml/ns/javaee/web-app_3_1.xsd">
    <context-param>
        <param-name>contextConfigLocation</param-name>
        <param-value>/WEB-INF/spring-context/portlet-application-context.xml</param-value>
    </context-param>
    <servlet>
        <servlet-name>ViewRendererServlet</servlet-name>
        <servlet-class>com.liferay.portletmvc4spring.ViewRendererServlet</servlet-class>
        <load-on-startup>1</load-on-startup>
    </servlet>
    <servlet-mapping>
        <servlet-name>ViewRendererServlet</servlet-name>
        <url-pattern>/WEB-INF/servlet/view</url-pattern>
    </servlet-mapping>
    <filter>
        <filter-name>delegatingFilterProxy</filter-name>
        <filter-class>org.springframework.web.filter.DelegatingFilterProxy</filter-class>
    </filter>
    <filter-mapping>
        <filter-name>delegatingFilterProxy</filter-name>
        <url-pattern>/WEB-INF/servlet/view</url-pattern>
        <dispatcher>FORWARD</dispatcher>
        <dispatcher>INCLUDE</dispatcher>
    </filter-mapping>
    <listener>
        <listener-class>org.springframework.web.context.ContextLoaderListener</listener-class>
    </listener>
</web-app>
\end{verbatim}

The \texttt{\textless{}context-param/\textgreater{}} element gives the
path to the portlet application context (discussed later):

\begin{verbatim}
<context-param>
    <param-name>contextConfigLocation</param-name>
    <param-value>/WEB-INF/spring-context/portlet-application-context.xml</param-value>
</context-param>
\end{verbatim}

The \texttt{\textless{}servlet/\textgreater{}} and
\texttt{\textless{}servlet-mapping/\textgreater{}} elements set the
servlet and the internal location for its views.

\begin{verbatim}
<servlet>
    <servlet-name>ViewRendererServlet</servlet-name>
    <servlet-class>com.liferay.portletmvc4spring.ViewRendererServlet</servlet-class>
    <load-on-startup>1</load-on-startup>
</servlet>
<servlet-mapping>
    <servlet-name>ViewRendererServlet</servlet-name>
    <url-pattern>/WEB-INF/servlet/view</url-pattern>
</servlet-mapping>
\end{verbatim}

The
\href{https://liferay.github.io/portletmvc4spring/apidocs/com/liferay/portletmvc4spring/ViewRendererServlet.html}{\texttt{ViewRendererServlet}}.
converts portlet requests into servlet requests and enables the view to
be rendered using the Spring Web MVC infrastructure and the
infrastructure's renderers for JSP, Thymeleaf, Velocity, and more.

The filter and filter mappings are set to forward and include servlet
views as necessary.

\begin{verbatim}
<filter>
    <filter-name>delegatingFilterProxy</filter-name>
    <filter-class>org.springframework.web.filter.DelegatingFilterProxy</filter-class>
</filter>
<filter-mapping>
    <filter-name>delegatingFilterProxy</filter-name>
    <url-pattern>/WEB-INF/servlet/view</url-pattern>
    <dispatcher>FORWARD</dispatcher>
    <dispatcher>INCLUDE</dispatcher>
</filter-mapping>
\end{verbatim}

A listener is configured for processing the application's contexts.

\begin{verbatim}
<listener>
    <listener-class>org.springframework.web.context.ContextLoaderListener</listener-class>
</listener>
\end{verbatim}

Liferay's project archetypes generate all this boilerplate code.

\section{portlet.xml}\label{portlet.xml}

The \texttt{portlet.xml} file describes the portlet application to the
portlet container. Here's an example:

\begin{verbatim}
<?xml version="1.0"?>

<portlet-app xmlns="http://xmlns.jcp.org/xml/ns/portlet" xmlns:xsi="http://www.w3.org/2001/XMLSchema-instance" xsi:schemaLocation="http://xmlns.jcp.org/xml/ns/portlet http://xmlns.jcp.org/xml/ns/portlet/portlet-app_3_0.xsd" version="3.0">
    <portlet>
        <portlet-name>portlet1</portlet-name>
        <display-name>com.mycompany.my.form.jsp.portlet</display-name>
        <portlet-class>com.liferay.portletmvc4spring.DispatcherPortlet</portlet-class>
        <init-param>
            <name>contextConfigLocation</name>
            <value>/WEB-INF/spring-context/portlet/portlet1-context.xml</value>
        </init-param>
        <expiration-cache>0</expiration-cache>
        <supports>
            <mime-type>text/html</mime-type>
            <portlet-mode>view</portlet-mode>
        </supports>
        <resource-bundle>content.portlet1</resource-bundle>
        <portlet-info>
            <title>com.mycompany.my.form.jsp.portlet</title>
            <short-title>com.mycompany.my.form.jsp.portlet</short-title>
            <keywords>com.mycompany.my.form.jsp.portlet</keywords>
        </portlet-info>
        <security-role-ref>
            <role-name>administrator</role-name>
        </security-role-ref>
        <security-role-ref>
            <role-name>guest</role-name>
        </security-role-ref>
        <security-role-ref>
            <role-name>power-user</role-name>
        </security-role-ref>
        <security-role-ref>
            <role-name>user</role-name>
        </security-role-ref>
    </portlet>
    <filter>
        <filter-name>SpringSecurityPortletFilter</filter-name>
        <filter-class>com.liferay.portletmvc4spring.security.SpringSecurityPortletFilter</filter-class>
        <lifecycle>ACTION_PHASE</lifecycle>
        <lifecycle>RENDER_PHASE</lifecycle>
        <lifecycle>RESOURCE_PHASE</lifecycle>
    </filter>
    <filter-mapping>
        <filter-name>SpringSecurityPortletFilter</filter-name>
        <portlet-name>portlet1</portlet-name>
    </filter-mapping>
</portlet-app>
\end{verbatim}

This application has one portlet named \texttt{portlet1}.

\begin{verbatim}
<portlet-name>portlet1</portlet-name>
<display-name>com.mycompany.my.form.jsp.portlet</display-name>
<portlet-class>com.liferay.portletmvc4spring.DispatcherPortlet</portlet-class>
\end{verbatim}

The \texttt{\textless{}portlet-name/\textgreater{}} is internal and the
\texttt{\textless{}display-name/\textgreater{}} is shown to users.
\texttt{\textless{}portlet-class/\textgreater{}} specifies the portlet's
Java class.

\textbf{Important:} All PortletMVC4Spring portlets must specify
\texttt{\textless{}portlet-class\textgreater{}com.liferay.portletmvc4spring.DispatcherPortlet\textless{}/portlet-class\textgreater{}}.

The \texttt{\textless{}supports/\textgreater{}} element must declare the
mime type that the portlet templates use.

The \texttt{\textless{}resource-bundle/\textgreater{}} sets the path to
the portlet's localized Java message properties. For example, the
element refers to properties at \texttt{content/portlet1.properties}:

\begin{verbatim}
<resource-bundle>content.portlet1</resource-bundle>
\end{verbatim}

The \texttt{\textless{}portlet-info/\textgreater{}} element lists the
portlet's titles and reserved keyword.

The \texttt{\textless{}security-role-ref/\textgreater{}} elements
declare default user roles the portlet accounts for.

Lastly, the \texttt{\textless{}filter/\textgreater{}} named
\href{https://liferay.github.io/portletmvc4spring/apidocs/index.html}{\texttt{SpringSecurityPortletFilter}}
prevents Cross-Site Request Forgery (CSRF).

\begin{verbatim}
<filter>
    <filter-name>SpringSecurityPortletFilter</filter-name>
    <filter-class>com.liferay.portletmvc4spring.security.SpringSecurityPortletFilter</filter-class>
    <lifecycle>ACTION_PHASE</lifecycle>
    <lifecycle>RENDER_PHASE</lifecycle>
    <lifecycle>RESOURCE_PHASE</lifecycle>
</filter>
<filter-mapping>
    <filter-name>SpringSecurityPortletFilter</filter-name>
    <portlet-name>portlet1</portlet-name>
</filter-mapping>
\end{verbatim}

The
\href{https://docs.liferay.com/portlet-api/3.0/portlet-app_3_0.xsd}{\texttt{portlet\ XSD}}
defines the \texttt{portlet.xml}. The Liferay-specific portlet
descriptor is next.

\section{liferay-portlet.xml}\label{liferay-portlet.xml}

The \texttt{liferay-portlet.xml} file applies Liferay-specific settings
that provide more developer features. Here's an example:

\begin{verbatim}
<?xml version="1.0"?>
<!DOCTYPE liferay-portlet-app PUBLIC "-//Liferay//DTD Portlet Application 7.1.0//EN" "http://www.liferay.com/dtd/liferay-portlet-app_7_1_0.dtd">

<liferay-portlet-app>
    <portlet>
        <portlet-name>portlet1</portlet-name>
        <icon>/resources](./images/icon.png</icon>
        <requires-namespaced-parameters>false</requires-namespaced-parameters>
    </portlet>
    <role-mapper>
        <role-name>administrator</role-name>
        <role-link>Administrator</role-link>
    </role-mapper>
    <role-mapper>
        <role-name>guest</role-name>
        <role-link>Guest</role-link>
    </role-mapper>
    <role-mapper>
        <role-name>power-user</role-name>
        <role-link>Power User</role-link>
    </role-mapper>
    <role-mapper>
        <role-name>user</role-name>
        <role-link>User</role-link>
    </role-mapper>
</liferay-portlet-app>
\end{verbatim}

This \texttt{\textless{}portlet/\textgreater{}} element associates an
icon with the portlet and indicates that name-spaced parameters aren't
required.

The \texttt{\textless{}role-mapper/\textgreater{}} elements associate
the portlet with default Liferay DXP user roles.

The
\href{https://docs.liferay.com/dxp/portal/7.2-latest/definitions/liferay-portlet-app_7_2_0.dtd.html}{liferay-portlet-app
DTD} defines the \texttt{liferay-portlet.xml} file.

\section{liferay-display.xml}\label{liferay-display.xml}

The \texttt{liferay-display.xml} applies display characteristics to the
portlet. For example, this descriptor associates the portlet with a
Widget category in Liferay DXP's Add Widget menu.

\begin{verbatim}
<?xml version="1.0"?>
<!DOCTYPE display PUBLIC "-//Liferay//DTD Display 7.2.0//EN" "http://www.liferay.com/dtd/liferay-display_7_2_0.dtd">

<display>
<category name="category.sample">
    <portlet id="portlet1" />
</category>
</display>
\end{verbatim}

See the
\href{https://docs.liferay.com/dxp/portal/7.2-latest/definitions/liferay-display_7_2_0.dtd.html}{liferay-display
DTD} for details.

It's time to look at the application contexts.

\section{Portlet Application Context}\label{portlet-application-context}

This context applies to all of the application's portlets. This is where
you specify view resolvers, resource bundles, security beans, proxies,
and more. Here's an example:

\begin{verbatim}
<?xml version="1.0"?>

<beans
    xmlns="http://www.springframework.org/schema/beans"
    xmlns:context="http://www.springframework.org/schema/context"
    xmlns:xsi="http://www.w3.org/2001/XMLSchema-instance"
    xsi:schemaLocation="
        http://www.springframework.org/schema/beans http://www.springframework.org/schema/beans/spring-beans.xsd
        http://www.springframework.org/schema/context http://www.springframework.org/schema/context/spring-context.xsd">
    <context:annotation-config />
    <bean class="org.springframework.web.servlet.view.InternalResourceViewResolver" id="viewResolver">
        <property name="contentType" value="text/html;charset=UTF-8" />
        <property name="prefix" value="/WEB-INF/views/" />
        <property name="suffix" value=".jspx" />
        <property name="viewClass" value="com.liferay.portletmvc4spring.PortletJstlView" />
    </bean>
    <bean id="messageSource" class="org.springframework.context.support.ResourceBundleMessageSource">
        <property name="basenames">
            <list>
                <value>content.portlet1</value>
            </list>
        </property>
        <property name="defaultEncoding" value="UTF-8" />
    </bean>
    <bean id="springSecurityPortletConfigurer" class="com.liferay.portletmvc4spring.security.SpringSecurityPortletConfigurer" />
    <bean id="delegatingFilterProxy" class="org.springframework.web.filter.DelegatingFilterProxy">
        <property name="targetBeanName" value="springSecurityFilterChain" />
    </bean>
</beans>
\end{verbatim}

The view resolver bean above handles JSPX view templates. To resolve
Thymeleaf view templates, for example, you could specify these beans:

\begin{verbatim}
<bean class="org.thymeleaf.templateresolver.ServletContextTemplateResolver" id="templateResolver">
    <property name="prefix" value="/WEB-INF/views/"/>
    <property name="suffix" value=".html"/>
    <property name="templateMode" value="HTML"/>
</bean>
<bean class="org.thymeleaf.spring5.SpringTemplateEngine" id="templateEngine">
    <property name="templateResolver" ref="templateResolver"></property>
    <property name="enableSpringELCompiler" value="true" />
</bean>
<bean class="org.thymeleaf.spring5.view.ThymeleafViewResolver" id="viewResolver">
    <property name="templateEngine" ref="templateEngine"/>
    <property name="order" value="1"/>
</bean>
\end{verbatim}

The context's \texttt{springSecurityPortletConfigurer} bean facilitates
using Spring Security:

\begin{verbatim}
<bean id="springSecurityPortletConfigurer" 
    class="com.liferay.portletmvc4spring.security.SpringSecurityPortletConfigurer" />
\end{verbatim}

You can also designate contexts for each portlet in the application.

\section{Portlet Contexts}\label{portlet-contexts}

Beans specific to a portlet, go in the portlet's context. Since
annotations are the easiest way to develop PortletMVC4Spring portlets,
you should specify MVC annotation scanning in the portlet context:

\begin{verbatim}
<?xml version="1.0"?>

<beans
    xmlns="http://www.springframework.org/schema/beans"
    xmlns:context="http://www.springframework.org/schema/context"
    xmlns:xsi="http://www.w3.org/2001/XMLSchema-instance"
    xmlns:mvc="http://www.springframework.org/schema/mvc"
    xsi:schemaLocation="
        http://www.springframework.org/schema/beans http://www.springframework.org/schema/beans/spring-beans.xsd
        http://www.springframework.org/schema/context http://www.springframework.org/schema/context/spring-context.xsd
        http://www.springframework.org/schema/mvc http://www.springframework.org/schema/mvc/spring-mvc.xsd">
    <context:component-scan base-package="portlet1**"/>
    <mvc:annotation-driven/>
</beans>
\end{verbatim}

The portlet context naming convention is
\texttt{{[}portlet-name{]}-context.xml}. To associate your portlet with
its own context, edit your application's \texttt{portlet.xml} file and
add an \texttt{\textless{}init-param/\textgreater{}} element that maps
the \texttt{\textless{}portlet/\textgreater{}} element to the portlet's
context:

\begin{verbatim}
<init-param>
    <name>contextConfigLocation</name>
    <value>/WEB-INF/spring-context/portlet/portlet1-context.xml</value>
</init-param>
\end{verbatim}

What's left is to describe your application package.

\section{liferay-plugin-package.properties}\label{liferay-plugin-package.properties-1}

This file specifies the application's name, version, Java package
imports/exports, and OSGi metadata. Here's an example package properties
file:

\begin{verbatim}
author=N/A
change-log=
licenses=N/A
liferay-versions=7.1.0+
long-description=
module-group-id=com.mycompany
module-incremental-version=1
name=com.mycompany.my.form.jsp.portlet
page-url=
short-description=my portlet short description
tags=myTag
Bundle-Version: 1.0.0
Import-Package: com.liferay.portal.webserver,com.liferay.portal.kernel.servlet.filters.invoker
\end{verbatim}

It uses this OSGi metadata header to
\href{/docs/7-2/customization/-/knowledge_base/c/importing-packages}{import
required Java packages}:

\begin{verbatim}
Import-Package: com.liferay.portal.webserver,\
com.liferay.portal.kernel.servlet.filters.invoker
\end{verbatim}

On deploying the portlet application WAR file, the
\href{/docs/7-2/customization/-/knowledge_base/c/deploying-wars-wab-generator}{WAB
Generator} adds the specified OSGi metadata to the resulting web
application bundle (WAB) that's deployed to Liferay's runtime framework.

The
\href{https://docs.liferay.com/dxp/portal/7.2-latest/propertiesdoc/liferay-plugin-package_7_2_0.properties.html}{liferay-plugin-package
reference document} describes the
\texttt{liferay-plugin-package.properties} file.

Congratulations! You've successfully toured the PortletMVC4Spring
configuration files.

\section{Related Topics}\label{related-topics-48}

\href{/docs/7-2/reference/-/knowledge_base/r/portletmvc4spring-annotations}{PortletMVC4Spring
Annotations}

\href{/docs/7-2/appdev/-/knowledge_base/a/migrating-to-portletmvc4spring}{Migrating
to PortletMVC4Spring}

\chapter{Project Templates}\label{project-templates}

Liferay provides project templates that you can use to generate starter
projects formatted in an opinionated way. These templates can be used by
most build tools (e.g., Gradle, Maven, Dev Studio) to generate your
desired project structure.

If you're using
\href{/docs/7-2/reference/-/knowledge_base/r/blade-cli}{Blade CLI},
execute the following command to display a full list of project
templates:

\begin{verbatim}
blade create -l
\end{verbatim}

If you're using
\href{/docs/7-2/reference/-/knowledge_base/r/maven}{Maven}, you can view
and use the project templates as Maven archetypes. Execute the following
command to list them:

\begin{verbatim}
mvn archetype:generate -Dfilter=liferay
\end{verbatim}

Archetypes with the \texttt{com.liferay.project.templates} prefix are
the latest templates offered by Liferay.

If you're using
\href{/docs/7-2/reference/-/knowledge_base/r/liferay-dev-studio}{Dev
Studio}, navigate to \emph{File} → \emph{New} → \emph{Liferay Module
Project} and view the project templates from the \emph{Project Template
Name} drop-down menu.

In this section of reference articles, each project template is outlined
with the appropriate generation command and folder structure. Visit the
project template article you're most interested in to start building
your own project!

\chapter{Activator Template}\label{activator-template}

In this article, you'll learn how to create a Liferay activator as a
Liferay module. To create a Liferay activator via the command line using
Blade CLI or Maven, use one of the commands with the following
parameters:

\begin{verbatim}
blade create -t activator [-p packageName] [-c className] projectName
\end{verbatim}

or

\begin{verbatim}
mvn archetype:generate \
    -DarchetypeGroupId=com.liferay \
    -DarchetypeArtifactId=com.liferay.project.templates.activator \
    -DartifactId=[projectName] \
    -Dpackage=[packageName] \
    -DclassName=[className] \
    -DliferayVersion=7.2
\end{verbatim}

You can also insert the \texttt{-b\ maven} parameter in the Blade
command to generate a Maven project using Blade CLI.

The template for this kind of project is \texttt{activator}. Suppose you
want to create an activator project called \texttt{my-activator-project}
with a package name of \texttt{com.liferay.docs.activator} and a class
name of \texttt{Activator}. You could run the following command to
accomplish this:

\begin{verbatim}
blade create -t activator -p com.liferay.docs.activator -c Activator my-activator-project
\end{verbatim}

or

\begin{verbatim}
mvn archetype:generate \
    -DarchetypeGroupId=com.liferay \
    -DarchetypeArtifactId=com.liferay.project.templates.activator \
    -DgroupId=com.liferay \
    -DartifactId=my-activator-project \
    -Dpackage=com.liferay.docs.activator \
    -Dversion=1.0 \
    -DclassName=Activator \
    -Dauthor=Joe Bloggs \
    -DliferayVersion=7.2
\end{verbatim}

Note that in your class, you're implementing the
\texttt{org.osgi.framework.BundleActivator} interface.

After running the Blade command above, your project's directory
structure looks like this:

\begin{itemize}
\tightlist
\item
  \texttt{my-activator-project}

  \begin{itemize}
  \tightlist
  \item
    \texttt{gradle}

    \begin{itemize}
    \tightlist
    \item
      \texttt{wrapper}

      \begin{itemize}
      \tightlist
      \item
        \texttt{gradle-wrapper.jar}
      \item
        \texttt{gradle-wrapper.properties}
      \end{itemize}
    \end{itemize}
  \item
    \texttt{src}

    \begin{itemize}
    \tightlist
    \item
      \texttt{main}

      \begin{itemize}
      \tightlist
      \item
        \texttt{java}

        \begin{itemize}
        \tightlist
        \item
          \texttt{com/liferay/docs/activator}

          \begin{itemize}
          \tightlist
          \item
            \texttt{Activator.java}
          \end{itemize}
        \end{itemize}
      \item
        \texttt{resources}
      \end{itemize}
    \end{itemize}
  \item
    \texttt{bnd.bnd}
  \item
    \texttt{build.gradle}
  \item
    \texttt{gradlew}
  \end{itemize}
\end{itemize}

The Maven-generated project includes a \texttt{pom.xml} file and does
not include the Gradle-specific files, but otherwise, appears exactly
the same.

The generated module is functional and is deployable to a Liferay DXP
instance. To build upon the generated app, modify the project by adding
logic and additional files to the folders outlined above.

\chapter{API Template}\label{api-template}

In this tutorial, you'll learn how to create a Liferay API as a Liferay
module. To create a Liferay API via the command line using Blade CLI or
Maven, use one of the commands with the following parameters:

\begin{verbatim}
blade create -t api [-p packageName] [-c className] projectName
\end{verbatim}

or

\begin{verbatim}
mvn archetype:generate \
    -DarchetypeGroupId=com.liferay \
    -DarchetypeArtifactId=com.liferay.project.templates.api \
    -DartifactId=[projectName] \
    -Dpackage=[packageName] \
    -DclassName=[className] \
    -DliferayVersion=7.2
\end{verbatim}

You can also insert the \texttt{-b\ maven} parameter in the Blade
command to generate a Maven project using Blade CLI.

The template for this kind of project is \texttt{api}. The \texttt{api}
template creates a simple \texttt{api} module with an empty public
interface. For example, suppose you want to create an API project called
\texttt{my-api-project} with a package name of
\texttt{com.liferay.docs.api} and a class name of \texttt{MyApi}. You
could run the following command to accomplish this:

\begin{verbatim}
blade create -t api -p com.liferay.docs -c MyApi my-api-project
\end{verbatim}

or

\begin{verbatim}
mvn archetype:generate \
    -DarchetypeGroupId=com.liferay \
    -DarchetypeArtifactId=com.liferay.project.templates.api \
    -DgroupId=com.liferay \
    -DartifactId=my-api-project \
    -Dpackage=com.liferay.docs \
    -Dversion=1.0 \
    -DclassName=MyApi \
    -Dauthor=Joe Bloggs \
    -DliferayVersion=7.2
\end{verbatim}

After running the Blade command above, your project's directory
structure looks like this:

\begin{itemize}
\tightlist
\item
  \texttt{my-api-project}

  \begin{itemize}
  \tightlist
  \item
    \texttt{gradle}

    \begin{itemize}
    \tightlist
    \item
      \texttt{wrapper}

      \begin{itemize}
      \tightlist
      \item
        \texttt{gradle-wrapper.jar}
      \item
        \texttt{gradle-wrapper.properties}
      \end{itemize}
    \end{itemize}
  \item
    \texttt{src}

    \begin{itemize}
    \tightlist
    \item
      \texttt{main}

      \begin{itemize}
      \tightlist
      \item
        \texttt{java}

        \begin{itemize}
        \tightlist
        \item
          \texttt{com/liferay/docs/api}

          \begin{itemize}
          \tightlist
          \item
            \texttt{MyApi.java}
          \end{itemize}
        \end{itemize}
      \item
        \texttt{resources}

        \begin{itemize}
        \tightlist
        \item
          \texttt{com/liferay/docs/api}

          \begin{itemize}
          \tightlist
          \item
            \texttt{packageinfo}
          \end{itemize}
        \end{itemize}
      \end{itemize}
    \end{itemize}
  \item
    \texttt{bnd.bnd}
  \item
    \texttt{build.gradle}
  \item
    \texttt{gradlew}
  \end{itemize}
\end{itemize}

The Maven-generated project includes a \texttt{pom.xml} file and does
not include the Gradle-specific files, but otherwise, appears exactly
the same.

The generated module is a working application and is deployable to a
Liferay DXP instance. To build upon the generated app, modify the
project by adding logic and additional files to the folders outlined
above.

\chapter{Control Menu Entry Template}\label{control-menu-entry-template}

In this article, you'll learn how to create a Liferay Control Menu entry
as a Liferay module. To create a Liferay Control Menu entry via the
command line using Blade CLI or Maven, use one of the commands with the
following parameters:

\begin{verbatim}
blade create -t control-menu-entry [-p packageName] [-c className] projectName
\end{verbatim}

or

\begin{verbatim}
mvn archetype:generate \
    -DarchetypeGroupId=com.liferay \
    -DarchetypeArtifactId=com.liferay.project.templates.control.menu.entry \
    -DartifactId=[projectName] \
    -Dpackage=[packageName] \
    -DclassName=[className] \
    -DliferayVersion=7.2
\end{verbatim}

You can also insert the \texttt{-b\ maven} parameter in the Blade
command to generate a Maven project using Blade CLI.

The template for this kind of project is \texttt{control-menu-entry}.
Suppose you want to create a control menu entry project called
\texttt{my-control-menu-entry-project} with a package name of
\texttt{com.liferay.docs.entry.control.menu} and a class name of
\texttt{SampleProductNavigationControlMenuEntry}. You could run the
following command to accomplish this:

\begin{verbatim}
blade create -t control-menu-entry -p com.liferay.docs.entry -c Sample my-control-menu-entry-project
\end{verbatim}

or

\begin{verbatim}
mvn archetype:generate \
    -DarchetypeGroupId=com.liferay \
    -DarchetypeArtifactId=com.liferay.project.templates.control.menu.entry \
    -DgroupId=com.liferay \
    -DartifactId=my-control-menu-entry-project \
    -Dpackage=com.liferay.docs.entry \
    -Dversion=1.0 \
    -DclassName=Sample \
    -Dauthor=Joe Bloggs \
    -DliferayVersion=7.2
\end{verbatim}

After running the Blade command above, your project's directory
structure would look like this:

\begin{itemize}
\tightlist
\item
  \texttt{my-control-menu-entry-project}

  \begin{itemize}
  \tightlist
  \item
    \texttt{gradle}

    \begin{itemize}
    \tightlist
    \item
      \texttt{wrapper}

      \begin{itemize}
      \tightlist
      \item
        \texttt{gradle-wrapper.jar}
      \item
        \texttt{gradle-wrapper.properties}
      \end{itemize}
    \end{itemize}
  \item
    \texttt{src}

    \begin{itemize}
    \tightlist
    \item
      \texttt{main}

      \begin{itemize}
      \tightlist
      \item
        \texttt{java}

        \begin{itemize}
        \tightlist
        \item
          \texttt{com/liferay/docs/entry/control/menu}

          \begin{itemize}
          \tightlist
          \item
            \texttt{SampleProductNavigationControlMenuEntry.java}
          \end{itemize}
        \end{itemize}
      \item
        \texttt{resources}

        \begin{itemize}
        \tightlist
        \item
          \texttt{content}

          \begin{itemize}
          \tightlist
          \item
            \texttt{Language.properties}
          \end{itemize}
        \end{itemize}
      \end{itemize}
    \end{itemize}
  \item
    \texttt{bnd.bnd}
  \item
    \texttt{build.gradle}
  \item
    \texttt{gradlew}
  \end{itemize}
\end{itemize}

The Maven-generated project includes a \texttt{pom.xml} file and does
not include the Gradle-specific files, but otherwise, appears exactly
the same.

The generated module is functional and is deployable to a Liferay DXP
instance. To build upon the generated app, modify the project by adding
logic and additional files to the folders outlined above. You can visit
the
\href{/docs/7-1/reference/-/knowledge_base/r/control-menu-entry}{control-menu-entry}
sample project for a more expanded sample of a Control Menu entry.

\chapter{Form Field Template}\label{form-field-template}

In this article, you'll learn how to create a Liferay form field as a
Liferay module. To create a Liferay form field via the command line
using Blade CLI or Maven, use one of the commands with the following
parameters:

\begin{verbatim}
blade create -t form-field [-p packageName] [-c className] projectName
\end{verbatim}

or

\begin{verbatim}
mvn archetype:generate \
    -DarchetypeGroupId=com.liferay \
    -DarchetypeArtifactId=com.liferay.project.templates.form.field \
    -DartifactId=[projectName] \
    -Dpackage=[packageName] \
    -DclassName=[className] \
    -DliferayVersion=7.2
\end{verbatim}

You can also insert the \texttt{-b\ maven} parameter in the Blade
command to generate a Maven project using Blade CLI.

The template for this kind of project is \texttt{form-field}. Suppose
you want to create a form field project called
\texttt{my-form-field-project} with a package name of
\texttt{com.liferay.docs.form.field} and a class name prefix of
\texttt{MyFormField}. You could run one of the following commands to
accomplish this:

\begin{verbatim}
blade create -t form-field -p com.liferay.docs -c MyFormField my-form-field-project
\end{verbatim}

or

\begin{verbatim}
mvn archetype:generate \
    -DarchetypeGroupId=com.liferay \
    -DarchetypeArtifactId=com.liferay.project.templates.form.field \
    -DgroupId=com.liferay \
    -DartifactId=my-form-field-project \
    -Dpackage=com.liferay.docs \
    -Dversion=1.0 \
    -DclassName=MyFormField \
    -Dauthor=Joe Bloggs \
    -DliferayVersion=7.2
\end{verbatim}

After running the Blade command above, your project's directory
structure looks like this:

\begin{itemize}
\tightlist
\item
  \texttt{my-form-field-project}

  \begin{itemize}
  \tightlist
  \item
    \texttt{gradle}

    \begin{itemize}
    \tightlist
    \item
      \texttt{wrapper}

      \begin{itemize}
      \tightlist
      \item
        \texttt{gradle-wrapper.jar}
      \item
        \texttt{gradle-wrapper.properties}
      \end{itemize}
    \end{itemize}
  \item
    \texttt{src}

    \begin{itemize}
    \tightlist
    \item
      \texttt{main}

      \begin{itemize}
      \tightlist
      \item
        \texttt{java}

        \begin{itemize}
        \tightlist
        \item
          \texttt{com/liferay/docs/form/field}

          \begin{itemize}
          \tightlist
          \item
            \texttt{MyFormFieldDDMFormFieldRenderer.java}
          \item
            \texttt{MyFormFieldDDMFormFieldType.java}
          \end{itemize}
        \end{itemize}
      \item
        \texttt{resources}

        \begin{itemize}
        \tightlist
        \item
          \texttt{content}

          \begin{itemize}
          \tightlist
          \item
            \texttt{Language.properties}
          \end{itemize}
        \item
          \texttt{META-INF}

          \begin{itemize}
          \tightlist
          \item
            \texttt{resources}

            \begin{itemize}
            \tightlist
            \item
              \texttt{config.js}
            \item
              \texttt{my-form-field-project.soy}
            \item
              \texttt{my-form-field-project\_field.js}
            \item
              \texttt{my-form-field-project\_field.es.js}
            \end{itemize}
          \end{itemize}
        \end{itemize}
      \end{itemize}
    \end{itemize}
  \item
    \texttt{bnd.bnd}
  \item
    \texttt{build.gradle}
  \item
    \texttt{gradlew}
  \end{itemize}
\end{itemize}

The Maven-generated project includes a \texttt{pom.xml} file and does
not include the Gradle-specific files, but otherwise, appears exactly
the same.

The generated module is a working form field and is deployable to a
Liferay DXP instance. To build upon the generated app, modify the
project by adding logic and additional files to the folders outlined
above.

\chapter{Fragment Template}\label{fragment-template}

In this article, you'll learn how to create a Liferay fragment as a
Liferay module. You can learn more about fragment modules in the
\href{/docs/7-2/customization/-/knowledge_base/c/jsp-overrides-using-osgi-fragments\#declaring-a-fragment-host}{Declaring
a Fragment Host} article and in section 3.14 of the
\href{https://osgi.org/download/r6/osgi.core-6.0.0.pdf}{OSGi Alliance's
core specification document}.

To create a Liferay fragment via the command line using Blade CLI or
Maven, use one of the commands with the following parameters:

\begin{verbatim}
blade create -t fragment [-h hostBundleName] [-H hostBundleVersion] projectName
\end{verbatim}

or

\begin{verbatim}
mvn archetype:generate \
    -DarchetypeGroupId=com.liferay \
    -DarchetypeArtifactId=com.liferay.project.templates.fragment \
    -DartifactId=[projectName] \
    -Dpackage=[packageName] \
    -DclassName=[className] \
    -DliferayVersion=7.2
\end{verbatim}

You can also insert the \texttt{-b\ maven} parameter in the Blade
command to generate a Maven project using Blade CLI.

The template for this kind of project is \texttt{fragment}. Suppose you
want to create a fragment project called \texttt{my-fragment-project}
with a host bundle symbolic name of \texttt{com.liferay.login.web} and
host bundle version of \texttt{1.0.0}. You could run the following
command to accomplish this:

\begin{verbatim}
blade create -t fragment -h com.liferay.login.web -H 1.0.0 my-fragment-project
\end{verbatim}

or

\begin{verbatim}
mvn archetype:generate \
    -DarchetypeGroupId=com.liferay \
    -DarchetypeArtifactId=com.liferay.project.templates.fragment \
    -DgroupId=com.liferay \
    -DartifactId=my-fragment-project \
    -Dversion=1.0 \
    -Dpackage= \
    -DhostBundleSymbolicName=com.liferay.login.web \
    -DhostBundleVersion=1.0.0 \
    -DliferayVersion=7.2
\end{verbatim}

The folder structure is created, but there are no files. The only files
created are the \texttt{bnd.bnd} and \texttt{build.gradle} files, which
specify your host bundle and its information, and your build tool's
files. After running the Blade command above, your project's directory
structure looks like this:

\begin{itemize}
\tightlist
\item
  \texttt{my-fragment-project}

  \begin{itemize}
  \tightlist
  \item
    \texttt{gradle}

    \begin{itemize}
    \tightlist
    \item
      \texttt{wrapper}

      \begin{itemize}
      \tightlist
      \item
        \texttt{gradle-wrapper.jar}
      \item
        \texttt{gradle-wrapper.properties}
      \end{itemize}
    \end{itemize}
  \item
    \texttt{src}

    \begin{itemize}
    \tightlist
    \item
      \texttt{main}

      \begin{itemize}
      \tightlist
      \item
        \texttt{java}

        \begin{itemize}
        \tightlist
        \item
          \texttt{my/fragment/project} -\texttt{resources}
        \end{itemize}
      \end{itemize}
    \end{itemize}
  \item
    \texttt{bnd.bnd}
  \item
    \texttt{build.gradle}
  \item
    \texttt{gradlew}
  \end{itemize}
\end{itemize}

The Maven-generated project includes a \texttt{pom.xml} file and does
not include the Gradle-specific files, but otherwise, appears exactly
the same.

The generated module is functional and is deployable to a Liferay DXP
instance. To build upon the generated app, modify the project by adding
logic and additional files to the folders outlined above.

\chapter{FreeMarker Portlet Template}\label{freemarker-portlet-template}

In this article, you'll learn how to create a Liferay FreeMarker portlet
application as a Liferay module. To create a Liferay FreeMarker portlet
application via the command line using Blade CLI or Maven, use one of
the commands with the following parameters:

\begin{verbatim}
blade create -t freemarker-portlet [-p packageName] [-c className] projectName
\end{verbatim}

or

\begin{verbatim}
mvn archetype:generate \
    -DarchetypeGroupId=com.liferay \
    -DarchetypeArtifactId=com.liferay.project.templates.freemarker.portlet \
    -DartifactId=[projectName] \
    -Dpackage=[packageName] \
    -DclassName=[className] \
    -DliferayVersion=7.2
\end{verbatim}

You can also insert the \texttt{-b\ maven} parameter in the Blade
command to generate a Maven project using Blade CLI.

The template for this kind of project is \texttt{freemarker-portlet}.
Suppose you want to create a FreeMarker portlet project called
\texttt{my-freemarker-portlet-project} with a package name of
\texttt{com.liferay.docs.freemarkerportlet} and a class name of
\texttt{MyFreemarkerPortlet}. Also, you'd like to create a service of
type \texttt{javax.portlet.Portlet} that extends the
\texttt{com.liferay.util.bridges.freemarker.FreeMarkerPortlet} class.
Here, \emph{service} means an OSGi service, not a Liferay API. Another
way to say \emph{service type} is to say \emph{component type}. You
could run the following command to accomplish this:

\begin{verbatim}
blade create -t freemarker-portlet -p com.liferay.docs.freemarkerportlet -c MyFreemarkerPortlet my-freemarker-portlet-project
\end{verbatim}

or

\begin{verbatim}
mvn archetype:generate \
    -DarchetypeGroupId=com.liferay \
    -DarchetypeArtifactId=com.liferay.project.templates.freemarker.portlet \
    -DgroupId=com.liferay \
    -DartifactId=my-freemarker-portlet-project \
    -Dpackage=com.liferay.docs.freemarkerportlet \
    -Dversion=1.0 \
    -DclassName=MyFreemarkerPortlet \
    -Dauthor=Joe Bloggs \
    -DliferayVersion=7.2
\end{verbatim}

After running the Blade command above, your project's directory
structure looks like this:

\begin{itemize}
\tightlist
\item
  \texttt{my-freemarker-portlet-project}

  \begin{itemize}
  \tightlist
  \item
    \texttt{gradle}

    \begin{itemize}
    \tightlist
    \item
      \texttt{wrapper}

      \begin{itemize}
      \tightlist
      \item
        \texttt{gradle-wrapper.jar}
      \item
        \texttt{gradle-wrapper.properties}
      \end{itemize}
    \end{itemize}
  \item
    \texttt{src}

    \begin{itemize}
    \tightlist
    \item
      \texttt{main}

      \begin{itemize}
      \tightlist
      \item
        \texttt{java}

        \begin{itemize}
        \tightlist
        \item
          \texttt{com/liferay/docs/freemarkerportlet}

          \begin{itemize}
          \tightlist
          \item
            \texttt{constants}

            \begin{itemize}
            \tightlist
            \item
              \texttt{MyFreemarkerPortletKeys.java}
            \end{itemize}
          \item
            \texttt{portlet}

            \begin{itemize}
            \tightlist
            \item
              \texttt{MyFreemarkerPortlet.java}
            \end{itemize}
          \end{itemize}
        \end{itemize}
      \item
        \texttt{resources}

        \begin{itemize}
        \tightlist
        \item
          \texttt{content}

          \begin{itemize}
          \tightlist
          \item
            \texttt{Language.properties}
          \end{itemize}
        \item
          \texttt{META-INF}

          \begin{itemize}
          \tightlist
          \item
            \texttt{resources}

            \begin{itemize}
            \tightlist
            \item
              \texttt{css}

              \begin{itemize}
              \tightlist
              \item
                \texttt{main.scss}
              \end{itemize}
            \end{itemize}
          \end{itemize}
        \item
          \texttt{templates}

          \begin{itemize}
          \tightlist
          \item
            \texttt{init.ftl}
          \item
            \texttt{view.ftl}
          \end{itemize}
        \end{itemize}
      \end{itemize}
    \end{itemize}
  \item
    \texttt{bnd.bnd}
  \item
    \texttt{build.gradle}
  \item
    \texttt{gradlew}
  \end{itemize}
\end{itemize}

The Maven-generated project includes a \texttt{pom.xml} file and does
not include the Gradle-specific files, but otherwise, appears exactly
the same.

The generated module is a working application and is deployable to a
Liferay DXP instance. To build upon the generated app, modify the
project by adding logic and additional files to the folders outlined
above.

\chapter{Layout Template}\label{layout-template}

In this article, you'll learn how to create a Liferay layout template as
a WAR project. To create a Liferay layout template via the command line
using Blade CLI or Maven, use one of the commands with the following
parameters:

\begin{verbatim}
blade create -t layout-template projectName
\end{verbatim}

or

\begin{verbatim}
mvn archetype:generate \
    -DarchetypeGroupId=com.liferay \
    -DarchetypeArtifactId=com.liferay.project.templates.layout.template \
    -DartifactId=[projectName] \
    -DliferayVersion=7.2
\end{verbatim}

You can also insert the \texttt{-b\ maven} parameter in the Blade
command to generate a Maven project using Blade CLI.

The template for this kind of project is \texttt{layout-template}.
Suppose you want to create a layout template project called
\texttt{my-layout-template-project}. You could run one of the following
commands to accomplish this:

\begin{verbatim}
blade create -t layout-template my-layout-template-project
\end{verbatim}

or

\begin{verbatim}
mvn archetype:generate \
    -DarchetypeGroupId=com.liferay \
    -DarchetypeArtifactId=com.liferay.project.templates.layout.template \
    -DgroupId=com.liferay \
    -DartifactId=my-layout-template-project \
    -Dversion=1.0 \
    -Dauthor=Joe Bloggs \
    -DliferayVersion=7.2
\end{verbatim}

After running the Blade command above, your project's directory
structure looks like this:

\begin{itemize}
\tightlist
\item
  \texttt{my-layout-template-project}

  \begin{itemize}
  \tightlist
  \item
    \texttt{gradle}

    \begin{itemize}
    \tightlist
    \item
      \texttt{wrapper}

      \begin{itemize}
      \tightlist
      \item
        \texttt{gradle-wrapper.jar}
      \item
        \texttt{gradle-wrapper.properties}
      \end{itemize}
    \end{itemize}
  \item
    \texttt{src}

    \begin{itemize}
    \tightlist
    \item
      \texttt{main}

      \begin{itemize}
      \tightlist
      \item
        \texttt{webapp}

        \begin{itemize}
        \tightlist
        \item
          \texttt{WEB-INF}

          \begin{itemize}
          \tightlist
          \item
            \texttt{liferay-layout-templates.xml}
          \item
            \texttt{liferay-plugin-package.properties}
          \end{itemize}
        \item
          \texttt{my-layout-template-project.ftl}
        \item
          \texttt{my-layout-template-project.png}
        \end{itemize}
      \end{itemize}
    \end{itemize}
  \item
    \texttt{build.gradle}
  \item
    \texttt{gradlew}
  \end{itemize}
\end{itemize}

The Maven-generated project includes a \texttt{pom.xml} file and does
not include the Gradle-specific files, but otherwise, appears exactly
the same.

The generated WAR is a working layout template and is deployable to a
Liferay DXP instance. To build upon the generated layout template,
modify the project by adding logic and additional files to the folders
outlined above.

\chapter{Modules Ext Template}\label{modules-ext-template}

In this article, you'll learn how to create an Ext module. To create an
Ext module via the command line using Blade CLI or Maven, use one of the
commands with the following parameters:

\begin{verbatim}
blade create -t modules-ext [-p packageName] [-m originalModuleName] [-M originalModuleVersion] projectName
\end{verbatim}

or

\begin{verbatim}
mvn archetype:generate \
    -DarchetypeGroupId=com.liferay \
    -DarchetypeArtifactId=com.liferay.project.templates.modules.ext \
    -DartifactId=[projectName] \
    -Dpackage=[packageName] \
    -DoriginalModuleName=[originalModuleName] \
    -DoriginalModuleVersion=[originalModuleVersion] \
    -DliferayVersion=7.2
\end{verbatim}

You can also insert the \texttt{-b\ maven} parameter in the Blade
command to generate a Maven project using Blade CLI.

The template for this kind of project is \texttt{modules-ext}. Suppose
you want to create an Ext module called \texttt{my-ext-module-project}
that overrides the \texttt{com.liferay.test.web} module (BSN) with
version \texttt{1.0.0}. If you have
\href{/docs/7-1/tutorials/-/knowledge_base/t/managing-the-target-platform-for-liferay-workspace}{Target
Platform} enabled, you're not required to specify the intended module
version to override. Also, the override module has a package path of
\texttt{com.liferay.docs.test}. You must use the exact path of the
original module when creating an Ext module. You could run the following
command to accomplish this:

\begin{verbatim}
blade create -t modules-ext -p com.liferay.docs.test -m com.liferay.test.web -M 1.0.0 my-ext-module-project
\end{verbatim}

or

\begin{verbatim}
mvn archetype:generate \
    -DarchetypeGroupId=com.liferay \
    -DarchetypeArtifactId=com.liferay.project.templates.modules.ext \
    -DartifactId=my-ext-module-project \
    -Dpackage=com.liferay.docs.test \
    -DoriginalModuleName=com.liferay.test.web \
    -DoriginalModuleVersion=1.0.0 \
    -DliferayVersion=7.2
\end{verbatim}

After running the Blade command above, your project's directory
structure looks like this:

\begin{itemize}
\tightlist
\item
  \texttt{my-ext-module-project}

  \begin{itemize}
  \tightlist
  \item
    \texttt{gradle}

    \begin{itemize}
    \tightlist
    \item
      \texttt{wrapper}

      \begin{itemize}
      \tightlist
      \item
        \texttt{gradle-wrapper.jar}
      \item
        \texttt{gradle-wrapper.properties}
      \end{itemize}
    \end{itemize}
  \item
    \texttt{src}

    \begin{itemize}
    \tightlist
    \item
      \texttt{main}

      \begin{itemize}
      \tightlist
      \item
        \texttt{java}

        \begin{itemize}
        \tightlist
        \item
          \texttt{com/liferay/docs/test}
        \end{itemize}
      \item
        \texttt{resources}
      \end{itemize}
    \end{itemize}
  \item
    \texttt{build.gradle}
  \item
    \texttt{gradlew}
  \end{itemize}
\end{itemize}

The Maven-generated project includes a \texttt{pom.xml} file and does
not include the Gradle-specific files, but otherwise, appears exactly
the same.

To build upon the generated project, modify the project by adding logic
and additional files to the folders outlined above.

\chapter{MVC Portlet Template}\label{mvc-portlet-template}

In this article, you'll learn how to create a Liferay MVC portlet
application as a Liferay module. To create a Liferay MVC portlet
application via the command line using Blade CLI or Maven, use one of
the commands with the following parameters:

\begin{verbatim}
blade create -t mvc-portlet [-p packageName] [-c className] projectName
\end{verbatim}

or

\begin{verbatim}
mvn archetype:generate \
    -DarchetypeGroupId=com.liferay \
    -DarchetypeArtifactId=com.liferay.project.templates.mvc.portlet \
    -DartifactId=[projectName] \
    -Dpackage=[packageName] \
    -DclassName=[className] \
    -DliferayVersion=7.2
\end{verbatim}

You can also insert the \texttt{-b\ maven} parameter in the Blade
command to generate a Maven project using Blade CLI.

The template for this kind of project is \texttt{mvc-portlet}. Suppose
you want to create an MVC portlet project called
\texttt{my-mvc-portlet-project} with a package name of
\texttt{com.liferay.docs.mvcportlet} and a class name of
\texttt{MyMvcPortlet}. Also, you'd like to create a service of type
\texttt{javax.portlet.Portlet} that extends the
\texttt{com.liferay.portal.kernel.portlet.bridges.mvc.MVCPortlet} class.
Here, \emph{service} means an OSGi service, not a Liferay API. Another
way to say \emph{service type} is to say \emph{component type}. You
could run the following command to accomplish this:

\begin{verbatim}
blade create -t mvc-portlet -p com.liferay.docs.mvcportlet -c MyMvcPortlet my-mvc-portlet-project
\end{verbatim}

or

\begin{verbatim}
mvn archetype:generate \
    -DarchetypeGroupId=com.liferay \
    -DarchetypeArtifactId=com.liferay.project.templates.mvc.portlet \
    -DgroupId=com.liferay \
    -DartifactId=my-mvc-portlet-project \
    -Dpackage=com.liferay.docs.mvcportlet \
    -Dversion=1.0 \
    -DclassName=MyMvcPortlet \
    -Dauthor=Joe Bloggs \
    -DliferayVersion=7.2
\end{verbatim}

After running the Blade command above, your project's directory
structure looks like this:

\begin{itemize}
\tightlist
\item
  \texttt{my-mvc-portlet-project}

  \begin{itemize}
  \tightlist
  \item
    \texttt{gradle}

    \begin{itemize}
    \tightlist
    \item
      \texttt{wrapper}

      \begin{itemize}
      \tightlist
      \item
        \texttt{gradle-wrapper.jar}
      \item
        \texttt{gradle-wrapper.properties}
      \end{itemize}
    \end{itemize}
  \item
    \texttt{src}

    \begin{itemize}
    \tightlist
    \item
      \texttt{main}

      \begin{itemize}
      \tightlist
      \item
        \texttt{java}

        \begin{itemize}
        \tightlist
        \item
          \texttt{com/liferay/docs/mvcportlet}

          \begin{itemize}
          \tightlist
          \item
            \texttt{constants}

            \begin{itemize}
            \tightlist
            \item
              \texttt{MyMvcPortletKeys.java}
            \end{itemize}
          \item
            \texttt{portlet}

            \begin{itemize}
            \tightlist
            \item
              \texttt{MyMvcPortlet.java}
            \end{itemize}
          \end{itemize}
        \end{itemize}
      \item
        \texttt{resources}

        \begin{itemize}
        \tightlist
        \item
          \texttt{content}

          \begin{itemize}
          \tightlist
          \item
            \texttt{Language.properties}
          \end{itemize}
        \item
          \texttt{META-INF}

          \begin{itemize}
          \tightlist
          \item
            \texttt{resources}

            \begin{itemize}
            \tightlist
            \item
              \texttt{init.jsp}
            \item
              \texttt{view.jsp}
            \end{itemize}
          \end{itemize}
        \end{itemize}
      \end{itemize}
    \end{itemize}
  \item
    \texttt{bnd.bnd}
  \item
    \texttt{build.gradle}
  \item
    \texttt{gradlew}
  \end{itemize}
\end{itemize}

The Maven-generated project includes a \texttt{pom.xml} file and does
not include the Gradle-specific files, but otherwise, appears exactly
the same.

The generated module is a working application and is deployable to a
Liferay DXP instance. To build upon the generated app, modify the
project by adding logic and additional files to the folders outlined
above.

\chapter{Panel App Template}\label{panel-app-template}

In this article, you'll learn how to create a Liferay panel app and
category as a Liferay module. To create a Liferay panel app and category
via the command line using Blade CLI or Maven, use one of the commands
with the following parameters:

\begin{verbatim}
blade create -t panel-app [-p packageName] [-c className] projectName
\end{verbatim}

or

\begin{verbatim}
mvn archetype:generate \
    -DarchetypeGroupId=com.liferay \
    -DarchetypeArtifactId=com.liferay.project.templates.panel.app \
    -DartifactId=[projectName] \
    -Dpackage=[packageName] \
    -DclassName=[className] \
    -DliferayVersion=7.2
\end{verbatim}

You can also insert the \texttt{-b\ maven} parameter in the Blade
command to generate a Maven project using Blade CLI.

The template for this kind of project is \texttt{panel-app}. Suppose you
want to create a panel app project called \texttt{my-panel-app-project}
with a package name prefix of \texttt{com.liferay.docs} and a class name
prefix of \texttt{Sample}. You could run the following command to
accomplish this:

\begin{verbatim}
blade create -t panel-app -p com.liferay.docs -c Sample my-panel-app-project
\end{verbatim}

or

\begin{verbatim}
mvn archetype:generate \
    -DarchetypeGroupId=com.liferay \
    -DarchetypeArtifactId=com.liferay.project.templates.panel.app \
    -DgroupId=com.liferay \
    -DartifactId=my-panel-app-project \
    -Dpackage=com.liferay.docs \
    -Dversion=1.0 \
    -DclassName=Sample \
    -Dauthor=Joe Bloggs \
    -DliferayVersion=7.2
\end{verbatim}

After running the Blade command above, your project's directory
structure would look like this

\begin{itemize}
\tightlist
\item
  \texttt{my-panel-app-project}

  \begin{itemize}
  \tightlist
  \item
    \texttt{gradle}

    \begin{itemize}
    \tightlist
    \item
      \texttt{wrapper}

      \begin{itemize}
      \tightlist
      \item
        \texttt{gradle-wrapper.jar}
      \item
        \texttt{gradle-wrapper.properties}
      \end{itemize}
    \end{itemize}
  \item
    \texttt{src}

    \begin{itemize}
    \tightlist
    \item
      \texttt{main}

      \begin{itemize}
      \tightlist
      \item
        \texttt{java}

        \begin{itemize}
        \tightlist
        \item
          \texttt{com/liferay/docs/}

          \begin{itemize}
          \tightlist
          \item
            \texttt{application/list}

            \begin{itemize}
            \tightlist
            \item
              \texttt{SamplePanelApp.java}
            \item
              \texttt{SamplePanelCategory.java}
            \end{itemize}
          \item
            \texttt{constants}

            \begin{itemize}
            \tightlist
            \item
              \texttt{SamplePanelCategoryKeys.java}
            \item
              \texttt{SamplePortletKeys.java}
            \end{itemize}
          \item
            \texttt{portlet}

            \begin{itemize}
            \tightlist
            \item
              \texttt{SamplePortlet.java}
            \end{itemize}
          \end{itemize}
        \end{itemize}
      \item
        \texttt{resources}

        \begin{itemize}
        \tightlist
        \item
          \texttt{content}

          \begin{itemize}
          \tightlist
          \item
            \texttt{Language.properties}
          \end{itemize}
        \item
          \texttt{META-INF}

          \begin{itemize}
          \tightlist
          \item
            \texttt{resources}

            \begin{itemize}
            \tightlist
            \item
              \texttt{init.jsp}
            \item
              \texttt{view.jsp}
            \end{itemize}
          \end{itemize}
        \end{itemize}
      \end{itemize}
    \end{itemize}
  \item
    \texttt{bnd.bnd}
  \item
    \texttt{build.gradle}
  \item
    \texttt{gradlew}
  \end{itemize}
\end{itemize}

The Maven-generated project includes a \texttt{pom.xml} file and does
not include the Gradle-specific files, but otherwise, appears exactly
the same.

The generated module is functional and is deployable to a Liferay DXP
instance. The generated module, by default, creates a panel category
with a panel app in Liferay DXP's Product Menu. To build upon the
generated app, modify the project by adding logic and additional files
to the folders outlined above.

\chapter{Portlet Configuration Icon}\label{portlet-configuration-icon}

In this article, you'll learn how to create a Liferay portlet
configuration icon as a Liferay module. To create a portlet
configuration icon via the command line using Blade CLI or Maven, use
one of the commands with the following parameters:

\begin{verbatim}
blade create -t portlet-configuration-icon [-p packageName] [-c className] projectName
\end{verbatim}

or

\begin{verbatim}
mvn archetype:generate \
    -DarchetypeGroupId=com.liferay \
    -DarchetypeArtifactId=com.liferay.project.templates.portlet.configuration.icon \
    -DartifactId=[projectName] \
    -Dpackage=[packageName] \
    -DclassName=[className] \
    -DliferayVersion=7.2
\end{verbatim}

You can also insert the \texttt{-b\ maven} parameter in the Blade
command to generate a Maven project using Blade CLI.

The template for this kind of project is
\texttt{portlet-configuration-icon}. Suppose you want to create a
portlet configuration icon project called
\texttt{my-portlet-config-icon} with a package name of
\texttt{com.liferay.docs.portlet.configuration.icon} and a class name of
\texttt{SamplePortletConfigurationIcon}. You could run the following
command to accomplish this:

\begin{verbatim}
blade create -t portlet-configuration-icon -p com.liferay.docs -c Sample my-portlet-config-icon
\end{verbatim}

or

\begin{verbatim}
mvn archetype:generate \
    -DarchetypeGroupId=com.liferay \
    -DarchetypeArtifactId=com.liferay.project.templates.portlet.configuration.icon \
    -DgroupId=com.liferay \
    -DartifactId=my-portlet-config-project \
    -Dpackage=com.liferay.docs \
    -Dversion=1.0 \
    -DclassName=Sample \
    -Dauthor=Joe Bloggs \
    -DliferayVersion=7.2
\end{verbatim}

After running the Blade command above, your project's directory
structure would look like this

\begin{itemize}
\tightlist
\item
  \texttt{my-portlet-config-icon}

  \begin{itemize}
  \tightlist
  \item
    \texttt{gradle}

    \begin{itemize}
    \tightlist
    \item
      \texttt{wrapper}

      \begin{itemize}
      \tightlist
      \item
        \texttt{gradle-wrapper.jar}
      \item
        \texttt{gradle-wrapper.properties}
      \end{itemize}
    \end{itemize}
  \item
    \texttt{src}

    \begin{itemize}
    \tightlist
    \item
      \texttt{main}

      \begin{itemize}
      \tightlist
      \item
        \texttt{java}

        \begin{itemize}
        \tightlist
        \item
          \texttt{com/liferay/docs/portlet/configuration/icon}

          \begin{itemize}
          \tightlist
          \item
            \texttt{SamplePortletConfigurationIcon.java}
          \end{itemize}
        \end{itemize}
      \item
        \texttt{resources}

        \begin{itemize}
        \tightlist
        \item
          \texttt{content}

          \begin{itemize}
          \tightlist
          \item
            \texttt{Language.properties}
          \end{itemize}
        \end{itemize}
      \end{itemize}
    \end{itemize}
  \item
    \texttt{bnd.bnd}
  \item
    \texttt{build.gradle}
  \item
    \texttt{gradlew}
  \end{itemize}
\end{itemize}

The Maven-generated project includes a \texttt{pom.xml} file and does
not include the Gradle-specific files, but otherwise, appears exactly
the same.

The generated module is functional and is deployable to a Liferay DXP
instance. The generated module, by default, creates a sample link in the
Hello World portlet's Options menu. To build upon the generated app,
modify the project by adding logic and additional files to the folders
outlined above. You can visit the
\href{https://github.com/liferay/liferay-blade-samples/tree/master/gradle/extensions/portlet-configuration-icon}{portlet-configuration-icon}
sample project for a more expanded sample of a portlet configuration
icon.

\chapter{Portlet Provider Template}\label{portlet-provider-template}

In this article, you'll learn how to create a Liferay portlet provider
as a Liferay module. To create a Liferay portlet provider via the
command line using Blade CLI or Maven, use one of the commands with the
following parameters:

\begin{verbatim}
blade create -t portlet-provider [-p packageName] [-c className] projectName
\end{verbatim}

or

\begin{verbatim}
mvn archetype:generate \
    -DarchetypeGroupId=com.liferay \
    -DarchetypeArtifactId=com.liferay.project.templates.portlet.provider \
    -DartifactId=[projectName] \
    -Dpackage=[packageName] \
    -DclassName=[className] \
    -DliferayVersion=7.2
\end{verbatim}

You can also insert the \texttt{-b\ maven} parameter in the Blade
command to generate a Maven project using Blade CLI.

The template for this kind of project is \texttt{portlet-provider}.
Suppose you want to create a portlet provider project called
\texttt{my-portlet-provider-project} with a package name of
\texttt{com.liferay.docs.portlet} and a class name prefix of
\texttt{Sample}. You could run the following command to accomplish this:

\begin{verbatim}
blade create -t portlet-provider -p com.liferay.docs -c Sample my-portlet-provider-project
\end{verbatim}

or

\begin{verbatim}
mvn archetype:generate \
    -DarchetypeGroupId=com.liferay \
    -DarchetypeArtifactId=com.liferay.project.templates.portlet.provider \
    -DgroupId=com.liferay \
    -DartifactId=my-portlet-provider-project \
    -Dpackage=com.liferay.docs \
    -Dversion=1.0 \
    -DclassName=Sample \
    -Dauthor=Joe Bloggs \
    -DliferayVersion=7.2
\end{verbatim}

After running the Blade command above, your project's directory
structure would look like this

\begin{itemize}
\tightlist
\item
  \texttt{my-portlet-provider-project}

  \begin{itemize}
  \tightlist
  \item
    \texttt{gradle}

    \begin{itemize}
    \tightlist
    \item
      \texttt{wrapper}

      \begin{itemize}
      \tightlist
      \item
        \texttt{gradle-wrapper.jar}
      \item
        \texttt{gradle-wrapper.properties}
      \end{itemize}
    \end{itemize}
  \item
    \texttt{src}

    \begin{itemize}
    \tightlist
    \item
      \texttt{main}

      \begin{itemize}
      \tightlist
      \item
        \texttt{java}

        \begin{itemize}
        \tightlist
        \item
          \texttt{com/liferay/docs}

          \begin{itemize}
          \tightlist
          \item
            \texttt{constants}

            \begin{itemize}
            \tightlist
            \item
              \texttt{SamplePortletKeys.java}
            \item
              \texttt{SampleWebKeys.java}
            \end{itemize}
          \item
            \texttt{portlet}

            \begin{itemize}
            \tightlist
            \item
              \texttt{SampleAddPortletProvider.java}
            \item
              \texttt{SamplePortlet.java}
            \end{itemize}
          \end{itemize}
        \end{itemize}
      \item
        \texttt{resources}

        \begin{itemize}
        \tightlist
        \item
          \texttt{META-INF}

          \begin{itemize}
          \tightlist
          \item
            \texttt{resources}

            \begin{itemize}
            \tightlist
            \item
              \texttt{init.jsp}
            \item
              \texttt{view.jsp}
            \end{itemize}
          \end{itemize}
        \end{itemize}
      \end{itemize}
    \end{itemize}
  \item
    \texttt{bnd.bnd}
  \item
    \texttt{build.gradle}
  \item
    \texttt{gradlew}
  \end{itemize}
\end{itemize}

The Maven-generated project includes a \texttt{pom.xml} file and does
not include the Gradle-specific files, but otherwise, appears exactly
the same.

The generated module is functional and is deployable to a Liferay DXP
instance. To build upon the generated app, modify the project by adding
logic and additional files to the folders outlined above. You can visit
the
\href{/docs/7-2/frameworks/-/knowledge_base/f/embedding-portlets-in-themes}{Providing
Portlets to Manage Requests} tutorial for instructions on customizing a
portlet provider project.

\chapter{Portlet Toolbar Contributor
Template}\label{portlet-toolbar-contributor-template}

In this article, you'll learn how to create a Liferay portlet toolbar
contributor as a Liferay module. To create a portlet toolbar contributor
entry via the command line using Blade CLI or Maven, use one of the
commands with the following parameters:

\begin{verbatim}
blade create -t portlet-toolbar-contributor [-p packageName] [-c className] projectName
\end{verbatim}

or

\begin{verbatim}
mvn archetype:generate \
    -DarchetypeGroupId=com.liferay \
    -DarchetypeArtifactId=com.liferay.project.templates.portlet.toolbar.contributor \
    -DartifactId=[projectName] \
    -Dpackage=[packageName] \
    -DclassName=[className] \
    -DliferayVersion=7.2
\end{verbatim}

You can also insert the \texttt{-b\ maven} parameter in the Blade
command to generate a Maven project using Blade CLI.

The template for this kind of project is
\texttt{portlet-toolbar-contributor}. Suppose you want to create a
portlet toolbar contributor project called
\texttt{my-portlet-toolbar-contributor} with a package name of
\texttt{com.liferay.docs.portlet.toolbar.contributor} and a class name
of \texttt{SamplePortletToolbarContributor}. You could run the following
command to accomplish this:

\begin{verbatim}
blade create -t portlet-toolbar-contributor -p com.liferay.docs -c Sample my-portlet-toolbar-contributor
\end{verbatim}

or

\begin{verbatim}
mvn archetype:generate \
    -DarchetypeGroupId=com.liferay \
    -DarchetypeArtifactId=com.liferay.project.templates.portlet.toolbar.contributor \
    -DgroupId=com.liferay \
    -DartifactId=my-portlet-toolbar-contributor \
    -Dpackage=com.liferay.docs \
    -Dversion=1.0 \
    -DclassName=Sample \
    -Dauthor=Joe Bloggs \
    -DliferayVersion=7.2
\end{verbatim}

After running the Blade command above, your project's directory
structure would look like this

\begin{itemize}
\tightlist
\item
  \texttt{my-portlet-toolbar-contributor}

  \begin{itemize}
  \tightlist
  \item
    \texttt{gradle}

    \begin{itemize}
    \tightlist
    \item
      \texttt{wrapper}

      \begin{itemize}
      \tightlist
      \item
        \texttt{gradle-wrapper.jar}
      \item
        \texttt{gradle-wrapper.properties}
      \end{itemize}
    \end{itemize}
  \item
    \texttt{src}

    \begin{itemize}
    \tightlist
    \item
      \texttt{main}

      \begin{itemize}
      \tightlist
      \item
        \texttt{java}

        \begin{itemize}
        \tightlist
        \item
          \texttt{com/liferay/docs/portlet/toolbar/contributor}

          \begin{itemize}
          \tightlist
          \item
            \texttt{SamplePortletToolbarContributor.java}
          \end{itemize}
        \end{itemize}
      \item
        \texttt{resources}

        \begin{itemize}
        \tightlist
        \item
          \texttt{content}

          \begin{itemize}
          \tightlist
          \item
            \texttt{Language.properties}
          \end{itemize}
        \end{itemize}
      \end{itemize}
    \end{itemize}
  \item
    \texttt{bnd.bnd}
  \item
    \texttt{build.gradle}
  \item
    \texttt{gradlew}
  \end{itemize}
\end{itemize}

The Maven-generated project includes a \texttt{pom.xml} file and does
not include the Gradle-specific files, but otherwise, appears exactly
the same.

The generated module is functional and is deployable to a Liferay DXP
instance. To build upon the generated app, modify the project by adding
logic and additional files to the folders outlined above. This generated
project, by default, creates a new button on the Hello World portlet's
toolbar. You can visit the
\href{https://github.com/liferay/liferay-blade-samples/tree/master/gradle/extensions/portlet-toolbar-contributor}{portlet-toolbar-contributor}
sample project for a more expanded sample of a portlet toolbar
contributor.

\chapter{REST Template}\label{rest-template}

In this article, you'll learn how to create a Liferay RESTful web
service packaged in a Liferay module. To create a Liferay RESTful web
service via the command line using Blade CLI or Maven, use one of the
commands with the following parameters:

\begin{verbatim}
blade create -t rest [-p packageName] [-c className] projectName
\end{verbatim}

or

\begin{verbatim}
mvn archetype:generate \
    -DarchetypeGroupId=com.liferay \
    -DarchetypeArtifactId=com.liferay.project.templates.rest \
    -DartifactId=[projectName] \
    -Dpackage=[packageName] \
    -DclassName=[className] \
    -DliferayVersion=7.2
\end{verbatim}

You can also insert the \texttt{-b\ maven} parameter in the Blade
command to generate a Maven project using Blade CLI.

The template for this kind of project is \texttt{rest}. Suppose you want
to create a RESTful web service project called \texttt{my-rest-project}
with a package name of \texttt{com.liferay.docs.application} and a class
name prefix of \texttt{Rest}. You could run one of the following
commands to accomplish this:

\begin{verbatim}
blade create -t rest -p com.liferay.docs -c Rest my-rest-project
\end{verbatim}

or

\begin{verbatim}
mvn archetype:generate \
    -DarchetypeGroupId=com.liferay \
    -DarchetypeArtifactId=com.liferay.project.templates.rest \
    -DgroupId=com.liferay \
    -DartifactId=my-rest-project \
    -Dpackage=com.liferay.docs \
    -Dversion=1.0 \
    -DclassName=Rest \
    -Dauthor=Joe Bloggs \
    -DliferayVersion=7.2
\end{verbatim}

After running the Blade command above, your project's directory
structure looks like this:

\begin{itemize}
\tightlist
\item
  \texttt{my-rest-project}

  \begin{itemize}
  \tightlist
  \item
    \texttt{gradle}

    \begin{itemize}
    \tightlist
    \item
      \texttt{wrapper}

      \begin{itemize}
      \tightlist
      \item
        \texttt{gradle-wrapper.jar}
      \item
        \texttt{gradle-wrapper.properties}
      \end{itemize}
    \end{itemize}
  \item
    \texttt{src}

    \begin{itemize}
    \tightlist
    \item
      \texttt{main}

      \begin{itemize}
      \tightlist
      \item
        \texttt{java}

        \begin{itemize}
        \tightlist
        \item
          \texttt{com/liferay/docs/application}

          \begin{itemize}
          \tightlist
          \item
            \texttt{RestApplication.java}
          \end{itemize}
        \end{itemize}
      \item
        \texttt{resources}

        \begin{itemize}
        \tightlist
        \item
          \texttt{configuration}

          \begin{itemize}
          \tightlist
          \item
            \texttt{com.liferay.portal.remote.cxf.common.configuration.CXFEndpointPublisherConfiguration-cxf.properties}
          \item
            \texttt{com.liferay.portal.remote.rest.extender.configuration.RestExtenderConfiguration-rest.properties}
          \end{itemize}
        \end{itemize}
      \end{itemize}
    \end{itemize}
  \item
    \texttt{bnd.bnd}
  \item
    \texttt{build.gradle}
  \item
    \texttt{gradlew}
  \end{itemize}
\end{itemize}

The Maven-generated project includes a \texttt{pom.xml} file and does
not include the Gradle-specific files, but otherwise, appears exactly
the same.

The generated module is a working RESTful web service and is deployable
to a Liferay DXP instance. To build upon the generated app, modify the
project by adding logic and additional files to the folders outlined
above.

\chapter{Service Builder Template}\label{service-builder-template}

In this article, you'll learn how to create a Liferay portlet
application that uses Service Builder as Liferay modules. To create a
Liferay Service Builder project via the command line using Blade CLI or
Maven, use one of the commands with the following parameters:

\begin{verbatim}
blade create -t service-builder [-p packageName] projectName
\end{verbatim}

or

\begin{verbatim}
mvn archetype:generate \
    -DarchetypeGroupId=com.liferay \
    -DarchetypeArtifactId=com.liferay.project.templates.service.builder \
    -DartifactId=[projectName] \
    -Dpackage=[packageName] \
    -DapiPath=[apiPath] \
    -DdependencyInjector=[dependencyInjector] \
    -DliferayVersion=7.2
\end{verbatim}

By default, the Service Builder project uses OSGi Declarative Services
(\texttt{ds}) for its dependency injector. If you prefer using Spring,
you can set the parameter \texttt{-\/-dependency-injector\ spring} with
Blade CLI or \texttt{-DdependencyInjector=spring} with Maven. See the
\href{/docs/7-2/frameworks/-/knowledge_base/f/dependency-injection}{Dependency
Injection} section for more information on these options.

You can also insert the \texttt{-b\ maven} parameter in the Blade
command to generate a Maven project using Blade CLI.

The template for this kind of project is \texttt{service-builder}.
Suppose you want to create a Service Builder project called
\texttt{tasks} with a package name of \texttt{com.liferay.docs.tasks}
using OSGi Declarative Services. You could run the following command to
accomplish this:

\begin{verbatim}
blade create -t service-builder -p com.liferay.docs.tasks tasks
\end{verbatim}

or

\begin{verbatim}
mvn archetype:generate \
    -DarchetypeGroupId=com.liferay \
    -DarchetypeArtifactId=com.liferay.project.templates.service.builder \
    -DgroupId=com.liferay \
    -DartifactId=tasks \
    -Dpackage=com.liferay.docs.tasks \
    -Dversion=1.0 \
    -DapiPath=com.liferay.api.path \
    -DdependencyInjector=ds \
    -DliferayVersion=7.2
\end{verbatim}

This task creates the \texttt{tasks-api} and \texttt{tasks-service}
folders. In many cases, a Service Builder project also requires a
\texttt{-web} folder to hold, for example, portlet classes. This should
be created manually. After running the Blade command above, your
project's directory structure looks like this:

\begin{itemize}
\tightlist
\item
  \texttt{tasks}

  \begin{itemize}
  \tightlist
  \item
    \texttt{gradle}

    \begin{itemize}
    \tightlist
    \item
      \texttt{wrapper}

      \begin{itemize}
      \tightlist
      \item
        \texttt{gradle-wrapper.jar}
      \item
        \texttt{gradle-wrapper.properties}
      \end{itemize}
    \end{itemize}
  \item
    \texttt{tasks-api}

    \begin{itemize}
    \tightlist
    \item
      \texttt{bnd.bnd}
    \item
      \texttt{build.gradle}
    \end{itemize}
  \item
    \texttt{tasks-service}

    \begin{itemize}
    \tightlist
    \item
      \texttt{bnd.bnd}
    \item
      \texttt{build.gradle}
    \item
      \texttt{service.xml}
    \end{itemize}
  \item
    \texttt{build.gradle}
  \item
    \texttt{gradlew}
  \item
    \texttt{settings.gradle}
  \end{itemize}
\end{itemize}

The Maven-generated project includes a \texttt{pom.xml} file and does
not include the Gradle-specific files, but otherwise, appears exactly
the same.

To generate your service and API classes for the \texttt{*-api} and
\texttt{*-service} modules, replace the \texttt{service.xml} file in the
\texttt{*-service} module. Depending on your build tool, you can build
your services by executing

\begin{verbatim}
blade gw buildService
\end{verbatim}

or

\begin{verbatim}
mvn service-builder:build
\end{verbatim}

from the \texttt{tasks} root directory. Note that \texttt{blade\ gw}
only works if the Gradle wrapper can be detected. To ensure the
availability of the Gradle wrapper, be sure to work in a Liferay
Workspace.

The \texttt{mvn\ service-builder:build} command only works if you're
using the \texttt{com.liferay.portal.tools.service.builder} plugin
version 1.0.145+. Maven projects using an earlier version of the Service
Builder plugin should update their POM accordingly.

The generated module is functional and is deployable to a Liferay DXP
instance. To build upon the generated app, modify the project by adding
logic and additional files to the folders outlined above.

For more information on Service Builder, see the
\href{/docs/7-2/appdev/-/knowledge_base/a/service-builder}{Service
Builder} section.

\chapter{Service Template}\label{service-template}

In this article, you'll learn how to create a Liferay service as a
Liferay module. To create a Liferay service via the command line using
Blade CLI or Maven, use one of the commands with the following
parameters:

\begin{verbatim}
blade create -t service [-p packageName] [-c className] [-s serviceName] projectName
\end{verbatim}

or

\begin{verbatim}
mvn archetype:generate \
    -DarchetypeGroupId=com.liferay \
    -DarchetypeArtifactId=com.liferay.project.templates.service \
    -DartifactId=[projectName] \
    -Dpackage=[packageName] \
    -DclassName=[className]
    -DserviceName=[serviceName] \
    -DliferayVersion=7.2
\end{verbatim}

You can also insert the \texttt{-b\ maven} parameter in the Blade
command to generate a Maven project using Blade CLI.

The template for this kind of project is \texttt{service}. Suppose you
want to create a service project called \texttt{my-service-project} with
a package name of \texttt{com.liferay.docs.service} and a class name of
\texttt{Service}. Also, you'd like to create a service of type
\texttt{com.liferay.portal.kernel.events.LifecycleAction} that also
implements that same service. You could run the following command to
accomplish this:

\begin{verbatim}
blade create -t service -p com.liferay.docs.service -c Service -s com.liferay.portal.kernel.events.LifecycleAction my-service-project
\end{verbatim}

or

\begin{verbatim}
mvn archetype:generate \
    -DarchetypeGroupId=com.liferay \
    -DarchetypeArtifactId=com.liferay.project.templates.service \
    -DgroupId=com.liferay \
    -DartifactId=my-service-project \
    -Dpackage=com.liferay.docs \
    -Dversion=1.0 \
    -DclassName=Service \
    -DclassName=com.liferay.portal.kernel.events.LifecycleAction \
    -Dauthor=Joe Bloggs \
    -DliferayVersion=7.2
\end{verbatim}

After running the Blade command above, your project's directory
structure would look like this

\begin{itemize}
\tightlist
\item
  \texttt{my-service-project}

  \begin{itemize}
  \tightlist
  \item
    \texttt{gradle}

    \begin{itemize}
    \tightlist
    \item
      \texttt{wrapper}

      \begin{itemize}
      \tightlist
      \item
        \texttt{gradle-wrapper.jar}
      \item
        \texttt{gradle-wrapper.properties}
      \end{itemize}
    \end{itemize}
  \item
    \texttt{src}

    \begin{itemize}
    \tightlist
    \item
      \texttt{main}

      \begin{itemize}
      \tightlist
      \item
        \texttt{java}

        \begin{itemize}
        \tightlist
        \item
          \texttt{com/liferay/docs/service}

          \begin{itemize}
          \tightlist
          \item
            \texttt{Service.java}
          \end{itemize}
        \end{itemize}
      \item
        \texttt{resources}
      \end{itemize}
    \end{itemize}
  \item
    \texttt{bnd.bnd}
  \item
    \texttt{build.gradle}
  \item
    \texttt{gradlew}
  \end{itemize}
\end{itemize}

The Maven-generated project includes a \texttt{pom.xml} file and does
not include the Gradle-specific files, but otherwise, appears exactly
the same.

The generated module is functional and is deployable to a Liferay DXP
instance. To build upon the generated app, modify the project by adding
logic and additional files to the folders outlined above.

\chapter{Service Wrapper Template}\label{service-wrapper-template}

In this article, you'll learn how to create a Liferay service wrapper as
a Liferay module. To create a Liferay service wrapper via the command
line using Blade CLI or Maven, use one of the commands with the
following parameters:

\begin{verbatim}
blade create -t service-wrapper [-p packageName] [-c className] [-s serviceWrapperClass] projectName
\end{verbatim}

or

\begin{verbatim}
mvn archetype:generate \
    -DarchetypeGroupId=com.liferay \
    -DarchetypeArtifactId=com.liferay.project.templates.service.wrapper \
    -DartifactId=[projectName] \
    -Dpackage=[packageName] \
    -DclassName=[className] \
    -DserviceWrapperClass=[serviceWrapperClass] \
    -DliferayVersion=7.2
\end{verbatim}

You can also insert the \texttt{-b\ maven} parameter in the Blade
command to generate a Maven project using Blade CLI.

The template for this kind of project is \texttt{service-wrapper}.
Suppose you want to create a service wrapper project called
\texttt{service-override} with a package name of
\texttt{com.liferay.docs.serviceoverride} and a class name of
\texttt{UserLocalServiceOverride}. Also, you'd like to create a service
of type \texttt{com.liferay.portal.kernel.service.ServiceWrapper} that
extends the \texttt{com.liferay.portal.service.UserLocalServiceWrapper}
class. You could run the following command to accomplish this:

\begin{verbatim}
blade create -t service-wrapper -p com.liferay.docs.serviceoverride -c UserLocalServiceOverride -s com.liferay.portal.kernel.service.UserLocalServiceWrapper service-override
\end{verbatim}

or

\begin{verbatim}
mvn archetype:generate \
    -DarchetypeGroupId=com.liferay \
    -DarchetypeArtifactId=com.liferay.project.templates.service.wrapper \
    -DgroupId=com.liferay \
    -DartifactId=service-override \
    -Dpackage=com.liferay.docs.serviceoverride \
    -Dversion=1.0 \
    -DclassName=UserLocalServiceOverride \
    -DserviceWrapperClass=com.liferay.portal.kernel.service.UserLocalServiceWrapper \
    -Dauthor=Joe Bloggs \
    -DliferayVersion=7.2
\end{verbatim}

Here, \emph{service} means an OSGi service, not a Liferay API. Another
way to say \emph{service type} is to say \emph{component type}.

After running the Blade command above, your project's directory
structure looks like this:

\begin{itemize}
\tightlist
\item
  \texttt{service-override}

  \begin{itemize}
  \tightlist
  \item
    \texttt{gradle}

    \begin{itemize}
    \tightlist
    \item
      \texttt{wrapper}

      \begin{itemize}
      \tightlist
      \item
        \texttt{gradle-wrapper.jar}
      \item
        \texttt{gradle-wrapper.properties}
      \end{itemize}
    \end{itemize}
  \item
    \texttt{src}

    \begin{itemize}
    \tightlist
    \item
      \texttt{main}

      \begin{itemize}
      \tightlist
      \item
        \texttt{java}

        \begin{itemize}
        \tightlist
        \item
          \texttt{com/liferay/docs/serviceoverride}

          \begin{itemize}
          \tightlist
          \item
            \texttt{UserLocalServiceOverride.java}
          \end{itemize}
        \end{itemize}
      \item
        \texttt{resources}
      \end{itemize}
    \end{itemize}
  \item
    \texttt{bnd.bnd}
  \item
    \texttt{build.gradle}
  \item
    \texttt{gradlew}
  \end{itemize}
\end{itemize}

The Maven-generated project includes a \texttt{pom.xml} file and does
not include the Gradle-specific files, but otherwise, appears exactly
the same.

The generated module is a working application and is deployable to a
Liferay DXP instance. To build upon the generated app, modify the
project by adding logic and additional files to the folders outlined
above.

\chapter{Simulation Panel Entry
Template}\label{simulation-panel-entry-template}

In this article, you'll learn how to create a Liferay simulation panel
entry as a Liferay module. To create a simulation panel entry via the
command line using Blade CLI or Maven, use one of the commands with the
following parameters:

\begin{verbatim}
blade create -t simulation-panel-entry [-p packageName] [-c className] projectName
\end{verbatim}

or

\begin{verbatim}
mvn archetype:generate \
    -DarchetypeGroupId=com.liferay \
    -DarchetypeArtifactId=com.liferay.project.templates.simulation.panel.entry \
    -DartifactId=[projectName] \
    -Dpackage=[packageName] \
    -DclassName=[className] \
    -DliferayVersion=7.2
\end{verbatim}

You can also insert the \texttt{-b\ maven} parameter in the Blade
command to generate a Maven project using Blade CLI.

The template for this kind of project is
\texttt{simulation-panel-entry}. Suppose you want to create a simulation
panel entry project called \texttt{my-simulation-panel-entry} with a
package name of \texttt{com.liferay.docs.application.list} and a class
name of \texttt{SampleSimulationPanelApp}. You could run the following
command to accomplish this:

\begin{verbatim}
blade create -t simulation-panel-entry -p com.liferay.docs -c Sample my-simulation-panel-entry
\end{verbatim}

or

\begin{verbatim}
mvn archetype:generate \
    -DarchetypeGroupId=com.liferay \
    -DarchetypeArtifactId=com.liferay.project.templates.simulation.panel.entry \
    -DgroupId=com.liferay \
    -DartifactId=my-simulation-panel-entry \
    -Dpackage=com.liferay.docs \
    -Dversion=1.0 \
    -DclassName=Sample \
    -Dauthor=Joe Bloggs \
    -DliferayVersion=7.2
\end{verbatim}

After running the Blade command above, your project's directory
structure would look like this

\begin{itemize}
\tightlist
\item
  \texttt{my-simulation-panel-entry}

  \begin{itemize}
  \tightlist
  \item
    \texttt{gradle}

    \begin{itemize}
    \tightlist
    \item
      \texttt{wrapper}

      \begin{itemize}
      \tightlist
      \item
        \texttt{gradle-wrapper.jar}
      \item
        \texttt{gradle-wrapper.properties}
      \end{itemize}
    \end{itemize}
  \item
    \texttt{src}

    \begin{itemize}
    \tightlist
    \item
      \texttt{main}

      \begin{itemize}
      \tightlist
      \item
        \texttt{java}

        \begin{itemize}
        \tightlist
        \item
          \texttt{com/liferay/docs/application/list}

          \begin{itemize}
          \tightlist
          \item
            \texttt{SampleSimulationPanelApp.java}
          \end{itemize}
        \end{itemize}
      \item
        \texttt{resources}

        \begin{itemize}
        \tightlist
        \item
          \texttt{content}

          \begin{itemize}
          \tightlist
          \item
            \texttt{Language.properties}
          \end{itemize}
        \item
          \texttt{META-INF}

          \begin{itemize}
          \tightlist
          \item
            \texttt{resources}

            \begin{itemize}
            \tightlist
            \item
              \texttt{simulation\_panel.jsp}
            \end{itemize}
          \end{itemize}
        \end{itemize}
      \end{itemize}
    \end{itemize}
  \item
    \texttt{bnd.bnd}
  \item
    \texttt{build.gradle}
  \item
    \texttt{gradlew}
  \end{itemize}
\end{itemize}

The Maven-generated project includes a \texttt{pom.xml} file and does
not include the Gradle-specific files, but otherwise, appears exactly
the same.

The generated module is functional and is deployable to a Liferay DXP
instance. To build upon the generated app, modify the project by adding
logic and additional files to the folders outlined above. You can visit
the
\href{https://github.com/liferay/liferay-blade-samples/tree/master/gradle/apps/simulation-panel-app}{simulation-panel-app}
sample project for a more expanded sample of a control menu entry.

\chapter{Social Bookmark Template}\label{social-bookmark-template}

In this article, you'll learn how to create a Liferay social bookmark as
a Liferay module. To create a social bookmark as a module via the
command line using Blade CLI or Maven, use one of the commands with the
following parameters:

\begin{verbatim}
blade create -t social-bookmark [-p packageName] [-c className] projectName
\end{verbatim}

or

\begin{verbatim}
mvn archetype:generate \
    -DarchetypeGroupId=com.liferay \
    -DarchetypeArtifactId=com.liferay.project.templates.social.bookmark \
    -DartifactId=[projectName] \
    -Dpackage=[packageName] \
    -DclassName=[className] \
    -DliferayVersion=7.2
\end{verbatim}

You can also insert the \texttt{-b\ maven} parameter in the Blade
command to generate a Maven project using Blade CLI.

The template for this kind of project is \texttt{social-bookmark}.
Suppose you want to create a social bookmark project called
\texttt{my-social-bookmark-project} with a package name of
\texttt{com.liferay.docs.socialbookmark} and a class name of
\texttt{TestSocialBookmark}. You could run the following command to
accomplish this:

\begin{verbatim}
blade create -t social-bookmark -p com.liferay.docs.socialbookmark -c Test my-social-bookmark-project
\end{verbatim}

or

\begin{verbatim}
mvn archetype:generate \
    -DarchetypeGroupId=com.liferay \
    -DarchetypeArtifactId=com.liferay.project.templates.social.bookmark \
    -DgroupId=com.liferay \
    -DartifactId=my-social-bookmark-project \
    -Dpackage=com.liferay.docs.socialbookmark \
    -Dversion=1.0 \
    -DclassName=Test \
    -Dauthor=Joe Bloggs \
    -DliferayVersion=7.2
\end{verbatim}

After running the Blade command above, your project's directory
structure looks like this:

\begin{itemize}
\tightlist
\item
  \texttt{my-social-bookmark-project}

  \begin{itemize}
  \tightlist
  \item
    \texttt{gradle}

    \begin{itemize}
    \tightlist
    \item
      \texttt{wrapper}

      \begin{itemize}
      \tightlist
      \item
        \texttt{gradle-wrapper.jar}
      \item
        \texttt{gradle-wrapper.properties}
      \end{itemize}
    \end{itemize}
  \item
    \texttt{src}

    \begin{itemize}
    \tightlist
    \item
      \texttt{main}

      \begin{itemize}
      \tightlist
      \item
        \texttt{java}

        \begin{itemize}
        \tightlist
        \item
          \texttt{com/liferay/docs/socialbookmark}

          \begin{itemize}
          \tightlist
          \item
            \texttt{social/bookmark}

            \begin{itemize}
            \tightlist
            \item
              \texttt{TestSocialBookmark}
            \end{itemize}
          \end{itemize}
        \end{itemize}
      \item
        \texttt{resources}

        \begin{itemize}
        \tightlist
        \item
          \texttt{content}

          \begin{itemize}
          \tightlist
          \item
            \texttt{Language.properties}
          \end{itemize}
        \item
          \texttt{META-INF}

          \begin{itemize}
          \tightlist
          \item
            \texttt{resources}

            \begin{itemize}
            \tightlist
            \item
              \texttt{icons.svg}
            \item
              \texttt{init.jsp}
            \item
              \texttt{page.jsp}
            \end{itemize}
          \end{itemize}
        \end{itemize}
      \end{itemize}
    \end{itemize}
  \item
    \texttt{bnd.bnd}
  \item
    \texttt{build.gradle}
  \item
    \texttt{gradlew}
  \end{itemize}
\end{itemize}

The Maven-generated project includes a \texttt{pom.xml} file and does
not include the Gradle-specific files, but otherwise, appears exactly
the same.

The generated module is a working application and is deployable to a
Liferay DXP instance. An unmodified module generated as described above
creates a social bookmark named \emph{Test} that searches the current
URL using Google Search.

\begin{figure}
\centering
\includegraphics{./images/social-bookmark-project-template.png}
\caption{Click the magnifying glass icon to search the current URL using
Google Search.}
\end{figure}

To build upon the generated app, modify the project by adding logic and
additional files to the folders outlined above. For more information on
developing social bookmarks, see the
\href{/docs/7-2/frameworks/-/knowledge_base/f/social-api}{Social API}
section of tutorials. For information on configuring social bookmarks
for the Blogs widget, see the
\href{/docs/7-2/user/-/knowledge_base/u/displaying-blogs}{Displaying
Blogs} article.

\chapter{Spring MVC Portlet Template}\label{spring-mvc-portlet-template}

In this article, you'll learn how to create a Liferay Spring MVC portlet
application as a WAR. To create a Liferay Spring MVC portlet via the
command line using Blade CLI or Maven, use one of the commands with the
following parameters:

\begin{verbatim}
blade create -t spring-mvc-portlet [-p packageName] [-c className] projectName
\end{verbatim}

or

\begin{verbatim}
mvn archetype:generate \
    -DarchetypeGroupId=com.liferay \
    -DarchetypeArtifactId=com.liferay.project.templates.spring.mvc.portlet \
    -DartifactId=[projectName] \
    -Dpackage=[packageName] \
    -DclassName=[className] \
    -DliferayVersion=7.2
\end{verbatim}

You can also insert the \texttt{-b\ maven} parameter in the Blade
command to generate a Maven project using Blade CLI.

The template for this kind of project is \texttt{spring-mvc-portlet}.
Suppose you want to create a Spring MVC portlet project called
\texttt{my-spring-mvc-portlet-project} with a package name of
\texttt{com.liferay.docs.springmvcportlet} and a class name of
\texttt{MySpringMvcPortlet}. Also, you'd like to create a
Spring-annotated portlet class named
\texttt{MySpringMvcPortletViewController}.

\begin{verbatim}
blade create -t spring-mvc-portlet -p com.liferay.docs.springmvcportlet -c MySpringMvcPortlet my-spring-mvc-portlet-project
\end{verbatim}

or

\begin{verbatim}
mvn archetype:generate \
    -DarchetypeGroupId=com.liferay \
    -DarchetypeArtifactId=com.liferay.project.templates.spring.mvc.portlet \
    -DgroupId=com.liferay \
    -DartifactId=my-spring-mvc-portlet-project \
    -Dpackage=com.liferay.docs.springmvcportlet \
    -Dversion=1.0 \
    -DclassName=MySpringMvcPortlet \
    -Dauthor=Joe Bloggs \
    -DliferayVersion=7.2
\end{verbatim}

After running the Blade command above, your project's directory
structure looks like this:

\begin{itemize}
\tightlist
\item
  \texttt{my-spring-mvc-portlet-project}

  \begin{itemize}
  \tightlist
  \item
    \texttt{gradle}

    \begin{itemize}
    \tightlist
    \item
      \texttt{wrapper}

      \begin{itemize}
      \tightlist
      \item
        \texttt{gradle-wrapper.jar}
      \item
        \texttt{gradle-wrapper.properties}
      \end{itemize}
    \end{itemize}
  \item
    \texttt{src}

    \begin{itemize}
    \tightlist
    \item
      \texttt{main}

      \begin{itemize}
      \tightlist
      \item
        \texttt{java}

        \begin{itemize}
        \tightlist
        \item
          \texttt{com/liferay/docs/springmvcportlet/portlet}

          \begin{itemize}
          \tightlist
          \item
            \texttt{MySpringMvcPortletViewController}
          \end{itemize}
        \end{itemize}
      \item
        \texttt{resources}

        \begin{itemize}
        \tightlist
        \item
          \texttt{content}

          \begin{itemize}
          \tightlist
          \item
            \texttt{Language.properties}
          \end{itemize}
        \end{itemize}
      \item
        \texttt{webapp}

        \begin{itemize}
        \tightlist
        \item
          \texttt{css}

          \begin{itemize}
          \tightlist
          \item
            \texttt{main.scss}
          \end{itemize}
        \item
          \texttt{WEB-INF}

          \begin{itemize}
          \tightlist
          \item
            \texttt{jsp}

            \begin{itemize}
            \tightlist
            \item
              \texttt{init.jsp}
            \item
              \texttt{view.jsp}
            \end{itemize}
          \item
            \texttt{spring-context}

            \begin{itemize}
            \tightlist
            \item
              \texttt{portlet}

              \begin{itemize}
              \tightlist
              \item
                \texttt{my-spring-mvc-portlet-project.xml}
              \end{itemize}
            \item
              \texttt{portlet-application-context.xml}
            \end{itemize}
          \item
            \texttt{tld}

            \begin{itemize}
            \tightlist
            \item
              \texttt{liferay-portlet.tld}
            \item
              \texttt{liferay-portlet-ext.tld}
            \item
              \texttt{liferay-security.tld}
            \item
              \texttt{liferay-theme.tld}
            \item
              \texttt{liferay-ui.tld}
            \item
              \texttt{liferay-util.tld}
            \end{itemize}
          \item
            \texttt{liferay-display.xml}
          \item
            \texttt{liferay-plugin-package.properties}
          \item
            \texttt{liferay-portlet.xml}
          \item
            \texttt{portlet.xml}
          \item
            \texttt{web.xml}
          \end{itemize}
        \item
          \texttt{icon.png}
        \end{itemize}
      \end{itemize}
    \end{itemize}
  \item
    \texttt{build.gradle}
  \item
    \texttt{gradlew}
  \end{itemize}
\end{itemize}

The Maven-generated project includes a \texttt{pom.xml} file and does
not include the Gradle-specific files, but otherwise, appears exactly
the same.

The generated WAR is a working application and is deployable to a
Liferay DXP instance. To build upon the generated app, modify the
project by adding logic and additional files to the folders outlined
above. You can visit the
\href{/docs/7-1/reference/-/knowledge_base/r/spring-mvc-portlet}{springmvc-portlet}
sample project for a more expanded sample of a Spring MVC portlet.

\chapter{Template Context Contributor
Template}\label{template-context-contributor-template}

In this article, you'll learn how to create a Liferay template context
contributor as a Liferay module. To create a template context
contributor via the command line using Blade CLI or Maven, use one of
the commands with the following parameters:

\begin{verbatim}
blade create -t template-context-contributor [-p packageName] [-c className] projectName
\end{verbatim}

or

\begin{verbatim}
mvn archetype:generate \
    -DarchetypeGroupId=com.liferay \
    -DarchetypeArtifactId=com.liferay.project.templates.template.context.contributor \
    -DartifactId=[projectName] \
    -Dpackage=[packageName] \
    -DclassName=[className] \
    -DliferayVersion=7.2
\end{verbatim}

You can also insert the \texttt{-b\ maven} parameter in the Blade
command to generate a Maven project using Blade CLI.

The template for this kind of project is
\texttt{template-context-contributor}. Suppose you want to create a
template context contributor project called
\texttt{my-template-context-contributor} with a package name of
\texttt{com.liferay.docs.theme.contributor} and a class name of
\texttt{SampleTemplateContextContributor}. You could run the following
command to accomplish this:

\begin{verbatim}
blade create -t template-context-contributor -p com.liferay.docs -c Sample my-template-context-contributor
\end{verbatim}

or

\begin{verbatim}
mvn archetype:generate \
    -DarchetypeGroupId=com.liferay \
    -DarchetypeArtifactId=com.liferay.project.templates.template.context.contributor \
    -DgroupId=com.liferay \
    -DartifactId=my-template-context-contributor \
    -Dpackage=com.liferay.docs \
    -Dversion=1.0 \
    -DclassName=Sample \
    -Dauthor=Joe Bloggs \
    -DliferayVersion=7.2
\end{verbatim}

After running the Blade command above, your project's directory
structure would look like this

\begin{itemize}
\tightlist
\item
  \texttt{my-template-context-contributor}

  \begin{itemize}
  \tightlist
  \item
    \texttt{gradle}

    \begin{itemize}
    \tightlist
    \item
      \texttt{wrapper}

      \begin{itemize}
      \tightlist
      \item
        \texttt{gradle-wrapper.jar}
      \item
        \texttt{gradle-wrapper.properties}
      \end{itemize}
    \end{itemize}
  \item
    \texttt{src}

    \begin{itemize}
    \tightlist
    \item
      \texttt{main}

      \begin{itemize}
      \tightlist
      \item
        \texttt{java}

        \begin{itemize}
        \tightlist
        \item
          \texttt{com/liferay/docs/context/contributor}

          \begin{itemize}
          \tightlist
          \item
            \texttt{SampleTemplateContextContributor.java}
          \end{itemize}
        \end{itemize}
      \item
        \texttt{resources}
      \end{itemize}
    \end{itemize}
  \item
    \texttt{bnd.bnd}
  \item
    \texttt{build.gradle}
  \item
    \texttt{gradlew}
  \end{itemize}
\end{itemize}

The Maven-generated project includes a \texttt{pom.xml} file and does
not include the Gradle-specific files, but otherwise, appears exactly
the same.

The generated module is functional and is deployable to a Liferay DXP
instance. To build upon the generated app, modify the project by adding
logic and additional files to the folders outlined above. You can visit
the
\href{https://github.com/liferay/liferay-blade-samples/tree/master/gradle/themes/template-context-contributor}{template-context-contributor}
sample project for a more expanded sample of a template context
contributor. Likewise, see the
\href{/docs/7-2/frameworks/-/knowledge_base/f/injecting-additional-context-variables-and-functionality-into-your-theme-templates}{Context
Contributors} tutorial for instructions on customizing a template
context contributor project.

\chapter{Theme Contributor Template}\label{theme-contributor-template}

In this article, you'll learn how to create a Liferay theme contributor
as a Liferay module. To create a theme contributor via the command line
using Blade CLI or Maven, use one of the commands with the following
parameters:

\begin{verbatim}
blade create -t theme-contributor [--contributorType contributorType] [-p packageName] projectName
\end{verbatim}

or

\begin{verbatim}
mvn archetype:generate \
    -DarchetypeGroupId=com.liferay \
    -DarchetypeArtifactId=com.liferay.project.templates.theme.contributor \
    -DartifactId=[projectName] \
    -Dpackage=[packageName] \
    -DcontributorType=[contributorType] \
    -DliferayVersion=7.2
\end{verbatim}

You can also insert the \texttt{-b\ maven} parameter in the Blade
command to generate a Maven project using Blade CLI.

The template for this kind of project is \texttt{theme-contributor}.
Suppose you want to create a theme contributor project called
\texttt{my-theme-contributor} with a package name of
\texttt{com.liferay.docs.theme.contributor} and a contributor type of
\texttt{my-contributor}. You could run the following command to
accomplish this:

\begin{verbatim}
blade create -t theme-contributor --contributor-type my-contributor -p com.liferay.docs.theme.contributor my-theme-contributor
\end{verbatim}

or

\begin{verbatim}
mvn archetype:generate \
    -DarchetypeGroupId=com.liferay \
    -DarchetypeArtifactId=com.liferay.project.templates.theme.contributor \
    -DgroupId=com.liferay \
    -DartifactId=my-theme-contributor \
    -Dpackage=com.liferay.docs.theme.contributor \
    -Dversion=1.0 \
    -DcontributorType=my-contributor \
    -DliferayVersion=7.2
\end{verbatim}

After running the Blade command above, your project's folder structure
would look like this:

\begin{itemize}
\tightlist
\item
  \texttt{my-theme-contributor}

  \begin{itemize}
  \tightlist
  \item
    \texttt{gradle}

    \begin{itemize}
    \tightlist
    \item
      \texttt{wrapper}

      \begin{itemize}
      \tightlist
      \item
        \texttt{gradle-wrapper.jar}
      \item
        \texttt{gradle-wrapper.properties}
      \end{itemize}
    \end{itemize}
  \item
    \texttt{src}

    \begin{itemize}
    \tightlist
    \item
      \texttt{main}

      \begin{itemize}
      \tightlist
      \item
        \texttt{java}

        \begin{itemize}
        \tightlist
        \item
          \texttt{com/liferay/docs/theme/contributor}
        \end{itemize}
      \item
        \texttt{resources}

        \begin{itemize}
        \tightlist
        \item
          \texttt{META-INF}

          \begin{itemize}
          \tightlist
          \item
            \texttt{resources}

            \begin{itemize}
            \tightlist
            \item
              \texttt{css}

              \begin{itemize}
              \tightlist
              \item
                \texttt{my-contributor}

                \begin{itemize}
                \tightlist
                \item
                  \texttt{\_body.scss}
                \item
                  \texttt{\_control\_menu.scss}
                \item
                  \texttt{\_product\_menu.scss}
                \item
                  \texttt{\_simulation\_panel.scss}
                \end{itemize}
              \item
                \texttt{my-contributor.scss}
              \end{itemize}
            \item
              \texttt{js}

              \begin{itemize}
              \tightlist
              \item
                \texttt{my-contributor.js}
              \end{itemize}
            \end{itemize}
          \end{itemize}
        \end{itemize}
      \end{itemize}
    \end{itemize}
  \item
    \texttt{bnd.bnd}
  \item
    \texttt{build.gradle}
  \item
    \texttt{gradlew}
  \end{itemize}
\end{itemize}

The Maven-generated project includes a \texttt{pom.xml} file and does
not include the Gradle-specific files, but otherwise, appears exactly
the same.

The generated module is functional and is deployable to a Liferay DXP
instance. To build upon the generated app, modify the project by adding
logic and additional files to the folders outlined above. You can visit
the
\href{/docs/7-2/reference/-/knowledge_base/r/theme-contributor}{Blade
Theme Contributor} sample project for a more expanded sample of a theme
contributor. Likewise, see the
\href{/docs/7-2/frameworks/-/knowledge_base/f/packaging-independent-ui-resources-for-your-site}{Theme
Contributors} tutorial for instructions on customizing a theme
contributor project.

\chapter{Theme Template}\label{theme-template}

In this article, you'll learn how to create a Liferay theme as a WAR
project. To create a Liferay theme via the command line using Blade CLI
or Maven, use one of the commands with the following parameters:

\begin{verbatim}
blade create -t theme projectName
\end{verbatim}

or

\begin{verbatim}
mvn archetype:generate \
    -DarchetypeGroupId=com.liferay \
    -DarchetypeArtifactId=com.liferay.project.templates.theme \
    -DartifactId=[projectName] \
    -DliferayVersion=7.2
\end{verbatim}

You can also insert the \texttt{-b\ maven} parameter in the Blade
command to generate a Maven project using Blade CLI.

The template for this kind of project is \texttt{theme}. Suppose you
want to create a theme project called \texttt{my-theme-project} as a WAR
file. You could run the following command to accomplish this:

\begin{verbatim}
blade create -t theme my-theme-project
\end{verbatim}

or

\begin{verbatim}
mvn archetype:generate \
    -DarchetypeGroupId=com.liferay \
    -DarchetypeArtifactId=com.liferay.project.templates.theme \
    -DgroupId=com.liferay \
    -DartifactId=my-theme-project \
    -Dversion=1.0 \
    -DliferayVersion=7.2
\end{verbatim}

After running the Blade command above, your project's folder structure
looks like this:

\begin{itemize}
\tightlist
\item
  \texttt{my-theme-project}

  \begin{itemize}
  \tightlist
  \item
    \texttt{gradle}

    \begin{itemize}
    \tightlist
    \item
      \texttt{wrapper}

      \begin{itemize}
      \tightlist
      \item
        \texttt{gradle-wrapper.jar}
      \item
        \texttt{gradle-wrapper.properties}
      \end{itemize}
    \end{itemize}
  \item
    \texttt{src}

    \begin{itemize}
    \tightlist
    \item
      \texttt{main}

      \begin{itemize}
      \tightlist
      \item
        \texttt{resources}

        \begin{itemize}
        \tightlist
        \item
          \texttt{resources-importer}

          \begin{itemize}
          \tightlist
          \item
            \texttt{sitemap.json}
          \end{itemize}
        \end{itemize}
      \item
        \texttt{webapp}

        \begin{itemize}
        \tightlist
        \item
          \texttt{css}

          \begin{itemize}
          \tightlist
          \item
            \texttt{\_custom.scss}
          \end{itemize}
        \item
          \texttt{WEB-INF}

          \begin{itemize}
          \tightlist
          \item
            \texttt{liferay-plugin-package.properties}
          \item
            \texttt{web.xml}
          \end{itemize}
        \end{itemize}
      \end{itemize}
    \end{itemize}
  \item
    \texttt{build.gradle}
  \item
    \texttt{gradlew}
  \end{itemize}
\end{itemize}

The Maven-generated project includes a \texttt{pom.xml} file and does
not include the Gradle-specific files, but otherwise, appears exactly
the same.

The generated theme is functional and is deployable to a Liferay DXP
instance. To build upon the generated project, modify the project by
adding logic and additional files to the folders outlined above. You can
visit the
\href{/docs/7-1/reference/-/knowledge_base/r/theme}{simple-theme}
project for a more expanded sample of a theme. Likewise, see the
\href{/docs/7-2/frameworks/-/knowledge_base/f/themes-introduction}{Themes}
section for more information on creating themes.

\chapter{WAR Core Ext}\label{war-core-ext}

In this article, you'll learn how to create a Liferay WAR core Ext
project. To create a WAR core Ext project via the command line using
Blade CLI or Maven, use one of the commands with the following
parameters:

\begin{verbatim}
blade create -t war-core-ext projectName
\end{verbatim}

or

\begin{verbatim}
mvn archetype:generate \
    -DarchetypeGroupId=com.liferay \
    -DarchetypeArtifactId=com.liferay.project.templates.war.core.ext \
    -DartifactId=[projectName] \
    -DliferayVersion=7.2
\end{verbatim}

You can also insert the \texttt{-b\ maven} parameter in the Blade
command to generate a Maven project using Blade CLI.

The template for this kind of project is \texttt{war-core-ext}. Suppose
you want to create a WAR core Ext project called
\texttt{my-war-core-ext-project}. You could run the following command to
accomplish this:

\begin{verbatim}
blade create -t war-core-ext my-war-core-ext-project
\end{verbatim}

or

\begin{verbatim}
mvn archetype:generate \
    -DarchetypeGroupId=com.liferay \
    -DarchetypeArtifactId=com.liferay.project.templates.war.core-ext \
    -DgroupId=com.liferay \
    -DartifactId=my-war-core-ext-project \
    -DliferayVersion=7.2
\end{verbatim}

After running the Blade command above, your project's folder structure
looks like this:

\begin{itemize}
\tightlist
\item
  \texttt{my-war-core-ext-project}

  \begin{itemize}
  \tightlist
  \item
    \texttt{gradle}

    \begin{itemize}
    \tightlist
    \item
      \texttt{wrapper}

      \begin{itemize}
      \tightlist
      \item
        \texttt{gradle-wrapper.jar}
      \item
        \texttt{gradle-wrapper.properties}
      \end{itemize}
    \end{itemize}
  \item
    \texttt{src}

    \begin{itemize}
    \tightlist
    \item
      \texttt{extImpl}

      \begin{itemize}
      \tightlist
      \item
        \texttt{java}
      \item
        \texttt{resources}

        \begin{itemize}
        \tightlist
        \item
          \texttt{META-INF}

          \begin{itemize}
          \tightlist
          \item
            \texttt{ext-model-hints.xml}
          \item
            \texttt{ext-spring.xml}
          \item
            \texttt{portal-log4j-ext.xml}
          \end{itemize}
        \end{itemize}
      \end{itemize}
    \item
      \texttt{extKernel}

      \begin{itemize}
      \tightlist
      \item
        \texttt{java}
      \item
        \texttt{resources}

        \begin{itemize}
        \tightlist
        \item
          \texttt{META-INF}
        \end{itemize}
      \end{itemize}
    \item
      \texttt{extUtilBridges}

      \begin{itemize}
      \tightlist
      \item
        \texttt{java}
      \item
        \texttt{resources}

        \begin{itemize}
        \tightlist
        \item
          \texttt{META-INF}
        \end{itemize}
      \end{itemize}
    \item
      \texttt{extUtilJava}

      \begin{itemize}
      \tightlist
      \item
        \texttt{java}
      \item
        \texttt{resources}

        \begin{itemize}
        \tightlist
        \item
          \texttt{META-INF}
        \end{itemize}
      \end{itemize}
    \item
      \texttt{extUtilTaglib}

      \begin{itemize}
      \tightlist
      \item
        \texttt{java}
      \item
        \texttt{resources}

        \begin{itemize}
        \tightlist
        \item
          \texttt{META-INF}
        \end{itemize}
      \end{itemize}
    \item
      \texttt{main}

      \begin{itemize}
      \tightlist
      \item
        \texttt{webapp}

        \begin{itemize}
        \tightlist
        \item
          \texttt{WEB-INF}

          \begin{itemize}
          \tightlist
          \item
            \texttt{ext-web}

            \begin{itemize}
            \tightlist
            \item
              \texttt{docroot}

              \begin{itemize}
              \tightlist
              \item
                \texttt{WEB-INF}

                \begin{itemize}
                \tightlist
                \item
                  \texttt{liferay-portlet-ext.xml}
                \item
                  \texttt{portlet-ext.xml}
                \item
                  \texttt{struts-config-ext.xml}
                \item
                  \texttt{tiles-defs-ext.xml}
                \item
                  \texttt{web.xml}
                \end{itemize}
              \end{itemize}
            \end{itemize}
          \item
            \texttt{liferay-plugin-package.properties}
          \end{itemize}
        \end{itemize}
      \end{itemize}
    \end{itemize}
  \item
    \texttt{build.gradle}
  \item
    \texttt{gradlew}
  \end{itemize}
\end{itemize}

The Maven-generated project includes a \texttt{pom.xml} file and does
not include the Gradle-specific files, but otherwise, appears exactly
the same.

\noindent\hrulefill

\textbf{Note:} If you generate a WAR Ext project using Gradle outside of
Liferay Workspace, you must set the \texttt{app.server.parent.dir}
property in the project's \texttt{gradle.properties}. The app server
location is required for this project to compile.

\noindent\hrulefill

The generated WAR Ext project is functional and is deployable to a
Liferay DXP instance. To build upon the generated project, modify the
project by adding logic and additional files to the folders outlined
above. Deploying WAR Ext projects is only supported for limited use
cases; it is recommended to leverage provided extension points offered
in Liferay DXP. You can visit the
\href{/docs/7-2/customization/-/knowledge_base/c/customization-with-ext}{Customization
with Ext} section for info on how to do this.

\chapter{WAR Hook Template}\label{war-hook-template}

In this article, you'll learn how to create a Liferay WAR hook project.
To create a Liferay WAR hook via the command line using Blade CLI or
Maven, use one of the commands with the following parameters:

\begin{verbatim}
blade create -t war-hook [-p packageName] [-c className] projectName
\end{verbatim}

or

\begin{verbatim}
mvn archetype:generate \
    -DarchetypeGroupId=com.liferay \
    -DarchetypeArtifactId=com.liferay.project.templates.war.hook \
    -DartifactId=[projectName]
    -Dpackage=[packageName] \
    -DclassName=[className] \
    -DliferayVersion=7.2
\end{verbatim}

You can also insert the \texttt{-b\ maven} parameter in the Blade
command to generate a Maven project using Blade CLI.

The template for this kind of project is \texttt{war-hook}. Suppose you
want to create a WAR hook project called \texttt{my-war-hook-project}
with a package name of \texttt{com.liferay.docs} and a class name of
\texttt{MyWarHook}. You could run the following command to accomplish
this:

\begin{verbatim}
blade create -t war-hook -p com.liferay.docs -c MyWarHook my-war-hook-project
\end{verbatim}

or

\begin{verbatim}
mvn archetype:generate \
    -DarchetypeGroupId=com.liferay \
    -DarchetypeArtifactId=com.liferay.project.templates.war.hook \
    -DgroupId=com.liferay \
    -DartifactId=my-war-hook-project \
    -Dpackage=com.liferay.docs \
    -DclassName=MyWarHook \
    -Dversion=1.0 \
    -DliferayVersion=7.2
\end{verbatim}

After running the Blade command above, your project's folder structure
looks like this:

\begin{itemize}
\tightlist
\item
  \texttt{my-war-hook-project}

  \begin{itemize}
  \tightlist
  \item
    \texttt{gradle}

    \begin{itemize}
    \tightlist
    \item
      \texttt{wrapper}

      \begin{itemize}
      \tightlist
      \item
        \texttt{gradle-wrapper.jar}
      \item
        \texttt{gradle-wrapper.properties}
      \end{itemize}
    \end{itemize}
  \item
    \texttt{src}

    \begin{itemize}
    \tightlist
    \item
      \texttt{main}

      \begin{itemize}
      \tightlist
      \item
        \texttt{java}

        \begin{itemize}
        \tightlist
        \item
          \texttt{com/liferay/docs}

          \begin{itemize}
          \tightlist
          \item
            \texttt{MyWarHookLoginPostAction}
          \item
            \texttt{MyWarHookStartupAction}
          \end{itemize}
        \end{itemize}
      \item
        \texttt{resources}

        \begin{itemize}
        \tightlist
        \item
          \texttt{portal.properties}
        \end{itemize}
      \item
        \texttt{webapp}

        \begin{itemize}
        \tightlist
        \item
          \texttt{WEB-INF}

          \begin{itemize}
          \tightlist
          \item
            \texttt{liferay-hook.xml}
          \item
            \texttt{liferay-plugin-package.properties}
          \item
            \texttt{web.xml}
          \end{itemize}
        \end{itemize}
      \end{itemize}
    \end{itemize}
  \item
    \texttt{build.gradle}
  \item
    \texttt{gradlew}
  \end{itemize}
\end{itemize}

The Maven-generated project includes a \texttt{pom.xml} file and does
not include the Gradle-specific files, but otherwise, appears exactly
the same.

The generated WAR hook is functional and is deployable to a Liferay DXP
instance. To build upon the generated project, modify the project by
adding logic and additional files to the folders outlined above.
Deploying WAR hooks is supported for 7.0, however, it is recommended to
optimize your WAR hooks to fragments or other applicable module
projects. You can visit the
\href{/docs/7-2/customization/-/knowledge_base/c/liferay-customization}{Liferay
Customization} section for info on how to do this for many project
types. See the
\href{/docs/6-2/tutorials/-/knowledge_base/t/customizing-liferay-portal}{Customizing
Liferay Portal} section for more information on WAR hooks.

\chapter{WAR MVC Portlet Template}\label{war-mvc-portlet-template}

In this article, you'll learn how to create a Liferay MVC portlet
project as a WAR file. To create a Liferay MVC portlet project as a WAR
via the command line using Blade CLI or Maven, use one of the commands
with the following parameters:

\begin{verbatim}
blade create -t war-mvc-portlet [-p packageName] projectName
\end{verbatim}

or

\begin{verbatim}
mvn archetype:generate \
    -DarchetypeGroupId=com.liferay \
    -DarchetypeArtifactId=com.liferay.project.templates.war.mvc.portlet \
    -DartifactId=[projectName]
    -Dpackage=[packageName] \
    -DliferayVersion=7.2
\end{verbatim}

You can also insert the \texttt{-b\ maven} parameter in the Blade
command to generate a Maven project using Blade CLI.

The template for this kind of project is \texttt{war-mvc-portlet}.
Suppose you want to create a WAR MVC portlet project called
\texttt{my-war-mvc-portlet-project} with a package name of
\texttt{com.liferay.docs.war.mvc} and a class name of
\texttt{MyWarMvcPortlet}. You could run the following command to
accomplish this:

\begin{verbatim}
blade create -t war-mvc-portlet -p com.liferay.docs.war.mvc my-war-mvc-portlet-project
\end{verbatim}

or

\begin{verbatim}
mvn archetype:generate \
    -DarchetypeGroupId=com.liferay \
    -DarchetypeArtifactId=com.liferay.project.templates.war.mvc.portlet \
    -DgroupId=com.liferay \
    -DartifactId=my-war-mvc-portlet-project \
    -Dpackage=com.liferay.docs.war.mvc \
    -Dversion=1.0 \
    -DliferayVersion=7.2
\end{verbatim}

After running the Blade command above, your project's folder structure
looks like this:

\begin{itemize}
\tightlist
\item
  \texttt{my-war-mvc-portlet-project}

  \begin{itemize}
  \tightlist
  \item
    \texttt{gradle}

    \begin{itemize}
    \tightlist
    \item
      \texttt{wrapper}

      \begin{itemize}
      \tightlist
      \item
        \texttt{gradle-wrapper.jar}
      \item
        \texttt{gradle-wrapper.properties}
      \end{itemize}
    \end{itemize}
  \item
    \texttt{src}

    \begin{itemize}
    \tightlist
    \item
      \texttt{main}

      \begin{itemize}
      \tightlist
      \item
        \texttt{java}

        \begin{itemize}
        \tightlist
        \item
          \texttt{com/liferay/docs/war/mvc}
        \end{itemize}
      \item
        \texttt{resources}

        \begin{itemize}
        \tightlist
        \item
          \texttt{content}

          \begin{itemize}
          \tightlist
          \item
            \texttt{Language.properties}
          \end{itemize}
        \end{itemize}
      \item
        \texttt{webapp}

        \begin{itemize}
        \tightlist
        \item
          \texttt{css}

          \begin{itemize}
          \tightlist
          \item
            \texttt{main.scss}
          \end{itemize}
        \item
          \texttt{WEB-INF}

          \begin{itemize}
          \tightlist
          \item
            \texttt{tld}

            \begin{itemize}
            \tightlist
            \item
              \texttt{liferay-portlet.tld}
            \item
              \texttt{liferay-portlet-ext.tld}
            \item
              \texttt{liferay-security.tld}
            \item
              \texttt{liferay-theme.tld}
            \item
              \texttt{liferay-ui.tld}
            \item
              \texttt{liferay-util.tld}
            \end{itemize}
          \item
            \texttt{liferay-display.xml}
          \item
            \texttt{liferay-plugin-package.properties}
          \item
            \texttt{liferay-portlet.xml}
          \item
            \texttt{portlet.xml}
          \item
            \texttt{web.xml}
          \end{itemize}
        \item
          \texttt{init.jsp}
        \item
          \texttt{view.jsp}
        \end{itemize}
      \end{itemize}
    \end{itemize}
  \item
    \texttt{build.gradle}
  \item
    \texttt{gradlew}
  \end{itemize}
\end{itemize}

The Maven-generated project includes a \texttt{pom.xml} file and does
not include the Gradle-specific files, but otherwise, appears exactly
the same.

The generated WAR MVC portlet is functional and is deployable to a
Liferay DXP instance. To build upon the generated project, modify the
project by adding logic and additional files to the folders outlined
above. Deploying WAR MVC portlets is supported for 7.0, however, it is
recommended to optimize your WAR portlet to a module project, if
possible. You can visit the
\href{/docs/7-2/tutorials/-/knowledge_base/t/upgrading-code-to-product-ver}{From
Liferay Portal 6 to 7} section for info on how to do this.

\chapter{Sample Projects}\label{sample-projects}

\noindent\hrulefill

\textbf{Note:} This section of articles does not provide documentation
for \emph{all} sample projects residing in the
\texttt{liferay-blade-samples} repo. The documentation for these samples
is in progress and will grow over time.

\noindent\hrulefill

Liferay provides sample projects that target different integration
points in Liferay DXP. These projects reside in the
\href{https://github.com/liferay/liferay-blade-samples}{liferay-blade-samples}
Github repository and can be easily copy/pasted to your local
environment. You can also generated them using
\href{/docs/7-2/reference/-/knowledge_base/r/generating-project-samples-with-blade-cli}{Blade
CLI}.

The sample projects are grouped into three different parent folders
based on the build tools used to generate them:

\begin{itemize}
\tightlist
\item
  \texttt{gradle}
\item
  \texttt{liferay-workspace}
\item
  \texttt{maven}
\end{itemize}

The provided sample projects are organized by their development
toolchains to cater to a variety of developers. Each folder offers the
same set of sample Liferay projects. Their only difference is that the
build files are specific to their toolchain. For example, the
\texttt{gradle} folder contains projects using standard OSS Gradle
plugins that can be added to any Gradle composite build. The same
concept also applies to the \texttt{liferay-workspace} and
\texttt{maven} projects.

\noindent\hrulefill

\textbf{Note:} The Liferay Workspace folder stores WAR-type samples in a
separate folder named
\href{https://github.com/liferay/liferay-blade-samples/tree/7.1/liferay-workspace/wars}{wars}.
The Gradle and Maven tool folders mix WAR samples with the other sample
types (apps, extensions, etc.).

\noindent\hrulefill

The \texttt{gradle} folder also uses the Liferay Gradle plugin (e.g.,
\texttt{com.liferay.plugin}) which encompasses additional functionality
for various types of Liferay projects. The Liferay Gradle plugin is
recommended for Gradle users developing for Liferay DXP.

Some samples also come configured with logging to help you fully
understand what the sample is accomplishing behind the scenes. For
example, OSGi module logging is implemented for several samples (e.g.,
\href{https://github.com/liferay/liferay-blade-samples/tree/7.1/gradle/apps/action-command-portlet}{action-command-portlet},
\href{/docs/7-2/reference/-/knowledge_base/r/document-action}{document-action},
\href{/docs/7-2/reference/-/knowledge_base/r/service-builder-application-using-external-database-via-jdbc}{service-builder/jdbc},
etc.), which lets OSGi modules supply their own logging configuration
defaults without external configuration. See the
\href{/docs/7-2/appdev/-/knowledge_base/a/adjusting-module-logging}{Adjusting
Module Logging} article for more information.

For a list of sample template projects available, visit the
\href{https://github.com/liferay/liferay-blade-samples\#liferay-extension-points-and-template-projects}{Liferay
extension points} sub-section in the Liferay Blade Samples repository.
This list is not comprehensive. A subset of missing extension point
samples can be found in the
\href{https://github.com/liferay/liferay-blade-samples\#liferay-extension-points-without-template-projects}{Liferay
extension points without template projects} sub-section. Visit the
repo's
\href{https://github.com/liferay/liferay-blade-samples\#contribution-guidelines}{Contribution
Guidelines} section for details on contributing to this repository.

\chapter{Apps}\label{apps}

This section focuses on Liferay sample applications. You can view these
sample apps by visiting the \texttt{apps} folder corresponding to your
preferred build tool:

\begin{itemize}
\tightlist
\item
  \href{https://github.com/liferay/liferay-blade-samples/tree/7.2/gradle/apps}{Gradle
  sample apps}
\item
  \href{https://github.com/liferay/liferay-blade-samples/tree/7.2/liferay-workspace/apps}{Liferay
  Workspace sample apps}
\item
  \href{https://github.com/liferay/liferay-blade-samples/tree/7.2/maven/apps}{Maven
  sample apps}
\end{itemize}

Visit a particular sample page to learn more!

\chapter{Service Builder Samples}\label{service-builder-samples}

This section focuses on Liferay Service Builder sample projects built
with various build tools. You can view these samples by visiting the
\texttt{apps/service-builder} folder corresponding to your preferred
build tool:

\begin{itemize}
\tightlist
\item
  \href{https://github.com/liferay/liferay-blade-samples/tree/7.2/gradle/apps/service-builder}{Gradle
  Service Builder sample apps}
\item
  \href{https://github.com/liferay/liferay-blade-samples/tree/7.2/liferay-workspace/apps/service-builder}{Liferay
  Service Builder Workspace sample apps}
\item
  \href{https://github.com/liferay/liferay-blade-samples/tree/7.2/maven/apps/service-builder}{Maven
  Service Builder sample apps}
\end{itemize}

The following Service Builder samples are documented:

\begin{itemize}
\tightlist
\item
  \href{/docs/7-2/reference/-/knowledge_base/r/service-builder-application-demonstrating-actionable-dynamic-query}{Service
  Builder application demonstrating Actionable Dynamic Query}
\item
  \href{/docs/7-2/reference/-/knowledge_base/r/service-builder-application-using-external-database-via-jdbc}{Service
  Builder application with JDBC connection}
\item
  \href{/docs/7-2/reference/-/knowledge_base/r/service-builder-application-using-external-database-via-jndi}{Service
  Builder application with JNDI connection}
\end{itemize}

Visit a particular sample page to learn more!

\chapter{Service Builder Application Demonstrating Actionable Dynamic
Query}\label{service-builder-application-demonstrating-actionable-dynamic-query}

This sample is similar to the
\href{https://github.com/liferay/liferay-blade-samples/tree/7.2/gradle/apps/service-builder/basic}{\texttt{basic}
Service Builder sample}, which lets you perform CRUD (create, read,
update, delete) operations on service builder entities. The distinctive
feature of the Service Builder Actionable Dynamic Query (ADQ) sample is
that it also lets you perform a mass update on all existing service
builder entities.

\begin{figure}
\centering
\includegraphics{./images/adq-sample.png}
\caption{This sample provides options to add entities and perform a mass
update.}
\end{figure}

To see the ADQ Service Builder sample in action, complete the following
steps:

\begin{enumerate}
\def\labelenumi{\arabic{enumi}.}
\item
  Add the sample to a page by navigating to \emph{Add}
  (\includegraphics{./images/icon-add.png}) → \emph{Widgets} →
  \emph{Sample} and dragging it to the page.
\item
  Select the app's \emph{Add} button and add an entity. Do this several
  times to create multiple entities.
\item
  Click the \emph{Mass Update} button and click \emph{Save} to invoke
  the update.

  After invoking the update, each entity's \texttt{field3} value (whose
  value is less than 100) is incremented.
\end{enumerate}

You've leveraged the actionable dynamic query API in your sample!

\section{What API(s) and/or code components does this sample
highlight?}\label{what-apis-andor-code-components-does-this-sample-highlight}

This sample demonstrates Liferay DXP's actionable dynamic query API.
Specifically, it demonstrates how to create an ADQ, add criteria to an
ADQ, specify an action for the ADQ, and execute the ADQ.

\section{How does this sample leverage the API(s) and/or code
component?}\label{how-does-this-sample-leverage-the-apis-andor-code-component}

An action request is sent to the \texttt{JSPPortlet} with a \texttt{cmd}
request parameter. When the \texttt{JSPPortlet}'s \texttt{processAction}
method processes the request, the value of the \texttt{cmd} parameter is
parsed and then the portlet's \texttt{massUpdate} method is invoked. The
\texttt{massUpdate} method, in turn, invokes the \texttt{massUpdate}
method defined in the \texttt{adq-service} module's
\texttt{BarLocalServiceImpl}. This is where the sample leverages the
actionable dynamic query API:

\begin{verbatim}
public void massUpdate() {
    ActionableDynamicQuery adq = getActionableDynamicQuery();

    adq.setAddCriteriaMethod(
        new ActionableDynamicQuery.AddCriteriaMethod() {

            @Override
            public void addCriteria(DynamicQuery dynamicQuery) {
                dynamicQuery.add(RestrictionsFactoryUtil.lt("field3", 100));
            }

        });

    adq.setPerformActionMethod(
        new ActionableDynamicQuery.PerformActionMethod<Bar>() {

            @Override
            public void performAction(Bar bar) {
                int field3 = bar.getField3();

                field3++;
                bar.setField3(field3);

                updateBar(bar);
            }

        });

    try {
        adq.performActions();
    }
    catch (Exception e) {
        e.printStackTrace();
    }
}
\end{verbatim}

For more information on the actionable dynamic query API, visit its
dedicated
\href{/docs/7-0/tutorials/-/knowledge_base/t/dynamic-query\#actionable-dynamic-queries}{tutorial}.

\chapter{Service Builder Application Using External Database via
JDBC}\label{service-builder-application-using-external-database-via-jdbc}

This sample demonstrates how to connect a Liferay Service Builder
application to an external database via a JDBC connection. Here, an
external database means any database other than Liferay DXP's database.
For this sample to work correctly, you must prepare such an external
database and configure Liferay DXP to use it. Follow the steps below to
make the required preparations before deploying the application.

\begin{enumerate}
\def\labelenumi{\arabic{enumi}.}
\item
  Create the external database to which your Service Builder application
  will connect. For example, create a MariaDB database called
  \texttt{external}. Add a table to this database called
  \texttt{country} with a \texttt{BIGINT} column called \texttt{Id} and
  a \texttt{VARCHAR(255)} column called \texttt{Name}. Add at least one
  record to this table. Here are the MariaDB commands to accomplish
  this:

\begin{verbatim}
create database external character set utf8;

use external;

create table country(id bigint not null primary key, name varchar(255));

insert into country(id, name) values(1, 'Australia');
\end{verbatim}

  Make sure that your database commands were successful: Running
  \texttt{select\ *\ from\ country;} should return the record you added.
\item
  Create a \texttt{portal-ext.properties} file in your Liferay DXP
  instance's \texttt{{[}LIFERAY\_HOME{]}} folder (this folder should be
  marked by the presence of a \texttt{.liferay-home} file). In your
  \texttt{portal-ext.properties} file, define the details of your JDBC
  data source connection:

\begin{verbatim}
jdbc.ext.driverClassName=org.mariadb.jdbc.Driver
jdbc.ext.password=userpassword
jdbc.ext.url=jdbc:mariadb://localhost/external?useUnicode=true&characterEncoding=UTF-8&useFastDateParsing=false
jdbc.ext.username=yourusername
\end{verbatim}

  Note that Liferay DXP's primary data source is specified by the
  \texttt{jdbc.default} prefix. These details are often specified in a
  \texttt{portal-setup-wizard.properties} file. Here, we've chosen to
  use the \texttt{jdbc.ext} prefix for our alternate data source.
\item
  Create a
  \texttt{com.liferay.blade.samples.jdbcservicebuilder.service-log4j-ext.xml}
  in your Liferay instance's \texttt{{[}LIFERAY\_HOME{]}/osgi/log4}
  folder. Create this folder if it doesn't yet exist. Add this content
  to the XML file that you created:

\begin{verbatim}
<?xml version="1.0"?>
<!DOCTYPE log4j:configuration SYSTEM "log4j.dtd">

<log4j:configuration xmlns:log4j="http://jakarta.apache.org/log4j/">
    <category name="com.liferay.blade.samples.jdbcservicebuilder.service.impl">
        <priority value="INFO" />
    </category>
</log4j:configuration>
\end{verbatim}

  This XML file defines the log level for the classes in the
  \texttt{com.liferay.blade.samples.jdbcservicebuilder.service.impl}
  package. The
  \texttt{com.liferay.blade.samples.jdbcservicebuilder.service.impl.CountryLocalServiceImpl}
  is the class that will produce log messages when the sample portlet is
  viewed.
\end{enumerate}

Now your sample is ready for deployment! Make sure to build and deploy
each of the three modules that comprise the sample application:

\begin{itemize}
\tightlist
\item
  \texttt{jdbc-api}
\item
  \texttt{jdbc-service}
\item
  \texttt{jdbc-web}
\end{itemize}

After these modules have been deployed, add the \texttt{-web} portlet to
a Liferay DXP page.

\begin{figure}
\centering
\includegraphics{./images/jdbc-sb-sample.png}
\caption{This sample prints out the values previously inputted into the
database.}
\end{figure}

A sample table is printed in the portlet's view, representing the info
inputted into the database.

\section{What API(s) and/or code components does this sample
highlight?}\label{what-apis-andor-code-components-does-this-sample-highlight-1}

The sample configures the data source using Spring Beans and
demonstrates two ways to access data from an external database defined
by a JDBC connection:

\begin{itemize}
\tightlist
\item
  extract data directly from the raw data source by explicitly
  specifying a SQL query.
\item
  read data using the helper methods that Service Builder generates in
  your application's persistence layer.
\end{itemize}

\section{How does this sample leverage the API(s) and/or code
component?}\label{how-does-this-sample-leverage-the-apis-andor-code-component-1}

Once you've added the \texttt{-web} portlet to a page, the
\texttt{CountryLocalService.useJDBC} method is invoked. This method
accesses the database defined by the JDBC connection you specified and
logs information about the rows in the \texttt{country} table to Liferay
DXP's log.

\section{Configuring the Data Source}\label{configuring-the-data-source}

The \texttt{-service} module's
\texttt{src/main/resources/META-INF/spring/ext-spring.xml} file
configures the external data source connection and applies the alias
\texttt{extDataSource} to the data source. The \texttt{service.xml} file
\texttt{entity} element specifies the data source via the attribute
assignment \texttt{data-source="extDataSource"}. The
\texttt{ext-spring.xml} and \texttt{service.xml} files demonstrate the
configuration steps explained in
\href{/docs/7-2/appdev/-/knowledge_base/a/connecting-the-data-source-using-spring-beans}{Connecting
the Data Source Using Spring Beans}.

\section{Accessing Data}\label{accessing-data}

The first way of accessing data from the external database is to extract
it directly from the raw data source by explicitly specifying a SQL
query. This technique is demonstrated by the
\texttt{CountryLocalServiceImpl.useJDBC} method. That method obtains the
Spring-defined data source that's injected into the
\texttt{countryPersistence} bean, opens a new connection, and reads data
from the data source. This is the technique used by the sample
application to write the data to Liferay DXP's log.

The second way of accessing data from the external database is to read
data using the helper methods that Service Builder generates in your
application's persistence layer. This technique is demonstrated by the
\texttt{UseJDBC.getCountries} method which first obtains an instance of
the \texttt{CountryLocalService} OSGi service and then invokes
\texttt{countryLocalService.getCountries}. The
\texttt{countryLocalService.getCountries} and
\texttt{countryLocalService.getCountriesCount} methods are two examples
of the persistence layer helper methods that Service Builder generates.
This is the technique used by the sample application to actually display
the data. The portlet's \texttt{view.jsp} uses the
\texttt{\textless{}search-container\textgreater{}} JSP tag to display a
list of results. The results are obtained by the
\texttt{UseJDBC.getCountries} method mentioned above.

\chapter{Service Builder Application Using External Database via
JNDI}\label{service-builder-application-using-external-database-via-jndi}

The \texttt{apps/service-builder/jndi} sample demonstrates how to
connect a Liferay Service Builder application to an external database
via a JNDI connection configured on the application server. Here, an
external database means any database other than Liferay DXP's database.
For this sample to work correctly, you must prepare such an external
database and configure your application server to use it.

\noindent\hrulefill

\textbf{Important:} Connecting to an external data source using JNDI is
broken in Portal CE 7.2 GA1 and GA2, and in DXP 7.2 releases prior to
FP5/SP2. See
\href{https://issues.liferay.com/browse/LPS-107733}{LPS-107733} for
details.

\noindent\hrulefill

Follow the steps below to make the required preparations before
deploying the application.

\begin{enumerate}
\def\labelenumi{\arabic{enumi}.}
\item
  Create an external database based on sample application's
  \texttt{service.xml}.

  \texttt{service.xml}:

\begin{verbatim}
<?xml version="1.0"?>
<!DOCTYPE service-builder PUBLIC "-//Liferay//DTD Service Builder 7.2.0//EN" "http://www.liferay.com/dtd/liferay-service-builder_7_2_0.dtd">

<service-builder package-path="com.liferay.blade.samples.jndiservicebuilder">
    <namespace>REGION</namespace>
    <!--<entity data-source="sampleDataSource" local-service="true" name="Foo" remote-service="false" session-factory="sampleSessionFactory" table="bar" tx-manager="sampleTransactionManager uuid="true"">-->
    <entity
        data-source="extDataSource"
        local-service="true"
        name="Region"
        remote-service="false"
        table="region"
        uuid="false"
    >
        <column db-name="id" name="regionId" primary="true" type="long" />
        <column db-name="name" name="regionName" type="String" />
    </entity>
</service-builder>
\end{verbatim}

  The entity's data source name \texttt{extDataSource} is arbitrary but
  must be specified in the data source configuration in the application
  server (next step).

  Here are MariaDB commands to create the database:

\begin{verbatim}
create database external character set utf8;

use external;

create table region(id bigint not null primary key, name varchar(255));

insert into region(id, name) values(1, 'Tasmania');
\end{verbatim}

  The database name is arbitrary; the data source configuration in your
  application server (next step), however, must specify this same
  database. The database table called \texttt{region} represents the
  service entity. The table has a \texttt{BIGINT} column called
  \texttt{Id} and a \texttt{VARCHAR(255)} column called \texttt{Name}.

  Add at least one record to this table. Running
  \texttt{select\ *\ from\ region;} should return the record you added.
\item
  In your application server configuration, define a JNDI connection to
  your database and map it to the \texttt{data-source} name (i.e.,
  \texttt{extDataSource}) that the sample \texttt{service.xml} entities
  specify.

  For example, if Tomcat is your application server, open your
  \texttt{{[}LIFERAY\_HOME{]}/tomcat-version/conf/server.xml} file and
  add a \texttt{Resource} element like this one inside of the
  \texttt{\textless{}GlobalNamingResources\textgreater{}} element:

\begin{verbatim}
<Resource
    name="jdbc/externalDataSource"
    auth="Container"
    type="javax.sql.DataSource"
    factory="org.apache.tomcat.jdbc.pool.DataSourceFactory"
    driverClassName="org.mariadb.jdbc.Driver"
    url="jdbc:mariadb://localhost/external?useUnicode=true&amp;characterEncoding=UTF-8&amp;useFastDateParsing=false"
    username="[place user name here]"
    password="[place password here]"
    maxActive="20"
    maxIdle="5"
    maxWait="10000"
/>
\end{verbatim}

  Replace the user name and password values and see the
  \href{/docs/7-2/deploy/-/knowledge_base/d/database-templates}{Database
  Templates} for the URL parameters to use for your database.
\item
  If you are using Tomcat, open your
  \texttt{{[}LIFERAY\_HOME{]}/tomcat-version/conf/context.xml} file and
  add this resource link element inside of the
  \texttt{\textless{}Context\textgreater{}} element:

\begin{verbatim}
<ResourceLink name="jdbc/externalDataSource" global="jdbc/externalDataSource" type="javax.sql.DataSource"/>
\end{verbatim}

  Now your data source is defined at Tomcat's scope.
\item
  Create a
  \texttt{com.liferay.blade.samples.jndiservicebuilder.service-log4j-ext.xml}
  in your Liferay DXP instance's \texttt{{[}LIFERAY\_HOME{]}/osgi/log4}
  folder. Create this folder if it doesn't yet exist. Add this content
  to the XML file that you created:

\begin{verbatim}
<?xml version="1.0"?>
<!DOCTYPE log4j:configuration SYSTEM "log4j.dtd">

<log4j:configuration xmlns:log4j="http://jakarta.apache.org/log4j/">
    <category name="com.liferay.blade.samples.jndiservicebuilder.service.impl">
        <priority value="INFO" />
    </category>
</log4j:configuration>
\end{verbatim}

  This XML file defines the log level for the classes in the
  \texttt{com.liferay.blade.samples.jndiservicebuilder.service.impl}
  package. The
  \texttt{com.liferay.blade.samples.jndiservicebuilder.service.impl.RegionLocalServiceImpl}
  is the class that will produce log messages when the sample portlet is
  viewed.
\end{enumerate}

Now your sample is ready for deployment! Make sure to build and deploy
each of the three modules that comprise the sample application:

\begin{itemize}
\tightlist
\item
  \texttt{jndi-api}
\item
  \texttt{jndi-service}
\item
  \texttt{jndi-web}
\end{itemize}

After these modules have been deployed, add the \texttt{jndi-web}
portlet to a Liferay DXP page.

\begin{figure}
\centering
\includegraphics{./images/jndi-sb-sample.png}
\caption{This sample prints out the values previously inputted into the
database.}
\end{figure}

A sample table is printed in the portlet's view, representing the info
inputted into the database.

\section{What API(s) and/or code components does this sample
highlight?}\label{what-apis-andor-code-components-does-this-sample-highlight-2}

This sample demonstrates two ways to access data from an external
database defined by a JNDI connection:

\begin{itemize}
\tightlist
\item
  extract data directly from the raw data source by explicitly
  specifying a SQL query.
\item
  read data using the helper methods that Service Builder generates in
  your application's persistence layer.
\end{itemize}

\section{How does this sample leverage the API(s) and/or code
component?}\label{how-does-this-sample-leverage-the-apis-andor-code-component-2}

Once you've added the \texttt{jndi-web} portlet to a page, the
\texttt{RegionLocalServiceUtil.useJNDI} method is invoked. This method
accesses the database defined by the JNDI connection you specified and
logs information about the rows in the \texttt{region} table to Liferay
DXP's log.

The first way of accessing data from the external database is to extract
data directly from the raw data source by explicitly specifying a SQL
query. This technique is demonstrated by the
\texttt{RegionLocalServiceImpl.useJNDI} method. That method obtains the
Spring-defined data source that's injected into the
\texttt{regionPersistence} bean, opens a new connection, and reads data
from the data source. This is the technique used by the sample
application to write the data to Liferay DXP's log.

The second way of accessing data from the external database is to read
data using the helper methods that Service Builder generates in your
application's persistence layer. This technique is demonstrated by the
\texttt{UseJNDI.getRegions} method which first obtains an instance of
the \texttt{RegionLocalService} OSGi service and then invokes
\texttt{regionLocalService.getRegions}. The
\texttt{regionLocalService.getRegions} and
\texttt{regionLocalService.getRegionsCount} methods are two examples of
the persistence layer helper methods that Service Builder generates.
This is the technique used by the sample application to actually display
the data. The portlet's \texttt{view.jsp} uses the
\texttt{\textless{}search-container\textgreater{}} JSP tag to display a
list of results. The results are obtained by the
\texttt{UseJNDI.getRegions} method mentioned above.

\section{Additional Information}\label{additional-information}

\begin{itemize}
\tightlist
\item
  \href{/docs/7-2/appdev/-/knowledge_base/a/connecting-service-builder-to-an-external-database}{Connecting
  to an External Data Source}
\end{itemize}

\chapter{Workflow Samples}\label{workflow-samples}

This section focuses on Liferay's Workflow Framework sample projects
built with various build tools. You can view these samples by visiting
the \texttt{apps/workflow} folder corresponding to your preferred build
tool:

\begin{itemize}
\tightlist
\item
  \href{https://github.com/liferay/liferay-blade-samples/tree/7.2/gradle/apps/workflow}{Gradle
  Workflow sample apps}
\item
  \href{https://github.com/liferay/liferay-blade-samples/tree/7.2/liferay-workspace/apps/workflow}{Liferay
  Workspace Workflow sample apps}
\item
  \href{https://github.com/liferay/liferay-blade-samples/tree/7.2/maven/apps/workflow}{Maven
  Workflow sample apps}
\end{itemize}

The following Workflow samples are documented:

\begin{itemize}
\tightlist
\item
  \href{/docs/7-2/reference/-/knowledge_base/r/workflow-application}{Workflow
  application}
\item
  \href{/docs/7-2/reference/-/knowledge_base/r/workflow-application-with-asset-integration}{Workflow
  application with Asset Integration}
\end{itemize}

Visit a particular sample page to learn more!

\chapter{Workflow Asset Application}\label{workflow-asset-application}

This sample demonstrates workflow enabling a model entity that is an
asset.

To see the Workflow sample in action, complete the following steps:

\begin{enumerate}
\def\labelenumi{\arabic{enumi}.}
\item
  Add the sample widget to a page by navigating to \emph{Add}
  (\includegraphics{./images/icon-add.png}) → \emph{Widgets} →
  \emph{Sample} → \emph{Workflow Asset} and dragging it to the page.
\item
  Go to \emph{Control Panel} → \emph{Workflow} → \emph{Process Builder}
  → \emph{Configuration} and assign a workflow to the Qux entity.
\item
  Select the app's \emph{Add} button and add an entity. Do this several
  times to create multiple entities.
\item
  Go to \emph{User} → \emph{My Workflow Tasks} → \emph{Assigned to My
  Roles} and assigned the task to me and Approve the Task.
\end{enumerate}

Now you've taken the entity and successfully run it through a workflow.

\section{What API(s) and/or code components does this sample
highlight?}\label{what-apis-andor-code-components-does-this-sample-highlight-3}

This sample demonstrates Liferay DXP's Workflow Handler API.
Specifically, it demonstrates how to create a \texttt{WorkflowHandler}
for your custom entity that is
\href{/docs/7-2/frameworks/-/knowledge_base/f/asset-framework}{asset
enabled}.

\section{How does this sample leverage the API(s) and/or code
component?}\label{how-does-this-sample-leverage-the-apis-andor-code-component-3}

The basic implementation of \texttt{WorkflowHandler} is done via
extension of the \texttt{BaseWorkflowHandler} class. This is where the
sample leverages the basic methods required for the entity's
\texttt{WorkflowHandler}.

\begin{verbatim}
@Override
public String getClassName() {
  return Qux.class.getName();
}

@Override
public String getTitle(long classPK, Locale locale) {
  return String.valueOf(classPK);
}

@Override
public String getType(Locale locale) {
  return ResourceActionsUtil.getModelResource(locale, getClassName());
}

@Override
public Qux updateStatus(
    int status, Map<String, Serializable> workflowContext)
  throws PortalException {

  long userId = GetterUtil.getLong(
    (String)workflowContext.get(WorkflowConstants.CONTEXT_USER_ID));

  long classPK = GetterUtil.getLong(
    (String)workflowContext.get(
      WorkflowConstants.CONTEXT_ENTRY_CLASS_PK));

  return _quxLocalService.updateStatus(userId, classPK, status);
}
\end{verbatim}

For more information on the workflow framework, visit its dedicated
\href{/docs/7-2/frameworks/-/knowledge_base/f/the-workflow-framework}{documentation}.

\chapter{Workflow Application}\label{workflow-application}

The
\href{https://github.com/liferay/liferay-blade-samples/tree/7.2/gradle/apps/workflow/basic}{\texttt{basic}}
sample demonstrates workflow enabling an entity that is not an asset.

To see the Workflow sample in action, complete the following steps:

\begin{enumerate}
\def\labelenumi{\arabic{enumi}.}
\item
  Add the sample widget to a page by navigating to \emph{Add}
  (\includegraphics{./images/icon-add.png}) → \emph{Widgets} →
  \emph{Sample} → \emph{Workflow Basic} and dragging it to the page.
\item
  Go to \emph{Control Panel} → \emph{Workflow} → \emph{Process Builder}
  → \emph{Configuration} and assign a workflow to the \texttt{Baz}
  entity.
\item
  Select the app's \emph{Add} button and add an entity. Do this several
  times to create multiple entities.
\item
  Go to \emph{User} → \emph{My Workflow Tasks} → \emph{Assigned to My
  Roles} and assigned the task to me and Approve the Task.
\end{enumerate}

Now you've taken the entity and successfully run it through a workflow.

\section{What API(s) and/or code components does this sample
highlight?}\label{what-apis-andor-code-components-does-this-sample-highlight-4}

This sample demonstrates Liferay DXP's Workflow Handler API.
Specifically, it demonstrates how to create a \texttt{WorkflowHandler}
for your custom entity.

\section{How does this sample leverage the API(s) and/or code
component?}\label{how-does-this-sample-leverage-the-apis-andor-code-component-4}

The basic implementation of \texttt{WorkflowHandler} is done via
extension of the \texttt{BaseWorkflowHandler} class. This is where the
sample leverages the basic methods required for the entity's
\texttt{WorkflowHandler}.

\begin{verbatim}
@Override
public String getClassName() {
  return Baz.class.getName();
}

@Override
public String getTitle(long classPK, Locale locale) {
  return String.valueOf(classPK);
}

@Override
public String getType(Locale locale) {
  return ResourceActionsUtil.getModelResource(locale, getClassName());
}

@Override
public Baz updateStatus(
    int status, Map<String, Serializable> workflowContext)
  throws PortalException {

  long userId = GetterUtil.getLong(
    (String)workflowContext.get(WorkflowConstants.CONTEXT_USER_ID));

  long classPK = GetterUtil.getLong(
    (String)workflowContext.get(
      WorkflowConstants.CONTEXT_ENTRY_CLASS_PK));

  return _bazLocalService.updateStatus(userId, classPK, status);
}
\end{verbatim}

For more information on the workflow framework, visit its dedicated
\href{/docs/7-2/frameworks/-/knowledge_base/f/the-workflow-framework}{documentation}.

\chapter{Greedy Policy Option
Application}\label{greedy-policy-option-application}

The Greedy Policy Option sample provides two portlets that can be added
to a Liferay DXP page: Greedy Portlet and Reluctant Portlet.

\begin{figure}
\centering
\includegraphics{./images/greedy-policy-app.png}
\caption{The Greedy Policy Option app provides two portlets that only
print text. You'll dive deeper later to discover their interesting
capabilities involving services.}
\end{figure}

These two portlets do not provide anything useful out-of-the-box. They
are, however, very effective at demonstrating the ability to reference
services using greedy and reluctant policy options. You'll learn how to
do this later.

\section{What API(s) and/or code components does this sample
highlight?}\label{what-apis-andor-code-components-does-this-sample-highlight-5}

This sample provides two modules referencing services using greedy and
reluctant policy options.

\begin{itemize}
\item
  \texttt{service-reference}: Provides an OSGi service interface called
  \texttt{SomeService}, a default implementation of it, and portlets
  that refer to service instances. One portlet refers to new higher
  ranked instances of the service automatically. The other portlet is
  reluctant to use new higher ranked instances and continues to use its
  bound service. The reluctant portlet can, however, be configured
  dynamically to use other service instances.
\item
  \texttt{higher-ranked-service}: Has a higher ranked
  \texttt{SomeService} implementation.
\end{itemize}

Here are each module's file structures:

\begin{itemize}
\tightlist
\item
  \texttt{service-reference/}

  \begin{itemize}
  \tightlist
  \item
    \texttt{bnd.bnd}
  \item
    \texttt{configs/}

    \begin{itemize}
    \tightlist
    \item
      \texttt{com.liferay.blade.reluctant.vs.greedy.portlet.portlet.ReluctantPortlet.config}
      → \texttt{ReluctantPortlet} configuration file
    \end{itemize}
  \item
    \texttt{src/main/java/com/liferay/blade/reluctant/vs/greedy/portlet/}

    \begin{itemize}
    \tightlist
    \item
      \texttt{api/}

      \begin{itemize}
      \tightlist
      \item
        \texttt{SomeService.java} → Service interface
      \end{itemize}
    \item
      \texttt{constants/}

      \begin{itemize}
      \tightlist
      \item
        \texttt{ReluctantPortletVsGreedyPortletKeys.java} → Portlet
        constants
      \end{itemize}
    \item
      \texttt{portlet/}

      \begin{itemize}
      \tightlist
      \item
        \texttt{DefaultSomeService.java} → Zero ranked service
        implementation
      \item
        \texttt{GreedyPortlet.java} → Refers to \texttt{SomeService}
        using a greedy service policy option
      \item
        \texttt{ReluctantPortletPortlet.java} → Refers to
        \texttt{SomeService} using a reluctant service policy option by
        default.
      \end{itemize}
    \end{itemize}
  \end{itemize}
\item
  \texttt{higher-ranked-service/}

  \begin{itemize}
  \tightlist
  \item
    \texttt{bnd.bnd}
  \item
    \texttt{src/main/java/com/liferay/blade/reluctant/vs/greedy/svc/HigherRankedService.java}
    → Service implementation with service ranking value of \texttt{100}
  \end{itemize}
\end{itemize}

\section{How does this sample leverage the API(s) and/or code
component?}\label{how-does-this-sample-leverage-the-apis-andor-code-component-5}

Here are the things you can learn using the sample modules:

\begin{enumerate}
\def\labelenumi{\arabic{enumi}.}
\item
  \hyperref[binding-a-newly-deployed-components-service-reference-to-the-highest-ranking-service-instance-thats-available-initially]{Binding
  a component's service reference to the highest ranked service instance
  that's available initially.}
\item
  \hyperref[deploying-a-module-with-a-higher-ranked-service-instance-for-binding-to-greedy-references-immediately]{Deploying
  a module with a higher ranked service instance for binding to greedy
  references immediately.}
\item
  \hyperref[configuring-a-component-to-reference-a-different-service-instance-dynamically]{Configuring
  a component to reference a different service instance dynamically.}
\end{enumerate}

Let's walk through the demonstration.

\section{Binding a newly deployed component's service reference to the
highest ranking service instance that's available
initially}\label{binding-a-newly-deployed-components-service-reference-to-the-highest-ranking-service-instance-thats-available-initially}

On deploying a component that references a service, it binds to the
highest ranking service instance that matches its target filter (if
specified).

The portlet classes refer to instances of interface
\texttt{SomeService}. The \texttt{doSomething} method returns a
\texttt{String}.

\begin{verbatim}
public interface SomeService {

    public String doSomething();

}
\end{verbatim}

Class \texttt{DefaultSomeService} implements \texttt{SomeService}. Its
\texttt{doSomething} method returns the \texttt{String} ``I am
Default!''.

\begin{verbatim}
@Component
public class DefaultSomeService implements SomeService {

    @Override
    public String doSomething() {
        return "I am Default!";
    }

}
\end{verbatim}

When module's portlets refer to \texttt{DefaultSomeService}, they
display the \texttt{String} ``I am Default!''.

The \texttt{ReluctantPortlet} class's \texttt{SomeService} reference's
policy option is the default: static and reluctant. This policy option
keeps the reference bound to its current service instance unless that
instance stops or the reference is reconfigured to refer to a different
service instance.

\begin{verbatim}
@Component(
   immediate = true,
   property = {
       "com.liferay.portlet.display-category=category.sample",
       "com.liferay.portlet.instanceable=true",
       "javax.portlet.display-name=Reluctant Portlet",
       "javax.portlet.init-param.template-path=/",
       "javax.portlet.init-param.view-template=/view.jsp",
       "javax.portlet.name=" + ReluctantVsGreedyPortletKeys.Reluctant,
       "javax.portlet.resource-bundle=content.Language",
       "javax.portlet.security-role-ref=power-user,user"
    },
    service = Portlet.class
  )
public class ReluctantPortlet extends MVCPortlet {

   @Override
   public void doView(
           RenderRequest renderRequest, RenderResponse renderResponse)
       throws IOException, PortletException {

       renderRequest.setAttribute("doSomething", _someService.doSomething());

       super.doView(renderRequest, renderResponse);
   }

    @Reference
    private SomeService _someService;

}
\end{verbatim}

The \texttt{ReluctantPortlet}'s method \texttt{doView} sets render
request attribute \texttt{doSomething} to the value returned from the
\texttt{SomeService} instance's \texttt{doSomething} method (e.g.,
\texttt{DefaultService} returns ``I am default!'').

The \texttt{GreedyPortlet} class is similar to
\texttt{ReluctantPortlet}, except its \texttt{SomeService} reference's
policy option is static and greedy (i.e.,
\texttt{ReferencePolicyOption.GREEDY}).

\begin{verbatim}
public class GreedyPortlet extends MVCPortlet {

  @Override
    public void doView(
        RenderRequest renderRequest, RenderResponse renderResponse)
    throws IOException, PortletException {

    renderRequest.setAttribute("doSomething", _someService.doSomething());

    super.doView(renderRequest, renderResponse);
    }

    @Reference (policyOption = ReferencePolicyOption.GREEDY)
    private SomeService _someService;

}
\end{verbatim}

The greedy policy option lets the component switch to using a higher
ranked \texttt{SomeService} instance if one becomes active in the
system. The section
\hyperref[deploying-a-module-with-a-higher-ranked-service-instance-for-binding-to-greedy-references-immediately]{\emph{Deploying
a module with a higher ranked service instance for binding to greedy
references immediately}} demonstrates this portlet switching to a higher
ranked service.

It's time to see this module's portlets and service in action.

\begin{enumerate}
\def\labelenumi{\arabic{enumi}.}
\item
  Stop module \texttt{higher-ranked-service} if it's active.
\item
  Deploy the \texttt{service-reference} module.
\item
  Add the \emph{Reluctant Portlet} from the \emph{Add} → \emph{Widgets}
  → \emph{Sample} category to a site page.

  The portlet displays the message ``SomeService says I am
  Default!''--whose latter part comes from the render request attribute
  set by the \texttt{DefaultService} instance.

  \begin{figure}
  \centering
  \includegraphics{./images/reluctant-portlet-using-default.png}
  \caption{\emph{Reluctant Portlet} displays the message ``SomeService
  says I am Default!''}
  \end{figure}
\item
  Add the \emph{Greedy Portlet} from the \emph{Add} → \emph{Widgets} →
  \emph{Sample} category to a site page.

  The portlet displays the message ``SomeService says I am better, use
  me!''. Both portlets are referencing a \texttt{DefaultService}
  instance.

  \begin{figure}
  \centering
  \includegraphics{./images/greedy-portlet-using-default.png}
  \caption{\emph{Greedy Portlet} displays the message ``SomeService says
  I am better, use me!''}
  \end{figure}
\end{enumerate}

Since \texttt{DefaultService} is the only active \texttt{SomeService}
instance in the system, the portlets refer to it for their
\texttt{SomeService} fields.

The \texttt{DefaultService} and portlets \emph{Reluctant Portlet} and
\emph{Greedy Portlet} are active. Let's activate a higher ranked
\texttt{SomeService} instance and see how the portlets react to it.

\section{Deploying a module with a higher ranked service instance for
binding to greedy references
immediately}\label{deploying-a-module-with-a-higher-ranked-service-instance-for-binding-to-greedy-references-immediately}

Module \texttt{higher-ranked-service} provides a \texttt{SomeService}
implementation called \texttt{HigherRankedService}.
\texttt{HigherRankedService}'s service ranking is \texttt{100}--that's
\texttt{100} more than \texttt{DefaultService}'s ranking \texttt{0}. Its
\texttt{doSomething} method returns the \texttt{String} ``I am better,
use me!''.

\begin{enumerate}
\def\labelenumi{\arabic{enumi}.}
\tightlist
\item
  Deploy the \texttt{higher-ranked-service} module.
\item
  Refresh your page that has the portlets \emph{Reluctant Portlet} and
  \emph{Greedy Portlet}.
\end{enumerate}

\emph{Reluctant Portlet} continues displaying message ``SomeService says
I am better, use me!''. It's ``reluctant'' to unbind from the
\texttt{DefaultService} instance and bind to the newly activated
\texttt{HigherRankedService} service.

\emph{Greedy Portlet} displays a new message ``SomeService says I am
better, use me!''. The part of the message ``I am better, use me!''
comes from the \texttt{HigherRankedService} instance to which it refers.

\begin{figure}
\centering
\includegraphics{./images/greedy-portlet-using-higher-ranked-service.png}
\caption{The \emph{Greedy Portlet} is using a
\texttt{HigherRankedService} instance}
\end{figure}

Next, learn how to bind the \emph{Reluctant Portlet} to a
\texttt{HigherRankedService} instance.

\section{Configuring a component to reference a different service
instance
dynamically}\label{configuring-a-component-to-reference-a-different-service-instance-dynamically}

The \emph{Reluctant Portlet} is currently bound to a
\texttt{DefaultService} instance. It's ``reluctant'' to unbind from it
and bind to a different service. OSGi Configuration Administration lets
you reconfigure service references to filter on and bind to different
service instances.

The \texttt{service-reference} module's configuration files and
\texttt{com.liferay.blade.reluctant.vs.greedy.portlet.portlet.ReluctantPortlet.config}
and
\texttt{com.liferay.blade.reluctant.vs.greedy.portlet.portlet.ReluctantPortlet.cfg}
configure the \texttt{ReluctantPortlet} component to use a
\texttt{HigherRankedService} instance.

\begin{verbatim}
_someService.target=(component.name=com.liferay.blade.reluctant.vs.greedy.service.HigherRankedService)
\end{verbatim}

The service configuration filters on a service whose
\texttt{component.name} is
\texttt{com.liferay.blade.reluctant.vs.greedy.service.HigherRankedService}.

Note: For deploying to 7.0, use file with suffix \texttt{.config}. For
earlier versions (i.e., Liferay DXP 7.0 Fix Packs earlier than Fix Pack
8 and Liferay CE Portal 7.0 GA3 or earlier), use the file with suffix
\texttt{.cfg}.

Here are the steps to reconfigure \texttt{ReluctantPortlet} to use
\texttt{HigherRankedService}:

\begin{enumerate}
\def\labelenumi{\arabic{enumi}.}
\tightlist
\item
  Copy the configuration file to
  \texttt{{[}Liferay-Home{]}/osgi/configs}.
\item
  Refresh your browser.
\end{enumerate}

\emph{Reluctant Portlet} displays a new message ``SomeService says I am
better, use me!''.

\begin{figure}
\centering
\includegraphics{./images/reluctant-portlet-using-higher-ranked-service.png}
\caption{\emph{Reluctant Portlet} is using the
\texttt{HigherRankedService} instance instead of a
\texttt{DefaultService} instance.}
\end{figure}

\emph{Reluctant Portlet} is using \texttt{HigherRankedService} instance
instead of a \texttt{DefaultService} instance. You've configured
\emph{Reluctant Portlet} to use a \texttt{HigherRankedService} instance!

\section{Where Is This Sample?}\label{where-is-this-sample}

There are three different versions of this sample, each built with a
different build tool:

\begin{itemize}
\tightlist
\item
  \href{https://github.com/liferay/liferay-blade-samples/tree/7.2/gradle/apps/greedy-policy-option-portlet}{Gradle}
\item
  \href{https://github.com/liferay/liferay-blade-samples/tree/7.2/liferay-workspace/apps/greedy-policy-option-portlet}{Liferay
  Workspace}
\item
  \href{https://github.com/liferay/liferay-blade-samples/tree/7.2/maven/apps/greedy-policy-option-portlet}{Maven}
\end{itemize}

\chapter{Kotlin Portlet}\label{kotlin-portlet}

The Kotlin Portlet sample provides an input form that accepts a name.
Once submitting a name, the portlet renders a greeting message.

\begin{figure}
\centering
\includegraphics{./images/kotlin-portlet.png}
\caption{After saving the inputted name, it's displayed as a greeting on
the portlet page.}
\end{figure}

\section{What API(s) and/or code components does this sample
highlight?}\label{what-apis-andor-code-components-does-this-sample-highlight-6}

This sample highlights the use of the
\href{https://kotlinlang.org/}{Kotlin} programming language in
conjunction with Liferay's MVC framework. Specifically, this sample
leverages the
\href{https://docs.liferay.com/dxp/portal/7.2-latest/javadocs/portal-kernel/com/liferay/portal/kernel/portlet/bridges/mvc/MVCActionCommand.html}{MVCActionCommand}
interface.

\section{How does this sample leverage the API(s) and/or code
component?}\label{how-does-this-sample-leverage-the-apis-andor-code-component-6}

This sample uses the MVC Action Command's \texttt{processAction(...)}
method to process the inputted text (i.e., name). The text is set as an
attribute in the \texttt{KotlinGreeterActionCommandKt.kt} class using an
\texttt{ActionRequest} and then is retrieved in the JSP using a
\texttt{RenderRequest}.

\section{Where Is This Sample?}\label{where-is-this-sample-1}

This sample is built with the following build tools:

\begin{itemize}
\tightlist
\item
  \href{https://github.com/liferay/liferay-blade-samples/tree/7.2/gradle/apps/kotlin-portlet}{Gradle}
\item
  \href{https://github.com/liferay/liferay-blade-samples/tree/7.2/liferay-workspace/apps/kotlin-portlet}{Liferay
  Workspace}
\end{itemize}

\chapter{Shared Language Keys}\label{shared-language-keys}

The Shared Language Keys sample provides a JSP portlet that displays
language keys.

\begin{figure}
\centering
\includegraphics{./images/language-web-portlet.png}
\caption{The sample JSP portlet displays three language keys.}
\end{figure}

The language keys displayed in the portlet come from two different
modules.

\section{What API(s) and/or code components does this sample
highlight?}\label{what-apis-andor-code-components-does-this-sample-highlight-7}

This sample is broken into two modules:

\begin{itemize}
\tightlist
\item
  \texttt{language}
\item
  \texttt{language-web}
\end{itemize}

The \texttt{language-web} module provides a JSP portlet with unique
language keys that it displays. The \texttt{language} module provides a
resource module which only holds language keys. Its sole purpose is to
share language keys with the JSP portlet provided in
\texttt{language-web}. This sample conveys Liferay's recommended
approach to sharing language keys through OSGi services.

\section{How does this sample leverage the API(s) and/or code
component?}\label{how-does-this-sample-leverage-the-apis-andor-code-component-7}

You must deploy both \texttt{language-web} and \texttt{language} modules
to simulate this sample's targeted demonstration.

First, note the language keys provided by each module:

\begin{itemize}
\tightlist
\item
  \texttt{language-web}

  \begin{itemize}
  \tightlist
  \item
    \texttt{blade\_language\_web\_LanguageWebPortlet.caption=Hello\ from\ BLADE\ Language\ Web!}
  \item
    \texttt{blade\_language\_web\_override\_LanguageWebPortlet.caption=I\ have\ overridden\ the\ key\ from\ BLADE\ Language\ Module!}
  \end{itemize}
\item
  \texttt{language}

  \begin{itemize}
  \tightlist
  \item
    \texttt{blade\_language\_LanguageWebPortlet.caption=Hello\ from\ the\ BLADE\ Language\ Module!}
  \item
    \texttt{blade\_language\_web\_override\_LanguageWebPortlet.caption=Hello\ from\ the\ BLADE\ Language\ Module\ but\ you\ won\textquotesingle{}t\ see\ me!}
  \end{itemize}
\end{itemize}

When you place the sample BLADE Language Web portlet on a Liferay DXP
page, you're presented with three language keys:

\begin{figure}
\centering
\includegraphics{./images/shared-language-keys.png}
\caption{The Language Web portlet displays three phrases, two of which
are shared from a different module.}
\end{figure}

The first message is provided by the \texttt{language-web} module. The
second message is from the \texttt{language} module. The third message
is provided by both modules; as you can see, the \texttt{language-web}'s
message is used, overriding the \texttt{language} module's identically
named language key.

This sample shows what takes precedence when displaying language keys.
The order for this example goes

\begin{enumerate}
\def\labelenumi{\arabic{enumi}.}
\tightlist
\item
  \texttt{language-web} module language keys
\item
  \texttt{language} module language keys
\item
  Liferay DXP language keys
\end{enumerate}

So how does sharing language keys work?

By default, the \texttt{ResourceBundleLoaderAnalyzerPlugin} expands
modules with \texttt{/content/Language.properties} files to add provided
capabilities:

\begin{itemize}
\tightlist
\item
  \texttt{bundle.symbolic.name}
\item
  \texttt{resource.bundle.base.name}
\end{itemize}

Then the deployed \texttt{LanguageExtender} scans modules with those
capabilities to automatically register an associated
\texttt{ResourceBundleLoader}.

You can leverage this functionality to use keys from common language
modules by republishing an aggregate \texttt{ResourceBundleLoader}. This
can be done two ways:

\begin{enumerate}
\def\labelenumi{\arabic{enumi}.}
\item
  Via Components

  You can get a reference to the registered service in your components
  as detailed in the
  \href{/docs/7-2/customization/-/knowledge_base/c/overriding-a-modules-language-keys}{Overriding
  a Module's Language Keys} tutorial. The main disadvantage of this
  approach is that it forces you to provide a specific implementation of
  the \texttt{ResourceBundleLoader}, making it harder to modularize in
  the future.
\item
  Via Provide Capability

  The same \texttt{LanguageExtender} that registers the services
  supports an extended syntax that lets you register an aggregate of a
  collection of bundles:

\begin{verbatim}
-liferay-aggregate-resource-bundles: \
    blade.language
\end{verbatim}

  This approach has the advantage of easier extensibility. When language
  keys change, only the common language modules must be built and
  redeployed for the modules referencing them to recognize their
  updates.
\end{enumerate}

For more information on sharing language keys, visit the
\href{/docs/7-2/frameworks/-/knowledge_base/f/localization}{Internationalization}
tutorials.

\section{Where Is This Sample?}\label{where-is-this-sample-2}

There are three different versions of this sample, each built with a
different build tool:

\begin{itemize}
\tightlist
\item
  \href{https://github.com/liferay/liferay-blade-samples/tree/7.2/gradle/apps/shared-language-keys}{Gradle}
\item
  \href{https://github.com/liferay/liferay-blade-samples/tree/7.2/liferay-workspace/apps/shared-language-keys}{Liferay
  Workspace}
\item
  \href{https://github.com/liferay/liferay-blade-samples/tree/7.2/maven/apps/shared-language-keys}{Maven}
\end{itemize}

\chapter{Simulation Panel App}\label{simulation-panel-app}

The Simulation Panel App provides new functionality in Liferay DXP's
Simulation Menu. When deploying this sample with no customizations, the
\emph{Simulation Sample} feature is provided in the Simulation Menu with
four options.

\section{What API(s) and/or code components does this sample
highlight?}\label{what-apis-andor-code-components-does-this-sample-highlight-8}

This sample leverages the
\href{https://docs.liferay.com/dxp/apps/web-experience/latest/javadocs/com/liferay/application/list/PanelApp.html}{PanelApp}
API.

\section{How does this sample leverage the API(s) and/or code
component?}\label{how-does-this-sample-leverage-the-apis-andor-code-component-8}

This sample leverages the \texttt{PanelApp} interface as an OSGi service
via the \texttt{@Component} annotation:

\begin{verbatim}
@Component(
    immediate = true,
    property = {
        "panel.app.order:Integer=500",
        "panel.category.key=" + SimulationPanelCategory.SIMULATION
    },
    service = PanelApp.class
)
\end{verbatim}

There are also two properties provided via the \texttt{@Component}
annotation:

\begin{itemize}
\tightlist
\item
  \texttt{panel.app.order}: the order in which the panel app is
  displayed among other panel apps in the chosen category. Entries are
  ordered from top to bottom. For example, an entry with order
  \texttt{1} will be listed above an entry with order \texttt{2}. If the
  order is not specified, it's chosen at random based on which service
  was registered first in the OSGi container.
\item
  \texttt{panel.category.key}: the host panel category for your panel
  app, which should be the Simulation Menu category.
\end{itemize}

The simulation panel app extends the
\href{https://docs.liferay.com/ce/apps/web-experience/latest/javadocs/com/liferay/application/list/BaseJSPPanelApp.html}{BaseJSPPanelApp},
which provides a skeletal implementation of the
\href{https://docs.liferay.com/ce/apps/web-experience/latest/javadocs/com/liferay/application/list/PanelApp.html}{PanelApp}
interface with JSP support. JSPs, however, are not the only way to
provide frontend functionality to your panel categories/apps. You can
create your own class implementing \texttt{PanelApp} to use other
technologies, such as FreeMarker.

\section{Where Is This Sample?}\label{where-is-this-sample-3}

There are three different versions of this sample, each built with a
different build tool:

\begin{itemize}
\tightlist
\item
  \href{https://github.com/liferay/liferay-blade-samples/tree/7.2/gradle/apps/simulation-panel-app}{Gradle}
\item
  \href{https://github.com/liferay/liferay-blade-samples/tree/7.2/liferay-workspace/apps/simulation-panel-app}{Liferay
  Workspace}
\item
  \href{https://github.com/liferay/liferay-blade-samples/tree/7.2/maven/apps/simulation-panel-app}{Maven}
\end{itemize}

\chapter{Extensions}\label{extensions}

This section focuses on Liferay sample extensions. You can view these
sample extensions by visiting the \texttt{extensions} folder
corresponding to your preferred build tool:

\begin{itemize}
\tightlist
\item
  \href{https://github.com/liferay/liferay-blade-samples/tree/7.2/gradle/extensions}{Gradle
  sample extensions}
\item
  \href{https://github.com/liferay/liferay-blade-samples/tree/7.2/liferay-workspace/extensions}{Liferay
  Workspace sample extensions}
\item
  \href{https://github.com/liferay/liferay-blade-samples/tree/7.2/maven/extensions}{Maven
  sample extensions}
\end{itemize}

Visit a particular sample page to learn more!

\chapter{Control Menu Entry}\label{control-menu-entry}

The Control Menu Entry sample provides a customizable button that is
added to Liferay Portal's default Control Menu. When deploying this
sample with no customizations, an additional button is added to the User
(right side) portion of the Control Menu.

\begin{figure}
\centering
\includegraphics{./images/controlmenuentry.png}
\caption{The User area of the Control Menu is provided an additional
link button when the Control Menu Entry sample is deployed to Liferay
DXP.}
\end{figure}

The button navigates the user to Liferay's website:
https://www.liferay.com.

\section{What API(s) and/or code components does this sample
highlight?}\label{what-apis-andor-code-components-does-this-sample-highlight-9}

This sample leverages the
\href{https://docs.liferay.com/dxp/apps/web-experience/latest/javadocs/com/liferay/product/navigation/control/menu/ProductNavigationControlMenuEntry.html}{ProductNavigationControlMenuEntry}
API.

\section{How does this sample leverage the API(s) and/or code
component?}\label{how-does-this-sample-leverage-the-apis-andor-code-component-9}

This sample first leverages the
\texttt{ProductNavigationControlMenuEntry} interface as an OSGi service
via the \texttt{@Component} annotation:

\begin{verbatim}
@Component(
    immediate = true,
    property = {
        "product.navigation.control.menu.category.key=" + ProductNavigationControlMenuCategoryKeys.USER,
        "product.navigation.control.menu.entry.order:Integer=1"
    },
    service = ProductNavigationControlMenuEntry.class
)
\end{verbatim}

There are also two properties provided via the \texttt{@Component}
annotation:

\begin{itemize}
\tightlist
\item
  \texttt{product.navigation.control.menu.category.key}: the category in
  which your entry should reside. The default Control Menu provides
  three categories: \emph{SITES} (left portion), \emph{TOOLS} (middle
  portion), and \emph{USER} (right portion).
\item
  \texttt{product.navigation.control.menu.entry.order:Integer}: the
  order in which your entry will be displayed in the category. Entries
  are ordered from left to right. For example, an entry with order
  \texttt{1} will be listed to the left of an entry with order
  \texttt{2}. If the order is not specified, it's chosen at random based
  on which service was registered first in the OSGi container.
\end{itemize}

This sample also implements the
\texttt{ProductNavigationControlMenuEntry} interface. The following
methods are implemented:

\begin{itemize}
\tightlist
\item
  \texttt{getIcon(HttpServletRequest)}
\item
  \texttt{getLabel(Locale)}
\item
  \texttt{getURL(HttpServletRequest)}
\item
  \texttt{isShow(HttpServletRequest)}
\end{itemize}

Refer to this sample's \texttt{BladeProductNavigationControlMenuEntry}
class for Javadocs describing these methods.

\section{Where Is This Sample?}\label{where-is-this-sample-4}

There are three different versions of this sample, each built with a
different build tool:

\begin{itemize}
\tightlist
\item
  \href{https://github.com/liferay/liferay-blade-samples/tree/7.2/gradle/extensions/control-menu-entry}{Gradle}
\item
  \href{https://github.com/liferay/liferay-blade-samples/tree/7.2/liferay-workspace/extensions/control-menu-entry}{Liferay
  Workspace}
\item
  \href{https://github.com/liferay/liferay-blade-samples/tree/7.2/maven/extensions/control-menu-entry}{Maven}
\end{itemize}

\chapter{Document Action}\label{document-action}

The Document Action sample shows how to add a context menu option to an
entry in the Documents and Media portlet. When deploying this sample
with no customizations, an additional menu option is available in the
Documents and Media Admin portlet and the Documents and Media portlet.
This sample creates a \emph{Blade Basic Info} option that displays basic
information about the entry (e.g., file name, type, version, etc.). For
example, the Admin portlet provides the new option as illustrated in the
images below:

\begin{figure}
\centering
\includegraphics{./images/documents-and-media-admin-portlet.png}
\caption{The new \emph{Blade Basic Info} option is available from the
entry's Options menu.}
\end{figure}

\section{What API(s) and/or code components does this sample
highlight?}\label{what-apis-andor-code-components-does-this-sample-highlight-10}

This sample leverages the
\href{@product-ref@/7.2-latest/javadocs/portal-kernel/com/liferay/portal/kernel/portlet/configuration/icon/PortletConfigurationIcon.html}{PortletConfigurationIcon}
API.

\section{How does this sample leverage the API(s) and/or code
component?}\label{how-does-this-sample-leverage-the-apis-andor-code-component-10}

There are four Java classes used in this sample:

\begin{itemize}
\tightlist
\item
  \texttt{BladeActionConfigurationIcon}: Adds the new context menu
  option to the Document Detail screen options
  (\includegraphics{./images/icon-options.png}) (top right corner) of
  the Documents and Media Admin portlet. See the
  \href{/docs/7-0/tutorials/-/knowledge_base/t/configuring-your-admin-apps-actions-menu}{Configuring
  Your Admin App's Actions Menu} tutorial for more details.
\item
  \texttt{BladeActionDisplayContext}: Adds the Display Context for the
  document action. More about Display Contexts are described later.
\item
  \texttt{BladeActionDisplayContextFactory}: Adds the Display Context
  factory for the document action.
\item
  \texttt{BladeDocumentActionPortlet}: Provides the portlet class, which
  extends the
  \href{https://portals.apache.org/pluto/portlet-2.0-apidocs/javax/portlet/GenericPortlet.html}{GenericPortlet}.
  This class generates what is shown when the context menu option is
  selected.
\end{itemize}

A Display Context is a Java class that controls access to a portlet
screen's UI elements. For example, the Document Library would use
Display Contexts to provide its screens all their UI elements. It would
use one Display Context for its document edit screen, another for its
document view screen, etc. A portlet ideally uses a different Display
Context for each of its screens.

A screen's JSP calls on the Display Context (DC) to get elements to
render and to decide whether to render certain types of elements. Some
of the DC methods return a collection of UI elements (e.g., a menu
object of menu items), while other DC methods return booleans that
determine whether to show particular element types. The DC decides which
objects to display, while the JSP organizes the rendered objects and
implements the screen's look and feel. You don't have to decide which
elements to display in your JSP; simply call the DC methods to populate
UI components with objects to render.

To customize or extend a portlet screen that uses a DC, you can extend
the DC and override the methods that control access to the elements that
interest you. For example, you can turn off displaying certain types of
elements (e.g., actions) by overriding the DC method that makes that
decision. You can add new custom elements (e.g., new actions) or remove
existing elements (e.g., a delete action) from a collection of elements
a DC method returns. The beauty of customizing via a DC is that you
don't have to modify the JSP. You only modify the particular methods
that are related to the UI customization goals. And JSP updates won't
break the DC customizations. Replacing a JSP, on the other hand, can
lead to missing an important JSP modification that a new Liferay version
introduces.

As you create custom portlets, you may want to implement DCs. You can
benefit from the separation of concerns that DCs provide and customers
can extend your portlet DCs to specify which UI elements to display. And
they don't need to worry about missing out on the updates you make to
the JSPs.

\section{Where Is This Sample?}\label{where-is-this-sample-5}

There are three different versions of this sample, each built with a
different build tool:

\begin{itemize}
\tightlist
\item
  \href{https://github.com/liferay/liferay-blade-samples/tree/7.2/gradle/extensions/document-action}{Gradle}
\item
  \href{https://github.com/liferay/liferay-blade-samples/tree/7.2/liferay-workspace/extensions/document-action}{Liferay
  Workspace}
\item
  \href{https://github.com/liferay/liferay-blade-samples/tree/7.2/maven/extensions/document-action}{Maven}
\end{itemize}

\chapter{Gogo Shell Command}\label{gogo-shell-command}

{ This document has been updated and ported to Liferay Learn and is no
longer maintained here.}

The Gogo Shell Command sample demonstrates adding a custom command to
Liferay DXP's Gogo shell environment. All Liferay DXP installations have
a Gogo shell environment, which lets system administrators interact with
Liferay DXP's module framework on a local server machine.

This example adds a new custom Gogo shell command called
\texttt{usercount} under the \texttt{blade} scope. It prints out the
number of registered users on your Liferay DXP installation.

To test this sample, follow the instructions below:

\begin{enumerate}
\def\labelenumi{\arabic{enumi}.}
\item
  Start a Liferay DXP installation.
\item
  Navigate to the Control Panel → \emph{Configuration} → \emph{Gogo
  Shell}.
\item
  Execute \texttt{help} to view all the available commands. The sample
  Gogo shell command is listed.

  \begin{figure}
  \centering
  \includegraphics{./images/gogo-shell-1.png}
  \caption{The sample Gogo shell command is listed with all the
  available commands.}
  \end{figure}
\item
  Execute \texttt{usercount} to execute the new custom command. The
  number of users on your running Liferay Portal installation is
  printed.

  \begin{figure}
  \centering
  \includegraphics{./images/gogo-shell-2.png}
  \caption{The outcome of executing the \texttt{usercount} command.}
  \end{figure}
\end{enumerate}

\section{What API(s) and/or code components does this sample
highlight?}\label{what-apis-andor-code-components-does-this-sample-highlight-11}

This sample demonstrates creating a new Gogo shell command by leveraging
\texttt{osgi.command.*} properties in a Java class.

\section{How does this sample leverage the API(s) and/or code
component?}\label{how-does-this-sample-leverage-the-apis-andor-code-component-11}

To add this new Gogo shell command, you must implement the logic in a
Java class with the following two properties:

\begin{itemize}
\tightlist
\item
  \texttt{osgi.command.function}: the command's name, which must match
  the method name in the registered service implementation.
\item
  \texttt{osgi.command.scope}: the general scope or namespace for the
  command.
\end{itemize}

These properties are set in your class's \texttt{@Component} annotation
like this:

\begin{verbatim}
@Component(
    property = {"osgi.command.function=usercount", "osgi.command.scope=blade"},
    service = Object.class
)
\end{verbatim}

The logic for the \texttt{usercount} command is specified in the method
with the same name:

\begin{verbatim}
public void usercount() {
    System.out.println(
        "# of users: " + getUserLocalService().getUsersCount());
}
\end{verbatim}

This method uses \emph{Declarative Services} to get a reference for the
\texttt{UserLocalService} to invoke the \texttt{getUsersCount} method.
This lets you find the number of users currently in the system.

For more information on using the Gogo shell, see the
\href{/docs/7-2/customization/-/knowledge_base/c/using-the-felix-gogo-shell}{Using
the Felix Gogo Shell} tutorial.

\section{Where Is This Sample?}\label{where-is-this-sample-6}

There are three different versions of this sample, each built with a
different build tool:

\begin{itemize}
\tightlist
\item
  \href{https://github.com/liferay/liferay-blade-samples/tree/7.2/gradle/extensions/gogo}{Gradle}
\item
  \href{https://github.com/liferay/liferay-blade-samples/tree/7.2/liferay-workspace/extensions/gogo}{Liferay
  Workspace}
\item
  \href{https://github.com/liferay/liferay-blade-samples/tree/7.2/maven/extensions/gogo}{Maven}
\end{itemize}

\chapter{Index Settings Contributor}\label{index-settings-contributor}

The Index Settings Contributor sample demonstrates how to add a custom
type mapping to Liferay DXP. You can demo this sample by completing the
following steps:

\begin{enumerate}
\def\labelenumi{\arabic{enumi}.}
\item
  Navigate to the \emph{Control Panel} → \emph{Configuration} →
  \emph{Search} menu.
\item
  Click \emph{Execute} for the \emph{Reindex all search indexes} action.

  All properties defined in your \texttt{.json} file are added to
  Liferay DXP's search engine. This sample adds the following index
  properties:

  \begin{itemize}
  \tightlist
  \item
    \texttt{sampleDate}
  \item
    \texttt{sampleDouble}
  \item
    \texttt{sampleLong}
  \item
    \texttt{sampleText}
  \end{itemize}

  You'll verify this next.
\item
  Find your Liferay DXP's instance ID. This can be found in the
  \emph{Control Panel} → \emph{Configuration} → \emph{Virtual Instances}
  menu.
\item
  Navigate to the following URL:

\begin{verbatim}
http://localhost:9200/liferay-[INSTANCE_ID]/_mapping/LiferayDocumentType?pretty
\end{verbatim}

  Be sure to insert your instance ID into the URL.
\item
  Verify the added properties are listed.

  \begin{figure}
  \centering
  \includegraphics{./images/index-settings-contributor.png}
  \caption{This sample added four new index properties.}
  \end{figure}
\end{enumerate}

\section{What API(s) and/or code components does this sample
highlight?}\label{what-apis-andor-code-components-does-this-sample-highlight-12}

This sample leverages the
\href{https://docs.liferay.com/dxp/apps/foundation/latest/javadocs/com/liferay/portal/search/elasticsearch/settings/IndexSettingsContributor.html}{IndexSettingsContributor}
API.

\section{How does this sample leverage the API(s) and/or code
component?}\label{how-does-this-sample-leverage-the-apis-andor-code-component-12}

Liferay's search engine provides an API to define custom mappings. To
use it, follow these fundamental steps:

\begin{enumerate}
\def\labelenumi{\arabic{enumi}.}
\item
  Define the new mapping. In this sample, the mapping is defined in the
  \texttt{META-INF/mappings/resources/index-type-mappings.json} file.
  Notice that the default document for Liferay DXP is called
  \texttt{LiferayDocumentType}. The mapping's features can be found in
  \href{https://www.elastic.co/guide/en/elasticsearch/reference/7.x/mapping.html}{Elasticsearch's
  docs}.
\item
  Inject the mapping into Elasticsearch. The
  \texttt{IndexSettingsContributor} class' components are invoked during
  the reindexing stage and receive a \texttt{TypeMappingsHelper} as a
  hook to add new mappings.
\end{enumerate}

This sample has two classes:

\begin{itemize}
\item
  \texttt{ResourceUtil}: reads the \texttt{.json} file.
\item
  \texttt{IndexSettingsContributor}: allows the addition of type
  mappings on Liferay DXP's search engine.
\end{itemize}

The \texttt{IndexSettingsContributor}'s \texttt{contribute} method adds
the type mappings:

\begin{verbatim}
@Override
public void contribute(
    String indexName, TypeMappingsHelper typeMappingsHelper) {
    try {
        String mappings = ResourceUtil.readResouceAsString(
            "META-INF/resources/mappings/index-type-mappings.json");

        typeMappingsHelper.addTypeMappings(indexName, mappings);
    }
    catch (Exception e) {
        e.printStackTrace();
    }
}
\end{verbatim}

For the \texttt{ResourceUtil.readResouceAsString} parameter, you should
pass the path for the \texttt{.json} file that contains the properties
to be added.

Also, it is important to highlight the
\texttt{IndexSettingsContributor}'s \texttt{@Component} annotation that
registers a new service to the OSGi container:

\begin{verbatim}
@Component(
    immediate = true,
    service = com.liferay.portal.search.elasticsearch6.settings.IndexSettingsContributor.class
)

> If using Elasticsearch 7, the value of the `service` property is instead `com.liferay.portal.search.elasticsearch7.settings.IndexSettingsContributor.class`.
\end{verbatim}

This sample demonstrates the essentials needed to contribute your own
index settings.

\section{Where Is This Sample?}\label{where-is-this-sample-7}

There are three different versions of this sample, each built with a
different build tool:

\begin{itemize}
\tightlist
\item
  \href{https://github.com/liferay/liferay-blade-samples/blob/7.2/gradle/extensions/index-settings-contributor}{Gradle}
\item
  \href{https://github.com/liferay/liferay-blade-samples/blob/7.2/liferay-workspace/extensions/index-settings-contributor}{Liferay
  Workspace}
\item
  \href{https://github.com/liferay/liferay-blade-samples/blob/7.2/maven/extensions/index-settings-contributor}{Maven}
\end{itemize}

\chapter{Indexer Post Processor}\label{indexer-post-processor}

The Indexer Post Processor sample demonstrates using the
\texttt{IndexerPostProcessor} interface, which is provided to customize
search queries and documents before they're sent to the search engine,
and/or customize result summaries when they're returned to end users.
This basic demonstration prints a message in the log when one of the
\texttt{*IndexerPostProcessor} methods is called.

To see this sample's messages in Liferay DXP's log, you must add a
logging category to the portal. Navigate to \emph{Control Panel} →
\emph{Configuration} → \emph{Server Administration} and click on
\emph{Log Levels} → \emph{Add Category}. Then fill out the form:

\begin{itemize}
\tightlist
\item
  \emph{Logger Name}:
  \texttt{com.liferay.blade.samples.indexerpostprocessor}
\item
  \emph{Log Level}: \texttt{INFO}
\end{itemize}

Once you save the new logging category, you can witness the sample
indexer post processor in action. For example, you can test the sample's
\texttt{BlogsIndexerPostProcessor} implementation by creating a blog
entry. When you publish the blog, the following message is logged in the
console:

\begin{verbatim}
18:27:30,737 INFO  [http-nio-8080-exec-8][BlogsIndexerPostProcessor:76] postProcessDocument
\end{verbatim}

\section{What API(s) and/or code components does this sample
highlight?}\label{what-apis-andor-code-components-does-this-sample-highlight-13}

This sample leverages the
\href{https://docs.liferay.com/dxp/portal/7.2-latest/javadocs/portal-kernel/com/liferay/portal/kernel/search/IndexerPostProcessor.html}{IndexerPostProcessor}
API.

\section{How does this sample leverage the API(s) and/or code
component?}\label{how-does-this-sample-leverage-the-apis-andor-code-component-13}

This sample contains four implementations of the
\texttt{IndexerPostProcessor} interface:

\begin{itemize}
\tightlist
\item
  \texttt{BlogsIndexerPostProcessor}
\item
  \texttt{MultipleEntityIndexerPostProcessor}
\item
  \texttt{MultipleIndexerPostProcessor}
\item
  \texttt{UserEntityIndexerPostProcessor}
\end{itemize}

All these classes leverage the interface as an OSGi service via the
\texttt{@Component} annotation. For example, the \texttt{@Component}
annotation of the \texttt{UserEntityIndexerPostProcessor} looks like
this:

\begin{verbatim}
@Component(
    immediate = true,
    property = {
        "indexer.class.name=com.liferay.portal.kernel.model.User",
        "indexer.class.name=com.liferay.portal.kernel.model.UserGroup"
    },
    service = IndexerPostProcessor.class
)
\end{verbatim}

There's one property type provided via the \texttt{@Component}
annotation:

\begin{itemize}
\tightlist
\item
  \texttt{indexer.class.name}: the fully qualified class name of the
  indexed entity or an \texttt{Indexer} class itself.
\end{itemize}

This sample's implementations of the \texttt{IndexerPostProcessor}
interface override the following methods:

\begin{itemize}
\tightlist
\item
  \texttt{postProcessContextBooleanFilter}
\item
  \texttt{postProcessContextQuery}
\item
  \texttt{postProcessDocument}
\item
  \texttt{postProcessFullQuery}
\item
  \texttt{postProcessSearchQuery(BooleanQuery,\ BooleanFilter)}
\item
  \texttt{postProcessSearchQuery(BooleanQuery,\ SearchContext)}
\item
  \texttt{postProcessSummary}
\end{itemize}

For more information on Liferay's Search API, refer to the
\href{/docs/7-2/frameworks/-/knowledge_base/f/search}{Introduction to
Liferay Search} article.

\section{Where Is This Sample?}\label{where-is-this-sample-8}

There are three different versions of this sample, each built with a
different build tool:

\begin{itemize}
\tightlist
\item
  \href{https://github.com/liferay/liferay-blade-samples/tree/7.2/gradle/extensions/indexer-post-processor}{Gradle}
\item
  \href{https://github.com/liferay/liferay-blade-samples/tree/7.2/liferay-workspace/extensions/indexer-post-processor}{Liferay
  Workspace}
\item
  \href{https://github.com/liferay/liferay-blade-samples/tree/7.2/maven/extensions/indexer-post-processor}{Maven}
\end{itemize}

\chapter{Model Listener}\label{model-listener}

The Model Listener sample demonstrates adding a custom model listener to
a Liferay Portal out-of-the-box entity. When deploying this sample with
no customizations, a custom model listener is added to the portal's
layouts, listening for \texttt{onBeforeCreate} events. This means that
any page creation will trigger this listener, which will execute before
the new page is created.

For example, if a new page is added with the name \emph{My Test Page},
the following message is printed to the console:

\begin{figure}
\centering
\includegraphics{./images/model-listener-1.png}
\caption{The sample model listener's message in the console.}
\end{figure}

You can also verify that the model listener sample was executed by
navigating to the new page's \emph{Options} → \emph{Configure Page} →
\emph{SEO} option. The HTML Title field looks like this:

\begin{figure}
\centering
\includegraphics{./images/model-listener-2.png}
\caption{The page's HTML title updated by the model listener sample.}
\end{figure}

\section{What API(s) and/or code components does this sample
highlight?}\label{what-apis-andor-code-components-does-this-sample-highlight-14}

This sample leverages the
\href{https://docs.liferay.com/dxp/portal/7.2-latest/javadocs/portal-kernel/com/liferay/portal/kernel/model/ModelListener.html}{ModelListener}
API.

\section{How does this sample leverage the API(s) and/or code
component?}\label{how-does-this-sample-leverage-the-apis-andor-code-component-14}

Model Listeners are used to listen for persistence events on models and
take actions as a result of those events. Actions can be executed on an
entity's database table before or after a \texttt{create},
\texttt{remove}, \texttt{update}, \texttt{addAssociation}, or
\texttt{removeAssociation} event. It's possible to have more than one
model listener on a single model too; the execution order is not
guaranteed.

There are two steps to create a new model listener:

\begin{itemize}
\tightlist
\item
  Implement a Model Listener class
\item
  Register the new service in Liferay's OSGi runtime
\end{itemize}

This sample adds the model listener logic in a new Java class named
\texttt{CustomLayoutListener} that extends
\href{https://docs.liferay.com/dxp/portal/7.1-latest/javadocs/portal-kernel/com/liferay/portal/kernel/model/BaseModelListener.html}{BaseModelListener}.

\begin{verbatim}
public class CustomLayoutListener extends BaseModelListener<Layout> {

    @Override
    public void onBeforeCreate(Layout model) throws ModelListenerException {
        System.out.println(
            "About to create layout: " + model.getNameCurrentValue());

        model.setTitle("Title generated by model listener!");
    }

}
\end{verbatim}

Important things to note in this code snippet are

\begin{itemize}
\tightlist
\item
  The entity to be targeted by this model listener is specified as the
  parameterized type (e.g., \texttt{Layout}).
\item
  The overridden methods dictate the type of event(s) that are listened
  for (e.g., \texttt{onBeforeCreate}); they also trigger the logic
  execution.
\end{itemize}

The final step is registering the service in Liferay's OSGi runtime,
which is accomplished by the following annotation (if using Declarative
Services):

\begin{verbatim}
@Component(immediate = true, service = ModelListener.class)
\end{verbatim}

For more information on model listeners, see the
\href{/docs/7-2/customization/-/knowledge_base/c/model-listeners}{Creating
Model Listeners} tutorial.

\section{Where Is This Sample?}\label{where-is-this-sample-9}

There are three different versions of this sample, each built with a
different build tool:

\begin{itemize}
\tightlist
\item
  \href{https://github.com/liferay/liferay-blade-samples/tree/7.2/gradle/extensions/model-listener}{Gradle}
\item
  \href{https://github.com/liferay/liferay-blade-samples/tree/7.2/liferay-workspace/extensions/model-listener}{Liferay
  Workspace}
\item
  \href{https://github.com/liferay/liferay-blade-samples/tree/7.2/maven/extensions/model-listener}{Maven}
\end{itemize}

\chapter{Screen Name Validator}\label{screen-name-validator}

The Screen Name Validator sample provides a way to validate a user's
inputted screen name. During validation, the screen name is tested
client-side and server-side.

This sample checks if a user's screen name contains reserved words that
are configured in the \emph{Control Panel} → \emph{Configuration} →
\emph{System Settings} → \emph{Foundation} → \emph{ScreenName Validator}
menu. The default values for the screen name validator's reserved words
are \emph{admin} and \emph{user}.

\begin{figure}
\centering
\includegraphics{./images/screenname-validator-config.png}
\caption{Enter reserved words for the screen name validator.}
\end{figure}

You can test this sample by following the following steps:

\begin{enumerate}
\def\labelenumi{\arabic{enumi}.}
\tightlist
\item
  Deploy the Screen Name Validator to your portal installation.
\item
  Navigate to the \emph{Control Panel} → \emph{Users} → \emph{Users and
  Organizations} menu.
\item
  Create a new user by selecting the \emph{Add User}
  (\includegraphics{./images/icon-add.png}) button.
\item
  Adding a screen name that contains the word \emph{admin} or
  \emph{user}.
\end{enumerate}

\begin{figure}
\centering
\includegraphics{./images/screenname-validator-test.png}
\caption{The error message displays when inputting a reserved word for
the screen name.}
\end{figure}

\section{What API(s) and/or code components does this sample
highlight?}\label{what-apis-andor-code-components-does-this-sample-highlight-15}

This sample leverages the
\href{@product-ref@/7.2-latest/javadocs/portal-kernel/com/liferay/portal/kernel/security/auth/ScreenNameValidator.html}{ScreenNameValidator}
API.

\section{How does this sample leverage the API(s) and/or code
component?}\label{how-does-this-sample-leverage-the-apis-andor-code-component-15}

To customize this sample, modify its
\texttt{com.liferay.blade.samples.screenname.validator.internal.CustomScreenNameValidator}
class.

You can also customize this sample's configuration by adding more
properties in its
\texttt{com.liferay.blade.samples.screenname.validator.CustomScreenNameConfiguration}
class.

For more information on customizing the Validation sample to fit your
needs, see the Javadoc provided in this sample's Java classes.

\section{Where Is This Sample?}\label{where-is-this-sample-10}

There are three different versions of this sample, each built with a
different build tool:

\begin{itemize}
\tightlist
\item
  \href{https://github.com/liferay/liferay-blade-samples/tree/7.2/gradle/extensions/screen-name-validator}{Gradle}
\item
  \href{https://github.com/liferay/liferay-blade-samples/tree/7.2/liferay-workspace/extensions/screen-name-validator}{Liferay
  Workspace}
\item
  \href{https://github.com/liferay/liferay-blade-samples/tree/7.2/maven/extensions/screen-name-validator}{Maven}
\end{itemize}

\chapter{Servlet}\label{servlet}

The Servlet sample provides an OSGi Whiteboard Servlet in Liferay DXP.
When deploying this sample and configuring the servlet, a \emph{Hello
World} message is displayed when accessing the servlet page URL. Log
info is also outputted to your console.

\begin{figure}
\centering
\includegraphics{./images/servlet-sample.png}
\caption{The servlet displays \emph{Hello World} from the configured
servlet page URL.}
\end{figure}

\begin{figure}
\centering
\includegraphics{./images/servlet-sample-log.png}
\caption{The servlet also logs info in the console.}
\end{figure}

To configure the servlet in Liferay DXP, complete the following steps:

\begin{enumerate}
\def\labelenumi{\arabic{enumi}.}
\item
  Navigate to the \emph{Control Panel} → \emph{Configuration} →
  \emph{Server Administration} → \emph{Log Levels}.
\item
  Select \emph{Add Category}.
\item
  Insert \emph{com.liferay.blade.samples.servlet.BladeServlet} for the
  Logger Name and \emph{INFO} for the Log Level.
\item
  Navigate to the http://localhost:8080/o/blade/servlet URL.
\end{enumerate}

\section{What API(s) and/or code components does this sample
highlight?}\label{what-apis-andor-code-components-does-this-sample-highlight-16}

This sample leverages the
\href{https://tomcat.apache.org/tomcat-5.5-doc/servletapi/javax/servlet/http/HttpServlet.html}{HttpServlet}
API.

\section{How does this sample leverage the API(s) and/or code
component?}\label{how-does-this-sample-leverage-the-apis-andor-code-component-16}

To customize this sample, modify its
\texttt{com.liferay.blade.samples.servlet.BladeServlet} class. This
class extends the \texttt{HttpServlet} class. Creating your own servlet
for Liferay DXP is useful when you need to implement servlet actions.
For example, if you wanted to implement the CMIS server by yourself with
\href{https://chemistry.apache.org/}{Apache Chemistry}, you would need
to implement your own servlet, managing requests at a low level.

\section{Where Is This Sample?}\label{where-is-this-sample-11}

There are three different versions of this sample, each built with a
different build tool:

\begin{itemize}
\tightlist
\item
  \href{https://github.com/liferay/liferay-blade-samples/tree/7.2/gradle/extensions/servlet}{Gradle}
\item
  \href{https://github.com/liferay/liferay-blade-samples/tree/7.2/liferay-workspace/extensions/servlet}{Liferay
  Workspace}
\item
  \href{https://github.com/liferay/liferay-blade-samples/tree/7.2/maven/extensions/servlet}{Maven}
\end{itemize}

\chapter{Overrides}\label{overrides}

This section focuses on Liferay sample overrides. You can view these
sample overrides by visiting the \texttt{overrides} folder corresponding
to your preferred build tool:

\begin{itemize}
\tightlist
\item
  \href{https://github.com/liferay/liferay-blade-samples/tree/7.2/gradle/overrides}{Gradle
  sample overrides}
\item
  \href{https://github.com/liferay/liferay-blade-samples/tree/7.2/liferay-workspace/overrides}{Liferay
  Workspace sample overrides}
\item
  \href{https://github.com/liferay/liferay-blade-samples/tree/7.2/maven/overrides}{Maven
  sample overrides}
\end{itemize}

Visit a particular sample page to learn more!

\chapter{Module JSP Override}\label{module-jsp-override}

The Module JSP Override sample conveys how to override an application's
JSP by leveraging OSGi fragment modules. This is not the recommended
practice for overriding JSPs in 7.0. See the
\href{/docs/7-2/customization/-/knowledge_base/c/customizing-jsps}{Customizing
JSPs} article for better options.

This example overrides the default \texttt{login.jsp} file in the
\texttt{com.liferay.login.web} bundle by adding the red text
\emph{changed} to the Sign In form.

\begin{figure}
\centering
\includegraphics{./images/hook-jsp.png}
\caption{The customized Sign In form with the new \emph{changed} text.}
\end{figure}

\section{What API(s) and/or code components does this sample
highlight?}\label{what-apis-andor-code-components-does-this-sample-highlight-17}

This sample demonstrates how to create a fragment host module and
configure it to override an existing module's JSP.

\section{How does this sample leverage the API(s) and/or code
component?}\label{how-does-this-sample-leverage-the-apis-andor-code-component-17}

You can create your own JSP override by

\begin{itemize}
\tightlist
\item
  Declaring the fragment host.
\item
  Providing the JSP that will override the original one.
\end{itemize}

To properly declare the fragment host in the \texttt{bnd.bnd} file, you
must specify the host module's (where the original JSP is located)
Bundle Symbolic Name and the host module's exact version to which the
fragment belongs. In this example, this is configured like this:

\begin{verbatim}
Fragment-Host: com.liferay.login.web;bundle-version="1.0.0"
\end{verbatim}

Then you must provide the new JSP intended to override the original one.
Be sure to mimic the host module's folder structure when overriding its
JAR. For this example, since the original JSP is in the folder
\texttt{/META-INF/resources/login.jsp}, the new JSP file resides in the
folder \texttt{src/main/resources/META-INF/resources/login.jsp}.

If needed, you can also target the original JSP following one of the two
possible naming conventions: \texttt{original} or \texttt{portal}. This
pattern looks like

\begin{verbatim}
<liferay-util:include
    page="/login.original.jsp"
    servletContext="<%= application %>"
/>
\end{verbatim}

or

\begin{verbatim}
<liferay-util:include
    page="/login.portal.jsp"
    servletContext="<%= application %>"
/>
\end{verbatim}

This approach can be used to override any application JSP (i.e., JSPs
residing in a module). You can also add new JSPs to an existing module
with this technique. For more information on other ways to customize
JSPs, see the
\href{/docs/7-2/customization/-/knowledge_base/c/customizing-jsps}{Customizing
JSPs} articles.

\section{Where Is This Sample?}\label{where-is-this-sample-12}

There are three different versions of this sample, each built with a
different build tool:

\begin{itemize}
\tightlist
\item
  \href{https://github.com/liferay/liferay-blade-samples/tree/7.2/gradle/overrides/module-jsp-override}{Gradle}
\item
  \href{https://github.com/liferay/liferay-blade-samples/tree/7.2/liferay-workspace/overrides/module-jsp-override}{Liferay
  Workspace}
\item
  \href{https://github.com/liferay/liferay-blade-samples/tree/7.2/maven/overrides/module-jsp-override}{Maven}
\end{itemize}

\chapter{Resource Bundle Override}\label{resource-bundle-override}

This example overrides the default
\texttt{javax.portlet.title.com\_liferay\_login\_web\_portlet\_LoginPortlet}
language key for Liferay DXP's default Login portlet. After deploying
this sample to Liferay DXP, the Login portlet's \emph{Sign In} title is
modified to display \emph{Login Portlet Override}.

\begin{figure}
\centering
\includegraphics{./images/hook-resourcebundle.png}
\caption{The customized Login portlet displays the new language key.}
\end{figure}

For reference, the Login portlet's language keys are stored in the
\href{https://github.com/liferay/liferay-portal}{liferay-portal} Github
repo's \texttt{modules/apps/login/login-web/src/main/resources/content}
folder.

\section{What API(s) and/or code components does this sample
highlight?}\label{what-apis-andor-code-components-does-this-sample-highlight-18}

This sample leverages the
\href{https://bnd.bndtools.org/chapters/220-contracts.html}{\texttt{Provide-Capability}}
OSGi manifest header.

\section{How does this sample leverage the API(s) and/or code
component?}\label{how-does-this-sample-leverage-the-apis-andor-code-component-18}

This sample conveys the recommended approach to override a portlet's
language keys file for any module that is deployed to Liferay DXP's OSGi
runtime (not applicable to Liferay DXP's core language keys).

The steps to override a portlet's language keys are

\begin{itemize}
\tightlist
\item
  Provide the new language keys that will override the original ones.
\item
  Prioritize the new module's resource bundle.
\end{itemize}

This sample's \texttt{src/main/resources/content} folder holds the
language properties file to override. Since this example's goal is to
override only the English keys, the \texttt{Language\_en.properties} is
added. You can add more language properties files for additional
language key locales you want to override (e.g.,
\texttt{Language\_en.properties} for Spanish).

Once your language keys are in place, you must use OSGi manifest headers
to specify your custom language keys are for the target module. To
compliment the target module's resource bundle, you must aggregate your
resource bundle with the target module's resource bundle. This is done
by ranking your module first to prioritize its resource bundle over the
target module resource bundle. See this sample's \texttt{bnd.bnd} as an
example for setting the \texttt{Provide-Capability} OSGi header:

\begin{verbatim}
Provide-Capability:\
    liferay.resource.bundle;\
        resource.bundle.base.name="content.Language",\
    liferay.resource.bundle;\
        bundle.symbolic.name=com.liferay.login.web;\
        resource.bundle.aggregate:String="(bundle.symbolic.name=com.liferay.blade.login.web.resource.bundle.override),(bundle.symbolic.name=com.liferay.login.web)";\
        resource.bundle.base.name="content.Language";\
        service.ranking:Long="2";\
        servlet.context.name=login-web
\end{verbatim}

For more information on the \texttt{Provide-Capability} header and its
parts, see the
\href{/docs/7-2/customization/-/knowledge_base/c/overriding-a-modules-language-keys\#prioritize-your-modules-resource-bundle}{Prioritze
Your Module's Resource Bundle} section.

This approach can be used to override any portlet's language keys (i.e.,
\texttt{language.properties} files that are inside a module deployed to
Liferay DXP's OSGi runtime). If you need to override Liferay DXP's core
language keys, see the
\href{/docs/7-2/customization/-/knowledge_base/c/overriding-global-language-keys}{Overriding
Global Language Keys} article.

For more information on using a resource bundle to override a module's
language keys, see the
\href{/docs/7-2/customization/-/knowledge_base/c/overriding-a-modules-language-keys}{Overriding
a Module's Language Keys} tutorial.

\section{Where Is This Sample?}\label{where-is-this-sample-13}

There are three different versions of this sample, each built with a
different build tool:

\begin{itemize}
\tightlist
\item
  \href{https://github.com/liferay/liferay-blade-samples/tree/7.2/gradle/overrides/login-web-resource-bundle-override}{Gradle}
\item
  \href{https://github.com/liferay/liferay-blade-samples/tree/7.2/liferay-workspace/overrides/login-web-resource-bundle-override}{Liferay
  Workspace}
\item
  \href{https://github.com/liferay/liferay-blade-samples/tree/7.2/maven/overrides/login-web-resource-bundle-override}{Maven}
\end{itemize}

\chapter{Themes}\label{themes}

This section focuses on Liferay sample themes. You can view these sample
themes by visiting the \texttt{themes} folder corresponding to your
preferred build tool:

\begin{itemize}
\tightlist
\item
  \href{https://github.com/liferay/liferay-blade-samples/tree/7.2/gradle/themes}{Gradle
  sample themes}
\item
  \href{https://github.com/liferay/liferay-blade-samples/tree/7.2/liferay-workspace/themes}{Liferay
  Workspace sample themes}
\item
  \href{https://github.com/liferay/liferay-blade-samples/tree/7.2/maven/themes}{Maven
  sample themes}
\end{itemize}

Visit a particular sample page to learn more!

\chapter{Simple Theme}\label{simple-theme}

The Simple Theme sample provides the base files for a theme, using the
\href{/docs/7-2/reference/-/knowledge_base/r/theme-builder-gradle-plugin}{Theme
Builder Gradle plugin}. When deploying this sample with no
customizations, a theme based off of the \texttt{\_styled} base theme is
created.

\begin{figure}
\centering
\includegraphics{./images/theme.png}
\caption{A theme based off of the Styled base theme is created when the
Theme Blade sample is deployed to Liferay Portal.}
\end{figure}

For more information on themes, visit the
\href{/docs/7-2/frameworks/-/knowledge_base/f/themes-introduction}{Introduction
to Themes} tutorial.

\section{What API(s) and/or code components does this sample
highlight?}\label{what-apis-andor-code-components-does-this-sample-highlight-19}

This sample demonstrates a way to create a simple theme in Liferay DXP.

\section{How does this sample leverage the API(s) and/or code
component?}\label{how-does-this-sample-leverage-the-apis-andor-code-component-19}

To modify this sample, add the \texttt{images}, \texttt{js}, or
\texttt{templates} folder, along with your modified files, to the
\texttt{src/main/webapp} folder. The sample already provides the
\texttt{src/main/resources/resources-importer},
\texttt{src/main/webapp/WEB-INF}, and \texttt{src/main/webapp/css}
folders for you. Add your style modifications to the provided
\texttt{css/\_custom.scss} file. For a complete explanation of a theme's
files, see the
\href{/docs/7-2/reference/-/knowledge_base/r/theme-reference-guide}{Theme
Reference Guide}.

\section{Where Is This Sample?}\label{where-is-this-sample-14}

There are three different versions of this sample, each built with a
different build tool:

\begin{itemize}
\tightlist
\item
  \href{https://github.com/liferay/liferay-blade-samples/tree/7.2/gradle/themes/simple-theme}{Gradle}
\item
  \href{https://github.com/liferay/liferay-blade-samples/tree/7.2/liferay-workspace/wars/simple-theme}{Liferay
  Workspace}
\item
  \href{https://github.com/liferay/liferay-blade-samples/tree/7.2/maven/themes/simple-theme}{Maven}
\end{itemize}

\chapter{Template Context
Contributor}\label{template-context-contributor}

The Template Context Contributor sample injects a new variable into
Liferay DXP's theme context. When deploying this sample with no
customizations, you can use the \texttt{\$\{sample\_text\}} variable
from any theme.

\section{What API(s) and/or code components does this sample
highlight?}\label{what-apis-andor-code-components-does-this-sample-highlight-20}

Many developers prefer using templating frameworks like FreeMarker and
Velocity, but don't have access to the common objects offered to those
working with JSPs. Context contributors allow non-JSP developers an easy
way to inject variables into their Liferay templates.

This sample leverages the
\href{@product-ref@/7.2-latest/javadocs/portal-kernel/com/liferay/portal/kernel/template/TemplateContextContributor.html}{TemplateContextContributor}
API.

\section{How does this sample leverage the API(s) and/or code
component?}\label{how-does-this-sample-leverage-the-apis-andor-code-component-20}

You can easily modify this sample by customizing its
\texttt{BladeTemplateContextContributor.java} Java class. For example,
the default context contributor sample provides the
\texttt{\$\{sample\_text\}} variable by injecting it into Liferay's
\texttt{contextObjects}, which is a map provided by default to offer
common variables to non-JSP template developers. You can easily inject
your own variables into the \texttt{contextObjects} map usable by any
theme deployed to Liferay DXP.

Are you working with templates that aren't themes (e.g., ADTs, DDM
templates, etc.)? You can change the context in which your variables are
injected by modifying the \texttt{property} attribute in the
\texttt{@Component} annotation. If you want your variable available for
all templates, change it to

\begin{verbatim}
property = {"type=" + TemplateContextContributor.TYPE_GLOBAL}
\end{verbatim}

For more information on customizing the Template Context Contributor
sample to fit your needs, see the Javadoc listed in this sample's
\texttt{com.liferay.blade.samples.theme.contributorBladeTemplateContextContributor}
class. For more information on context contributors and how to create
them in Liferay DXP, visit the
\href{/docs/7-2/frameworks/-/knowledge_base/f/injecting-additional-context-variables-and-functionality-into-your-theme-templates}{Context
Contributors} tutorial.

\section{Where Is This Sample?}\label{where-is-this-sample-15}

There are three different versions of this sample, each built with a
different build tool:

\begin{itemize}
\tightlist
\item
  \href{https://github.com/liferay/liferay-blade-samples/tree/7.2/gradle/themes/template-context-contributor}{Gradle}
\item
  \href{https://github.com/liferay/liferay-blade-samples/tree/7.2/liferay-workspace/themes/template-context-contributor}{Liferay
  Workspace}
\item
  \href{https://github.com/liferay/liferay-blade-samples/tree/7.2/maven/themes/template-context-contributor}{Maven}
\end{itemize}

\chapter{Theme Contributor}\label{theme-contributor}

The Theme Contributor sample contributes updates to the UI of the theme
body, Control Menu, Product Menu, and Simulation Panel. When deploying
this sample with no customizations, the colors of the theme and
aforementioned menus are updated.

\begin{figure}
\centering
\includegraphics{./images/theme-contributor-yellow.png}
\caption{Your Liferay DXP pages and menu fonts now have a yellow tint.}
\end{figure}

Also, there's a simple JavaScript update that is provided, which logs a
message to the browser's console window that states \emph{Hello Blade
Theme Contributor!}.

\begin{figure}
\centering
\includegraphics{./images/theme-contributor-console-output.png}
\caption{The message is printed to your browser's console window using
JavaScript.}
\end{figure}

\section{What API(s) and/or code components does this sample
highlight?}\label{what-apis-andor-code-components-does-this-sample-highlight-21}

This sample demonstrates a way to contribute updates to a Liferay DXP
theme. Theme Contributors let you package UI resources (e.g., CSS and
JS) independent of a theme to include on a Liferay DXP page.

\section{How does this sample leverage the API(s) and/or code
component?}\label{how-does-this-sample-leverage-the-apis-andor-code-component-21}

To modify this sample, replace the corresponding JS or SCSS file with
the JavaScript or styles that you want, or add your own JS or SCSS
files. For example, this sample provides an update to the Control Menu's
\texttt{background-color} in its
\texttt{src/main/resources/META-INF/resources/css/blade.theme.contributor/\_control\_menu.scss}
file:

\begin{verbatim}
body {
        .control-menu {
                background-color: darkkhaki;
        }
}
\end{verbatim}

All of the SCSS files used in this sample are imported into the main
\texttt{blade.theme.contributor.scss} file:

\begin{verbatim}
@import "bourbon";
@import "mixins";

@import "blade.theme.contributor/body";
@import "blade.theme.contributor/control_menu";
@import "blade.theme.contributor/product_menu";
@import "blade.theme.contributor/simulation_panel";
\end{verbatim}

If you add your own \texttt{SCSS} files, you must add them to the list
of imports in the \texttt{blade.theme.contributor.scss} file.

Likewise, the sample \texttt{blade.theme.contributor.js} logs a message
to your browser's console window using the following JS logic:

\begin{verbatim}
console.log('Hello Blade Theme Contributor!');
\end{verbatim}

For more information on Theme Contributors, visit the
\href{/docs/7-2/frameworks/-/knowledge_base/f/packaging-independent-ui-resources-for-your-site}{Theme
Contributors} tutorial.

\section{Where Is This Sample?}\label{where-is-this-sample-16}

There are three different versions of this sample, each built with a
different build tool:

\begin{itemize}
\tightlist
\item
  \href{https://github.com/liferay/liferay-blade-samples/tree/7.2/gradle/themes/theme-contributor}{Gradle}
\item
  \href{https://github.com/liferay/liferay-blade-samples/tree/7.2/liferay-workspace/themes/theme-contributor}{Liferay
  Workspace}
\item
  \href{https://github.com/liferay/liferay-blade-samples/tree/7.2/maven/themes/theme-contributor}{Maven}
\end{itemize}

\chapter{Ext}\label{ext}

This section focuses on Liferay Ext modules. You can view these sample
apps by visiting the \texttt{ext} folder corresponding to your preferred
build tool:

\begin{itemize}
\tightlist
\item
  \href{https://github.com/liferay/liferay-blade-samples/tree/7.2/gradle/ext}{Gradle
  sample apps}
\item
  \href{https://github.com/liferay/liferay-blade-samples/tree/7.2/liferay-workspace/ext}{Liferay
  Workspace sample apps}
\end{itemize}

Visit the sample page to learn more!

\chapter{Login Web Ext}\label{login-web-ext}

The Login Ext Module sample demonstrates how to customize a default
Liferay module's source code. This example replaces the default
\texttt{login.jsp} file in the \texttt{com.liferay.login.web} bundle by
adding the text \emph{Hello from com.liferay.login.web.ext module! 2 + 2
= 4} to the Sign In form.

\begin{figure}
\centering
\includegraphics{./images/login-ext.png}
\caption{The Login Ext module customizes the original Login module.}
\end{figure}

It also prints the following text to the console when you select
\emph{Forgot Password} from the Sign In form:

\begin{verbatim}
In com.liferay.login.web.internal.portlet.action.ForgotPasswordMVCRenderCommand render
\end{verbatim}

Before deploying the sample, you must stop the original bundle you
intend to override. This is because the Ext sample's generated JAR
includes the original bundle source plus your modified source files.
Follow the instructions below to do this:

\begin{enumerate}
\def\labelenumi{\arabic{enumi}.}
\item
  Connect to your portal instance using
  \href{/docs/7-1/reference/-/knowledge_base/r/using-the-felix-gogo-shell}{Gogo
  Shell}.
\item
  Search for the bundle ID of the original bundle to override. To find
  the \texttt{com.liferay.login.web} bundle, execute this command:

\begin{verbatim}
lb -s | grep com.liferay.login.web
\end{verbatim}

  This returns output similar to this:

\begin{verbatim}
1580|Active   |   10|com.liferay.login.web (4.0.5)
\end{verbatim}

  Make note of the ID (e.g., \texttt{1580}).
\item
  Stop the bundle:

\begin{verbatim}
stop 1580
\end{verbatim}
\end{enumerate}

Once the original bundle is stopped, deploy the Ext module. Note that
you cannot leverage Blade or Gradle's \texttt{deploy} command to do
this. The \texttt{deploy} command deploys the module to the
\texttt{osgi\textbackslash{}marketplace\textbackslash{}override} folder
by default, which does not configure Ext modules properly for usage. You
should build and copy the Ext module's JAR to the \texttt{deploy} folder
manually, or leverage Liferay Dev Studio's
\href{/docs/7-2/reference/-/knowledge_base/r/deploying-a-project\#liferay-dev-studio}{deployment}
feature.

\section{What API(s) and/or code components does this sample
highlight?}\label{what-apis-andor-code-components-does-this-sample-highlight-22}

This sample demonstrates how to create an Ext module and configure it to
replace a default module bundle.

\section{How does this sample leverage the API(s) and/or code
component?}\label{how-does-this-sample-leverage-the-apis-andor-code-component-22}

You can create your own Ext module project by

\begin{itemize}
\tightlist
\item
  Declaring the original module name and version.
\item
  Providing the source code that will replace the original.
\end{itemize}

To declare the original module in the \texttt{build.gradle} file
properly (only supports Gradle), you must specify the original module's
Bundle Symbolic Name and the original module's exact version. In this
example, this is configured like this:

\begin{verbatim}
originalModule group: "com.liferay", name: "com.liferay.login.web", version: "4.0.5"
\end{verbatim}

If you're leveraging
\href{/docs/7-2/reference/-/knowledge_base/r/liferay-workspace}{Liferay
Workspace}, you should put your Ext module project in the \texttt{/ext}
folder (default); you can specify a different Ext folder name in
workspace's \texttt{gradle.properties} by adding

\begin{verbatim}
liferay.workspace.ext.dir=EXT_DIR
\end{verbatim}

If you are developing an Ext module project in standalone mode (not
associated with Liferay Workspace), you must declare the Ext Gradle
plugin in your \texttt{build.gradle}:

\begin{verbatim}
apply plugin: 'com.liferay.osgi.ext.plugin'
\end{verbatim}

Then you must provide your own code intended to replace the original
one. \textbf{Be sure to mimic the original module's folder structure
when overriding its JAR.}

The following file types can be overlaid with an Ext module:

\begin{itemize}
\tightlist
\item
  CSS
\item
  Java
\item
  JavaScript
\item
  Language files (\texttt{Language.properties})
\item
  Scss
\item
  Soy
\item
  etc.
\end{itemize}

The
\href{https://github.com/liferay/liferay-portal/blob/master/modules/sdk/gradle-plugins/src/main/java/com/liferay/gradle/plugins/LiferayOSGiExtPlugin.java}{Ext
Gradle Plugin} helps compile your code into the JAR. For example,
\texttt{.scss} files are compiled into \texttt{.css} files, which are
included in your module's JAR file artifact. This is done by the
\texttt{buildCSS} task.

\section{Where Is This Sample?}\label{where-is-this-sample-17}

There are two different versions of this sample, each built with a
different build tool:

\begin{itemize}
\tightlist
\item
  \href{https://github.com/liferay/liferay-blade-samples/tree/7.2/gradle/ext/login-web-ext}{Gradle}
\item
  \href{https://github.com/liferay/liferay-blade-samples/tree/7.2/liferay-workspace/ext/login-web-ext}{Liferay
  Workspace}
\end{itemize}

\chapter{Segmentation and Personalization
Reference}\label{segmentation-and-personalization-reference}

Browse this section's reference articles for additional information on
the Segmentation and Personalization framework.

\chapter{Defining Segmentation
Criteria}\label{defining-segmentation-criteria}

There are three categories for defining Segment criteria:

\begin{itemize}
\tightlist
\item
  User Properties
\item
  Organization Properties
\item
  Session Properties
\end{itemize}

There are several types of information that can be collected by the User
Segment interface. Some data is entered in text boxes, while others use
selectors to select specific criteria or tools like a date picker. In
addition, some fields use an operator, which, depending on the specific
context lets you select the relationship between the user or agent data
and the criteria:

\begin{itemize}
\item
  \emph{equals}
\item
  \emph{not equals}
\item
  \emph{greater than}
\item
  \emph{greater than or equals}
\item
  \emph{less than}
\item
  \emph{less than or equals}
\item
  \emph{contains}
\item
  \emph{does not contain}
\end{itemize}

Depending on the nature of the criteria, the operator selection may
contain different combinations. For example, the \emph{Date} selection
described below contains options for all the above option except
\emph{contains} and \emph{does not contain}, whereas the \emph{Email
Address} selection has \emph{equals}, \emph{not equals}, \emph{contains}
and \emph{does not contain}.

In between each criteria and each category, you can define an ``and'' or
``or'' conjunction. For ``and'' all criteria must be true in order for
the criteria to be satisfied. With ``or'' it will be true if any of the
defined criteria are true. You can also mix operators to create complex
cases.

\section{User Properties}\label{user-properties}

The following are the criteria available for defining user properties:

\textbf{Date Modified:} Provides a date picker and an relationship
selector to select the date that user information was last changed

\textbf{Email Address:} Provides a text box to enter the email provided
in the user's\\
profile.

\textbf{First Name:} Enter the first name provided in the user's
profile.

\textbf{Group:} Select a site that the user is a member of.

\textbf{Job Title:} Enter the job title provided in the user's profile.

\textbf{Last Name:} Enter the last name provided in the user's profile.

\textbf{Role:} Select a role that the user is a member of.

\textbf{Screen Name:} Enter the users' screen name.

\textbf{Team:} Select a team that the user is a member of.

\textbf{User Group:} Select a user group that the user is a member of.

\textbf{User:} Select a specific user from a list.

\textbf{Name:} The full name of the user.

\section{Organization Properties}\label{organization-properties}

\textbf{Date Modified:} Enter the date that the organization information
was last modified.

\textbf{Name:} Enter the name of the organization.

\textbf{Hierarchy Path:} Enter the name of an ancestor organization.

\textbf{Organization:} Select a specific organization.

\textbf{Parent Organization:} Select a specific parent organization.

\textbf{Type:} Select the type of organization, if organization types
have been defined.

\section{Session Properties}\label{session-properties}

\textbf{Browser:} Enter a property from the browser.

\textbf{Cookies:} Enter the name of a browser cookie.

\textbf{Device Brand:} Enter the brand name of the device being used.

\textbf{Device Model:} Enter the model name of the device being used.

\textbf{Device Screen Resolution Height:} Enter the screen resolution
height value.

\textbf{Device Screen Resolution Width:} Enter the screen resolution
width value.

\textbf{Language:} Select the current Language.

\textbf{Last Sign In Date:} Select the date of the user's last sign in.

\textbf{Local Date:} Select the current date where the user is located.

\textbf{Referrer URL:} Enter the URL that the user last visited.

\textbf{Signed In:} Select whether the user is signed in.

\textbf{URL:} Enter the current URL.

\textbf{User Agent:} Enter a User Agent property.

\chapter{Tooling}\label{tooling}

You can write code for Liferay DXP using any standard toolset. Liferay
is tool-agnostic, which frees you to work with whatever you're already
productive using.

Liferay has also created its own tools that streamline Liferay DXP
development. These tools integrate with popular build environments
(e.g., Gradle, Maven, and NodeJS). They include

\begin{itemize}
\tightlist
\item
  \href{/docs/7-2/reference/-/knowledge_base/r/blade-cli}{Blade CLI}: a
  command line interface used to build and manage Liferay Workspaces and
  Liferay DXP projects. This CLI is intended for Gradle or Maven
  development.
\item
  \href{/docs/7-2/reference/-/knowledge_base/r/liferay-workspace}{Liferay
  Workspace}: a generated Gradle/Maven environment built to hold and
  manage Liferay DXP projects.
\item
  \href{/docs/7-2/reference/-/knowledge_base/r/liferay-dev-studio}{Liferay
  Dev Studio}: an Eclipse-based IDE supporting development for Liferay
  DXP.
\item
  \href{/docs/7-2/reference/-/knowledge_base/r/intellij}{Liferay
  IntelliJ Plugin}: a plugin providing support for Liferay DXP
  development with IntelliJ IDEA.
\item
  \href{/docs/7-2/reference/-/knowledge_base/r/theme-generator}{Liferay
  Theme Generator}: a generator that creates themes, layouts templates,
  and themelets for Liferay DXP development.
\item
  \href{/docs/7-2/reference/-/knowledge_base/r/js-generator}{Liferay JS
  Generator}: a generator that creates JavaScript portlets with
  JavaScript tooling.
\end{itemize}

Liferay also provides a plethora of
\href{/docs/7-2/reference/-/knowledge_base/r/gradle-plugins}{Gradle} and
\href{/docs/7-2/reference/-/knowledge_base/r/maven-plugins}{Maven
plugins} you can apply to your projects. Many of these are already built
into tools such as Liferay Workspace.

Want samples or predefined project templates? Liferay has you covered
with 30+
\href{/docs/7-2/reference/-/knowledge_base/r/project-templates}{project
templates} and many more
\href{/docs/7-2/reference/-/knowledge_base/r/sample-projects}{project
samples}.

If you're a newbie looking for the best development tool for Liferay
DXP, or even a seasoned veteran looking for a tool you may like more
than your current setup, this section answers your tooling questions.

\chapter{Creating a Project}\label{creating-a-project}

Liferay provides many project templates you can use to generate starter
projects formatted in an opinionated way. Visit the
\href{/docs/7-2/reference/-/knowledge_base/r/project-templates}{Project
Templates} reference section for more information on the available
project templates. Each project template has different configurable
options, so be sure to research a project template before generating it.

You can use your desired tool to generate a project. The following tools
are preconfigured for Liferay project generation:

\begin{itemize}
\tightlist
\item
  \href{/docs/7-2/reference/-/knowledge_base/r/blade-cli}{Blade CLI}
\item
  \href{/docs/7-2/reference/-/knowledge_base/r/liferay-dev-studio}{Liferay
  Dev Studio}
\item
  \href{/docs/7-2/reference/-/knowledge_base/r/intellij}{Liferay
  IntelliJ Plugin}
\item
  \href{/docs/7-2/reference/-/knowledge_base/r/maven}{Maven}
\end{itemize}

It's recommended to create Liferay projects within a
\href{/docs/7-2/reference/-/knowledge_base/r/liferay-workspace}{Liferay
Workspace}. Most tools, however, support creating projects in a
standalone environment (except for IntelliJ). Visit the appropriate
section to learn how to create a project with the highlighted tool.

\section{Blade CLI}\label{blade-cli}

\begin{enumerate}
\def\labelenumi{\arabic{enumi}.}
\item
  Print the available project templates by executing this:

\begin{verbatim}
blade create -l
\end{verbatim}

  Note the project template you want to generate; you'll use it in the
  next step.
\item
  Run the following command to create a Gradle project with Blade CLI:

\begin{verbatim}
blade create -t [projectTemplate] [option1] [option2] ... [optionN] [projectName]
\end{verbatim}
\end{enumerate}

\noindent\hrulefill

\textbf{Note:} If you want to generate a project for a previous version
(e.g., Liferay Portal 7.0), you can specify this using the \texttt{-v}
flag. For example, to create a project for Liferay Portal 7.0, you would
include \texttt{-v\ 7.0} in your create command sequence.

\noindent\hrulefill

The available configuration options are documented for each
\href{/docs/7-2/reference/-/knowledge_base/r/project-templates}{project
template}. Run \texttt{blade\ create\ -\/-help} for the entire list of
available options.

\section{Liferay Dev Studio}\label{liferay-dev-studio}

\begin{enumerate}
\def\labelenumi{\arabic{enumi}.}
\item
  Navigate to \emph{File} → \emph{New} → \emph{Liferay Module Project}.
\item
  Specify the project name, location, build type, Liferay DXP version,
  and template type.

  \begin{figure}
  \centering
  \includegraphics{./images/liferay-project-wizard.png}
  \caption{The New Liferay Module Project wizard offers project
  templates for JAR and WAR-based projects.}
  \end{figure}
\item
  Click \emph{Next} and you're given additional configuration options
  based on the project template you selected. For example, if you
  selected a template that requires a component class, you must
  configure it in the wizard.

  \begin{figure}
  \centering
  \includegraphics{./images/component-class-wizard.png}
  \caption{Specify your component class's details in the Portlet
  Component Class Wizard.}
  \end{figure}

  You can specify your component class's name, package name, and its
  properties. The properties you assign are the ones found in the
  \texttt{@Component} annotation's \texttt{property\ =\ \{...\}}
  assignment.
\end{enumerate}

\noindent\hrulefill

\begin{verbatim}
 **Note:** You can also create a new component class for a pre-existing
 module project. Navigate to *File* &rarr; *New* &rarr; *Liferay Component
 Class*. This is a similar wizard to the previous component class wizard,
 except you can select a component class template. 
\end{verbatim}

\noindent\hrulefill

\begin{enumerate}
\def\labelenumi{\arabic{enumi}.}
\setcounter{enumi}{3}
\tightlist
\item
  Click \emph{Finish} to create your project.
\end{enumerate}

\section{Liferay IntelliJ Plugin}\label{liferay-intellij-plugin}

\begin{enumerate}
\def\labelenumi{\arabic{enumi}.}
\item
  Navigate to \emph{File} → \emph{New} → \emph{Liferay Module}.

  \begin{figure}
  \centering
  \includegraphics{./images/intellij-new-liferay-module.png}
  \caption{Selecting \emph{Liferay Module} opens the New Liferay Modules
  wizard.}
  \end{figure}
\item
  Select the project you want to create.

  \begin{figure}
  \centering
  \includegraphics{./images/intellij-modules.png}
  \caption{Choose the project template to create your module.}
  \end{figure}
\item
  Configure your project's SDK (i.e., JDK), package name, class name,
  and service name, if necessary. Then click \emph{Next}.
\item
  Give your project a name. Then click \emph{Finish}.
\end{enumerate}

\section{Maven}\label{maven}

\begin{enumerate}
\def\labelenumi{\arabic{enumi}.}
\item
  Execute the following Maven command:

\begin{verbatim}
mvn archetype:generate -Dfilter=liferay
\end{verbatim}
\item
  Select the archetype you want to leverage and proceed through the
  configuration prompts.
\end{enumerate}

\noindent\hrulefill

\textbf{Note:} Maven projects can also be generated using Blade CLI.
Follow \hyperref[blade-cli]{Blade CLI's} project creation instructions
and insert the \texttt{-b\ maven} parameter in the Blade command.

\noindent\hrulefill

Archetypes prefixed with
\texttt{com.liferay.project.templates.{[}TYPE{]}} or
\texttt{com.liferay.faces.archetype:{[}TYPE{]}} are compatible with 7.0.
All other Liferay archetypes are legacy archetypes targeted for previous
versions of Liferay DXP.

See Maven's
\href{http://maven.apache.org/archetype/maven-archetype-plugin/generate-mojo.html}{Archetype
Generation} documentation for further details on how to modify the Maven
archetype generation process.

\chapter{Deploying a Project}\label{deploying-a-project}

Deploying a project to Liferay DXP can be completed using your tool of
choice. The following tools are preconfigured (or can be easily
configured) for Liferay project generation:

\begin{itemize}
\tightlist
\item
  \href{/docs/7-2/reference/-/knowledge_base/r/blade-cli}{Blade CLI}
\item
  \href{https://gradle.org/}{Gradle}
\item
  \href{/docs/7-2/reference/-/knowledge_base/r/liferay-dev-studio}{Liferay
  Dev Studio}
\item
  \href{/docs/7-2/reference/-/knowledge_base/r/intellij}{Liferay
  IntelliJ Plugin}
\item
  \href{/docs/7-2/reference/-/knowledge_base/r/maven}{Maven}
\end{itemize}

The deployment process is the same across all tools; the deployment
command/action builds and deploys your project based on the build tool's
deployment configuration. For example, leveraging Blade CLI in a default
Gradle Liferay Workspace uses the underlying Gradle deployment
configuration. The build tool's deployment configuration is found by
reading the Liferay Home folder. The Liferay Home folder is
preconfigured in most cases; if it's not, ways to configure it are
included below. All tools support JAR and WAR-style project deployment.

It's recommended to deploy Liferay projects within a
\href{/docs/7-2/reference/-/knowledge_base/r/liferay-workspace}{Liferay
Workspace}. Most tools, however, support deploying projects in a
standalone environment (except for IntelliJ). Visit the appropriate
section to learn how to deploy a project with the highlighted tool.

\section{Blade CLI}\label{blade-cli-1}

This is the recommended way to deploy Gradle and Maven projects in a
Liferay Workspace via command line. Blade CLI is leveraged by Dev Studio
and IntelliJ too.

Run this command to deploy your project:

\begin{verbatim}
blade deploy
\end{verbatim}

If you prefer not to use your underlying build tool's (Gradle or Maven)
module deployment configuration, and instead, you want to deploy
straight to Liferay DXP's OSGi container, run this command:

\begin{verbatim}
blade deploy -l
\end{verbatim}

\section{Gradle}\label{gradle}

Deploying with pure Gradle is not recommended unless you prefer to
develop outside of a Liferay Workspace. \hyperref[blade-cli]{Blade CLI}
is a better tool for deploying Liferay Gradle projects in most cases.

\begin{enumerate}
\def\labelenumi{\arabic{enumi}.}
\item
  Apply the Liferay Gradle plugin in your project's
  \texttt{build.gradle} file:

\begin{verbatim}
apply plugin: "com.liferay.plugin"
\end{verbatim}
\item
  Extend the Liferay extension object to set your Liferay Home and
  \texttt{deploy} folder:

\begin{verbatim}
liferay {
    liferayHome = "../../../../liferay-ce-portal-7.1.1-ga2"
    deployDir = file("${liferayHome}/deploy")
}
\end{verbatim}
\item
  Run this command to deploy your project:

\begin{verbatim}
./gradlew deploy
\end{verbatim}
\end{enumerate}

\section{Liferay Dev Studio}\label{liferay-dev-studio-1}

These steps assume you've
\href{/docs/7-2/reference/-/knowledge_base/r/installing-a-liferay-server-in-dev-studio}{installed
a Liferay server in Dev Studio}.

\begin{enumerate}
\def\labelenumi{\arabic{enumi}.}
\item
  Right-click the server from the Servers window and select \emph{Add
  and Remove\ldots{}}.
\item
  Add the project(s) you'd like to deploy from the Available window to
  the Configured window. Then click \emph{Finish}.

  \begin{figure}
  \centering
  \includegraphics{./images/add-and-remove-ide.png}
  \caption{Using the this deployment method is convenient when deploying
  multiple projects.}
  \end{figure}
\item
  Verify your project builds, deploys, and starts successfully by
  viewing the results in the Console window.
\end{enumerate}

Dev Studio's deployment mechanism executes the \texttt{watch} task
behind the scenes. For more information on what to expect from the
\texttt{watch} task, see
\href{/docs/7-2/reference/-/knowledge_base/r/blade-cli}{this article}.

\section{Liferay IntelliJ Plugin}\label{liferay-intellij-plugin-1}

These steps assume you've
\href{/docs/7-2/reference/-/knowledge_base/r/installing-a-server-in-intellij}{installed
a Liferay server in IntelliJ}. A configured Liferay Workspace is
required to create and deploy Liferay projects in IntelliJ.

\begin{enumerate}
\def\labelenumi{\arabic{enumi}.}
\item
  Right-click your project from within the Liferay Workspace folder
  structure and select \emph{Liferay} → \emph{Deploy}.

  This automatically loads a build progress window viewable at the
  bottom of your IntelliJ instance.

  \begin{figure}
  \centering
  \includegraphics{./images/intellij-project-build.png}
  \caption{Verify that your project built successfully.}
  \end{figure}
\item
  Verify that your project builds successfully from the build progress
  window. Then navigate back to your server's window and confirm it
  starts in your configured Liferay DXP instance.
\end{enumerate}

\section{Maven}\label{maven-1}

If you're developing your project in a Liferay Workspace, skip to step
3.

\begin{enumerate}
\def\labelenumi{\arabic{enumi}.}
\item
  Add the following plugin configuration to your Liferay Maven project's
  parent \texttt{pom.xml} file.

\begin{verbatim}
<build>
    <plugins>
        <plugin>
            <groupId>com.liferay</groupId>
            <artifactId>com.liferay.portal.tools.bundle.support</artifactId>
            <version>3.4.1</version>
            <executions>
                <execution>
                    <id>deploy</id>
                    <goals>
                        <goal>deploy</goal>
                    </goals>
                    <phase>pre-integration-test</phase>
                </execution>
            </executions>
        </plugin>
    </plugins>
</build>
\end{verbatim}

  This POM configuration applies Liferay's
  \href{/docs/7-2/reference/-/knowledge_base/r/bundle-support-plugin}{Bundle
  Support plugin}. This plugin is applied in Liferay Workspace by
  default. The Bundle Support configuration defines the
  \href{https://maven.apache.org/guides/mini/guide-configuring-plugins.html\#Using_the_executions_Tag}{executions}
  tag, which configures the Bundle Support plugin to run during the
  \texttt{pre-integration-test} phase of your Maven project's build
  lifecycle. The
  \href{http://maven.apache.org/guides/introduction/introduction-to-the-lifecycle.html\#A_Build_Phase_is_Made_Up_of_Plugin_Goals}{\texttt{deploy}}
  goal is defined for that lifecycle phase.
\item
  Define your Liferay home folder in your POM. You can do this by adding
  the following logic within the \texttt{plugin} tags, but outside of
  the \texttt{execution} tags:

\begin{verbatim}
<configuration>
    <liferayHome>C:/liferay/liferay-ce-portal-7.1-ga1</liferayHome>
</configuration>
\end{verbatim}
\item
  Run this command to deploy your project:

\begin{verbatim}
mvn verify
\end{verbatim}
\end{enumerate}

\chapter{Blade CLI}\label{blade-cli-2}

\href{https://github.com/liferay/liferay-blade-cli/}{Blade CLI} is a
command line tool that makes it easy for Liferay developers to create,
manage, and deploy Liferay projects (Gradle or Maven). Blade CLI can

\begin{itemize}
\tightlist
\item
  Create Liferay projects usable in any IDE or development environment
\item
  Create/manage Liferay DXP instances
\item
  Deploy Liferay projects
\item
  And more
\end{itemize}

The table below describes all Blade CLI commands for the latest Blade
CLI release.

Command \textbar{} Description \texttt{convert} \textbar{} Converts a
Plugins SDK plugin project to a Gradle Workspace project. See the
\href{/docs/7-1/reference/-/knowledge_base/r/migrating-traditional-plugins-to-workspace-web-applications\#running-the-migration-command}{Running
the Migration Command} command for details. \texttt{create} \textbar{}
Creates a new Liferay project from available templates. See the
\href{/docs/7-2/reference/-/knowledge_base/r/creating-a-project\#blade-cli}{Creating
a Project section for Blade CLI} for more information. \texttt{deploy}
\textbar{} Builds and deploys projects to Liferay DXP. See the
\href{/docs/7-2/reference/-/knowledge_base/r/deploying-a-project\#blade-cli}{Deploying
a Project section for Blade CLI} for more information.
\texttt{extension\ install} \textbar{} Installs an extension into Blade
CLI. \texttt{extension\ uninstall} \textbar{} Uninstalls an extension
from Blade CLI. \texttt{gw} \textbar{} Executes a Gradle command using
the Gradle Wrapper, if detected (e.g., \texttt{blade\ gw\ tasks}).
\texttt{help} \textbar{} Provides information for Blade CLI's commands.
\texttt{init} \textbar{} Initializes a new Liferay Workspace. See the
\href{/docs/7-2/reference/-/knowledge_base/r/creating-a-liferay-workspace\#blade-cli}{Creating
a Liferay Workspace} article for more information. \texttt{samples}
\textbar{} Generates a sample project. See the
\href{/docs/7-2/reference/-/knowledge_base/r/generating-project-samples-with-blade-cli}{Generating
Project Samples with Blade CLI} article for more information.
\texttt{server\ init} \textbar{} Initializes the Liferay server
configured in Liferay Workspace's \texttt{gradle.properties} file. Set
the \texttt{liferay.workspace.bundle.url} property to configure the
server to initialize. \texttt{server\ start} \textbar{} Starts the
Liferay server in the background. You can add the \texttt{-d} flag to
start the server in debug mode. Debug mode can be customized by adding
the \texttt{-p} tag to set the custom remote debugging port (defaults
are \texttt{8000} for Tomcat and \texttt{8787} for Wildfly) and/or the
boolean \texttt{-s} tag to set whether you want to suspend the started
server until the debugger is connected. See the
\href{/docs/7-2/reference/-/knowledge_base/r/managing-your-liferay-server-with-blade-cli}{Managing
Your Liferay Server with Blade CLI} article for more information.
\texttt{server\ stop} \textbar{} Stops the Liferay server.
\texttt{server\ run} \textbar{} Starts the Liferay server in the
foreground. See the \texttt{server\ start} property for more
information. \texttt{sh} \textbar{} Connects to Liferay DXP, executes
succeeding Gogo command, and returns output. For example,
\texttt{blade\ sh\ lb} lists Liferay DXP's bundles using the Gogo shell.
See the
\href{/docs/7-2/reference/-/knowledge_base/r/managing-your-liferay-server-with-blade-cli}{Managing
Your Liferay Server with Blade CLI} article for more information.
\texttt{update} \textbar{} Updates Blade CLI to the latest version. See
the
\href{/docs/7-2/reference/-/knowledge_base/r/updating-blade-cli}{Updating
Blade CLI} article for details. \texttt{upgradeProps} \textbar{}
Analyzes your old \texttt{portal-ext.properties} and your newly
installed 7.x server to show you properties moved to OSGi configuration
files or removed from the product. \texttt{watch} \textbar{} Watches for
changes to a deployed project and automatically redeploys it when
changes are detected. This command does not rebuild your project and
copy it to Portal every time a change is detected, but rather, installs
it into the runtime as a reference. This means that the Portal does not
make a cached copy of the project. This allows the Portal to see changes
that are made to your project's files immediately. When you cancel the
\texttt{watch} task, your module is uninstalled automatically. The
\texttt{blade\ deploy\ -w} command works similarly to
\texttt{blade\ watch}, except it manually recompiles and deploys your
project every time a change is detected. This causes slower update
times, but does preserve your deployed project in Portal when it's shut
down. \texttt{version} \textbar{} Displays version information about
Blade CLI.

For information on command options, run the command with the
\texttt{-\/-help} flag (e.g., \texttt{blade\ samples\ -\/-help}).

Continue on to learn about leveraging Blade CLI to create and test
Liferay DXP instances and projects.

\chapter{Installing Blade CLI}\label{installing-blade-cli}

You can install Blade CLI using the Liferay Project SDK installer. This
installs JPM and Blade CLI into your user home folder and optionally
initializes a
\href{/docs/7-2/reference/-/knowledge_base/r/liferay-workspace}{Liferay
Workspace} folder. Do not use the Liferay Project SDK installer to
update Blade CLI; instead, follow the instructions for
\href{/docs/7-2/reference/-/knowledge_base/r/updating-blade-cli}{updating
Blade CLI}.

If you must configure proxy settings for Blade CLI, follow the
\href{/docs/7-2/reference/-/knowledge_base/r/installing-blade-cli-with-proxy-requirements}{Installing
Blade CLI with Proxy Requirements} instructions instead.

Follow the steps below to download and install Blade CLI:

\begin{enumerate}
\def\labelenumi{\arabic{enumi}.}
\item
  Download the latest
  \href{https://sourceforge.net/projects/lportal/files/Liferay\%20IDE/}{Liferay
  Project SDK installer} that corresponds with your operating system
  (e.g., Windows, MacOS, or Linux). The Project SDK installer is listed
  under \emph{Liferay IDE}, so the folder versions are based on IDE
  releases. You can select an installer without Dev Studio DXP if you
  don't intend to use it. The Project SDK installer is available for
  versions 3.2.0+. Do \textbf{not} select the large green download
  button; this downloads Liferay Portal instead.
\item
  Run the installer.
\item
  Select the Java Runtime to use with Blade CLI. Then click \emph{OK}.
\item
  Click \emph{Next} to step through the installer's introduction.
\item
  If you'd like to initialize a Liferay Workspace, you can set its
  location.

  \begin{figure}
  \centering
  \includegraphics{./images/blade-installer-workspace-init.png}
  \caption{Determine where your Liferay Workspace should reside, if you
  want one.}
  \end{figure}

  Select \emph{Don't initialize Liferay Workspace directory} if you only
  want to install Blade CLI. Then click \emph{Next}.
\item
  If you initialized a Liferay Workspace folder, an additional option
  appears for selecting the Liferay product type to use with your
  workspace. Choose the product type and click \emph{Next}.

  \begin{figure}
  \centering
  \includegraphics{./images/installer-workspace-type.png}
  \caption{Select the product version you'll use with your Liferay
  Workspace.}
  \end{figure}
\item
  Click \emph{Next} to begin installing Blade CLI/Liferay Workspace on
  your computer.
\end{enumerate}

That's it! Blade CLI is installed on your machine! If you specified a
location to initialize a Liferay Workspace folder, that is also
available.

If Blade CLI doesn't work properly on your machine, visit the
\href{/docs/7-2/reference/-/knowledge_base/r/common-errors-with-blade-cli}{Common
Errors with Blade CLI} article for solutions to common problems.

\chapter{Installing Blade CLI with Proxy
Requirements}\label{installing-blade-cli-with-proxy-requirements}

If you have proxy server requirements and want to use Blade CLI, you
must configure your http(s) proxy for it using JPM:

\begin{enumerate}
\def\labelenumi{\arabic{enumi}.}
\item
  Install JPM and Blade CLI using the Liferay Project SDK installer.
  Read the
  \href{/docs/7-2/reference/-/knowledge_base/r/installing-blade-cli}{Installing
  Blade CLI} tutorial for more details.
\item
  Execute the following command to configure your proxy requirements for
  Blade CLI:

\begin{verbatim}
jpm command --jvmargs "-Dhttp(s).proxyHost=[your proxy host] -Dhttp(s).proxyPort=[your proxy port]" jpm
\end{verbatim}
\end{enumerate}

Excellent! You've configured Blade CLI with your proxy settings using
JPM.

\chapter{Managing Your Liferay Server with Blade
CLI}\label{managing-your-liferay-server-with-blade-cli}

You can manage a Liferay server using Blade CLI. Managing a server with
Blade CLI should be done in a
\href{/docs/7-2/reference/-/knowledge_base/r/liferay-workspace}{Liferay
Workspace}.

Blade CLI has commands for installing, starting, stopping, inspecting,
and modifying a Liferay server:

\begin{enumerate}
\def\labelenumi{\arabic{enumi}.}
\item
  Make sure you've created a Liferay Workspace. See the
  \href{/docs/7-2/reference/-/knowledge_base/r/creating-a-liferay-workspace\#blade-cli}{Creating
  a Liferay Workspace} article for more information.
\item
  Initialize a Liferay server by running

\begin{verbatim}
blade server init
\end{verbatim}

  This downloads the Liferay DXP bundle set in your workspace's
  \texttt{gradle.propeties} file. See the
  \href{/docs/7-2/reference/-/knowledge_base/r/adding-a-liferay-bundle-to-liferay-workspace}{Adding
  a Liferay Bundle to Workspace} article for more information.

  You can initialize a server based on a
  \href{/docs/7-2/reference/-/knowledge_base/r/liferay-workspace\#testing-projects}{defined
  environment} by running the following command:

\begin{verbatim}
blade server init --environment [ENVIRONMENT]
\end{verbatim}

  For example, you could pass in the \texttt{uat} variable to generate a
  bundle with the configs set in the \texttt{configs/uat} workspace
  folder.
\item
  Start your Liferay server (Tomcat or Wildfly/JBoss) by running

\begin{verbatim}
blade server start
\end{verbatim}

  This starts the server in the background. You can tail the logs by
  adding the \texttt{-t} flag. If you prefer starting the server in the
  foreground, run \texttt{blade\ server\ run}. Additionally, if you
  prefer starting the server in debug mode, add the \texttt{-d} flag.
  See the \href{/docs/7-2/reference/-/knowledge_base/r/blade-cli}{Blade
  CLI} article for additional flags you can set when starting your
  Liferay server.
\item
  Examine your server's OSGi container by using Blade CLI's \texttt{sh}
  command, which provides access to your server using the Felix Gogo
  shell. For example, to check if you successfully deployed your
  application from the previous section, you could run:

\begin{verbatim}
blade sh lb
\end{verbatim}

  Your output lists a long list of modules that are active/installed in
  your server's OSGi container.

  \begin{figure}
  \centering
  \includegraphics{./images/blade-sh.png}
  \caption{Blade CLI accesses the Gogo shell script to run the
  \texttt{lb} command.}
  \end{figure}

  You can run any Gogo command using \texttt{blade\ sh}. This command
  requires
  \href{/docs/7-2/frameworks/-/knowledge_base/f/using-developer-mode-with-themes}{Developer
  Mode} to be enabled. Developer Mode is enabled in workspace by
  default. See the
  \href{/docs/7-2/customization/-/knowledge_base/c/using-the-felix-gogo-shell}{Using
  the Felix Gogo Shell} section for more information on this tool.
\item
  Once you're finished modifying your Liferay bundle, you can package it
  as a sharable file by running this command:

\begin{verbatim}
blade gw distBundle[Zip|Tar]
\end{verbatim}

  This lets you create a ZIP or TAR file to share with others. This
  option is only available with Gradle at this time. The above command
  leverages Blade CLI's \texttt{gw} option, which executes the project's
  Gradle wrapper.
\end{enumerate}

\noindent\hrulefill

\begin{verbatim}
 **Note:** You can avoid deploying a module inside your workspace's
 `modules/` folder when `distBundle[Zip|Tar]` is called by adding the
 following snippet to your workspace's `build.gradle` file:
 
 ```groovy
 distBundle {
     exclude "com.liferay.jsp.spy*.jar"
 }
 ```
 
 You can replace the JAR name above with the module JAR you want to exclude.
 This is useful for those who want to have a module in their workspace that
 is used for development or debug purposes only, and it should not be deployed
 to production. This works for Gradle builds only at this time.
\end{verbatim}

\noindent\hrulefill

\begin{verbatim}
<!-- TODO: Add way for producing a distributable workspace using Blade, when
available. It can only be done currently with ./gradlew distBundle[Zip|Tar].
-->
\end{verbatim}

\begin{enumerate}
\def\labelenumi{\arabic{enumi}.}
\setcounter{enumi}{5}
\item
  Turn off your Liferay server:

\begin{verbatim}
blade server stop
\end{verbatim}
\end{enumerate}

To reference all of Blade CLI's available options, see the
\href{/docs/7-2/reference/-/knowledge_base/r/blade-cli}{Blade CLI}
article.

Awesome! You learned how to interact with Liferay DXP using Blade CLI.

\chapter{Generating Project Samples with Blade
CLI}\label{generating-project-samples-with-blade-cli}

Liferay provides many useful
\href{https://github.com/liferay/liferay-blade-samples}{sample projects}
for those interested in learning best practices for Liferay DXP
projects. You can learn more about these samples by visiting
\href{/docs/7-2/reference/-/knowledge_base/r/sample-projects}{Sample
Projects}.

Rather than cloning the repository, you can generate these samples using
Blade CLI for convenience.

\begin{enumerate}
\def\labelenumi{\arabic{enumi}.}
\item
  List the available sample projects:

\begin{verbatim}
blade samples
\end{verbatim}

  Note the sample project you want to generate; you'll use it in the
  next step.
\item
  Run the following command to generate a sample project:

\begin{verbatim}
blade samples <NAME>
\end{verbatim}

  For example, to generate the
  \href{https://github.com/liferay/liferay-blade-samples/tree/master/gradle/apps/ds-portlet}{portlet-ds}
  sample, execute

\begin{verbatim}
blade samples ds-portlet
\end{verbatim}

  The sample is generated in the current folder.
\end{enumerate}

\noindent\hrulefill

\textbf{Note:} Interested in generating legacy versions of Blade
samples? Pass in the \texttt{-v} param followed by the Liferay DXP
version to target. For example,

\begin{verbatim}
blade samples -v 7.0 ds-portlet
\end{verbatim}

\noindent\hrulefill

Awesome! You've successfully generated a Liferay sample project using
Blade CLI!

\chapter{Updating Blade CLI}\label{updating-blade-cli}

Blade CLI is updated frequently, so you should update your Blade CLI
environment for new features. You can check the released versions of
Blade CLI on Nexus by inspecting the
\href{https://repository-cdn.liferay.com/nexus/content/repositories/liferay-public-releases/com/liferay/blade/com.liferay.blade.cli/}{\texttt{com.liferay.blade.cli}}
artifact. You can check your current installed version by running
\texttt{blade\ version}.

To update your Blade CLI installation to the latest stable version, run

\begin{verbatim}
blade update
\end{verbatim}

Although Blade CLI is frequently released, if you want bleeding edge
features not yet available, you can install the latest snapshot version:

\begin{verbatim}
blade update -s
\end{verbatim}

This pulls the latest snapshot version of Blade CLI and installs it to
your local machine. Running \texttt{blade\ version} after installing a
snapshot displays output similar to this:

\begin{verbatim}
blade version 3.3.1.SNAPSHOT201811301746
\end{verbatim}

Be careful; snapshot versions are unstable and should only be used for
experimental purposes.

Awesome! You've successfully learned how to update Blade CLI.

\chapter{Converting Plugins SDK Projects with Blade
CLI}\label{converting-plugins-sdk-projects-with-blade-cli}

Blade CLI can automatically migrate a Plugins SDK project to a Liferay
Workspace. During the process, the Ant-based Plugins SDK project is
copied to the applicable workspace folder based on its project type
(e.g., \texttt{wars}) and is converted to a Gradle-based Liferay
Workspace project. This drastically speeds up the migration process when
upgrading to a Liferay Workspace from a legacy Plugins SDK.

\noindent\hrulefill

\textbf{Note:} There is no Maven command for the migration process yet,
so you must complete it manually for Maven-based workspaces.

\noindent\hrulefill

To copy your Plugins SDK project and convert it to Gradle, use the Blade
\texttt{convert} command:

\begin{enumerate}
\def\labelenumi{\arabic{enumi}.}
\item
  Navigate to the root folder of your workspace in a command line tool.
\item
  Execute the following command:

\begin{verbatim}
blade convert -s [PLUGINS_SDK_PATH] [PLUGINS_SDK_PROJECT_NAME]
\end{verbatim}

  You must provide the path of the Plugins SDK your project resides in
  and the project name you want to convert. If you prefer converting all
  the Plugins SDK projects at once, replace the project name variable
  with \texttt{-a} (i.e., specifying all plugins).
\end{enumerate}

\noindent\hrulefill

\begin{verbatim}
 **Note:** If the `convert` task doesn't work as described above, you may
 need to update your Blade CLI version. See the
 [Updating Blade CLI](/docs/7-2/reference/-/knowledge_base/r/updating-blade-cli)
 article for more information.
\end{verbatim}

\noindent\hrulefill

\begin{verbatim}
This Gradle conversion process also works for themes; they're converted to
automatically leverage NodeJS. If you're converting a Java-based theme, add
the `-t` option to your command too. This will incorporate the
[Theme Builder](/docs/7-2/reference/-/knowledge_base/r/theme-builder-gradle-plugin)
Gradle plugin for the theme instead. For more information on upgrading
6.2 themes, see the
[Upgrade a 6.2 Theme to 7.2](/docs/7-2/tutorials/-/knowledge_base/t/upgrading-6-2-themes-to-7-2).
\end{verbatim}

\noindent\hrulefill

\textbf{Note:} When converting a Service Builder project, the
\texttt{convert} task automatically extracts the project's service
interfaces and implementations into OSGi modules (i.e., \emph{-impl and
}-api) and places them in the workspace's \texttt{modules} folder. Your
portlet and controller logic remain a WAR and reside in the
\texttt{wars} folder.

\noindent\hrulefill

Your project is successfully converted to a Gradle-based workspace
project! Great job!

\chapter{Extending Blade CLI}\label{extending-blade-cli}

You can extend Blade in three different ways:

\begin{itemize}
\tightlist
\item
  \href{/docs/7-2/reference/-/knowledge_base/r/creating-custom-commands-for-blade-cli}{Creating
  Custom Commands for Blade CLI}
\item
  \href{/docs/7-2/reference/-/knowledge_base/r/creating-custom-project-templates-for-blade-cli}{Creating
  Custom Project Templates for Blade CLI}
\item
  \href{/docs/7-2/reference/-/knowledge_base/r/installing-new-extensions-for-blade-cli}{Installing
  New Extensions for Blade CLI}
\end{itemize}

There are a few use cases to consider when extending Blade CLI. For
example, if you only want to add a new command that applies globally to
all types of workspaces, you can create and install a new custom command
as explained in the links above.

Alternatively, you may want a set of custom commands that only apply to
a specific workspace environment. Normally, Liferay developers who use
Blade CLI run a series of Blade commands that make sense in the
\emph{default} Liferay Workspace. What if the workspace, however, should
support a containerized environment (e.g., Docker) or some other
specialized environment? The commands used in the development workflow
must complete the workflow differently.

To customize Blade CLI's development workflow, you must create a Blade
\emph{profile}. Blade profiles \emph{override} existing Blade commands
or add \emph{new} commands in a preserved environment that can be
applied to any Liferay Workspace. For example, \texttt{blade\ init} for
a profile \texttt{myprofile} would override the default \texttt{init}
command to do something before/after the normal \texttt{init} command.
For more information, see
\href{/docs/7-2/reference/-/knowledge_base/r/creating-a-blade-profile}{Creating
a Blade Profile}.

\noindent\hrulefill

\textbf{Note:} Blade CLI leverages the profile system internally for
Maven support. The Maven specific code is stored in an extension JAR and
embedded inside the default Blade JAR.

\noindent\hrulefill

Continue on to learn more!

\chapter{Creating Custom Commands for Blade
CLI}\label{creating-custom-commands-for-blade-cli}

To create a custom command for Blade CLI, follow these steps:

\noindent\hrulefill

\textbf{Note:} This article creates a Gradle-based command project.
These steps can be completed for a Maven-based project too.

\noindent\hrulefill

\begin{enumerate}
\def\labelenumi{\arabic{enumi}.}
\item
  \href{/docs/7-2/reference/-/knowledge_base/r/creating-a-project}{Create
  a generic OSGi module}.
\item
  You'll leverage \href{http://jcommander.org/}{JCommander} and the
  Blade CLI API to create your custom command. Add these dependencies in
  your build file. For example, a \texttt{build.gradle} file's
  \texttt{dependencies} block looks like this:

\begin{verbatim}
dependencies {
    compileOnly group: "com.beust", name: "jcommander", version: "1.72"
    compileOnly group: "com.liferay.blade", name: "com.liferay.blade.cli", version: "latest.release"
}
\end{verbatim}
\item
  Build a Command class by extending the
  \href{https://github.com/liferay/liferay-blade-cli/blob/master/cli/src/main/java/com/liferay/blade/cli/command/BaseCommand.java}{\texttt{BaseCommand}}
  class:

\begin{verbatim}
import com.liferay.blade.cli.command.BaseCommand;

public class Hello extends BaseCommand<HelloArgs> {

    @Override
    public void execute() throws Exception {
        HelloArgs helloArgs = getArgs();

        getBladeCLI().out("Hello " + helloArgs.getName());
    }

    @Override
    public Class<HelloArgs> getArgsClass() {
        return HelloArgs.class;
    }

}
\end{verbatim}

  This registers your new command with Blade. You must define the
  \texttt{execute()} command for all classes extending
  \texttt{BaseCommand}. The \texttt{BaseCommand} class expects an
  arguments class as its parameter. You'll create this next.
\item
  Create a class that holds your command's arguments:

\begin{verbatim}
import com.beust.jcommander.Parameter;
import com.beust.jcommander.Parameters;

import com.liferay.blade.cli.command.BaseArgs;

@Parameters(commandDescription = "Executes a hello command", commandNames = "hello")
public class HelloArgs extends BaseArgs {

    public String getName() {
        return _name;
    }

    @Parameter(description = "The name to say hello to", names = "--name", required = true)
    private String _name;

}
\end{verbatim}

  This class extends the
  \href{https://github.com/liferay/liferay-blade-cli/blob/master/cli/src/main/java/com/liferay/blade/cli/command/BaseArgs.java}{\texttt{BaseArgs}}
  class. Notice that the class declaration has the \texttt{@Parameters}
  JCommander annotation. This sets your command's description and name.
  The \texttt{@Parameter} annotation applied to the private string
  \texttt{\_name} sets how the command's parameter is called and whether
  it's required.
\item
  Since Blade looks for custom commands using the
  \texttt{com.liferay.blade.cli.command.BaseCommand} service interface,
  you must use a standard JRE service loader mechanism to finish
  registering your new command with Blade CLI.

  Create a file named \texttt{com.liferay.blade.cli.command.BaseCommand}
  in the \texttt{src/main/resources/META-INF/services/} folder. This
  class should list all of your custom commands' fully qualified class
  names:

\begin{verbatim}
com.liferay.extensions.sample.command.Hello
\end{verbatim}
\end{enumerate}

\noindent\hrulefill

\begin{verbatim}
 **Note:** Java's Service Loader Interface (SPI) is used to load the
 fully qualified classes in the `META-INF/services` folder. You can learn
 more about SPIs
 [here](https://docs.oracle.com/javase/7/docs/api/java/util/ServiceLoader.html).
\end{verbatim}

\noindent\hrulefill

\begin{enumerate}
\def\labelenumi{\arabic{enumi}.}
\setcounter{enumi}{5}
\tightlist
\item
  Generate the extension's JAR file (e.g., \texttt{gradlew\ build}).
\end{enumerate}

Awesome! You've created a custom command! You can deploy multiple custom
commands in a single JAR, so you can continue adding custom command
projects to this module, if desired. See the
\href{/docs/7-2/reference/-/knowledge_base/r/installing-new-extensions-for-blade-cli}{Installing
New Extensions} article to install the command (JAR) to Blade CLI.

You can examine a working custom command project
\href{https://github.com/liferay/liferay-blade-cli/blob/master/extensions/sample-command}{here}.

\chapter{Creating Custom Project Templates for Blade
CLI}\label{creating-custom-project-templates-for-blade-cli}

Blade comes with 32+ project templates, but many times you may feel that
those are too simple or don't fit the need for your development team.
You can create new custom project templates that fit your team's
workflow and have Blade use them instead.

\noindent\hrulefill

\textbf{Important:} An extension JAR can only contain one template. If
you have multiple templates, each must be deployed in a separate JAR.

\noindent\hrulefill

Implementing a custom project template should mimic that of a Maven
archetype. The best way to illustrate this is by visualizing a
\href{https://github.com/liferay/liferay-blade-cli/tree/master/extensions/sample-template}{sample
template}'s folder structure:

\begin{itemize}
\tightlist
\item
  \texttt{src/}

  \begin{itemize}
  \tightlist
  \item
    \texttt{main/resources/}

    \begin{itemize}
    \tightlist
    \item
      \texttt{META-INF}

      \begin{itemize}
      \tightlist
      \item
        \texttt{maven}

        \begin{itemize}
        \tightlist
        \item
          \texttt{archetype-metadata.xml}
        \end{itemize}
      \item
        \texttt{archetype-post-generate.groovy} (optional; only invoked
        by Maven projects)
      \end{itemize}
    \item
      \texttt{archetype-resources}

      \begin{itemize}
      \tightlist
      \item
        Folder structure to be generated
      \end{itemize}
    \end{itemize}
  \end{itemize}
\item
  \texttt{bnd.bnd}
\item
  \texttt{{[}build.gradle\textbar{}pom.xml{]}}
\end{itemize}

You can read more about Maven archetypes and their features and
capabilities
\href{https://maven.apache.org/guides/introduction/introduction-to-archetypes.html}{here}.

To create a custom project template that can be generated using Blade
CLI, follow these steps:

\begin{enumerate}
\def\labelenumi{\arabic{enumi}.}
\item
  Create a generic Maven archetype following the folder structure
  outlined above. Follow Maven's documentation to configure the
  archetype project appropriately.
\item
  Open the template's \texttt{bnd.bnd} file and ensure it sets the
  following configurations:

\begin{verbatim}
Bundle-Description: TEMPLATE_DESCRIPTION
Bundle-Name: TEMPLATE_NAME
Bundle-SymbolicName: SYMBOLIC_NAME
Bundle-Version: TEMPLATE_VERSION
Liferay-Versions: LIFERAY_VERSION_RANGE
-removeheaders:\
    Import-Package,\
    Private-Package,\
    Require-Capability
\end{verbatim}

  For example, a template's \texttt{bnd.bnd} could look like this:

\begin{verbatim}
Bundle-Description: Creates a Sample as a module project.
Bundle-Name: Liferay Project Templates Sample
Bundle-SymbolicName: com.liferay.project.templates.sample
Bundle-Version: 1.0.0
Liferay-Versions: [7,8)
-removeheaders:\
    Import-Package,\
    Private-Package,\
    Require-Capability
\end{verbatim}

  The \texttt{Bundle-SymbolicName} of your template JAR must have the
  pattern \texttt{*.project.templates.\textless{}name\textgreater{}.*}.
  The \texttt{-removeheaders} definition is a packaging requirement for
  all project templates. For more information on Bnd versioning, visit
  \href{https://bnd.bndtools.org/chapters/170-versioning.html}{Bnd's
  official docs}.
\item
  Generate the extension's JAR file (e.g., \texttt{gradlew\ build}).
\end{enumerate}

It's that easy! You've created a custom project template. See the
\href{/docs/7-2/reference/-/knowledge_base/r/installing-new-extensions-for-blade-cli}{Installing
New Extensions} article to install the project template (JAR) to Blade
CLI.

You can examine a working custom project template
\href{https://github.com/liferay/liferay-blade-cli/blob/master/extensions/sample-template}{here}.

\chapter{Installing New Extensions for Blade
CLI}\label{installing-new-extensions-for-blade-cli}

After you've created a new extension for Blade CLI, you must install it
so it's available for use. You can learn how to create
\href{/docs/7-2/reference/-/knowledge_base/r/creating-custom-commands-for-blade-cli}{custom
commands} and
\href{/docs/7-2/reference/-/knowledge_base/r/creating-custom-project-templates-for-blade-cli}{custom
project templates} in their respective articles.

When Blade CLI starts, it looks in the user's
\texttt{\$\{user.home\}/.blade/extensions} folder for JAR files. All JAR
files are searched to see if they contain valid Blade extensions. You'll
learn how to install new extensions next.

\section{Installing a New Extension}\label{installing-a-new-extension}

To install an extension, you must move the extension JAR to the user's
\texttt{\$\{user.home\}/.blade/extensions} folder. You can do this
automatically from Blade CLI by running

\begin{verbatim}
blade extension install /path/to/my_extension.JAR
\end{verbatim}

You can verify that the extension is available by running the following
commands, depending on extension type:

\textbf{Custom Command:}

\begin{verbatim}
blade help
\end{verbatim}

\textbf{Custom Project Template:}

\begin{verbatim}
blade create -l
\end{verbatim}

Great! You've installed a new extension!

\section{Uninstalling an Extension}\label{uninstalling-an-extension}

You can uninstall a Blade extension by running this:

\begin{verbatim}
blade extension uninstall EXTENSION_NAME.jar
\end{verbatim}

This removes the extension JAR from the user's
\texttt{\$\{user.home\}/.blade/extensions} folder.

\chapter{Creating a Blade Profile}\label{creating-a-blade-profile}

There are two steps to follow when adding a new Blade profile:

\begin{itemize}
\tightlist
\item
  \hyperref[creating-a-new-profile]{Creating a new profile}
\item
  \hyperref[setting-a-profile]{Setting the profile in Liferay Workspace}
\end{itemize}

You'll learn how to create a profile first.

\section{Creating a New Profile}\label{creating-a-new-profile}

To create a new Blade profile, follow these steps:

\begin{enumerate}
\def\labelenumi{\arabic{enumi}.}
\item
  \href{/docs/7-2/reference/-/knowledge_base/r/creating-a-project}{Create
  a generic OSGi module}.
\item
  To create a new command, create the command and arguments classes
  extending
  \href{https://github.com/liferay/liferay-blade-cli/blob/master/cli/src/main/java/com/liferay/blade/cli/command/BaseCommand.java}{\texttt{BaseCommand}}
  and
  \href{https://github.com/liferay/liferay-blade-cli/blob/master/cli/src/main/java/com/liferay/blade/cli/command/BaseArgs.java}{\texttt{BaseArgs}},
  respectively, as described in the
  \href{/docs/7-2/reference/-/knowledge_base/r/creating-custom-commands-for-blade-cli}{Creating
  Custom Commands} article. These classes should reside in the profile
  module's \texttt{src/main/java/PACKAGE\_NAME} folder. These classes
  register your command and arguments to Blade CLI.
\item
  To override a default command, follow the same steps outlined
  \href{/docs/7-2/reference/-/knowledge_base/r/creating-custom-commands-for-blade-cli}{here}:

  \begin{itemize}
  \tightlist
  \item
    Create a command class
  \item
    Create an arguments class
  \item
    Define your commands' fully qualified class names for the service
    loader
  \end{itemize}

  Instead of extending the
  \href{https://github.com/liferay/liferay-blade-cli/blob/master/cli/src/main/java/com/liferay/blade/cli/command/BaseCommand.java}{\texttt{BaseCommand}}
  and
  \href{https://github.com/liferay/liferay-blade-cli/blob/master/cli/src/main/java/com/liferay/blade/cli/command/BaseArgs.java}{\texttt{BaseArgs}},
  classes, however, extend the command/arguments classes defined for the
  command you intend to override. Make sure to also set the
  \texttt{@Parameters} annotation's \texttt{commandNames} argument to
  the command to override.

  For example, if you intend to override the default \texttt{deploy}
  command, your arguments class declaration would look like this:

\begin{verbatim}
@Parameters(commandDescription = "Overridden Deploy Command", commandNames = "deploy")
public class OverriddenArgs extends DeployArgs {

}
\end{verbatim}

  The corresponding command class override would look like this:

\begin{verbatim}
public class OverriddenCommand extends BaseCommand<OverriddenArgs> {

    @Override
    public void execute() throws Exception {
            OverriddenArgs args = getArgs();

        getBladeCLI().out("OverriddenCommand says " + args.isWatch());
    }

    @Override
    public Class<OverriddenArgs> getArgsClass() {
        return OverriddenArgs.class;
    }
}
\end{verbatim}

  You can search for the default command/arguments classes
  \href{https://github.com/liferay/liferay-blade-cli/tree/master/cli/src/main/java/com/liferay/blade/cli/command}{here}.
\item
  To associate a command to your new profile, set the
  \href{https://github.com/liferay/liferay-blade-cli/blob/master/cli/src/main/java/com/liferay/blade/cli/command/BladeProfile.java}{\texttt{BladeProfile}}
  annotation in your command class:

\begin{verbatim}
@BladeProfile("myprofile")
public class NewCommand extends BaseCommand<NewArgs> {

}
\end{verbatim}

  The annotation's parameter should specify the profile you want to
  associate the command with (e.g., \texttt{myprofile}).
\end{enumerate}

Excellent! You've created a new Blade profile and learned how to add new
commands or override default commands by leveraging the profile.

\noindent\hrulefill

\textbf{Note:} Command classes spanning multiple JARs can use/share the
same profile name. For example, if you want to extend the internal
(provided) \texttt{maven} profile extension with new commands, you can
do it externally the same way you'd create a new profile.

\noindent\hrulefill

You can reference the
\href{https://github.com/liferay/liferay-blade-cli/tree/master/extensions/sample-profile}{sample
profile project} to examine a new command and overridden command's setup
in a custom profile.

Next, you'll learn how to set your new profile for use in a Liferay
Workspace.

\section{Setting a Profile}\label{setting-a-profile}

To set your new Blade profile in a Liferay Workspace, open the
\texttt{\$\{workspaceDir\}/.blade.properties} file and set the
\texttt{profile.name} property to your profile name:

\begin{verbatim}
profile.name=myprofile
\end{verbatim}

This specifies which Blade profile is active and uses its defined
commands. The default setting is \texttt{gradle}. You can also set this
property to \texttt{maven} out-of-the-box, which is applied for
Maven-based workspaces. You can only set one profile for a workspace.

You can specify the Blade profile for a workspace when initializing it
too. This is done by passing the profile name as an argument when
creating the workspace:

\begin{verbatim}
blade init -P [profile-name] [workspace-name]
\end{verbatim}

For example, if you execute the following command:

\begin{verbatim}
blade init -P myprofile my-new-custom-workspace
\end{verbatim}

Your \texttt{my-new-custom-workspace} has the following properties set
in its \texttt{\$\{workspaceDir\}/.blade.properties} file:

\begin{verbatim}
liferay.version.default=7.2
profile.name=myprofile
\end{verbatim}

\noindent\hrulefill

\textbf{Note:} The \texttt{-P} profile parameter can be used for any
command to specify the profile to use for that command. This is helpful
if you want to run a command not associated with the workspace's current
profile.

\noindent\hrulefill

Awesome! You've set your new Blade profile!

\chapter{Common Errors with Blade
CLI}\label{common-errors-with-blade-cli}

If Blade CLI isn't working as expected, you may find answers here.

The following issues are documented:

\begin{itemize}
\tightlist
\item
  \hyperref[the-blade-command-is-not-available-in-my-cli]{The blade
  command is not available in my CLI}
\item
  \hyperref[i-cant-update-my-blade-cli-version]{I can't update my Blade
  CLI version}
\end{itemize}

Visit the appropriate section to learn more.

\section{The blade command is not available in my
CLI}\label{the-blade-command-is-not-available-in-my-cli}

The Liferay Project SDK installer attempts to add JPM to your path. For
Windows, it uses the Windows registry. For Mac/Linux, it updates
\texttt{.bashrc} or \texttt{.zshrc}.

At a minimum, Mac/Linux users must open a new shell after the installer
finishes for the new features to be available. If, however, you're using
a different shell (e.g., Korn, csh) or you've customized your CLI via
\texttt{.profile} or some other configuration file, you must add JPM to
your path manually.

To add JPM's required \texttt{bin} folder, execute the appropriate
command based on your operating system.

macOS:

\begin{verbatim}
echo 'export PATH="$PATH:$HOME/Library/PackageManager/bin"' >> ~/.bash_profile
\end{verbatim}

Linux:

\begin{verbatim}
echo 'export PATH="$PATH:$HOME/jpm/bin"' >> ~/.bash_profile
\end{verbatim}

Once you open a new shell, the \texttt{blade} command should be
available.

\section{I can't update my Blade CLI
version}\label{i-cant-update-my-blade-cli-version}

If you run \texttt{blade\ version} after updating, but don't see the
expected version installed, you may have two separate Blade CLI
installations on your machine. This is typically caused if you installed
an earlier version of Blade CLI and then used the Liferay Project SDK
installer (at any time prior) to update the older Blade CLI instance.
This is not recommended. Doing this installs Blade CLI in the global and
user home folder of your machine. The latest Blade CLI update process
installs to your user home folder, so you must delete the legacy Blade
files in your global folder, if present. To do this, navigate to your
\texttt{GLOBAL\_FOLDER/JPM4J} folder and delete

\begin{itemize}
\tightlist
\item
  \texttt{/bin/blade}
\item
  \texttt{/commands/blade}
\end{itemize}

The newest Blade CLI installation in your user home folder is now
recognized and available.

\chapter{Liferay Dev Studio}\label{liferay-dev-studio-2}

Liferay Dev Studio is an extension for the
\href{https://www.eclipse.org/ide/}{Eclipse} platform for developing
Liferay DXP plugins. It works with build tools such as Gradle and Maven
and configuration tools like BndTools.

Dev Studio makes Liferay development easier by providing an intuitive
GUI. It contains

\begin{itemize}
\tightlist
\item
  Liferay-specific wizards for creating/developing Liferay DXP projects.
\item
  Liferay server management and project testing capabilities.
\item
  Editors for Service Builder files, workflow definitions, POM files,
  and more.
\item
  Snippets for tag libraries
\item
  etc.
\end{itemize}

Here you'll learn how to

\begin{itemize}
\tightlist
\item
  \href{/docs/7-2/reference/-/knowledge_base/r/installing-liferay-dev-studio}{Install
  Dev Studio}
\item
  \href{/docs/7-2/reference/-/knowledge_base/r/setting-proxy-requirements-for-dev-studio}{Set
  Proxy Requirements}
\item
  \href{/docs/7-2/reference/-/knowledge_base/r/installing-a-liferay-server-in-dev-studio}{Install
  a Liferay server}
\item
  \href{/docs/7-2/reference/-/knowledge_base/r/importing-projects-in-dev-studio}{Import
  a Liferay project}
\item
  \href{/docs/7-2/reference/-/knowledge_base/r/using-the-gogo-shell-in-dev-studio}{Use
  the Gogo Shell}
\item
  \href{/docs/7-2/reference/-/knowledge_base/r/searching-product-source-in-dev-studio}{Search
  Liferay DXP source}
\item
  \href{/docs/7-2/reference/-/knowledge_base/r/debugging-product-source-in-dev-studio}{Debug
  Liferay DXP source}
\item
  \href{/docs/7-2/reference/-/knowledge_base/r/updating-liferay-dev-studio}{Update
  Dev Studio}
\end{itemize}

You can also find
\href{/docs/7-2/reference/-/knowledge_base/r/gradle-in-dev-studio}{Gradle}
and
\href{/docs/7-2/reference/-/knowledge_base/r/maven-in-dev-studio}{Maven}
reference articles that highlight popular use cases for those respective
build tools in Dev Studio.

For help
\href{/docs/7-2/reference/-/knowledge_base/r/creating-a-project\#liferay-dev-studio}{creating}
and
\href{/docs/7-2/reference/-/knowledge_base/r/deploying-a-project\#liferay-dev-studio}{deploying}
projects with Dev Studio, or
\href{/docs/7-2/reference/-/knowledge_base/r/creating-a-liferay-workspace\#dev-studio}{creating
a Liferay Workspace}, visit their respective articles.

Let Dev Studio aid in your conquest for Liferay DXP development.
Continue on to learn how!

\chapter{Installing Liferay Dev
Studio}\label{installing-liferay-dev-studio}

Liferay Dev Studio is a plugin for Eclipse that brings many
Liferay-specific features to the table. You can install it into your
existing Eclipse environment, or Liferay provides a bundled version
included in its Project SDK. Here you'll learn the different methods
available for installing Dev Studio:

\begin{itemize}
\tightlist
\item
  \hyperref[install-the-dev-studio-bundle]{install the Dev Studio bundle
  from scratch}
\item
  \hyperref[install-dev-studio-into-eclipse]{install Dev Studio into an
  existing Eclipse instance using an update URL}
\item
  \hyperref[install-dev-studio-into-eclipse-from-a-zip-file]{install Dev
  Studio into an existing Eclipse instance using a ZIP file}
\end{itemize}

\textbf{Important:} If you're installing Dev Studio into an existing
Eclipse environment, you must be on Eclipse Photon or newer. For
instructions on upgrading to Photon, see Eclipse's
\href{https://wiki.eclipse.org/FAQ_How_do_I_upgrade_Eclipse_IDE\%3F\#Upgrading_existing_Eclipse_IDE_and_Installed_Features_to_newer_release}{upgrade
documentation}. With this particular upgrade, you should also deactivate
the current available update sites in the \emph{Window} →
\emph{Preferences} → \emph{Install/Update} → \emph{Available Software
Sites} menu to ensure a successful upgrade (e.g., Oxygen).

\section{Install the Dev Studio
Bundle}\label{install-the-dev-studio-bundle}

\begin{enumerate}
\def\labelenumi{\arabic{enumi}.}
\item
  Download and install \href{http://java.oracle.com}{Java}. Liferay runs
  on Java, so you'll need it to run everything else. Because you'll be
  developing apps for Liferay DXP in Dev Studio, the Java Development
  Kit (JDK) is required. It is an enhanced version of the Java
  Environment used for developing new Java technology. You can download
  the Java SE JDK from the Java
  \href{http://www.oracle.com/technetwork/java/javase/downloads/index.html}{Downloads}
  page.
\item
  Download Liferay's latest
  \href{https://sourceforge.net/projects/lportal/files/Liferay\%20IDE/}{Project
  SDK with Dev Studio}. Go to the
  \href{https://customer.liferay.com/group/customer/downloads}{Downloads}
  page in Liferay's Help Center. Select \emph{Developer Tools} in the
  Product drop-down and \emph{Developer Studio} for the file type. Then
  select the executable that correlates to your operating system.

  The Project SDK includes Dev Studio DXP,
  \href{/docs/7-2/reference/-/knowledge_base/r/liferay-workspace}{Liferay
  Workspace}, and
  \href{/docs/7-2/reference/-/knowledge_base/r/blade-cli}{Blade CLI}.
\item
  Run the Project SDK executable and step through the installer to
  install everything to your machine. For help with setting up proxy
  settings (if necessary), see the
  \href{/docs/7-2/reference/-/knowledge_base/r/setting-proxy-requirements-for-dev-studio}{Dev
  Studio Proxy Settings} and
  \href{/docs/7-2/reference/-/knowledge_base/r/setting-proxy-requirements-for-liferay-workspace}{Liferay
  Workspace Proxy Settings} articles for more information.
\end{enumerate}

Congratulations! You've installed Liferay Dev Studio! It's now available
in the Project SDK folder's \texttt{liferay-developer-studio}. To run
Dev Studio, execute the \texttt{DeveloperStudio} executable. A Liferay
Workspace has also been initialized in that same folder.

\section{Install Dev Studio into
Eclipse}\label{install-dev-studio-into-eclipse}

If you already have an Eclipse environment that you're using for other
things, it's easy to add Dev Studio to your existing Eclipse
installation.

\begin{enumerate}
\def\labelenumi{\arabic{enumi}.}
\item
  In Eclipse, select \emph{Help} → \emph{Install New Software}.
\item
  In the \emph{Work with} field, copy in the following URL:
  https://releases.liferay.com/tools/ide/latest/stable/
\item
  You'll see the Liferay Dev Studio components in the list below. Check
  them off and click \emph{Next}.
\item
  Accept the terms of the agreements and click \emph{Next}, and Dev
  Studio is installed. Like other Eclipse plugins you'll have to restart
  Eclipse to enable it.
\end{enumerate}

\section{Install Dev Studio into Eclipse from a ZIP
File}\label{install-dev-studio-into-eclipse-from-a-zip-file}

To install Liferay Dev Studio into Eclipse from a Zip file, follow these
steps:

\begin{enumerate}
\def\labelenumi{\arabic{enumi}.}
\item
  Go to the
  \href{https://community.liferay.com/en_GB/project/-/asset_publisher/TyF2HQPLV1b5/content/ide-installation-instructions}{Dev
  Studio} downloads page. Under \emph{Other Downloads}, select the
  \emph{Liferay IDE {[}version{]} Archived Update-site} option to
  download it.
\item
  In Eclipse, go to \emph{Help} → \emph{Install New Software\ldots{}}.
\item
  In the \emph{Add} dialog, click the \emph{Archive} button and browse
  to the location of the downloaded Dev Studio Zip file.
\item
  You'll see the Dev Studio components in the list below. Check them off
  and click \emph{Next}.
\item
  Accept the terms of the agreements and click \emph{Next}, and Liferay
  Dev Studio is installed. Like other Eclipse plugins you'll have to
  restart Eclipse to enable it.
\end{enumerate}

Awesome! You've installed Liferay Dev Studio. Now you can begin Liferay
development using a popular and supported IDE.

\chapter{Setting Proxy Requirements for Dev
Studio}\label{setting-proxy-requirements-for-dev-studio}

If you have proxy server requirements and want to configure your http(s)
proxy\\
to work with Liferay Dev Studio, follow the instructions below.

\begin{enumerate}
\def\labelenumi{\arabic{enumi}.}
\item
  Navigate to Eclipse's \emph{Window} → \emph{Preferences} →
  \emph{General} → \emph{Network Connections} menu.
\item
  Set the \emph{Active Provider} drop-down selector to \emph{Manual}.
\item
  Under \emph{Proxy entries}, configure both proxy HTTP and HTTPS by
  clicking the field and selecting the \emph{Edit} button.

  \begin{figure}
  \centering
  \includegraphics{./images/ide-network-connections.png}
  \caption{You can configure your proxy settings in Dev Studio's Network
  Connections menu.}
  \end{figure}
\item
  For each schema (HTTP and HTTPS), enter your proxy server's host,
  port, and authentication settings (if necessary). Do not leave
  whitespace at the end of your proxy host or port settings.
\item
  Once you've configured your proxy entry, click \emph{Apply and Close}.
\end{enumerate}

If you're working with a Liferay Workspace in Dev Studio, you'll need to
configure your proxy settings for that environment too. See the
\href{/docs/7-2/reference/-/knowledge_base/r/setting-proxy-requirements-for-liferay-workspace}{Setting
Proxy Requirements for Liferay Workspace} article for more details.

Awesome! You've successfully configured Dev Studio's proxy settings!

\section{Additional Proxy Settings}\label{additional-proxy-settings}

Some Eclipse plugins do not properly check the \texttt{core.net} proxy
infrastructure when setting proxy settings via \emph{Window} →
\emph{Preferences} → \emph{General} → \emph{Network Connections}.
Therefore, you may need to configure additional proxy settings.

To do so, open the \texttt{eclipse.ini} file associated with your
Eclipse installation and add the following entries:

\begin{verbatim}
-vmargs
-Dhttp.proxyHost=www.somehost.com
-Dhttp.proxyPort=1080
-Dhttp.proxyUser=userId
-Dhttp.proxyPassword=somePassword
-Dhttps.proxyHost=www.somehost.com
-Dhttps.proxyPort=1080
-Dhttps.proxyUser=userId
-Dhttps.proxyPassword=somePassword
\end{verbatim}

After saving the file, restart Eclipse. Now your additional proxy
settings are applied!

\chapter{Installing a Liferay Server in Dev
Studio}\label{installing-a-liferay-server-in-dev-studio}

Dev Studio offers a single GUI for managing a Liferay server and its
deployed projects. A server is installed and managed from the Servers
view (lower left corner in Eclipse).

For reference, here's how the Dev Studio server buttons work with your
Liferay DXP instance:

\begin{itemize}
\tightlist
\item
  \emph{Start} (\includegraphics{./images/icon-start-server.png}):
  Starts the server.
\item
  \emph{Stop} (\includegraphics{./images/icon-stop-server.png}): Stops
  the server.
\item
  \emph{Debug} (\includegraphics{./images/icon-debug-server.png}):
  Starts the server in debug mode. For more information on debugging in
  Dev Studio, see the
  \href{/docs/7-2/reference/-/knowledge_base/r/debugging-product-source-in-dev-studio}{Debugging
  Liferay DXP source in Dev Studio} article.
\end{itemize}

Follow these steps to install your server. Note you must have already
downloaded and de-compressed the server bundle:

\begin{enumerate}
\def\labelenumi{\arabic{enumi}.}
\item
  In the Servers view, click the \emph{No Servers are available} link.
  If you already have a server installed, you can install a new server
  by right-clicking in the Servers view and selecting \emph{New} →
  \emph{Server}. This brings up a wizard that walks you through the
  process of defining a new server.
\item
  Select the type of server you would like to create from the list of
  available options. For a standard server, open the \emph{Liferay,
  Inc.} folder and select the \emph{Liferay 7.x} option. You can change
  the server name to something more unique too; this is the name
  displayed in the Servers view. Then click \emph{Next}.

  \begin{figure}
  \centering
  \includegraphics{./images/define-new-server.png}
  \caption{Choose the type of server you want to create.}
  \end{figure}

  \textbf{Note:} If you've already configured previous Liferay servers,
  you'll be provided the \emph{Server runtime environment} field, which
  lets you choose previously configured runtime environments. If you
  want to re-add an existing server, select one from the dropdown menu.
  You can also add a new server by selecting \emph{Add}, or you can edit
  existing servers by selecting \emph{Configure runtime environments}.
  Once you've configured the server runtime environment, select
  \emph{Finish}. If you selected an existing server, your server
  installation is finished; you can skip steps 3-5.
\item
  Enter a name for your server. This is the name for the Liferay DXP
  runtime configuration used by Dev Studio. This is not the display name
  used in the Servers tab.
\item
  Browse to the installation folder of the Liferay DXP bundle. For
  example,
  \texttt{C:\textbackslash{}liferay-ce-portal-7.2.0-m2\textbackslash{}tomcat-9.0.10}.

  \begin{figure}
  \centering
  \includegraphics{./images/specify-bundle-directory.png}
  \caption{Specify the installation folder of the bundle.}
  \end{figure}
\item
  Select a runtime JRE and click \emph{Finish}. Your new server appears
  under the Servers view.

  \begin{figure}
  \centering
  \includegraphics{./images/new-server-added.png}
  \caption{Your new server appears under the \emph{Servers} view.}
  \end{figure}
\end{enumerate}

Congratulations! Your server is now available in Liferay Dev Studio!

\chapter{Importing Projects in Dev
Studio}\label{importing-projects-in-dev-studio}

There are two types of Liferay projects you can import into Dev Studio:

\begin{itemize}
\tightlist
\item
  Liferay Module Project (this also includes WAR-style projects)
\item
  Liferay Workspace Project
\end{itemize}

You cannot import Liferay projects individually that reside in a
\href{/docs/7-2/reference/-/knowledge_base/r/liferay-workspace}{Liferay
Workspace}. You can either import a non-workspace Liferay project (or
group of projects if the parent folder is specified) or an entire
workspace project with all its Liferay projects.

To import a pre-existing Liferay project into Dev Studio, follow the
steps outlined below:

\begin{enumerate}
\def\labelenumi{\arabic{enumi}.}
\item
  Right-click in the Project Explorer and select \emph{Import} →
  \emph{Liferay Module Project}. If you're interested in importing an
  entire Liferay Workspace, select the \emph{Liferay Workspace Project}
  instead.

  \begin{figure}
  \centering
  \includegraphics{./images/import-liferay-project.png}
  \caption{You can import a single project or folder of projects.}
  \end{figure}

  Once you've selected your project(s), the project build type is
  displayed.
\item
  Click \emph{Finish}.
\end{enumerate}

Now your Liferay project is available from the Package Explorer.

\chapter{Using the Gogo Shell in Dev
Studio}\label{using-the-gogo-shell-in-dev-studio}

If you're using Dev Studio to develop and deploy your projects, you may
be interested in managing them after they're deployed with Dev Studio
too. You can do this with the Dev Studio's
\href{/docs/7-2/customization/-/knowledge_base/c/using-the-felix-gogo-shell}{Gogo
Shell} feature.

\begin{enumerate}
\def\labelenumi{\arabic{enumi}.}
\item
  Right-click your started portal instance in the Servers view.
\item
  Select \emph{Open Gogo Shell}.

  \begin{figure}
  \centering
  \includegraphics{./images/open-gogo-shell.png}
  \caption{Select \emph{Open Gogo Shell} to open a terminal window in
  Dev Studio using Gogo shell.}
  \end{figure}

  A Gogo shell terminal appears, allowing you to enter Gogo commands to
  inspect your Liferay instance and the projects deployed to it.
\item
  A common use case for the Gogo Shell is verifying successful project
  deployment. Enter the \texttt{lb} command to view a list of deployed
  bundles. If the project status is active, then it deployed
  successfully.

  \begin{figure}
  \centering
  \includegraphics{./images/gogo-deploy-successful.png}
  \caption{You can check to see if your project deployed successfully to
  Liferay using the Gogo shell.}
  \end{figure}
\end{enumerate}

\textbf{Important:} Dev Studio's Gogo shell usage requires
\href{/docs/7-2/frameworks/-/knowledge_base/f/using-developer-mode-with-themes}{Developer
Mode} to be enabled. Developer Mode is enabled in
\href{/docs/7-2/reference/-/knowledge_base/r/liferay-workspace}{Liferay
Workspace} by default.

Excellent! You've learned how to manage your deployed projects with Dev
Studio's Gogo Shell integration.

\chapter{Searching Liferay DXP Source in Dev
Studio}\label{searching-liferay-dxp-source-in-dev-studio}

In Liferay Dev Studio, you can search through Liferay DXP's source code
to aid in the development of your project. Liferay provides great
resources to help with development (e.g., official documentation,
\href{https://docs.liferay.com/}{docs.liferay.com},
\href{/docs/7-2/reference/-/knowledge_base/r/sample-projects}{sample
projects}, etc.), but sometimes searching through Liferay's codebase
(i.e., platform and official apps) for patterns is just as useful. For
example, if you're creating an application that extends a class provided
in Liferay's \texttt{portal-kernel} JAR, you can inspect that class and
research how it's used in other areas of Liferay DXP's codebase.

To do this, you must be developing in a
\href{/docs/7-2/reference/-/knowledge_base/r/liferay-workspace}{Liferay
Workspace}. Liferay Workspace is able to provide this functionality by
targeting a specific Liferay DXP version, which indexes the configured
Liferay DXP source code to provide advanced search. See the
\href{/docs/7-2/reference/-/knowledge_base/r/managing-the-target-platform}{Managing
the Target Platform in Liferay Workspace} tutorial for more information
on how this works.

Workspace does not perform portal source indexing by default. You must
enable this functionality by adding the following property to your
workspace's \texttt{gradle.properties} file:

\begin{verbatim}
target.platform.index.sources=true
\end{verbatim}

\noindent\hrulefill

\textbf{Note:} Portal source indexing is disabled in Gradle workspace
version 2.0.3+ (Target Platform plugin version 2.0.0+).

\noindent\hrulefill

In this tutorial, you'll explore three use cases where advanced search
would be useful.

\begin{itemize}
\tightlist
\item
  \hyperref[search-class-hierarchy]{Search class hierarchy}
\item
  \hyperref[search-method-declarations]{Search declarations}
\item
  \hyperref[search-annotation-references]{Search references}
\end{itemize}

These examples are just a small subset of what you can search in Liferay
Dev Studio. See Eclipse's documentation on
\href{http://help.eclipse.org/oxygen/index.jsp?topic=\%2Forg.eclipse.jdt.doc.user\%2Fconcepts\%2Fconcept-java-search.htm&resultof=\%22\%6a\%61\%76\%61\%22\%20}{Java
Search} for a comprehensive guide.

\section{Search Class Hierarchy}\label{search-class-hierarchy}

Inspecting classes that extend a similar superclass can help you find
useful patterns and examples for how you can develop your own app. For
example, suppose your app extends the
\href{https://docs.liferay.com/dxp/portal/7.2-latest/javadocs/portal-kernel/com/liferay/portal/kernel/portlet/bridges/mvc/MVCPortlet.html}{MVCPortlet}
class. You can search classes that extend that same class in Dev Studio.
Complete the steps below for a simple example:

\begin{enumerate}
\def\labelenumi{\arabic{enumi}.}
\item
  Right-click the \texttt{MVCPortlet} declaration.
\item
  Select \emph{Open Type Hierarchy}.
\end{enumerate}

This opens a window that lets you inspect all classes residing in the
target platform that extend \texttt{MVCPortlet}.

\begin{figure}
\centering
\includegraphics{./images/open-type-hierarchy.png}
\caption{Browse the Type Hierarchy window and open the provided classes
for examples on how to extend a class.}
\end{figure}

Great! Now you can search for all extensions and implementations of a
class/interface to aid in your quest for developing the perfect app.

\section{Search Method Declarations}\label{search-method-declarations}

Sometimes you want a search to be more granular, exploring the
declarations of a specific method provided by a class/interface. Dev
Studio's advanced search has no limits; Liferay Workspace's target
platform indexing provides method exploration too!

Suppose in the
\href{https://docs.liferay.com/dxp/portal/7.2-latest/javadocs/portal-kernel/com/liferay/portal/kernel/portlet/bridges/mvc/MVCPortlet.html}{MVCPortlet}
class you're extending, you want to search for declarations of its
\texttt{doView} method you're overriding. Here's how to do it:

\begin{enumerate}
\def\labelenumi{\arabic{enumi}.}
\item
  Right-click the \texttt{doView} method declaration in your custom
  app's class.
\item
  Select \emph{Declarations} → \emph{Workspace}.
\end{enumerate}

\begin{figure}
\centering
\includegraphics{./images/inspect-declared-method.png}
\caption{All declarations of the method are returned in the Search
window.}
\end{figure}

The rendered Search window displays the other occurrences in the target
platform where that method was overridden.

\section{Search Annotation
References}\label{search-annotation-references}

Annotations used in Liferay DXP's source code can sometimes be cryptic.
You can find out how they can be used in your own application by
searching for where these types of annotations exist in Liferay's target
platform.

For example, you may find some documentation on using the
\texttt{@Reference} annotation in an OSGi module and implement it in
your custom app. It could be useful to reference real world examples in
Liferay DXP's apps to check how it was used elsewhere. You can complete
this search like this:

\begin{enumerate}
\def\labelenumi{\arabic{enumi}.}
\item
  Right-click the \texttt{@Reference} annotation in a class.
\item
  Select \emph{References} → \emph{Workspace}.
\end{enumerate}

\begin{figure}
\centering
\includegraphics{./images/inspect-references-ide.png}
\caption{All matching annotations are displayed in the Search window.}
\end{figure}

The rendered Search window displays the other occurrences in the target
platform where that annotation was used.

Excellent! You now have the tools to search the configured target
platform specified in your Liferay Workspace!

\chapter{Debugging Liferay DXP Source in Dev
Studio}\label{debugging-liferay-dxp-source-in-dev-studio}

You can debug Liferay DXP source code in Dev Studio to help resolve
errors. Debugging Liferay DXP code follows most of the same techniques
associated with debugging in Eclipse. If you need help with general
debugging, you can visit Eclipse's documentation. Here's some helpful
Eclipse links to visit:

\begin{itemize}
\tightlist
\item
  \href{http://help.eclipse.org/oxygen/index.jsp?topic=\%2Forg.eclipse.jdt.doc.user\%2Fconcepts\%2Fcdebugger.htm&cp=1_2_9}{Debugger}
\item
  \href{http://help.eclipse.org/oxygen/index.jsp?topic=\%2Forg.eclipse.jdt.doc.user\%2Fconcepts\%2Fclocdbug.htm&cp=1_2_11}{Local
  Debugging}
\item
  \href{http://help.eclipse.org/oxygen/index.jsp?topic=\%2Forg.eclipse.jdt.doc.user\%2Fconcepts\%2Fcremdbug.htm&cp=1_2_12}{Remote
  Debugging}
\end{itemize}

There are a couple Liferay-specific configurations to know before
debugging Liferay DXP code:

\begin{itemize}
\tightlist
\item
  \hyperref[configure-your-target-platform]{Configure your target
  platform.}
\item
  \hyperref[configure-a-liferay-server-and-start-it-in-debug-mode]{Configure
  a Liferay server and start it in debug mode.}
\end{itemize}

Let's explore these Liferay-specific debugging configurations.

\section{Configure Your Target
Platform}\label{configure-your-target-platform}

To configure your target platform, you must be developing in a
\href{/docs/7-2/reference/-/knowledge_base/r/liferay-workspace}{Liferay
Workspace}. Liferay Workspace is able to provide debugging capabilities
by targeting a specific Liferay DXP version, which indexes the
configured Liferay DXP source code. Liferay Workspace does not perform
portal source indexing by default. You must enable this functionality by
adding the following property to your workspace's
\texttt{gradle.properties} file:

\begin{verbatim}
target.platform.index.sources=true
\end{verbatim}

\noindent\hrulefill

\textbf{Note:} Portal source indexing is disabled in Gradle workspace
version 2.0.3+ (Target Platform plugin version 2.0.0+).

\noindent\hrulefill

Without specifying a target platform, Liferay DXP's source code cannot
be accessed by Dev Studio. See the
\href{/docs/7-2/reference/-/knowledge_base/r/managing-the-target-platform}{Managing
the Target Platform in Liferay Workspace} tutorial for more information
on how this works.

\textbf{Important:} The target platform should match the Liferay server
version you configure in the next section.

Once the target platform is configured in your workspace, Eclipse has
access to all of Liferay DXP's source code. Next, you'll configure a
Liferay server and learn how to start it in Debug mode.

\section{Configure a Liferay Server and Start It in Debug
Mode}\label{configure-a-liferay-server-and-start-it-in-debug-mode}

Configuring your target platform gives Eclipse Liferay DXP's source code
to reference. Now you must configure a Liferay server matching the
target platform version so you can deploy the custom code you wish to
debug.

\begin{enumerate}
\def\labelenumi{\arabic{enumi}.}
\item
  Set up your Liferay DXP server to run in Dev Studio. See the
  \href{/docs/7-2/reference/-/knowledge_base/r/installing-a-liferay-server-in-dev-studio}{Installing
  a Server in Dev Studio} for more details.
\item
  Start the server in debug mode. To do this, click the debug button in
  the Servers pane of Dev Studio.

  \begin{figure}
  \centering
  \includegraphics{./images/ide-debug.png}
  \caption{The red box in this screenshot highlights the debug button.
  Click this button to start the server in debug mode.}
  \end{figure}
\end{enumerate}

Awesome! You're now equipped to begin debugging in Liferay Dev Studio!

\chapter{Updating Liferay Dev Studio}\label{updating-liferay-dev-studio}

If you're already using Liferay Dev Studio but must update your
environment, follow the steps below:

\begin{enumerate}
\def\labelenumi{\arabic{enumi}.}
\item
  In Dev Studio, go to \emph{Help} → \emph{Install New
  Software\ldots{}}.
\item
  In the \emph{Work with} field, copy in the URL
  http://releases.liferay.com/tools/ide/latest/stable/.
\item
  You'll see the Dev Studio components in the list below. Check them off
  and click \emph{Next}.

  \begin{figure}
  \centering
  \includegraphics{./images/ide-updatesite-install.png}
  \caption{Make sure to check all the Dev Studio components you wish to
  install.}
  \end{figure}
\item
  Accept the terms of the agreements. Click \emph{Next}, and Dev Studio
  is updated. You must restart Dev Studio for the updates to take
  effect.
\end{enumerate}

You're now on the latest version of Liferay Dev Studio!

\chapter{Gradle in Dev Studio}\label{gradle-in-dev-studio}

\href{http://gradle.org/}{Gradle} is a popular open source build
automation system. You can take full advantage of Gradle in Liferay Dev
Studio through
\href{https://projects.eclipse.org/releases/photon}{Buildship}, a
collection of Eclipse plugins that provide support for building software
using Gradle with Liferay Dev Studio. Buildship is bundled with Liferay
Dev Studio versions 3.0 and higher.

\begin{figure}
\centering
\includegraphics{./images/buildship-in-liferayide.png}
\caption{Navigate to \emph{Help} → \emph{Installation Details} to view
plugins included in Dev Studio.}
\end{figure}

This reference article highlights some useful tips for leveraging Gradle
in Dev Studio.

\begin{itemize}
\tightlist
\item
  \hyperref[creating-pure-gradle-projects]{Creating Pure Gradle
  Projects}
\item
  \hyperref[importing-pure-gradle-projects]{Importing Pure Gradle
  Projects}
\item
  \hyperref[gradle-tasks-and-executions]{Gradle Tasks and Executions}
\end{itemize}

Note that
\href{/docs/7-2/reference/-/knowledge_base/r/creating-a-project\#liferay-dev-studio}{creating
Liferay Gradle projects} and
\href{/docs/7-2/reference/-/knowledge_base/r/deploying-a-project\#liferay-dev-studio}{deploying
Gradle projects} with Dev Studio are covered in their respective
articles.

The first thing you'll learn about in this tutorial is creating pure
Gradle projects in Dev Studio.

\section{Creating Pure Gradle
Projects}\label{creating-pure-gradle-projects}

Most of Dev Studio's wizards rely on your usage of Liferay Workspace.
This is for good reason; it's the recommended developer environment for
Liferay projects. You can, however, create pure Gradle projects and
manually configure them to be deployable to Liferay DXP.

You can create a pure Gradle project by using the Gradle Project wizard.

\begin{enumerate}
\def\labelenumi{\arabic{enumi}.}
\item
  Navigate to \emph{File} → \emph{New} → \emph{Project\ldots{}}.
\item
  Select \emph{Gradle} → \emph{Gradle Project}. Then click \emph{Next} →
  \emph{Next}.
\item
  Enter a valid project name. You can also specify your project location
  and working sets.
\item
  Optionally, you can navigate to the next page and specify your Gradle
  distribution and other advanced options. Once you're finished, select
  \emph{Finish}.
\end{enumerate}

\begin{figure}
\centering
\includegraphics{./images/new-gradle-project.png}
\caption{You can specify your Gradle distribution and advanced options
such as home directories, JVM options, and program arguments.}
\end{figure}

Excellent! You've created a pure Gradle project using Dev Studio.

\section{Importing Pure Gradle
Projects}\label{importing-pure-gradle-projects}

You can also import existing pure Gradle projects in Dev Studio.

\begin{enumerate}
\def\labelenumi{\arabic{enumi}.}
\item
  Go to \emph{File} → \emph{Import\ldots{}}.
\item
  Select \emph{Gradle} → \emph{Existing Gradle Project} → \emph{Next} →
  \emph{Next}.

  \begin{figure}
  \centering
  \includegraphics{./images/import-gradle-project.png}
  \caption{You can specify what Gradle project to import from the
  \emph{Import Gradle Project} wizard.}
  \end{figure}
\item
  Click the \emph{Browse\ldots{}} button to choose a Gradle project.
\item
  Optionally, you can navigate to the next page and specify your Gradle
  distribution and other advanced options. Once you're finished, click
  \emph{Next} again to review the import configuration. Select
  \emph{Finish} once you've confirmed your Gradle project import.
\end{enumerate}

Next you'll learn about Gradle tasks and executions, and learn how to
run and display them in Dev Studio.

\section{Gradle Tasks and Executions}\label{gradle-tasks-and-executions}

Dev Studio provides two views to enhance your development experience
using Gradle: Gradle Tasks and Gradle Executions. You can open these
views by following the instructions below.

\begin{enumerate}
\def\labelenumi{\arabic{enumi}.}
\item
  Go to \emph{Window} → \emph{Show View} → \emph{Other\ldots{}}.
\item
  Navigate to the \emph{Gradle} folder and open \emph{Gradle Tasks} and
  \emph{Gradle Executions}.
\end{enumerate}

Gradle tasks and executions views open automatically once you create or
import a Gradle project.

The Gradle Tasks view lets you display the Gradle tasks available for
you to use in your Gradle project. Users can execute a task listed under
the Gradle project by double-clicking it.

\begin{figure}
\centering
\includegraphics{./images/gradle-tasks.png}
\caption{Navigate into your preferred Gradle project to view its
available Gradle tasks.}
\end{figure}

Once you've executed a Gradle task, you can open the Gradle Executions
view to inspect its output.

\begin{figure}
\centering
\includegraphics{./images/gradle-executions.png}
\caption{The Gradle Executions view helps you visualize the Gradle build
process.}
\end{figure}

Keep in mind that if you change the Gradle build scripts inside your
Gradle projects (e.g., \texttt{build.gradle} or
\texttt{settings.gradle}), you must refresh the project so Dev Studio
can account for the change and display it properly in your views. To
refresh a Gradle project, right-click on the project and select
\emph{Gradle} → \emph{Refresh Gradle Project}.

\begin{figure}
\centering
\includegraphics{./images/refresh-gradle-project.png}
\caption{Make sure to always refresh your Gradle project in Dev Studio
after build script edits.}
\end{figure}

If you prefer Eclipse refresh your Gradle projects automatically,
navigate to \emph{Window} → \emph{Preferences} → \emph{Gradle} and
enable the \emph{Automatic Project Synchronization} checkbox. If you'd
like to enable Gradle's automatic synchronization for just one Gradle
project, you can right-click a Gradle project and select
\emph{Properties} → \emph{Gradle} and enable auto sync that way.

Excellent! You're now equipped with the knowledge to add, import, and
build your Gradle projects in Liferay Dev Studio!

\chapter{Maven in Dev Studio}\label{maven-in-dev-studio}

You can take full advantage of \href{https://maven.apache.org/}{Maven}
in Liferay Dev Studio with its built-in Maven support. In this article,
you'll learn about the following topics:

\begin{itemize}
\tightlist
\item
  \hyperref[installing-maven-plugins-for-dev-studio]{Installing Maven
  Plugins for Liferay Dev Studio}
\item
  \hyperref[importing-maven-projects]{Importing Maven Projects}
\item
  \hyperref[using-the-pom-graphic-editor]{Using the POM Graphic Editor}
\end{itemize}

Note that
\href{/docs/7-2/reference/-/knowledge_base/r/creating-a-project\#liferay-dev-studio}{creating}
and
\href{/docs/7-2/reference/-/knowledge_base/r/deploying-a-project\#liferay-dev-studio}{deploying}
Maven projects with Dev Studio are covered in their respective articles.

First you'll install the necessary Maven plugins for Dev Studio.

\section{Installing Maven Plugins for Dev
Studio}\label{installing-maven-plugins-for-dev-studio}

To support Maven projects in Dev Studio properly, you first need a
mechanism to recognize Maven projects as Dev Studio projects. For Dev
Studio to recognize the Maven project and for it to be able to leverage
Java EE tooling features (e.g., the Servers view) with the project, the
project must be a flexible web project. Dev Studio relies on the
following Eclipse plugins to provide this capability:

\begin{itemize}
\tightlist
\item
  \texttt{m2e} (Maven integration for Eclipse)
\item
  \texttt{m2e-wtp} (Maven integration for WTP)
\end{itemize}

All you have to do is install them so you can begin developing Maven
projects for Liferay DXP.

When first installing Dev Studio, the installation startup screen asks
if you want to install the Maven plugins automatically. Don't worry if
you missed this during setup. You'll learn how to install the required
Maven plugins for Dev Studio manually below.

\begin{enumerate}
\def\labelenumi{\arabic{enumi}.}
\item
  Navigate to \emph{Help} → \emph{Install New Software}. In the
  \emph{Work with} field, insert the following value:

\begin{verbatim}
http://releases.liferay.com/tools/ide/latest/stable/
\end{verbatim}
\item
  Check the \emph{Liferay IDE Maven Support} option. This bundles all
  the required Maven plugins you need to begin developing Maven projects
  for Liferay DXP.

  \begin{figure}
  \centering
  \includegraphics{./images/maven-install-ide-plugins.png}
  \caption{You can install all the necessary Maven plugins for Dev
  Studio by installing the \emph{Liferay IDE Maven Support} option.}
  \end{figure}

  If the \emph{Liferay IDE Maven Support} option does not appear, then
  it's already installed. To verify that it's installed, uncheck the
  \emph{Hide items that are already installed} checkbox and look for
  \emph{Liferay IDE Maven Support} in the list of installed plugins.
  Also, if you want to view everything that is bundled with the
  \emph{Liferay IDE Maven Support} option, uncheck the \emph{Group items
  by category} checkbox.
\item
  Click \emph{Next}, review the install details, accept the term and
  license agreements, and select \emph{Finish}.
\end{enumerate}

\noindent\hrulefill

\textbf{Note:} Both Maven and Eclipse have their own standard build
project lifecycles that are independent from each other. For both to
work together and run seamlessly within Dev Studio, a lifecycle mapping
is required to link both lifecycles into one combined lifecycle.
Normally, this would have to be done manually by the user. Fortunately,
the m2e-liferay plugin combines the lifecycle metadata mapping and
Eclipse build lifecycles, to provide a seamless user experience. The
lifecycle mappings for your project can be viewed by right-clicking your
project and selecting \emph{Properties} → \emph{Maven} → \emph{Lifecycle
Mapping}.

\noindent\hrulefill

Awesome! Your Dev Studio is ready to develop Maven projects for Liferay
DXP!

You'll learn about importing Maven projects in Dev Studio next.

\section{Importing Maven Projects}\label{importing-maven-projects}

To import a pre-existing, non-Liferay Maven project into Dev Studio,
follow the steps outlined below:

\begin{enumerate}
\def\labelenumi{\arabic{enumi}.}
\item
  Navigate to \emph{File} → \emph{Import} → \emph{Maven} →
  \emph{Existing Maven Projects} and click \emph{Next}.

  \begin{figure}
  \centering
  \includegraphics{./images/import-maven-project.png}
  \caption{Dev Studio offers the Maven folder in the Import wizard.}
  \end{figure}
\item
  Click \emph{Browse\ldots{}} and select the root folder for your Maven
  project. Once you've selected it, the \texttt{pom.xml} for that
  project should be visible in the Projects menu.

  \begin{figure}
  \centering
  \includegraphics{./images/select-maven-import.png}
  \caption{Use the Import Maven Projects wizard to import your
  pre-existing project.}
  \end{figure}
\item
  Click \emph{Finish}.
\end{enumerate}

Now your Maven project is available from the Package Explorer. To import
a Liferay project built with Maven, see the
\href{/docs/7-2/reference/-/knowledge_base/r/importing-projects-in-dev-studio}{Importing
Projects in Dev Studio} Next you'll learn about Dev Studio's POM
graphical editor.

\section{Using the POM Graphic
Editor}\label{using-the-pom-graphic-editor}

You're provided a graphical POM editor when opening your Maven project's
\texttt{pom.xml} in Dev Studio. This gives you several different ways to
leverage the power of Maven in your project:

\begin{itemize}
\item
  \textbf{Overview:} provides a graphical interface where you can add to
  and edit the \texttt{pom.xml} file.
\item
  \textbf{Dependencies:} provides a graphical interface for adding and
  editing dependencies in your project, as well as modifying the
  \texttt{dependencyManagement} section of the \texttt{pom.xml} file.
\item
  \textbf{Effective POM:} provides a read-only version of your project
  POM merged with its parent POM(s), \texttt{settings.xml}, and the
  settings in Eclipse for Maven.
\item
  \textbf{Dependency Hierarchy:} provides a hierarchical view of project
  dependencies and an interactive listing of resolved dependencies.
\item
  \textbf{pom.xml:} provides an editor for your POM's source XML.
\end{itemize}

The figure below shows the \texttt{pom.xml} file editor and its modes.

\begin{figure}
\centering
\includegraphics{./images/pom-editor-features.png}
\caption{Liferay Dev Studio provides five interactive modes to help you
edit and organize your POM..}
\end{figure}

By taking advantage of these interactive modes, Dev Studio makes
modifying and organizing your POM and its dependencies a snap!

\chapter{IntelliJ}\label{intellij}

The
\href{https://plugins.jetbrains.com/plugin/10739-liferay-intellij-plugin}{Liferay
IntelliJ plugin} provides support for Liferay DXP development in
\href{https://www.jetbrains.com/idea/}{IntelliJ IDEA}. Liferay's
IntelliJ plugin provides the following built-in features:

\begin{itemize}
\tightlist
\item
  Creating a Liferay Workspace (Gradle and Maven based)
\item
  Creating Liferay projects (Gradle and Maven based)
\item
  Liferay DXP Tomcat/Wildfly server support for project deployment and
  debugging
\item
  Support for adding line markers for each entity in the service editor
\item
  Syntax checking, highlighting, and code completion (e.g., Bnd and XML
  files)
\end{itemize}

In these articles, you'll learn how to install the Liferay IntelliJ
plugin and leverage its features to improve Liferay development with
IntelliJ IDEA.

\chapter{Installing the Liferay IntelliJ
Plugin}\label{installing-the-liferay-intellij-plugin}

There are two ways to install the Liferay IntelliJ plugin:

\begin{itemize}
\tightlist
\item
  \hyperref[installing-via-intellij-marketplace]{via IntelliJ
  Marketplace}
\item
  \hyperref[installing-via-zip-file]{via Zip file}
\end{itemize}

Follow the steps pertaining to your preferred installation process.

\section{Installing Via IntelliJ
Marketplace}\label{installing-via-intellij-marketplace}

\begin{enumerate}
\def\labelenumi{\arabic{enumi}.}
\item
  In IntelliJ, navigate to \emph{File} → \emph{Settings} →
  \emph{Plugins}.
\item
  In the Marketplace tab, search for \emph{Liferay} in the provided
  search bar.
\item
  Click \emph{Install} next to the Liferay IntelliJ Plugin.

  \begin{figure}
  \centering
  \includegraphics{./images/intellij-marketplace-installation.png}
  \caption{IntelliJ Marketplace offers a streamlined way to install
  plugins.}
  \end{figure}
\item
  After the plugin has downloaded, select \emph{Restart IDE}.
\end{enumerate}

Once IntelliJ restarts, the Liferay IntelliJ plugin is installed and
ready for use.

\section{Installing Via Zip File}\label{installing-via-zip-file}

\begin{enumerate}
\def\labelenumi{\arabic{enumi}.}
\item
  Navigate to the
  \href{https://plugins.jetbrains.com/plugin/10739-liferay-intellij-plugin}{JetBrains'
  Liferay IntelliJ plugin} page and download it to your local machine.
\item
  In IntelliJ, navigate to \emph{File} → \emph{Settings} →
  \emph{Plugins}.
\item
  Click the gear icon from the top menu and select \emph{Install Plugin
  from Disk\ldots{}}.
\item
  Select the Liferay IntelliJ plugin and click \emph{OK}.
\item
  Navigate to the Installed tab in the top menu and select \emph{Restart
  IDE}.
\end{enumerate}

Once IntelliJ restarts, the Liferay IntelliJ plugin is installed and
ready for use.

Great job! You're now ready to develop for Liferay DXP in IntelliJ!

\chapter{Installing a Server in
IntelliJ}\label{installing-a-server-in-intellij}

Installing a Liferay server in IntelliJ is easy. In just a few steps,
you'll have your server up and running.

\noindent\hrulefill

\textbf{Note:} Tomcat and Wildfly are the only supported Liferay app
server bundles available to install in IntelliJ.

\noindent\hrulefill

Follow these steps to install your server:

\begin{enumerate}
\def\labelenumi{\arabic{enumi}.}
\item
  Right-click your Liferay Workspace and select \emph{Liferay} →
  \emph{InitBundle}.

  This downloads the Liferay DXP bundle specified in your workspace's
  \texttt{gradle.properties} file. You can change the default bundle by
  updating the \texttt{liferay.workspace.bundle.url} property. For
  example, this is required to update the default bundle version and/or
  type (e.g., Wildfly). The downloaded bundle is stored in the
  workspace's \texttt{bundles} folder.
\item
  Navigate to the top right Configurations dropdown menu and select
  \emph{Edit Configurations}. From here, you can configure your server's
  run and debug configurations.

  \begin{figure}
  \centering
  \includegraphics{./images/intellij-server-dropdown.png}
  \caption{You have several options to choose from the server dropdown
  menu.}
  \end{figure}
\item
  Click the \emph{Add New Configuration} button
  (\includegraphics{./images/icon-intellij-add-config.png}) and select
  \emph{Liferay Server}.
\item
  Give your server a better name and modify any other configurations, if
  necessary. Then select \emph{OK} .

  \begin{figure}
  \centering
  \includegraphics{./images/intellij-run-debug-wizard.png}
  \caption{Set your Liferay server's configurations in the Run/Debug
  Configurations menu.}
  \end{figure}
\end{enumerate}

Your server is now available in IntelliJ! Make sure to select it in the
Configurations dropdown before executing the configuration buttons
(below).

For reference, here's how the IntelliJ configuration buttons work with
your Liferay DXP instance:

\begin{itemize}
\tightlist
\item
  \emph{Start}
  (\includegraphics{./images/icon-intellij-start-server.png}): Starts
  the server.
\item
  \emph{Stop}
  (\includegraphics{./images/icon-intellij-stop-server.png}): Stops the
  server.
\item
  \emph{Debug}
  (\includegraphics{./images/icon-intellij-debug-server.png}): Starts
  the server in debug mode. For more information on debugging in
  IntelliJ, see the
  \href{https://www.jetbrains.com/help/idea/debugging-code.html}{IntelliJ
  Debugging} article.
\end{itemize}

Now you're ready to use your server in IntelliJ!

\chapter{Updating Liferay IntelliJ
Plugin}\label{updating-liferay-intellij-plugin}

If you're already using the Liferay IntelliJ plugin but need to update
your environment, follow the steps below:

\begin{enumerate}
\def\labelenumi{\arabic{enumi}.}
\item
  In IntelliJ, navigate to \emph{File} → \emph{Settings} →
  \emph{Plugins}.
\item
  Click the \emph{Updates} tab. If the Liferay IntelliJ plugin is
  outdated, it will be listed as an available update.

  \begin{figure}
  \centering
  \includegraphics{./images/update-intellij-plugin.png}
  \caption{Check for updates periodically to ensure you're leveraging
  the latest features.}
  \end{figure}
\item
  Click the \emph{Update} button (if available) to update the Liferay
  IntelliJ plugin.
\item
  Once it's downloaded, click the \emph{Restart IDE} button.

  \begin{figure}
  \centering
  \includegraphics{./images/intellij-update-restart.png}
  \caption{The Available Updates prompt also prints the plugin version
  to which you're updating.}
  \end{figure}
\end{enumerate}

Once IntelliJ restarts, the Liferay IntelliJ plugin is updated and ready
for use.

\chapter{Liferay JS Generator}\label{liferay-js-generator}

The Liferay JS Generator generates JavaScript widgets using pure
JavaScript tooling. You don't have to have a deep understanding of Java
to write a widget for Liferay DXP. See the
\href{/docs/7-2/frameworks/-/knowledge_base/f/creating-and-bundling-javascript-widgets-with-javascript-tooling}{Liferay
JS Generator developer documentation} for more information on
configuring generated JavaScript widgets. This section covers these
reference topics for the Liferay JS Generator:

\begin{itemize}
\tightlist
\item
  How to install the Liferay JS Generator and use it to create a JS
  widget
\item
  An explanation of JS Portlet Extender's method signature
\item
  A reference list of the available configuration options for system
  settings and instance settings
\end{itemize}

\noindent\hrulefill

\textbf{Note:} The Liferay Bundle Generator is deprecated as of v2.7.1
of the \href{https://github.com/liferay/liferay-js-toolkit}{Liferay JS
Toolkit}. It has been renamed the Liferay JS Generator. If you're still
running the Liferay Bundle Generator, we recommend that you install the
Liferay JS Generator instead at your earliest convenience, as the
Liferay Bundle Generator will be removed in future versions.

\noindent\hrulefill

The available commands for bundles created with the Liferay JS Generator
are listed below:

\noindent\hrulefill

\begin{longtable}[]{@{}
  >{\raggedright\arraybackslash}p{(\columnwidth - 2\tabcolsep) * \real{0.5000}}
  >{\raggedright\arraybackslash}p{(\columnwidth - 2\tabcolsep) * \real{0.5000}}@{}}
\toprule\noalign{}
\begin{minipage}[b]{\linewidth}\raggedright
Command
\end{minipage} & \begin{minipage}[b]{\linewidth}\raggedright
Description
\end{minipage} \\
\midrule\noalign{}
\endhead
\bottomrule\noalign{}
\endlastfoot
\texttt{npm\ run\ build} & Places the output of liferay-npm-bundler in
the designated output folder. The standard output is a JAR file that can
be deployed manually to Liferay DXP. \\
\texttt{npm\ run\ deploy} & Deploys the application to the configured
server. \\
\texttt{npm\ run\ start} & Tests the application in a local webpack
installation instead of a Liferay DXP server. This speeds up development
because you can see live changes without the need to deploy. Note,
however, that because it's outside a Liferay instance, you don't have
access to Liferay's APIs. \\
\texttt{npm\ run\ translate} & Runs the translation features for your
bundle. Note that this feature requires Microsoft Translator
credentials. See
\href{/docs/7-2/frameworks/-/knowledge_base/f/using-translation-features-in-your-widget}{Using
Translation Features in Your Widget} for more information. \\
\end{longtable}

\noindent\hrulefill

\noindent\hrulefill

\textbf{Note:} By default, the webpack server uses port 8080, which
conflicts with the port used by Tomcat. You can point the webpack server
to a different port by setting the \texttt{port} key in
\texttt{.npmbuildrc}:

\begin{verbatim}
"webpack": {
  "port": 2070
}
\end{verbatim}

\noindent\hrulefill

Read this section to learn how to install the Liferay JS Generator and
understand its configuration.

\chapter{Installing the JS Generator and Generating a
Bundle}\label{installing-the-js-generator-and-generating-a-bundle}

{This document has been replaced by an article on Liferay Learn and is
no longer maintained here. The Liferay CLI tool is used for creating
JavaScript application projects for Liferay versions 7.1+.}

Here you'll learn how to install the
\href{https://www.npmjs.com/package/generator-liferay-bundle}{Liferay JS
Generator} and use it to create JavaScript widgets. See the
\href{/docs/7-2/appdev/-/knowledge_base/a/developing-an-angular-application}{Angular
Application},
\href{/docs/7-2/appdev/-/knowledge_base/a/developing-a-react-application}{React
Application}, and
\href{/docs/7-2/appdev/-/knowledge_base/a/developing-a-vue-application}{Vue
Application} articles to learn how to use your existing Angular, React,
and Vue apps in Liferay DXP.

\noindent\hrulefill

\textbf{Note:} To use the Liferay JS Generator, you must have the
Liferay JS Portlet Extender activated in your Liferay DXP instance. It's
activated by default. You can confirm this by opening the Control Menu,
navigating to the \emph{App Manager}, and searching for
\texttt{com.liferay.frontend.js.portlet.extender}.

\noindent\hrulefill

Follow these steps to create your JavaScript widget:

\begin{enumerate}
\def\labelenumi{\arabic{enumi}.}
\item
  Install \href{http://nodejs.org/}{Node.js}. Note that Node Package
  Manager (npm) is installed with this as well. You'll use npm to
  install the remaining dependencies and generator, and
  \href{/docs/7-2/reference/-/knowledge_base/r/setting-up-your-npm-environment}{configure
  your npm environment}.
\item
  Install \href{http://yeoman.io/}{Yeoman} for the generator:

\begin{verbatim}
npm install -g yeoman
\end{verbatim}
\item
  Install the Liferay JS Generator:

\begin{verbatim}
npm install -g generator-liferay-js
\end{verbatim}
\item
  Run the generator with the command below, select the JavaScript widget
  you want to create, and answer the prompts that follow.

\begin{verbatim}
yo liferay-js
\end{verbatim}

  \begin{figure}
  \centering
  \includegraphics{./images/liferay-js-generator-prompts.png}
  \caption{The liferay-js generator prompts you for widget options.}
  \end{figure}
\item
  If you specified your app server information when your widget was
  generated, you can deploy your widget by running the command below.
  You can verify this by checking the value of the \texttt{liferayDir}
  entry in the widget's \texttt{.npmbuildrc}.

\begin{verbatim}
npm run deploy
\end{verbatim}
\end{enumerate}

Great! Now you know how to install and run the Liferay JS Generator.

\chapter{Understanding the JS Portlet Extender
Configuration}\label{understanding-the-js-portlet-extender-configuration}

Bundles generated with the Liferay JS Generator require specific method
signatures, MANIFEST headers, and configuration within their
\texttt{package.json} file to use the JS Portlet Extender. This
configuration is provided by default.

\section{Manifest Header}\label{manifest-header}

The OSGi bundle contains the MANIFEST header shown below, which
specifies a dependency on the JS Portlet Extender:

\begin{verbatim}
Require-Capability: osgi.extender;filter:="(osgi.extender=liferay.npm.portlet)"
\end{verbatim}

\section{Main Entry Point}\label{main-entry-point}

The main module of your JavaScript widget must export a JavaScript
function with the signature below. Bundles created with the Liferay JS
Generator have this out-of-the-box:

\begin{verbatim}
function({portletNamespace, contextPath, portletElementId, configuration}) {
  ...
}
\end{verbatim}

The entry point function receives one object parameter with four fields:

\begin{itemize}
\item
  \texttt{portletNamespace}: the unique namespace of the widget as
  defined in the Portlet specification.
\item
  \texttt{contextPath}: the URL path that can be used to retrieve bundle
  resources from the browser (it doesn't contain the protocol, host, or
  port, just the absolute path).
\item
  \texttt{portletElementId}: the DOM identifier of the widget's
  \texttt{\textless{}div\textgreater{}} node that can be used to render
  HTML.
\item
  \texttt{configuration} (optional): since JS Portlet Extender version
  1.1.0, this field contains the system (OSGi) and portlet instance
  (preferences as described in the Portlet spec) configuration for the
  widget. It has two subfields:

  \begin{itemize}
  \item
    \texttt{system:} contains the system level configuration (defined in
    Control Panel → System Settings)
  \item
    \texttt{portletInstance:} contains the per-widget configuration
    (defined in the Configuration menu option of the widget)
  \end{itemize}
\end{itemize}

Note that all values are received as strings, no matter what their type
is in OSGi configuration store.

The JavaScript-based widget's main \texttt{index.js} file configuration
is shown below for reference. Note that system settings and localization
are enabled in the example below:

\begin{verbatim}
export default function main({portletNamespace, contextPath, portletElementId, configuration}) {
    
    const node = document.getElementById(portletElementId);

    node.innerHTML =`
        <div>
            <span class="tag">${Liferay.Language.get('porlet-namespace')}:</span>
            <span class="value">${portletNamespace}</span>
        </div>
        <div>
            <span class="tag">${Liferay.Language.get('context-path')}:</span>
            <span class="value">${contextPath}</span>
        </div>
        <div>
            <span class="tag">${Liferay.Language.get('portlet-element-id')}:</span>
            <span class="value">${portletElementId}</span>
        </div>
        
        <div>
            <span class="tag">${Liferay.Language.get('configuration')}:</span>
            <span class="value">
                ${JSON.stringify(configuration, null, 2)}
            </span>
        </div>
        
    `;
    
}
\end{verbatim}

The JavaScript file containing the main entry point function is
specified in the \texttt{main} entry of the \texttt{package.json} file.
Below is the \texttt{main} entry for the \emph{JavaScript based widget}:

\begin{verbatim}
"main": "index.js"
\end{verbatim}

\chapter{Configuration JSON Available
Options}\label{configuration-json-available-options}

If you've
\href{/docs/7-2/reference/-/knowledge_base/r/installing-the-js-generator-and-generating-a-bundle}{created
an OSGi bundle with the Liferay JS Generator} and want to provide system
settings or instance settings for your widget, you must provide a
\texttt{configuration.json} file. This reference guide lists the
available configuration options for \texttt{configuration.json} along
with example code.

\section{JSON Format}\label{json-format}

The \texttt{configuration.json} must follow the basic pattern shown
below:

\begin{verbatim}
{
  "system": {
    "category": "{category identifier}",
    "name": "{name of configuration}",
    "fields": {
      "{field id 1}": {
        "type": "{field type}",
        "name": "{field name}",
        "description": "{field description}",
        "default": "{default value}",
        "options": {
          "{option id 1}": "{option name 1}",
          "{option id 2}": "{option name 2}",

          "{option id n}": "{option name n}"
        }
      },
      "{field id 2}": {},

      "{field id n}": {}
    }
  },
  "portletInstance": {
    "name": "{name of configuration}",
    "fields": {
      "{field id 1}": {
        "type": "{field type}",
        "name": "{field name}",
        "description": "{field description}",
        "default": "{default value}",
        "options": {
          "{option id 1}": "{option name 1}",
          "{option id 2}": "{option name 2}",

          "{option id n}": "{option name n}"
        }
      },
      "{field id 2}": {},

      "{field id n}": {}
    }
  }
}
\end{verbatim}

The available options are described in the table below:

\noindent\hrulefill

\begin{longtable}[]{@{}
  >{\raggedright\arraybackslash}p{(\columnwidth - 2\tabcolsep) * \real{0.5000}}
  >{\raggedright\arraybackslash}p{(\columnwidth - 2\tabcolsep) * \real{0.5000}}@{}}
\toprule\noalign{}
\begin{minipage}[b]{\linewidth}\raggedright
Option
\end{minipage} & \begin{minipage}[b]{\linewidth}\raggedright
Value
\end{minipage} \\
\midrule\noalign{}
\endhead
\bottomrule\noalign{}
\endlastfoot
\texttt{\{category\ identifier\}} & Describes the identifier of the
configuration category where the settings must be placed. It's
equivalent to the category field of the
\texttt{@ExtendedObjectClassDefinition} annotation explained
\href{/docs/7-2/frameworks/-/knowledge_base/f/categorizing-the-configuration}{here}.
The category field of \texttt{configuration.json} is optional and, when
not set, the project's name specified in \texttt{package.json} is used.
You need JS Portlet Extender 1.1.0+ for this feature to work. Otherwise,
the system configuration will show up under \emph{Platform} →
\emph{Third Party} in System Settings. \\
\texttt{\{name\ of\ configuration\}} & the configuration's name as a
string or a localization key. If no value is given, the bundler falls
back to the project's name, then description given in
\texttt{package.json}. \\
\texttt{\{field\ id\}} & the field's name as a string or a localization
key \\
\texttt{\{field\ type\}} & specifies the field's type, which can be one
of the following types: ~- \texttt{number}: an integer number~-
\texttt{float}: a floating point number~- \texttt{string}: a string~-
\texttt{boolean}: true or false~- \texttt{password}: a password
(string) \\
\texttt{\{field\ name\}} & the field's name as a string or a
localization key \\
\texttt{\{field\ description\}} & an optional string or a localization
key that describes the field's purpose and appears as hint text near
it \\
\texttt{\{default\ value\}} & an optional default value for the field \\
\texttt{options} & an optional section that defines a fixed set of
values for the field \\
\texttt{\{option\ id\}} & a string that defines the option's ID \\
\texttt{\{option\ name\}} & the option's name as a string or a
localization key \\
\end{longtable}

\noindent\hrulefill

An example configuration is shown below:

\begin{verbatim}
{
  "system": {
    "category": "third-party",
    "name": "My project",
    "fields": {
      "a-number": {
        "type": "number",
        "name": "A number",
        "description": "An integer number",
        "default": "42"
      },
      "a-string": {
        "type": "string",
        "name": "A string",
        "description": "An arbitrary length string",
        "default": "this is a string"
      },
      "a-password": {
        "type": "password",
        "name": "A password",
        "description": "A secret string",
        "default": "s3.cr3t"
      },
      "a-boolean": {
        "type": "boolean",
        "name": "A boolean",
        "description": "A true|false value",
        "default": true
      },
      "an-option": {
        "type": "string",
        "name": "An option",
        "description": "A restricted values option",
        "required": true,
        "default": "A",
        "options": {
          "A": "Option a",
          "B": "Option b"
        }
      }
    }
  },
  "portletInstance": {
    "name": "Widget configuration",
    "fields": {
      "a-float": {
        "type": "float",
        "name": "A float",
        "description": "A floating point number",
        "default": "1.1"
      }
    }
  }
}
\end{verbatim}

\chapter{Adapting Existing Apps to Run on Liferay
DXP}\label{adapting-existing-apps-to-run-on-liferay-dxp}

There are two ways to get your existing front-end applications running
on Liferay DXP:

\begin{enumerate}
\def\labelenumi{\arabic{enumi}.}
\item
  \href{/docs/7-2/appdev/-/knowledge_base/a/web-front-ends}{Migrate your
  project} to a Liferay JS Toolkit project.
\item
  Since v2.15.0 of the Liferay JS Toolkit, create projects normally, as
  you would with
  \href{https://facebook.github.io/create-react-app/}{create-react-app},
  \href{https://cli.angular.io/}{Angular CLI} (any project containing
  \texttt{@angular/cli} as a dependency or devDependency), and
  \href{https://cli.vuejs.org/}{Vue CLI} (any project containing
  \texttt{@vue/cli-service} as a dependency or devDependency), and adapt
  them to run on Liferay DXP.
\end{enumerate}

Only adapt your project if you intend it to be platform-agnostic. If you
want to integrate with Liferay DXP fully and have access to all the
\href{/docs/7-2/frameworks/-/knowledge_base/f/creating-and-bundling-javascript-widgets-with-javascript-tooling}{features
and benefits} that it provides,
\href{/docs/7-2/appdev/-/knowledge_base/a/web-front-ends}{migrate your
project} to a true Liferay JS Toolkit project instead.

The reason for this is some of Liferay DXP's features may not be
available because the native frameworks expect certain things. For
example, Angular assumes that it controls a whole Single Page
Application as opposed to the small portion of the page that it controls
in a portlet-based platform such as Liferay DXP. Since webpack bundles
all JavaScript in a single file to consume per app, if there are five
widgets on a page, you have five copies of the framework in the
JavaScript interpreter. To prevent this,
\href{/docs/7-2/appdev/-/knowledge_base/a/web-front-ends}{migrate your
project} to a true Liferay JS Toolkit project instead.

To adapt your project, it must have the structure shown below:

\begin{itemize}
\item
  \textbf{Angular CLI projects} must use \texttt{app-root} as the
  application's Dom selector.
\item
  \textbf{creact-react-app projects} must use \texttt{ReactDom.render()}
  call in your entry point with a \texttt{document.getElementById()}
  parameter.
\item
  \textbf{Vue CLI projects} must use \texttt{\#app} as the application's
  DOM selector.
\end{itemize}

When your project meets the requirements, you can follow these steps to
use the Liferay JS Generator to adapt it:

\begin{enumerate}
\def\labelenumi{\arabic{enumi}.}
\item
  Open the command line and navigate to your project's folder.
\item
  Run the Liferay JS Generator's \texttt{adapt} subtarget:

\begin{verbatim}
yo liferay-js:adapt
\end{verbatim}

  \begin{figure}
  \centering
  \includegraphics{./images/liferay-js-generator-adapt-run.png}
  \caption{You can run the adapt subtarget of the Liferay JS Generator
  to adapt your existing apps for Liferay.}
  \end{figure}
\item
  Answer the prompts. An example configuration appears below:

\begin{verbatim}
? Under which category should your widget be listed? category.sample
? Do you have a local installation of Liferay for development? Yes
? Where is your local installation of Liferay placed? /home/user/liferay
\end{verbatim}

  Your project is adapted to use the Liferay JS Toolkit and run on
  Liferay DXP!

  \begin{figure}
  \centering
  \includegraphics{./images/liferay-js-generator-adapt-complete.png}
  \caption{You can run the adapt subtarget of the Liferay JS Generator
  to adapt your existing apps for Liferay.}
  \end{figure}
\item
  The adapt process automatically adds a few npm scripts to the
  project's \texttt{package.json} so you can build and deploy your
  project to Liferay DXP. Note that you can swap \texttt{npm} for
  \texttt{yarn} below if you prefer to use yarn instead.

  Run the command below to add a deployable JAR to the
  \texttt{build.liferay} folder in your project. For example, if you
  want to build the JAR and copy it to another app server, you can run
  this command:

\begin{verbatim}
npm run build:liferay
\end{verbatim}

  Run the command below to deploy the adapted app to your Liferay DXP
  instance:

\begin{verbatim}
npm run deploy:liferay
\end{verbatim}
\end{enumerate}

Great! Now you know how to use the Liferay JS Generator to adapt your
existing apps to run on Liferay DXP. See the
\href{https://github.com/liferay/liferay-docs/tree/master/en/developer/reference/code/adapted-react-app/}{React
Guestbook App} for a working example of an adapted app.

\begin{figure}
\centering
\includegraphics{./images/liferay-js-generator-adapt-deployed.png}
\caption{Your adapted app runs in Liferay in no time.}
\end{figure}

\chapter{Liferay Workspace}\label{liferay-workspace}

{This document has been updated and ported to Liferay Learn and is no
longer maintained here.}

A \emph{Liferay Workspace} is a generated environment that is built to
hold and manage your Liferay projects. This workspace is intended to aid
in the management of Liferay projects by providing various build scripts
and configured properties. You can download the
\href{https://sourceforge.net/projects/lportal/files/Liferay\%20IDE/}{Liferay
Project SDK installer} and run it to install
\href{/docs/7-2/reference/-/knowledge_base/r/blade-cli}{Blade CLI}
(default CLI for workspace), initialize a new Liferay Workspace, and
download Dev Studio DXP.

Liferay Workspace is the official way to create/manage 7.0 projects
using Gradle. Do you prefer Maven over Gradle? You can also generate a
Maven-based workspace.

You'll cover the following topics in this section:

\begin{itemize}
\tightlist
\item
  \href{/docs/7-2/reference/-/knowledge_base/r/installing-liferay-workspace}{Installing
  Liferay Workspace}
\item
  \href{/docs/7-2/reference/-/knowledge_base/r/creating-a-liferay-workspace}{Creating
  a Liferay Workspace}
\item
  \href{/docs/7-2/reference/-/knowledge_base/r/importing-a-liferay-workspace-into-an-ide}{Importing
  a Liferay Workspace}
\item
  \href{/docs/7-2/reference/-/knowledge_base/r/setting-proxy-requirements-for-liferay-workspace}{Setting
  Proxy Requirements}
\item
  \href{/docs/7-2/reference/-/knowledge_base/r/adding-a-liferay-bundle-to-liferay-workspace}{Adding
  a Bundle}
\item
  \href{/docs/7-2/reference/-/knowledge_base/r/setting-environment-configurations-for-liferay-workspace}{Setting
  Environment Configurations}
\item
  \href{/docs/7-2/reference/-/knowledge_base/r/building-node-js-themes-in-liferay-workspace}{Building
  Node.js Themes}
\item
  \href{/docs/7-2/reference/-/knowledge_base/r/building-gradle-maven-themes-in-liferay-workspace}{Building
  Gradle/Maven Themes}
\item
  \href{/docs/7-2/reference/-/knowledge_base/r/managing-the-target-platform}{Managing
  the Target Platform}
\item
  \href{/docs/7-2/reference/-/knowledge_base/r/validating-modules-against-the-target-platform}{Validating
  Modules Against the Target Platform}
\item
  \href{/docs/7-2/reference/-/knowledge_base/r/leveraging-docker}{Leveraging
  Docker}
\item
  \href{/docs/7-2/reference/-/knowledge_base/r/updating-liferay-workspace}{Updating
  Liferay Workspace}
\item
  \href{/docs/7-2/reference/-/knowledge_base/r/updating-default-plugins-provided-by-liferay-workspace}{Updating
  Default Plugins Provided by Liferay Workspace}
\end{itemize}

Liferay Workspaces can be used in many different development
environments, which makes it flexible and applicable to many different
developers. For example, a Liferay Workspace easily integrates with
Eclipse and IntelliJ, providing a seamless development experience. See
how to
\href{/docs/7-2/reference/-/knowledge_base/r/installing-liferay-workspace}{install}
and
\href{/docs/7-2/reference/-/knowledge_base/r/creating-a-liferay-workspace}{create}
a Liferay Workspace for more information.

You'll learn about workspace's anatomy and development lifecycle next.

\section{Workspace Anatomy}\label{workspace-anatomy}

A Liferay Workspace offers a development environment that can be
configured to fit your development needs. Properties are available to
help manage default and optional folders. This provides you the power to
customize your workspace's folder structure any way you'd like. The
top-level files/folder of a Liferay (Gradle) Workspace are outlined
below:

\begin{itemize}
\tightlist
\item
  \texttt{bundles} (generated): the default folder for Liferay DXP
  bundles.
\item
  \texttt{configs}: holds the configuration files for different
  environments. These files serve as your global configuration files for
  all Liferay DXP servers and projects residing in your workspace. To
  learn more about using the \texttt{configs} folder, see the
  \hyperref[testing-projects]{Testing Projects} section.
\item
  \texttt{ext} (generated): holds the Ext OSGi modules and Ext plugins.
\item
  \texttt{gradle}: holds the Gradle Wrapper used by your workspace.
\item
  \texttt{modules}: holds your custom modules. This can also hold
  front-end portlets created with the
  \href{/docs/7-2/reference/-/knowledge_base/r/js-generator}{Liferay JS
  Toolkit}.
\item
  \texttt{themes}: holds Node.js-style themes that use the Liferay JS
  Theme Toolkit, which are built using the Liferay Theme Generator.
\item
  \texttt{wars}: holds traditional WAR-style web application projects
  and theme projects (i.e., generated by the
  \href{/docs/7-2/reference/-/knowledge_base/r/theme-template}{\texttt{theme}}
  project template).
\item
  \texttt{build.gradle}: the common Gradle build file.
\item
  \texttt{gradle.properties}: specifies the workspace's project
  locations and Liferay DXP server configuration globally.
\item
  \texttt{gradle-local.properties}: sets user-specific properties for
  your workspace. This lets multiple users use a single workspace,
  letting them configure specific properties for the workspace on their
  own machine.
\item
  \texttt{gradlew}: executes the Gradle command wrapper.
\item
  \texttt{settings.gradle}: applies plugins to the workspace and
  configures its dependencies.
\end{itemize}

If you're using a workspace generated for Maven projects, your folder
hierarchy is the same, except the Gradle build files are swapped out for
a \texttt{pom.xml} file.

Visit your workspace's \texttt{gradle.properties} file for a list of
properties (with descriptions) you can define to adapt your workspace.
For a Maven-based workspace, see the
\href{/docs/7-2/reference/-/knowledge_base/r/bundle-support-plugin}{Bundle
Support Plugin} article for info on adapting your Maven workspace.

If you'd like to keep the global Gradle properties the same, but want to
change them for yourself only (perhaps for local testing), you can
override the \texttt{gradle.properties} file with your own
\texttt{gradle-local.properties} file.

Next, you'll learn about workspace's development lifecycle.

\section{Development Lifecycle}\label{development-lifecycle}

Liferay Workspaces offer a full development lifecycle for your projects
to make your Liferay development easier than ever. The development
lifecycle includes

\begin{itemize}
\item
  \hyperref[creating-projects]{Creating projects}
\item
  \hyperref[building-projects]{Building projects}
\item
  \hyperref[deploying-projects]{Deploying projects}
\item
  \hyperref[testing-projects]{Testing projects}
\item
  \hyperref[releasing-projects]{Releasing projects}
\item
  \hyperref[development-lifecycle]{Test}
\end{itemize}

You'll learn about each lifecycle option next.

\section{Creating Projects}\label{creating-projects}

Workspace provides a slew of
\href{/docs/7-2/reference/-/knowledge_base/r/project-templates}{project
templates} that you can use to create many different types of Liferay
projects. Workspace also provides development support for front-end
portlets generated with the
\href{/docs/7-2/reference/-/knowledge_base/r/js-generator}{Liferay JS
Toolkit}. They're stored in the \texttt{modules} folder by default.

You can also configure where to generate certain projects (modules,
themes, WARs, etc.). These settings are documented in the
\texttt{gradle.properties} file. See the
\href{/docs/7-2/reference/-/knowledge_base/r/creating-a-project}{Creating
a Project} article for more information.

Liferay Workspace manages theme projects in two separate folders based
on how they're created:

\begin{itemize}
\tightlist
\item
  \href{/docs/7-2/reference/-/knowledge_base/r/theme-generator}{Liferay
  Theme Generator} (Node.js-based themes that use the Liferay JS Theme
  Toolkit)
\item
  \href{/docs/7-2/reference/-/knowledge_base/r/theme-template}{Project
  template/archetype} (Gradle/Maven-based themes)
\end{itemize}

Liferay Workspace offers an environment where developers can use the
Liferay Theme Generator to create themes and their work can be
seamlessly integrated into their overall DevOps strategy. You can
leverage the Liferay Theme Generator to
\href{/docs/7-2/reference/-/knowledge_base/r/building-node-js-themes-in-liferay-workspace}{create
Node.js-based themes inside workspace} or you can leverage it externally
and copy themes into Workspace.

Workspace also offers a
\href{/docs/7-2/reference/-/knowledge_base/r/building-gradle-maven-themes-in-liferay-workspace}{traditional
Java-based theme approach} (leveraging Gradle/Maven) for those that
can't use the Liferay JS Theme Toolkit's tools in their CI environment.

\section{Building Projects}\label{building-projects}

Liferay Workspace abstracts many build requirements away so you can
focus on developing projects instead of worrying about how to build
them. This is done by incorporating a slew of plugins under the hood to
allow for easily accessible tooling. See the
\href{/docs/7-2/reference/-/knowledge_base/r/gradle-plugins}{Gradle
Plugins} and
\href{/docs/7-2/reference/-/knowledge_base/r/maven-plugins}{Maven
Plugins} sections for information on some of the plugins provided by
workspace.

Gradle-based workspaces also include a Gradle wrapper in its ROOT folder
(e.g., \texttt{gradlew}), which you can leverage to execute Gradle
commands. This means that you can run familiar Gradle build commands
(e.g., \texttt{build}, \texttt{clean}, \texttt{compile}, etc.) from a
Liferay Workspace without having Gradle installed on your machine. For
Maven-based workspaces, Maven build commands are supported (e.g.,
\texttt{package}, \texttt{verify}, \texttt{deploy}, etc.).

Liferay Workspace lets you build your projects out-of-the-box without
the hassle of manual build configurations.

\section{Deploying Projects}\label{deploying-projects}

Liferay Workspace provides easy-to-use deployment mechanisms that let
you deploy your project to a Liferay server without any custom
configuration. To learn more about deploying projects from a workspace,
visit the
\href{/docs/7-2/reference/-/knowledge_base/r/deploying-a-project}{Deploying
a Project} article.

\section{Testing Projects}\label{testing-projects}

Liferay provides many configuration settings for 7.0. Configuring
several different Liferay DXP installations to simulate/test certain
behaviors can become cumbersome and time consuming. With Liferay
Workspace, you can easily organize environment settings and generate an
environment installation with those settings.

Liferay Workspace provides the \texttt{configs} folder, which lets you
configure different environments in the same workspace. For example, you
could configure separate Liferay DXP environment settings for
development, testing, and production in a single Liferay Workspace. So
how does it work?

The \texttt{configs} folder offers six subfolders:

\texttt{common}: holds a common configuration that you want applied to
all environments.

\texttt{dev}: holds the development configuration.

\texttt{docker}: holds the configuration for a Docker container.

\texttt{local}: holds the configuration intended for testing locally.

\texttt{prod}: holds the configuration for a production site.

\texttt{uat}: holds the configuration for a UAT site.

You're not limited to just these environments. You can create any
subfolder in the \texttt{configs} folder (e.g., \texttt{aws},
\texttt{test}, etc.) to simulate any environment. Each environment
folder can supply its own database, \texttt{portal-ext.properties},
Elasticsearch, etc. The files in each folder overlay your Liferay DXP
installation, which you generate from within workspace.

\begin{figure}
\centering
\includegraphics{./images/workspace-configs.png}
\caption{The \texttt{configs/common} and
\texttt{configs/{[}environment{]}} overlay you Liferay DXP bundle when
it's generated.}
\end{figure}

When workspace generates a Liferay DXP bundle, these things happen:

\begin{enumerate}
\def\labelenumi{\arabic{enumi}.}
\item
  Configuration files found in the \texttt{configs/common} folder are
  applied to the Liferay DXP bundle.
\item
  The configured workspace environment (\texttt{dev}, \texttt{local},
  etc.) is applied on top of any existing configurations from the
  \texttt{common} folder.
\end{enumerate}

See the
\href{/docs/7-2/reference/-/knowledge_base/r/setting-environment-configurations-for-liferay-workspace}{Setting
Environment Configurations for Liferay Workspace} article for more
information.

\section{Releasing Projects}\label{releasing-projects}

Liferay Workspace does not provide a built-in release mechanism, but
there are easy ways to use external release tools with workspace. The
most popular choice is uploading your projects to a Maven Nexus
repository. You could also use other release tools like
\href{https://www.jfrog.com/artifactory/}{Artifactory}.

Uploading projects to a remote repository is useful if you need to share
them with other non-workspace projects. Also, if you're ready for your
projects to be in the spotlight, uploading them to a public remote
repository gives other developers the chance to use them.

For more instructions on how to set up a Maven Nexus repository for your
workspace's projects, see the
\href{/docs/7-2/reference/-/knowledge_base/r/creating-a-maven-repository}{Creating
a Maven Repository} and
\href{/docs/7-2/reference/-/knowledge_base/r/deploying-liferay-maven-artifacts-to-a-repository}{Deploying
Liferay Maven Artifacts to a Repository} articles.

\chapter{Installing Liferay
Workspace}\label{installing-liferay-workspace}

{This document has been updated and ported to Liferay Learn and is no
longer maintained here.}

You can install Liferay Workspace using the Liferay Project SDK
installer. This installs JPM and
\href{/docs/7-2/reference/-/knowledge_base/r/blade-cli}{Blade CLI} into
your user home folder and optionally initializes a Liferay Workspace
folder. This is the same installer covered in the
\href{/docs/7-2/reference/-/knowledge_base/r/installing-blade-cli}{Installing
Blade CLI} article.

Follow the steps below to download and install Liferay Workspace:

\begin{enumerate}
\def\labelenumi{\arabic{enumi}.}
\item
  Download the latest
  \href{https://sourceforge.net/projects/lportal/files/Liferay\%20IDE/}{Liferay
  Project SDK installer} that corresponds with your operating system
  (e.g., Windows, MacOS, or Linux). The Project SDK installer is listed
  under \emph{Liferay IDE}, so the folder versions are based on IDE
  releases. You can select an installer that does not include Dev Studio
  DXP, if you don't intend to use it. The Project SDK installer is
  available for versions 3.2.0+. Do \textbf{not} select the large green
  download button; this downloads Liferay Portal instead.
\item
  Run the installer. Click \emph{Next} to step through the installer's
  introduction.
\item
  Set the folder where your Liferay Workspace should be initialized.

  \begin{figure}
  \centering
  \includegraphics{./images/blade-installer-workspace-init.png}
  \caption{Determine where your Liferay Workspace should reside.}
  \end{figure}

  Then click \emph{Next}.
\item
  Choose the Liferay product type you intend to use with the workspace.
  Then click \emph{Next}.

  \begin{figure}
  \centering
  \includegraphics{./images/installer-workspace-type.png}
  \caption{Select the product version you'll use with your Liferay
  Workspace.}
  \end{figure}
\end{enumerate}

\noindent\hrulefill

\begin{verbatim}
 **Note:** You'll be prompted for your liferay.com username and password
 before downloading the Liferay DXP bundle. Your credentials are not saved
 locally; they're saved as a token in the `~/.liferay` folder. The token is
 used by your workspace if you ever decide to redownload a DXP bundle.
 Furthermore, the bundle that is downloaded in your workspace is also
 copied to your `~/.liferay/bundles` folder, so if you decide to initialize
 another Liferay DXP instance of the same version, the bundle is not
 re-downloaded. See the
 [Adding a Liferay Bundle to Liferay Workspace](/docs/7-2/reference/-/knowledge_base/r/adding-a-liferay-bundle-to-liferay-workspace)
 for more information on this topic.
\end{verbatim}

\noindent\hrulefill

\begin{enumerate}
\def\labelenumi{\arabic{enumi}.}
\setcounter{enumi}{4}
\tightlist
\item
  Click \emph{Next} to begin installing Liferay Workspace on your
  machine.
\end{enumerate}

That's it! Liferay Workspace is now installed on your machine!

\chapter{Creating a Liferay
Workspace}\label{creating-a-liferay-workspace}

{This document has been updated and ported to Liferay Learn and is no
longer maintained here.}

You can create a Liferay Workspace using the following tools:

\begin{itemize}
\tightlist
\item
  \hyperref[blade-cli]{Blade CLI}
\item
  \hyperref[dev-studio]{Dev Studio}
\item
  \hyperref[intellij]{IntelliJ}
\item
  \hyperref[maven]{Maven}
\end{itemize}

Visit the appropriate section to learn how to create a workspace with
the highlighted tool.

\section{Blade CLI}\label{blade-cli-3}

\begin{enumerate}
\def\labelenumi{\arabic{enumi}.}
\item
  Navigate to the folder where you want your workspace generated.
\item
  Run the following command to build a Gradle-based workspace:

\begin{verbatim}
blade init -v 7.2 [WORKSPACE_NAME]
\end{verbatim}
\end{enumerate}

\noindent\hrulefill

\begin{verbatim}
 **Note**: The version you set when first initializing your workspace is
 stored in the workspace's `.blade.properties` file with the
 `liferay.version.default` property. This version is applied when creating
 projects based on the corresponding project template versions.

 If you wish to develop projects for a different Liferay DXP version, you can
 pass a different version in the Blade init command. For example,

 ```bash
 blade init -v 7.0 [WORKSPACE_NAME]
 ```
\end{verbatim}

\noindent\hrulefill

You can also create a Maven-based workspace with Blade CLI. See the
\hyperref[maven]{Maven} section for more information.

\section{Dev Studio}\label{dev-studio}

\begin{enumerate}
\def\labelenumi{\arabic{enumi}.}
\item
  Select \emph{File} → \emph{New} → \emph{Liferay Workspace Project}.

  \begin{figure}
  \centering
  \includegraphics{./images/selecting-liferay-workspace.png}
  \caption{By selecting \emph{Liferay Workspace Project}, you begin the
  process of creating a new workspace for your Liferay projects.}
  \end{figure}

  A New Liferay Workspace dialog appears, presenting several
  configuration options.
\item
  Give your workspace project a name.
\item
  Choose the location where you'd like your workspace to reside.
  Checking the \emph{Use default location} checkbox places your Liferay
  Workspace in the Eclipse workspace you're working in.
\item
  Select the build tool you want your workspace to be built with (i.e.,
  Gradle or Maven).
\item
  Choose the Liferay Portal version you plan to develop for (i.e., 7.2,
  7.1, or 7.0).
\item
  Select the specific target platform version corresponding to the GA
  release you're developing for (e.g., 7.2.0 → 7.2 GA1). For more
  information on target platform benefits, see the
  \href{/docs/7-2/reference/-/knowledge_base/r/managing-the-target-platform}{Managing
  the Target Platform} articles.
\item
  Check the \emph{Download Liferay bundle} checkbox if you'd like to
  auto-generate a Liferay instance in your workspace. You'll be prompted
  to name the server and provide the server's download URL, if selected.
\end{enumerate}

\noindent\hrulefill

\begin{verbatim}
 **Note:** You can configure a pre-existing Liferay bundle in your
 workspace by creating a folder for the bundle in your workspace and
 configuring it in the workspace's `gradle.properties` file by setting the
 `liferay.workspace.home.dir` property.
\end{verbatim}

\noindent\hrulefill

\begin{enumerate}
\def\labelenumi{\arabic{enumi}.}
\setcounter{enumi}{7}
\item
  Check the \emph{Add project to working set} checkbox if you want your
  workspace to be a part of a larger working set you've already created
  in Dev Studio. For more information on working sets, visit
  \href{https://help.eclipse.org/mars/index.jsp?topic=\%2Forg.eclipse.platform.doc.user\%2Fconcepts\%2Fcworkset.htm}{Eclipse
  Help}.
\item
  Click \emph{Finish} to create your Liferay Workspace.

  \begin{figure}
  \centering
  \includegraphics{./images/new-workspace-menu.png}
  \caption{Dev Studio provides an easy-to-follow menu to create your
  Liferay Workspace.}
  \end{figure}
\end{enumerate}

A dialog appears prompting you to open the Liferay Workspace
perspective. Click \emph{Yes}, and your perspective will switch to
Liferay Workspace.

\section{IntelliJ}\label{intellij-1}

\begin{enumerate}
\def\labelenumi{\arabic{enumi}.}
\item
  Open the New Project wizard by selecting \emph{File} → \emph{New} →
  \emph{Project}. If you're starting IntelliJ for the first time, you
  can do this by selecting \emph{Create New Project} in the opening
  window.
\item
  Select \emph{Liferay} from the left menu.
\item
  Choose the build type for your workspace (i.e., Gradle or Maven). Then
  click \emph{Next}.

  \begin{figure}
  \centering
  \includegraphics{./images/intellij-workspace-build.png}
  \caption{Choose \emph{Liferay Gradle Workspace} or \emph{Liferay Maven
  Workspace}, depending on the build you prefer.}
  \end{figure}
\item
  Specify your workspace's name, location, intended Liferay DXP version,
  \href{/docs/7-2/reference/-/knowledge_base/r/managing-the-target-platform}{target
  platform}, and SDK (i.e., Java JDK). Then click \emph{Finish}.

  \begin{figure}
  \centering
  \includegraphics{./images/intellij-workspace-project.png}
  \caption{Specify your workspace's configurations.}
  \end{figure}
\item
  A window opens for additional build configurations for the build type
  you selected (i.e., Gradle or Maven). Verify the settings and click
  \emph{OK}.
\end{enumerate}

\section{Maven}\label{maven-2}

\begin{enumerate}
\def\labelenumi{\arabic{enumi}.}
\item
  Execute the following Maven command:

\begin{verbatim}
mvn archetype:generate -Dfilter=liferay
\end{verbatim}
\item
  Select the \texttt{com.liferay.project.templates.workspace} archetype
  to generate.
\item
  Step through the remaining prompts to generate the workspace project.
\end{enumerate}

A Maven-based Liferay Workspace can also be generated using Blade CLI.
Follow \hyperref[blade-cli]{Blade CLI's} workspace creation instructions
and insert the \texttt{-b\ \ maven} parameter in the Blade command.

\chapter{Importing a Liferay Workspace into an
IDE}\label{importing-a-liferay-workspace-into-an-ide}

{This document has been updated and ported to Liferay Learn and is no
longer maintained here.}

Liferay supports two IDEs with preconfigured Liferay Workspace wizards
and functionalities

\begin{itemize}
\tightlist
\item
  \hyperref[dev-studio]{Dev Studio}
\item
  \hyperref[intellij]{IntelliJ}
\end{itemize}

These aren't the only IDEs you can leverage to use Liferay Workspace,
but they are the ones with out-of-the-box support for it.

Visit the appropriate section to learn how to import a workspace with
the highlighted tool.

\section{Dev Studio}\label{dev-studio-1}

\begin{enumerate}
\def\labelenumi{\arabic{enumi}.}
\item
  Navigate to \emph{File} → \emph{Import} → \emph{Liferay} →
  \emph{Liferay Workspace Project}.
\item
  Click \emph{Next} and browse for your workspace project.

  \begin{figure}
  \centering
  \includegraphics{./images/liferay-workspace-import.png}
  \caption{You can import an existing Liferay Workspace into your
  current Dev Studio session.}
  \end{figure}
\item
  Once you've selected you workspace, click \emph{Finish}.
\end{enumerate}

\section{IntelliJ}\label{intellij-2}

\begin{enumerate}
\def\labelenumi{\arabic{enumi}.}
\item
  Select \emph{File} → \emph{New} → \emph{Project from Existing
  Sources\ldots{}}.
\item
  Select the workspace you want to import. Then click \emph{OK}.

  \begin{figure}
  \centering
  \includegraphics{./images/intellij-import-workspace.png}
  \caption{Specify your workspace's configurations.}
  \end{figure}
\item
  Click the \emph{Import project from external model} radio button and
  select the build tool your workspace uses (e.g., Gradle or Maven).
\item
  Configure the project import (if necessary) and then click
  \emph{Finish}. See the
  \href{https://www.jetbrains.com/help/idea/creating-and-managing-projects.html\#importing-project}{Import
  a Project} section of IntelliJ's official documentation for more
  information.
\item
  Step through the remaining import prompts and then open your imported
  workspace as you desire (i.e., in the current window or a new window).
\end{enumerate}

\chapter{Setting Proxy Requirements for Liferay
Workspace}\label{setting-proxy-requirements-for-liferay-workspace}

{This document has been updated and ported to Liferay Learn and is no
longer maintained here.}

If you're working behind a corporate firewall that requires using a
proxy server to access external repositories, you need to add some extra
configuration to make Liferay Workspace work within your environment.
You'll learn how to set proxy requirements for both Gradle and Maven
environments.

\section{Gradle}\label{gradle-1}

\begin{enumerate}
\def\labelenumi{\arabic{enumi}.}
\item
  Open your \texttt{\textasciitilde{}/.gradle/gradle.properties} file.
  Create this file if it does not exist.
\item
  Add the following properties to the file:

\begin{verbatim}
systemProp.http.proxyHost=www.somehost.com
systemProp.http.proxyPort=1080
systemProp.https.proxyHost=www.somehost.com
systemProp.https.proxyPort=1080
\end{verbatim}

  Make sure to replace the proxy host and port values with your own.
\item
  If the proxy server requires authentication, also add the following
  properties:

\begin{verbatim}
systemProp.http.proxyUser=userId
systemProp.http.proxyPassword=yourPassword
systemProp.https.proxyUser=userId
systemProp.https.proxyPassword=yourPassword
\end{verbatim}
\end{enumerate}

Excellent! Your proxy settings are set in your Liferay Workspace's
Gradle environment.

\section{Maven}\label{maven-3}

\begin{enumerate}
\def\labelenumi{\arabic{enumi}.}
\item
  Open your \texttt{\textasciitilde{}/.m2/settings.xml} file. Create
  this file if it does not exist.
\item
  Add the following XML snippet to the file:

\begin{verbatim}
<?xml version="1.0" encoding="UTF-8"?>
    <settings xmlns="http://maven.apache.org/SETTINGS/1.0.0"
        xmlns:xsi="http://www.w3.org/2001/XMLSchema-instance"
        xsi:schemaLocation="http://maven.apache.org/SETTINGS/1.0.0 http://maven.apache.org/xsd/settings-1.0.0.xsd">
        <proxies>
            <proxy>
                <id>httpProxy</id>
                <active>true</active>
                <protocol>http</protocol>
                <host>www.somehost.com</host>
                <port>1080</port>
            </proxy>
            <proxy>
                <id>httpsProxy</id>
                <active>true</active>
                <protocol>https</protocol>
                <host>www.somehost.com</host>
                <port>1080</port>
            </proxy>
        </proxies>
    </settings>
\end{verbatim}

  Make sure to replace the proxy host and port values with your own.
\item
  If the proxy server requires authentication, also add the
  \texttt{username} and \texttt{password} proxy properties. For example,
  the HTTP proxy authentication configuration would look like this:

\begin{verbatim}
<proxy>
  <id>httpProxy</id>
  <active>true</active>
  <protocol>http</protocol>
  <host>www.somehost.com</host>
  <port>1080</port>
  <username>userID</username>
  <password>somePassword</password>
</proxy>
\end{verbatim}
\end{enumerate}

Excellent! Your Maven proxy settings are now set.

\chapter{Adding a Liferay Bundle to Liferay
Workspace}\label{adding-a-liferay-bundle-to-liferay-workspace}

{This document has been updated and ported to Liferay Learn and is no
longer maintained here.}

Liferay Workspaces can generate and hold a Liferay Server. This lets you
build/test your workspace's plugins against a running Liferay instance.
Follow the instructions below to get started.

\begin{enumerate}
\def\labelenumi{\arabic{enumi}.}
\item
  Open your workspace's root \texttt{gradle.properties} file.
\item
  Set the \texttt{liferay.workspace.bundle.url} property to the bundle's
  download URL you want to generate and install. For example,

\begin{verbatim}
liferay.workspace.bundle.url=https://releases-cdn.liferay.com/portal/7.2.0-ga1/liferay-ce-portal-tomcat-7.2.0-ga1-20190531153709761.7z
\end{verbatim}

  For DXP subscribers, it would look like this:

\begin{verbatim}
liferay.workspace.bundle.url=https://api.liferay.com/downloads/portal/7.2.10/liferay-dxp-tomcat-7.2.10-ga1-20190531140450482.7z
\end{verbatim}
\end{enumerate}

\noindent\hrulefill

\begin{verbatim}
 **Note:** The DXP download URL must be set to the bundle hosted on
 *api.liferay.com*. It can be tricky to find the fully qualified bundle
 name/number for the DXP bundle you want. You cannot access Liferay's API
 site directly to find it, so you must start to download DXP manually from
 Liferay's Customer Portal, take note of the file name, and append it to
 `https://api.liferay.com/downloads/portal/`.
\end{verbatim}

\noindent\hrulefill

\begin{verbatim}
DXP subscribers must also set the `liferay.workspace.bundle.token.download`
property to `true` to allow your workspace to access Liferay's API site.
\end{verbatim}

\begin{enumerate}
\def\labelenumi{\arabic{enumi}.}
\setcounter{enumi}{2}
\item
  Navigate to your workspace's root folder and run

\begin{verbatim}
blade server init
\end{verbatim}
\item
  Verify your bundle was downloaded. The bundle is generated in the
  \texttt{bundles} folder by default. You can change this by setting the
  \texttt{gradle.properties} file's \texttt{liferay.workspace.home.dir}
  property to a different folder.
\end{enumerate}

You can also produce a distributable Liferay bundle (Zip or Tar) from
within a workspace. To do this, navigate to your workspace's root folder
and run the following command:

\begin{verbatim}
./gradlew distBundle[Zip|Tar]
\end{verbatim}

Your distribution file is available from the workspace's \texttt{/build}
folder.

\noindent\hrulefill

\textbf{Note:} You can define different environments for your Liferay
bundle for easy testing. You can learn more about this in the
\href{/docs/7-2/reference/-/knowledge_base/r/liferay-workspace\#testing-projects}{Testing
Projects} section.

\noindent\hrulefill

You're all set to develop projects for a nested Liferay DXP bundle.

\chapter{Setting Environment Configurations for Liferay
Workspace}\label{setting-environment-configurations-for-liferay-workspace}

{This document has been updated and ported to Liferay Learn and is no
longer maintained here.}

Liferay Workspace offers the \texttt{configs} folder, which provides a
way to organize multiple environment settings and generate a Liferay
bundle for each environment configuration.

To simulate using the \texttt{configs} folder, you'll explore a typical
scenario. Suppose you want a local Liferay DXP installation for testing
and a UAT installation for simulating a production site. Assume you want
the following configuration for the two environments:

\textbf{Local Environment}

\begin{itemize}
\tightlist
\item
  Use MySQL database pointing to localhost
\item
  Skip setup wizard
\end{itemize}

\textbf{UAT Environment}

\begin{itemize}
\tightlist
\item
  Use MySQL database pointing to a live server
\item
  Skip setup wizard
\end{itemize}

To configure these two environments in your workspace, follow the steps
below:

\begin{enumerate}
\def\labelenumi{\arabic{enumi}.}
\item
  Open the \texttt{configs/common} folder and add the
  \texttt{portal-setup-wizard.properties} file with the
  \texttt{setup.wizard.enabled=false} property.
\item
  Open the \texttt{configs/local} folder and configure the MySQL
  database settings for localhost in a \texttt{portal-ext.properties}
  file.
\item
  Open the \texttt{configs/uat} folder and configure the MySQL database
  settings for the live server in a \texttt{portal-ext.properties} file.
\item
  Now that your two environments are configured, generate one of them:

\begin{verbatim}
blade server init --environment uat
\end{verbatim}
\item
  To generate a distributable Liferay DXP installation of the
  environment to the workspace's \texttt{/build} folder, run

\begin{verbatim}
./gradlew distBundle[Zip|Tar] -Pliferay.workspace.environment=uat
\end{verbatim}
\end{enumerate}

\noindent\hrulefill

\begin{verbatim}
 **Note:** You may prefer to set your workspace environment in the
 `gradle.properties` file instead of passing it via Gradle command. If so,
 it's recommended to set the workspace environment variable inside the
 `[USER_HOME]/.gradle/gradle.properties` file.
 
 ```properties
 liferay.workspace.environment=local
 ```
 
 The variable is set to `local` by default.
\end{verbatim}

\noindent\hrulefill

\begin{verbatim}
You've successfully configured two environments and generated one of them.
\end{verbatim}

Awesome! You can now test various Liferay DXP bundle environments using
Liferay Workspace.

\chapter{Building Node.js Themes in Liferay
Workspace}\label{building-node.js-themes-in-liferay-workspace}

{This document has been updated and ported to Liferay Learn and is no
longer maintained here.}

Liferay Workspace reserves the \texttt{themes} folder only for themes
that are created with the Themes Generator. There are no Blade
CLI-provided commands or Maven archetypes to generate a theme for this
folder. You must leverage the
\href{/docs/7-2/reference/-/knowledge_base/r/theme-generator}{Liferay
Theme Generator} from within the \texttt{themes} folder to create them;
you can also copy a generated theme into the folder.

You'll demo this theme management capability next. Be sure the Liferay
Theme Generator's required tooling is installed.

\begin{enumerate}
\def\labelenumi{\arabic{enumi}.}
\item
  Navigate to your workspace's \texttt{themes} folder and run the
  following command:

\begin{verbatim}
yo liferay-theme
\end{verbatim}

  Follow the prompts to create your theme.
\item
  Navigate into your new theme and run

\begin{verbatim}
./gradlew build
\end{verbatim}

  Liferay Workspace builds the front-end theme using Gradle. Under the
  hood, Liferay's
  \href{/docs/7-2/reference/-/knowledge_base/r/node-gradle-plugin}{Node
  Gradle Plugin} is applied and used to build your theme.
\item
  Workspace is smart enough to differentiate between theme types. For
  instance, you can't copy a theme built with the Theme Generator into
  the \texttt{wars} folder and expect it to build. You can test if your
  project is recognized by workspace by running this command from
  workspace's root folder:

\begin{verbatim}
./gradlew projects
\end{verbatim}

  Your CLI should display your new theme under the \texttt{themes}
  project.

  ```bash Root project `liferay-workspace' +--- Project `:themes'
\end{enumerate}

\noindent\hrulefill

\begin{verbatim}
    \--- Project ':themes:my-generated-theme'
\end{verbatim}

\noindent\hrulefill ```

\begin{verbatim}
If you moved a WAR-style theme (Gradle/Maven-based) into the `themes`
folder, it is not recognized by the Gradle `projects` command.
\end{verbatim}

\noindent\hrulefill

\textbf{Note:} Workspace identifies whether a theme was generated by the
Theme Generator by checking whether it has a \texttt{package.json} file.
Any theme without this file is not compatible in the \texttt{themes}
folder.

\noindent\hrulefill

Excellent! You learned how generated themes are recognized in workspace
and where they should reside. For more information on building
Gradle/Maven-based themes in workspace, see its dedicated
\href{/docs/7-2/reference/-/knowledge_base/r/building-gradle-maven-themes-in-liferay-workspace}{article}.

\chapter{Building Gradle/Maven Themes in Liferay
Workspace}\label{building-gradlemaven-themes-in-liferay-workspace}

{This document has been updated and ported to Liferay Learn and is no
longer maintained here.}

Liferay Workspace provides the \texttt{wars} folder for any WAR-style
project. Themes created with
\href{/docs/7-2/reference/-/knowledge_base/r/blade-cli}{Blade CLI} or
Maven using the
\href{/docs/7-2/reference/-/knowledge_base/r/theme-template}{\texttt{theme}}
project template or archetype are automatically generated here when
creating the project within Workspace.

Follow the steps below to build a Gradle/Maven theme in workspace's
\texttt{wars} folder:

\begin{enumerate}
\def\labelenumi{\arabic{enumi}.}
\item
  Follow the
  \href{/docs/7-2/reference/-/knowledge_base/r/creating-a-project}{Creating
  a Project} article to generate a project based on a project template
  or archetype. Make sure to select the \texttt{theme} template.

  Themes built using Liferay's \texttt{theme} project template are
  always WARs and should always reside in Workspace's \texttt{wars}
  folder. They should never be moved to the \texttt{themes} folder; that
  folder is reserved for
  \href{/docs/7-2/reference/-/knowledge_base/r/building-node-js-themes-in-liferay-workspace}{themes
  generated by the Theme Generator}.
\item
  Navigate into your new theme and run

\begin{verbatim}
./gradlew build
\end{verbatim}

  Liferay Workspace builds the theme using Gradle. Under the hood,
  Liferay's
  \href{/docs/7-2/reference/-/knowledge_base/r/theme-builder-gradle-plugin}{Theme
  Builder Gradle Plugin} is applied and used to build your theme. It
  works similarly in a Maven workspace. See the
  \href{/docs/7-1/frameworks/-/knowledge_base/frameworks/building-themes-in-a-maven-project}{Building
  Themes in a Maven Project} article for more information.
\end{enumerate}

Awesome! You know how WAR-style themes are built in workspace and where
they should reside.

\chapter{Managing the Target
Platform}\label{managing-the-target-platform}

{This document has been updated and ported to Liferay Learn and is no
longer maintained here.}

\noindent\hrulefill

\textbf{Note:} The Target Platform articles currently assume you're
using Gradle as a build tool. If your projects are built with Maven, you
can still leverage the Target Platform features, but it is not built
into Liferay Workspace \emph{yet}
(\href{https://issues.liferay.com/browse/LPS-90524}{LPS-90524}). See the
\href{/docs/7-2/reference/-/knowledge_base/r/targeting-a-platform-with-maven}{Targeting
a Platform with Maven} article to set the Target Platform for
Maven-based projects.

\noindent\hrulefill

Liferay Workspace helps you target a specific release of Liferay DXP, so
dependencies get resolved properly. This makes upgrades easy: specify
your target platform, and Workspace points to the new version. All your
dependencies are updated to the latest ones provided in the targeted
release.

\noindent\hrulefill

\textbf{Note:} There are times when configuring dependencies based on a
version range is better than tracking exact versions. See the
\href{/docs/7-2/customization/-/knowledge_base/c/semantic-versioning}{Semantic
Versioning} tutorial for more details.

\noindent\hrulefill

Next, you'll discover how all of this is possible.

\section{Dependency Management with
BOMs}\label{dependency-management-with-boms}

You can target a version by importing a predefined bill of materials
(BOM). This only requires that you specify a property in your
workspace's \texttt{gradle.properties} file (see
\href{/docs/7-2/reference/-/knowledge_base/r/setting-the-target-platform}{this
article} for details).

\noindent\hrulefill

\textbf{Note:} The Target Platform feature is only supported for Gradle
projects at this time.

\noindent\hrulefill

Each Liferay DXP version has a predefined BOM that you can specify for
your workspace to reference. Each BOM defines the artifacts and their
versions used in the specific release. BOMs list all dependencies in a
management fashion, so it doesn't \textbf{add} dependencies to your
project; it only \textbf{provides} your build tool (e.g., Gradle or
Maven) the versions needed for the project's defined artifacts. This
means you don't need to specify your dependency versions; the BOM
automatically defines the appropriate artifact versions based on the
BOM.

You can override a BOM's defined artifact version by specifying a
different version in your project's \texttt{build.gradle}. Artifact
versions defined in your project's build files override those specified
in the predefined BOM. Note that overriding the BOM can be dangerous;
make sure the new version is compatible in the targeted platform.

For more information on BOMs, see the
\href{https://maven.apache.org/guides/introduction/introduction-to-dependency-mechanism\#Importing_Dependencies}{Importing
Dependencies} section in Maven's official documentation. To view a BOM
file and its mapping of artifacts and versions, visit
\href{https://repository.liferay.com}{repository.liferay.com} and search
for the BOM artifacts (e.g.,
\href{https://repository.liferay.com/nexus/index.html\#nexus-search;quick~release.portal.bom}{release.portal.bom}
and
\href{https://repository.liferay.com/nexus/index.html\#nexus-search;quick~release.dxp.bom}{release.dxp.bom}).

Pretty cool, right? Next, you'll learn how to leverage platform
targeting in Dev Studio.

\section{Leveraging Target Platform in Dev
Studio}\label{leveraging-target-platform-in-dev-studio}

\href{/docs/7-2/reference/-/knowledge_base/r/liferay-dev-studio}{Liferay
Dev Studio 3.2+} helps you streamline targeting a specific version even
more. Dev Studio can index the configured Liferay DXP source code to

\begin{itemize}
\tightlist
\item
  provide advanced Java search (Open Type and Reference Searching)
  (\href{/docs/7-2/reference/-/knowledge_base/r/searching-product-source-in-dev-studio}{article})
\item
  debug Liferay DXP sources
  (\href{/docs/7-2/reference/-/knowledge_base/r/debugging-product-source-in-dev-studio}{article})
\end{itemize}

To enable this functionality, set the following property in your
workspace's \texttt{gradle.properties} file:

\begin{verbatim}
target.platform.index.sources=true
\end{verbatim}

\noindent\hrulefill

\textbf{Note:} Portal source indexing is disabled in Gradle workspace
version 2.0.3+ (Target Platform plugin version 2.0.0+). See the
\href{/docs/7-2/reference/-/knowledge_base/r/updating-liferay-workspace}{Updating
Liferay Workspace} article for instructions on how to update your
workspace.

\noindent\hrulefill

These options in Dev Studio are only available when developing in a
Liferay Workspace, or if you have the
\href{/docs/7-2/reference/-/knowledge_base/r/target-platform-gradle-plugin}{Target
Platform Gradle plugin} applied to your multi-module Gradle project with
specific configurations. See the
\href{/docs/7-2/reference/-/knowledge_base/r/targeting-a-platform-outside-of-workspace}{Targeting
a Platform Outside of Workspace} article for more info on applying the
Target Platform Gradle plugin.

Continue on to learn how to set the target platform.

\chapter{Setting the Target Platform}\label{setting-the-target-platform}

{This document has been updated and ported to Liferay Learn and is no
longer maintained here.}

Setting the target platform version to develop for takes two steps:

\begin{enumerate}
\def\labelenumi{\arabic{enumi}.}
\item
  Open the workspace's \texttt{gradle.properties} file and set the
  \texttt{liferay.workspace.target.platform.version} property to the
  version you want to target. For example,

\begin{verbatim}
liferay.workspace.target.platform.version=7.2.0
\end{verbatim}
\end{enumerate}

\noindent\hrulefill

\begin{verbatim}
 **Note:** You must explicitly uncomment the property in your workspace's
 `gradle.properties` file to set it. Target Platform is not enabled by
 default.
\end{verbatim}

\noindent\hrulefill

\begin{verbatim}
If you're using Liferay DXP, you can set the property like this:

```properties
liferay.workspace.target.platform.version=7.2.10
```

The versions following a GA1 release of DXP follow fix pack versions (e.g.,
`7.2.10.fp1`, `7.2.10.fp2`, etc.).
\end{verbatim}

\begin{enumerate}
\def\labelenumi{\arabic{enumi}.}
\setcounter{enumi}{1}
\item
  Once the target platform is configured, check to make sure no
  dependencies in your Gradle build files specify a version. The
  versions are now imported from the configured target platform's BOM.
  For example, a simple MVC portlet's \texttt{build.gradle} may look
  something like this:

\begin{verbatim}
dependencies {
    compileOnly group: "com.liferay.portal", name: "com.liferay.portal.kernel"
    compileOnly group: "com.liferay.portal", name: "com.liferay.util.taglib"
    compileOnly group: "javax.portlet", name: "portlet-api"
    compileOnly group: "javax.servlet", name: "javax.servlet-api"
    compileOnly group: "jstl", name: "jstl"
    compileOnly group: "org.osgi", name: "osgi.cmpn"
}
\end{verbatim}
\end{enumerate}

\noindent\hrulefill

\textbf{Note}: The \texttt{liferay.workspace.target.platform.version}
property also sets the distro JAR, which can be used to validate your
projects during the build process. See the
\href{/docs/7-2/reference/-/knowledge_base/r/validating-modules-against-the-target-platform}{Validating
Modules Against the Target Platform} articles for more info.

\noindent\hrulefill

\noindent\hrulefill

\textbf{Note:} The target platform functionality is available in Liferay
Workspace version 1.9.0+. If you have an older version, you must update
it to leverage platform targeting. See the
\href{/docs/7-2/reference/-/knowledge_base/r/updating-liferay-workspace}{Updating
Liferay Workspace} article to do this.

\noindent\hrulefill

You've configured your target platform in workspace. You're all set!

\chapter{Targeting a Platform Outside of
Workspace}\label{targeting-a-platform-outside-of-workspace}

{This document has been updated and ported to Liferay Learn and is no
longer maintained here.}

If you prefer to not use Liferay Workspace, but still want to target a
platform, you must apply the
\href{/docs/7-2/reference/-/knowledge_base/r/target-platform-gradle-plugin}{Target
Platform Gradle plugin} to the root \texttt{build.gradle} file of your
custom multi-module Gradle build.

To do this, follow the steps below.

\begin{enumerate}
\def\labelenumi{\arabic{enumi}.}
\item
  Open your project's \texttt{build.gradle} file and add this:

\begin{verbatim}
buildscript {
    dependencies {
        classpath group: "com.liferay", name: "com.liferay.gradle.plugins.target.platform", version: "2.0.0"
    }
    repositories {
        maven {
            url "https://repository-cdn.liferay.com/nexus/content/groups/public"
        }
    }
}
\end{verbatim}

  This sets the dependency on the Target Platform Gradle plugin and
  configures the repository that provides the necessary artifacts for
  your project build.
\item
  Apply Liferay's Target Platform Gradle plugin to the
  \texttt{build.xml} file:

\begin{verbatim}
apply plugin: "com.liferay.target.platform"
\end{verbatim}
\item
  Set the Target Platform plugin's dependencies:

\begin{verbatim}
dependencies {
    targetPlatformBoms group: "com.liferay.portal", name: "release.portal.bom", version: "7.2.0"
    targetPlatformBoms group: "com.liferay.portal", name: "release.portal.bom.compile.only", version: "7.2.0"
    targetPlatformBoms group: "com.liferay.portal", name: "release.portal.bom.third.party", version: "7.2.0"
}
\end{verbatim}

  These dependencies are described below:

  \texttt{com.liferay.ce.portal.bom}: provides all the artifacts
  included in Liferay DXP.

  \texttt{com.liferay.ce.portal.compile.only}: provides artifacts that
  are not included in Liferay DXP, but are necessary to reference during
  the build (e.g., \texttt{org.osgi.core}).

  \texttt{release.portal.bom.third.party}: provides all third party
  artifacts that make up the Liferay Portal bundle.

  Liferay DXP users must replace the artifact names and versions:

  \begin{itemize}
  \tightlist
  \item
    \texttt{release.portal.bom} → \texttt{release.dxp.bom}
  \item
    \texttt{release.portal.bom.compile.only} →
    \texttt{release.dxp.bom.compile.only}
  \item
    \texttt{release.portal.bom.third.party} →
    \texttt{release.dxp.bom.third.party}
  \item
    \texttt{7.2.0} → \texttt{7.2.10}
  \end{itemize}
\item
  If you're interested in
  \href{/docs/7-2/reference/-/knowledge_base/r/searching-product-source-in-dev-studio}{advanced
  search} and/or
  \href{/docs/7-2/reference/-/knowledge_base/r/debugging-product-source-in-dev-studio}{debugging}
  Liferay DXP's source using
  \href{/docs/7-2/reference/-/knowledge_base/r/liferay-dev-studio}{Liferay
  Dev Studio}, you must also apply the following configuration:

\begin{verbatim}
targetPlatformIDE {
    includeGroups "com.liferay", "com.liferay.portal"
}
\end{verbatim}

  This indexes the target platform's source code and makes it available
  to Dev Studio.
\end{enumerate}

Now you can define your target platform!

\chapter{Targeting a Platform with
Maven}\label{targeting-a-platform-with-maven}

{This document has been updated and ported to Liferay Learn and is no
longer maintained here.}

Although a Maven-based Liferay Workspace does not offer a configurable
property to set the target platform, you can still leverage the Target
Platform framework by adding a few dependencies to your project.

\begin{enumerate}
\def\labelenumi{\arabic{enumi}.}
\item
  Open your workspace's root \texttt{pom.xml} file and add the following
  dependencies:

\begin{verbatim}
<dependencyManagement>
    <dependencies>
        <dependency>
            <groupId>com.liferay.portal</groupId>
            <artifactId>release.portal.bom</artifactId>
            <version>7.2.0</version>
            <type>pom</type>
            <scope>import</scope>
        </dependency>
        <dependency>
            <groupId>com.liferay.portal</groupId>
            <artifactId>release.portal.bom.compile.only</artifactId>
            <version>7.2.0</version>
            <type>pom</type>
            <scope>import</scope>
        </dependency>
        <dependency>
            <groupId>com.liferay.portal</groupId>
            <artifactId>release.portal.bom.third.party</artifactId>
            <version>7.2.0</version>
            <type>pom</type>
            <scope>import</scope>
        </dependency>
    </dependencies>
</dependencyManagement>
\end{verbatim}

  These dependencies are described below:

  \texttt{com.liferay.ce.portal.bom}: provides all the artifacts
  included in Liferay DXP.

  \texttt{com.liferay.ce.portal.compile.only}: provides artifacts that
  are not included in Liferay DXP, but are necessary to reference during
  the build (e.g., \texttt{org.osgi.core}).

  \texttt{release.portal.bom.third.party}: provides all third party
  artifacts that make up the Liferay Portal bundle.

  Liferay DXP users must replace the artifact names and versions:

  \begin{itemize}
  \tightlist
  \item
    \texttt{release.portal.bom} → \texttt{release.dxp.bom}
  \item
    \texttt{release.portal.bom.compile.only} →
    \texttt{release.dxp.bom.compile.only}
  \item
    \texttt{release.portal.bom.third.party}
  \item
    \texttt{7.2.0} → \texttt{7.2.10}
  \end{itemize}
\item
  Go through the remaining POMs in your workspace and remove
  \texttt{\textless{}version\textgreater{}} tags for all
  Liferay-specific artifacts. These versions are now being provided by
  the Target Platform framework.
\end{enumerate}

Great! You can now target a platform in your Maven-based workspace.

\chapter{Validating Modules Against the Target
Platform}\label{validating-modules-against-the-target-platform}

{This document has been updated and ported to Liferay Learn and is no
longer maintained here.}

\noindent\hrulefill

\textbf{Important:} Validating modules with the \texttt{resolve} task is
deprecated. It only functions as it's documented here in versions prior
to Liferay Workspace (Gradle only) version 2.0.3. It is being redesigned
for workspace versions 2.0.3+ and is still in development at this time.

\noindent\hrulefill

After you write a module in Liferay Workspace, you can validate it
before deployment to make sure of several things:

\begin{itemize}
\tightlist
\item
  Will my app deploy successfully?
\item
  Will there be some sort of missing requirement?
\item
  If there's an issue, how do I diagnose it?
\end{itemize}

These are all common worries that can be frustrating.

Instead of deploying your app and checking for errors in the log, you
can validate your app before deployment. This is done by calling Liferay
Workspace's \texttt{resolve} task, which validates your modules against
a targeted platform.

You'll cover the following topics in this section:

\begin{itemize}
\tightlist
\item
  \hyperref[resolving-your-modules]{Resolving your modules}.
\item
  \hyperref[modifying-the-target-platforms-capabilities]{Modifying the
  target platform's capabilities}.
\item
  \hyperref[including-the-resolver-in-your-gradle-build]{Including the
  resolver in your Gradle build}.
\end{itemize}

Continue on to learn how this works.

\section{Resolving Your Modules}\label{resolving-your-modules}

You can resolve your modules before deployment. This can be done by
calling the \texttt{resolve} Gradle task provided by Liferay Workspace.

\begin{verbatim}
./gradlew resolve
\end{verbatim}

This task gathers all the capabilities provided by

\begin{itemize}
\tightlist
\item
  the specified version of Liferay DXP (i.e.,
  \href{/docs/7-2/reference/-/knowledge_base/r/managing-the-target-platform}{targeted
  platform})
\item
  the current workspace's modules
\end{itemize}

Some capabilities/information gathered by the \texttt{resolve} task that
are validated include

\begin{itemize}
\tightlist
\item
  declared required capabilities
\item
  module versions
\item
  package imports/use constraints
\item
  service references
\end{itemize}

It also computes a list of run requirements for your project. Then it
compares the current project's requirements against the gathered
capabilities. If your project requires something not available in the
gathered list of capabilities, the task fails.

The task can only validate OSGi modules. It does not work with WAR-style
projects, themes, or npm portlets.

\noindent\hrulefill

\textbf{Note:} The \texttt{resolve} task can be executed from a specific
project folder or from the workspace's root folder. Running the task
from the root folder validates all the modules in your workspace.

\noindent\hrulefill

The \texttt{resolve} task can automatically gather the available
capabilities from your workspace, but you must specify this for your
targeted Liferay DXP version. To do this, open your workspace's
\texttt{gradle.properties} file and set the
\texttt{liferay.workspace.target.platform.version} property to the
version you want to target. For example,

\begin{verbatim}
liferay.workspace.target.platform.version=7.2.0
\end{verbatim}

If you're using Liferay DXP, you can set the property like this:

\begin{verbatim}
liferay.workspace.target.platform.version=7.2.10
\end{verbatim}

The versions following a GA1 release of DXP follow fix pack versions
(e.g., \texttt{7.2.10.fp1}, \texttt{7.2.10.fp2}, etc.).

Setting the target platform property provides a static \emph{distro} JAR
for the specified version of Liferay DXP, which contains all the
metadata (i.e., capabilities, packages, versions, etc.) running in that
version. The distro JAR is a complete snapshot of everything provided in
the OSGi runtime; this serves as the target platform's list of
capabilities that your modules are validated against.

You can now validate your module projects before deploying them! If the
resolver throws errors, see the article on
\href{/docs/7-2/reference/-/knowledge_base/r/how-to-resolve-common-output-errors-reported-by-the-resolve-task}{how
to resolve common output errors reported by the \texttt{resolve} task}.
Sometimes, you must modify the \texttt{resolve} task's default behavior
to successfully validate your app. See the next section for more
information.

\section{Modifying the Target Platform's
Capabilities}\label{modifying-the-target-platforms-capabilities}

In a perfect world, everything the \texttt{resolve} task gathers and
checks against would work during your development process.
Unfortunately, there are exceptions that may force you to modify the
default functionality of the \texttt{resolve} task.

There are two scenarios you may run into during development that require
a modification for your project to pass the resolver check.

\begin{itemize}
\tightlist
\item
  You're depending on a third party library that is not available in the
  targeted Liferay DXP instance or the current workspace.
\item
  You're depending on a customized distribution of Liferay DXP.
\end{itemize}

You'll explore these use cases next.

\section{Depending on Third Party Libraries Not Included in Liferay
DXP}\label{depending-on-third-party-libraries-not-included-in-liferay-dxp}

The \texttt{resolve} task, by default, gathers all of Liferay DXP's
capabilities and the capabilities of your workspace's modules. What if,
however, your module depends on a third party project that is not
included in either space (e.g.,
\href{https://opensource.google.com/projects/guava}{Google Guava})?. The
\texttt{resolve} task fails by default if your project depends on this
project type. You probably plan to have this project deployed and
available at runtime, so it's not a concern, but the resolver doesn't
know that; you must customize the resolver to bypass this.

There are three ways you can do this:

\begin{itemize}
\tightlist
\item
  \href{/docs/7-2/customization/-/knowledge_base/c/adding-third-party-libraries-to-a-module}{Embed
  the third party library in your module}
\item
  \href{/docs/7-2/reference/-/knowledge_base/r/adding-a-third-party-librarys-capabilities-to-the-resolvers-capabilities}{Add
  the third party library's capabilities to the current static set of
  resolver capabilities}
\item
  \href{/docs/7-2/reference/-/knowledge_base/r/skipping-the-resolving-process-for-a-module}{Skip
  the resolving process for your module}
\end{itemize}

\noindent\hrulefill

\textbf{Note:} You should only embed a third party library in your
module if it's the only module that depends on it. You should not bypass
the resolver failure this way if more than one project in the OSGi
container depends on that library.

\noindent\hrulefill

For help resolving third party dependency errors, see the
\href{/docs/7-1/frameworks/-/knowledge_base/frameworks/adding-third-party-libraries-to-a-module}{Resolving
Third Party Library Package Dependencies} tutorial.

\section{Depending on a Customized Distribution of Liferay
DXP}\label{depending-on-a-customized-distribution-of-liferay-dxp}

There are times when manually specifying your project's list of
dependent JARs does not suffice. If your app requires a customized
Liferay DXP instance to run, you must regenerate the target platform's
default list of capabilities with an updated list. Two examples of a
customized Liferay DXP instance are described below:

\textbf{Example 1: Leveraging an External Feature}

There are many external features/frameworks available that are not
included in the downloadable bundle by default. After deploying a
feature/framework, it's available for your module projects to leverage.
When validating your app, however, the \texttt{resolve} task does not
have access to external capabilities not included by default. For
example, Audience Targeting is an example of this type of external
framework. If you're creating a Liferay Audience Targeting rule that
depends on the Audience Targeting framework, you can't easily provide a
slew of JARs for your module. In this case, you should install the
platform your code depends on and regenerate an updated list of
capabilities that your Liferay DXP instance provides.

\textbf{Example 2: Leveraging a Customized Core Feature}

You can extend Liferay DXP's core features to provide a customized
experience for your intended audience. Once deployed, you can assume
these customizations are present and build other things on top of them.
The new capabilities resulting from your customizations are not
available, however, in the target platform's default list of
capabilities. Therefore, when your application relies on non-default
capabilities, it fails during the \texttt{resolve} task. To get around
this, you must regenerate a new list of capabilities that your
customized Liferay DXP instance provides.

To regenerate the target platform's capabilities (distro JAR) based on
the current workspace's Liferay DXP instance, follow the
\href{/docs/7-2/reference/-/knowledge_base/r/depending-on-a-customized-distribution-of-product}{Depending
on a Customized Distribution of Liferay DXP} article.

\section{Including the Resolver in Your Gradle
Build}\label{including-the-resolver-in-your-gradle-build}

By default, Liferay Workspace provides the \texttt{resolve} task as an
independent executable. It's provided by the
\href{/docs/7-2/reference/-/knowledge_base/r/target-platform-gradle-plugin}{Target
Platform} Gradle plugin and is not integrated in any other Gradle
processes. This gives you control over your Gradle build without
imposing strategies you may not want included in your default build
process.

With that said, the \texttt{resolve} task can be useful to include in
your build process if you want to check for errors in your module
projects before deployment. Instead of resolving your projects
separately from your standard build, you can build and resolve them all
in one shot.

In Liferay Workspace, the recommended path for doing this is adding it
to the default \texttt{check} Gradle task. The \texttt{check} task is
provided by default in a workspace by the
\href{https://docs.gradle.org/current/userguide/java_plugin.html\#_lifecycle_tasks}{Java}
plugin. Adding the \texttt{resolve} task to the \texttt{check} lifecycle
task also promotes the \texttt{resolve} task to run for CI and other
test tools that typically run the \texttt{check} task for verification.
Of course, Gradle's \texttt{build} task also depends on the
\texttt{check} task, so you can run \texttt{gradlew\ build} and run the
resolver too.

You can learn how to include the resolver in your Gradle build by
visiting
\href{/docs/7-2/reference/-/knowledge_base/r/including-the-resolver-in-your-gradle-build}{this
article}.

Continue on for various step-by-step instructions for
configuring/manipulating the resolver task.

\chapter{Adding a Third Party Library's Capabilities to the Resolver's
Capabilities}\label{adding-a-third-party-librarys-capabilities-to-the-resolvers-capabilities}

{This document has been updated and ported to Liferay Learn and is no
longer maintained here.}

You can add your third party dependencies to the target platform's
default list of capabilities by listing them as provided modules.

\begin{enumerate}
\def\labelenumi{\arabic{enumi}.}
\item
  Open your workspace's root \texttt{build.gradle} file.
\item
  Add a code snippet similar to this:

\begin{verbatim}
dependencies {
    providedModules group: "GROUP_ID", name: "NAME", version: "VERSION"
}
\end{verbatim}

  For example, if you wanted to add
  \href{https://opensource.google.com/projects/guava}{Google Guava} as a
  provided module, it would look like this:

\begin{verbatim}
dependencies {
    providedModules group: "com.google.guava", name: "guava", version: "23.0"
}
\end{verbatim}
\end{enumerate}

This both provides the third party dependency to the resolver, and it
downloads and includes it in your Liferay DXP bundle's
\texttt{osgi/modules} folder when you initialize it (e.g.,
\texttt{blade\ server\ init}).

\chapter{Skipping the Resolving Process for a
Module}\label{skipping-the-resolving-process-for-a-module}

{This document has been updated and ported to Liferay Learn and is no
longer maintained here.}

It may be easiest to skip validating a particular module during the
resolve process.

\begin{enumerate}
\def\labelenumi{\arabic{enumi}.}
\item
  Open your workspace's root \texttt{build.gradle} file.
\item
  Insert the following Gradle code at the bottom of the file:

\begin{verbatim}
targetPlatform {
    resolveOnlyIf { project ->
        project.name != 'PROJECT_NAME'
    }
}
\end{verbatim}

  Be sure to replace the \texttt{PROJECT\_NAME} filler with your
  module's name (e.g., \texttt{test-api}).
\item
  (Optional) If you prefer to disable the Target Platform plugin
  altogether, you can add a slightly different directive to your
  \texttt{build.gradle} file:

\begin{verbatim}
targetPlatform {
    onlyIf { project ->
        project.name != 'PROJECT_NAME'
    }
}
\end{verbatim}

  This both skips the \texttt{resolve} task execution and disables BOM
  dependency management.
\end{enumerate}

Now the \texttt{resolve} task skips your module project.

\chapter{Depending on a Customized Distribution of Liferay
DXP}\label{depending-on-a-customized-distribution-of-liferay-dxp-1}

{This document has been updated and ported to Liferay Learn and is no
longer maintained here.}

To regenerate the target platform's capabilities (distro JAR) based on
the current workspace's Liferay DXP instance, follow the steps below:

\begin{enumerate}
\def\labelenumi{\arabic{enumi}.}
\item
  Start the Liferay DXP instance stored in your workspace. Make sure the
  platform you want to depend on is installed.
\item
  Download the
  \href{https://search.maven.org/\#search\%7Cga\%7C1\%7Cbiz.aqute.remote.agent}{BND
  Remote Agent JAR file} and copy it into the \texttt{osgi/modules}
  folder.
\item
  From the root folder of your workspace, run the following command:

\begin{verbatim}
bnd remote distro -o custom_distro.jar release.portal.distro 7.2.0
\end{verbatim}

  Liferay DXP users must replace the \texttt{release.portal.distro}
  artifact name with \texttt{release.dxp.distro} and use the
  \texttt{7.2.10} version syntax.

  This connects to the newly deployed BND agent running in Liferay DXP
  and generates a new distro JAR named \texttt{custom\_distro.jar}. All
  other capabilities inherit their functionality based on your Liferay
  DXP instance, so verify the workspace bundle is the version you plan
  to release in production.
\item
  Navigate to your workspace's root \texttt{build.gradle} file and add
  the following dependency:

\begin{verbatim}
dependencies {
    targetPlatformDistro files('custom_distro.jar')
}
\end{verbatim}
\end{enumerate}

Now your workspace is pointing to a custom distro JAR file instead of
the default one provided. Run the \texttt{resolve} task to validate your
modules against the new set of capabilities.

\chapter{Including the Resolver in Your Gradle
Build}\label{including-the-resolver-in-your-gradle-build-1}

{This document has been updated and ported to Liferay Learn and is no
longer maintained here.}

To call the \texttt{resolve} task during Gradle's \texttt{check} task
automatically, follow the instructions below:

\begin{enumerate}
\def\labelenumi{\arabic{enumi}.}
\item
  Open your workspace's root \texttt{build.gradle} file.
\item
  Add the following directive:

\begin{verbatim}
check.dependsOn resolve
\end{verbatim}

  The \texttt{resolve} task is now called during the \texttt{check}
  task.

  You can also configure this for specific projects in a workspace if
  you don't want all modules to be included in the global
  \texttt{check}.
\item
  (Optional) If the \texttt{resolve} task runs during every Gradle
  build, you may want to prevent the build from failing if there are
  errors reported by the resolver. To do this, open your workspace's
  root \texttt{build.gradle} file and add the following code:

\begin{verbatim}
targetPlatform {
    ignoreResolveFailures = true
}
\end{verbatim}

  This reports the failures without failing the build. Note, this can
  only be configured in the workspace's root \texttt{build.gradle} file.
\end{enumerate}

Awesome! You can now run the \texttt{resolve} task in your current
Gradle lifecycle.

\chapter{How to Resolve Common Output Errors Reported by the Resolve
Task}\label{how-to-resolve-common-output-errors-reported-by-the-resolve-task}

{This document has been updated and ported to Liferay Learn and is no
longer maintained here.}

Liferay Workspace provides the \texttt{resolve} Gradle task to validate
modules. This is very useful for finding issues and reporting them as
output before deployment. For general help with OSGi related issues,
visit the
\href{/docs/7-2/appdev/-/knowledge_base/a/troubleshooting-application-development-issues}{Troubleshooting
FAQ} section.

For help interpreting the \texttt{resolve} task's output, see the list
below for common output errors, what they mean, and how to fix them.

\section{Missing Import Error}\label{missing-import-error}

When your module refers to an unavailable import, the container throws
this error. For example, suppose you have a module \texttt{test-service}
that depends on the \texttt{com.google.common.base} package. If the
container can't find that package, it throws this error:

\begin{verbatim}
Resolution exception in project 'modules:test-service': Unresolved requirements in root project 'modules:test-service':
    Mandatory:
        [osgi.wiring.package ] com.google.common.base; version=[23.0.0,24.0.0)
        [osgi.identity       ] test.service
\end{verbatim}

This kind of error can also occur when separate modules require
different versions of another module. If you have \emph{module A}
requiring \emph{module Test version 1} and \emph{module B} requiring
\emph{module Test version 4}, without running the resolver, both modules
A and B would compile successfully. When they were deployed, however,
one would fail in the OSGi runtime because both dependencies cannot be
satisfied. These types of scenarios are difficult to diagnose, but with
the \texttt{resolve} task, can be found with ease.

To fix missing import errors, you may need to adjust the
\href{/docs/7-2/customization/-/knowledge_base/c/exporting-packages}{export}
and/or
\href{/docs/7-2/customization/-/knowledge_base/c/importing-packages}{import}
configuration of your modules. Also, see the
\href{/docs/7-2/customization/-/knowledge_base/c/adding-third-party-libraries-to-a-module}{Resolving
Third Party Library Package Dependencies} tutorial for more information
on resolving import errors. Sometimes, this kind of error can be solved
by editing the \texttt{resolve} task's list of capabilities. See the
\href{/docs/7-2/customization/-/knowledge_base/c/adding-third-party-libraries-to-a-module}{Resolving
Third Party Library Package Dependencies} section to learn how to do
this.

\section{Missing Service Reference}\label{missing-service-reference}

If your module references a non-existent service, an error is thrown.
This is helpful because service reference issues are hard to diagnose
during deployment without using the
\href{/docs/7-2/customization/-/knowledge_base/c/using-the-felix-gogo-shell}{Gogo
Shell}.

For example, if your module \texttt{test-portlet} references a service
(e.g., \texttt{test.api.TestApi}) it does not have access to, the
following error is thrown:

\begin{verbatim}
Resolution exception in project 'modules:test-portlet': Unresolved requirements in project 'modules:test-portlet':
    Mandatory:
        [osgi.identity ] test.portlet
        [osgi.service  ] objectClass=test.api.TestApi
\end{verbatim}

To fix this, you must make the service available to your module. If
you're expecting the service to be provided by your target platform,
check to make sure it's being provided. If it's a service provided by a
custom module, check that service provider module and ensure it's
correctly providing that service to your module. To check the target
platform for available services, follow the steps below:

\begin{enumerate}
\def\labelenumi{\arabic{enumi}.}
\item
  Start your target platform instance.
\item
  Open the Gogo shell.
\item
  List all services containing a keyword by running
  \texttt{services\ \textbar{}\ grep\ \ \ \ \ "SERVICE\_NAME"}. It's
  easiest to do this rather than listing all services since there are
  usually too many to sift through.
\item
  You can also list services provided by a component. Run
  \texttt{lb\ -s} to list all provided bundles by their bundle symbolic
  name (BSN). Find the BSN for the desired component and then run
  \texttt{scr:info\ \textless{}BSN\textgreater{}}.
\end{enumerate}

If you're unable to track down your missing service, it may be provided
by a customized Liferay DXP core feature or an external Liferay DXP
feature. If this is the case, it isn't included in the target platform's
default capabilities. You can make the custom service capability
available to reference by
\href{/docs/7-2/reference/-/knowledge_base/r/depending-on-a-customized-distribution-of-product}{generating
a new custom distro JAR}.

\section{Missing Fragment Host}\label{missing-fragment-host}

Referring to a non-existent fragment host throws an error. For example,
if your \texttt{test.login} fragment is configured to modify a fragment
host named \texttt{com.liferay.login.web} that cannot be referenced, the
following error is thrown:

\begin{verbatim}
Resolution exception in project 'modules:test.login': Unresolved requirements in project 'modules:test-login':
    Mandatory:
        [osgi.identity    ] test.login
        [osgi.wiring.host ] com.liferay.login.web; version=1.0.10
\end{verbatim}

Configuring a fragment host in your module is typically done with the
\texttt{Fragment-Host} header in the \texttt{bnd.bnd} file:

\begin{verbatim}
Fragment-Host: com.liferay.login.web;bundle-version="[1.0.0,1.0.1)"
\end{verbatim}

To fix this, inspect your target platform to ensure it includes the JAR
you're attempting to add a fragment for. Your fragment host header may
be referencing an incorrect bundle symbolic name (BSN) or version. The
easiest way to check this is by using the
\href{/docs/7-2/customization/-/knowledge_base/c/using-the-felix-gogo-shell}{Gogo
Shell}. Follow the steps below to find the bundle symbolic name:

\begin{enumerate}
\def\labelenumi{\arabic{enumi}.}
\item
  Start your target platform instance.
\item
  Open the Gogo shell.
\item
  List all installed bundles by BSN with the command \texttt{lb\ -s}.
  You can search through the output to find the BSN. If you already know
  the BSN and want to check the version, run
  \texttt{lb\ -s\ \textbar{}\ grep\ "\textless{}BSN\textgreater{}"}.
\end{enumerate}

Once you know the correct BSN/version to reference, update your
\texttt{Fragment-Host} header to resolve the error.

For more information on fragments, see the
\href{/docs/7-2/customization/-/knowledge_base/c/jsp-overrides-using-osgi-fragments}{JSP
Overrides Using OSGi Fragments} tutorial.

\chapter{Validating Modules Outside of
Workspace}\label{validating-modules-outside-of-workspace}

{This document has been updated and ported to Liferay Learn and is no
longer maintained here.}

If you prefer to not use Liferay Workspace, but still want to validate
modules against a target platform, you must apply the
\href{/docs/7-2/reference/-/knowledge_base/r/target-platform-gradle-plugin}{Target
Platform Gradle plugin} to the root \texttt{build.gradle} file of your
multi-module Gradle build. Follow the
\href{/docs/7-2/reference/-/knowledge_base/r/targeting-a-platform-outside-of-workspace}{Targeting
a Platform Outside of Workspace} section to do this.

Once you have the Target Platform plugin and its BOM dependencies
configured, you must configure the \texttt{targetPlatformDistro}
dependency. Follow the instructions below to do this.

\begin{enumerate}
\def\labelenumi{\arabic{enumi}.}
\item
  Open your project's root \texttt{build.gradle} file.
\item
  Add the \texttt{targetPlatformDistro} dependency to the list of
  dependencies. It should look like this:

\begin{verbatim}
dependencies {
    targetPlatformBoms group: "com.liferay.portal", name: "release.portal.bom", version: "7.2.0"
    targetPlatformBoms group: "com.liferay.portal", name: "release.portal.bom.compile.only", version: "7.2.0"
    targetPlatformDistro group: "com.liferay.portal", name "release.portal.distro", version: "7.2.0"
}
\end{verbatim}

  Liferay DXP users must replace the artifact names and versions:

  \begin{itemize}
  \tightlist
  \item
    \texttt{release.portal.bom} → \texttt{release.dxp.bom}
  \item
    \texttt{release.portal.bom.compile.only} →
    \texttt{release.dxp.bom.compile.only}
  \item
    \texttt{release.portal.distro} → \texttt{release.dxp.distro}
  \item
    \texttt{7.2.0} → \texttt{7.2.10}
  \end{itemize}
\end{enumerate}

Now you can validate your non-workspace modules against a target
platform!

\chapter{Leveraging Docker}\label{leveraging-docker}

{This document has been updated and ported to Liferay Learn and is no
longer maintained here.}

Docker has become increasingly popular in today's development lifecycle,
by providing an automated way to package software and its dependencies
into a standardized unit that can be shared cross-platform. Read
Docker's extensive \href{https://docs.docker.com/}{documentation} to
learn more.

Liferay provides Docker images for

\begin{itemize}
\tightlist
\item
  \href{https://hub.docker.com/r/liferay/portal}{Liferay Portal}
\item
  \href{https://hub.docker.com/r/liferay/dxp}{Liferay DXP}
\item
  \href{https://hub.docker.com/r/liferay/commerce}{Liferay Commerce}
\item
  \href{https://hub.docker.com/r/liferay/portal-snapshot}{Liferay Portal
  Snapshots}
\end{itemize}

You can pull Liferay's Docker images from those resources and manage
them yourself. Liferay Workspace, however, provides an easy way to
integrate Docker development into your existing development workflow
with preconfigured Gradle tasks.

The following Docker commands (Gradle-based) are available in Liferay
Workspace:

Command \textbar{} Description \texttt{buildDockerImage} \textbar{}
Builds the Docker image with all modules/configurations deployed.
\texttt{createDockerContainer} \textbar{} Creates a Docker container
from the Liferay DXP image and mounts the workspace's
\texttt{/build/docker} folder to the container's \texttt{/etc/liferay}
folder. \texttt{createDockerfile} \textbar{} Creates a
\texttt{Dockerfile} to build the Docker image. \texttt{dockerDeploy}
\textbar{} Deploys the project to the container's \texttt{deploy} folder
by copying the project archive file to workspace's
\texttt{build/docker/deploy} folder. This command can also be executed
from workspace's root folder to deploy all projects and copy all Docker
configurations (i.e., from the \texttt{configs/common} and
\texttt{configs/docker} folders) to the container.
\texttt{logsDockerContainer} \textbar{} Prints the portal runtime's
logs. You can exit log tracking mode while maintaining a running
container (e.g., {[}Ctrl\textbar Command{]} + C).
\texttt{pullDockerImage} \textbar{} Pulls the Docker image.
\texttt{removeDockerContainer} \textbar{} Removes the container from
Docker's system. \texttt{startDockerContainer} \textbar{} Starts the
Docker container. \texttt{stopDockerContainer} \textbar{} Stops the
Docker container.

\noindent\hrulefill

\textbf{Note:} Leveraging Docker in Liferay Workspace is only available
for Gradle projects at this time.

\noindent\hrulefill

In this section, you'll learn how to

\begin{itemize}
\tightlist
\item
  \href{/docs/7-2/reference/-/knowledge_base/r/creating-a-product-docker-container}{Create
  a Docker container based on a provided Liferay DXP image}.
\item
  \href{/docs/7-2/reference/-/knowledge_base/r/configuring-a-docker-container}{Configure
  the container}.
\item
  \href{/docs/7-2/reference/-/knowledge_base/r/building-a-custom-docker-image}{Build
  a custom image}.
\end{itemize}

Continue on to learn more.

\chapter{Creating a Liferay DXP Docker
Container}\label{creating-a-liferay-dxp-docker-container}

{This document has been updated and ported to Liferay Learn and is no
longer maintained here.}

To create a Liferay DXP Docker container in Liferay Workspace, complete
the steps below.

\begin{enumerate}
\def\labelenumi{\arabic{enumi}.}
\item
  Choose the Docker image you need. This is configured in your
  workspace's \texttt{gradle.properties} file by customizing this
  property:

\begin{verbatim}
liferay.workspace.docker.image.liferay
\end{verbatim}

  To find the possible property values you can set, see the official
  Liferay DXP Docker Hub's Tags section (e.g.,
  \href{https://hub.docker.com/r/liferay/portal/tags}{Liferay Portal
  Docker Tags}). For example, if you want to base your container on the
  Liferay Portal 7.2 GA1 image, you would set this property:

\begin{verbatim}
liferay.workspace.docker.image.liferay=liferay/portal:7.2.0-ga1
\end{verbatim}
\item
  Run the following command from your workspace's root folder:

\begin{verbatim}
./gradlew createDockerContainer
\end{verbatim}
\end{enumerate}

This command creates a new container named
\texttt{{[}projectName{]}-liferayapp}. A new \texttt{build/docker}
folder is generated in your workspace. This folder is mounted into the
container's file system. This means files in workspace's
\texttt{build/docker} folder are also available in the container's
\texttt{/etc/liferay} folder.

Any projects in your workspace are automatically compiled and copied to
the \texttt{build/docker/deploy} folder when the container is created;
this means that when the container is started, all your projects are
deployed to the container. All configurations are also applied to the
container.

\noindent\hrulefill

\textbf{Note:} During your container's startup, you may run into the
following error:

\begin{verbatim}
/etc/liferay/entrypoint.sh: line 3:    11 Killed
${LIFERAY_HOME}/tomcat/bin/catalina.sh run
\end{verbatim}

This usually means you have not allocated enough memory to your Docker
engine to successfully run your container. See Docker's
\href{https://docs.docker.com}{documentation} to learn how to increase
resources available to Docker.

\noindent\hrulefill

Once your container is created, you can
\href{/docs/7-2/reference/-/knowledge_base/r/configuring-a-docker-container}{configure
it}.

\chapter{Configuring a Docker
Container}\label{configuring-a-docker-container}

{This document has been updated and ported to Liferay Learn and is no
longer maintained here.}

Before starting your container, you may want to add additional portal
configurations. This could include things like

\begin{itemize}
\tightlist
\item
  Property overrides (e.g., \texttt{portal-ext.properties})
\item
  Marketplace app overrides
\item
  App server configurations
\item
  License files
\end{itemize}

You can do this by applying files (and their accompanying folder
structures, if necessary) to your workspace's \texttt{configs/docker}
folder. This folder is treated as your Liferay Home for Docker
development; you add additional files that overlay your workspace's
\texttt{configs/common} folder and your Liferay DXP container's default
configuration.

As an example, you'll enable the
\href{/docs/7-2/customization/-/knowledge_base/c/using-the-felix-gogo-shell}{Gogo
shell} for your container.

\begin{enumerate}
\def\labelenumi{\arabic{enumi}.}
\item
  Add a \texttt{portal-ext.properties} file to your workspace's
  \texttt{configs/docker} folder.
\item
  Add the following property to the \texttt{portal-ext.properties} file:

\begin{verbatim}
module.framework.properties.osgi.console=0.0.0.0:11311
\end{verbatim}

  This lets you access your container using Gogo shell via telnet
  session.
\item
  Start the container.

  Once the container is started, the configurations stored in
  \texttt{configs/common} and \texttt{configs/docker} are transferred to
  the \texttt{build/docker/files} folder, which applies all
  configurations to the container's file system. For more information on
  workspace's \texttt{configs} folder, see
  \href{/docs/7-2/reference/-/knowledge_base/r/liferay-workspace\#testing-projects}{this
  section}.
\end{enumerate}

\noindent\hrulefill

\begin{verbatim}
 **Note:** You can call the `deployDocker` Gradle task from your
 workspace's root folder to initiate the Docker configuration transfer to
 the `build/docker/files` folder manually. It's executed automatically when
 creating or starting the container.
\end{verbatim}

\noindent\hrulefill

You can now apply configurations to your Liferay DXP Docker container.

\chapter{Building a Custom Docker
Image}\label{building-a-custom-docker-image}

{This document has been updated and ported to Liferay Learn and is no
longer maintained here.}

You can preserve your container's configuration by building it as an
image.

\begin{enumerate}
\def\labelenumi{\arabic{enumi}.}
\item
  Build your custom Liferay DXP image by running

\begin{verbatim}
./gradlew buildDockerImage
\end{verbatim}

  A \texttt{Dockerfile} is generated for your container when building
  your image. The \texttt{Dockerfile} is generated in your workspace's
  \texttt{build/docker} folder. For more information on how to configure
  the \texttt{Dockerfile}, see Docker's
  \href{https://docs.docker.com/engine/reference/builder/}{Dockerfile
  reference documentation}.

  You can generate a \texttt{Dockerfile} manually at any time by running

\begin{verbatim}
./gradlew createDockerfile
\end{verbatim}
\item
  Run \texttt{docker\ image\ ls} to verify the image's availability.
\end{enumerate}

You can now build Liferay DXP Docker images in Liferay Workspace!

\chapter{Updating Liferay Workspace}\label{updating-liferay-workspace}

{This document has been updated and ported to Liferay Learn and is no
longer maintained here.}

Liferay Workspace is continuously being updated with new features. If
you created your workspace a while ago, you may be missing out on some
of the latest features that could improve your Liferay DXP development
experience. Updating your Liferay Workspace is easy; you'll learn how to
do it for Gradle and Maven-based workspaces next.

\section{Gradle}\label{gradle-2}

\begin{enumerate}
\def\labelenumi{\arabic{enumi}.}
\item
  Find the latest Liferay Workspace version. To do this, view the
  Workspace Gradle plugin's
  \href{https://repository-cdn.liferay.com/nexus/content/repositories/liferay-public-releases/com/liferay/com.liferay.gradle.plugins.workspace/}{released
  versions} on Liferay's repository. Copy the version to which you want
  to upgrade.
\item
  Open your Liferay Workspace's \texttt{settings.gradle} file. This file
  resides in your Workspace's root folder.
\item
  In the \texttt{dependencies} block, you'll find code similar to below:

\begin{verbatim}
dependencies {
    classpath group: "com.liferay", name: "com.liferay.gradle.plugins.workspace", version: "[WORKSPACE_VERSION]"
}
\end{verbatim}

  Update the \texttt{com.liferay.gradle.plugins.workspace} dependency's
  \texttt{version} to the version number you copied in step 1.
\item
  Execute any Gradle command to initiate the update process for your
  Workspace (e.g., \texttt{blade\ gw\ tasks}).
\end{enumerate}

\noindent\hrulefill

\textbf{Note:} The Gradle wrapper provided in a Gradle-based Liferay
Workspace must be updated if you're migrating from a workspace before
version \texttt{1.10.14} to the latest available version. To update your
Gradle wrapper, run

\begin{verbatim}
./gradlew wrapper --gradle-version=4.10.2
\end{verbatim}

\noindent\hrulefill

Awesome! You've upgraded your Gradle-based Liferay Workspace!

\section{Maven}\label{maven-4}

\begin{enumerate}
\def\labelenumi{\arabic{enumi}.}
\item
  Find the latest Liferay Workspace version. To do this, view the Bundle
  Support plugin's
  \href{https://repository-cdn.liferay.com/nexus/content/repositories/liferay-public-releases/com/liferay/com.liferay.portal.tools.bundle.support/}{released
  versions} on Liferay's repository. Copy the version to which you want
  to upgrade.
\item
  Open your Liferay Workspace's root \texttt{pom.xml} file.
\item
  Within the \texttt{plugin} tags, you'll find code similar to below:

\begin{verbatim}
<plugin>
    <groupId>com.liferay</groupId>
    <artifactId>com.liferay.portal.tools.bundle.support</artifactId>
    <version>3.4.2</version>
    ...
</plugin>
\end{verbatim}

  Update the \texttt{com.liferay.portal.tools.bundle.support} artifact's
  \texttt{version} to the version number you copied in step 1.
\item
  Execute any Maven command to initiate the update process for your
  Workspace (e.g., \texttt{mvn\ verify}).
\end{enumerate}

Awesome! You've upgraded your Maven-based Liferay Workspace!

\chapter{Updating Default Plugins Provided by Liferay
Workspace}\label{updating-default-plugins-provided-by-liferay-workspace}

{This document has been updated and ported to Liferay Learn and is no
longer maintained here.}

Liferay Workspace comes with a slew of plugins like these:

\begin{itemize}
\tightlist
\item
  \href{https://github.com/liferay/liferay-portal/tree/master/modules/util/css-builder}{CSS
  Builder}
\item
  \href{https://github.com/liferay/liferay-portal/tree/master/modules/util/javadoc-formatter}{Javadoc
  Formatter}
\item
  \href{https://github.com/liferay/liferay-portal/tree/master/modules/util/lang-builder}{Lang
  Builder}
\item
  \href{https://github.com/liferay/liferay-portal/tree/master/modules/util/portal-tools-service-builder}{Service
  Builder}
\item
  \href{https://github.com/liferay/liferay-portal/tree/master/modules/util/source-formatter}{Source
  Formatter}
\item
  \href{https://github.com/liferay/liferay-portal/tree/master/modules/util/portal-tools-theme-builder}{Theme
  Builder}
\item
  etc.
\end{itemize}

Bundled plugins are updated with each release of workspace. Suppose you
need a new feature in the
\href{https://repository.liferay.com/nexus/content/repositories/liferay-public-releases/com/liferay/com.liferay.source.formatter/}{Source
Formatter plugin}, but the latest workspace version has not yet been
updated to include it. You can upgrade it yourself!

To upgrade one of workspace's bundled plugins, follow these steps:

\begin{enumerate}
\def\labelenumi{\arabic{enumi}.}
\item
  Find the bundle symbolic name (BSN) for the plugin you want to update.
  You can find this value in the
  \href{https://github.com/liferay/liferay-portal/blob/master/modules/sdk/gradle-plugins/src/main/resources/com/liferay/gradle/plugins/dependencies/portal-tools.properties}{\texttt{portal-tools.properties}}
  file. For example, the Source Formatter's BSN is
  \texttt{com.liferay.source.formatter}.
\item
  Open your workspace's \texttt{build.gradle} file and copy the plugin's
  BSN followed by \texttt{.version} and set the desired plugin version
  you want to use. For example,

\begin{verbatim}
com.liferay.source.formatter.version=1.0.819
\end{verbatim}

  If you're most interested in the latest and greatest plugins, you can
  set the above property to \texttt{latest.release} to always use the
  latest available version. This could, however, cause your workspace to
  become unstable.
\end{enumerate}

That's it! You're no longer tied to particular plugin versions provided
by your workspace.

\chapter{Maven}\label{maven-5}

\href{https://maven.apache.org/}{Maven} is a viable option for managing
Liferay projects if you don't want to use Liferay's default Gradle
management system. Liferay provides several
\href{/docs/7-2/reference/-/knowledge_base/r/maven-plugins}{Maven
plugins} for generating and managing your project. Liferay also provides
easy to obtain Maven artifacts that are required for Liferay Maven
module development. Here, you'll learn how to

\begin{itemize}
\tightlist
\item
  Install Liferay Maven artifacts
\item
  Create/Manage a Maven Repository
\item
  Apply Maven plugins
\end{itemize}

Because Liferay DXP is tool-agnostic, Maven is fully supported for
Liferay DXP development. Read on for details about these topics.

\section{Installing Liferay Maven
Artifacts}\label{installing-liferay-maven-artifacts}

To create Liferay projects using Maven, you'll need its dependencies.
This isn't a problem---Liferay provides them as Maven artifacts. You can
retrieve them from a remote repository.

There are two remote repositories that contain Liferay artifacts:
Central Repository and Liferay Repository. The Central Repository is the
default repository used to download artifacts if you don't have a remote
repository configured. Using the Central Repository to install Liferay
Maven artifacts only requires that you
\href{/docs/7-2/customization/-/knowledge_base/c/configuring-dependencies}{specify
your module's dependencies} in its \texttt{pom.xml} file.

When packaging your module, the automatic Maven artifact installation
process only downloads the artifacts necessary for that module from the
Central Repository.

The Central Repository \emph{usually} offers the latest Liferay Maven
artifacts, but the Liferay Repository \emph{guarantees} access to the
latest artifacts released by Liferay. Other than a slight delay in
artifact releases, the two repositories are identical. When the Liferay
repository is configured in your \texttt{settings.xml} file, archetypes
are generated based on that repository's contents. The Liferay Maven
repository offers a good alternative for those who want the most
up-to-date Maven artifacts produced by Liferay.

\noindent\hrulefill

\textbf{Note:} If you've configured the Liferay Nexus repository to
access Liferay Maven artifacts and you've already been syncing from the
Central Repository, you might have to clear out parts of your local
repository to force Maven to re-download the newer artifacts. Also,
don't leave the Liferay repository enabled when publishing artifacts to
Maven Central. You must comment out the Liferay Repository credentials
when publishing your artifacts.

\noindent\hrulefill

Next, you'll learn about managing your Maven artifacts.

\section{Managing Maven Artifacts in a
Repository}\label{managing-maven-artifacts-in-a-repository}

You can share Liferay artifacts and modules with teammates or manage
your repositories using a GUI by using
\href{http://www.sonatype.org/nexus/}{Sonatype Nexus}. It's a Maven
repository management server for creating and managing release servers,
snapshot servers, and proxy servers. There are several other Maven
repository management servers you can use (for example,
\href{https://www.jfrog.com/artifactory/}{Artifactory}), but this
section focuses on Nexus.

You'll learn how to

\begin{itemize}
\tightlist
\item
  \href{/docs/7-2/reference/-/knowledge_base/r/creating-a-maven-repository}{Create
  a repository}
\item
  \href{/docs/7-2/reference/-/knowledge_base/r/configuring-local-maven-settings-to-access-repositories}{Configure
  a repository}
\item
  \href{/docs/7-2/reference/-/knowledge_base/r/deploying-liferay-maven-artifacts-to-a-repository}{Deploy
  artifacts to a repository}
\end{itemize}

Before using repository servers, you must specify them in your Maven
environment settings. Your repository settings let Maven find the
repository and retrieve and install artifacts. You can configure your
local Maven settings in the \texttt{{[}USER\_HOME{]}/.m2/settings.xml}
file.

\noindent\hrulefill

\textbf{Note}: You must only configure a repository server if you're
sharing artifacts (e.g., Liferay artifacts and/or your modules) with
others. If you're installing Liferay artifacts from the Central/Liferay
Repository and aren't interested in sharing artifacts, you don't need a
repository server specified in your Maven settings. You can find out
more about installing artifacts from the Central Repository or Liferay's
own Nexus repository in the
\href{/docs/7-2/reference/-/knowledge_base/r/installing-remote-liferay-maven-artifacts}{Installing
Remote Liferay Maven Artifacts} article.

\noindent\hrulefill

To deploy to a remote repository, your Liferay project should be
packaged using Maven. Maven provides a packaging command that creates an
artifact (JAR) that can be easily deployed to your remote repository.

Once you've created a deployable artifact, you can configure your module
project to communicate with your remote repository and use Maven's
\texttt{deploy} command to send it on its way. Once your module project
resides on the remote repository, other developers can configure your
remote repository in their projects and set dependencies in their
project POMs to reference it.

\section{Applying Maven Plugins}\label{applying-maven-plugins}

There are several important Maven plugins that provide important
functionality to Liferay Maven projects. The available Liferay Maven
plugins are available in the
\href{/docs/7-2/reference/-/knowledge_base/r/maven-plugins}{Maven
Plugins} section.

The following tasks are covered in this section:

\begin{itemize}
\tightlist
\item
  \href{/docs/7-2/reference/-/knowledge_base/r/building-an-osgi-module-jar-with-maven}{Building
  an OSGi module JAR}
\item
  \href{/docs/7-2/reference/-/knowledge_base/r/building-a-theme-with-maven}{Building
  themes}
\item
  \href{/docs/7-2/reference/-/knowledge_base/r/compiling-sass-files-in-a-maven-project}{Compiling
  Sass files}
\item
  \href{/docs/7-2/reference/-/knowledge_base/r/using-service-builder-in-a-maven-project}{Using
  Service Builder}
\end{itemize}

Read on to learn more!

\chapter{Installing Liferay Maven
Artifacts}\label{installing-liferay-maven-artifacts-1}

If you haven't configured your project to retrieve artifacts from a
custom Maven repository, your project will leverage the artifacts from
the Central Repository. You can view these artifacts from the
\href{https://search.maven.org/}{Maven Central Repository} site. Use the
Latest Version column as a guide to see what's available for the version
of Liferay DXP you're developing for.

If you'd like to access Liferay's latest released Maven artifacts,
configure Maven to use
\href{https://repository-cdn.liferay.com}{Liferay's Nexus repository}
instead. To do this, open your project's parent \texttt{pom.xml} and add
this:

\begin{verbatim}
<repositories>
    <repository>
        <id>liferay-public-releases</id>
        <name>Liferay Public Releases</name>
        <url>https://repository-cdn.liferay.com/nexus/content/repositories/liferay-public-releases</url>
    </repository>
</repositories>

<pluginRepositories>
    <pluginRepository>
        <id>liferay-public-releases</id>
        <url>https://repository-cdn.liferay.com/nexus/content/repositories/liferay-public-releases/</url>
    </pluginRepository>
</pluginRepositories>
\end{verbatim}

The above configuration retrieves artifacts from Liferay's release
repository.

If you're most interested in retrieving Liferay's latest snapshot
artifacts, follow the instructions below to configure Liferay's Nexus
repository to access them.

\begin{enumerate}
\def\labelenumi{\arabic{enumi}.}
\item
  Open your project's parent \texttt{pom.xml} and add this:

\begin{verbatim}
<repositories>
    <repository>
        <id>liferay-public-snapshots</id>
        <name>Liferay Public Snapshots</name>
        <url>https://repository-cdn.liferay.com/nexus/content/repositories/liferay-public-snapshots</url>
    </repository>
</repositories>

<pluginRepositories>
    <pluginRepository>
        <id>liferay-public-snapshots</id>
        <url>https://repository-cdn.liferay.com/nexus/content/repositories/liferay-public-snapshots/</url>
    </pluginRepository>
</pluginRepositories>
\end{verbatim}
\item
  Enable your project to access snapshot artifacts by adding this code
  to your parent project's \texttt{pom.xml}:

\begin{verbatim}
<snapshots>
    <enabled>true</enabled>
</snapshots>
\end{verbatim}
\end{enumerate}

You're now equipped to access Liferay's Maven artifacts via the

\begin{itemize}
\tightlist
\item
  Central Repository
\item
  Liferay Repository (releases)
\item
  Liferay repository (snapshots)
\end{itemize}

Great job!

\chapter{Creating a Maven Repository}\label{creating-a-maven-repository}

To create a Maven repository using Nexus, download
\href{https://help.sonatype.com/display/NXRM2/Download}{Nexus} and
follow the instructions on Nexus'
\href{https://help.sonatype.com/display/NXRM2/Installing+and+Running}{Installation
page} to install and start it.

To create your own repository using Nexus, follow these steps:

\begin{enumerate}
\def\labelenumi{\arabic{enumi}.}
\item
  Open your web browser; navigate to your Nexus repository server (e.g.,
  \url{http://localhost:8081/nexus}) and log in. The default user name
  is \texttt{admin} with password \texttt{admin123}.
\item
  Click on \emph{Repositories} and navigate to \emph{Add\ldots{}} →
  \emph{Hosted Repository}.

  \begin{figure}
  \centering
  \includegraphics{./images/maven-nexus-create-repo.png}
  \caption{Adding a repository to hold your Liferay artifacts is easy
  with Nexus.}
  \end{figure}

  To learn more about each type of Nexus repository, read Sonatype's
  \href{http://books.sonatype.com/nexus-book/reference/confignx-sect-manage-repo.html}{Managing
  Repositories} guide.
\item
  Enter repository properties appropriate for the type of artifacts it
  will hold. If you're installing release version artifacts into the
  repository, specify \emph{Release} as the repository policy. Below are
  example repository property values:

  \begin{itemize}
  \tightlist
  \item
    \textbf{Repository ID:} \emph{liferay-releases}
  \item
    \textbf{Repository Name:} \emph{Liferay Release Repository}
  \item
    \textbf{Provider:} \emph{Maven2}
  \item
    \textbf{Repository Policy:} \emph{Release}
  \end{itemize}

  To create a snapshot repository, choose \emph{Snapshot} for the
  Repository Policy and update the ID and name accordingly.
\item
  Click \emph{Save}.
\end{enumerate}

Voila! You've created a repository for your Liferay releases (i.e.,
\texttt{liferay-releases}) and/or Liferay snapshots (i.e.,
\texttt{liferay-snapshots}). To learn how to deploy your Liferay Maven
artifacts to a Nexus repository, see the
\href{/docs/7-2/reference/-/knowledge_base/r/deploying-liferay-maven-artifacts-to-a-repository}{Deploying
Liferay Maven Artifacts to a Repository} tutorial.

See the
\href{/docs/7-2/reference/-/knowledge_base/r/configuring-local-maven-settings-to-access-repositories}{Configuring
Local Maven Settings to Access Repositories} to configure your new
repository servers in your Maven settings to install artifacts to them.

\chapter{Configuring Local Maven Settings to Access
Repositories}\label{configuring-local-maven-settings-to-access-repositories}

To configure your Maven environment to access your repository servers,
do the following:

\begin{enumerate}
\def\labelenumi{\arabic{enumi}.}
\item
  Navigate to your \texttt{{[}USER\_HOME{]}/.m2/settings.xml} file.
  Create it if it doesn't yet exist.
\item
  Configure your repository server settings. Here are contents from a
  \texttt{settings.xml} file that has \texttt{liferay-releases} and
  \texttt{liferay-snapshots} repository servers configured:

\begin{verbatim}
<?xml version="1.0"?>
<settings>
    <servers>
        <server>
            <id>liferay-releases</id>
            <username>admin</username>
            <password>admin123</password>
        </server>
        <server>
            <id>liferay-snapshots</id>
            <username>admin</username>
            <password>admin123</password>
        </server>
    </servers>
</settings>
\end{verbatim}

  The user name \texttt{admin} and password \texttt{admin123} are the
  credentials of the default Nexus administrator account. If you changed
  these credentials for your Nexus server, make sure to update
  \texttt{settings.xml} with these changes.
\end{enumerate}

Now that your repositories are configured, they're ready to receive all
the Liferay Maven artifacts you'll download and the Liferay module
artifacts you'll create!

\chapter{Deploying Liferay Maven Artifacts to a
Repository}\label{deploying-liferay-maven-artifacts-to-a-repository}

Deploying artifacts to a remote repository is important if you intend to
share your Maven projects with others. First, you must have a remote
repository that can hold deployed Maven artifacts. If you do not
currently have a remote repository,
\href{/docs/7-2/reference/-/knowledge_base/r/creating-a-maven-repository}{create
one}. Also make sure your \texttt{{[}USER\_HOME{]}/.m2/settings.xml}
file specifies your remote repository's ID, user name, and password
(configuration instructions
\href{/docs/7-2/reference/-/knowledge_base/r/configuring-local-maven-settings-to-access-repositories}{here}).

To follow this article, you'll need a Liferay module built with Maven.
For demonstration purposes, this tutorial uses the \texttt{portlet.ds}
sample module project. To follow along with this module, download the
\href{https://portal.liferay.dev/documents/113763090/114000186/portlet.ds.zip}{portlet.ds}
Zip.

Now it's time to deploy a Maven artifact to your Nexus repository.

\begin{enumerate}
\def\labelenumi{\arabic{enumi}.}
\item
  Create a folder anywhere on your machine to serve as the parent folder
  for your Liferay modules. Unzip the \texttt{portlet.ds} module project
  into that folder.
\item
  Create a \texttt{pom.xml} file inside this folder. Copy the following
  logic into the parent POM:

\begin{verbatim}
<?xml version="1.0" encoding="UTF-8"?>
<project
    xmlns="http://maven.apache.org/POM/4.0.0"
    xmlns:xsi="http://www.w3.org/2001/XMLSchema-instance"
    xsi:schemaLocation="http://maven.apache.org/POM/4.0.0 http://maven.apache.org/xsd/maven-4.0.0.xsd"
>

    <modelVersion>4.0.0</modelVersion>
    <groupId>liferay.sample</groupId>
    <artifactId>liferay.sample.maven</artifactId>
    <version>1.0.0</version>
    <name>Liferay Maven Module Projects</name>
    <packaging>pom</packaging>

    <distributionManagement>
        <repository>
            <id>liferay-releases</id>
            <url>http://localhost:8081/nexus/content/repositories/liferay-releases</url>
        </repository>
    </distributionManagement>

    <modules>
        <module>portlet.ds</module>
    </modules>
</project>
\end{verbatim}

  The tags \texttt{\textless{}modelVersion\textgreater{}} through
  \texttt{\textless{}packaging\textgreater{}} are POM tags that are used
  frequently in parent POMs. Visit Maven's
  \href{https://maven.apache.org/pom.html}{POM Reference} documentation
  for more information.

  The \texttt{\textless{}distributionManagement\textgreater{}} tag
  specifies the deployment repository for all module projects residing
  in the parent folder. This repository should also be specified in your
  \texttt{{[}USER\_HOME{]}/.m2/settings.xml}. Both the parent POM and
  \texttt{settings.xml} file's repository declarations are required to
  deploy your modules to that remote repository.

  Finally, you must list the modules residing in the parent folder that
  you want deployed using the \texttt{\textless{}modules\textgreater{}}
  tag. The \texttt{portlet.ds} module is specified within that tag.
\item
  Open the \texttt{portlet.ds} module's \texttt{pom.xml} file. If you
  did not download the \texttt{portlet.ds} module project Zip, you can
  reference its POM below.

\begin{verbatim}
<project
    xmlns="http://maven.apache.org/POM/4.0.0"
    xmlns:xsi="http://www.w3.org/2001/XMLSchema-instance"
    xsi:schemaLocation="http://maven.apache.org/POM/4.0.0 http://maven.apache.org/xsd/maven-4.0.0.xsd"
>

    <modelVersion>4.0.0</modelVersion>
    <artifactId>portlet.ds</artifactId>
    <version>1.0.0</version>
    <packaging>jar</packaging>

    <parent>
        <groupId>liferay.sample</groupId>
        <artifactId>liferay.sample.maven</artifactId>
        <version>1.0.0</version>
        <relativePath>../pom.xml</relativePath>
    </parent>

    <dependencies>
        <dependency>
            <groupId>javax.portlet</groupId>
            <artifactId>portlet-api</artifactId>
            <version>2.0</version>
            <scope>provided</scope>
        </dependency>
        <dependency>
            <groupId>org.osgi</groupId>
            <artifactId>org.osgi.service.component.annotations</artifactId>
            <version>1.3.0</version>
            <scope>provided</scope>
        </dependency>
    </dependencies>
</project>
\end{verbatim}

  The \texttt{portlet.ds} module's POM specifies its own attributes
  first, followed by the parent POM's attributes. Declaring the
  \texttt{\textless{}parent\textgreater{}} tag like above links the
  \texttt{portlet.ds} module to its parent POM, which is necessary to
  deploy to the remote repository. Then the module's dependencies are
  listed. These dependencies are downloaded from the Central Repository
  and installed to your local \texttt{.m2} repository when you package
  the \texttt{portlet.ds} module.
\item
  Now that you've configured your parent POM and module POM, package
  your Maven project. Navigate to your module project (e.g.,
  \texttt{project.ds}) using the command line and run the Maven package
  command:

\begin{verbatim}
mvn package
\end{verbatim}

  This downloads and installs all your module's dependencies and
  packages the project into a JAR file. Navigate to your module
  project's generated build folder (e.g., \texttt{/target}). You'll
  notice there is a newly generated JAR file. This is the artifact
  you'll deploy to your Nexus repository.
\item
  Run Maven's deploy command to deploy your module project's artifact to
  your configured remote repository.

\begin{verbatim}
mvn deploy
\end{verbatim}

  Your console shows output from the artifact being deployed into your
  repository server.
\end{enumerate}

To verify that your artifact is deployed, navigate to the
\emph{Repositories} page of your Nexus server and select your
repository. A window appears below showing the Liferay artifact now
deployed to your repository.

\begin{figure}
\centering
\includegraphics{./images/maven-verify-deployment.png}
\caption{Your repository server now provides access to your Liferay
Maven artifacts.}
\end{figure}

Awesome! You can now share your Liferay module projects with anyone by
deploying them as artifacts to your remote repository!

\chapter{Building an OSGi Module JAR with
Maven}\label{building-an-osgi-module-jar-with-maven}

If you have an existing Liferay module built with Maven that you created
from scratch, or you're upgrading your Maven project from a previous
version of Liferay DXP, your project probably can't generate an
executable OSGi JAR. Don't fret! You can do this by making a few minor
configurations in your module's POMs.

\noindent\hrulefill

\textbf{Note:} If you used Liferay's Maven archetypes to generate your
module project, the project already has the Maven plugins required to
generate an OSGi JAR.

\noindent\hrulefill

Continue on to see how this is done.

\begin{enumerate}
\def\labelenumi{\arabic{enumi}.}
\item
  In your project's \texttt{pom.xml} file, add the
  \href{http://njbartlett.name/2015/03/27/announcing-bnd-maven-plugin.html}{BND
  Maven Plugin} declaration:

\begin{verbatim}
<plugin>
    <groupId>biz.aQute.bnd</groupId>
    <artifactId>bnd-maven-plugin</artifactId>
    <version>3.3.0</version>
    <executions>
        <execution>
            <goals>
                <goal>bnd-process</goal>
            </goals>
        </execution>
    </executions>
    <dependencies>
        <dependency>
            <groupId>biz.aQute.bnd</groupId>
            <artifactId>biz.aQute.bndlib</artifactId>
            <version>3.2.0</version>
        </dependency>
        <dependency>
            <groupId>com.liferay</groupId>
            <artifactId>com.liferay.ant.bnd</artifactId>
            <version>2.0.41</version>
        </dependency>
    </dependencies>
</plugin>
\end{verbatim}

  The BND Maven plugin prepares all your Maven module's resources (e.g.,
  \texttt{MANIFEST.MF}) and inserts them into the generated
  \texttt{{[}Maven\ Project{]}/target/classes} folder. This plugin
  prepares your module to be packaged as an OSGi JAR deployable to
  Liferay DXP.
\end{enumerate}

\noindent\hrulefill

\begin{verbatim}
 **Note:** Although WABs can be generated using the `bnd-maven-plugin`,
 this is not supported by Liferay. WABs should be created as a standard WAR
 project and deployed to the
 [Liferay WAB Generator](/docs/7-2/customization/-/knowledge_base/c/deploying-wars-wab-generator),
 which generates a WAB for you.
\end{verbatim}

\noindent\hrulefill

\begin{enumerate}
\def\labelenumi{\arabic{enumi}.}
\setcounter{enumi}{1}
\item
  In your project's \texttt{pom.xml} file, add the
  \href{http://maven.apache.org/plugins/maven-jar-plugin/}{Maven JAR
  Plugin} declaration:

\begin{verbatim}
<build>
    <plugins>
        <plugin>
            <groupId>org.apache.maven.plugins</groupId>
            <artifactId>maven-jar-plugin</artifactId>
            <version>2.6</version>
            <configuration>
                <archive>
                    <manifestFile>${project.build.outputDirectory}/META-INF/MANIFEST.MF</manifestFile>
                </archive>
            </configuration>
        </plugin>
    </plugins>
</build>
\end{verbatim}

  The Maven JAR plugin builds your Maven project as a JAR file,
  including the resources generated by the BND Maven plugin. The above
  configuration also sets the default project \texttt{MANIFEST.MF} file
  path for your project, which is essential when packaging your module
  using the BND Maven plugin. By default, the Maven JAR Plugin ignores
  the \texttt{target/classes/META-INF/MANIFEST.MF} generated by the BND
  Maven plugin, so you must explicitly set it as the manifest file so
  it's included properly in the generated JAR file.
\item
  Add a \href{http://bnd.bndtools.org/}{\texttt{bnd.bnd} file} to your
  Liferay Maven project, residing in the same folder as your project's
  \texttt{pom.xml} file.
\item
  Build your Maven OSGi JAR by running

\begin{verbatim}
mvn package
\end{verbatim}

  Your Maven JAR is generated in your project's \texttt{/target} folder.
  You can deploy it manually into Liferay DXP's \texttt{/deploy} folder,
  or you can configure your project to deploy automatically to Liferay
  DXP by following the
  \href{/docs/7-2/reference/-/knowledge_base/r/deploying-a-project\#maven}{Deploying
  a Project} article.
\end{enumerate}

Fantastic! You've configured your Liferay Maven project to package
itself into a deployable OSGi module.

\chapter{Building a Theme with Maven}\label{building-a-theme-with-maven}

Liferay's Theme Builder is used to build Liferay DXP theme files in your
project. You can incorporate the Theme Builder into your Maven project
to generate WAR-style
\href{/docs/7-2/frameworks/-/knowledge_base/f/themes-introduction}{themes}
deployable to Liferay DXP.

The easiest way to create a Liferay theme with Maven is to create a new
Maven project using Liferay's provided
\href{/docs/7-2/reference/-/knowledge_base/r/theme-template}{Theme
archetype}; Theme Builder is configured in the new project by default.
In some cases, however, this may not be convenient. For instance, if you
have a legacy theme project and don't want to start over, generating a
new project is not ideal.

For cases like this, you should manually configure your Maven project to
leverage Theme Builder. You'll learn how to do this next.

\begin{enumerate}
\def\labelenumi{\arabic{enumi}.}
\item
  Configure Liferay's
  \href{/docs/7-2/reference/-/knowledge_base/r/theme-builder-plugin}{Theme
  Builder} plugin in your project's \texttt{pom.xml} file:

\begin{verbatim}
<build>
    <plugins>
        <plugin>
            <groupId>com.liferay</groupId>
            <artifactId>com.liferay.portal.tools.theme.builder</artifactId>
            <version>1.1.7</version>
            <executions>
                <execution>
                    <phase>generate-resources</phase>
                    <goals>
                        <goal>build</goal>
                    </goals>
                    <configuration>
                        <diffsDir>${maven.war.src}</diffsDir>
                        <name>${project.artifactId}</name>
                        <outputDir>${project.build.directory}/${project.build.finalName}</outputDir>
                        <parentDir>${project.build.directory}/deps/com.liferay.frontend.theme.styled.jar</parentDir>
                        <parentName>_styled</parentName>
                        <templateExtension>ftl</templateExtension>
                        <unstyledDir>${project.build.directory}/deps/com.liferay.frontend.theme.unstyled.jar</unstyledDir>
                    </configuration>
                </execution>
            </executions>
        </plugin>
    </plugins>
</build>
\end{verbatim}

  The above configuration applies the Theme Builder plugin and then
  defines the Theme Builder's execution and configuration.

  \begin{itemize}
  \tightlist
  \item
    The
    \href{https://maven.apache.org/guides/mini/guide-configuring-plugins.html\#Using_the_executions_Tag}{executions}
    tag configures the Theme Builder to run during the
    \texttt{generate-resources} phase of your Maven project's build
    lifecycle. The \texttt{build}
    \href{http://maven.apache.org/guides/introduction/introduction-to-the-lifecycle.html\#A_Build_Phase_is_Made_Up_of_Plugin_Goals}{goal}
    is defined for that lifecycle phase.
  \item
    The \href{https://maven.apache.org/pom.html\#Plugins}{configuration}
    defines tag several important properties. For more info on these
    properties, see the
    \href{/docs/7-2/reference/-/knowledge_base/r/theme-builder-plugin}{Theme
    Builder Plugin} article.
  \end{itemize}
\item
  Apply the CSS Builder plugin, which is required to use Theme Builder:

\begin{verbatim}
<plugin>
    <groupId>com.liferay</groupId>
    <artifactId>com.liferay.css.builder</artifactId>
    <version>2.1.3</version>
    <executions>
        <execution>
            <id>default-build</id>
            <phase>compile</phase>
            <goals>
                <goal>build</goal>
            </goals>
        </execution>
    </executions>
    <configuration>
        <docrootDirName>target/${project.build.finalName}</docrootDirName>
        <outputDirName>/</outputDirName>
        <portalCommonPath>target/deps/com.liferay.frontend.css.common.jar</portalCommonPath>
    </configuration>
</plugin>
\end{verbatim}

  You can learn more about the CSS Builder's Maven configuration by
  visiting the
  \href{/docs/7-2/reference/-/knowledge_base/r/compiling-sass-files-in-a-maven-project}{Compiling
  Sass Files in a Maven Project} tutorial.
\item
  You can configure your project to exclude Sass files from being
  packaged in your theme. This is optional, but is a nice convenience to
  keep any unnecessary source code out of your WAR. Since the Theme
  Builder creates a WAR-style theme, you should apply the
  \href{https://maven.apache.org/plugins/maven-war-plugin/}{maven-war-plugin}
  so it instructs the WAR file packaging process to exclude Sass files:

\begin{verbatim}
<plugin>
    <artifactId>maven-war-plugin</artifactId>
    <version>3.0.0</version>
    <configuration>
        <packagingExcludes>**/*.scss</packagingExcludes>
    </configuration>
</plugin>
\end{verbatim}
\item
  Insert the \texttt{\textless{}packaging\textgreater{}} tag in your
  project's POM so your project is correctly packaged as a WAR file.
  This tag can be placed with your project's \texttt{groupId},
  \texttt{artifactId}, and \texttt{version} specifications like this:

\begin{verbatim}
<groupId>com.liferay</groupId>
<artifactId>com.liferay.project.templates.theme</artifactId>
<version>1.0.0</version>
<packaging>war</packaging>
\end{verbatim}
\item
  Building themes requires certain dependencies. You can
  \href{/docs/7-2/customization/-/knowledge_base/c/configuring-dependencies}{configure
  these dependenices} in your project's \texttt{pom.xml} as directories
  or JAR files. If you choose to use JARs, you must apply the
  \href{http://maven.apache.org/plugins/maven-dependency-plugin/}{maven-dependency-plugin}
  and have it copy JAR dependencies into your project from Maven
  Central:

\begin{verbatim}
<plugin>
    <artifactId>maven-dependency-plugin</artifactId>
    <executions>
        <execution>
            <phase>generate-sources</phase>
            <goals>
                <goal>copy</goal>
            </goals>
            <configuration>
                <artifactItems>
                    <artifactItem>
                        <groupId>com.liferay</groupId>
                        <artifactId>com.liferay.frontend.css.common</artifactId>
                        <version>${com.liferay.frontend.css.common.version}</version>
                    </artifactItem>
                    <artifactItem>
                        <groupId>com.liferay</groupId>
                        <artifactId>com.liferay.frontend.theme.styled</artifactId>
                        <version>${com.liferay.frontend.theme.styled.version}</version>
                    </artifactItem>
                    <artifactItem>
                        <groupId>com.liferay</groupId>
                        <artifactId>com.liferay.frontend.theme.unstyled</artifactId>
                        <version>${com.liferay.frontend.theme.unstyled.version}</version>
                    </artifactItem>
                </artifactItems>
                <outputDirectory>${project.build.directory}/deps</outputDirectory>
                <stripVersion>true</stripVersion>
            </configuration>
        </execution>
    </executions>
</plugin>
\end{verbatim}

  This configuration copies the
  \texttt{com.liferay.frontend.css.common},
  \texttt{com.liferay.frontend.theme.styled}, and
  \texttt{com.liferay.frontend.theme.unstyled} dependencies into your
  Maven project. Notice that you've set the \texttt{stripVersion} tag to
  \texttt{true} and you're setting the artifact versions within each
  \texttt{artifactItem} tag. You'll set these versions and a few other
  properties for your Maven project next.
\item
  Configure the properties for your project in its \texttt{pom.xml}
  file:

\begin{verbatim}
<properties>
    <com.liferay.css.builder.version>2.1.3</com.liferay.css.builder.version>
    <com.liferay.frontend.css.common.version>2.0.4</com.liferay.frontend.css.common.version>
    <com.liferay.frontend.theme.styled.version>2.0.28</com.liferay.frontend.theme.styled.version>
    <com.liferay.frontend.theme.unstyled.version>2.2.5</com.liferay.frontend.theme.unstyled.version>
    <com.liferay.portal.tools.theme.builder.version>1.1.7</com.liferay.portal.tools.theme.builder.version>
</properties>
\end{verbatim}

  The properties above set the versions for the CSS and Theme Builder
  plugins and their dependencies.
\end{enumerate}

You've successfully configured your Maven project to build a Liferay
theme with Theme Builder!

\chapter{Compiling Sass Files in a Maven
Project}\label{compiling-sass-files-in-a-maven-project}

If your Liferay Maven project uses Sass files to style its UI, you must
configure the project to convert its Sass files into CSS files so they
are recognizable for Maven's build lifecycle. It would be a real pain to
convert your Sass files into CSS files manually before building your
Maven project!

Liferay provides the
\href{/docs/7-2/reference/-/knowledge_base/r/css-builder-plugin}{CSS
Builder} plugin, which converts Sass files into CSS files so the Maven
build can parse your style sheets.

Here's how to apply Liferay's CSS builder to your Maven project.

\begin{enumerate}
\def\labelenumi{\arabic{enumi}.}
\item
  Open your project's \texttt{pom.xml} file and apply Liferay's CSS
  Builder:

\begin{verbatim}
<plugin>
    <groupId>com.liferay</groupId>
    <artifactId>com.liferay.css.builder</artifactId>
    <version>2.1.0</version>
    <executions>
        <execution>
            <id>default-build</id>
            <phase>compile</phase>
            <goals>
                <goal>build</goal>
            </goals>
        </execution>
    </executions>
    <configuration>
        <docrootDirName>${com.liferay.portal.tools.theme.builder.outputDir}</docrootDirName>
        <outputDirName>/</outputDirName>
        <portalCommonPath>target/deps/com.liferay.frontend.css.common.jar</portalCommonPath>
    </configuration>
</plugin>
\end{verbatim}

  The above configuration applies the CSS Builder and then defines the
  CSS Builder's execution and configuration.

  \begin{itemize}
  \tightlist
  \item
    The
    \href{https://maven.apache.org/guides/mini/guide-configuring-plugins.html\#Using_the_executions_Tag}{executions}
    tag configures the CSS Builder to run during the \texttt{compile}
    phase of your Maven project's build lifecycle. The \texttt{build}
    \href{http://maven.apache.org/guides/introduction/introduction-to-the-lifecycle.html\#A_Build_Phase_is_Made_Up_of_Plugin_Goals}{goal}
    is defined for that lifecycle phase.
  \item
    The \href{https://maven.apache.org/pom.html\#Plugins}{configuration}
    tag defines two important properties. For more info on these
    properties, see the
    \href{/docs/7-2/reference/-/knowledge_base/r/css-builder-plugin}{CSS
    Builder Plugin} article.
  \end{itemize}
\item
  If you're using \href{http://bourbon.io/}{Bourbon} in your Sass files,
  you'll need to
  \href{/docs/7-2/customization/-/knowledge_base/c/configuring-dependencies}{add
  an additional plugin dependency} to your project's POM. If you're not
  using Bourbon, skip this step. Add the following plugin dependency:

\begin{verbatim}
<plugin>
    <artifactId>maven-dependency-plugin</artifactId>
    <executions>
        <execution>
            <phase>generate-sources</phase>
            <goals>
                <goal>copy</goal>
            </goals>
            <configuration>
                <artifactItems>
                    <artifactItem>
                        <groupId>com.liferay</groupId>
                        <artifactId>com.liferay.frontend.css.common</artifactId>
                        <version>2.0.4</version>
                    </artifactItem>
                </artifactItems>
                <outputDirectory>${project.build.directory}/deps</outputDirectory>
                <stripVersion>true</stripVersion>
            </configuration>
        </execution>
    </executions>
</plugin>
\end{verbatim}

  The
  \href{http://maven.apache.org/plugins/maven-dependency-plugin/}{maven-dependency-plugin}
  copies the \texttt{com.liferay.frontend.css.common} dependency from
  Maven Central to your project's build folder so the CSS Builder can
  leverage it.
\item
  Use this command to compile your Maven project's Sass files:

\begin{verbatim}
mvn compile
\end{verbatim}
\end{enumerate}

\noindent\hrulefill

\textbf{Note:} Liferay's CSS Builder is supported for Oracle's JDK and
uses a native compiler for increased speed. If you're using an IBM JDK,
you may experience issues when building your Sass files (e.g., when
building a theme). It's recommended to switch to using the Oracle JDK,
but if you prefer using the IBM JDK, you must use the fallback Ruby
compiler. To do this, add the following tag to your CSS Builder
configuration in your POM:

\begin{verbatim}
<sassCompilerClassName>ruby</sasscompilerClassName>
\end{verbatim}

Be aware that the Ruby-based compiler doesn't perform as well as the
native compiler, so expect longer compile times.

\noindent\hrulefill

Awesome! You can now compile Sass files in your Liferay Maven project.

\chapter{Using Service Builder in a Maven
Project}\label{using-service-builder-in-a-maven-project}

The easiest way to add
\href{/docs/7-2/appdev/-/knowledge_base/a/service-builder}{Service
Builder} to your Maven project is to create a new Maven project using
Liferay's provided Service Builder archetype; it's configured in the new
project by default. You can learn how to generate a Service Builder
Maven project by visiting the
\href{/docs/7-2/reference/-/knowledge_base/r/using-the-service-builder-template}{Service
Builder Template} article.

In some cases, you should not use this template due to a number of
reasons:

\begin{itemize}
\tightlist
\item
  You're updating a legacy Maven project to follow OSGi modular
  architecture.
\item
  You have a pre-existing modular Maven project and don't want to start
  over.
\end{itemize}

For these cases, you can configure Service Builder in your project
manually. Follow the instructions below to configure Service Builder in
your Maven project!

\begin{enumerate}
\def\labelenumi{\arabic{enumi}.}
\item
  Apply the Service Builder plugin in your Maven project's
  \texttt{pom.xml} file:

\begin{verbatim}
<build>
    <plugins>
        <plugin>
            <groupId>com.liferay</groupId>
            <artifactId>com.liferay.portal.tools.service.builder</artifactId>
            <version>1.0.276</version>
            <configuration>
                <apiDirName>../basic-api/src/main/java</apiDirName>
                <autoImportDefaultReferences>true</autoImportDefaultReferences>
                <autoNamespaceTables>true</autoNamespaceTables>
                <buildNumberIncrement>true</buildNumberIncrement>
                <hbmFileName>src/main/resources/META-INF/module-hbm.xml</hbmFileName>
                <implDirName>src/main/java</implDirName>
                <inputFileName>service.xml</inputFileName>
                <modelHintsFileName>src/main/resources/META-INF/portlet-model-hints.xml</modelHintsFileName>
                <mergeModelHintsConfigs>src/main/resources/META-INF/portlet-model-hints.xml</mergeModelHintsConfigs>
                <osgiModule>true</osgiModule>
                <propsUtil>com.liferay.blade.samples.servicebuilder.service.util.PropsUtil</propsUtil>
                <resourcesDirName>src/main/resources</resourcesDirName>
                <springFileName>src/main/resources/META-INF/spring/module-spring.xml</springFileName>
                <springNamespaces>beans,osgi</springNamespaces>
                <sqlDirName>src/main/resources/META-INF/sql</sqlDirName>
                <sqlFileName>tables.sql</sqlFileName>
                <testDirName>src/main/test</testDirName>
            </configuration>
        </plugin>
    </plugins>
</build>
\end{verbatim}

  The \texttt{configuration} tag used above is an example of what a
  Service Builder project's configuration could look like. All the
  configuration attributes above should be modified to fit your project.

  The above code configures Service Builder for a sample
  \texttt{basic-service} module. When running Service Builder with this
  configuration, the project's API classes are generated in the
  \texttt{basic-api} module's \texttt{src/main/java} folder. You can
  also specify your hibernate module mappings, implementation directory
  name, model hints file, Spring configurations, SQL configurations,
  etc. You can reference all the configurable Service Builder properties
  in the
  \href{/docs/7-2/reference/-/knowledge_base/r/service-builder-plugin}{Service
  Builder Plugin} reference article.
\item
  Apply the WSDD Builder plugin declaration directly after the Service
  Builder plugin declaration:

\begin{verbatim}
<plugin>
    <groupId>com.liferay</groupId>
    <artifactId>com.liferay.portal.tools.wsdd.builder</artifactId>
    <version>1.0.10</version>
    <configuration>
        <inputFileName>service.xml</inputFileName>
        <outputDirName>src/main/java</outputDirName>
        <serverConfigFileName>src/main/resources/server-config.wsdd</serverConfigFileName>
    </configuration>
</plugin>
\end{verbatim}

  The WSDD Builder is required to generate your project's remote
  services.

  See the
  \href{/docs/7-2/reference/-/knowledge_base/r/wsdd-builder-plugin}{WSDD
  Builder Plugin} article for more information on the configuration
  properties.
\end{enumerate}

Terrific! You've successfully configured your Maven project to use
Service Builder by applying the Service Builder and WSDD Builder plugins
in your project's POM.

\chapter{Upgrading Your Maven Build
Environment}\label{upgrading-your-maven-build-environment}

\noindent\hrulefill

\textbf{Note:} This upgrade article only applies to projects residing in
a pre Liferay Portal 7.0 Maven environment that are not upgrading to
Liferay Workspace. If you're interested in upgrading to a Maven-based
Liferay Workspace (recommended), see the
\href{/docs/7-2/tutorials/-/knowledge_base/t/upgrading-code-to-product-ver}{Upgrading
Code to 7.0} tutorials for more information.

\noindent\hrulefill

If you're an avid Maven user and have been using it for Liferay Portal
6.2 project development or older, you must upgrade your Maven build to
be compatible with 7.0 development. There are two main parts of the
Maven environment upgrade process that you must address:

\begin{itemize}
\tightlist
\item
  \hyperref[upgrading-to-new-product-ver-maven-plugins]{Upgrading to new
  7.0 Maven plugins}
\item
  \hyperref[updating-liferay-maven-artifact-dependencies]{Updating
  Liferay Maven artifact dependencies}
\end{itemize}

For more information on using Maven with 7.0, see the
\href{/docs/7-2/reference/-/knowledge_base/r/maven}{Maven} section.

Liferay also offers a Maven development environment tailored
specifically for 7.0 development. Learn more about this in the
\href{/docs/7-2/reference/-/knowledge_base/r/liferay-workspace}{Liferay
Workspace} section.

You'll start off by upgrading your Maven environment's Liferay Maven
plugins.

\section{Upgrading to New 7.0 Maven
Plugins}\label{upgrading-to-new-7.0-maven-plugins}

The biggest change for your project's build plugins is the removal of
the \texttt{liferay-maven-plugin}. Liferay now provides several
individual Maven plugins that accomplish specific tasks. For example,
you can configure Maven plugins for Liferay's CSS Builder, Service
Builder, Theme Builder, etc. With smaller plugins available to
accomplish specific tasks in your project, you no longer have to rely on
one large plugin that provides many things you may not want.

For example, suppose your Liferay Portal 6.2 project was using the
\texttt{liferay-maven-plugin} for Liferay CSS Builder only. First, you
should remove the legacy \texttt{liferay-maven-plugin} plugin dependency
from your project's \texttt{pom.xml} file:

\begin{verbatim}
<plugin>
    <groupId>com.liferay.maven.plugins</groupId>
    <artifactId>liferay-maven-plugin</artifactId>
    <version>${liferay.version}</version>
    <configuration>
        <autoDeployDir>${liferay.auto.deploy.dir}</autoDeployDir>
        <liferayVersion>${liferay.version}</liferayVersion>
        <pluginType>portlet</pluginType>
    </configuration>
</plugin>
\end{verbatim}

Then, add the CSS Builder plugin dependency to your project's
\texttt{pom.xml} file:

\begin{verbatim}
<plugin>
    <groupId>com.liferay</groupId>
    <artifactId>com.liferay.css.builder</artifactId>
    <version>2.1.3</version>
    <executions>
        <execution>
            <id>default-build</id>
            <phase>generate-sources</phase>
            <goals>
                <goal>build</goal>
            </goals>
        </execution>
    </executions>
        <configuration>
            <docrootDirName>src/main/webapp</docrootDirName>
        </configuration>
</plugin>
\end{verbatim}

Some common Liferay Maven plugins are listed below, with their
corresponding artifact IDs and articles explaining how to configure
them:

\textbf{Common Liferay Maven Plugins}

\begin{longtable}[]{@{}
  >{\raggedright\arraybackslash}p{(\columnwidth - 4\tabcolsep) * \real{0.2083}}
  >{\raggedright\arraybackslash}p{(\columnwidth - 4\tabcolsep) * \real{0.4583}}
  >{\raggedright\arraybackslash}p{(\columnwidth - 4\tabcolsep) * \real{0.3333}}@{}}
\toprule\noalign{}
\begin{minipage}[b]{\linewidth}\raggedright
Name
\end{minipage} & \begin{minipage}[b]{\linewidth}\raggedright
Artifact ID
\end{minipage} & \begin{minipage}[b]{\linewidth}\raggedright
Tutorial
\end{minipage} \\
\midrule\noalign{}
\endhead
\bottomrule\noalign{}
\endlastfoot
Bundle Support &
\href{https://search.maven.org/search?q=com.liferay.portal.tools.bundle.support}{com.liferay.portal.tools.bundle.support}
&
\href{/docs/7-2/reference/-/knowledge_base/r/bundle-support-plugin}{Bundle
Support Plugin} \\
CSS Builder &
\href{https://search.maven.org/search?q=com.liferay.css.builder}{com.liferay.css.builder}
& \href{/docs/7-2/reference/-/knowledge_base/r/css-builder-plugin}{CSS
Builder Plugin} \\
DB Support &
\href{https://search.maven.org/search?q=com.liferay.portal.tools.db.support}{com.liferay.portal.tools.db.support}
& \href{/docs/7-2/reference/-/knowledge_base/r/db-support-plugin}{DB
Support Plugin} \\
Deployment Helper &
\href{https://search.maven.org/search?q=com.liferay.deployment.helper}{com.liferay.deployment.helper}
&
\href{/docs/7-2/reference/-/knowledge_base/r/deployment-helper-plugin}{Deployment
Helper Plugin} \\
Javadoc Formatter &
\href{https://search.maven.org/search?q=com.liferay.javadoc.formatter}{com.liferay.javadoc.formatter}
&
\href{/docs/7-2/reference/-/knowledge_base/r/javadoc-formatter-plugin}{Javadoc
Formatter Plugin} \\
Lang Builder &
\href{https://search.maven.org/search?q=com.liferay.lang.builder}{com.liferay.lang.builder}
& \href{/docs/7-2/reference/-/knowledge_base/r/lang-builder-plugin}{Lang
Builder Plugin} \\
REST Builder &
\href{https://search.maven.org/search?q=com.liferay.portal.tools.rest.builder}{com.liferay.portal.tools.rest.builder}
& \href{/docs/7-2/reference/-/knowledge_base/r/rest-builder-plugin}{REST
Builder Plugin} \\
Service Builder &
\href{https://search.maven.org/search?q=com.liferay.portal.tools.service.builder}{com.liferay.portal.tools.service.builder}
&
\href{/docs/7-2/reference/-/knowledge_base/r/service-builder-plugin}{Service
Builder Plugin} \\
Source Formatter &
\href{https://search.maven.org/search?q=com.liferay.source.formatter}{com.liferay.source.formatter}
&
\href{/docs/7-2/reference/-/knowledge_base/r/source-formatter-plugin}{Source
Formatter Plugin} \\
Theme Builder &
\href{https://search.maven.org/search?q=com.liferay.portal.tools.theme.builder}{com.liferay.portal.tools.theme.builder}
&
\href{/docs/7-2/reference/-/knowledge_base/r/theme-builder-plugin}{Theme
Builder Plugin} \\
TLD Formatter &
\href{https://search.maven.org/search?q=com.liferay.tld.formatter}{com.liferay.tld.formatter}
& \href{/docs/7-2/reference/-/knowledge_base/r/tld-formatter-plugin}{TLD
Formatter Plugin} \\
WSDD Builder &
\href{https://search.maven.org/search?q=com.liferay.portal.tools.wsdd.builder}{com.liferay.portal.tools.wsdd.builder}
& \href{/docs/7-2/reference/-/knowledge_base/r/wsdd-builder-plugin}{WSDD
Builder Plugin} \\
XML Formatter &
\href{https://search.maven.org/search?q=com.liferay.xml.formatter}{com.liferay.xml.formatter}
& \href{/docs/7-2/reference/-/knowledge_base/r/xml-formatter-plugin}{XML
Formatter Plugin} \\
\end{longtable}

\noindent\hrulefill

\textbf{Note:} When upgrading to a Liferay Workspace built with Maven,
many of these plugins are applied to the workspace by default.

\noindent\hrulefill

In Liferay Portal 6.2, you were also required to specify all your app
server configuration settings. For example, your parent POM probably
contained settings similar to these:

\begin{verbatim}
<properties>
    <liferay.app.server.deploy.dir>
        E:\liferay-portal-6.2.0-ce-ga1\tomcat-7.0.42\webapps
    </liferay.app.server.deploy.dir>

    <liferay.app.server.lib.global.dir>
        E:\liferay-portal-6.2.0-ce-ga1\tomcat-7.0.42\lib\ext
    </liferay.app.server.lib.global.dir>

    <liferay.app.server.portal.dir>
        E:\liferay-portal-6.2.0-ce-ga1\tomcat-7.0.42\webapps\root
    </liferay.app.server.portal.dir> 

    <liferay.auto.deploy.dir>
        E:\liferay-portal-6.2.0-ce-ga1\deploy
    </liferay.auto.deploy.dir>

    <liferay.version>
        6.2.0
    </liferay.version>

    <liferay.maven.plugin.version>
        6.2.0
    </liferay.maven.plugin.version
 
</properties>
\end{verbatim}

This is no longer required in 7.0 because Liferay's Maven tools no
longer rely on your Liferay DXP installation's specific versions. You
should remove them from your POM file.

Awesome! You've learned about the new Maven plugins available to you for
7.0 development. Next, you'll learn about updating your Liferay Maven
artifacts.

\section{Updating Liferay Maven Artifact
Dependencies}\label{updating-liferay-maven-artifact-dependencies}

Many Liferay Portal 6.2 artifact dependencies you were using have
changed in 7.0. See the table below for popular Liferay Maven artifacts
that have changed:

\begin{longtable}[]{@{}ll@{}}
\toprule\noalign{}
Liferay Portal 6.2 Artifact ID & 7.0 Artifact ID \\
\midrule\noalign{}
\endhead
\bottomrule\noalign{}
\endlastfoot
\texttt{portal-service} & \texttt{com.liferay.portal.kernel} \\
\texttt{util-bridges} & \texttt{com.liferay.util.bridges} \\
\texttt{util-java} & \texttt{com.liferay.util.java} \\
\texttt{util-slf4j} & \texttt{com.liferay.util.slf4j} \\
\texttt{util-taglib} & \texttt{com.liferay.util.taglib} \\
\end{longtable}

For more information on resolving dependencies in 7.0, see the
\href{/docs/7-2/customization/-/knowledge_base/c/configuring-dependencies}{Resolving
a Plugin's Dependencies} article.

Of course, you must also update the artifacts you're referencing for
your projects. If you're using the Central Repository to install Liferay
Maven artifacts, you won't need to do anything more than update the
artifacts in your POMs. If, however, you're working behind a proxy or
don't have Internet access, you must update your company-shared or local
repository with the latest 7.0 Maven artifacts.

With these updates, you can easily upgrade your Liferay Maven build so
you can begin developing projects for 7.0.

\chapter{Theme Generator}\label{theme-generator}

The Liferay Theme Generator generates themes for Liferay DXP. It is just
one of Liferay JS Theme Toolkit's
\href{https://github.com/liferay/liferay-themes-sdk/tree/master/packages}{tools}.

A couple versions of the Liferay Theme Generator are available. The
version you must install depends on the version of Liferay DXP you're
developing on. The required versions are listed in the table below:

\noindent\hrulefill

\begin{longtable}[]{@{}
  >{\raggedright\arraybackslash}p{(\columnwidth - 4\tabcolsep) * \real{0.1717}}
  >{\raggedright\arraybackslash}p{(\columnwidth - 4\tabcolsep) * \real{0.3333}}
  >{\raggedright\arraybackslash}p{(\columnwidth - 4\tabcolsep) * \real{0.4949}}@{}}
\toprule\noalign{}
\begin{minipage}[b]{\linewidth}\raggedright
Liferay Version
\end{minipage} & \begin{minipage}[b]{\linewidth}\raggedright
Liferay Theme Generator Version
\end{minipage} & \begin{minipage}[b]{\linewidth}\raggedright
Command
\end{minipage} \\
\midrule\noalign{}
\endhead
\bottomrule\noalign{}
\endlastfoot
6.2 & 7.x.x &
\texttt{npm\ install\ -g\ generator-liferay-theme@\^{}7.x.x} \\
7.0 & 7.x.x or 8.x.x & Same as above or below \\
7.1 & 8.x.x &
\texttt{npm\ install\ -g\ generator-liferay-theme@\^{}8.x.x} \\
7.2 & 9.x.x &
\texttt{npm\ install\ -g\ generator-liferay-theme@\^{}9.x.x} \\
\end{longtable}

\noindent\hrulefill

See
\href{https://learn.liferay.com/w/dxp/building-applications/tooling/reference/node-version-information\#version-compatibility-matrix}{Version
Compatibility Matrix} for more information.

\section{Sub-generators}\label{sub-generators}

The Liferay Theme Generator includes the sub-generators listed in the
table below:

\noindent\hrulefill

\begin{longtable}[]{@{}
  >{\raggedright\arraybackslash}p{(\columnwidth - 4\tabcolsep) * \real{0.1376}}
  >{\raggedright\arraybackslash}p{(\columnwidth - 4\tabcolsep) * \real{0.2661}}
  >{\raggedright\arraybackslash}p{(\columnwidth - 4\tabcolsep) * \real{0.5963}}@{}}
\toprule\noalign{}
\begin{minipage}[b]{\linewidth}\raggedright
Sub-generator
\end{minipage} & \begin{minipage}[b]{\linewidth}\raggedright
Command to run
\end{minipage} & \begin{minipage}[b]{\linewidth}\raggedright
Description
\end{minipage} \\
\midrule\noalign{}
\endhead
\bottomrule\noalign{}
\endlastfoot
Layouts & \texttt{yo\ liferay-theme:layout} & Generate layout templates
with an interactive VIM. \\
Themelets & \texttt{yo\ liferay-theme:themelet} & Create small,
reusable, pieces of CSS and HTML for your themes. \\
\end{longtable}

\noindent\hrulefill

\section{Layouts Sub-generator}\label{layouts-sub-generator}

The Layouts sub-generator provides the controls to create a layout
template that meets your needs. You can add and remove rows and columns
on-the-fly as you require. See
\href{/docs/7-2/reference/-/knowledge_base/r/creating-layout-templates-with-the-themes-generator}{Generating
Layout Templates} for more information.

\section{Themelets Sub-generator}\label{themelets-sub-generator}

Themelets are small, extendable, and reusable pieces of code. They only
require the CSS and/or JavaScript files that you want to add to your
theme. This modular approach reduces the need for duplicated code across
themes and makes it easy for developers to share code snippets with
other developers. Themelets are applicable for changes both small and
large, from modifying the appearance of an individual piece of UI to
adding new features. If there is something you have to manually code for
every theme you create, it's a good candidate for a themelet. See
\href{/docs/7-2/reference/-/knowledge_base/r/creating-themelets-with-the-themes-generator}{Generating
Themelets} for more information.

While you can create themes using the tools you prefer, the Liferay
Theme Generator is designed with theme development for Liferay DXP in
mind. Its toolset and features help streamline theme development.

\chapter{Installing the Theme Generator and Creating a
Theme}\label{installing-the-theme-generator-and-creating-a-theme}

The steps below show how to install the Liferay Theme Generator and
generate a theme.

\begin{figure}
\centering
\includegraphics{./images/theme-generator-theme-example.png}
\caption{The tools are in your hands to create any theme you can
imagine.}
\end{figure}

Your first step in generating a theme is installing
\href{http://nodejs.org/}{NodeJS} (along with Node Package Manager(npm))
if it's not already installed. We recommend installing
\href{https://nodejs.org/download/release/v10.15.1/}{v10.15.1}, which is
the version Liferay Portal 7.2 supports (See the
\href{https://learn.liferay.com/w/dxp/building-applications/tooling/reference/node-version-information\#version-compatibility-matrix}{compatibility
matrix}). Once NodeJS is installed and you've
\href{/docs/7-2/reference/-/knowledge_base/r/setting-up-your-npm-environment}{set
up your npm environment}, you can follow these steps to install the
Liferay Theme Generator and generate a theme:

\begin{enumerate}
\def\labelenumi{\arabic{enumi}.}
\item
  Use npm to install the \href{http://yeoman.io/}{Yeoman} dependency:

\begin{verbatim}
npm install -g yo
\end{verbatim}
\end{enumerate}

\noindent\hrulefill

\begin{verbatim}
 **Note:** Gulp is included as a local dependency in generated themes, so you
 are not required to install it. It can be accessed by running
 `node_modules\.bin\gulp` followed by the Gulp task from a generated theme's
 root folder.
\end{verbatim}

\noindent\hrulefill

\begin{enumerate}
\def\labelenumi{\arabic{enumi}.}
\setcounter{enumi}{1}
\item
  Install the Liferay Theme Generator with the command below:

\begin{verbatim}
npm install -g generator-liferay-theme
\end{verbatim}

  If you're on Windows, follow the instructions in step 3 to install
  Sass, otherwise you can skip to step 4.
\item
  The generator uses node-sass. If you're on Windows, you must also
  install
  \href{https://github.com/nodejs/node-gyp\#installation}{node-gyp and
  Python}.
\item
  Run the generator and follow the prompts to create your theme:

\begin{verbatim}
yo liferay-theme
\end{verbatim}

  \begin{figure}
  \centering
  \includegraphics{./images/theme-generator-theme-prompt.png}
  \caption{You can generate a theme by answering just a few
  configuration questions.}
  \end{figure}
\end{enumerate}

\noindent\hrulefill

\begin{verbatim}
 **Note:** Since Liferay DXP Fix Pack 2 and Liferay Portal 7.2 CE GA2, Font
 Awesome is available globally as a system setting, which is enabled by
 default. If you're using Font Awesome icons in your theme, answer yes (y)
 to the Font Awesome question to include Font Awesome imports in your
 theme. This ensures that your icons won't break if a Site Administrator
 disables the global setting.
\end{verbatim}

\noindent\hrulefill

\begin{enumerate}
\def\labelenumi{\arabic{enumi}.}
\setcounter{enumi}{4}
\tightlist
\item
  Navigate to your theme folder and run \texttt{gulp\ deploy} to deploy
  your new theme to the server.
\end{enumerate}

Now you have a powerful theme development tool at your disposal. The sky
is the limit!

\chapter{Generating Layout Templates with the Theme
Generator}\label{generating-layout-templates-with-the-theme-generator}

This article shows how to use the Liferay Theme Generator's Layouts
sub-generator to create a layout template.

\begin{figure}
\centering
\includegraphics{./images/layout-template-1-2-1-columns.png}
\caption{The \emph{1-2-1 Columns} page layout creates a nice flow for
your content.}
\end{figure}

Your first step in creating a layout template with the Liferay Theme
Generator's Layouts sub-generator is installing the
\href{/docs/7-2/reference/-/knowledge_base/r/installing-the-theme-generator-and-creating-a-theme}{Liferay
Theme Generator} if it's not already installed. Once the generator is
installed, you can follow these steps to create a layout template:

\begin{enumerate}
\def\labelenumi{\arabic{enumi}.}
\tightlist
\item
  Open the Command Line and navigate to the folder where you want to
  create your layout template.
\end{enumerate}

\noindent\hrulefill

\begin{verbatim}
 **Note:** Run the Layouts sub-generator from the theme's root folder to
 bundle it with the theme. This adds the layout template to the theme's
 `src/layouttpl/custom` folder. This **only works** for generated themes.
\end{verbatim}

\noindent\hrulefill

\begin{enumerate}
\def\labelenumi{\arabic{enumi}.}
\setcounter{enumi}{1}
\item
  Run The Layouts sub-generator with the command below, and use the
  options listed below to create your layout:

\begin{verbatim}
yo liferay-theme:layout
\end{verbatim}

  \begin{figure}
  \centering
  \includegraphics{./images/layout-column-widths.png}
  \caption{You must specify the width for each column in the row.}
  \end{figure}

  \begin{figure}
  \centering
  \includegraphics{./images/layout-prompt.png}
  \caption{The Layouts sub-generator automates the layout creation
  process.}
  \end{figure}

  \begin{itemize}
  \item
    \textbf{Add a row:} Adds a row below the last row.
  \item
    \textbf{Insert row:} Displays a vi to insert your row. Use your
    arrow keys to choose where to insert your row, highlighted in blue,
    then press Enter to insert the row.
  \end{itemize}

  \begin{figure}
  \centering
  \includegraphics{./images/insert-row.png}
  \caption{Rows can be inserted using the layout vi.}
  \end{figure}

  \begin{itemize}
  \tightlist
  \item
    \textbf{Remove row:} Displays a vi to remove your row. Use your
    arrow keys to select the row you want to remove, highlighted in red,
    then press Enter to remove the row.
  \end{itemize}

  \begin{figure}
  \centering
  \includegraphics{./images/remove-row.png}
  \caption{Rows are removed using the layout vi.}
  \end{figure}

  \begin{itemize}
  \tightlist
  \item
    \textbf{Finish Layout:} Complete the layout template.
  \end{itemize}

  \begin{figure}
  \centering
  \includegraphics{./images/finish-layout.png}
  \caption{Select the \emph{Finish layout} option to complete your
  design.}
  \end{figure}
\item
  Run \texttt{gulp\ deploy} to deploy your layout template to the server
  you specified, or deploy your theme if the layout is
  \href{/docs/7-2/frameworks/-/knowledge_base/f/including-layout-templates-with-a-theme}{bundled
  with it}.
\end{enumerate}

\noindent\hrulefill

\begin{verbatim}
 **Note:** Gulp is included as a local dependency of the generator, so you 
 are not required to install it. It can be accessed by running 
 `node_modules\.bin\gulp` followed by the Gulp task from a generated theme's 
 root folder.
\end{verbatim}

\noindent\hrulefill

Awesome! You just created a layout template with the Theme Generator's
Layouts sub-generator. Your layout template project should have a file
structure similar to the one below:

\begin{itemize}
\tightlist
\item
  \texttt{my-liferay-layout-layouttpl/}

  \begin{itemize}
  \tightlist
  \item
    \texttt{docroot/}

    \begin{itemize}
    \tightlist
    \item
      \texttt{WEB-INF/}

      \begin{itemize}
      \tightlist
      \item
        \texttt{liferay-layout-templates.xml}
      \item
        \texttt{liferay-plugin-package.properties}
      \end{itemize}
    \item
      \texttt{my\_liferay\_layout.png}
    \item
      \texttt{my\_liferay\_layout.tpl}
    \end{itemize}
  \item
    \texttt{node\_modules/}

    \begin{itemize}
    \tightlist
    \item
      (lots of packages)
    \end{itemize}
  \item
    \texttt{gulpfile.js}
  \item
    \texttt{liferay-plugin.json}
  \item
    \texttt{package-lock.json}
  \item
    \texttt{package.json}
  \end{itemize}
\end{itemize}

\chapter{Generating Themelets with the Theme
Generator}\label{generating-themelets-with-the-theme-generator}

This steps below show how to use the Liferay Theme Generator's Themelets
sub-generator to create a themelet.

\begin{figure}
\centering
\includegraphics{./images/product-menu-animation-themelet.png}
\caption{Themelets can be used to modify one aspect of the UI, that you
can then reuse in your other themes.}
\end{figure}

Your first step in creating a themelet is installing the
\href{/docs/7-2/reference/-/knowledge_base/r/installing-the-theme-generator-and-creating-a-theme}{Liferay
Theme Generator} if it's not already installed. Once the generator is
installed, you can follow these steps to create a themelet:

\begin{enumerate}
\def\labelenumi{\arabic{enumi}.}
\item
  Open the Command Line and navigate to the folder you want to create
  your themelet in.
\item
  Run \texttt{yo\ liferay-theme:themelet} and follow the prompts to
  generate the themelet.

  \begin{figure}
  \centering
  \includegraphics{./images/themelet-prompt.png}
  \caption{The Themelet sub-generator automates the themelet creation
  process, making it quick and easy.}
  \end{figure}

  The generated themelet contains a \texttt{package.json} file with
  configuration information and a \texttt{src/css} folder that contains
  a \texttt{\_custom.scss} file. Just like a theme, add your CSS changes
  to the \texttt{src/css} folder, and add your JavaScript changes to the
  \texttt{src/js} folder.
\item
  To use your themelet, you must install it globally first. This makes
  the themelet visible to the generator. To install your themelet
  globally, navigate into its root folder and run \texttt{npm\ link}.
  Note, you may need to run the command using \texttt{sudo\ npm\ link}.
  This creates a globally-installed symbolic link for the themelet in
  your npm packages folder. Now your themelet is available to install in
  your themes.
\end{enumerate}

Awesome! You just created a themelet with the Theme Generator's
Themelets sub-generator.

\chapter{Liferay Upgrade Planner}\label{liferay-upgrade-planner}

The Liferay Upgrade Planner provides an automated way to adapt your
installation's data and legacy plugins to your desired Liferay DXP
upgrade version. We recommend leveraging this tool for any of the
following upgrades:

\begin{itemize}
\tightlist
\item
  Liferay Portal 6.2 → Liferay DXP 7.0, 7.1, or 7.2
\item
  Liferay DXP 7.0 → Liferay DXP 7.1 or 7.2
\item
  Liferay DXP 7.1 → Liferay DXP 7.2
\end{itemize}

For step-by-step instructions for following the two upgrade paths, see
the following documentation:

\begin{itemize}
\tightlist
\item
  \href{/docs/7-2/deploy/-/knowledge_base/d/upgrading-to-product-ver}{Data
  Upgrade}
\item
  \href{/docs/7-2/tutorials/-/knowledge_base/t/upgrading-code-to-product-ver}{Code
  Upgrade}
\end{itemize}

The Upgrade Planner is provided in
\href{/docs/7-2/reference/-/knowledge_base/r/liferay-dev-studio}{Liferay
Dev Studio} (versions 3.6+). Here's what the Upgrade Planner does:

\begin{itemize}
\tightlist
\item
  Updates your development environment.
\item
  Identifies code affected by the API changes.
\item
  Describes each API change related to the code.
\item
  Suggests how to adapt the code.
\item
  Provides options, in some cases, to adapt code automatically.
\item
  Transfers database and server data to your new environment.
\end{itemize}

Even if you prefer tools other than Dev Studio (which is based on
Eclipse), you should upgrade your data and legacy plugins using the
Upgrade Planner first--you can use your favorite tools afterward.

To start the Upgrade Planner in Dev Studio, do this:

\begin{enumerate}
\def\labelenumi{\arabic{enumi}.}
\item
  Navigate to \emph{Project} → \emph{New Liferay Upgrade Plan\ldots{}}.
\item
  In the New Liferay Upgrade Plan wizard, assign your plan a name and
  choose an upgrade plan outline. The data and code upgrade processes
  are separate, so you must step through each process independently.
\item
  Choose your current Liferay version and the new version you're
  upgrading to.
\item
  If you chose to complete a code upgrade, you must also select the
  folder where your legacy plugins reside (e.g., Plugins SDK for Liferay
  6.2 projects).
\item
  Click \emph{Finish}.
\end{enumerate}

\begin{figure}
\centering
\includegraphics{./images/upgrade-plan-wizard.png}
\caption{Configure your upgrade plan before beginning the upgrade
process.}
\end{figure}

Switch to the new Liferay Upgrade Planner perspective (prompted
automatically). You're now offered several windows in the UI:

\begin{itemize}
\tightlist
\item
  \emph{Project Explorer:} displays your legacy plugin environment and
  new development environment. It also displays your
  \href{/docs/7-2/tutorials/-/knowledge_base/t/fixing-upgrade-problems}{upgrade
  problems} that are detected during the \emph{Fix Upgrade Problems}
  step.
\item
  \emph{Liferay Upgrade Plan:} outlines the upgrade plan's steps and
  step summaries.
\item
  \emph{Liferay Upgrade Plan Info:} shows official documentation that
  describes the upgrade step.
\end{itemize}

To progress through your upgrade plan, click the steps outlined in the
Liferay Upgrade Plan window. Each step can have several options:

\begin{itemize}
\tightlist
\item
  \emph{Click to preview:} previews what an automated step will perform.
\item
  \emph{Click to perform:} executes an automated process provided with
  the step. This is only offered for steps where the Upgrade Planner can
  assist.
\item
  \emph{Click when complete:} marks the step as complete. This is only
  offered when the Upgrade Planner cannot provide automated assistance
  and, instead, only offers documentation to assist in completing the
  step manually.
\item
  \emph{Restart:} marks a completed step as unfinished. The step is
  performed again if automation is involved.
\item
  \emph{Skip:} skips the step and jumps to the next step in the outline.
\end{itemize}

\begin{figure}
\centering
\includegraphics{./images/preview-upgrade-planner-changes.png}
\caption{You can preview the Upgrade Planner's automated updates before
you perform them.}
\end{figure}

Great! You now have a good understanding of the Liferay Upgrade
Planner's UI and how to get started. Visit the
\href{/docs/7-2/deploy/-/knowledge_base/d/upgrading-to-product-ver}{Data
Upgrade} and
\href{/docs/7-2/tutorials/-/knowledge_base/t/upgrading-code-to-product-ver}{Code
Upgrade} sections for more information on those upgrade processes.

\chapter{Using the Upgrade Planner with Proxy
Requirements}\label{using-the-upgrade-planner-with-proxy-requirements}

If you have proxy server requirements and want to configure your http(s)
proxy\\
to work with the Liferay Upgrade Planner, follow the instructions below.

\begin{enumerate}
\def\labelenumi{\arabic{enumi}.}
\item
  In Dev Studio's \texttt{DeveloperStudio.ini}/\texttt{eclipse.ini}
  file, add the following parameters:

\begin{verbatim}
-Djdk.http.auth.proxying.disabledSchemes=
-Djdk.http.auth.tunneling.disabledSchemes=
\end{verbatim}
\item
  Launch Dev Studio.
\item
  Go to \emph{Window} → \emph{Preferences} → \emph{General} →
  \emph{Network Connections}.
\item
  Set the \emph{Active Provider} drop-down selector to \emph{Manual}.
\item
  Under \emph{Proxy entries}, configure both proxy HTTP and HTTPS by
  clicking the field and selecting the \emph{Edit} button.

  \begin{figure}
  \centering
  \includegraphics{./images/upgrade-planner-proxy.png}
  \caption{You can configure your proxy settings in Dev Studio's Network
  Connections menu.}
  \end{figure}
\item
  For each schema (HTTP and HTTPS), enter your proxy server's host,
  port, and authentication settings (if necessary). Do not leave
  whitespace at the end of your proxy host or port settings.
\item
  Once you've configured your proxy entry, click \emph{Apply and Close}.
\end{enumerate}

Awesome! You've successfully configured the Upgrade Planner's proxy
settings!

\chapter{Web Experience Management
Reference}\label{web-experience-management-reference}

Browse this section's reference articles for additional information on
the Web Experience Management framework.

\chapter{Fragment Specific Tags}\label{fragment-specific-tags}

There are Liferay-specific tags for creating editable, text, image, and
link fields, and for embedding widgets.

\section{Making Text Editable}\label{making-text-editable}

You can make text of a fragment editable by enclosing it in an
\texttt{\textless{}lfr-editable\textgreater{}} tag like this:

\begin{verbatim}
<lfr-editable id="unique-id" type="text">
   This is editable text!
</lfr-editable>
\end{verbatim}

If you need formatting options like text or color styles, use
\texttt{rich-text}:

\begin{verbatim}
<lfr-editable id="unique-id" type="rich-text">
   This is editable text that I can make bold or italic! 
</lfr-editable>
\end{verbatim}

The \texttt{lfr-editable} tag doesn't render without a unique
\texttt{id}.

\noindent\hrulefill

\textbf{Note:} If you want to make text inside an HTML element editable,
you must use the \texttt{rich-text} type. The \texttt{text} type strips
HTML formatting out of the text before rendering.

\noindent\hrulefill

\section{Making Images Editable}\label{making-images-editable}

Images use the same \texttt{\textless{}lfr-editable\textgreater{}} tag
as text, but with the \texttt{image} type, like this:

\begin{verbatim}
<lfr-editable id="unique-id" type="image">
   <img src="...">
</lfr-editable>
\end{verbatim}

After you add the \texttt{lfr-editable} tag with the type \texttt{image}
to a Fragment, when you add that Fragment to a page, you can then click
on the editable image to select an image or configure content mapping
for the image.

\begin{figure}
\centering
\includegraphics{./images/fragment-image-editor.png}
\caption{You have several options for defining an image on a Content
Page.}
\end{figure}

Most images can be handled like this, but to add an editable background
image you must add an additional property to set the background image
ID, \texttt{data-lfr-background-image-id}. The background image ID is
set in the main \texttt{div} for the Fragment and is the same as your
editable image ID.

\begin{verbatim}
<div data-lfr-background-image-id="unique-id">
   <lfr-editable id="unique-id" type="image">
      <img src="...">
   </lfr-editable>
</div>
\end{verbatim}

Content mapping connects editable fields in your Fragment with fields
from an Asset type like Web Content or Blogs. For example, you can map
an image field to display a preview image for a Web Content Article. For
more information on mapping fields, see
\href{/docs/7-2/user/-/knowledge_base/u/content-page-elements\#editable-elements}{Editable
Elements}.

\section{Creating Editable Links}\label{creating-editable-links}

There is also a specific syntax for creating editable link elements:

\begin{verbatim}
<lfr-editable id="unique-id" type="link">
    <a href="default-target-url-goes-here">Link text goes here</a>
</lfr-editable>
\end{verbatim}

Marketers can edit the link text, target URL, and basic link
styling---primary button, secondary button, link.

\begin{figure}
\centering
\includegraphics{./images/fragment-link-editor.png}
\caption{You have several options for defining a link's appearance and
behavior.}
\end{figure}

For more information on editable links, see
\href{/docs/7-2/user/-/knowledge_base/u/content-page-elements\#editable-links}{Editable
Links}.

\section{Including Widgets Within A
Fragment}\label{including-widgets-within-a-fragment}

To include a widget, you must know its registered name. For example, the
Site Navigation Menu portlet is registered as \texttt{nav}. Each
registered portlet has an \texttt{lfr-widget-{[}name{]}} tag that's used
to embed it. For example: the Navigation Menu tag is
\texttt{\textless{}lfr-widget-nav\ /\textgreater{}}. You could embed it
in a block like this:

\begin{verbatim}
<div class="nav-widget">
    <lfr-widget-nav>
    </lfr-widget-nav>
</div>
\end{verbatim}

These are the widgets that can be embedded and their accompanying tags:

\noindent\hrulefill

\begin{longtable}[]{@{}ll@{}}
\toprule\noalign{}
Widget Name & Tag \\
\midrule\noalign{}
\endhead
\bottomrule\noalign{}
\endlastfoot
DDL Display &
\texttt{\textless{}lfr-widget-dynamic-data-list\textgreater{}} \\
Form & \texttt{\textless{}lfr-widget-form\textgreater{}} \\
Asset Publisher &
\texttt{\textless{}lfr-widget-asset-list\textgreater{}} \\
Breadcrumb & \texttt{\textless{}lfr-widget-breadcrumb\textgreater{}} \\
Categories Navigation &
\texttt{\textless{}lfr-widget-categories-nav\textgreater{}} \\
Flash & \texttt{\textless{}lfr-widget-flash\textgreater{}} \\
Media Gallery &
\texttt{\textless{}lfr-widget-media-gallery\textgreater{}} \\
Navigation Menu & \texttt{\textless{}lfr-widget-nav\textgreater{}} \\
Polls Display & \texttt{\textless{}lfr-widget-polls\textgreater{}} \\
Related Assets &
\texttt{\textless{}lfr-widget-related-assets\textgreater{}} \\
Site Map & \texttt{\textless{}lfr-widget-site-map\textgreater{}} \\
Tag Cloud & \texttt{\textless{}lfr-widget-tag-cloud\textgreater{}} \\
Tags Navigation &
\texttt{\textless{}lfr-widget-tags-nav\textgreater{}} \\
Web Content Display &
\texttt{\textless{}lfr-widget-web-content\textgreater{}} \\
RSS Publisher (Deprecated) &
\texttt{\textless{}lfr-widget-rss\textgreater{}} \\
Iframe & \texttt{\textless{}lfr-widget-iframe\textgreater{}} \\
\end{longtable}

\noindent\hrulefill

\section{Enabling Embedding for Your
Widget}\label{enabling-embedding-for-your-widget}

If you have a custom widget that you want to embed in a fragment, you
can configure that widget to be embeddable. To embed your widget, it
must be an OSGi Component. Inside the \texttt{@Component} annotation for
the portlet class you want to embed, add this property:

\begin{verbatim}
com.liferay.fragment.entry.processor.portlet.alias=app-name
\end{verbatim}

When you deploy your widget, it's available to add. The name you specify
in the property must be appended to the \texttt{lfr-widget} tag like
this:

\begin{verbatim}
<lfr-widget-app-name>
</lfr-widget-app-name>
\end{verbatim}

\noindent\hrulefill

\textbf{Note:} According to the W3C HTML standards, custom elements
cannot be self closing. Therefore, even though you cannot add anything
between the opening and closing
\texttt{\textless{}lfr-widget...\textgreater{}} tags, you cannot use the
self closing notation for the tag.

\chapter{Fragment Configuration
Types}\label{fragment-configuration-types}

There are four configurable Fragment types available to implement:

\begin{itemize}
\tightlist
\item
  \texttt{checkbox}
\item
  \texttt{colorPalette}
\item
  \texttt{select}
\item
  \texttt{text}
\end{itemize}

For more information on how to make a Fragment configurable, see
\href{/docs/7-2/frameworks/-/knowledge_base/f/making-a-fragment-configurable}{Making
a Fragment Configurable}.

In this article, you'll explore JSON examples of each Fragment type.

\section{Checkbox Configuration}\label{checkbox-configuration}

The following JSON configuration creates a checkbox you can implement
for cases where a boolean value selection is necessary:

\begin{verbatim}
{
    "fieldSets": [
        {
            "fields": [
                {
                    "name": "hideBody",
                    "label": "Hide Body",
                    "description": "hide-body",
                    "type": "checkbox",
                    "defaultValue": false
                }
                ...
            ]
        }
    ]
}
\end{verbatim}

\begin{figure}
\centering
\includegraphics{./images/fragment-config-checkbox.png}
\caption{The checkbox configuration is useful when a boolean selection
is necessary.}
\end{figure}

\section{Color Palette Configuration}\label{color-palette-configuration}

The following JSON configuration creates a color selector you can
implement for cases where you must select a color:

\begin{verbatim}
{
    "fieldSets": [
        {
            "fields": [
                {
                    "name": "textColor",
                    "label": "Text color",
                    "type": "colorPalette",
                    "dataType": "object",
                    "defaultValue": {
                        "cssClass": "white",
                        "rgbValue": "rgb(255,255,255)"
                    }
                }
                ...
            ]
        }
    ]
}
\end{verbatim}

The \texttt{colorPalette} type stores an object with two values:
\texttt{cssClass} and \texttt{rgbValue}.

For example, if you implement the snippet above, you can use it in the
FreeMarker context like this:

\begin{verbatim}
<h3 class="text-${configuration.textColor.cssClass}">Example</h3>
\end{verbatim}

If you were to choose the color white, the \texttt{h3} tag heading would
have the class \texttt{text-white\textquotesingle{}}.

\begin{figure}
\centering
\includegraphics{./images/fragment-config-colorpalette.png}
\caption{The \texttt{colorPalette} configuration is useful when a color
selection is necessary.}
\end{figure}

\section{Select Configuration}\label{select-configuration}

The following JSON configuration creates a selector you can implement
for cases where you must select a predefined option:

\begin{verbatim}
{
    "fieldSets": [
        {
            "fields": [
                {
                    "name": "numberOfFeatures",
                    "label": "Number Of Features",
                    "description": "number-of-features",
                    "type": "select",
                    "dataType": "int",
                    "typeOptions": {
                        "validValues": [
                            {"value": "1"},
                            {"value": "2"},
                            {"value": "3"}
                        ]
                    },
                    "defaultValue": "3"
                }
                ...
            ]
        }
    ]
}
\end{verbatim}

Configuration values inserted into the FreeMarker context honor the
defined \texttt{datatype} value specified in the JSON file.

\begin{figure}
\centering
\includegraphics{./images/fragment-config-select.png}
\caption{The \texttt{select} configuration is useful when an option
choice is necessary.}
\end{figure}

\section{Text Configuration}\label{text-configuration}

The following JSON configuration creates an input text field you can
implement for cases where you must manually input a text option:

\begin{verbatim}
{
    "fieldSets": [
        {
            "fields": [
                {
                    "name": "buttonText",
                    "label": "Button Text",
                    "description": "button-text",
                    "type": "text",
                    "typeOptions": {
                        "placeholder": "Placeholder"
                    },
                    "dataType": "string",
                    "defaultValue": "Go Somewhere"
                }
                ...
            ]
        }
    ]
}
\end{verbatim}

Configuration values inserted into the FreeMarker context honor the
defined \texttt{datatype} value specified in the JSON file.

\begin{figure}
\centering
\includegraphics{./images/fragment-config-text.png}
\caption{The \texttt{text} configuration is useful when an input text
option is necessary.}
\end{figure}

\chapter{Escaping Fragment Configuration Text
Values}\label{escaping-fragment-configuration-text-values}

When you define text configuration and other options for a Fragment,
Fragment developers can declare any text value they want. With this
freedom comes risk; malicious code could be inserted into the text
field, wreaking havoc for other users of the Fragment.

In this article, you'll learn how to escape Fragment text values so
Fragment authors are protected from XSS attacks.

\section{Escaping Values in
HTML/FreeMarker}\label{escaping-values-in-htmlfreemarker}

You must take special care when adding a text value in your Fragment's
HTML. For example, if a user includes malicious code within
\texttt{\textless{}script\textgreater{}} tags, it runs when the page is
rendered.

To solve this problem, a utility is available in the FreeMarker context
via the \texttt{htmlUtil} class.

For generic cases, an \texttt{escape} method is available:

\begin{verbatim}
<div class="fragment_38816">
    ${htmlUtil.escape(configuration.text)}"
</div>
\end{verbatim}

This escapes your Fragment configuration's \texttt{text} field,
preventing malicious code from affecting Fragment authors.

For more information on escaping methods, see the
\href{https://docs.liferay.com/dxp/portal/7.2-latest/javadocs/portal-kernel/com/liferay/portal/kernel/util/HtmlUtil.html}{HtmlUtil}
class.

\section{Escaping Values in
JavaScript}\label{escaping-values-in-javascript}

When including JavaScript in your Fragment, you must be aware of
potential attack vectors, like setting an attribute or appending HTML
children. To prevent these attacks, you should use the
\texttt{Liferay.Util.escapeHTML} function. You can call it from your
Fragment's JavaScript like this:

\begin{verbatim}
function (fragmentElement, configuration) {
    const escapedValue = Liferay.Util.escapeHTML(configuration.text)
}
\end{verbatim}

This escapes your Fragment configuration's \texttt{text} field,
preventing malicious code in your Fragment's JavaScript code.

\chapter{Asset Display Page Taglib}\label{asset-display-page-taglib}

Once you have created an Asset Display Page associated with your Asset
type, you can add the option to select an Asset Display Page in the form
where you create the Asset. The
\texttt{\textless{}liferay-asset:select-asset-display-page\ /\textgreater{}}
taglib renders a form field for selecting an Asset Display Page for your
asset.

There are three options when selecting a display page:

\begin{itemize}
\item
  The default display page for the asset type if one has been
  configured.
\item
  Any other selectable display page.
\item
  None
\end{itemize}

If you select no default display page, a display page is not defined.

\section{Display Page Attributes}\label{display-page-attributes}

When you use the display page selector taglib, you can define the
following attributes:

\texttt{classNameId\ (long)} (required): a class name ID of the asset
type to select an asset display page for.

\texttt{classPK\ (long)}: a primary key of the asset entry to select an
asset display page for.

\texttt{classTypeId\ (long)}: a class type ID of the asset type to
select an asset display page for.

\texttt{eventName\ (String)}: event name which fires when a user selects
a display page using the item selector.

\texttt{groupId\ (long)} (required): the entity's group ID to select an
asset display page for.

\texttt{showPortletLayouts\ (boolean)}: allow selection of pages that
have Asset Publisher configured as a default Asset Publisher for the
page.

The attribute \texttt{showPortletLayout} provides backwards
compatibility for display pages created for Journal Articles in older
versions. When \texttt{showPortletLayouts} is set to true, you can
select any public or private pages with an Asset Publisher widget on it
configured as the \emph{Default Asset Publisher for the page}.

When submitting a form with the taglib, it populates the request with
the following parameters:

\texttt{displayPageType\ (int)}: 1 = Default, 2 = Specific, 3 = None.

\texttt{assetDisplayPageId\ (long)}: ID of selected Asset Display Page.

\texttt{layoutUuid}: Layout UUID in case of a portlet page with default
Asset Publisher.

\chapter{Crafting XML Workflow
Definitions}\label{crafting-xml-workflow-definitions}

You don't need a fancy visual designer to build workflows. To be clear,
Kaleo Designer may make you a faster workflow designer through its
graphical interface. If you plan to build lots of workflow processes, a
Digital Enterprise subscription gets you access to Kaleo Designer. But
with a little copy and paste from existing workflows and a little
handcrafted XML, you can build any workflow and attain workflow
wizard-hood in the process. Follow this set of tutorials to learn what
elements you can put into your definitions.

\section{Existing Workflow
Definitions}\label{existing-workflow-definitions}

Only one workflow definition is installed by default: Single Approver.
Several more, however, are embedded in the source code of your Liferay
DXP installation. If you're comfortable extracting the XML files from a
JAR file embedded in an LPKG file, you're welcome to follow the steps
below to obtain the workflow definitions.

To extract the definitions for yourself,

\begin{enumerate}
\def\labelenumi{\arabic{enumi}.}
\item
  Navigate to

\begin{verbatim}
[Liferay Home]/osgi/marketplace
\end{verbatim}
\item
  Using an archive manager, open the LPKG

\begin{verbatim}
Liferay CE Forms and Workflow.lpkg
\end{verbatim}
\item
  Open the JAR file named

\begin{verbatim}
com.liferay.portal.workflow.kaleo.runtime.impl-[version].jar
\end{verbatim}
\item
  In the JAR file, navigate to

\begin{verbatim}
META-INF/definitions/
\end{verbatim}
\item
  Extract the four XML workflow definition files. These definitions
  provide good reference material for many of the workflow features and
  elements described in these articles. In fact, most of the XML
  snippets you see here are lifted directly from these definitions.
\end{enumerate}

\section{Schema}\label{schema}

The XML structure of a workflow definition is defined in an XSD file:

\begin{verbatim}
liferay-worklow-definition-7_2_0.xsd
\end{verbatim}

Declare the schema at the top of the workflow definition file:

\begin{verbatim}
<?xml version="1.0"?>
<workflow-definition
    xmlns="urn:liferay.com:liferay-workflow_7.2.0"
    xmlns:xsi="http://www.w3.org/2001/XMLSchema-instance"
    xsi:schemaLocation="urn:liferay.com:liferay-workflow_7.2.0
        http://www.liferay.com/dtd/liferay-workflow-definition_7_2_0.xsd">
\end{verbatim}

To read 464 lines of beautifully formatted XML that defines how to write
more XML (it's practically poetic), check out the XSD
\href{https://www.liferay.com/dtd/liferay-workflow-definition_7_2_0.xsd}{here}.
Otherwise, move on to entering the definition's metadata.

\section{Metadata}\label{metadata}

Give the definition a name, description, and version:

\begin{verbatim}
<name>Category Specific Approval</name>
<description>A single approver can approve a workflow content.</description>
<version>1</version>
\end{verbatim}

All these tags are optional. If present the first time a definition is
saved, the \texttt{\textless{}name\textgreater{}} tag serves as a unique
identifier for the definition. If not specified (or added sometime after
the first save), a random unique name is generated and used to identify
the workflow.

Once the schema and metadata are in place, it's time to turn up the
funky beats and get into the flow (the workflow). Learn about workflow
nodes in the next article.

\chapter{Workflow Definition Nodes}\label{workflow-definition-nodes}

After your definition's schema and metadata are in place, begin defining
the process. \emph{Node} elements, with their sub-elements, are
fundamental building blocks making up workflow definitions.

\section{State Nodes}\label{state-nodes}

State nodes don't require user input. The workflow does whatever is
specified in the state node's \texttt{actions} tag (a notification
and/or a custom script), and then moves to the provided transition.
Workflows start and end with a state. The initial state node often only
contains a transition:

\begin{verbatim}
<state>
    <name>created</name>
    <initial>true</initial>
    <transitions>
        <transition>
            <name>Determine Branch</name>
            <target>determine-branch</target>
            <default>true</default>
        </transition>
    </transitions>
</state>
\end{verbatim}

If a notification or script is required in your state node, use an
\texttt{actions} tag. Here's an \texttt{action} element containing a
Groovy script. This is found in many terminal state nodes and marks the
asset as approved in the workflow.

\begin{verbatim}
<actions>
    <action>
        <name>Approve</name>
        <description>Approve</description>
        <script>
            <![CDATA[
            com.liferay.portal.kernel.workflow.WorkflowStatusManagerUtil.
                updateStatus(com.liferay.portal.kernel.workflow.WorkflowConstants.     
                    getLabelStatus("approved"), workflowContext);]]>
        </script>
        <script-language>groovy</script-language>
        <execution-type>onEntry</execution-type>
    </action>
</actions>
\end{verbatim}

\section{Conditions}\label{conditions}

Conditions let you inspect the asset (or its execution context) and do
something, like send it to a particular transition.

Here's the \texttt{determine-branch} condition from the Category
Specific Approval workflow definition:

\begin{verbatim}
<condition>
    <name>determine-branch</name>
    <script>
        <![CDATA[
            import com.liferay.asset.kernel.model.AssetCategory;
            import com.liferay.asset.kernel.model.AssetEntry;
            import com.liferay.asset.kernel.model.AssetRenderer;
            import com.liferay.asset.kernel.model.AssetRendererFactory;
            import com.liferay.asset.kernel.service.AssetEntryLocalServiceUtil;
            import com.liferay.portal.kernel.util.GetterUtil;
            import com.liferay.portal.kernel.workflow.WorkflowConstants;
            import com.liferay.portal.kernel.workflow.WorkflowHandler;
            import com.liferay.portal.kernel.workflow.WorkflowHandlerRegistryUtil;

            import java.util.List;

            String className = (String)workflowContext.get(WorkflowConstants.CONTEXT_ENTRY_CLASS_NAME);

            WorkflowHandler workflowHandler = WorkflowHandlerRegistryUtil.getWorkflowHandler(className);

            AssetRendererFactory assetRendererFactory = workflowHandler.getAssetRendererFactory();

            long classPK = GetterUtil.getLong((String)workflowContext.get(WorkflowConstants.CONTEXT_ENTRY_CLASS_PK));

            AssetRenderer assetRenderer = workflowHandler.getAssetRenderer(classPK);

            AssetEntry assetEntry = assetRendererFactory.getAssetEntry(assetRendererFactory.getClassName(), assetRenderer.getClassPK());

            List<AssetCategory> assetCategories = assetEntry.getCategories();

            returnValue = "Content Review";

            for (AssetCategory assetCategory : assetCategories) {
                String categoryName = assetCategory.getName();

                if (categoryName.equals("legal")) {
                    returnValue = "Legal Review";

                    return;
                }
            }
        ]]>
    </script>
    <script-language>groovy</script-language>
    <transitions>
        <transition>
            <name>Legal Review</name>
            <target>legal-review</target>
            <default>false</default>
        </transition>
        <transition>
            <name>Content Review</name>
            <target>content-review</target>
            <default>false</default>
        </transition>
    </transitions>
</condition>
\end{verbatim}

This example checks the asset category to choose the processing path,
whether to transition to the \emph{Legal Review} task or the
\emph{Content Review} task.

The \texttt{returnValue} variable points from the condition to a
transition, and its value must match a valid transition name. This
script looks up the asset in question, retrieves its asset category, and
sets an initial \texttt{returnValue}. Then it checks to see if the asset
has been marked with the \emph{legal} category. If not it goes through
\emph{Content Review} (to the content-review task in the workflow), and
if it does it goes through \emph{Legal Review} (to the legal-review task
in the workflow).

\section{Forks and Joins}\label{forks-and-joins}

Forks split the workflow process, and joins bring the process back to a
unified branch. Processing must always be brought back using a Join (or
a Join XOR), and the number of forks and joins in a workflow definition
must be equal.

\begin{verbatim}
<fork>
    <name>fork-1</name>
    <transitions>
        <transition>
            <name>transition-1</name>
            <target>task-1</target>
            <default>true</default>
        </transition>
        <transition>
            <name>transition-2</name>
            <target>task-2</target>
            <default>false</default>
        </transition>
    </transitions>
</fork>
<join>
    <name>join-1</name>
    <transitions>
        <transition>
            <name>transition-4</name>
            <target>EndNode</target>
            <default>true</default>
        </transition>
    </transitions>
</join>
\end{verbatim}

The workflow doesn't move past the join until the asset transitions to
it from both of the forks. To fork the workflow process, but then allow
the processing to continue when only one fork is completed, use a Join
XOR.

A Join XOR differs from a join in one important way: it removes the
constraint that both forks must be completed before processing can
continue. The asset must complete just one of the forks before
processing continues.

\begin{verbatim}
<join-xor>
    <name>join-xor</name>
    <transitions>
        <transition>
            <name>transition3</name>
            <target>EndNode</target>
            <default>true</default>
        </transition>
    </transitions>
</join-xor>
\end{verbatim}

\section{Task Nodes}\label{task-nodes}

\href{/docs/7-2/reference/-/knowledge_base/r/workflow-task-nodes}{Task
nodes} are at the core of the workflow definition. They're the part
where a user interacts with the asset in some way. Tasks can also have
sub-elements, including notifications, assignments, and task timers.

Here's the \texttt{content-review} task from the Category Specific
Approval workflow, with some of the \texttt{role} assignment tags cut
out for brevity:

\begin{verbatim}
<task>
    <name>content-review</name>
    <actions>
        <notification>
            <name>Review Notification</name>
            <template>You have a new submission waiting for your review in the workflow.</template>
            <template-language>text</template-language>
            <notification-type>email</notification-type>
            <notification-type>user-notification</notification-type>
            <execution-type>onAssignment</execution-type>
        </notification>
    </actions>
    <assignments>
        <roles>
            <role>
                <role-type>organization</role-type>
                <name>Organization Administrator</name>
            </role>
            ...
        </roles>
    </assignments>
    <task-timers>
        <task-timer>
            <name></name>
            <delay>
                <duration>1</duration>
                <scale>hour</scale>
            </delay>
            <blocking>false</blocking>
            <timer-actions>
                <timer-notification>
                    <name></name>
                    <template></template>
                    <template-language>text</template-language>
                    <notification-type>user-notification</notification-type>
                </timer-notification>
            </timer-actions>
        </task-timer>
    </task-timers>
    <transitions>
        <transition>
            <name>approve</name>
            <target>approved</target>
            <default>true</default>
        </transition>
        <transition>
            <name>reject</name>
            <target>update</target>
            <default>false</default>
        </transition>
    </transitions>
</task>
\end{verbatim}

Learn more about workflow tasks in the next article.

\chapter{Workflow Task Nodes}\label{workflow-task-nodes}

Task nodes are fundamental parts of a workflow definition. When you
define your organization's business processes and design corresponding
workflows, you likely first envision the tasks. As the name implies,
tasks are the part of the workflow where \emph{work} is done. A user
enters the picture and must interact with the submitted asset. Users
often take the role of reviewer, deciding if an asset from the workflow
is acceptable for publication or needs more work.

Unlike other workflow nodes, task nodes have Assignments, because a user
is expected to \emph{do something} (often approve or reject the
submitted asset) when a workflow process enters the task node.

Commonly, task nodes contain task timers, assignments, actions (which
can include notifications and scripts), and transitions. Notifications
and actions aren't limited to task nodes, but task nodes and their
assignments deserve their own article (this one).

Check out the Review task in the Single Approver definition, noting that
several \texttt{\textless{}role\textgreater{}} tags are excluded from
this snippet for brevity:

\begin{verbatim}
<task>
    <name>review</name>
    <actions>
        <notification>
            <name>Review Notification</name>
            <template>${userName} sent you a ${entryType} for review in the workflow.</template>
            <template-language>freemarker</template-language>
            <notification-type>email</notification-type>
            <notification-type>user-notification</notification-type>
            <execution-type>onAssignment</execution-type>
        </notification>
        <notification>
            <name>Review Completion Notification</name>
            <template><![CDATA[Your submission was reviewed<#if taskComments?has_content> and the reviewer applied the following ${taskComments}</#if>.]]></template>
            <template-language>freemarker</template-language>
            <notification-type>email</notification-type>
            <recipients>
                <user />
            </recipients>
            <execution-type>onExit</execution-type>
        </notification>
    </actions>
    <assignments>
        <roles>
            <role>
                <role-type>organization</role-type>
                <name>Organization Administrator</name>
            </role>
              ...
        </roles>
    </assignments>
    <transitions>
        <transition>
            <name>approve</name>
            <target>approved</target>
        </transition>
        <transition>
            <name>reject</name>
            <target>update</target>
            <default>false</default>
        </transition>
    </transitions>
</task>
\end{verbatim}

There are two \texttt{actions} in the review task, both
\texttt{\textless{}notification\textgreater{}}s. Each notification may
contain a name, template, notification-type, execution-type, and
recipients. Besides notifications, You can also use the
\texttt{\textless{}action\textgreater{}} tag. These have a name and a
\href{/docs/7-2/user/-/knowledge_base/u/leveraging-the-script-engine-in-workflow}{script}
and are more often used in state nodes than tasks.

\section{Assignments}\label{assignments}

Workflow tasks are completed by a user. Assignments make sure the right
users can access the tasks. You can choose how you want to configure
your assignments.

You can choose to add assignments to specific roles, to multiple roles
of a role type (organization, site, or regular role types), to the asset
creator, to resource actions, or to specific users. Additionally, you
can write a script to define the assignment. For an example, see the
\texttt{single-approver-definition-scripted-assignment.xml}.

\begin{verbatim}
<assignments>
    <roles>
        <role>
            <role-type>organization</role-type>
            <name>Organization Administrator</name>
        </role>
    </roles>
</assignments>
\end{verbatim}

The above assignment specifies that an Organization Administrator must
complete the task.

\begin{verbatim}
<assignments>
    <user>
        <user-id>20156</user-id>
    </user>
</assignments>
\end{verbatim}

The above assignment specifies that only the user with the user ID of
20156 may complete the task. Alternatively, specify the
\texttt{\textless{}screen-name\textgreater{}} or
\texttt{\textless{}email-address\textgreater{}} of the user.

\begin{verbatim}
<assignments>
    <scripted-assignment>
        <script>
            <![CDATA[
                    import com.liferay.portal.kernel.model.Group;
                    import com.liferay.portal.kernel.model.Role;
                    import com.liferay.portal.kernel.service.GroupLocalServiceUtil;
                    import com.liferay.portal.kernel.service.RoleLocalServiceUtil;
                    import com.liferay.portal.kernel.util.GetterUtil;
                    import com.liferay.portal.kernel.workflow.WorkflowConstants;

                    long companyId = GetterUtil.getLong((String)workflowContext.get(WorkflowConstants.CONTEXT_COMPANY_ID));

                    long groupId = GetterUtil.getLong((String)workflowContext.get(WorkflowConstants.CONTEXT_GROUP_ID));

                    Group group = GroupLocalServiceUtil.getGroup(groupId);

                    roles = new ArrayList<Role>();

                    Role adminRole = RoleLocalServiceUtil.getRole(companyId, "Administrator");

                    roles.add(adminRole);

                    if (group.isOrganization()) {
                        Role role = RoleLocalServiceUtil.getRole(companyId, "Organization Content Reviewer");

                        roles.add(role);
                    }
                    else {
                        Role role = RoleLocalServiceUtil.getRole(companyId, "Site Content Reviewer");

                        roles.add(role);
                    }

                    user = null;
                ]]>
            </script>
        <script-language>groovy</script-language>
    </scripted-assignment>
</assignments>
\end{verbatim}

The above assignment assigns the task to the \emph{Administrator} role,
then checks whether the \emph{group} of the asset is an Organization. If
it is, the \emph{Organization Content Reviewer} role is assigned to it.
If it's not, the task is assigned to the \emph{Site Content Reviewer}
role.

Note the
\texttt{roles\ =\ new\ ArrayList\textless{}Role\textgreater{}();} line
above. In a scripted assignment, the \texttt{roles} variable is where
you specify any roles the task is assigned to. For example, when
\texttt{roles.add(adminRole);} is called, the Administrator role is
added to the assignment.

Assigning tasks to Roles, Organizations, or Asset Creators is a
straightforward concept, but what does it mean to assign a workflow task
to a Resource Action? Imagine an \emph{UPDATE} resource action. If your
workflow definition specifies the UPDATE action in an assignment, then
anyone who has permission to update the type of asset being processed in
the workflow is assigned to the task. You can configure multiple
assignments for a task.

\section{Resource Action Assignments}\label{resource-action-assignments}

\emph{Resource actions} are operations performed by users on an
application or entity. For example, a user might have permission to
update Message Boards Messages. This is called an UPDATE resource
action, because the user can update the resource. If you're uncertain
about what resource actions are, refer to the developer tutorial on the
\href{/docs/7-2/frameworks/-/knowledge_base/f/defining-application-permissions}{permission
system} for a more detailed explanation.

To find all the resource actions that have been created, you need access
to the Roles Admin application in the Control Panel (in other words, you
need permission for the VIEW action on the roles resource).

\begin{itemize}
\tightlist
\item
  Navigate to Control Panel → Users → Roles.
\item
  Add a new Regular Role. See the
  \href{/docs/7-2/user/-/knowledge_base/u/roles-and-permissions}{article
  on managing roles} for more information.
\item
  Once the role is added, navigate to the Define Permissions interface
  for the role.
\item
  Find the resource whose action should define your workflow assignment.
\end{itemize}

Here's what the assignment's XML looks like:

\begin{verbatim}
<assignments>
    <resource-actions>
        <resource-action>UPDATE</resource-action>
    </resource-actions>
</assignments>
\end{verbatim}

Now when the workflow proceeds to the task with the resource action
assignment, users with \texttt{UPDATE} permission on the resource (for
example, Message Boards Messages) are notified of the task and can
assign it to themselves (if the notification is set to Task Assignees).
Specifically, users see the tasks in their \emph{My Workflow Tasks}
application under the tab \emph{Assigned to My Roles}.

Use all upper case letters for resource action names. Here are some
common resource actions:

\begin{verbatim}
UPDATE
ADD
DELETE
VIEW
PERMISSIONS
SUBSCRIBE
ADD_DISCUSSION
\end{verbatim}

Determine the probable resource action name from the permissions screen
for a resource. For example, in Message Boards, one of the permissions
displayed on that screen is \emph{Add Discussion}. Convert that to all
uppercase and replace the space with an underscore, and you have the
action name.

\section{Task Timers}\label{task-timers}

Task timers trigger an action after a specified time period passes.
Timers are useful for ensuring a task does not go unattended for a long
time. Available timer actions include sending an additional
notification, reassigning the asset, or creating a timer action.

\begin{verbatim}
<task-timers>
    <task-timer>
        <name></name>
        <delay>
            <duration>1</duration>
            <scale>hour</scale>
        </delay>
        <blocking>false</blocking>
        <recurrence>
            <duration>10</duration>
            <scale>minute</scale>
        </recurrence>
        <timer-actions>
            <timer-notification>
                <name></name>
                <template></template>
                <template-language>text</template-language>
                <notification-type>user-notification</notification-type>
            </timer-notification>
        </timer-actions>
    </task-timer>
</task-timers>
\end{verbatim}

The above task timer creates a notification. Specify a time period in
the \texttt{\textless{}delay\textgreater{}} tag, and specify what action
to take when the time expires in the
\texttt{\textless{}timer-actions\textgreater{}} block. The
\texttt{\textless{}blocking\textgreater{}} element specifies whether the
timer actions may recur. If blocking is set to \texttt{false}, timer
actions may recur. In a \texttt{recurrence} element, specify the
recurrence interval using a \texttt{duration} and a \texttt{scale}, as
demonstrated above. The above recurrence element specifies that the
timer actions run again every ten minutes after the initial occurrence.
Setting blocking to true prevents timer actions from recurring.

\begin{verbatim}
<timer-actions>
    <reassignments>
       <assignments>
         <roles>
          <role>
              <role-type></role-type>
              <name></name>
          </role>
          ...
         </roles>
       </assignments>
    </reassignments>
</timer-actions>
\end{verbatim}

The above snippet demonstrates how to set up a reassignment action.

Like \texttt{\textless{}action\textgreater{}} elements,
\texttt{\textless{}timer-action\textgreater{}} elements can contain
scripts.

\begin{verbatim}
<timer-actions>
    <timer-action>
        <name>doSomething</name>
        <description>Do something cool when time runs out.</description>
        <script>
           ...
        </script>
        <script-language>groovy</script-language>
    </timer-action>
</timer-actions>
\end{verbatim}

The above example isn't functional but it demonstrates setting up a
\texttt{\textless{}script\textgreater{}} in your task timer.
\href{/docs/7-2/user/-/knowledge_base/u/leveraging-the-script-engine-in-workflow}{Read
the \emph{Scripting in Workflow} article} for more information.

\noindent\hrulefill

\textbf{Note:} A \texttt{timer-action} can contain all the same tags as
an \texttt{action}, with one exception: \texttt{execution-type}. Timer
actions are always triggered once the time is up, so specifying and
execution type of \texttt{onEntry}, for example, isn't meaningful inside
a timer.

\noindent\hrulefill

Tasks are at the core of the workflow definition. Once you understand
how to create tasks and the other
\href{/docs/7-2/reference/-/knowledge_base/r/workflow-definition-nodes}{workflow
nodes} and add transitions between the nodes, you're on the cusp of
workflow wizard-hood.

\chapter{Workflow Notifications}\label{workflow-notifications}

While an asset is in a workflow, relevant Users should be notified about
certain events, like when a review task is completed. Any workflow node
with an \texttt{\textless{}actions\textgreater{}} element can have
notifications.

\begin{verbatim}
<actions>
    <action>
        <notification>
            <name>Creator Modification Notification</name>
            <template>Your submission was rejected by ${userName}, please modify and resubmit.</template>
            <template-language>freemarker</template-language>
            <notification-type>email</notification-type>
            <notification-type>user-notification</notification-type>
            <execution-type>onAssignment</execution-type>
        </notification>
    </actions>
</actions>
\end{verbatim}

The above Creator Modification Notification sends a notification message
in two ways: via email and via user notification (this goes to the
Notifications widget in the User's Site). The message is defined in a
FreeMarker template and sent once a task assignment is created. But who
receives the notification? If no recipients are explicitly specified via
a \texttt{recipients} tag, the asset's creator receives the
notification.

\section{Notification Options}\label{notification-options}

There are several elements that can be specified in a
\texttt{\textless{}notification\textgreater{}}:

\begin{description}
\tightlist
\item[\textbf{Name}]
Set the name of the notification in the
\texttt{\textless{}name\textgreater{}} element. This information is used
to display the notification in the \emph{My Workflow Tasks} widget of a
User's personal Site.
\item[\textbf{Description}]
The \texttt{\textless{}description\textgreater{}} element contains the
subject of the notification if the notification type is \texttt{email}.
The syntax is determined by the template language you're using.
\item[\textbf{Template}]
The \texttt{\textless{}template\textgreater{}} element contains the
message of the notification. The syntax is determined by the template
language you're using.
\item[\textbf{Template Language}]
Choose from \texttt{freemarker}, \texttt{velocity}, or plain
\texttt{text} in the \texttt{\textless{}template-language\textgreater{}}
tag.
\item[\textbf{Notification Type}]
Choose whether to send an \texttt{email}, \texttt{user-notification}
(via the Notification widget), \texttt{im} (instant message), or
\texttt{private-message} in the
\texttt{\textless{}notification-type\textgreater{}} tag.
\end{description}

\begin{verbatim}
<notification-type>email</notification-type>
\end{verbatim}

\begin{description}
\tightlist
\item[\textbf{Execution Type}]
Choose to link the sending of the notification to entry into the node
(\texttt{onEntry}), when a task is assigned (\texttt{onAssignment}), or
when the workflow processing is leaving a node (\texttt{onExit}). If you
specify a notification to be sent on assignment, the assignee is
notified automatically.
\item[\textbf{Recipients}]
Decide who should receive the notification in the
\texttt{\textless{}recipients\textgreater{}} tag:
\end{description}

\begin{verbatim}
<recipients>
    [SEE BELOW FOR THE AVAILABLE RECIPIENT TAGS]
</recipients>
\end{verbatim}

Available recipient tags are

\begin{itemize}
\tightlist
\item
  \texttt{\textless{}user\textgreater{}}: notify the User that sent the
  asset through the workflow. Specify the tag as
  \texttt{\textless{}user\ /\textgreater{}}. To notify a specific user,
  enter the \texttt{userId}:
\end{itemize}

\begin{verbatim}
<recipients>
    <user />
</recipients>
<recipients>
    <user>
        <user-id>20139</user-id>
    </user>
</recipients>
\end{verbatim}

\begin{itemize}
\tightlist
\item
  \texttt{\textless{}roles\textgreater{}}: notify specific Roles, either
  by ID or by their type and name.
\end{itemize}

\begin{verbatim}
<recipients>
    <roles>
        <role>
            <role-id>33621</role-id>
        </role>
    </roles>
</recipients>
<recipients>
    <roles>
        <role>
            <role-type>regular</role-type>
            <name>Power User</name>
            <auto-create>false</auto-create>
        </role>
    </roles>
</recipients>
\end{verbatim}

\begin{itemize}
\item
  \texttt{\textless{}assignees\ /\textgreater{}}: notify the task
  assignees.
\item
  \texttt{\textless{}scripted-recipient\textgreater{}}: use a script to
  identify notification recipients.
\end{itemize}

\begin{verbatim}
<recipients>
    <scripted-recipient>
        <script>
            <![CDATA[Script logic goes here]]>
        </script>
        <script-language>groovy</script-language>
    </scripted-recipient>
</recipients>
\end{verbatim}

If the notification type is \texttt{email}, you can specify the
\texttt{recipientType} attribute of the
\texttt{\textless{}recipients\textgreater{}} tag as \emph{To},
\emph{CC}, or \emph{BCC}.

\begin{verbatim}
<recipients receptionType="cc">
    <roles>
        <role>
            <role-type>regular</role-type>
            <name>Manager</name>
        </role>
    </roles>
</recipients>
\end{verbatim}

By default, \texttt{recipientType} is \texttt{to}.

As always, read the
\href{https://www.liferay.com/dtd/liferay-workflow-definition_7_1_0.xsd}{schema
for all the details}.

\chapter{Breaking Changes}\label{breaking-changes}

{ This document has been updated and ported to Liferay Learn and is no
longer maintained here.}

This document presents a chronological list of changes that break
existing functionality, APIs, or contracts with third party Liferay
developers or users. We try our best to minimize these disruptions, but
sometimes they are unavoidable.

Here are some of the types of changes documented in this file:

\begin{itemize}
\tightlist
\item
  Functionality that is removed or replaced
\item
  API incompatibilities: Changes to public Java or JavaScript APIs
\item
  Changes to context variables available to templates
\item
  Changes in CSS classes available to Liferay themes and portlets
\item
  Configuration changes: Changes in configuration files, like
  \texttt{portal.properties}, \texttt{system.properties}, etc.
\item
  Execution requirements: Java version, J2EE Version, browser versions,
  etc.
\item
  Deprecations or end of support: For example, warning that a certain
  feature or API will be dropped in an upcoming version.
\item
  Recommendations: For example, recommending using a newly introduced
  API that replaces an old API, in spite of the old API being kept in
  Liferay Portal for backwards compatibility.
\end{itemize}

\section{Breaking Changes List}\label{breaking-changes-list}

\section{Removed Support for JSP Templates in
Themes}\label{removed-support-for-jsp-templates-in-themes}

\begin{itemize}
\tightlist
\item
  \textbf{Date:} 2018-Nov-14
\item
  \textbf{JIRA Ticket:}
  \href{https://issues.liferay.com/browse/LPS-87064}{LPS-87064}
\end{itemize}

\subsection{What changed?}\label{what-changed}

Themes can no longer leverage JSP templates. Also, related logic has
been removed from the public APIs
\texttt{com.liferay.portal.kernel.util.ThemeHelper} and
\texttt{com.liferay.taglib.util.ThemeUtil}.

\subsection{Who is affected?}\label{who-is-affected}

This affects anyone who has themes using JSP templates or is using the
removed methods.

\subsection{How should I update my
code?}\label{how-should-i-update-my-code}

If you have a theme using JSP templates, consider migrating it to
FreeMarker.

\subsection{Why was this change made?}\label{why-was-this-change-made}

JSP is not a real template engine and is rarely used. FreeMarker is the
recommended template engine moving forward.

The removal of JSP templates allows for an increased focus on existing
and new template engines.

\section{Lodash Is No Longer Included by
Default}\label{lodash-is-no-longer-included-by-default}

\begin{itemize}
\tightlist
\item
  \textbf{Date:} 2018-Nov-27
\item
  \textbf{JIRA Ticket:}
  \href{https://issues.liferay.com/browse/LPS-87677}{LPS-87677}
\end{itemize}

\subsection{What changed?}\label{what-changed-1}

Previously, Lodash was included in every page by default and made
available through the global \texttt{window.\_} and scoped
\texttt{AUI.\_} variables. Lodash is no longer included by default and
those variables are now undefined.

\subsection{Who is affected?}\label{who-is-affected-1}

This affects any developer who used the \texttt{AUI.\_} or
\texttt{window.\_} variables in their custom scripts.

\subsection{How should I update my
code?}\label{how-should-i-update-my-code-1}

You should provide your own Lodash version for your custom developments
to use following any of the possible strategies to add third party
libraries.

As a temporary measure, you can bring back the old behavior by setting
the \emph{Enable Lodash} property in Liferay Portal's \emph{Control
Panel} → \emph{Configuration} → \emph{System Settings} → \emph{Third
Party} → \emph{Lodash} to \texttt{true}.

\subsection{Why was this change made?}\label{why-was-this-change-made-1}

This change was made to avoid bundling and serving additional library
code on every page that was mostly unused and redundant.

\section{Moved Two Staging Properties to OSGi
Configuration}\label{moved-two-staging-properties-to-osgi-configuration}

\begin{itemize}
\tightlist
\item
  \textbf{Date:} 2018-Dec-12
\item
  \textbf{JIRA Ticket:}
  \href{https://issues.liferay.com/browse/LPS-88018}{LPS-88018}
\end{itemize}

\subsection{What changed?}\label{what-changed-2}

Two Staging properties have been moved from \texttt{portal.properties}
to an OSGi configuration named
\texttt{ExportImportServiceConfiguration.java} in the
\texttt{export-import-service} module.

\subsection{Who is affected?}\label{who-is-affected-2}

This affects anyone using the following portal properties:

\begin{itemize}
\tightlist
\item
  \texttt{staging.delete.temp.lar.on.failure}
\item
  \texttt{staging.delete.temp.lar.on.success}
\end{itemize}

\subsection{How should I update my
code?}\label{how-should-i-update-my-code-2}

Instead of overriding the \texttt{portal.properties} file, you can
manage the properties from Portal's configuration administrator. This
can be accessed by navigating to Liferay Portal's \emph{Control Panel} →
\emph{Configuration} → \emph{System Settings} → \emph{Infrastructure} →
\emph{Export/Import} and editing the settings there.

If you would like to include the new configuration in your application,
follow the instructions for
\href{https://dev.liferay.com/develop/tutorials/-/knowledge_base/7-1/making-applications-configurable}{making
applications configurable}.

\subsection{Why was this change made?}\label{why-was-this-change-made-2}

This change was made as part of the modularization efforts to ease
portal configuration changes.

\section{Remove Link Application URLs to Page
Functionality}\label{remove-link-application-urls-to-page-functionality}

\begin{itemize}
\tightlist
\item
  \textbf{Date:} 2018-Dec-14
\item
  \textbf{JIRA Ticket:}
  \href{https://issues.liferay.com/browse/LPS-85948}{LPS-85948}
\end{itemize}

\subsection{What changed?}\label{what-changed-3}

The \emph{Link Portlet URLs to Page} option in the Look and Feel portlet
was marked as deprecated in Liferay Portal 7.1, allowing the user to
show and hide the option through a configuration property. In Liferay
Portal 7.2, this has been removed and can no longer be configured.

\subsection{Who is affected?}\label{who-is-affected-3}

This affects administrators who used the option in the UI and developers
who leveraged the option in the portlet.

\subsection{How should I update my
code?}\label{how-should-i-update-my-code-3}

You should update any portlets leveraging this feature, since any
preconfigured reference to the property is ignored in the portal.

\subsection{Why was this change made?}\label{why-was-this-change-made-3}

A limited number of portlets use this property; there are better ways to
achieve the same results.

\section{Moved TermsOfUseContentProvider out of
kernel.util}\label{moved-termsofusecontentprovider-out-of-kernel.util}

\begin{itemize}
\tightlist
\item
  \textbf{Date:} 2019-Jan-07
\item
  \textbf{JIRA Ticket:}
  \href{https://issues.liferay.com/browse/LPS-88869}{LPS-88869}
\end{itemize}

\subsection{What changed?}\label{what-changed-4}

The \texttt{TermsOfUseContentProvider} interface's package changed:

\texttt{com.liferay.portal.kernel.util} →
\texttt{com.liferay.portal.kernel.term.of.use}

The \texttt{TermsOfUseContentProviderRegistryUtil} class' name and
package changed:

\texttt{TermsOfUseContentProviderRegistryUtil} →
\texttt{TermsOfUseContentProviderUtil}

and \texttt{com.liferay.portal.kernel.util} →
\texttt{com.liferay.portal.internal.terms.of.use}

The logic of getting \texttt{TermsOfUseContentProvider} was also
changed. Instead of always returning the first service registered, which
is random and depends on the order of registered services, the
\texttt{TermsOfUseContentProvider} service is tracked and updated with
\texttt{com.liferay.portal.kernel.util.ServiceProxyFactory}. As a
result, the \texttt{TermsOfUseContentProvider} now respects service
ranking.

\subsection{Who is affected?}\label{who-is-affected-4}

This affects anyone who used
\texttt{com.liferay.portal.kernel.util.TermsOfUseContentProviderRegistryUtil}
to lookup the
\texttt{com.liferay.portal.kernel.util.TermsOfUseContentProvider}
service.

\subsection{How should I update my
code?}\label{how-should-i-update-my-code-4}

If \texttt{com.liferay.portal.kernel.util.TermsOfUseContentProvider} is
used, update the import package name. If there is any usage in
\texttt{portal-web}, update
\texttt{com.liferay.portal.kernel.util.TermsOfUseContentProviderRegistryUtil}
to
\texttt{com.liferay.portal.kernel.term.of.use.TermsOfUseContentProviderUtil}.
Remove usages of
\texttt{com.liferay.portal.kernel.util.TermsOfUseContentProviderRegistryUtil}
in modules and use the \texttt{@Reference} annotation to fetch the
\texttt{com.liferay.portal.kernel.term.of.use.TermsOfUseContentProvider}
service instead.

\subsection{Why was this change made?}\label{why-was-this-change-made-4}

This is one of several steps to clean up kernel provider interfaces to
reduce the chance of package version lock down.

\section{Removed HibernateConfigurationConverter and
Converter}\label{removed-hibernateconfigurationconverter-and-converter}

\begin{itemize}
\tightlist
\item
  \textbf{Date:} 2019-Jan-07
\item
  \textbf{JIRA Ticket:}
  \href{https://issues.liferay.com/browse/LPS-88870}{LPS-88870}
\end{itemize}

\subsection{What changed?}\label{what-changed-5}

The interface \texttt{com.liferay.portal.kernel.util.Converter} and its
implementation
\texttt{com.liferay.portal.spring.hibernate.HibernateConfigurationConverter}
were removed.

\subsection{Who is affected?}\label{who-is-affected-5}

This removes the support of generating customized
\texttt{portlet-hbm.xml} files implemented by
\texttt{HibernateConfigurationConverter}. Refer to
\href{https://issues.liferay.com/browse/LPS-5363}{LPS-5363} for more
information.

\subsection{How should I update my
code?}\label{how-should-i-update-my-code-5}

You should remove usages of \texttt{HibernateConfigurationConverter}.
Make sure the generated \texttt{portlet-hbm.xml} is accurate.

\subsection{Why was this change made?}\label{why-was-this-change-made-5}

This is one of several steps to clean up kernel provider interfaces to
reduce the chance of package version lock down.

\section{Switched to Use JDK Function and
Supplier}\label{switched-to-use-jdk-function-and-supplier}

\begin{itemize}
\tightlist
\item
  \textbf{Date:} 2019-Jan-08
\item
  \textbf{JIRA Ticket:}
  \href{https://issues.liferay.com/browse/LPS-88911}{LPS-88911}
\end{itemize}

\subsection{What changed?}\label{what-changed-6}

The \texttt{Function} and \texttt{Supplier} interfaces in package
\texttt{com.liferay.portal.kernel.util} were removed. Their usages were
replaced with \texttt{java.util.function.Function} and
\texttt{java.util.function.Supplier}.

\subsection{Who is affected?}\label{who-is-affected-6}

This affects anyone who implemented the \texttt{Function} and
\texttt{Supplier} interfaces in package
\texttt{com.liferay.portal.kernel.util}.

\subsection{How should I update my
code?}\label{how-should-i-update-my-code-6}

You should replace usages of
\texttt{com.liferay.portal.kernel.util.Function} and
\texttt{com.liferay.portal.kernel.util.Supplier} with
\texttt{java.util.function.Function} and
\texttt{java.util.function.Supplier}, respectively.

\subsection{Why was this change made?}\label{why-was-this-change-made-6}

This is one of several steps to clean up kernel provider interfaces to
reduce the chance of package version lock down.

\section{Deprecated com.liferay.portal.service.InvokableService
Interface}\label{deprecated-com.liferay.portal.service.invokableservice-interface}

\begin{itemize}
\tightlist
\item
  \textbf{Date:} 2019-Jan-08
\item
  \textbf{JIRA Ticket:}
  \href{https://issues.liferay.com/browse/LPS-88912}{LPS-88912}
\end{itemize}

\subsection{What changed?}\label{what-changed-7}

The \texttt{InvokableService} and \texttt{InvokableLocalService}
interfaces in package \texttt{com.liferay.portal.kernel.service} were
removed.

\subsection{Who is affected?}\label{who-is-affected-7}

This affects anyone who used \texttt{InvokableService} and
\texttt{InvokableLocalService} in package
\texttt{com.liferay.portal.kernel.service}.

\subsection{How should I update my
code?}\label{how-should-i-update-my-code-7}

You should remove usages of \texttt{InvokableService} and
\texttt{InvokableLocalService}. Make sure to use the latest version of
Service Builder to generate implementations for services in case there
is any compile errors after removal.

\subsection{Why was this change made?}\label{why-was-this-change-made-7}

This is one of several steps to clean up kernel provider interfaces to
reduce the chance of package version lock down.

\section{Dropped Support of
ServiceLoaderCondition}\label{dropped-support-of-serviceloadercondition}

\begin{itemize}
\tightlist
\item
  \textbf{Date:} 2019-Jan-08
\item
  \textbf{JIRA Ticket:}
  \href{https://issues.liferay.com/browse/LPS-88913}{LPS-88913}
\end{itemize}

\subsection{What changed?}\label{what-changed-8}

The interface \texttt{ServiceLoaderCondition} and its implementation
\texttt{DefaultServiceLoaderCondition} in package
\texttt{com.liferay.portal.kernel.util} were removed.

\subsection{Who is affected?}\label{who-is-affected-8}

This affects anyone using \texttt{ServiceLoaderCondition} and
\texttt{DefaultServiceLoaderCondition}.

\subsection{How should I update my
code?}\label{how-should-i-update-my-code-8}

You should remove usages of \texttt{ServiceLoaderCondition}. Update
usages of \texttt{load} methods in
\texttt{com.liferay.portal.kernel.util.ServiceLoader} according to the
updated method signatures.

\subsection{Why was this change made?}\label{why-was-this-change-made-8}

This is one of several steps to clean up kernel provider interfaces to
reduce the chance of package version lock down.

\section{Switched to Use JDK
Predicate}\label{switched-to-use-jdk-predicate}

\begin{itemize}
\tightlist
\item
  \textbf{Date:} 2019-Jan-14
\item
  \textbf{JIRA Ticket:}
  \href{https://issues.liferay.com/browse/LPS-89139}{LPS-89139}
\end{itemize}

\subsection{What changed?}\label{what-changed-9}

The interface \texttt{com.liferay.portal.kernel.util.PredicateFilter}
was removed and replaced with \texttt{java.util.function.Predicate}. As
a result, the following implementations were removed:

\begin{itemize}
\tightlist
\item
  \texttt{com.liferay.portal.kernel.util.AggregatePredicateFilter}
\item
  \texttt{com.liferay.portal.kernel.util.PrefixPredicateFilter}
\item
  \texttt{com.liferay.portal.kernel.portlet.JavaScriptPortletResourcePredicateFilter}
\item
  \texttt{com.liferay.dynamic.data.mapping.form.values.query.internal.model.DDMFormFieldValuePredicateFilter}
\end{itemize}

The \texttt{com.liferay.portal.kernel.util.ArrayUtil\_IW} class was
regenerated.

\subsection{Who is affected?}\label{who-is-affected-9}

This affects anyone who used \texttt{PredicateFilter},
\texttt{AggregatePredicateFilter}, \texttt{PrefixPredicateFilter},
\texttt{JavaScriptPortletResourcePredicateFilter}, and
\texttt{DDMFormFieldValuePredicateFilter}.

\subsection{How should I update my
code?}\label{how-should-i-update-my-code-9}

You should replace usages of
\texttt{com.liferay.portal.kernel.util.PredicateFilter} with
\texttt{java.util.function.Predicate}. Additionally, remove usages of
\texttt{AggregatePredicateFilter}, \texttt{PrefixPredicateFilter},
\texttt{JavaScriptPortletResourcePredicateFilter}, and
\texttt{DDMFormFieldValuePredicateFilter}.

\subsection{Why was this change made?}\label{why-was-this-change-made-9}

This is one of several steps to clean up kernel provider interfaces to
reduce the chance of package version lock down.

\section{Removed Unsafe Functional Interfaces in Package
com.liferay.portal.kernel.util}\label{removed-unsafe-functional-interfaces-in-package-com.liferay.portal.kernel.util}

\begin{itemize}
\tightlist
\item
  \textbf{Date:} 2019-Jan-15
\item
  \textbf{JIRA Ticket:}
  \href{https://issues.liferay.com/browse/LPS-89223}{LPS-89223}
\end{itemize}

\subsection{What changed?}\label{what-changed-10}

The \texttt{com.liferay.portal.osgi.util.test.OSGiServiceUtil} class was
removed. Also, the following interfaces were removed from the
\texttt{com.liferay.portal.kernel.util} package:

\begin{itemize}
\tightlist
\item
  \texttt{UnsafeConsumer}
\item
  \texttt{UnsafeFunction}
\item
  \texttt{UnsafeRunnable}
\end{itemize}

\subsection{Who is affected?}\label{who-is-affected-10}

This affects anyone using the class/interfaces mentioned above.

\subsection{How should I update my
code?}\label{how-should-i-update-my-code-10}

The \texttt{com.liferay.portal.osgi.util.test.OSGiServiceUtil} class has
been deprecated since Liferay Portal 7.1. If usages for this class still
exist, replace it with its direct replacement:
\texttt{com.liferay.osgi.util.service.OSGiServiceUtil}. Replace usages
of \texttt{UnsafeConsumer}, \texttt{UnsafeFunction} and
\texttt{UnsafeRunnable} with their corresponding interfaces in package
\texttt{com.liferay.petra.function}.

\subsection{Why was this change
made?}\label{why-was-this-change-made-10}

This is one of several steps to clean up kernel provider interfaces to
reduce the chance of package version lock down.

\section{Deprecated NTLM in Portal
Distribution}\label{deprecated-ntlm-in-portal-distribution}

\begin{itemize}
\tightlist
\item
  \textbf{Date:} 2019-Jan-21
\item
  \textbf{JIRA Ticket:}
  \href{https://issues.liferay.com/browse/LPS-88300}{LPS-88300}
\end{itemize}

\subsection{What changed?}\label{what-changed-11}

NTLM modules have been moved from the \texttt{portal-security-sso}
project to a new project named \texttt{portal-security-sso-ntlm}. This
new project is deprecated and available to download from Liferay
Marketplace.

\subsection{Who is affected?}\label{who-is-affected-11}

This affects anyone using NTLM as an authentication system.

\subsection{How should I update my
code?}\label{how-should-i-update-my-code-11}

If you want to continue using NTLM as an authentication system, you must
download the corresponding modules from Liferay Marketplace.
Alternatively, you can migrate to Kerberos (recommended), which requires
no changes and is compatible with Liferay Portal 7.0+.

\subsection{Why was this change
made?}\label{why-was-this-change-made-11}

This change was made to avoid using an old proprietary solution (NTLM).
Kerberos is now recommended, which is a standard protocol and a more
secure method of authentication compared to NTLM.

\section{Deprecated OpenID in Portal
Distribution}\label{deprecated-openid-in-portal-distribution}

\begin{itemize}
\tightlist
\item
  \textbf{Date:} 2019-Jan-21
\item
  \textbf{JIRA Ticket:}
  \href{https://issues.liferay.com/browse/LPS-88906}{LPS-88906}
\end{itemize}

\subsection{What changed?}\label{what-changed-12}

OpenID modules have been moved to a new project named
\texttt{portal-security-sso-openid}. This new project is deprecated and
available to download from Liferay Marketplace.

\subsection{Who is affected?}\label{who-is-affected-12}

This affects anyone using OpenID as an authentication system.

\subsection{How should I update my
code?}\label{how-should-i-update-my-code-12}

If you want to continue using OpenID as an authentication system, you
must download the corresponding module from Liferay Marketplace.
Alternatively, you should migrate to OpenID Connect, available on
Liferay Portal Distribution.

\subsection{Why was this change
made?}\label{why-was-this-change-made-12}

This change was made to avoid using a deprecated solution (OpenID).
OpenID Connect is now recommended, which is a more secure method of
authentication since it runs on top of OAuth.

\section{Deprecated Google SSO in Portal
Distribution}\label{deprecated-google-sso-in-portal-distribution}

\begin{itemize}
\tightlist
\item
  \textbf{Date:} 2019-Jan-21
\item
  \textbf{JIRA Ticket:}
  \href{https://issues.liferay.com/browse/LPS-88905}{LPS-88905}
\end{itemize}

\subsection{What changed?}\label{what-changed-13}

Google SSO modules have been moved from the \texttt{portal-security-sso}
project to a new project named \texttt{portal-security-sso-google}. This
new project is deprecated and available to download from Liferay
Marketplace.

\subsection{Who is affected?}\label{who-is-affected-13}

This affects anyone using Google SSO as an authentication system.

\subsection{How should I update my
code?}\label{how-should-i-update-my-code-13}

If you want to continue using Google SSO as an authentication system,
you must download the corresponding module from Liferay Marketplace.
Alternatively, you can use OpenID Connect.

\subsection{Why was this change
made?}\label{why-was-this-change-made-13}

This change was made to avoid using an old solution for authentication
(Google SSO). OpenID Connect is the recommended specification to use
Google implementation for authentication.

\section{Updated AlloyEditor v2.0 Includes New Major Version of
React}\label{updated-alloyeditor-v2.0-includes-new-major-version-of-react}

\begin{itemize}
\tightlist
\item
  \textbf{Date:} 2019-Feb-04
\item
  \textbf{JIRA Ticket:}
  \href{https://issues.liferay.com/browse/LPS-90079}{LPS-90079}
\end{itemize}

\subsection{What changed?}\label{what-changed-14}

AlloyEditor was upgraded to version 2.0.0, which includes a major
upgrade from React v15 to v16.

The \texttt{React.createClass} was
\href{https://reactjs.org/blog/2017/04/07/react-v15.5.0.html}{deprecated
in React v15.5.0} (April 2017) and
\href{https://reactjs.org/blog/2017/09/26/react-v16.0.html}{removed in
React v16.0.0} (September 2017). All the buttons bundled with
AlloyEditor have been updated to use the ES6 class syntax instead of
\texttt{React.createClass}.

\subsection{Who is affected?}\label{who-is-affected-14}

This affects anyone who built their own buttons using
\texttt{React.createClass}. The \texttt{createClass} function is no
longer available, and attempts to access it at runtime will trigger an
error.

\subsection{How should I update my
code?}\label{how-should-i-update-my-code-14}

You should update your code in one of two ways:

\begin{itemize}
\item
  Port custom buttons from the \texttt{React.createClass} API to use the
  ES6 \texttt{class} API, as described in
  \href{https://reactjs.org/docs/react-component.html}{the React
  documentation}. For example, see the changes made in moving to an
  \href{https://github.com/liferay/alloy-editor/blob/b082c312179ae6626cb2ddcc04ad3ebc5b355e1b/src/components/buttons/button-ol.jsx}{ES6
  class-based button} from
  \href{https://github.com/liferay/alloy-editor/blob/2826ab9ceabe17c6ba0d38985baf8a787c23db43/src/ui/react/src/components/buttons/button-ol.jsx}{the
  previous \texttt{createClass}-based implementation}.
\item
  Provide a compatibility adapter. The
  \href{https://www.npmjs.com/package/create-react-class}{create-react-class
  package} (described
  \href{https://reactjs.org/docs/react-without-es6.html}{here}) can be
  injected into the page to restore the \texttt{createClass} API.
\end{itemize}

\subsection{Why was this change
made?}\label{why-was-this-change-made-14}

This change was made to use a newer major version of React, which brings
performance and compatibility improvements and reduces the bundle size
by removing deprecated APIs.

\section{Deprecated dl.tabs.visible
property}\label{deprecated-dl.tabs.visible-property}

\begin{itemize}
\tightlist
\item
  \textbf{Date:} 2019-Apr-10
\item
  \textbf{JIRA Ticket:}
  \href{https://issues.liferay.com/browse/LPS-93948}{LPS-93948}
\end{itemize}

\subsection{What changed?}\label{what-changed-15}

The \texttt{dl.tabs.visible} property let users toggle the visibility of
a Documents and Media widget's navigation tabs when placed on a widget
page. This configuration option has been removed, so the navigation tab
will never appear on widget pages.

\subsection{Who is affected?}\label{who-is-affected-15}

This affects anyone who set the \texttt{dl.tabs.visible} property to
\texttt{true}.

\subsection{How should I update my
code?}\label{how-should-i-update-my-code-15}

No code changes are necessary.

\subsection{Why was this change
made?}\label{why-was-this-change-made-15}

Documents \& Media has been reviewed from a UX perspective, and removing
the navigation tabs in widget pages was part of a UI clean up process.

\section{Move the User Menu out of the Product
Menu}\label{move-the-user-menu-out-of-the-product-menu}

\begin{itemize}
\tightlist
\item
  \textbf{Date:} 2019-Apr-19
\item
  \textbf{JIRA Ticket:}
  \href{https://issues.liferay.com/browse/LPS-87868}{LPS-87868}
\end{itemize}

\subsection{What changed?}\label{what-changed-16}

The User Menu was removed from the Product Menu, and the user menu
entries were moved to the new Personal Menu, a dropdown menu triggered
by the user avatar.

\subsection{Who is affected?}\label{who-is-affected-16}

This affects anyone who has customized the User Menu section of the
Product Menu.

\subsection{How should I update my
code?}\label{how-should-i-update-my-code-16}

If you would like to keep your custom user menu entries and have them
available in the Personal Menu, you need to implement the
\texttt{PersonalMenuEntry} interface. All panel apps registered with the
\texttt{PanelCategoryKeys.USER},
\texttt{PanelCategoryKeys.USER\_MY\_ACCOUNT}, and
\texttt{PanelCategoryKeys.USER\_SIGN\_OUT} panel category keys should be
converted to \texttt{PersonalMenuEntry}.

\subsection{Why was this change
made?}\label{why-was-this-change-made-16}

Product navigation has been reviewed from a UX perspective, and removing
the User Menu from the Product Menu and splitting the menu to its own
provides a better user experience.

\section{Removed Hong Kong and Macau from the List of
Countries}\label{removed-hong-kong-and-macau-from-the-list-of-countries}

\begin{itemize}
\tightlist
\item
  \textbf{Date:} 2019-Apr-26
\item
  \textbf{JIRA Ticket:}
  \href{https://issues.liferay.com/browse/LPS-82203}{LPS-82203}
\end{itemize}

\subsection{What changed?}\label{what-changed-17}

Hong Kong and Macau have been removed from the list of countries and
listed as regions of China as Xianggang (region code: CN-91) and Aomen
(region code: CN-92), respectively.

\subsection{Who is affected?}\label{who-is-affected-17}

This affects anyone who used Hong Kong or Macau in their addresses.

\subsection{How should I update my
code?}\label{how-should-i-update-my-code-17}

No code changes are necessary. However, if you have hardcoded the
\texttt{countryId} of Hong Kong and Macau in your code, they should be
updated to China's \texttt{countryId}. References to Hong Kong and Macau
should be done with their corresponding \texttt{regionId}.

\subsection{Why was this change
made?}\label{why-was-this-change-made-17}

After the handover of Hong Kong in 1997 and of Macau in 1999, Hong Kong
and Macau are now the special administrative regions of China.

\section{JGroups Was Upgraded From 3.6.16 to
4.1.1}\label{jgroups-was-upgraded-from-3.6.16-to-4.1.1}

\begin{itemize}
\tightlist
\item
  \textbf{Date:} 2019-Aug-15
\item
  \textbf{JIRA Ticket:}
  \href{https://issues.liferay.com/browse/LPS-97897}{LPS-97897}
\end{itemize}

\subsection{What changed?}\label{what-changed-18}

JGroups was upgraded from version 3.6.16 to 4.1.1.

\subsection{Who is affected?}\label{who-is-affected-18}

This affects anyone using Cluster Link.

\subsection{How should I update my
code?}\label{how-should-i-update-my-code-18}

The \texttt{cluster.link.channel.properties.*} property in
\texttt{portal.properties} no longer accepts a connection string as a
value; it now requires a file path to a configuration XML file. Some of
the protocol properties from 3.6.16 are removed and no longer parsed by
4.1.1; you should update the protocol properties accordingly.

\subsection{Why was this change
made?}\label{why-was-this-change-made-18}

This upgrade was made to fix a security issue.

\section{\texorpdfstring{Liferay \texttt{AssetEntries\_AssetCategories}
Is No Longer
Used}{Liferay AssetEntries\_AssetCategories Is No Longer Used}}\label{liferay-assetentries_assetcategories-is-no-longer-used}

\begin{itemize}
\tightlist
\item
  \textbf{Date:} 2019-Sep-11
\item
  \textbf{JIRA Tickets:}
  \href{https://issues.liferay.com/browse/LPS-99973}{LPS-99973},
  \href{https://issues.liferay.com/browse/LPS-76488}{LPS-76488}
\end{itemize}

\subsection{What changed?}\label{what-changed-19}

Previously, Liferay used a mapping table and a corresponding interface
for the relationship between \texttt{AssetEntry} and
\texttt{AssetCategory} in \texttt{AssetEntryLocalService} and
\texttt{AssetCategoryLocalService}. This mapping table and the
corresponding interface have been replaced by the table
\texttt{AssetEntryAssetCategoryRel} and the service
\texttt{AssetEntryAssetCategoryRelLocalService}.

\subsection{Who is affected?}\label{who-is-affected-19}

This affects any content or code that relies on calling the old
interfaces for the \texttt{AssetEntries\_AssetCategories} relationship,
through the \texttt{AssetEntryLocalService} and
\texttt{AssetCategoryLocalService}.

\subsection{How should I update my
code?}\label{how-should-i-update-my-code-19}

Use the new methods in \texttt{AssetEntryAssetCategoryRelLocalService}
to retrieve the same data as before. The method signatures haven't
changed; they have just been relocated to a different service.

\textbf{Example}

Old way:

\begin{verbatim}
List<AssetEntry> entries =
AssetEntryLocalServiceUtil.getAssetCategoryAssetEntries(categoryId);

for (AssetEntry entry: entries) {
  ...
}
\end{verbatim}

New way:

\begin{verbatim}
long[] assetEntryPKs =
_assetEntryAssetCategoryRelLocalService.getAssetEntryPrimaryKeys(assetCategoryId);

for (long assetEntryPK: assetEntryPKs) {
  AssetEntry = _assetEntryLocalService.getEntry(assetEntryPK);
  ...
}

...

@Reference
private AssetEntryAssetCategoryRelLocalService _assetEntryAssetCategoryRelLocalService;

@Reference
private AssetEntryLocalService _assetEntryLocalService;
\end{verbatim}

\subsection{Why was this change
made?}\label{why-was-this-change-made-19}

This change was made due to changes resulting from
\href{https://issues.liferay.com/browse/LPS-76488}{LPS-76488}, which let
developers control the order of a list of assets for a given category.

\section{Auto Tagging Must Be Reconfigured
Manually}\label{auto-tagging-must-be-reconfigured-manually}

\begin{itemize}
\tightlist
\item
  \textbf{Date: 2019-Oct-2}
\item
  \textbf{JIRA Ticket:}
  \href{https://issues.liferay.com/browse/LPS-97123}{LPS-97123}
\end{itemize}

\subsection{What changed?}\label{what-changed-20}

Auto Tagging configurations were renamed and reorganized. There's no
longer an automatic upgrade process, so you must reconfigure Auto
Tagging manually.

\subsection{Who is affected?}\label{who-is-affected-20}

This affects DXP 7.2 installations that are upgraded to SP1 and have
Auto Tagging configured and enabled.

\subsection{How should I update my
code?}\label{how-should-i-update-my-code-20}

You must reconfigure Auto Tagging through System Settings (please see
the
\href{https://help.liferay.com/hc/en-us/articles/360029041551-Configuring-Asset-Auto-Tagging}{official
documentation} for details). Any code referencing the old configuration
interfaces must be updated to use the new ones.

\subsection{Why was this change
made?}\label{why-was-this-change-made-20}

This change unifies the previously split configuration interfaces,
improving the user experience.

\section{Blogs Image Properties Were Moved to System
Settings}\label{blogs-image-properties-were-moved-to-system-settings}

\begin{itemize}
\tightlist
\item
  \textbf{Date: 2019-Oct-2}
\item
  \textbf{JIRA Ticket:}
  \href{https://issues.liferay.com/browse/LPS-95298}{LPS-95298}
\end{itemize}

\subsection{What changed?}\label{what-changed-21}

Blogs image configuration was moved from \texttt{portal.properties} to
System Settings. There's no automatic upgrade process, so custom Blogs
image properties must be reconfigured manually.

\subsection{Who is affected?}\label{who-is-affected-21}

This affects DXP 7.2 installations that are upgraded to SP1 and have
custom values for the \texttt{blogs.image.max.size} and
\texttt{blogs.image.extensions} properties.

\subsection{How should I update my
code?}\label{how-should-i-update-my-code-21}

If you would like to keep your custom Blogs image property values, you
must reconfigure them through the System Settings under
\emph{Configuration} → \emph{Blogs} → \emph{File Uploads}. Any code
referencing the old properties must be updated to use the new
configuration interfaces.

\subsection{Why was this change
made?}\label{why-was-this-change-made-21}

This change was made so Blogs image properties can be configured without
a restart.

\section{Removed Cache Bootstrap
Feature}\label{removed-cache-bootstrap-feature}

\begin{itemize}
\tightlist
\item
  \textbf{Date:} 2020-Jan-8
\item
  \textbf{JIRA Ticket:}
  \href{https://issues.liferay.com/browse/LPS-96563}{LPS-96563}
\end{itemize}

\subsection{What changed?}\label{what-changed-22}

The cache bootstrap feature has been removed. These properties can no
longer be used to enable/configure cache bootstrap:

\texttt{ehcache.bootstrap.cache.loader.enabled},
\texttt{ehcache.bootstrap.cache.loader.properties.default},
\texttt{ehcache.bootstrap.cache.loader.properties.\$\{specific.cache.name\}}.

\subsection{Who is affected?}\label{who-is-affected-22}

This affects anyone using the properties listed above.

\subsection{How should I update my
code?}\label{how-should-i-update-my-code-22}

There's no direct replacement for the removed feature. If you have code
that depends on it, you must implement it yourself.

\subsection{Why was this change
made?}\label{why-was-this-change-made-22}

This change was made to avoid security issues.

\chapter{CDI Portlet Predefined
Beans}\label{cdi-portlet-predefined-beans}

Liferay DXP provides injectable portlet artifacts for
\href{/docs/7-2/frameworks/-/knowledge_base/f/cdi-dependency-injection}{CDI}
called Portlet Predefined Beans, as specified by
\href{https://jcp.org/en/jsr/detail?id=362}{JSR 362}. There are two
types of predefined beans:

\begin{itemize}
\item
  Portlet Request Scoped Beans
  (\href{https://docs.liferay.com/portlet\%20\%7C\%20-\%20\%7C\%20api/3.0/javadocs/javax/portlet/annotations/PortletRequestScoped.html}{\texttt{@PortletRequestScoped}})
\item
  Dependent Scoped Beans
  (\href{https://docs.oracle.com/javaee/7/api/javax/enterprise/context/Dependent.html}{\texttt{@Dependent}
  scoped})
\end{itemize}

The table below describes these attributes for each bean:

\textbf{Artifact:} The bean's type.

\textbf{Bean EL Name:} Expression Language (EL) name for accessing the
bean in a JSP or JSF page.

\textbf{Qualifier:} Annotation applied to the bean for defining and
selecting a bean implementation.

\textbf{Valid during (phase):} The
\href{/docs/7-2/frameworks/-/knowledge_base/f/portlets}{portlet phases}
in which the bean is valid.

\section{Portlet Request Scoped
Beans}\label{portlet-request-scoped-beans}

These beans have the \texttt{@PortletRequestScoped} annotation. Here are
their artifact types, bean EL names, and annotation qualifiers, along
with their valid portlet phases.

Table 1: Portlet Request Scoped Beans\footnote{Martin Scott Nicklous,
  Java™ Portlet Specification 3.0, page 122.}

\noindent\hrulefill

\begin{longtable}[]{@{}llll@{}}
\toprule\noalign{}
Artifact & Bean EL Name & Qualifier & Valid during \\
\midrule\noalign{}
\endhead
\bottomrule\noalign{}
\endlastfoot
\texttt{PortletConfig} & \texttt{portletConfig} & - & all \\
\texttt{PortletRequest} & \texttt{portletRequest} & - & all \\
\texttt{PortletResponse} & \texttt{portletResponse} & - & all \\
\texttt{ActionRequest} & \texttt{actionRequest} & - & action \\
\texttt{ActionResponse} & \texttt{actionResponse} & - & action \\
\texttt{HeaderRequest} & \texttt{headerRequest} & - & header \\
\texttt{HeaderResponse} & \texttt{headerResponse} & - & header \\
\texttt{RenderRequest} & \texttt{renderRequest} & - & render \\
\texttt{RenderResponse} & \texttt{renderResponse} & - & render \\
\texttt{EventRequest} & \texttt{eventRequest} & - & event \\
\texttt{EventResponse} & \texttt{eventResponse} & - & event \\
\texttt{ResourceRequest} & \texttt{resourceRequest} & - & resource \\
\texttt{ResourceResponse} & \texttt{resourceResponse} & - & resource \\
\texttt{StateAwareResponse} & \texttt{stateAwareResponse} & - & action,
event \\
\texttt{MimeResponse} & \texttt{mimeResponse} & - & header, render,
resource \\
\texttt{ClientDataRequest} & \texttt{clientDataRequest} & - & action,
resource \\
\texttt{RenderParameters} & \texttt{renderParams} & - & all \\
\texttt{MutableRenderParameters} & \texttt{mutableRenderParams} & - &
action, event \\
\texttt{ActionParameters} & \texttt{actionParams} & - & action \\
\texttt{ResourceParameters} & \texttt{resourceParams} & - & resource \\
\texttt{PortletContext} & \texttt{portletContext} & - & all \\
\texttt{PortletMode} & \texttt{portletMode} & - & all \\
\texttt{WindowState} & \texttt{windowState} & - & all \\
\texttt{PortletPreferences} & \texttt{portletPreferences} & - & all \\
\texttt{Cookies(List\textless{}Cookie\textgreater{})} & \texttt{cookies}
& - & all \\
\texttt{PortletSession} & \texttt{portletSession} & - & all \\
\texttt{Locales(List\textless{}Locale\textgreater{})} & \texttt{locales}
& - & all \\
\end{longtable}

\noindent\hrulefill

\section{Dependent Scoped Beans}\label{dependent-scoped-beans}

These beans use the \texttt{@Dependent} scope. They're of type
\texttt{java.lang.String}, which is \texttt{final}. This disqualifies
them from being proxied. To prevent using dependent scoped beans in a
scope broader than their original scope, you should only inject them
into \texttt{@PortletRequestScoped} beans.

Table 2: Dependent Scoped Beans\footnote{Martin Scott Nicklous, Java™
  Portlet Specification 3.0, page 123.}

\noindent\hrulefill

\begin{longtable}[]{@{}llll@{}}
\toprule\noalign{}
Artifact & Bean EL Name & Qualifier & Valid during \\
\midrule\noalign{}
\endhead
\bottomrule\noalign{}
\endlastfoot
\texttt{Namespace} (String) & \texttt{namespace} & \texttt{@Namespace} &
all \\
\texttt{ContextPath} (String) & \texttt{contextPath} &
\texttt{@ContextPath} & all \\
\texttt{WindowID} (String) & \texttt{windowId} & \texttt{@WindowId} &
all \\
\texttt{Portlet\ name} (String) & \texttt{portletName} &
\texttt{@PortletName} & all \\
\end{longtable}

\noindent\hrulefill

\section{Related Topics}\label{related-topics-49}

\href{/docs/7-2/frameworks/-/knowledge_base/f/cdi-dependency-injection}{CDI
Dependency Injection}

\chapter{Item Selector Criterion and Return
Types}\label{item-selector-criterion-and-return-types}

Liferay DXP contains Item Selector criterion
(\href{https://docs.liferay.com/dxp/apps/item-selector/latest/javadocs/com/liferay/item/selector/ItemSelectorCriterion.html}{\texttt{ItemSelectorCriterion}})
and return type
(\href{https://docs.liferay.com/dxp/apps/item-selector/latest/javadocs/com/liferay/item/selector/ItemSelectorReturnType.html}{\texttt{ItemSelectorReturnType}})
classes that developers can use in Item Selectors. The following
sections in this document list the available classes:

\begin{itemize}
\tightlist
\item
  \hyperref[item-selector-criterion-classes]{Criterion Classes}
\item
  \hyperref[item-selector-return-type-classes]{Return Type Classes}
\end{itemize}

If there isn't a criterion or return type for your needs, you can create
your own by following the instructions in
\href{/docs/7-2/frameworks/-/knowledge_base/f/creating-custom-criterion-and-return-types}{Creating
Custom Criterion and Return Types}. For more information on Item
Selectors in general, including definitions of criterion and return
types, see the
\href{/docs/7-2/frameworks/-/knowledge_base/f/item-selector}{Item
Selector introduction}.

\section{Item Selector Criterion
Classes}\label{item-selector-criterion-classes}

\textbf{Assets:}

\href{https://docs.liferay.com/dxp/apps/asset/latest/javadocs/com/liferay/asset/display/page/item/selector/criterion/AssetDisplayPageSelectorCriterion.html}{AssetDisplayPageSelectorCriterion}:
Asset display page.

\textbf{Blogs:}

\href{https://docs.liferay.com/dxp/apps/blogs/latest/javadocs/com/liferay/blogs/item/selector/criterion/BlogsItemSelectorCriterion.html}{BlogsItemSelectorCriterion}:
Blogs item.

\textbf{Page Fragments:}

\href{https://docs.liferay.com/dxp/apps/fragment/latest/javadocs/com/liferay/fragment/item/selector/criterion/FragmentItemSelectorCriterion.html}{FragmentItemSelectorCriterion}:
\href{/docs/7-2/frameworks/-/knowledge_base/f/page-fragments}{Page
fragment}.

\textbf{Item Selector:}

\href{https://docs.liferay.com/dxp/apps/item-selector/latest/javadocs/com/liferay/item/selector/criteria/audio/criterion/AudioItemSelectorCriterion.html}{AudioItemSelectorCriterion}:
Audio file.

\href{https://docs.liferay.com/dxp/apps/item-selector/latest/javadocs/com/liferay/item/selector/criteria/file/criterion/FileItemSelectorCriterion.html}{FileItemSelectorCriterion}:
Document Library file.

\href{https://docs.liferay.com/dxp/apps/item-selector/latest/javadocs/com/liferay/item/selector/criteria/image/criterion/ImageItemSelectorCriterion.html}{ImageItemSelectorCriterion}:
Image file.

\href{https://docs.liferay.com/dxp/apps/item-selector/latest/javadocs/com/liferay/item/selector/criteria/upload/criterion/UploadItemSelectorCriterion.html}{UploadItemSelectorCriterion}:
Uploadable file.

\href{https://docs.liferay.com/dxp/apps/item-selector/latest/javadocs/com/liferay/item/selector/criteria/url/criterion/URLItemSelectorCriterion.html}{URLItemSelectorCriterion}:
URL.

\href{https://docs.liferay.com/dxp/apps/item-selector/latest/javadocs/com/liferay/item/selector/criteria/video/criterion/VideoItemSelectorCriterion.html}{VideoItemSelectorCriterion}:
Video file.

\textbf{Journal (Web Content):}

\href{https://docs.liferay.com/dxp/apps/journal/latest/javadocs/com/liferay/journal/item/selector/criterion/JournalItemSelectorCriterion.html}{JournalItemSelectorCriterion}:
Web content article.

\textbf{Knowledge Base:}

\href{https://docs.liferay.com/dxp/apps/knowledge-base/latest/javadocs/com/liferay/knowledge/base/item/selector/criterion/KBAttachmentItemSelectorCriterion.html}{KBAttachmentItemSelectorCriterion}:
Knowledge base attachment.

\textbf{Layout:}

\href{https://docs.liferay.com/dxp/apps/layout/latest/javadocs/com/liferay/layout/item/selector/criterion/LayoutItemSelectorCriterion.html}{LayoutItemSelectorCriterion}:
Page layout.

\textbf{Organizations:}

\href{https://docs.liferay.com/dxp/apps/organizations/latest/javadocs/com/liferay/organizations/item/selector/OrganizationItemSelectorCriterion.html}{OrganizationItemSelectorCriterion}:
Organization.

\textbf{Roles:}

\href{https://docs.liferay.com/dxp/apps/roles/latest/javadocs/com/liferay/roles/item/selector/RoleItemSelectorCriterion.html}{RoleItemSelectorCriterion}:
Role.

\textbf{Site Navigation:}

\href{https://docs.liferay.com/dxp/apps/site-navigation/latest/javadocs/com/liferay/site/navigation/item/selector/criterion/SiteNavigationMenuItemItemSelectorCriterion.html}{SiteNavigationMenuItemItemSelectorCriterion}:
Site navigation menu item.

\href{https://docs.liferay.com/dxp/apps/site-navigation/latest/javadocs/com/liferay/site/navigation/item/selector/criterion/SiteNavigationMenuItemSelectorCriterion.html}{SiteNavigationMenuItemSelectorCriterion}:
Site navigation menu.

\textbf{Sites:}

\href{https://docs.liferay.com/dxp/apps/site/latest/javadocs/com/liferay/site/item/selector/criterion/SiteItemSelectorCriterion.html}{SiteItemSelectorCriterion}:
Site.

\textbf{User Groups Admin:}

\href{https://docs.liferay.com/dxp/apps/user-groups-admin/latest/javadocs/com/liferay/user/groups/admin/item/selector/UserGroupItemSelectorCriterion.html}{UserGroupItemSelectorCriterion}:
User group.

\textbf{Users Admin:}

\href{https://docs.liferay.com/dxp/apps/users-admin/latest/javadocs/com/liferay/users/admin/item/selector/UserItemSelectorCriterion.html}{UserItemSelectorCriterion}:
User.

\textbf{Wiki:}

\href{https://docs.liferay.com/dxp/apps/wiki/latest/javadocs/com/liferay/wiki/item/selector/criterion/WikiAttachmentItemSelectorCriterion.html}{WikiAttachmentItemSelectorCriterion}:
Wiki attachment.

\href{https://docs.liferay.com/dxp/apps/wiki/latest/javadocs/com/liferay/wiki/item/selector/criterion/WikiPageItemSelectorCriterion.html}{WikiPageItemSelectorCriterion}:
Wiki page.

\section{Item Selector Return Type
Classes}\label{item-selector-return-type-classes}

\textbf{Adaptive Media:}

\href{https://docs.liferay.com/dxp/apps/adaptive-media/latest/javadocs/com/liferay/adaptive/media/image/item/selector/AMImageFileEntryItemSelectorReturnType.html}{AMImageFileEntryItemSelectorReturnType}:
Adaptive Media image file.

\href{https://docs.liferay.com/dxp/apps/adaptive-media/latest/javadocs/com/liferay/adaptive/media/image/item/selector/AMImageURLItemSelectorReturnType.html}{AMImageURLItemSelectorReturnType}:
Adaptive Media image URL.

\textbf{Item Selector:}

\href{https://docs.liferay.com/dxp/apps/item-selector/latest/javadocs/com/liferay/item/selector/criteria/Base64ItemSelectorReturnType.html}{Base64ItemSelectorReturnType}:
The entity's Base64 encoding as a \texttt{String}.

\href{https://docs.liferay.com/dxp/apps/item-selector/latest/javadocs/com/liferay/item/selector/criteria/DownloadURLItemSelectorReturnType.html}{DownloadURLItemSelectorReturnType}:
The entity's download URL as a \texttt{String}.

\href{https://docs.liferay.com/dxp/apps/item-selector/latest/javadocs/com/liferay/item/selector/criteria/FileEntryItemSelectorReturnType.html}{FileEntryItemSelectorReturnType}:
File entry information as a JSON object.

\href{https://docs.liferay.com/dxp/apps/item-selector/latest/javadocs/com/liferay/item/selector/criteria/URLItemSelectorReturnType.html}{URLItemSelectorReturnType}:
The entity's URL as a \texttt{String}.

\href{https://docs.liferay.com/dxp/apps/item-selector/latest/javadocs/com/liferay/item/selector/criteria/UUIDItemSelectorReturnType.html}{UUIDItemSelectorReturnType}:
The entity's universally unique identifier (UUID) as a \texttt{String}.

\textbf{Site:}

\href{https://docs.liferay.com/dxp/apps/site/latest/javadocs/com/liferay/site/item/selector/criteria/SiteItemSelectorReturnType.html}{SiteItemSelectorReturnType}:
The Site's information as a JSON object.

\textbf{Wiki:}

\href{https://docs.liferay.com/dxp/apps/wiki/latest/javadocs/com/liferay/wiki/item/selector/WikiPageTitleItemSelectorReturnType.html}{WikiPageTitleItemSelectorReturnType}:
The wiki page's title.

\href{https://docs.liferay.com/dxp/apps/wiki/latest/javadocs/com/liferay/wiki/item/selector/WikiPageURLItemSelectorReturnType.html}{WikiPageURLItemSelectorReturnType}:
The wiki page's URL.

\chapter{Java APIs}\label{java-apis}

Here you'll find Javadoc for Liferay DXP and Liferay DXP apps. Note that
each link to the Javadoc listed here opens in a new window.

For help finding module attributes and configuring dependencies, see
\href{/docs/7-2/customization/-/knowledge_base/c/configuring-dependencies}{Configuring
Dependencies}.

\section{7.0 Java APIs}\label{java-apis-1}

This table contains links to the Javadoc for 7.0 API modules. The root
location for these modules' Javadoc is
\href{https://docs.liferay.com/dxp/portal/7.2-latest/javadocs/}{here}.

Core

com.liferay.portal.kernel (portal-kernel): ~for developing applications
on Liferay DXP

com.liferay.util.bridges (util-bridges): ~for using various
non-proprietary computing languages, frameworks, and utilities on
Liferay DXP

com.liferay.util.java (util-java): ~for using various Java-related
frameworks and utilities on Liferay DXP

com.liferay.util.slf4j (util-slf4j): ~for using the Simple Logging
Facade for Java (SLF4J)

com.liferay.portal.impl (portal-impl): ~refer to this only if you are an
advanced Liferay developer that needs a deeper understanding of 7.0's
implementation in order to contribute to it

\section{Liferay DXP App Java APIs}\label{liferay-dxp-app-java-apis}

The tables in this section link to the API modules for apps in these
categories:

\begin{itemize}
\tightlist
\item
  \hyperref[collaboration]{Collaboration}
\item
  \hyperref[forms-and-workflow]{Forms and Workflow}
\item
  \hyperref[foundation]{Foundation}
\item
  \hyperref[web-experience]{Web Experience}
\end{itemize}

Note that the root location for these modules' Javadoc is
{[}https://docs.liferay.com/dxp/apps{]}( \#\# Collaboration

Announcements

com.liferay.announcements.api

Blogs

com.liferay.blogs.api

com.liferay.blogs.item.selector.api

com.liferay.blogs.recent.bloggers.api

Comment

com.liferay.comment.api

Document Library

com.liferay.document.library.api

com.liferay.document.library.content.api

com.liferay.document.library.file.rank.api

com.liferay.document.library.repository.authorization.api

com.liferay.document.library.repository.cmis.api

com.liferay.document.library.repository.external.api

com.liferay.document.library.sync.api

Flags

com.liferay.flags.api

Invitation

com.liferay.invitation.invite.members.api

Item Selector

com.liferay.item.selector.api

com.liferay.item.selector.criteria.api

Mentions

com.liferay.mentions.api

Message Boards

com.liferay.message.boards.api

Ratings

com.liferay.ratings.api

Reading Time

com.liferay.reading.time.api

Social

com.liferay.social.activities.api

com.liferay.social.activity.api

com.liferay.social.bookmarks.api

com.liferay.social.user.statistics.api

Subscription

com.liferay.subscription.api

Upload

com.liferay.upload.api

Wiki

com.liferay.wiki.api

\section{Forms and Workflow}\label{forms-and-workflow}

Calendar

com.liferay.calendar.api

Dynamic Data Lists

com.liferay.dynamic.data.lists.api

Dynamic Data Mapping

com.liferay.dynamic.data.mapping.api

Polls

com.liferay.polls.api

Portal Reports Engine

com.liferay.portal.reports.engine.api

Portal Rules Engine

com.liferay.portal.rules.engine.api

Portal Workflow

com.liferay.portal.workflow.api

com.liferay.portal.workflow.kaleo.api

com.liferay.portal.workflow.kaleo.definition.api

com.liferay.portal.workflow.kaleo.runtime.api

\section{Foundation}\label{foundation}

Captcha

com.liferay.captcha.api

Configuration Admin

com.liferay.configuration.admin.api

Contacts

com.liferay.contacts.api

Friendly URL

com.liferay.friendly.url.api

Frontend Editor

com.liferay.frontend.editor.api

Frontend Image Editor

com.liferay.frontend.image.editor.api

Frontend JS

com.liferay.frontend.js.loader.modules.extender.api

Map

com.liferay.map.api

Mobile Device Rules

com.liferay.mobile.device.rules.api

Organizations

com.liferay.organizations.api

com.liferay.organizations.item.selector.api

Password Policies Admin

com.liferay.password.policies.admin.api

Portal Background Task

com.liferay.portal.background.task.api

Portal Cache

com.liferay.portal.cache.api

Portal Configuration

com.liferay.portal.configuration.upgrade.api

Portal Instances

com.liferay.portal.instances.api

Portal Lock

com.liferay.portal.lock.api

Portal Remote

com.liferay.portal.remote.soap.extender.api

Portal Scripting

com.liferay.portal.scripting.api

Portal Search

com.liferay.portal.search.api

com.liferay.portal.search.engine.adapter.api

com.liferay.portal.search.web.api

Portal Security Audit

com.liferay.portal.security.audit.api

com.liferay.portal.security.audit.event.generators.api

com.liferay.portal.security.audit.storage.api

Portal Security SSO

com.liferay.portal.security.sso.cas.api

com.liferay.portal.security.sso.facebook.connect.api

com.liferay.portal.security.sso.ntlm.api

com.liferay.portal.security.sso.openid.api

com.liferay.portal.security.sso.openid.connect.api

com.liferay.portal.security.sso.opensso.api

com.liferay.portal.security.sso.token.api

Portal Security

com.liferay.portal.security.exportimport.api

com.liferay.portal.security.ldap.api

com.liferay.portal.security.permission.api

com.liferay.portal.security.service.access.policy.api

com.liferay.portal.security.service.access.quota.api

Portal Security SSO Google

com.liferay.portal.security.sso.google.api

Portal Settings

com.liferay.portal.settings.api

Portal Template

com.liferay.portal.template.soy.api

Portal URL Builder

com.liferay.portal.url.builder.api

Portal

com.liferay.portal.custom.jsp.bag.api

com.liferay.portal.dao.orm.custom.sql.api

com.liferay.portal.instance.lifecycle.api

com.liferay.portal.jmx.api

com.liferay.portal.output.stream.container.api

com.liferay.portal.spring.extender.api

com.liferay.portal.upgrade.api

Roles

com.liferay.roles.admin.api

com.liferay.roles.item.selector.api

Text Localizer

com.liferay.text.localizer.address.api

User-associated Data

com.liferay.user.associated.data.api

User Groups Admin

com.liferay.user.groups.admin.api

com.liferay.user.groups.admin.item.selector.api

Users Admin

com.liferay.users.admin.api

com.liferay.users.admin.item.selector.api

XStream

com.liferay.xstream.configurator.api

\section{Web Experience}\label{web-experience}

Application List

com.liferay.application.list.api

Asset

com.liferay.asset.api

com.liferay.asset.categories.navigation.api

com.liferay.asset.category.property.api

com.liferay.asset.display.api

com.liferay.asset.display.page.api

com.liferay.asset.display.page.item.selector.api

com.liferay.asset.entry.rel.api

com.liferay.asset.publisher.api

com.liferay.asset.tag.stats.api

com.liferay.asset.tags.api

com.liferay.asset.tags.navigation.api

Export Import

com.liferay.exportimport.api

com.liferay.exportimport.changeset.api

Fragment

com.liferay.fragment.api

com.liferay.fragment.item.selector.api

HTML Preview

com.liferay.html.preview.api

Journal

com.liferay.journal.api

com.liferay.journal.content.asset.addon.entry.api

com.liferay.journal.item.selector.api

Layout

com.liferay.layout.api

com.liferay.layout.admin.api

com.liferay.layout.item.selector.api

com.liferay.layout.page.template.api

com.liferay.layout.prototype.api

com.liferay.layout.set.prototype.api

Portlet Display Template

com.liferay.portlet.display.template.api

Product Navigation

com.liferay.product.navigation.control.menu.api

com.liferay.product.navigation.product.menu.api

com.liferay.product.navigation.simulation.api

RSS

com.liferay.rss.api

Site Navigation

com.liferay.site.navigation.api

com.liferay.site.navigation.admin.api

com.liferay.site.navigation.item.selector.api

com.liferay.site.navigation.language.api

Site

com.liferay.site.api

com.liferay.site.item.selector.api

Staging

com.liferay.staging.api

Trash

com.liferay.trash.api

\section{JavaScript and CSS}\label{javascript-and-css}

\href{https://liferay.design/lexicon/}{\textbf{Lexicon}}: A system for
building applications in and outside of Liferay DXP, designed to be
fluid and extensible, as well as provide a consistent and documented
API.

\href{https://clayui.com/}{\textbf{Clay}}: The web implementation of
Lexicon.

\href{http://getbootstrap.com/}{\textbf{Bootstrap}}: The base CSS
library onto which Lexicon is built. Liferay DXP uses Bootstrap natively
and all of its CSS classes and JavaScript features are available within
portlets, templates, and themes.

\href{http://alloyui.com}{\textbf{AlloyUI}}: AlloyUI and all of its
JavaScript APIs are available within portlets, templates, and themes.

\section{Descriptor Definitions}\label{descriptor-definitions}

\href{https://docs.liferay.com/dxp/portal/7.2-latest/definitions/}{\textbf{DTDs}}:
Describes the XML files used in configuring Liferay DXP apps, 7.0
plugins, and Liferay DXP 7.2.

\chapter{Meaningful Schema
Versioning}\label{meaningful-schema-versioning}

Liferay's data schema version convention communicates a schema's
compatibility with older versions of the software. It tells you whether
a schema's changes maintain or break compatibility with existing
software. For example, if a new data schema removes a field your
software expects, the schema breaks compatibility. But if a new schema's
changes are non-breaking (e.g., adds a new field), the schema is
compatible and can be used with existing software.

Since Liferay DXP 7.1, Liferay uses a meaningful schema version
convention (similar to \href{http://semver.org}{Semantic Versioning}) to
define new
\href{/docs/7-2/frameworks/-/knowledge_base/f/creating-an-upgrade-process-for-your-app}{upgrade
steps} and support rollback of schema micro versions. The schema version
defines the status of the database schema and its data belonging to that
module or Core in a certain moment. The concept of schema versioning is
different from bundle versioning. The biggest concern in versioning a
data schema is \textbf{backward-compatibility} between the new schema
and the code that operates on the data.

Here's Liferay's schema version convention:

\textbf{MAJOR.MINOR.MICRO}

Each part means something:

\textbf{MAJOR:} Contains breaking schema/data changes that are
incompatible with the previous version of the code.

\textbf{MINOR:} Contains schema/data changes compatible with the
previous version of the code. The changes are required for the new
version of the code to work (the application will fail without applying
the schema/data changes)

\textbf{MICRO:} Contains schema/data changes that are compatible with
the previous version of the code. The changes are optional.

If you're not sure what kind of schema version change represents your
upgrade step, ask yourself these questions:

\begin{enumerate}
\def\labelenumi{\arabic{enumi}.}
\item
  Will the previous code version (previous FP, SP, or GA) work with
  these schema/data changes?

  \begin{itemize}
  \tightlist
  \item
    If not, it is a major change.
  \item
    If yes, continue.
  \end{itemize}
\item
  Are the schema/data changes required for the application to work?
  (Obviously, all changes are intended to improve the application but in
  some cases the application is fully functional without them)

  \begin{itemize}
  \tightlist
  \item
    If yes, it is a minor change.
  \item
    If not, it is a micro change.
  \end{itemize}
\end{enumerate}

Next are some concrete examples of micro, minor, and major changes.

\section{Micro change examples}\label{micro-change-examples}

Here are common micro changes:

\begin{itemize}
\tightlist
\item
  Increasing \texttt{VARCHAR} field sizes.
\item
  Modifying DB indexes.
\item
  Modifying data values to adapt to current logic. These include
  backwards compatible data changes only. These changes commonly occur
  when data updates are missed for new functionalities.
\item
  Converting a field from a String to a CLOB, as long as the field has
  few records and isn't used in \texttt{DISTINCT} or \texttt{GROUP\ BY}
  SQL clauses.
\end{itemize}

\section{Minor change examples}\label{minor-change-examples}

Here are common minor changes:

\begin{itemize}
\tightlist
\item
  Adding a new DB field.
\item
  Adding a new DB table.
\end{itemize}

\textbf{Important:} The changes above are major if they require
modifying current existing data or extract information to populate the
new field or table. In such cases, the data can become incorrect if you
rolled back to the previous code version and then, after some time,
installed the new code again.

\section{Major change examples}\label{major-change-examples}

Here are common major changes:

\begin{itemize}
\tightlist
\item
  Making data modifications that are not backward compatible.
\item
  Removing a DB field
\item
  Removing a DB table.
\item
  Altering a column name.
\item
  Decreasing the size of a \texttt{VARCHAR} field.
\item
  Converting a field from a String to a CLOB, where the field is has
  many records or is used in \texttt{DISTINCT} or \texttt{GROUP\ BY} SQL
  clauses.
\item
  Adding a new DB field or table that requires modifying current
  existing data or extracts information to populate the new field or
  table.
\end{itemize}

Now you can ascribe meaningful versions to your module's data schemas.

\chapter{Portlet 3.0 API Opt In}\label{portlet-3.0-api-opt-in}

A portlet must specify version 3.0 to ``opt in'' to the Portlet 3.0 API.
The 3.0 Portlet API version can be specified in the following ways.

\section{\texorpdfstring{Standard Portlet \texttt{@PortletApplication}
Annotation}{Standard Portlet @PortletApplication Annotation}}\label{standard-portlet-portletapplication-annotation}

Standard portlets need only specify the
\href{https://docs.liferay.com/portlet-api/3.0/javadocs/javax/portlet/annotations/PortletApplication.html}{\texttt{@PortletApplication}}
annotation.

\begin{verbatim}
@PortletApplication(version="3.0") // 3.0 is the default for this annotation attribute
@PortletConfiguration(portletName="myPortlet")
public class MyPortlet {
    ...
}
\end{verbatim}

\section{\texorpdfstring{Liferay MVC Portlet \texttt{@Component}
Annotation}{Liferay MVC Portlet @Component Annotation}}\label{liferay-mvc-portlet-component-annotation}

Declarative Services portlets, such as \texttt{MVCPortlet}, can specify
version 3.0 in their \texttt{@Component} annotation.

\begin{verbatim}
@Component(properties="javax.portlet.version=3.0", service=javax.portlet.Portlet.class)
public class MyDeclarativeServicesPortlet {
    ...
}
\end{verbatim}

\section{\texorpdfstring{\texttt{portlet.xml}
Descriptor}{portlet.xml Descriptor}}\label{portlet.xml-descriptor}

All portlets can specify version 3.0 in their \texttt{portlet.xml}
descriptor.

\begin{verbatim}
<?xml version="1.0"?>

<portlet-app xmlns="http://xmlns.jcp.org/xml/ns/portlet"
             xmlns:xsi="http://www.w3.org/2001/XMLSchema-instance"
             xsi:schemaLocation="http://xmlns.jcp.org/xml/ns/portlet http://xmlns.jcp.org/xml/ns/portlet/portlet-app_3_0.xsd"
             version="3.0">
    ...
</portlet-app>
\end{verbatim}

\chapter{Portlet Descriptor to OSGi Service Property
Map}\label{portlet-descriptor-to-osgi-service-property-map}

{ This document has been updated and ported to Liferay Learn and is no
longer maintained here.}

This article maps portlet XML descriptor values to OSGi service
properties for publishing OSGi Portlets.

OSGi service definitions can use properties. OSGi service properties
centralize and simplify portlet configuration. They are typically
represented as key-value pairs or, more generally, as a Map-like object.

Portlet spec property keys are prefixed by

\begin{verbatim}
javax.portlet.
\end{verbatim}

Liferay property keys are prefixed by

\begin{verbatim}
com.liferay.portlet.
\end{verbatim}

The mappings essentially flatten what is found in the XML descriptor.
The property names resemble the original descriptor names.

This article covers these descriptor mappings:

\begin{itemize}
\item
  \hyperref[portlet-descriptor-mappings]{Portlet descriptor mappings}
\item
  \hyperref[liferay-descriptor-mappings]{Liferay descriptor mappings}

  \begin{itemize}
  \item
    \hyperref[liferay-display]{From \texttt{liferay-display.xml}}
  \item
    \hyperref[liferay-portlet]{From \texttt{liferay-portlet.xml}}
  \end{itemize}
\end{itemize}

The standard portlet descriptor mappings are first.

\section{Portlet Descriptor Mappings}\label{portlet-descriptor-mappings}

\textbf{Note:} XPath notation derived from the \textbf{Portlet XSD}
\hyperref[four]{4} is used in this document for simplicity.

\noindent\hrulefill

portlet.xml XPath \textbar{} OSGi Portlet Service Property\textbar{}
\texttt{/portlet-app/container-runtime-option}\textbar not
supported\textbar{}
\texttt{/portlet-app/custom-portlet-mode}\textbar not
supported\textbar{}
\texttt{/portlet-app/custom-window-state}\textbar not
supported\textbar{}
\texttt{/portlet-app/default-namespace}\textbar{}\texttt{javax.portlet.default-namespace=\textless{}String\textgreater{}}\textbar{}
\texttt{/portlet-app/event-definition}\textbar{}\texttt{javax.portlet.event-definition=\textless{}QNameLocalPart\textgreater{};\textless{}QNameURI\textgreater{}{[};\textless{}PayloadType\textgreater{}{]}{[},\textless{}AliasQNameLocalPart\textgreater{};\textless{}AliasQNameURI\textgreater{}{]}}
\hyperref[two]{2}\textbar{}
\texttt{/portlet-app/filter}\texttt{/portlet-app/filter/init-param/name}\texttt{/portlet-app/filter-mapping}\textbar{}\hyperref[three]{3}\texttt{javax.portlet.init-param.\textless{}name\textgreater{}=\textless{}value\textgreater{}}
\hyperref[three]{3}, \hyperref[nine]{9}\hyperref[three]{3}\textbar{}
\texttt{/portlet-app/public-render-parameter}\textbar not
supported\textbar{} \texttt{/portlet-app/resource-bundle}\textbar not
supported\textbar{}
\texttt{/portlet-app/security-constraint}\textbar not
supported\textbar{} \texttt{/portlet-app/user-attribute}\textbar not
supported\textbar{}
\texttt{/portlet-app/version}\textbar{}\texttt{javax.portlet.version=\textless{}value\textgreater{}}\textbar{}
\texttt{/portlet-app/portlet/async-supported}\textbar{}\texttt{javax.portlet.async-supported=\textless{}boolean\textgreater{}}\textbar{}
\texttt{/portlet-app/portlet/cache-scope}\textbar not
supported\textbar{}
\texttt{/portlet-app/portlet/container-runtime-option}\textbar{}\texttt{javax.portlet.container-runtime-option.\textless{}name\textgreater{}=\textless{}value\textgreater{}}
\hyperref[two]{2}\textbar{}
\texttt{/portlet-app/portlet/dependency}\textbar{}\texttt{javax.portlet.dependency=\textless{}name\textgreater{};\textless{}scope\textgreater{};\textless{}version\textgreater{}}
\hyperref[two]{2}, \hyperref[six]{6}\textbar{}
\texttt{/portlet-app/portlet/description}\textbar{}\texttt{javax.portlet.description=\textless{}String\textgreater{}}\textbar{}
\texttt{/portlet-app/portlet/display-name}\textbar{}\texttt{javax.portlet.display-name=\textless{}String\textgreater{}}\textbar{}
\texttt{/portlet-app/portlet/expiration-cache}\textbar{}\texttt{javax.portlet.expiration-cache=\textless{}int\textgreater{}}\textbar{}
\texttt{/portlet-app/portlet/init-param/name}\textbar{}\texttt{javax.portlet.init-param.\textless{}name\textgreater{}=\textless{}value\textgreater{}}\textbar{}
\texttt{/portlet-app/portlet/listener}\textbar{}\texttt{javax.portlet.listener=\textless{}listener-class\textgreater{};\textless{}ordinal\textgreater{}}
\hyperref[two]{2},\hyperref[eight]{8}\textbar{}
\texttt{/portlet-app/portlet/multipart-config/file-size-threshold}\textbar{}\texttt{javax.portlet.multipart.file-size-threshold=\textless{}Integer\textgreater{}}\textbar{}
\texttt{/portlet-app/portlet/multipart-config/location}\textbar{}\texttt{javax.portlet.multipart.location=\textless{}String\textgreater{}}\textbar{}
\texttt{/portlet-app/portlet/multipart-config/max-file-size}\textbar{}\texttt{javax.portlet.multipart.max-file-size=\textless{}Long\textgreater{}}\textbar{}
\texttt{/portlet-app/portlet/multipart-config/max-request-size}\textbar{}\texttt{javax.portlet.multipart.max-request-size=\textless{}Long\textgreater{}}\textbar{}
\texttt{/portlet-app/portlet/portlet-class}\textbar{}\hyperref[one]{1}\textbar{}
\texttt{/portlet-app/portlet/portlet-info/keywords}\textbar{}\texttt{javax.portlet.info.keywords=\textless{}String\textgreater{}}\textbar{}
\texttt{/portlet-app/portlet/portlet-info/short-title}\textbar{}\texttt{javax.portlet.info.short-title=\textless{}String\textgreater{}}\textbar{}
\texttt{/portlet-app/portlet/portlet-info/title}\textbar{}\texttt{javax.portlet.info.title=\textless{}String\textgreater{}}\textbar{}
\texttt{/portlet-app/portlet/portlet-name}
\hyperref[ten]{10}\textbar{}\texttt{javax.portlet.name=\textless{}String\textgreater{}}
\hyperref[ten]{10}\textbar{}
\texttt{/portlet-app/portlet/portlet-preferences}\textbar{}\texttt{javax.portlet.preferences=\textless{}String\textgreater{}}OR\texttt{javax.portlet.preferences=classpath:\textless{}path\_to\_file\_in\_jar\textgreater{}}\textbar{}
\texttt{/portlet-app/portlet/portlet-preferences/preferences-validator}\textbar{}\texttt{javax.portlet.preferences-validator=\textless{}String\textgreater{}}
\hyperref[one]{1}\textbar{}
\texttt{/portlet-app/portlet/resource-bundle}\textbar{}\texttt{javax.portlet.resource-bundle=\textless{}String\textgreater{}}\textbar{}
\texttt{/portlet-app/portlet/security-role-ref}\textbar{}\texttt{javax.portlet.security-role-ref=\textless{}String\textgreater{}{[},\textless{}String\textgreater{}{]}}\hyperref[two]{2}\textbar{}
\texttt{/portlet-app/portlet/supported-locale}\textbar{}\texttt{javax.portlet.supported-locale=\textless{}String\textgreater{}}
\hyperref[two]{2}\textbar{}
\texttt{/portlet-app/portlet/supported-processing-event}\textbar{}\texttt{javax.portlet.supported-processing-event=\textless{}QNameLocalPart\textgreater{}}
OR
\texttt{javax.portlet.supported-processing-event=\textless{}QNameLocalPart\textgreater{};\textless{}QNameURI\textgreater{}}
\hyperref[two]{2}\textbar{}
\texttt{/portlet-app/portlet/supported-public-render-parameter}\textbar{}\texttt{javax.portlet.supported-public-render-parameter=\textless{}String\textgreater{}}\hyperref[two]{2}\textbar{}
\texttt{/portlet-app/portlet/supported-publishing-event}\textbar{}\texttt{javax.portlet.supported-publishing-event=\textless{}QNameLocalPart\textgreater{}}
OR
\texttt{javax.portlet.supported-publishing-event=\textless{}QNameLocalPart\textgreater{};\textless{}QNameURI\textgreater{}}
\hyperref[two]{2}\textbar{}
\texttt{/portlet-app/portlet/supports/mime-type}\textbar{}\texttt{javax.portlet.mime-type=\textless{}mime-type\textgreater{}}\textbar{}
\texttt{/portlet-app/portlet/supports/portlet-mode}\textbar{}\texttt{javax.portlet.portlet-mode=\textless{}mime-type\textgreater{};\textless{}portlet-mode\textgreater{}{[},\textless{}portlet-mode\textgreater{}{]}*}\textbar{}
\texttt{/portlet-app/portlet/supports/window-state}\textbar{}\texttt{javax.portlet.window-state=\textless{}mime-type\textgreater{};\textless{}window-state\textgreater{}{[},\textless{}window-state\textgreater{}{]}*}\textbar{}

\noindent\hrulefill

\section{Liferay Descriptor Mappings}\label{liferay-descriptor-mappings}

\section{Liferay Display}\label{liferay-display}

\noindent\hrulefill

liferay-display.xml XPath \textbar{} OSGi Portlet Service
Property\textbar{}
\texttt{/display/category{[}@name{]}}\textbar{}\texttt{com.liferay.portlet.display-category=\textless{}value\textgreater{}}\textbar{}

\noindent\hrulefill

\section{Liferay Portlet}\label{liferay-portlet}

\textbf{Note:} XPath notation derived from \textbf{Liferay Portlet}
\hyperref[five]{5} is used in this document for simplicity.

\noindent\hrulefill

liferay-portlet.xml XPath \textbar{} OSGi Liferay Portlet Service
Property\textbar{}
\texttt{/liferay-portlet-app/portlet/action-timeout}\textbar{}\texttt{com.liferay.portlet.action-timeout=\textless{}int\textgreater{}}\textbar{}
\texttt{/liferay-portlet-app/portlet/action-url-redirect}\textbar{}\texttt{com.liferay.portlet.action-url-redirect=\textless{}boolean\textgreater{}}\textbar{}
\texttt{/liferay-portlet-app/portlet/active}\textbar{}\texttt{com.liferay.portlet.active=\textless{}boolean\textgreater{}}\textbar{}
\texttt{/liferay-portlet-app/portlet/add-default-resource}\textbar{}\texttt{com.liferay.portlet.add-default-resource=\textless{}boolean\textgreater{}}\textbar{}
\texttt{/liferay-portlet-app/portlet/ajaxable}\textbar{}\texttt{com.liferay.portlet.ajaxable=\textless{}boolean\textgreater{}}\textbar{}
\texttt{/liferay-portlet-app/portlet/asset-renderer-factory}\textbar{}\hyperref[three]{3}\textbar{}
\texttt{/liferay-portlet-app/portlet/atom-collection-adapter}\textbar{}\hyperref[three]{3}\textbar{}
\texttt{/liferay-portlet-app/portlet/autopropagated-parameters}\textbar{}\texttt{com.liferay.portlet.autopropagated-parameters=\textless{}String\textgreater{}}\hyperref[two]{2}\textbar{}
\texttt{/liferay-portlet-app/portlet/configuration-action-class}\textbar{}\hyperref[three]{3}\textbar{}
\texttt{/liferay-portlet-app/portlet/configuration-path}\textbar{}\texttt{com.liferay.portlet.configuration-path=\textless{}String\textgreater{}}\textbar{}
\texttt{/liferay-portlet-app/portlet/control-panel-entry-category}\textbar{}\texttt{com.liferay.portlet.control-panel-entry-category=\textless{}String\textgreater{}}\textbar{}
\texttt{/liferay-portlet-app/portlet/control-panel-entry-class}\textbar{}\hyperref[three]{3}\textbar{}
\texttt{/liferay-portlet-app/portlet/control-panel-entry-weight}\textbar{}\texttt{com.liferay.portlet.control-panel-entry-weight=\textless{}double\textgreater{}}\textbar{}
\texttt{/liferay-portlet-app/portlet/css-class-wrapper}\textbar{}\texttt{com.liferay.portlet.css-class-wrapper=\textless{}String\textgreater{}}\textbar{}
\texttt{/liferay-portlet-app/portlet/custom-attributes-display}\textbar{}\hyperref[three]{3}\textbar{}
\texttt{/liferay-portlet-app/portlet/ddm-display}\textbar{}\hyperref[three]{3}\textbar{}
\texttt{/liferay-portlet-app/portlet/facebook-integration}\textbar{}\texttt{com.liferay.portlet.facebook-integration=\textless{}String\textgreater{}}\textbar{}
\texttt{/liferay-portlet-app/portlet/footer-portal-css}\textbar{}\texttt{com.liferay.portlet.footer-portal-css=\textless{}String\textgreater{}}\hyperref[two]{2}\textbar{}
\texttt{/liferay-portlet-app/portlet/footer-portal-javascript}\textbar{}\texttt{com.liferay.portlet.footer-portal-javascript=\textless{}String\textgreater{}}\hyperref[two]{2}\textbar{}
\texttt{/liferay-portlet-app/portlet/footer-portlet-css}\textbar{}\texttt{com.liferay.portlet.footer-portlet-css=\textless{}String\textgreater{}}\hyperref[two]{2}\textbar{}
\texttt{/liferay-portlet-app/portlet/footer-portlet-javascript}\textbar{}\texttt{com.liferay.portlet.footer-portlet-javascript=\textless{}String\textgreater{}}\hyperref[two]{2}\textbar{}
\texttt{/liferay-portlet-app/portlet/friendly-url-mapper-class}\textbar{}\hyperref[three]{3}\textbar{}
\texttt{/liferay-portlet-app/portlet/friendly-url-mapping}\textbar{}\texttt{com.liferay.portlet.friendly-url-mapping=\textless{}String\textgreater{}}\textbar{}
\texttt{/liferay-portlet-app/portlet/friendly-url-routes}\textbar{}\texttt{com.liferay.portlet.friendly-url-routes=\textless{}String\textgreater{}}\textbar{}
\texttt{/liferay-portlet-app/portlet/header-portal-css}\textbar{}\texttt{com.liferay.portlet.header-portal-css=\textless{}String\textgreater{}}\hyperref[two]{2}\textbar{}
\texttt{/liferay-portlet-app/portlet/header-portal-javascript}\textbar{}\texttt{com.liferay.portlet.header-portal-javascript=\textless{}String\textgreater{}}\hyperref[two]{2}\textbar{}
\texttt{/liferay-portlet-app/portlet/header-portlet-css}\textbar{}\texttt{com.liferay.portlet.header-portlet-css=\textless{}String\textgreater{}}\hyperref[two]{2}\textbar{}
\texttt{/liferay-portlet-app/portlet/header-portlet-javascript}\textbar{}\texttt{com.liferay.portlet.header-portlet-javascript=\textless{}String\textgreater{}}\hyperref[two]{2}\textbar{}
\texttt{/liferay-portlet-app/portlet/header-request-attribute-prefix}\textbar{}\texttt{com.liferay.portlet.header-request-attribute-prefix=\textless{}String\textgreater{}}
\hyperref[seven]{7}\textbar{}
\texttt{/liferay-portlet-app/portlet/header-timeout}\textbar{}\texttt{header-timeout=\textless{}int\textgreater{}}\textbar{}
\texttt{/liferay-portlet-app/portlet/icon}\textbar{}\texttt{com.liferay.portlet.icon=\textless{}String\textgreater{}}\textbar{}
\texttt{/liferay-portlet-app/portlet/include}\textbar not
supported\textbar{}
\texttt{/liferay-portlet-app/portlet/indexer-class}\textbar{}\hyperref[three]{3}\textbar{}
\texttt{/liferay-portlet-app/portlet/instanceable}\textbar{}\texttt{com.liferay.portlet.instanceable=\textless{}boolean\textgreater{}}\textbar{}
\texttt{/liferay-portlet-app/portlet/layout-cacheable}\textbar{}\texttt{com.liferay.portlet.layout-cacheable=\textless{}boolean\textgreater{}}\textbar{}
\texttt{/liferay-portlet-app/portlet/maximize-edit}\textbar{}\texttt{com.liferay.portlet.maximize-edit=\textless{}boolean\textgreater{}}\textbar{}
\texttt{/liferay-portlet-app/portlet/maximize-help}\textbar{}\texttt{com.liferay.portlet.maximize-help=\textless{}boolean\textgreater{}}\textbar{}
\texttt{/liferay-portlet-app/portlet/open-search-class}\textbar{}\hyperref[three]{3}\textbar{}
\texttt{/liferay-portlet-app/portlet/parent-struts-path}\textbar{}\texttt{com.liferay.portlet.parent-struts-path=\textless{}String\textgreater{}}\textbar{}
\texttt{/liferay-portlet-app/portlet/permission-propagator}\textbar{}\hyperref[three]{3}\textbar{}
\texttt{/liferay-portlet-app/portlet/poller-processor-class}\textbar{}\hyperref[three]{3}\textbar{}
\texttt{/liferay-portlet-app/portlet/pop-message-listener-class}\textbar{}\hyperref[three]{3}\textbar{}
\texttt{/liferay-portlet-app/portlet/pop-up-print}\textbar{}\texttt{com.liferay.portlet.pop-up-print=\textless{}boolean\textgreater{}}\textbar{}
\texttt{/liferay-portlet-app/portlet/portlet-data-handler-class}\textbar{}\hyperref[three]{3}\textbar{}
\texttt{/liferay-portlet-app/portlet/portlet-layout-listener-class}\textbar{}\hyperref[three]{3}\textbar{}
\texttt{/liferay-portlet-app/portlet/portlet-name}\textbar not
supported\textbar{}
\texttt{/liferay-portlet-app/portlet/portlet-url-class}\textbar{}\hyperref[three]{3}\textbar{}
\texttt{/liferay-portlet-app/portlet/preferences-company-wide}\textbar{}\texttt{com.liferay.portlet.preferences-company-wide=\textless{}boolean\textgreater{}}\textbar{}
\texttt{/liferay-portlet-app/portlet/preferences-owned-by-group}\textbar{}\texttt{com.liferay.portlet.preferences-owned-by-group=\textless{}boolean\textgreater{}}\textbar{}
\texttt{/liferay-portlet-app/portlet/preferences-unique-per-layout}\textbar{}\texttt{com.liferay.portlet.preferences-unique-per-layout=\textless{}boolean\textgreater{}}\textbar{}
\texttt{/liferay-portlet-app/portlet/private-request-attributes}\textbar{}\texttt{com.liferay.portlet.private-request-attributes=\textless{}boolean\textgreater{}}\textbar{}
\texttt{/liferay-portlet-app/portlet/private-session-attributes}\textbar{}\texttt{com.liferay.portlet.private-session-attributes=\textless{}boolean\textgreater{}}\textbar{}
\texttt{/liferay-portlet-app/portlet/remoteable}\textbar{}\texttt{com.liferay.portlet.remoteable=\textless{}boolean\textgreater{}}\textbar{}
\texttt{/liferay-portlet-app/portlet/render-timeout}\textbar{}\texttt{com.liferay.portlet.render-timeout=\textless{}int\textgreater{}}\textbar{}
\texttt{/liferay-portlet-app/portlet/render-weight}\textbar{}\texttt{com.liferay.portlet.render-weight=\textless{}int\textgreater{}}\textbar{}
\texttt{/liferay-portlet-app/portlet/requires-namespaced-parameters}\textbar{}\texttt{com.liferay.portlet.requires-namespaced-parameters=\textless{}boolean\textgreater{}}\textbar{}
\texttt{/liferay-portlet-app/portlet/restore-current-view}\textbar{}\texttt{com.liferay.portlet.restore-current-view=\textless{}boolean\textgreater{}}\textbar{}
\texttt{/liferay-portlet-app/portlet/scheduler-entry}\textbar{}\hyperref[three]{3}\textbar{}
\texttt{/liferay-portlet-app/portlet/scopeable}\textbar{}\texttt{com.liferay.portlet.scopeable=\textless{}boolean\textgreater{}}\textbar{}
\texttt{/liferay-portlet-app/portlet/show-portlet-access-denied}\textbar{}\texttt{com.liferay.portlet.show-portlet-access-denied=\textless{}boolean\textgreater{}}\textbar{}
\texttt{/liferay-portlet-app/portlet/show-portlet-inactive}\textbar{}\texttt{com.liferay.portlet.show-portlet-inactive=\textless{}boolean\textgreater{}}\textbar{}
\texttt{/liferay-portlet-app/portlet/single-page-application}\textbar{}\texttt{com.liferay.portlet.single-page-application=\textless{}boolean\textgreater{}}\textbar{}
\texttt{/liferay-portlet-app/portlet/social-activity-interpreter-class}\textbar{}\hyperref[three]{3}\textbar{}
\texttt{/liferay-portlet-app/portlet/social-request-interpreter-class}\textbar{}\hyperref[three]{3}\textbar{}
\texttt{/liferay-portlet-app/portlet/social-interactions-configuration}\textbar{}\hyperref[three]{3}\textbar{}
\texttt{/liferay-portlet-app/portlet/staged-model-data-handler-class}\textbar{}\hyperref[three]{3}\textbar{}
\texttt{/liferay-portlet-app/portlet/struts-path}\textbar{}\texttt{com.liferay.portlet.struts-path=\textless{}String\textgreater{}}\textbar{}
\texttt{/liferay-portlet-app/portlet/system}\textbar{}\texttt{com.liferay.portlet.system=\textless{}boolean\textgreater{}}\textbar{}
\texttt{/liferay-portlet-app/portlet/template-handler}\textbar{}\hyperref[three]{3}\textbar{}
\texttt{/liferay-portlet-app/portlet/trash-handler}\textbar{}\hyperref[three]{3}\textbar{}
\texttt{/liferay-portlet-app/portlet/url-encoder-class}\textbar{}\hyperref[three]{3}\textbar{}
\texttt{/liferay-portlet-app/portlet/use-default-template}\textbar{}\texttt{com.liferay.portlet.use-default-template=\textless{}boolean\textgreater{}}\textbar{}
\texttt{/liferay-portlet-app/portlet/user-notification-definitions}\textbar not
supported\textbar{}
\texttt{/liferay-portlet-app/portlet/user-notification-handler-class}\textbar{}\hyperref[three]{3}\textbar{}
\texttt{/liferay-portlet-app/portlet/user-principal-strategy}\textbar{}\texttt{com.liferay.portlet.user-principal-strategy=\textless{}String\textgreater{}}\textbar{}
\texttt{/liferay-portlet-app/portlet/virtual-path}\textbar{}\texttt{com.liferay.portlet.virtual-path=\textless{}String\textgreater{}}\textbar{}
\texttt{/liferay-portlet-app/portlet/webdav-storage-class}\textbar{}\hyperref[three]{3}\textbar{}
\texttt{/liferay-portlet-app/portlet/webdav-storage-token}\textbar not
supported\textbar{}
\texttt{/liferay-portlet-app/portlet/workflow-handler}\textbar{}\hyperref[three]{3}\textbar{}
\texttt{/liferay-portlet-app/portlet/xml-rpc-method-class}\textbar{}\hyperref[three]{3}\textbar{}

\noindent\hrulefill

\begin{itemize}
\item
  {[}1{]} Portlets are registered as concrete objects.
\item
  {[}2{]} Multiples of these properties may be used. This results in an
  array of values.
\item
  {[}3{]} This type is registered as an OSGi service.
\item
  {[}4{]} https://xmlns.jcp.org/xml/ns/portlet/portlet-app\_3\_0.xsd
\item
  {[}5{]}
  \href{https://docs.liferay.com/dxp/portal/7.2-latest/definitions/liferay-portlet-app_7_2_0.dtd.html}{http://www.liferay.com/dtd/liferay-portlet-app\_7\_2\_0.dtd}
\item
  {[}6{]} Here's an example of using multiple
  \texttt{javax.portlet.dependency} properties.

  \emph{Old:}

\begin{verbatim}
<portlet>
    ...
    <dependency>
        <name>jquery</name>
        <scope>com.jquery</scope>
        <version>2.1.1</version>
    </dependency>
    <dependency>
        <name>jsutil</name>
        <scope>com.mycompany</scope>
        <version>1.0.0</version>
    </dependency>
    ...
</portlet>
\end{verbatim}

  \emph{New:}

\begin{verbatim}
@Component(
    immediate = true, property = {
        "javax.portlet.name=my_portlet",
        "javax.portlet.display-name=my-portlet",
        "javax.portlet.dependency=jquery;com.jquery;2.1.1",
        "javax.portlet.dependency=jsutil;com.mycompany;1.0.0"
    }, service = Portlet.class
)
public class MyPortlet extends GenericPortlet {
    ...
} 
\end{verbatim}
\item
  {[}7{]} Here's an example for the
  \texttt{com.liferay.portlet.header-request-attribute-prefix} property.

  \emph{Old:}

\begin{verbatim}
<portlet>
    ...
    <header-request-attribute-prefix>com.mycompany</header-request-attribute-prefix>
    ...
</portlet>
\end{verbatim}

  \emph{New:}

\begin{verbatim}
@Component(
    immediate = true, property = {
        "javax.portlet.name=my_portlet",
        "javax.portlet.display-name=my-portlet",
        "javax.portlet.dependency=jquery;com.jquery;2.1.1",
        "javax.portlet.dependency=jsutil;com.mycompany;1.0.0",
        "com.liferay.portlet.header-request-attribute-prefix=com.mycompany"
    }, service = Portlet.class
)
public class MyPortlet extends GenericPortlet {
    ...
} 
\end{verbatim}
\item
  {[}8{]} Here's an example for the \texttt{javax.portlet.listener}
  property.

  \emph{Old:}

\begin{verbatim}
<portlet>
    ...
    <listener>
        <listener-class>com.mycompany.MyPortletURLGenerationListener</listener-class>
        <ordinal>1</ordinal>
    </listener>
    ...
</portlet>
\end{verbatim}

  \emph{New:}

\begin{verbatim}
@Component(
    immediate = true,
    property = {"javax.portlet.name=myPortlet",
        "javax.portlet.listener=com.mycompany.MyPortletURLGenerationListener;1"
    }, service = Portlet.class
)
public class MyPortlet extends GenericPortlet {
    ...
} 
\end{verbatim}
\item
  {[}9{]} An \texttt{javax.portlet.init-param} property can be declared
  like this:

\begin{verbatim}
@Component(
    immediate = true,
    property = {"javax.portlet.name=myPortlet", 
        "javax.portlet.init-param.myInitParam=1234"},
    service = PortletFilter.class
)
public class MyFilter implements RenderFilter {
    ...
}
\end{verbatim}
\item
  {[}10{]} Liferay DXP creates each portlet's ID based on the portlet's
  name (i.e., the \texttt{portlet-name} descriptor in
  \texttt{liferay-portlet.xml} or the \texttt{javax.portlet.name} OSGi
  service property). Dashes, periods, and spaces are allowed in the
  portlet name, but they and all other JavaScript unsafe characters are
  stripped from the name value that's used for the portlet ID.
  Therefore, make your portlet name unique in light of the characters
  that are removed. Otherwise, if you try to deploy a portlet whose ID
  is the same as a portlet that's already deployed, your portlet
  deployment fails and Liferay DXP logs a message like this:

\begin{verbatim}
Portlet id [portletId] is already in use
\end{verbatim}
\end{itemize}

\chapter{Third Party Packages Portal
Exports}\label{third-party-packages-portal-exports}

{ This document has been updated and ported to Liferay Learn and is no
longer maintained here.}

The \texttt{com.liferay.portal.bootstrap} module exports many third
party Java packages that can cause problems if used improperly. If your
WAR's Gradle file, for example, uses the \texttt{compile} scope for a
\href{/docs/7-2/customization/-/knowledge_base/c/configuring-dependencies}{dependency}
that Liferay's OSGi runtime already provides, the dependency JAR is
included in the WAR's \texttt{WEB-INF/lib} and deployed in the resulting
WAB, and two versions of dependency classes wind up on the classpath.
This can cause weird errors that are hard to debug.

To find a list of the packages exported by
\texttt{com.liferay.portal.bootstrap}, go to the source file
\texttt{modules/core/portal-bootstrap/system.packages.extra.bnd}. If you
don't have access to the source code, the same list (in a less
user-friendly format) is in the
\texttt{META-INF/system.packages.extra.mf} file in
\texttt{{[}LIFERAY\_HOME{]}/osgi/core/com.liferay.portal.bootstrap.jar}.
These packages are installed and available in Liferay's OSGi runtime. If
your module or WAR uses one of them, specify the corresponding
dependency as being ``provided'' (provided by Liferay DXP). Here's how
to specify a provided dependency:

Maven:
\texttt{\textless{}scope\textgreater{}provided\textless{}/scope\textgreater{}}

Gradle: \texttt{providedCompile}

Now you can safely leverage third party packages Liferay DXP provides!
